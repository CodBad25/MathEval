
\bigskip

%\begin{minipage}{7.5cm}
%On considère la figure ci-contre dans laquelle:
%
%\begin{itemize}
%\item[$\bullet$~] Les points F, G et H sont alignés
%\item[$\bullet$~] (LH) est perpendiculaire à (FH)
%\item[$\bullet$~] EF $= 18$ cm ; FG $= 24$ cm ; EG $= 30$ cm ;
%
%GH = $38,4$ cm
%\item[$\bullet$~] $\widehat{\text{EGF}} = \widehat{\text{LGH}}$.
%\end{itemize}
%\end{minipage}\hfill
\begin{minipage}{7.5cm}
\psset{unit=1cm}
\begin{pspicture}(7.8,4.8)
\pspolygon(0.2,3.8)(0.2,1)(7.5,1)(7.5,4.6)(3.3,1)
\pswedge*(3.3,1){0.4}{0}{39}
\pswedge*(3.3,1){0.4}{141}{180}
\psframe(7.5,1)(7.2,1.3)
\rput(3.9,0.2){\emph{La figure n'est pas en vraie grandeur.}}
\uput[u](0.2,3.8){E} \uput[d](0.2,1){F} \uput[d](3.3,1){G} \uput[dr](7.5,1){H}\uput[ur](7.5,4.6){L}
\end{pspicture}
\end{minipage}
\begin{minipage}{7.5cm}
\begin{enumerate}
\item %Montrer que le triangle EFG est rectangle en F{}.
On a EF$^2 = 18^2 = 324$ ; \quad FG$^2 = 24^2 = 576$ et EG$^2 = 30^2 = 900$.
Or $900 = 324 + 576$, soit EG$^2 =$EF$^2 + $ FG$^2$.

Donc d'après la réciproque du théorème de Pythagore le triangle EFG est rectangle en F{}.

\item %Calculer la mesure de l'angle $\widehat{\text{EGF}}$.
Dans le triangle EFG   rectangle en F on a par exemple $\tan \widehat{\text{EGF}} = \dfrac{\text{EF}}{\text{FG}} = \dfrac{18}{24} = \dfrac34 = 0,75$.

Donc d'après la calculatrice $\widehat{\text{EGF}} \approx 36,9$, soit $37\degres{}$ au degré près.
%Donner l'arrondi au degré près.
\end{enumerate}
\end{minipage}
\begin{enumerate}[start=3]
\item %Montrer que les triangles EGF et LGH sont semblables.
Les triangles EGF et LGH ont deux de leurs angles de même mesure, donc les troisièmes aussi : ils sont donc semblables
\item %Parmi les propositions suivantes, quel est le coefficient d'agrandissement qui permet de passer du triangle EFG au triangle LHG ?

%Expliquer.

%\begin{center}
%\begin{tabularx}{0.75\linewidth}{|*{4}{>{\centering \arraybackslash}X|}}\hline
%0,625 &1,28	& 1,6	&2,6\\ \hline
%\end{tabularx}
%\end{center}
[GH] et [FG] sont les côtés adjacents aux angles $\widehat{\text{EGF}}$ et $\widehat{\text{LGH}}$ de même mesure.

Comme GH $ >$ FG, le coefficient d'agrandissement est égal à $\dfrac{\text{GH}}{\text{FG}} = \dfrac{38,4}{24} = 1,6$.
\item %Quel est le périmètre du triangle LGH ?
Le périmètre de EGF est égal à :

EF + FG + GE $= 18 + 24 + 30 = 72$~(cm).

D'après la question précédente le périmètre de LGH est  égal à à celui de EFG multiplié par 1,6, soit  :

$72 \times 1,6 = 115,2$~(cm).
\end{enumerate}


\medskip

