
\medskip

%\emph{Les trois parties de cet exercice sont indépendantes et peuvent être traitées séparément.}
%
%\medskip
%
%Une famille souhaite installer dans son jardin une cabane.
%
%La partie inférieure de cette cabane est modélisée par le rectangle BCDF:

\begin{center}
\psset{unit=1cm}
\begin{pspicture}(10,3.2)
%\psgrid
\pspolygon(0,0.2)(9.5,0.2)(5.6,2.5)(1.4,2.5)%AEFB
\psline(5.6,2.5)(5.6,0.2)%FD
\psline(1.4,2.5)(1.4,0.2)%BC
\psline[linestyle=dashed](1.4,1)(0.45,1)%NM
\psframe(1.4,1)(1.2,1.2)\psframe(1.4,0.2)(1.2,0.4)\psframe(5.6,0.2)(5.8,0.4)
\psarc(9.5,0.2){0.6}{150}{180}
\uput[d](0,0.2){A} \uput[d](1.4,0.2){C} \uput[d](5.6,0.2){D} \uput[dr](9.5,0.2){E} 
\uput[u](5.6,2.5){F} \uput[u](1.4,2.5){B} \uput[d](1.4,0.2){C} \uput[r](1.4,1){N}
\uput[ul](0.45,1){M}
\rput(8.7,3){Toboggan}\rput(0.7,3.1){Escalier}
\psline{->}(8.7,2.8)(7.4,1.43)
\psline{->}(0.6,2.9)(0.8,1.57)
\end{pspicture}
\end{center}

On précise que:

\begin{center}
\begin{tabularx}{\linewidth}{m{4.5cm} X}
$\bullet~$AB = 1,3 m ;&$\bullet~$DE = 2,04 m ;\\
$\bullet~$AC = 0,5 m ;&$\bullet~$Les triangles ABC, BMN et FDE sont rectangles.\\
$\bullet~$BC = DF = 1,2 m ;&\\ 
\end{tabularx}
\end{center}

\textbf{Partie A: Étude du toboggan}

\medskip

\begin{enumerate}
\item %Pour que le toboggan soit sécurisé, il faut que l'angle $\widehat{\text{DEF}}$ mesure $30\degres{}$, au degré près.

%Le toboggan de cette cabane est-il sécurisé ?
On a $\tan \widehat{\text{DEF}} = \dfrac{\text{DF}}{\text{DE}} = \dfrac{1,2}{2,04} \approx 0,588$.

La calculatrice donne $\widehat{\text{DEF}} \approx 30,4$, soit 30~\degres{} à l'unité près : le toboggan est sécurisé.
\item %Montrer que la rampe du toboggan, EF, mesure environ $2,37$~m.
Dans le triangle DEF rectangle en D le théorème de Pythagore donne :

EF$^2 = \text{ED}^2 + \text{DF}^2 = 1,2^2 + 2,04^2 = \np{5,6016}$, d'où :

EF $= \sqrt{\np{5,6016}} \approx 2,366 \approx 2,37$ au centième près.
\end{enumerate}

\medskip

\textbf{Partie B : Étude de l'échelle}

\medskip

%Pour consolider l'échelle, on souhaite ajouter une poutre supplémentaire [MN], comme indiqué sur le modèle.

\medskip

\begin{enumerate}
\item %Démontrer que les droites (AC) et (MN) sont parallèles.
On sait que (MN) et (AC) sont perpendiculaires à (BC), or, lorsque deux droites sont perpendiculaires à une même droite, elles sont parallèles, on en déduit que (MN) et (AC) sont parallèles.
\item %On positionne cette poutre [MN] telle que BN = 0,84 m. Calculer sa longueur MN.
D'après le théorème de Thalès : $\dfrac{\text{BN}}{\text{BC}} = \dfrac{\text{MN}}{\text{AC}}$, soit $\dfrac{0,84}{1,2} = \dfrac{\text{MN}}{0,5}$, d'où MN $ = 0,5 \times \dfrac{0,84}{1,2} : \dfrac{0,42}{1,2} = 0,35$~(m).
\end{enumerate}

\medskip

\textbf{Partie C : Étude du bac à sable}

\medskip

Un bac à sable est installé sous la cabane. Il s'agit d'un pavé droit dont les dimensions sont :

$\bullet~$Longueur : 200 cm

$\bullet~$Largeur : 180 cm

$\bullet~$Hauteur : 20 cm

\medskip

\begin{enumerate}
\item %Calculer le volume de ce bac à sable en cm$^3$.
On a $V = 200 \times 180 \times 20 = \np{720000}$~(cm$^3$)
\item %On admet que le volume du bac à sable est de $0,72$~m$^3$.

%On remplit entièrement ce bac avec un mélange de sable à maçonner et de sable fin dans le ratio 3 : 2.

%Vérifier que le volume nécessaire de sable à maçonner est de $0,432$ m$^3$ et que celui de sable fin est de $0,288$~m$^3$.
En divisant le volume en 5 parties le sable à maçonner en occupe 3, soit :

$0,72 \times \dfrac35 = 0,72 \times 0,6 = 0,432$~(m$^3$).

Par différence ou en calculant les $\dfrac25$ du volume total, le volume du sable fin est :

$0,72 - 0,432 = 0,72 \times \dfrac25 = 0,72 \times 0,4 = 0,288$~(m$^3$).
\item %Un magasin propose à l'achat le sable à maçonner et le sable fin, vendus en sac. D'après les indications ci-dessous, quel est le coût total du sable nécessaire pour remplir entièrement ce bac à sable sachant qu'on ne peut acheter que des sacs entiers ?
On a $\dfrac{0,432}{0,022} \approx 19,6$ : il faut donc acheter 20 sacs de sable à maçonner et comme $\dfrac{0,288}{0,016} = 18$ : il faut donc acheter 18 sacs de sable fin.

Le coût d'achat du sable est donc :

$20 \times 2,95 + 18 \times 5,95 = 59 + 107,10 = 166,10~$(\euro).
%\begin{center}
%\begin{tabular}{l |l}
%Un  sac de sable à maçonner:& Un sac de sable fin:\\
%Poids : 35 kg&Poids : 25 kg\\
%Volume : 0,022~m$^3$&Volume : 0,016~m$^3$\\
%Prix : 2,95~\euro&Prix : 5,95~\euro\\
%\end{tabular}
%\end{center}
\end{enumerate}

\bigskip

