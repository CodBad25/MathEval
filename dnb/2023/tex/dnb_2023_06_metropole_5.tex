
\medskip

Voici deux programmes de calcul.

\begin{center}
	\begin{tabularx}{\linewidth}{|X|X|}\hline
Programme A &Programme B\\
$\bullet~~$ Choisir un nombre				&$\bullet~~$ Choisir un nombre\\
$\bullet~~$ Multiplier ce nombre par $-2$	&$\bullet~~$ Soustraire 5 à ce nombre\\
$\bullet~~$Ajouter 5 à ce résultat.			&$\bullet~~$ Multiplier le résultat par 3\\
											&$\bullet~~$ Ajouter 11 au résultat\\ \hline
	\end{tabularx}
\end{center}

\begin{enumerate}
	\item
	\begin{enumerate}
		\item Montrer que, si on choisit $- 3$ comme nombre de départ, le résultat obtenu avec le programme A est 11.
		\item Quel résultat obtient-on avec le programme B si on choisit 5,5 comme nombre de départ ?
	\end{enumerate}
	\item En désignant par $x$ le nombre de départ, on obtient $- 2x + 5$ comme résultat avec le programme A.

	Montrer, qu'avec le même nombre de départ, le résultat du programme B est égal à $3x - 4$.
\end{enumerate}
\begin{minipage}{0.56\linewidth}
	\begin{enumerate}[resume]
		\item
		\begin{enumerate}
			\item On a représenté ci-contre les fonctions $f$ et $g$ définies par $f(x) = -2x + 5$
			et $g(x) = 3x - 4$.

			Associer, en justifiant, chaque droite à la fonction qui lui correspond.
			\item Par lecture graphique, donner, le plus précisément possible, le nombre dont l'image est la même par la fonction $f$ et la fonction $g$.
		\end{enumerate}
	\end{enumerate}
\end{minipage}\hfill
\begin{minipage}{0.42\linewidth}
	\psset{xunit=1cm,yunit=0.7cm,arrowsize=2pt 3}
	\begin{pspicture*}(-1.02,-4)(5,7.02)
		\psgrid[gridlabels=0pt,subgriddiv=1,gridwidth=0.3pt]
		\psaxes[linewidth=1.25pt,labelFontSize=\scriptstyle]{->}(0,0)(-1,-4)(5,7)
		\psplot[plotpoints=500,linewidth=1.25pt]{-1}{5}{x 3 mul 4 sub}\uput[dr](3.4,6){$(D_1)$}
		\psplot[plotpoints=500,linewidth=1.25pt]{-1}{5}{5 x 2 mul sub}\uput[ur](4.3,-4){$(D_2)$}
	\end{pspicture*}
\end{minipage}

\medskip

\begin{enumerate}[start=4]
	\item Déterminer par le calcul le nombre de départ pour lequel les programmes A et B donnent le même résultat.
\end{enumerate}
%%%%%%%%
