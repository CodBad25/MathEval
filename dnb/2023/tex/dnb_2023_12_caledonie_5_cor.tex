
\medskip

\begin{enumerate}
\item 
	\begin{enumerate}
		\item %La fonction $f$, dont la représentation graphique est en ann-exe  est-elle une fonction affine ? Justifier votre réponse.
		Une fonction affine est représentée par une droite : ce n'est pas le cas dans le graphique ci-dessous : $f$ n'est donc pas une fonction affine.
		\begin{center}
\psset{unit=0.75cm,arrowsize=2pt 3}
\begin{pspicture*}(-5,-5.15)(6,13.1)
\psgrid[gridlabels=0pt,subgriddiv=1,gridwidth=0.4pt]
\psaxes[linewidth=1.25pt,labelFontSize=\scriptstyle]{->}(0,0)(-5,-5)(6,13)
\psplot[plotpoints=2000,linewidth=1.25pt,linecolor=red]{-5}{5}{x 1 sub x 3 add mul}
\psplot[plotpoints=1000,linewidth=1.25pt]{-5}{5}{x 2 mul 1 add}
\psdots(-1,-1)(0,1)(1,3)
\rput{63.43}(4,9.5){$g(x) = 2x + 1$}
\uput[d](5.75,0){$x$}\uput[l](0,12.75){$y$}\uput[r](2.95,11.15){\red $\mathcal{C}_f$}
\end{pspicture*}
\end{center}

		\item %À l'aide de ce graphique, compléter le tableau de valeurs de la fonction $f$ sur l'ann-exe.
Voir ci-dessus.
	\end{enumerate}

%Parmi les trois formules suivantes, l'une correspond à l'expression de la fonction $f$.

%Elle a été saisie dans la cellule B2 puis étendue dans la cellule C2 du tableau 

%\begin{center}
%\begin{tabularx}{\linewidth}{|*{3}{>{\centering \arraybackslash}X|}}\hline
%=B1 + 3 &=(B1 + 3)$*$(B1 $-$ 1)& =SOMME(B1 : G1) \\ \hline
%\end{tabularx}
%\end{center}
\begin{enumerate}[resume] 
		\item %Noter la bonne formule sur votre copie.
		La bonne formule est la deuxième : =(B1 + 3)$*$(B1 $-$ 1).
	\end{enumerate}
\item On considère la fonction affine $g$ définie par $g(x) = 2x + 1$.
	\begin{enumerate}
		\item %Calculer l'image de $-2$ par la fonction $g$.
		L'image de $-2$ par $g$ est $g(- 2) = 2 \times (- 2) + 1 = - 4 + 1 = - 3$.
		\item %Calculer $g(3)$.
$g(3) = 2 \times 3 + 1 = 7$.
		\item %Déterminer l'antécédent de $2$ par la fonction $g$.
L'antécédent de $2$ par la fonction $g$ est le nombre $x$ tel que $g(x) = 2$, soit $2x + 1 = 2$ ou $2x = 1$, soit $x = \dfrac12$.

L'antécédent de $2$ par la fonction $g$ est le nombre $\dfrac12$.
		\item %Tracer, sur le graphique de l'ann-exe, la représentation graphique de la fonction $g$.
Pour tracer la représentation graphique de la fonction affine $g$ il suffit de trouver deux points de la droite  ; par précaution on en prend trois : (0~;~1) $(-1~;~-1)$ et (1~;~3). Voir graphique.
	\end{enumerate}
\item L'expression de la fonction $f$ ci-dessus est $f(x) = (x + 3)(x - 1)$.
	\begin{enumerate}
		\item %Développer et réduire l'expression $(x + 3)(x - 1)$.
On a pour tout nombre $s$, \: $f(x) = x^2 - x + 3x - 3 = x^2 + 2x - 3$.
		\item %Pour quelle(s) valeur(s) de $x$, a-t-on $f(x) = g(x)$ ?
$f(x) = g(x)$ si $x^2 + 2x - 3 = 2x + 1 $ ou $x^2 = 4$, soit $x^2 - 4 = 0$ et enfin $(x + 2)(x - 2) = 0$.
Deux possibilités = $x + 2 = 0$, soit $x = -2$ ou $x - 2 = 0$ soit $x = 2$.
	\end{enumerate}
\end{enumerate}

\bigskip

