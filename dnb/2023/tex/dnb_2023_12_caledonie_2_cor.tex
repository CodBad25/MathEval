
\medskip

%José, un agriculteur vivant dans la commune du Mont-Dore, veut préparer des paniers de légumes bio pour ses clients.
%
%Il a déjà récolté 39 salades, 78 carottes et 51 aubergines.
%
%Il veut que tous les paniers aient la même composition et utiliser tous les légumes.
%
%La décomposition de 39 en produit de facteurs premiers est : $3 \times 13$.

\medskip

\begin{enumerate}
\item 
	\begin{enumerate}
		\item %Décomposer en facteurs premiers les nombres 78 et 51.
$\bullet~~$De même que $39 = 3 \times 13$, on a $78 = 60 + 18 = 6 \times 10 + 6 \times 3 = 6 \times (10 + 3) = 6 \times 13 = 2 \times 3 \times 13$ ;
		
$\bullet~~$ $51 = 30 + 21 = 3 \times 10 + 3 \times7 = 3\times(10 + 7) = 3 \times 17$.
		\item %En déduire le nombre de paniers maximum que José peut préparer.
On a donc $\left\{\begin{array}{l c l}
39&=&3 \times 13\\
78 &=& 3 \times 26\\
51&=&3 \times 17
\end{array}\right.$

On peut donc faire 3 paniers identiques.
		\item %Combien de salades, de carottes et d'aubergines y aurait-il dans chaque panier?
		Il suffit de relever les seconds facteurs de chaque produit pour trouver que chacun des 3  paniers sera composé de 13 salades, 26 carottes et 17 aubergines.
	\end{enumerate}
\end{enumerate}

Finalement, José décide de préparer 13 paniers.

\begin{enumerate}[resume]
\item 
	\begin{enumerate}
		\item %Combien d'aubergines ne seront pas utilisées? Justifier votre réponse.
		On a :
$\left\{\begin{array}{l c l}
39&=&13 \times 3\\
78 &=& 13 \times 6\\
51&=&13 \times 3 + {\red 12}
\end{array}\right.$

Chacun des 13 paniers aura 3 salades, 6 carottes et 3 aubergines. Resterons 12 aubergines.
		\item %Combien doit-il cueillir au minimum d'aubergines supplémentaires pour pouvoir toutes les utiliser ?
Avec 1 aubergine de plus, on aura $52 = 13 \times 4$ : chacun des 13 paniers aura 4 aubergines.
			\end{enumerate}
\end{enumerate}

%José souhaite que ses 13 paniers contiennent également des tomates.

%Il estime qu'il en a entre $110$ et $125$~prêtes à être récoltées.

\begin{enumerate}[resume]
\item %Combien doit-il en cueillir au maximum pour éviter les pertes et pour que chaque panier ait toujours la même composition ?
On écrit les multiples de 13 aux environs de 110 et 125 :

%\textbf{Toute trace de recherche, même non aboutie, sera prise en compte.}
$110 < {\red 117 = 13 \times 9}< 125  < {\red 130 = 13 \times 10}$ : le seul multiple de 13 entre 110 et 125 est $117 = 13 \times 9$ ; si l'on récolte 117 tomates, on pourra en mettre exactement 9 dans chacun des 13 paniers.
\end{enumerate}


