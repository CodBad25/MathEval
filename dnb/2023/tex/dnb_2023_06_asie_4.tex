
\medskip

Dans cet exercice, aucune justification n'est demandée.

\smallskip

Une élève souhaite réaliser un programme avec un logiciel de programmation pour dessiner des frises constituées de carrés et de rectangles.

Pour cela, elle commence par créer les trois blocs ci-dessous:

\medskip

\begin{center}
\renewcommand{\arraystretch}{1.5}
\begin{tabular}{|p{3.5cm}|p{5.4cm}|p{5.1cm}|}
\hline
~~\newline
\begin{scratch}
\initmoreblocks{d\'efinir \namemoreblocks{initialisation}}
\blockpen{effacer tout}
\blockmove{aller à x: \ovalnum{-220} y: \ovalnum0}
\blockmove{s'orienter à \ovalnum{90}}
\end{scratch}\newline
~~\newline
La commande\newline \og s'orienter à 90 \fg{}\newline signifie que le lutin est tourné vers la droite.
&
~~\newline
\begin{scratch}[num blocks]
\initmoreblocks{d\'efinir \namemoreblocks{carré}}
\blockpen{stylo en position d'\'ecriture}
\blockrepeat{répéter \ovalnum{\hphantom{4}} fois}
{
\blockmove{avancer de \ovalnum{50} pas}
\blockmove{tourner \turnright{} de \ovalnum{\hphantom{90}} degrés}
}
\end{scratch}
&
~~\newline
\begin{scratch}
\initmoreblocks{d\'efinir \namemoreblocks{rectangle}}
\blockpen{stylo en position d'\'ecriture}
\blockrepeat{répéter \ovalnum{2} fois}
{
\blockmove{avancer de \ovalnum{100} pas}
\blockmove{tourner \turnright{} de \ovalnum{90} degrés}
\blockmove{avancer de \ovalnum{50} pas}
\blockmove{tourner \turnright{} de \ovalnum{90} degrés}
}
\end{scratch}\newline
~~\newline
\\
\hline
\hfill{~}\textbf{Bloc 1}\hfill{~} & \hfill{~}\textbf{Bloc 2}\hfill{~} &\hfill{~}\textbf{Bloc 3}\hfill{~} \\
\hline
\end{tabular}
\end{center}

%\medskip

\begin{enumerate}
\item Quelles sont les coordonnées du lutin après l'exécution du bloc 1?
\item Par quelles valeurs doit-on compléter les lignes 3 et 5 du bloc 2 pour obtenir un carré?
\item Construire ce que dessine le lutin lorsque le bloc 3 est utilisé. On prendra 1~cm pour 20 pas.
\item L'élève souhaite réaliser les deux frises ci-dessous.

\medskip

\begin{minipage}{0.65\linewidth}
\psset{unit=1cm}
\def\xmin {0}   \def\xmax {9} \def\ymin {-1}   \def\ymax {4}
\begin{pspicture}(\xmin,\ymin)(\xmax,\ymax)
%\psgrid[subgriddiv=1, gridcolor=gray] 
\psframe(0,2)(9,3) \uput[u](0.6,3){Frise 1}
\psline(1,2)(1,3) \psline(3,2)(3,3) \psline(4,2)(4,3) \psline(6,2)(6,3) \psline(7,2)(7,3) 
\psframe(0,0)(9,1) \uput[u](0.6,1){Frise 2}
\psline(1,0)(1,1) \psline(2,0)(2,1) \psline(3,0)(3,1) \psline(5,0)(5,1) \psline(7,0)(7,1) 
\end{pspicture}
\begin{enumerate}
\item Elle rédige le script ci-contre. Indiquer le numéro de la frise qu'elle va réaliser lorsque le drapeau vert est cliqué.
\item Écrire un script qui permet de réaliser la frise qui n'a pas été obtenue.
\end{enumerate}
\vspace*{1cm}
\end{minipage}
\hfill
\begin{minipage}{0.3\linewidth}
\begin{scratch}
\blockinit{Quand \greenflag est cliqué}
\blockmove{initialisation}
\blockrepeat{répéter \ovalnum{3} fois}
{
\blockmove{carré}
\blockmove{avancer de \ovalnum{50} pas}
\blockmove{rectangle}
\blockmove{avancer de \ovalnum{100} pas}
}
\end{scratch}
\end{minipage}

\end{enumerate}

\bigskip

