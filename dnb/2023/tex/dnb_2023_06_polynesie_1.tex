
\medskip

Cet exercice est un questionnaire à choix multiples QCM

\textbf{Aucune justification n'est demandée}

Pour chaque question, trois réponses sont proposées, une seule est exacte.

Écrire sur votre copie, le numéro de la question et la réponse correspondante.

\medskip

\textbf{Question 1 :} soit $f$, la fonction définie par $f(x) = -2x + 3$.

Quelle est la représentation de la fonction $f$ ?

\medskip

\begin{tabularx}{\linewidth}{*3{>{\centering\arraybackslash}X}}
\textbf{Réponse A}\rule[-7pt]{0pt}{0pt}
\psset{unit=0.4cm}
\def\xmin {-5}   \def\xmax {5} \def\ymin {-3}   \def\ymax {6}
\begin{pspicture*}(\xmin,\ymin)(\xmax,\ymax)
\psgrid[subgriddiv=1, gridcolor=lightgray]
\psaxes[arrowsize=3pt 2, ticksize=-2pt 2pt,Dx=2,Dy=2,labelFontSize=\scriptstyle]{->}(0,0)(\xmin,\ymin)(\xmax,\ymax) 
\psplot[linecolor=blue]{\xmin}{\xmax}{1.5 x mul 3 add}
\end{pspicture*}
&
\textbf{Réponse B}\rule[-7pt]{0pt}{0pt}
\psset{unit=0.4cm}
\def\xmin {-5}   \def\xmax {5} \def\ymin {-3}   \def\ymax {6}
\begin{pspicture*}(\xmin,\ymin)(\xmax,\ymax)
\psgrid[subgriddiv=1, gridcolor=lightgray]
\psaxes[arrowsize=3pt 2, ticksize=-2pt 2pt,Dx=2,Dy=2,labelFontSize=\scriptstyle]{->}(0,0)(\xmin,\ymin)(\xmax,\ymax) 
\psplot[linecolor=blue]{\xmin}{\xmax}{-2 x mul 3 add}
\end{pspicture*}
&
\textbf{Réponse C}\rule[-7pt]{0pt}{0pt}
\psset{unit=0.4cm}
\def\xmin {-5}   \def\xmax {5} \def\ymin {-3}   \def\ymax {6}
\begin{pspicture*}(\xmin,\ymin)(\xmax,\ymax)
\psgrid[subgriddiv=1, gridcolor=lightgray]
\psaxes[arrowsize=3pt 2, ticksize=-2pt 2pt,Dx=2,Dy=2,labelFontSize=\scriptstyle]{->}(0,0)(\xmin,\ymin)(\xmax,\ymax) 
\psplot[linecolor=blue]{\xmin}{\xmax}{3 x mul 2 sub}
\end{pspicture*}
\end{tabularx}

\medskip

\begin{minipage}{0.65\linewidth}\textbf{Question 2 :}
On considère la fonction dont la représentation graphique est donnée ci-contre.

D'après le graphique, quelle est l'image de 1  par cette fonction ?

\begin{tabularx}{\linewidth}{*{2}{>{\centering \arraybackslash}X|}{>{\centering \arraybackslash}X}}
Réponse A  &Réponse B&  Réponse C\\ 
L'image de 1 est 2&L'image de 1 est  -2&L'image de 1 est 0\\
\end{tabularx}
\end{minipage}\hfill
\begin{minipage}{0.28\linewidth}
\psset{unit=0.6cm}
\def\xmin {-3}   \def\xmax {3} \def\ymin {-3}   \def\ymax {3}
\begin{pspicture*}(\xmin,\ymin)(\xmax,\ymax)
\psgrid[subgriddiv=1, gridcolor=lightgray]
\psaxes[arrowsize=3pt 2, ticksize=-2pt 2pt,Dx=2,Dy=2,labelFontSize=\scriptstyle]{->}(0,0)(\xmin,\ymin)(\xmax,\ymax) 
\psplot[linecolor=blue]{\xmin}{\xmax}{-5 x dup mul mul 17 add x mul 6 div}
\psdots[linecolor=blue](-2,1)(-1,-2)(1,2)(2,-1)
\uput[dl](-2,1){\blue A} \uput[dl](-1,-2){\blue B} 
\uput[ur](1,2){\blue C} \uput[ur](2,-1){\blue D} 
\end{pspicture*}
\end{minipage}

\medskip

\textbf{Question 3 :}

On donne ci-dessous un tableau de valeurs de la fonction $h$ définie par $h(x) = - x + 1$ réalisé à l'aide d'un tableur :

\begin{center}
%\renewcommand{\arraystretch}{1.2}
$\begin{array}{|>{\cellcolor{lightgray}}c| *{7}{>{\centering\arraybackslash}m{1cm}|}}
\cline{2-8}
\rowcolor{lightgray} &\text A & \text B & \text C & \text D & \text E & \text F & \text G\\
\hline
1 & \blue $x$      & -3 & -2 & -1 & 0 & 1 & 2 \\ \hline
2 & \blue $h(x)$  & 4 & 3 & 2 & 1 & 0 & -1 \\ \hline
\end{array}$
\end{center}

Quelle formule a-t-on saisie dans la case B2 avant de l'étirer vers la droite ?

\begin{tabularx}{\linewidth}{*{2}{>{\centering \arraybackslash}X|}{>{\centering \arraybackslash}X}}
Réponse A  &Réponse B&  Réponse C\\ 
= $- (-3) + 1$ &= $- x + 1$ & = $-$ B1 + 1\\
\end{tabularx}

\medskip

\textbf{Question 4 :}

Quelle est la forme développée de l'expression $(3x - 7)^2$ ?

\begin{tabularx}{\linewidth}{*{2}{>{\centering \arraybackslash}X|}{>{\centering \arraybackslash}X}}
Réponse A  &Réponse B&  Réponse C\\ 
$3x^2 - 49$& $9x^2 - 42x + 49$&$9x^2 - 49$.
\end{tabularx}

\bigskip

