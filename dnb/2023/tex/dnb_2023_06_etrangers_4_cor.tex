
\medskip

Des élèves organisent, pour leur classe, un jeu au cours duquel il est possible de gagner des lots. Pour cela, ils placent dans une urne trois boules noires numérotées de 1 à 3, et quatre boules rouges numérotées de 1 à 4, toutes indiscernables au toucher.

\bigskip

\textbf{Partie A : étude du jeu}

\medskip

\begin{enumerate}
\item On pioche au hasard une boule dans l'urne.
	\begin{enumerate}
		\item %Quelle est la probabilité de tirer une boule rouge ?
Il y a en tout 7 boules dont 4 sont rouge, la probabilité de tirer une boule rouge est donc de 
$\dfrac47$.
		\item %Quelle est la probabilité de tirer une boule dont le numéro est un nombre pair ?
Les nombres pairs sont 2 et 4, ils sont présents sur 3 boules différentes donc la probabilité de tirer une boule dont le numéro est un nombre pair est de $\dfrac37$.
	\end{enumerate}	
\item %Le jeu consiste à piocher, dans l'urne, une première boule, la remettre dans l'urne puis en piocher une seconde.

%Pour chacune des boules tirées, on note la couleur ainsi que le numéro.

%Pour gagner un lot, il faut tirer la boule rouge numérotée 1 et une boule noire.

%Quelle est la probabilité de gagner ?
\item On construit un tableau à double entrée donnant toutes les issues
\[\begin{tabular}{|m{2.8cm}|*{7}{c|}}\hline
\diagbox{\footnotesize 1\up{er}tirage}{\footnotesize 2\up{nd} tirage}&N1&N2&N3&\red R1&\red R2&\red R3&\red R4\\ \hline
N1&&&&$\bullet$&&&\\ \hline
N2&&&&$\bullet$&&&\\ \hline
N3&&&&$\bullet$&&&\\ \hline
\red R1&$\bullet$&$\bullet$&$\bullet$&&&&\\ \hline
\red R2&&&&&&&\\ \hline
\red R3&&&&&&&\\ \hline
\red R4&&&&&&&\\ \hline
\end{tabular}\]

%Il ya deux issues favorables qui consistent à :
%\begin{itemize}
%\item tirer la boule rouge numérotée 1 puis une boule noire
%\item tirer une boule noire puis la boule rouge numérotée 1
%\end{itemize}
%
Il y a 6 issues favorables donc la probabilité de gagner est de $\dfrac{6}{49}$.
\end{enumerate}

\bigskip

\textbf{Partie B : constitution des lots}

\medskip

%Pour constituer les lots, on dispose de 195 figurines et 234 autocollants. 
%
%Chaque lot sera composé de figurines ainsi que d'autocollants. 
%
%Tous les lots sont identiques.
%
%Toutes les figurines et tous les autocollants doivent être utilisés.

\medskip

\begin{enumerate}
\item %Peut-on faire 3 lots ?
On peut faire 3 lots puisque $\dfrac{195}{3} =65$ et $\dfrac{234}{3} = 78$ donc les 3 lots seront constitués de $65$~figurines et $78$~autocollants.
\item %Décomposer 195 en produit de facteurs premiers.
$195 = 5 \times 39 = 5 \times 3 \times 13 = 3 \times 5 \times 13$.
\item Sachant que la décomposition en produit de facteurs premiers de $234$ est $2 \times 3^2 \times 13$ :
	\begin{enumerate}
		\item %Combien de lots peut-on constituer au maximum ?
On peut donc diviser 195 et 234 par $3 \times 13 = 39$ au maximum. On pourra donc constituer au maximum $39$ lots.
		\item %De combien de figurines et d'autocollants sera alors composé chaque lot ?
Chaque lot sera alors composé de $\dfrac{195}{39} = 5$~figurines et $\dfrac{234}{39} = 6$~autocollants.
	\end{enumerate}
\end{enumerate}

\bigskip

