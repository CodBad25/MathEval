

\medskip

\textbf{Situation 1}

\medskip

$ 780 = 2 \times 390 = 2\times2\times195 = 2^{2}\times 5 \times 39= 2^{2}\times5\times3\times13=2^{2}\times3\times5\times13$.

\medskip

\textbf{Situation 2}

\medskip

\begin{enumerate}[label=\alph*)]
\item On tire une carte au hasard, on est donc dans une situation d'équiprobabilité.

\(\displaystyle P(\og8 ~~ \text{de pique}\fg) = \frac{\text{nombre d'issues favorables}}{\text{nombre total d'issues}} = \frac{1}{32}\)
\item Il y a 4 rois et 8 c\oe{}urs dans ce jeu, donc au total 11 issues favorables car on ne compte pas deux fois le roi de c\oe{}ur.

\(\displaystyle P(\og \text{roi ou c\oe{}ur}\fg) = \frac{\text{nombre d'issues favorables}}{\text{nombre total d'issues}} = \frac{11}{32}\).
	\end{enumerate}
	
\medskip

\textbf{Situation 3}

\medskip

$ A = (2x + 5)(3x - 4)= 6x^{2}- 8x + 15x - 20 = 6x^{2} +7x - 20$.
	
\medskip

\textbf{Situation 4}

\medskip
	
\begin{enumerate}[label=\alph*)]
	\item	Le solide est un prisme droit dont les bases sont des triangles rectangles.
	
On calcule l'aire de chacun de ces triangles :

		\(\displaystyle A_{base}=\dfrac{60\times80}{2}=2~400 ~~\text{cm}^{2}\)
		
		On en déduit le volume du prisme droit :
		
		\(\displaystyle V_{prisme}=2 400\times120 = 288~000 ~~\text{cm}^{3} \).
		
	\item	$ 288~000 ~~\text{cm}^{3}=288 ~~\text{dm}^{3}=288 ~~\text{L} $
	
		Le volume est donc de $ 288 ~~\text{L}.$
	\end{enumerate}

\medskip

\textbf{Situation 5}

\medskip

Le polygone 2 est un agrandissement du polygone 1.

Le coefficient d'agrandissement est $ 3 $ donc l'aire du polygone 1 est multipliée par $ 3^{2} $ donc par $ 9 $. On a ainsi :

$ A_{\text{polygone~~2}}=9\times A_{\text{polygone~~1}}= 9\times 11= 99$.

L'aire du polygone 2 est donc $ 99 ~~\text{cm}^{2} $.


