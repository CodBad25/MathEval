
\medskip

On considère le programme de calcul suivant:
\hspace{1cm}
\begin{tabular}[t]{|l|}
\multicolumn{1}{c}{Nombre de départ}\\
\multicolumn{1}{c}{\psline[linewidth=4pt,arrowsize=1pt 2,arrowinset=0]{->}(0,0.3)(0,-0.5)}\\
\multicolumn{1}{c}{}\\
\hline
\textbullet~~Calculer le carré de ce nombre\\
\textbullet~~Multiplier par 5\\
\textbullet~~Ajouter 4\\
\textbullet~~Multiplier par 2\\
\textbullet~~Enlever 8\\
\hline
\multicolumn{1}{c}{\psline[linewidth=4pt,arrowsize=1pt 2,arrowinset=0]{->}(0,0.3)(0,-0.5)}\\
\multicolumn{1}{c}{}\\
\multicolumn{1}{c}{Résultat}\\
\end{tabular}

\bigskip

\textbf{PARTIE A}

\begin{enumerate}
\item Montrer que si 3 est le nombre de départ, le programme donne un résultat égal à 90.
\item Un élève choisit 2 comme nombre de départ et un autre élève choisit $-2$. \\
Montrer qu'ils doivent obtenir le même résultat.
\item Si on nomme $x$ le nombre de départ, montrer que le résultat du programme peut s'écrire $10x^2$.
\end{enumerate}

\bigskip

\textbf{PARTIE B}

\medskip

Pour cette partie, un élève cherche le ou les nombre(s) qu'il doit choisir pour obtenir 30 comme résultat. 

\begin{enumerate}[resume]
\item Pour cela, il représente graphiquement la fonction $f$ associée au programme de calcul définie par: $f(x)=10x^2$.

Il obtient la courbe suivante:

%\medskip

\begin{center}
\scalebox{0.8}{
\psset{xunit=3.5cm,yunit=0.15cm,arrowsize=3pt 3}
\def\xmin {-2.4}   \def\xmax {2.4}
\def\ymin {-5}   \def\ymax {45}
\begin{pspicture*}(\xmin,\ymin)(\xmax,\ymax)
\psgrid[gridlabels=0pt,xunit=3.5cm,yunit=0.75cm,subgriddiv=5, gridcolor=gray](-3,-5)(12,45)
\psaxes[labelFontSize=\small, ticks=none, labels=y,Dy=5]{->}(0,0)(\xmin,-4.9)(\xmax,\ymax)
\uput[dl](0,0){$0$} \uput[ul](1.8,32){$\blue\mathcal{C}_f$}
\psplot[linecolor=blue]{\xmin}{\xmax}{10 x dup mul mul}
\multido{\n=0.2+0.2}{11}{\uput[d](\n,0){\small\np{\n}}}
\end{pspicture*}
}% scalebox
\end{center}

%\medskip

À l'aide du graphique, déterminer une valeur approchée des antécédents de 30 par la fonction $f$. Ne pas justifier.

\item L'élève souhaite trouver une valeur plus précise de l'antécédent \textbf{positif} trouvé à la question précédente. Pour cela il utilise une feuille de calcul dont un extrait est donné ci-dessous:

\begin{center}
%\renewcommand{\arraystretch}{1.2}
$\begin{array}{|>{\cellcolor{lightgray}}c| *{3}{>{\centering \arraybackslash}m{1.9cm}|}}
\hline
\rowcolor{lightgray} & \text A & \text B & \text C\\
\hline
1 & \textbf{Nombre de départ} & \textbf{Résultat} & \\ \hline
2 & 1,60   & 25,600  & \\ \hline
3 &  1,61 &   25,921 & \\ \hline
4 & 1,62  & 26,244  & \\ \hline
5 & 1,63  &  26,569 & \\ \hline
6 & 1,64 &  26,896 & \\ \hline
7 &  1,65 &  27,225 & \\ \hline
8 &  1,66 &  27,556 & \\ \hline
9 &  1,67 &  27,889 & \\ \hline
10 &  1,68 &  28,224 & \\ \hline
11 &  1,69 &  28,561 & \\ \hline
12 &  1,70&  28,900 & \\ \hline
13 &  1,71 &  29,241 & \\ \hline
14 &  1,72 &  29,584 & \\ \hline
15 &  1,73 &  29,929 & \\ \hline
16 &  1,74 &  30,276 & \\ \hline
17 &  1,75 &  30,625 & \\ \hline
18 &  1,76 &  30,976 & \\ \hline
19 &  1,77 &  31,329 & \\ \hline
20 &  1,78 &  31,684 & \\ \hline
21 &  1,79 &  32,041 & \\ \hline
22 &  1,80 &  32,400 & \\ \hline
23 &  &  & \\ \hline
\end{array}$
\end{center}

\begin{enumerate}
\item Quelle formule a-t-il pu entrer dans la cellule \texttt{B2} avant de l'étirer vers le bas? Ne pas justifier.
\item Dans ce tableau, quel est le nombre de départ donnant le résultat le plus proche de 30? Ne pas justifier.
\end{enumerate}

\item Déterminer la valeur exacte du nombre positif cherché par l'élève.

\end{enumerate}

\bigskip

