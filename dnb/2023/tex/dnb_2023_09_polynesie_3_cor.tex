
\medskip

%Dans cet exercice, toutes les longueurs sont exprimées en pixel.
\begin{minipage}{11cm}Dans cet exercice, toutes les longueurs sont exprimées en pixel.

Un professeur de mathématiques souhaite élaborer un programme avec ses élèves permettant de construire la figure ci-contre composée de 10 carrés.

Le côté du premier carré à tracer mesure $300$~pixels.

Le côté de chaque carré construit ensuite mesure 20\,\% de moins que celui du carré précédent.

La figure n'est pas en vraie grandeur.
\medskip

\end{minipage}\hfill
\begin{minipage}{3.5cm}
\psset{unit=1cm}
\begin{pspicture}(-2,-2)(2,2)
\psframe[linewidth=1pt](-1.6,-1.6)(1.6,1.6)
\psframe[linewidth=1pt](-1.28,-1.28)(1.28,1.28)
\psframe[linewidth=1pt](-1.024,-1.024)(1.024,1.024)
\psframe[linewidth=1pt](-0.8192,-0.8192)(0.8192,0.8192)
\psframe[linewidth=1pt](-0.65536,-0.65536)(0.65536,0.65536)
\psframe[linewidth=1pt](-0.5243,-0.5243)(0.5243,0.5243)
\psframe[linewidth=1pt](-0.4194,-0.4194)(0.4194,0.4194)
\psframe[linewidth=1pt](-0.3355,-0.3355)(0.3355,0.3355)
\psframe[linewidth=1pt](-0.2684,-0.2684)(0.2684,0.2684)
\psframe[linewidth=1pt](-0.2147,-0.2147)(0.2147,0.2147)
\end{pspicture}
\end{minipage}
%\textbf{Aucune justification n'est attendue pour les questions 2., 3. a., 3. b. et 4.}

\medskip

\begin{enumerate}
\item %Montrer que le côté du 2\up{e} carré mesure $240$ pixels.
Retirer 20\;\%, c'est multiplier par $1-\dfrac{20}{100}$ soit $0,8$.

Le côté du premier carré à tracer mesure $300$~pixels, donc le côté du 2\ieme{} carré mesure $300\times 0,8$, c'est-à-dire 240~pixels.

\item Le professeur distribue aux élèves le bloc \og Carré \fg{} d'instructions figurant ci-dessous qui permet de tracer un carré de côté donné.
\begin{center}
\begin{scratch}[num blocks]
\initmoreblocks{définir \namemoreblocks{\ovalmoreblocks{Carré}}}
\blockrepeat{répéter \ovalnum{} fois}
{\blockmove{avancer de \ovalmoreblocks{Côté}}
\blockmove{tourner \turnright{} de \ovalnum{} degrés}
}
\end{scratch}
\end{center}
Pour cela, il a créé une variable \og Côté \fg{} qui correspond à la longueur du côté du carré à tracer.

Voici le script avec les lignes 2 et 4 complétées:

\begin{center}
\begin{scratch}[num blocks]
\initmoreblocks{définir \namemoreblocks{\ovalmoreblocks{Carré}}}
\blockrepeat{répéter \ovalnum{4} fois}
{\blockmove{avancer de \ovalmoreblocks{Côté}}
\blockmove{tourner \turnright{} de \ovalnum{90} degrés}
}
\end{scratch}
\end{center}
\end{enumerate}

\begin{minipage}{0.6\linewidth}
\begin{enumerate}[resume]
\item Le script ci-contre permet de réaliser les dix carrés de la figure souhaitée.

On rappelle que l'instruction \og s'orienter à 180 \fg{} signifie que le lutin est dirigé vers le bas.
	\begin{enumerate}
		\item %Donner les coordonnées du stylo lorsqu'il commence à tracer le premier carré.
\og Côté \fg{} vaut 300 et on démarre chaque carré au point de coordonnées (Côté/2\;;\;Côté/2).  

Les coordonnées du stylo lorsqu'il commence à tracer le premier carré sont donc $(150\;;\;150)$.

		\item Parmi les 4 propositions ci-dessous,  celle qui correspond au tracé des deux premiers carrés est la proposition 3.
		
En effet, on démarre avec un Côté de 300, donc en $(150\;;\;150)$, puis on réduit le Côté de 20\,\% donc il vaut 240; on démarre alors le deuxième carré en $(120\;;\;120)$.
	\end{enumerate}
\end{enumerate}
\end{minipage}
\hfill
\begin{minipage}{0.35\linewidth}
\begin{scratch}[num blocks, scale=0.8]
\blockinit{Quand \greenflag est cliqué}
\blockpen{effacer tout}
\blockmove{s'orienter à \ovalnum{180} degrés}
\blockvariable{mettre \selectmenu{Côté} à \ovaloperator{\ovalnum{300}}}
\blockrepeat{répéter \ovalnum{10} fois}
{\blockpen{relever le stylo}
\blockmove{aller à x: \ovalmoreblocks{Côté} \pmb{/} \ovalnum{2} y: \ovalmoreblocks{Côté} \pmb{/} \ovalnum{2}}
\blockpen{stylo en position d'écriture}
\blockmoreblocks{Carré}
\blockvariable{mettre \selectmenu{Côté} à \ovaloperator{Côté \pmb{*} \ovalnum{0,8}}}
}

\end{scratch}
\end{minipage}

\medskip

\begin{enumerate}[resume,label=]
\item 	\begin{enumerate}[start=2,label=]
\begin{center}
\begin{tabularx}{\linewidth}{|*{4}{>{\centering \arraybackslash}X|}}\hline
Proposition 1&Proposition 2&Proposition 3&Proposition 4\\ \hline
\psset{unit=0.005cm}
\begin{pspicture}(-200,-200)(320,330)
\psaxes[linewidth=1.25pt,Dx=100,Dy=100,labelFontSize=\scriptscriptstyle]{->}(0,0)(-200,-200)(320,320)
\psframe(300,300)\psframe(30,30)(270,270)
\uput[d](310,0){\tiny $x$}\uput[l](0,310){\tiny $y$}
\end{pspicture}&
\psset{unit=0.005cm}
\begin{pspicture}(-200,-200)(320,330)
\psaxes[linewidth=1.25pt,Dx=100,Dy=100,labelFontSize=\scriptscriptstyle]{->}(0,0)(-200,-200)(320,320)
\psframe(-150,-150)(150,150)\psframe(-100,-100)(100,100)
\uput[d](310,0){\tiny $x$}\uput[l](0,310){\tiny $y$}
\end{pspicture}&
\psset{unit=0.005cm}
\begin{pspicture}(-200,-200)(320,330)
\psaxes[linewidth=1.25pt,Dx=100,Dy=100,labelFontSize=\scriptscriptstyle]{->}(0,0)(-200,-200)(320,320)
\psframe(-120,-120)(120,120)\psframe(-140,-140)(140,140)
\uput[d](310,0){\tiny $x$}\uput[l](0,310){\tiny $y$}
\end{pspicture}&
\psset{unit=0.005cm}
\begin{pspicture}(-200,-200)(320,330)
\psaxes[linewidth=1.25pt,Dx=100,Dy=100,labelFontSize=\scriptscriptstyle]{->}(0,0)(-200,-200)(320,320)
\psframe(300,300)\psframe(100,100)(200,200)
\uput[d](310,0){\tiny $x$}\uput[l](0,310){\tiny $y$}
\end{pspicture}\\ \hline
\end{tabularx}
\end{center}
\end{enumerate}
%\end{enumerate}
%
%\begin{enumerate}[resume,label=]
\item 	\begin{enumerate}[start=3]
		\item %Quelle est la longueur du dernier carré tracé avec le script précédent? Arrondir au pixel.
Le 1\ier{} carré a un côté de longueur 300.

Le 2\ieme{} carré a un côté de longueur $300\times 0,8 = 240$.

Le 3\ieme{} carré a un côté de longueur $240\times 0,8 = 300\times 0,8^2 = 192$.

Etc.

Le 10\ieme{} carré a un côté de longueur $300\times 0,8^9$ soit environ $40,27$.

La longueur du dernier carré est donc d'environ $4\times 40,27$ soit environ $161$.
	\end{enumerate}
\end{enumerate}

\begin{enumerate}[start=4]
\item On veut diminuer l'épaisseur des traits lorsqu'on passe de la construction d'un carré au suivant pour obtenir la figure suivante.

\begin{minipage}{11cm}
Pour cela, on souhaite utiliser les deux instructions suivantes : 
\begin{itemize}
\item[$\bullet$] Instruction A :

\begin{scratch}\blockpen{ajouter \ovalvariable{$- 1$} à la taille su stylo} \end{scratch}
\item[$\bullet$] Instruction B :

\begin{scratch}\blockpen{mettre la taille du stylo à \ovalvariable{$11$}} \end{scratch}
\end{itemize}
\end{minipage}
\hfill
\begin{minipage}{3.5cm}
\psset{unit=1cm}
\begin{pspicture}(-2,-2)(2,2)
\psframe[linewidth=3pt](-1.6,-1.6)(1.6,1.6)
\psframe[linewidth=2.7pt](-1.28,-1.28)(1.28,1.28)
\psframe[linewidth=2.4pt](-1.024,-1.024)(1.024,1.024)
\psframe[linewidth=2.1pt](-0.8192,-0.8192)(0.8192,0.8192)
\psframe[linewidth=1.8pt](-0.65536,-0.65536)(0.65536,0.65536)
\psframe[linewidth=1.5pt](-0.5243,-0.5243)(0.5243,0.5243)
\psframe[linewidth=1.2pt](-0.4194,-0.4194)(0.4194,0.4194)
\psframe[linewidth=0.9pt](-0.3355,-0.3355)(0.3355,0.3355)
\psframe[linewidth=0.6pt](-0.2684,-0.2684)(0.2684,0.2684)
\end{pspicture}
\end{minipage}
%Pour chaque instruction, indiquer les numéros des lignes du script de la question 2 entre lesquelles elle peut être insérée afin d'obtenir cette figure.

\medskip

On insère l'instruction A entre les lignes 9 et 10, et on peut insérer l'instruction B entre les lignes 2 et 3.
\end{enumerate}


