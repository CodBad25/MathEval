
\medskip

%Une usine fabrique des bougies parfumées en cire de forme cylindrique.
%
%\medskip
%
%\textbf{Les questions 1, 2 et 3 sont indépendantes}
%
%\begin{center}
%\begin{tabularx}{\linewidth}{|p{5.8cm}|X|}\hline
%\textbf{Document 1}
%
%\psset{unit=1cm,arrowsize=2pt 3}
%\begin{pspicture}(-2.7,0)(2.7,4.9)
%\psellipse(0,4)(0.7,0.15)
%\psellipticarc(0,0.8)(0.7,0.15){180}{0}
%\psellipticarc[linestyle=dashed](0,0.8)(0.7,0.15){0}{180}
%\psline(-0.7,0.8)(-0.7,4)\psline(0.7,0.8)(0.7,4)
%\pscurve[linewidth=1.5pt](0,4)(-0.1,4.15)(0.15,4.3)
%\rput(1.,4.4){la mèche}\rput(1.9,2.35){la bougie}
%\psline{<->}(0,0.5)(0.7,0.5)
%\uput[d](0.35,0.5){\small 3 cm}
%\psline{<->}(-1,0.8)(-1,4)
%\uput[l](-1,2.4){\small 12 cm}
%\end{pspicture}
%
%Rayon du cylindre : 3 cm 
%
%Hauteur du cylindre : 12 cm&
%\textbf{Document 2}
%
%Aire d'un disque  : rayon $^2 \times \pi$
%
%Volume d'un cylindre : Aire de la base $\times$ hauteur
%%\rule{6cm}
%
%\textbf{Document 3}
%
%\begin{itemize}
%\item Une bougie est composée de cire et de parfum.
%\item Le volume de cire nécessaire à la fabrication d'une bougie correspond au $\dfrac{9}{10}$ du volume de cette bougie.
%\item 1~cm$^3$ de cire a une masse de $0,7$~g.
%\end{itemize}\\ \hline
%\end{tabularx}
%\end{center}

\begin{enumerate}
\item 
	\begin{enumerate}
		\item %Montrer que le volume d'une bougie est d'environ 339 cm$^2$.
La base d'une bougie est un disque de rayon 3 et de hauteur 12 : son volume est donc égal à :
		
$\pi \times 3^2 \times 12 = 108\pi \approx 339,3$, soit à l'unité près 339~cm$^3$.
		\item %Quelle est la masse de cire nécessaire pour une bougie ? On donnera une valeur approchée au gramme près.
$\dfrac{9}{10}$ de ce volume est de la cire soit $\dfrac{9}{10} \times 108\pi = 97,2\pi$~cm$^3$ et à raison 0,7~g par cm$^3$, il faut $97,2\pi \times 0,7 = 68,04\pi \approx 213,8$ soit au gramme près environ 214~g de cire pour fabriquer une bougie.
	\end{enumerate}

%\begin{minipage}{8.4cm}
\item %Au mois de novembre, l'usine a fabriqué des bougies de 4 parfums différents:

%vanille, miel, lavande et jasmin.

%Le diagramme circulaire codé ci-contre donne la répartition, pour le mois de novembre, du nombre de bougies fabriquées en fonction de leur parfum.

%Les bougies au miel représentent 22\,\% de la production du mois de novembre.

%Quel est le pourcentage de bougies à la lavande fabriquées au mois de novembre ?
%\end{minipage}\hspace{0.4cm}
%\begin{minipage}{4.8cm}
%\psset{unit=0.8cm}
%\begin{pspicture}(-2.9,-2.9)(2.9,2.9)
%\pscircle(0,0){2.9}
%\psline(2.9;180)(0,0)(2.9;-90)
%\psline(2.9;5.4) \psline(2.9;84.6)
%\rput(1.45;46){Miel}\rput(1.45;135){Lavande}
%\rput(1.45;225){Vanille}\rput(1.45;-45){Jasmin}
%\psframe(0,0)(-0.3,-0.3)
%\psarc(0,0){0.5}{5.4}{84.6}\psarc(0,0){0.6}{5.4}{84.6}
%\psarc(0,0){0.4}{84.6}{180}\psarc(0,0){0.4}{270}{5.4}
%\end{pspicture}
%\end{minipage}
Les bougies à la vanille sont représentées par un secteur dont l'angle au centre a pour mesure $90\degres{}$ ; comme $90 = \dfrac{360}{4}$ elles représentent le $\dfrac14$ de la production soit $25\,\%$.

Come il y autant de bougies à la lavande que de bougies au jasmin, le pourcentage de bougies à la lavande (ou au jasmin) est égal à :

$\dfrac{100 - (22 + 25)}{2} = \dfrac{100 - 47}{2} = \dfrac{53}{2} = 26,5$~(\,\%).
\item %Durant les trois premiers mois de l'année suivante, l'entreprise se donne pour objectif de produire en moyenne \np{7900} bougies par mois.

%En janvier, elle fabrique \np{6500} bougies et \np{8000} en février.

%Quel est le nombre de bougies à produire en mars pour atteindre l'objectif ?
Si $m$ est le nombre de bougies à produire en mars on doit avoir comme moyenne : 

$\np{7900} = \dfrac{\np{6500} + \np{8000} + m}{3}$, soit $3\times \np{7900} = \np{14500} + m$ ou encore $m = \np{23700} - \np{14500} = \np{9200}$.
\end{enumerate}

\medskip

