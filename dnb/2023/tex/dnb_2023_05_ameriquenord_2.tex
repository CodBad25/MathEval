
\medskip

\begin{minipage}{0.48\linewidth}
On considère la figure ci-contre. On donne les mesures suivantes:

\begin{itemize}
\item[$\bullet~$] AN = 13 cm
\item[$\bullet~$] LN = 5 cm
\item[$\bullet~$] AL = 12 cm
\item[$\bullet~$] ON = 3 cm
\item[$\bullet~$] O appartient au segment [LN]
\item[$\bullet~$] H appartient au segment [NA]
\end{itemize}
\end{minipage}\hfill
\begin{minipage}{0.48\linewidth}
\psset{unit=1cm}
\begin{pspicture}(6.5,8.3)
\rput(3.25,8.1){Cette figure n'est pas à l'échelle. }
\pspolygon(1.9,0.4)(6,0.4)(1.9,7.6)%LNA
\psline(3.4,0.4)(3.4,5)%OH
\psframe(3.4,0.4)(3.6,0.6)
\uput[dl](1.9,0.4){L} \uput[dr](6,0.4){N} \uput[l](1.9,7.6){A} \uput[d](3.4,0.4){O} \uput[ur](3.4,5){H} 
\end{pspicture}
\end{minipage}


\begin{enumerate}
\item Montrer que le triangle LNA est rectangle en L.
\item Montrer que la longueur OH est égale à $7,2$~cm.
\item Calculer la mesure de l'angle $\widehat{\text{LNA}}$. Donner une valeur approchée à l'unité près. 
\item Pourquoi les triangles LNA et ONH sont-ils semblables ?
\item 
	\begin{enumerate}
		\item Quelle est l'aire du quadrilatère LOHA ?
		\item Quelle proportion de l'aire du triangle LNA représente l'aire du quadrilatère LOHA ?
	\end{enumerate}
\end{enumerate}

\bigskip

