
\medskip


\begin{enumerate}
\item %Vérifier que si la valeur de $x$ est 5 alors le résultat est $63$.
On a $(5 + 4) \times (2\times 5 - 3) = 9 \times 7 = 63$.
\item %Quel résultat obtient-on si la valeur de $x$ est $- 3$ ?
De même $(- 3 + 4) \times (2\times (- 3) - 3) = 1 \times (- 9) = - 9$.
\item %Parmi les expressions suivantes, recopier celle qui correspond au programme de calcul donné par le script.

%\[A = (x + 4) \times (2x - 3)\qquad  B = x + 4 \times 2x - 3\qquad  C = x + 4 \times (2x - 3)\]
$A = (x + 4) \times (2x - 3)$.
\item  %Pour quelle(s) valeur(s) de $x$ obtient-on un résultat égal à $0$ ?
Il faut résoudre l'équation $(x + 4) \times (2x - 3) = 0$, soit $\left\{\begin{array}{l c l}
x + 4&=&0\\
2x - 3&=&0
\end{array}\right.$ ou $\left\{\begin{array}{l c l}
x &=&- 4\\
2x &=&3
\end{array}\right.$ et enfin $\left\{\begin{array}{l c l}
x &=&- 4\\
x &=&\frac{3}{2}
\end{array}\right.$

$- 4$ et 1,5 donnent comme résultat 0.
\end{enumerate}

\vspace{0,5cm}

