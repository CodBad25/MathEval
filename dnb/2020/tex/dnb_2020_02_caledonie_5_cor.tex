
\medskip

%L'image satellite représente 6 bassins de forme rectangulaire.

\medskip

\begin{enumerate}
\item %À partir de cette image, estimer la longueur et la largeur (en m) d'un bassin.
Avec l'échelle donnée un bassin mesure environ 150~m sur 32~m.
\item %On considère un bassin dont la surface mesure \np{4500} m$^2$.

%Chaque bassin reçoit 2 larves de crevettes par mètre carré.

%Calculer la quantité de larves de crevettes qu'il faut prévoir pour $6$ bassins.
Il faudra donc $6 \times 2 \times \np{4500} = \np{54000}$~crevettes pour les six bassins.
\item %Toutes les larves de crevettes ne survivent pas lors du transfert en bassin. Il faut prévoir de commander 10\,\% de larves de crevettes supplémentaires pour $6$ bassins.

%Quelle quantité totale de larves de crevettes faut-il commander ?
Ajouter 10\,\% c'est multiplier par 1,1.

Il faut donc prévoir : $\np{54000} \times 1,1 = \np{59400}$~crevettes.
\end{enumerate}

\vspace{0,5cm}

