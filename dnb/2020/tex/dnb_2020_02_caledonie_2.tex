
\medskip

Laura a créé trois variables puis elle a réalisé le script ci-dessous.


\begin{multicols}{2}
~\vfill

\begin{center}
Créer une variable

\medskip

\begin{scratch}
\blockvariable{Étape 1}
\blockvariable{Étape 2}
\blockvariable{$\times$}
\end{scratch}
\end{center}

\vfill~

\columnbreak

\begin{scratch}
\blockinit{Quand \greenflag est cliqué}
\blocksensing{demander \ovalnum{valeur de x ?} et attendre}
\blockvariable{mettre{\selectmenu{$x$} à \ovalsensing{réponse}}}
\blockvariable{mettre{\selectmenu{Étape 1}} à \ovaloperator{\ovalvariable{$x$}  + \ovalnum{4}}}
\blockvariable{mettre{\selectmenu{Étape 2}} à \ovaloperator{\ovalnum{2} *\ovalvariable{$x$} - \ovalnum{3}}}
\blocklook{dire {\ovaloperator{regroupe {\ovalnum{Le résultat est : }} et \ovaloperator{\ovalvariable{~Étape 1} * \ovalvariable{Étape 2}}}}}
\end{scratch}
\end{multicols}

\begin{enumerate}
\item Vérifier que si la valeur de $x$ est 5 alors le résultat est $63$.
\item  Quel résultat obtient-on si la valeur de $x$ est $- 3$ ?
\item  Parmi les expressions suivantes, recopier celle qui correspond au programme de calcul donné par le script.

\[A = (x + 4) \times (2x - 3)\qquad  B = x + 4 \times 2x - 3\qquad  C = x + 4 \times (2x - 3)\]

\item  Pour quelle(s) valeur(s) de $x$ obtient-on un résultat égal à $0$ ?
\end{enumerate}

\vspace{0,5cm}

