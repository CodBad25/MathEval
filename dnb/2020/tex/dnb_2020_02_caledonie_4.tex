
\medskip

Les crevettes mangent des granulés qui sont stockés dans des réservoirs appelés silos.

Un silo est composé d'un cône de révolution surmonté d'un cylindre de même base de diamètre
DC = $2,8$~m. La hauteur du cylindre est égale à $2,4$~m.

\medskip

\parbox{0.54\linewidth}{\psset{unit=1cm}
\begin{pspicture}(5,2.5)
\uput[r](2.5,2.2){Rappel:}
\uput[r](2.5,1.5){$\text{Volume du cylindre} = \pi \times \text{rayon}^2 \times \text{hauteur}$}
\uput[r](2.5,0.5){$\text{Volume du cône} = \dfrac{\pi \times \text{rayon}^2 \times \text{hauteur}}{3}$}
\end{pspicture}

\medskip

\begin{enumerate}
\item Calculer le volume du cylindre. Arrondir à l'unité.
\item Montrer que la hauteur AB du cône est environ de $2,5$m.
\item Calculer le volume du silo. Arrondir à l'unité.
\item L'aquaculteur commande 16 m$^3$ de granulés pour crevettes.
\end{enumerate}
}
\hfill
\parbox{0.44\linewidth}{\psset{unit=1cm}
\begin{center}
La figure n'est pas à l'échelle.

\medskip

\begin{pspicture}(5,6.5)
%\psgrid
\psellipse(2.5,5.5)(2,0.5)
\psline(0.5,5.5)(0.5,2.5)\psline(4.5,5.5)(4.5,2.5)
%\psellipse(2.5,2.5)(2,0.5)
\psline(0.5,2.5)(2.5,0)(4.5,2.5)
\scalebox{1}[0.3]{\psarc[linewidth=2pt](2.5,8.3){1.95}{180}{0}}%
\scalebox{1}[0.28]{\psarc[linestyle=dashed,linewidth=2pt](2.5,8.6){1.95}{0}{180}}%
\psline[linewidth=0.5pt]{<->}(0.5,6.2)(4.5,6.2)
\psline[linewidth=0.5pt]{<->}(0.2,5.5)(0.2,2.5)
\psline[linewidth=0.5pt]{<->}(0.3,2.5)(2.3,0)
\uput[ur](2.5,2.5){A}\uput[d](2.5,0){B}\uput[r](4.5,2.5){C}\uput[l](0.2,2.5){D}
\uput[u](2.5,6.2){2,8~m}\uput[l](0.3,4){2,4~m}\uput[dl](1.5,1.25){2,9~m}
\psline[linestyle=dashed](0.5,2.5)(4.5,2.5)
\psline[linestyle=dashed](2.5,2.5)(2.5,0)
\rput(2.5,2.5){\psframe(0.3,-0.3)}
\end{pspicture}
\end{center}}

\medskip

Voici les informations dont il dispose:

\begin{center}
\begin{tabular}{|c|}\hline
\textbf{Informations sur les granulés :}\\ \hline
Masse volumique: 750 kg / m$^3$\\
Prix au kilogramme: 160 F CFP\\ \hline
\end{tabular}
\end{center}

Calculer le montant total (en F CFP) de la commande. Justifier la réponse.


\vspace{0,5cm}

