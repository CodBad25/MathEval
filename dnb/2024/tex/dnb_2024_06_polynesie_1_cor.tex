
\medskip

%Cet exercice est un questionnaire à choix multiples (QCM).
%
%Pour chaque question, quatre affirmations sont proposées. \textbf{Une seule affirmation est exacte.}
%
%\textbf{Sur la copie}, écrire le numéro de la question et l'affirmation choisie. Aucune justification n'est attendue.
%
%\medskip

\begin{enumerate}
\item 

ABC est un triangle tel que AB = 20 cm, BC = 21 cm et AC = 29 cm. 
%On peut affirmer que:

%\begin{center}
%\begin{tabularx}{\linewidth}{|*{4}{>{\small}X|}}\hline
%ABC est un triangle ABC en A&ABC est un triangle rectangle en B&
%ABC est un triangle rectangle en C &ABC n'est pas un triangle rectangle\\ \hline
%\end{tabularx}
%\end{center}
$\bullet~$On a AB$^2 = 20^2 = 400$ et BC$^2 = 21^2 = 441$. D'où AB$^2 + \text{BC}^2 = 400 + 441 = 841$.

D'autre part : AC$^2 = 29^2 = (30 - 1)^2 = 30^2 - 2 \times 30 + 1 = 900 - 60 +1 = 840 + 1 = 841$.

On a donc $400 + 441 = 841$, soit AB$^2 + $ BC$^2 = $ AC$^2$ : d'après la réciproque du théorème de Pythagore la triangle ABC est rectangle en B.

$\bullet~$Variante avec identité mais sans calculette :

AC$^2 - $ BC$^2 = 29^2 - 21^2 = (29 + 21)(29 - 21) = 50 \times 8 = 400 = 20^2 = $ AC$^2$, d'où : AC$^2 - $ BC$^2 = $ AC$^2$ soit AC$^2= $ BC$^2 + $ AC$^2$ \ldots

$\bullet~$Semblable mais plus technique : AC$^2 - $AB$^2 = 29^2 - 20^2 = (29 + 20)(29 - 20) = 49 \times 9 = 7^2 \times 3^2 = (7 \times 3)^2 = 21^2 = $ BC$^2$, d'où : AC$^2 - $ AB$^2 = $ BC$^2$ soit AC$^2 = $ BC$^2 + $ AC$^2$ \ldots
%\begin{minipage}{0.49\linewidth}
\item %Voici la représentation graphique d'une fonction $f$.
La droite représente une fonction affine $x \longmapsto ax + b$, avec $a \in \R, \: b \in \R$.

$\bullet~$l'ordonnée à l'origine est égale à $b = 1$ ;

$\bullet~$le coefficient directeur de la droite est égal à $\dfrac{1}{2} = a$.

Conclusion $x \longmapsto f(x) = \dfrac12 x + 1$.
%\begin{minipage}{0.55\linewidth}
\item %Sur la figure ci-contre, le carré \no 2 est l'image du carré \no 1 par :
%\end{minipage}\hfill
%\begin{minipage}{0.42\linewidth}
%\psset{unit=1cm,arrowsize=2pt 3}
%\begin{pspicture}(-2,-2)(4,1)
%\psframe[linewidth=1.5pt](-2,0)(-1,1)
%\psframe[linewidth=1.5pt](2,0)(4,-2)
%\psline(-1,1)(2,-2)\psline(-0.55,-0.1)(-0.55,0.1)\psline(-0.5,-0.1)(-0.5,0.1)
%\psline(0.55,-0.1)(0.55,0.1)\psline(0.5,-0.1)(0.5,0.1)\psline(1.55,-0.1)(1.55,0.1)\psline(1.5,-0.1)(1.5,0.1)
%\psline(-1,0)(2,0)
%\psline(0,0)(0,-0.1)\psline(1,0)(1,-0.1)
%\rput(-1.5,0.5){1}\rput(3,-1){2}
%\uput[d](-1,0){C}\uput[ur](0,0){O}\uput[u](2,0){E}
%\end{pspicture}
%\end{minipage}
%
%\begin{center}
%\begin{tabularx}{\linewidth}{|*{4}{>{\small}X|}}\hline
%la symétrie centrale de centre O&la translation qui transforme C en E&l'homothétie de centre O et de rapport 2&l'homothétie de centre O et de rapport $- 2$\\ \hline
%\end{tabularx}
%\end{center}
Comme $\vect{\text{OE}} = - 2\vect{\text{OC}}$ la transformation est l'homothétie de centre O et de rapport $- 2$

\item %Le cocktail Bora-Bora est composé de jus d'ananas, de jus de fruit de la passion et de jus de citron dans le ratio de 10~:~6~:~2. Pour réaliser $90$ cL de ce cocktail, il faut prévoir exactement :
On a le tableau de proportionnalité suivant :

\begin{center}
\begin{tabularx}{\linewidth}{*{3}{>{\centering \arraybackslash}X}c}
ananas&passion&citron&cocktail\\ 
10		&6					&2		&$10 + 6 + 2 = 18$\\ 
 		&$6{\red \times 5}$  & 		&$90 = 18 \times{\red  5}$\\
\end{tabularx}
\end{center}
%
%\begin{center}
%\begin{tabularx}{\linewidth}{|*{4}{X|}}\hline
%6 cL de jus de fruit de la passion&30 cL de jus de fruit de
%la passion&54 cL de jus de fruit de la passion&45 cL de jus de fruit de 
%la passion\\ \hline
%\end{tabularx}
%\end{center}
On passe donc de la ligne 2 à la ligne 3 en multipliant par 5, d'où $5 \times 6 = 30$~(cL) de  jus de fruit de la passion.
\item  %Un maraîcher a cueilli $408$ pommes et $168$ poires. Il décide de remplir des sacs pour ses clients comportant chacun le même nombre de pommes et le même nombre de poires, en utilisant tous les fruits cueillis.

%Le plus grand nombre de sacs qu'il peut ainsi remplir est :
%
%\begin{center}
%\begin{tabularx}{\linewidth}{|*{4}{X|}}\hline
%48 sacs&24 sacs&8 sacs&6 sacs\\ \hline
%\end{tabularx}
%\end{center}
$\bullet~408 = 4 \times 102 = 4 \times 2 \times 51 = 8 \times 3 \times 17$ ; 

$\bullet~168 = 8 \times 21 = 8 \times 3 \times 7$.

Donc $408 = 24 \times 17$ et $168 = 24 \times 7$.

On peut donc faire au maximum 24 sacs identiques contenant chacun 17 pommes et 7 poires.

\end{enumerate}

\bigskip

