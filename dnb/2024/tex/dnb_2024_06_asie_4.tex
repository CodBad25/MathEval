
Des amis habitent Strasbourg et préparent leurs vacances.

Cette année ils ont décidé de partir découvrir une grande ville française pendant une semaine. 

Pour s'y rendre, ils louent une voiture. Une fois arrivés sur place, ils feront ensuite tous leurs trajets à pied ou en transport en commun.

Une agence de location de voitures propose les trois formules suivantes pour une location sur une semaine :

\begin{center}
\begin{tabularx}{\linewidth}{*{3}{>{\centering \arraybackslash}X}}\hline
Formule A&Formule B&Formule C\\ \hline
0,50~\euro{} pour chaque kilomètre parcouru&
 Forfait fixe de 300~\euro{}
puis 0,25~\euro{} pour chaque kilomètre parcouru&
Forfait fixe de 900~\euro{} pour un kilométrage illimité.\\ \hline
\end{tabularx}

\bigskip

\textbf{Tableau indicatif des distances (en km) entre des villes françaises}
\medskip

\begin{tabularx}{\linewidth}{*{7}{>{\centering \arraybackslash}X}}
Bordeaux&&&&&&\\ \cline{1-1}
675& Grenoble&&&&&\\ \cline{1-2}
792& 771& Lille&&&&\\ \cline{1-3}
 555 &280&1005&Marseille&&&\\ \cline{1-4}
338 &741&584& 909& Nantes&&\\ \cline{1-5}
546& 585&215 &772& 379& Paris&\\ \cline{1-6}
907 &506& 498 &803& 864& 442&Strasbourg\\ \cline{1-7}
\end{tabularx}
\end{center}

Exemple: la distance la plus courte entre Nantes et Grenoble est de 741 km.

\medskip

\textbf{PARTIE A} : Les amis souhaitent se rendre à Marseille. Ils ont un budget de \np{1000} \euro{} pour le voyage.

\medskip

\begin{enumerate}
\item Quelle distance, en km, vont-ils parcourir pour le trajet aller-retour ?
\item En choisissant la formule B, montrer que la location de voiture coûtera $701,50$~\euro.
\item Quelle est la formule la plus avantageuse ?
\item Voici des informations pour le voyage :

\begin{center}
\begin{tabularx}{\linewidth}{|*{3}{>{\centering \arraybackslash}X|}}\hline
Information 1&Information 2&Information 3\\ \hline
\small Prix moyen du gazole en 2023 &\small Voiture proposée&\small  Coût total pour les péages\\
\small 1,87~\euro{} par litre&\small Type de carburant: gazole. Consommation: 5,6~L pour 100 km&\small  115,80~\euro \\ \hline
\end{tabularx}
\end{center}

Leur budget sera-t-il suffisant ?

\emph{Dans cette question, toute trace de recherche sera prise en compte dans la correction}.
\end{enumerate}
\medskip

\medskip

\textbf{PARTIE B }: Étude des formules 

\begin{center}
\begin{tabularx}{\linewidth}{|*{3}{>{\centering \arraybackslash}X|}}\hline
Formule A&Formule B&Formule C\\ \hline
0,50~\euro{} pour chaque kilomètre parcouru&Forfait fixe de 300~\euro{}
puis 0,25~\euro{} pour chaque kilomètre parcouru&Forfait fixe de $900$~\euro{} pour un kilométrage illimité.\\ \hline
\end{tabularx}
\end{center}

\begin{enumerate}[resume]
\item Soit $x$ le nombre de kilomètres parcourus, exprimer en fonction de $x$ le prix payé pour chaque formule de location.
\item On a représenté ci-dessous, pour chacune des formules, le coût de la location (en euros) en fonction de la distance parcourue (en kilomètres).

Associer chaque courbe à la formule de location correspondante. \emph{Ne pas justifier}.

\begin{center}
\psset{xunit=0.004cm,yunit=0.0075cm,arrowsize=2pt 3,labels=none}
\begin{pspicture}(-200,-50)(2600,1300)
\uput[r](0,1250){Coût de la location (en \euro)}
\uput[d](1800,0){Distance parcourue (en km)}
\psaxes[linewidth=1.25pt,labels=none,Dx=200,Dy=200]{->}(0,0)(0,0)(2600,1300)
\psaxes[linewidth=1.25pt,labels=none,Dx=200,Dy=200](0,0)(0,0)(2600,1300)
\psline[linewidth=1.25pt,linecolor=blue](2600,1300)\rput{45}(300,120){\blue Courbe 3}
\psline[linewidth=1.25pt,linecolor=red](0,300)(2600,950)\rput{25}(400,380){\red Courbe 2}
\psline[linewidth=1.25pt,linecolor=cyan](0,900)(2600,900)\uput[u](400,900){\cyan Courbe 1}
\uput[d](200,0){200}\uput[l](0,200){200}
\end{pspicture}
\end{center}

\item Résoudre l'équation 
\[0,25x + 300 = 0,5x.\]
 Interpréter ce résultat.
\item 
	\begin{enumerate}
		\item Si la distance parcourue est de \np{2500}~km, quelle formule doit-on choisir pour payer le moins cher ? Ne pas justifier.
		\item Donner une distance parcourue pour laquelle la formule A est la plus intéressante. Ne pas justifier.
		\item Déterminer graphiquement quelle formule de location est la moins chère en fonction de la distance parcourue pour une distance inférieure à \np{2600}~km.
	\end{enumerate}
\end{enumerate}

