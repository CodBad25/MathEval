
\medskip

\begin{minipage}{0.48\linewidth}
Pour obtenir l'octogone  EFGHIJKL ci-contre, on retire quatre triangles rectangles isocèles identiques des coins d'un carré ABCD de côté 5 m.

On donne :

AD $= 5$ m ; EF $= 2,2$ m.
\end{minipage}\hfill
\begin{minipage}{0.5\linewidth}
\psset{unit=0.8cm,arrowsize=2pt 3}
\begin{pspicture}(8.6,8.6)
%\psgrid
\def\barc{\psline(0,-0.1)(0,0.1)}
\def\bara{\psline(-0.05,-0.1)(-0.05,0.1)\psline(0.05,-0.1)(0.05,0.1)}
\def\barb{\psline(-0.05,-0.1)(-0.05,0.1)\psline(0,-0.1)(0,0.1)\psline(0.05,-0.1)(0.05,0.1)}
\psframe(1.4,0.5)(8.4,7.5)%DCBA
\pspolygon[fillstyle=solid,fillcolor=gray!30,linewidth=1.25pt](3.4,0.5)(6.4,0.5)(8.4,2.5)(8.4,5.5)(6.4,7.5)(3.4,7.5)(1.4,5.5)(1.4,2.5)
\rput(5,7.5){\bara}\rput(5,0.5){\bara}\rput{90}(8.4,4){\bara}\rput{90}(1.4,4){\bara}
\rput{45}(2.4,6.5){\barb}\rput{-45}(7.4,6.5){\barb}\rput{45}(7.4,1.5){\barb}\rput{-45}(2.4,1.5){\barb}
\rput(2.4,7.5){\barc}\rput(7.4,7.5){\barc}
\rput(2.4,0.5){\barc}\rput(7.4,0.5){\barc}
\rput{90}(1.4,6.5){\barc}\rput{90}(8.4,6.5){\barc}
\rput{90}(1.4,1.5){\barc}\rput{90}(8.4,1.5){\barc}
\psframe(1.4,0.5)(1.65,0.75)\psframe(1.4,7.5)(1.65,7.25)\psframe(8.4,7.5)(8.15,7.25)\psframe(8.4,0.5)(8.15,0.75)
\uput[ul](0.4,7.5){A} \uput[ur](8.4,7.5){B} \uput[dr](8.4,0.5){C} \uput[dl](1.4,0.5){D}
\uput[u](3.4,7.5){E} \uput[u](6.4,7.5){F} \uput[r](8.4,5.5){G} \uput[r](8.4,2.5){H}
\uput[d](6.4,0.5){I} \uput[d](3.4,0.5){J} \uput[l](1.4,2.5){K} \uput[l](1.4,5.5){L}
\psline[linewidth=0.6pt]{<->}(3.4,8)(6.4,8)\uput[u](5,8){2,2 m}
\psline[linewidth=0.6pt]{<->}(1,0.5)(1,7.5)\rput{90}(0.8,4){5 m}
\end{pspicture}
\end{minipage}

\medskip

\begin{enumerate}
\item 
	\begin{enumerate}
		\item Montrer que la longueur AE est égale à 1,4 m.
		\item Montrer que l'aire du triangle AEL est égale à 0,98 m$^2$.
		\item En déduire que l'aire de l'octogone grisé est égale à 21,08 m$^2$
	\end{enumerate}
\begin{minipage}{0.62\linewidth}
\item  Cet octogone a les mêmes dimensions que la surface d'une piscine de hauteur 1,50 m.

On souhaite remplir cette piscine aux trois quarts de sa hauteur.

\end{minipage}\hfill
\begin{minipage}{0.25\linewidth}
\includegraphics[width=3.5cm]{piscine}
\end{minipage}

	\begin{enumerate}
		\item  Montrer que le volume d'eau nécessaire est environ égal à $24~\text{m}^3$.
		\item  Sachant que le débit du robinet utilisé pour remplir la piscine est de $12$~ L/min, calculer la
durée de remplissage de ces $24~\text{m}^3$ d'eau.

Donner le résultat en heures et minutes.

\emph{Rappel}: 1 m$^3 = \np{1000}$~L.
	\end{enumerate}
\end{enumerate}
