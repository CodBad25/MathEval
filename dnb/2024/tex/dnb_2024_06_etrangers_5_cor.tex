
\medskip

\textbf{Partie A}

\begin{enumerate}
\item Le triangle SOM est un triangle rectangle en O, et donc, d'après le théorème de Pythagore, on a :

$\mathrm{SM}^2 = \mathrm{SO}^2 + \mathrm{OM}^2 = 30^2 + 9^2 = 900 + 81 = 981$.

Comme la longueur MS est positive, on en déduit :\quad $\mathrm{SM} = \sqrt{981} \approx 31,32$.

En arrondissant au dixième de centimètre, on a bien MS = \np[cm]{31,3}.

\item La base du cône est un cercle de rayon \np[cm]{9}, donc de circonférence :

$\mathcal{C} = 2\times \pi \times 9 = 18\pi \approx \np[cm]{56,55}$.

Puisque son tour de tête mesure \np[cm]{56}, les dimensions choisies sont adaptées : la circonférence du cône est à peine plus grande que le tour de tête de Léo, donc le chapeau sera assez grand, mais pas trop grand.

\item \begin{enumerate}
		\item Puisque le cercle de rayon SM a un rayon de $31,3$,  sa circonférence est de :

		$2\pi\times 31,3 \approx 196,66$.


		En arrondissant au dixième de centimètre, la longueur du cercle de centre S et de rayon SM, est bien égale à \np[cm]{196,7}.

		\item Voici le tableau complété :

		\begin{center}
			\renewcommand\arraystretch{2}
			\begin{tabularx}{\linewidth}{|m{8cm}|*{2}{>{\centering \arraybackslash}X|}}\hline
				Mesure de l'angle $\widehat{\text{M}'\text{SM}}$ (en degré) &360& 103\\ \hline
				Longueur de l'arc $\widearc{\text{M}'\text{M}}$ (en centimètre)\qquad \qquad(Valeur arrondie au dixième de centimètre)&196,7 & 56,5\\ \hline
			\end{tabularx}
		\end{center}

		\item Calculons la mesure de l'angle, en utilisant le tableau de proportionnalité et un produit en croix :

		$\widehat{\mathrm{M}'\mathrm{SM}} = \dfrac{360 \times 56,5}{196,7} =\dfrac{20340}{196,7} \approx 103,4$\,\degre{}.

		 Au degré près, l'angle correspondant à une longueur d'arc de \np[cm]{56,5} permettant à Léo de tracer le patron de son chapeau est de 103\,\degre{}.
\end{enumerate}
\end{enumerate}

\bigskip

\textbf{Partie B}

\medskip

\begin{enumerate}
\item Le volume total du chapeau est donné par :\quad $V_\mathrm{chapeau} = \dfrac{1}{3} \times \pi\times 9^2 \times 30 = 810\pi \approx \np{2544,7}$.

En arrondissant au $\mathrm{cm}^3$, on a donc bien un volume total d'environ \np[cm^3]{2545}.


\item Si les bonbons atteignent le milieu de la hauteur de son chapeau, cela signifie que la partie remplie de bonbons (celle qui est grisée sur la figure) est l'image du chapeau complet par une homothétie de rapport $\dfrac{1}{2}$, et de centre S, le sommet du chapeau.

Une homothétie de rapport $\dfrac{1}{2}$ multiplie toutes les longueurs par $\dfrac{1}{2}$ et les volumes par $\left(\dfrac{1}{2}\right)^3 = \dfrac{1}{2^3}=\dfrac{1}{8}=0,125 $.

Si le volume de bonbons est obtenu en multipliant le volume du chapeau par 0,125; cela signifie que le volume de bonbons est de 12,5\,\% du volume total de chapeau.

Or $12,5\,\% < 15\,\%$.

Il a donc raison dans son estimation : si la hauteur de bonbons n'atteint que la moitié de la hauteur du chapeau, alors le volume de bonbons est inférieur à 15\,\% du volume total de son chapeau.
\end{enumerate}
