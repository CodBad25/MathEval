
\medskip

Un club de natation propose un après-midi découverte pour les enfants. 

\bigskip

\textbf{PARTIE A}

\medskip

La présidente du club veut offrir des petits sachets cadeaux tous identiques contenant des autocollants et des drapeaux avec le logo du club. Elle a acheté 330 autocollants et 132 drapeaux et veut tous les utiliser. Elle veut que, dans chaque sachet, il y ait exactement le même nombre d'autocollants et que, dans chaque sachet, il y ait exactement le même nombre de drapeaux.

\medskip

\begin{enumerate}
\item Pourquoi n'est-il pas possible de faire 15 sachets ?
\item 
	\begin{enumerate}
		\item Décomposer 330 et 132 en produits de facteurs premiers.
		\item En déduire le plus grand nombre de sachets que la présidente pourra réaliser.
		\item Dans ce cas, combien mettra-t-elle d'autocollants et de drapeaux dans chaque sachet ?
	\end{enumerate}
\end{enumerate}

\bigskip

\textbf{PARTIE B}

\medskip

La piscine a la forme d'un pavé droit représenté ci-dessous.

\begin{minipage}{7cm}
Elle est remplie aux $\dfrac{9}{10}$ du volume.

1 m$^3$ d'eau coûte 4,14~\euro.

Combien coûte le remplissage de la piscine ?
\end{minipage}\hfill
\begin{minipage}{7cm}
\psset{unit=1cm,arrowsize=2pt 3}
\begin{pspicture}(6,4.4)
%\psgrid
\psframe(2,0.5)(5.5,2.1)%face avant
\psline(5.5,2.1)(4,3.9)(0.8,3.9)(2,2.1)%haut
\psline(2,0.5)(0.8,2.3)(0.8,3.9)(2,2.1)%gauche
\psline[linestyle=dashed](0.8,2.3)(4,2.3)(4,3.9)
\psline[linestyle=dashed](4,2.3)(5.5,0.5)
\psline[linewidth=0.6pt]{<->}(2,0.4)(5.5,0.4)\uput[d](3.75,0.4){15 m}
\psline[linewidth=0.6pt]{<->}(5.6,0.5)(5.6,2.1)\uput[r](5.6,1.3){2 m}
\psline[linewidth=0.6pt]{<->}(1.9,0.4)(0.7,2.2)\uput[dl](1.3,1.3){25 m}
\end{pspicture}
\end{minipage}
