
\medskip

La quadrilatère ABCD ci-dessous est constitué de deux triangles équilatéraux de côté 5 cm.

\psset{unit=0.75cm}
\begin{center}
\begin{pspicture}(6.5,5)
%\psgrid
\pspolygon(0.2,0.2)(5.2,0.2)(7.5,4.53)(2.7,4.53)%BCDA
\psline(5.2,0.2)(2.7,4.53)
\uput[ul](2.7,4.53){A} \uput[dl](0.2,0.2){B} \uput[dr](5.2,0.2){C} \uput[ur](7.5,4.53){D} \uput[d](2.7,0.2){5 cm}
\psline(2.7,0.3)(2.7,0.1)
\def\marque{\psline(0,0.1)(0,-0.1)}
\rput{60}(1.45,2.36){\marque}\rput{-60}(3.95,2.36){\marque}\rput{60}(6.35,2.36){\marque}
\rput(5.1,4.53){\marque}
\end{pspicture}
\end{center}

\medskip

\begin{enumerate}
\item
	\begin{enumerate}
		\item Reproduire le quadrilatère ABCD en vraie grandeur.
		\item Quelle est sa nature?
		\item Démontrer que l'angle $\widehat{\text{BCD}}$ mesure $120\degres$.
	\end{enumerate}	
\item Le programme ci-dessous permet de créer le bloc Motif qui trace le quadrilatère ABCD.

Recopier et compléter les lignes 5 et 6 de ce programme.

On utilise l'échelle suivante : 10 pas dans le programme représentent 1 cm dans la réalité.

\medskip

\begin{minipage}{0.55\linewidth}
\begin{scratch}[scale=0.75,num blocks]
\renewcommand*\numblock[1]{\large #1~~}
\initmoreblocks{d\'efinir \namemoreblocks{Motif}}
\blockrepeat{r\'ep\'eter \ovalnum{2} fois}
{
\blockmove{avancer de \ovalnum{50 } pas}
\blockmove{tourner \turnleft{} de \ovalnum{60} degr\'es}
\blockmove{avancer de \ovalnum{\ldots } pas}
\blockmove{tourner \turnleft{} de \ovalnum{\ldots} degr\'es}
}
\end{scratch}
\end{minipage}\hfill
\begin{minipage}{0.45\linewidth}
\psset{unit=0.5cm,arrowsize=2pt 3}
\def\losan{\pspolygon(0.2,0.2)(5.2,0.2)(7.5,4.53)(2.7,4.53)}
\psset{unit=0.25cm,arrowsize=2pt 3}
\begin{pspicture}(4,3.8)
\rput(1.3,2.){\losan}
\rput(1.3,-2.4){Point de départ}
\psline[linecolor=blue]{->}(1.3,-1.4)(1.3,2)
\end{pspicture} 

\end{minipage}

\item Recopier et compléter les trois phrases suivantes afin d'associer chaque figure au programme qui permet de la tracer.

Le programme A permet de tracer la figure \ldots.

Le programme B permet de tracer la figure \ldots.

Le programme C permet de tracer la figure \ldots.
\end{enumerate}

\medskip

\begin{center}
\psset{linecolor=blue}
\begin{tabularx}{\linewidth}{*{3}{>{\centering \arraybackslash}X}}
\fbox{Programme A} & \fbox{Programme B} & \fbox{Programme C}\\
& & \\
\begin{scratch}[scale=0.75]
\blockinit{quand \greenflag est cliqu\'e}
\blockmove{aller à x: \ovalnum{0} y: \ovalnum{0}}
\blockmove{s'orienter à \ovalnum{90}}
\blockpen{effacer tout}
\blockpen{stylo en position d'\'ecriture}
\blockrepeat{r\'ep\'eter \ovalnum{5} fois}
{
\blockmoreblocks{Motif}
\blockmove{tourner \turnleft{} de \ovalnum{72} degr\'es}
}
\end{scratch}
&
\begin{scratch}[scale=0.75]
\blockinit{quand \greenflag est cliqu\'e}
\blockmove{aller à x: \ovalnum{0} y: \ovalnum{0}}
\blockmove{s'orienter à \ovalnum{90}}
\blockpen{effacer tout}
\blockpen{stylo en position d'\'ecriture}
\blockrepeat{r\'ep\'eter \ovalnum{5} fois}
{
\blockmoreblocks{Motif}
\blockmove{tourner \turnleft{} de \ovalnum{72} degr\'es}
\blockmove{avancer de \ovalnum{25 } pas}
}
\end{scratch}
&
\begin{scratch}[scale=0.75]
\blockinit{quand \greenflag est cliqu\'e}
\blockmove{aller à x: \ovalnum{0} y: \ovalnum{0}}
\blockmove{s'orienter à \ovalnum{90}}
\blockpen{effacer tout}
\blockpen{stylo en position d'écriture}
\blockrepeat{r\'ep\'eter \ovalnum{5} fois}
{
\blockmoreblocks{Motif}
\blockmove{avancer de \ovalnum{25 } pas}
}
\end{scratch}\\
&& \\
&& \\
Figure 1&Figure 2&Figure 3\\
\psset{unit=0.25cm}
\def\losan{\pspolygon(-1.8,7.2)(3.2,7.2)(5.7,11.53013)(0.7,11.53013)}
\begin{pspicture}(0,0)(2.5,10)
\multido{\n=-5+2.5}{4}{\rput(\n,0){\losan}}
\end{pspicture}
&
\psset{unit=0.25cm}
\def\losan{\pspolygon(-1.8,0.2)(3.2,0.2)(5.7,4.53013)(0.7,4.53013)}
\begin{pspicture}(-2,-10)(2,10)
\def\losan{\pspolygon(0,0)(5,0)(7.5,4.33013)(2.5,4.33013)}
\multido{\n=0+72}{5}{\rput{\n}(0,0){\losan}}
\end{pspicture}
&
\psset{unit=0.25cm}
\def\losan{\pspolygon(0,0)(5,0)(7.5,4.33013)(2.5,4.33013)}
\begin{pspicture}(-7,-10)(4,12)
%\psgrid
\multido{\i=18+72,\I=90+72,\n=72+72}{5}{
\psline(2.13;\i)(2.13;\I)
\rput{\n}(2.13;\i){\losan}}
\end{pspicture}
\\
\end{tabularx}
\end{center}
