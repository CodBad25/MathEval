
\medskip

\begin{minipage}{0.5\linewidth}
Un agriculteur possède un champ de blé ayant la forme
d'un triangle ABC rectangle en B représenté ci-contre.

On donne AB $= 200$ m et BC $= 150$ m.

Pour moissonner son champ, il utilise une moissonneuse batteuse qui, à chaque passage, coupe
des bandes de 12 mètres de large parallèles à la droite (AB). On a donc BE~$=~12$~m.

Il commence à passer le long du côté [AB]. Le segment en pointillés [DE] représente la limite du premier passage de la moissonneuse batteuse.

Après avoir fait 5 passages, il a moissonné le quadrilatère ABGF{}.

\medskip

\begin{enumerate}
\item 
	\begin{enumerate}
		\item Montrer que BG $= 60$ m.
		\item En déduire que CG $= 90$ m.
	\end{enumerate}	
\item Démontrer que la longueur GF est de $120$ m.
\end{enumerate}
\end{minipage}\hfill
\begin{minipage}{0.45\linewidth}
\psset{unit=1cm}
\begin{pspicture}(6.2,10.2)
%\psgrid
\pspolygon(0.1,0.2)(6.1,0.2)(0.1,10.3)
\uput[dr](6.1,0.2){C}
\psline(2.4,0.2)(2.4,6.4)
\psline[linestyle=dashed](0.56,0.2)(0.56,9.5)
\psline[linestyle=dashed](1.02,0.2)(1.02,8.8)
\psline[linestyle=dashed](1.48,0.2)(1.48,8)
\psline[linestyle=dashed](1.94,0.2)(1.94,7.2)
\multido{\n=0.33+0.46}{5}{\rput(\n,0.2){/}}
\uput[ul](0.1,10.3){A}\uput[dl](0.1,0.2){B}\psframe(0.1,0.2)(0.3,0.4)\psframe(2.4,0.2)(2.6,0.4)
\uput[d](0.56,0.2){E}\uput[ur](0.56,9.5){D}
\uput[d](2.4,0.2){G}\uput[ur](2.4,6.4){F}
\end{pspicture}
\end{minipage}

\begin{enumerate}[start=3]
\item
	\begin{enumerate}
		\item Démontrer que l'aire du triangle rectangle CGF est de \np{5400}~m$^2$.
		\item Le quadrilatère ABGF a une surface de \np{9600} m$^2$ qui a été moissonnée en 80 minutes.
		
On admet que le temps de travail de la moissonneuse batteuse est proportionnel à la surface moissonnée.

Calculer le temps de travail qu'il faut pour moissonner la partie restante CGF de son champ.
	\end{enumerate}
\item L'année suivante, il décide de clôturer son champ ABC afin d'y mettre des animaux pour l'été. Quelle longueur de clôture doit-il acheter ?
\end{enumerate}

\bigskip

