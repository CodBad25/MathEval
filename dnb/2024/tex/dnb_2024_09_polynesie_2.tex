
\medskip

On a relevé dans une feuille de calcul les températures maximales Tmax (en \degres{} C) atteintes à Strasbourg le 25 juin de chaque année de 2010 à 2018 (source : meteociel.fr).

\begin{center}
\begin{tabularx}{\linewidth}{|>{\centering\arraybackslash\cellcolor{lightgray}} c|c|*{9}{>{\centering \arraybackslash}X|}}\hline
\rowcolor{lightgray}	&A			&B		&C&D&E&F&G&H&I&J\\ \hline
1	&Année		&2010 	&2011	&2012	&2013	&2014	&2015	&2016	&2017	&2018\\ \hline
2	&Tmax		& 29 	&$23,1$	& $22,6$	&$17,4$	&$23,4$	&$25,7$	&$25,2$	&26		&24\\ \hline
3	&			&		&		&		&		&		&		&		&		&\\ \hline
4	&Moyenne	&		&		&		&		&		&		&		&		&\\ \hline
5	&Médiane	&24		&		&		&		&		&		&		&		&\\ \hline
6	&Étendue	&$11,6$	&		&		&		&		&		&		&		&\\ \hline
\end{tabularx}
\end{center}

\medskip

\begin{enumerate}
\item On a oublié de calculer la moyenne de cette série.

Quelle formule peut-on saisir dans la cellule B4 pour que ce calcul soit effectué ? 
\item Donner, sans détailler les calculs, une valeur approchée au degré Celsius près de la moyenne de la série.
\item Donner une interprétation de la médiane de cette série.
\item Pour cette question seulement, on considère la série des températures maximales atteintes à Strasbourg le 25 juin de chaque année de 2010 à 2019. 

On sait que l'étendue des températures de cette nouvelle série est égale à $18,5\degres{}$~C.

Déterminer la température maximale atteinte à Strasbourg le 25 juin 2019.
\end{enumerate}
\medskip

Les questions suivantes portent sur la série des températures maximales atteintes à Strasbourg le 25 juin de chaque année de 2010 à 2018.

\medskip

\begin{enumerate}[resume]
\item  On crée 9 fiches, une par année, sur lesquelles figure la température maximale atteinte le 25 juin de l'année. On prend une fiche au hasard. Chacune des fiches a la même probabilité d'être tirée.
	\begin{enumerate}
		\item Quelle est la probabilité que la température écrite sur cette fiche soit égale à $26\degres{}$~C ?
		\item Quelle est la probabilité que la température écrite sur cette fiche soit inférieure ou égale à 24$\degres{}$~C?
		\item A-t-on raison de dire que l'on a plus de 40\,\% de chance de prendre une fiche sur laquelle la température est supérieure à $25\degres{}$~C?
	\end{enumerate}
\end{enumerate}

%\bigskip

