
\medskip

%Cet exercice est un questionnaire à choix multiples (QCM).
%
%\textbf{Aucune justification n'est demandée}.
%
%Pour chaque question, trois réponses (A, B et C) sont proposées.
%
%\textbf{Une seule réponse est exacte.}
%
%Recopier sur la copie le numéro de la question et la réponse choisie.
%
%\begin{center}
%\begin{tabularx}{\linewidth}{|m{3.4cm}|m{3cm}|*{3}{>{\centering \arraybackslash}X|}}\hline
%\multicolumn{2}{|m{6.4cm}|}{}&Réponse A & Réponse B & Réponse C \\ \hline
%\multicolumn{2}{|m{6.4cm}|}{\textbf{Question 1}
%
%Quelle est l'écriture scientifique de \np{13420} ?} &
%$1,342 \times 10^{-4}$ & $1,342 \times 10^4$ & $\np{1342} \times 10^1$\\ \hline
%\multicolumn{2}{|m{6.4cm}|}{\textbf{Question 2}
%
%On a relevé, en mètres, les onze meilleures performances du lancer de marteau chez les hommes: 
%
%85,14~;~85,14~;~85,20~;~85,60~;~85,68~;~85,74~;
%
%86,04~;~86,34~;~86,51~;~86,66~;~86,74.
%
%Quelle est la médiane de cette série ?}&85,74&85,86&85,89\\ \hline
%
%\multirow{2}{3.2cm}{\psset{unit=0.8cm}
%\begin{pspicture}(-2.2,-2.4)(2.2,2.4)
%\pscircle(0,0){2.1}
%\def\ange{\pspolygon(0;0)(1.6;0)(2.1;11.25)(1.6;22.5)}
%\multido{\n=0.0+22.5}{16}{\rput{\n}(0;0){\ange}}
%\psline[linewidth=1.25pt](2.3;112.5)(2.3;292.5)\uput[ur](2.3;115){\small$(d)$}
%\pspolygon[fillstyle=solid,fillcolor=lightgray](0;0)(1.6;157.5)(2.1;168.75)(1.6;180)
%\multido{\n=1+1,\na=146.25+-22.50}{15}{\rput(1.2;\na){\footnotesize \n}}
%\end{pspicture}}&\textbf{Question 3} 
%Quelle est l'image du motif gris par la symétrie d'axe $(d)$ ?&Le motif 8&Le motif 15&Le motif 5
%\\ \cline{2-5}
%&\textbf{Question 4} 
%Quelle est l'image du motif gris par la rotation de centre O et d'angle $90\degres{}$ dans le sens antihoraire ?&Le motif 4&Le motif 12&Le motif 13\\ \hline
%\multirow{2}{3.2cm}{\psset{xunit=0.5cm,yunit=0.35cm,arrowsize=2pt 3}
%\begin{pspicture*}(-1.9,-2.9)(4,6)
%\psgrid[gridlabels=0pt,subgriddiv=1,gridwidth=0.15pt]
%\psaxes[linewidth=1.25pt,labelFontSize=\scriptstyle]{->}(0,0)(-0.9,-2.9)(4,6)
%\psplot[plotpoints=500,linewidth=1.25pt]{-2}{4}{4 2 x mul sub}
%\uput[dl](2.9,-1.3){\small $(d)$}
%\end{pspicture*}}&\textbf{Question 5} 
%
%Quelle est l'image de 2 par la fonction $f$ ? &0&1&4\\ \cline{2-5}
%&\textbf{Question 6}
%
%Quel est le coefficient directeur de la droite $(d)$ ?&2 &$-0,5$&$- 2$\\ \hline
%\end{tabularx}
%\end{center}
\textbf{Question 1} $\np{13420} = \np{1,342} \times 10^4$ : réponse B

\textbf{Question 2} La médiane est la sixième valeur qui partage les 10 performances en deux séries de 5 nombres : la médiane est donc 85,74 ; réponse A.

\textbf{Question 3} Le motif gris a pour symétrique le motif 5 : réponse C

\textbf{Question 4} Le motif gris a pour image le motif 12 : réponse B.

\textbf{Question 5} $f$ étant représentée par la droite $(d)$, 2 a pour image 0 : réponse A.

\textbf{Question 6} Le coefficient directeur de la droite peut se calculer avec les points de coordonnées (0~;~4) et (2~;~0), soit comme le quotient $\dfrac{0 - 4}{2 - 0} = \dfrac{-4}{2} = - 2$ : réponse C.
\bigskip

