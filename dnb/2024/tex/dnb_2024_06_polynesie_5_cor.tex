
\medskip

%Dans cet exercice, les deux parties sont indépendantes.
%
%On considère les fonctions $f$ et $g$ définies par 

\begin{center} $f(x) = (x + 2)^2 - x$\quad et \quad  $g(x) = 7x + 4$.\end{center}

%\smallskip

\textbf{Partie A}

\medskip

\begin{enumerate}
\item %Calculer 
$f(- 4) = (- 4 + 2)^2 - (-4) = 4 + 4 = 8$.
\item %Déterminer un antécédent de $3$ par la fonction $g$.
Il faut trouver $x$ tel que $g(x) =  7x + 4 = 3$ ou $7x = - 1$ et $x = - \dfrac17$.

Donc $g\left(- \dfrac17\right) = 3$.
\end{enumerate}

\medskip

\textbf{Partie B}

\medskip

Trois élèves, Paul, Jane et Morgane, cherchent à résoudre l'équation $f(x) = g(x)$ par trois méthodes différentes.

\medskip

\begin{enumerate}
\item %Paul utilise un tableur.
%{\Huge \rput{45}(0,0.5){+}}\vspace{-0.25cm}
Il calcule ainsi les images des entiers compris entre $-3$ et $3$ par les fonctions $f$ et $g$.

\begin{center}
\begin{tabularx}{\linewidth}{|l|*{8}{>{\centering \arraybackslash}X|}}\hline
	&A &B&C&D&E &F &G &H \\ \hline
1	&\cellcolor{black}&$-3$&$ -2$& $-1$& 0 &1 &2 &3\\ \hline
2	&$f(x)$&4&2 &2&4&8&14&22\\ \hline
3	& $g(x)$& $-17$& $-10$& $-3$& 4& 11& 18& 25\\ \hline
\end{tabularx}
\end{center}

	\begin{enumerate}
		\item %Quelle formule a-t-il saisie en cellule B3 puis étirée vers la droite pour compléter la ligne 3 du tableau ?
Il a écrit en B3 : =7*B1+4
		\item %Avec cette méthode, quelle(s) solution(s) trouve-t-il à l'équation $f(x) = g(x)$ ?
Sur cette partie du tableur il voit que les images par $f$ et $g$ sont les mêmes  (4) pour $x = 0$.
	\end{enumerate}
\item %Jane utilise un logiciel de programmation.

%Le programme qu'elle a créé permet de tester l'égalité $f(x) = g(x)$ pour une valeur de $x$ choisie par l'utilisateur. Ce programme se trouve en ......

%Elle décide de tester toutes les valeurs entières entre $-5$ et 3.
	\begin{enumerate}
		\item %Compléter sur l'AN...., à rendre avec la copie, la ligne 4 du programme de Jane afin d'obtenir l'image par la fonction $g$ du nombre choisi.

\begin{scratch}[num blocks]
\renewcommand*\numblock[1]{ligne \itshape#1}
\blockinit{quand \greenflag est cliqué}
\blockmove{demander \ovalnum{Choisir un nombre} et attendre}
\blockvariable{mettre \selectmenu{image par f} à \ovaloperator{\ovaloperator{\ovalmove{réponse}+\ovalnum{2}}*\ovaloperator{\ovalmove{réponse}+\ovalnum{2}}-\ovalmove{réponse}}}
\blockvariable{mettre \selectmenu{image par g} à \ovaloperator{\ovalnum{7}*\ovalmove{réponse}}+\ovalnum{4}}
 \blockifelse{si\ovaloperator{\ovalvariable{image par f} =\ovalvariable{image par g}}  alors}
 {\blocklook{dire \ovalnum{le nombre choisi est une solution de f(x)=g(x)} pendant \ovalnum{2} secondes}}
 {\blocklook{dire \ovalnum{le nombre choisi n'est pas une solution de f(x)=g(x)} pendant \ovalnum{2} secondes}}
\end{scratch}

		\item %Quelle réponse donne le programme si le nombre choisi est $0$ ?
Réponse : le nombre choisi est une solution de l'équation $f(x) = g(x)$.
		\item %En déduire une solution de l'équation $f(x) = g(x)$.
		On retrouve que 0 est une solution de l'équation $f(x) = g(x)$ puisque, d'après la question précédente, $f(0) = g(0)$
	\end{enumerate}
\item %Morgane décide de résoudre cette équation par le calcul.
	\begin{enumerate}
		\item %Démontrer que l'équation $f(x) = g(x)$ peut se ramener à l'équation $x^2 - 4x = 0$.
		On a $f(x) = g(x)$  si et seulement si $(x + 2)^2 - x = 7x + 4$ ou en développant :
		
		$x^2 + 4x + 4 - x = 7x + 4$ ou encore $x^2 - 4x = 0$
		\item %Factoriser l'expression $x^2 - 4x$.
		On a $x^2 - 4x = x(x - 4)$
		\item %En déduire les solutions de l'équation $f(x) = g(x)$.
		D'après le résultat précédent l'équation $x^2 - 4x = 0$ s'écrit 
		
		$x(x - 4) = 0 $. Or le produit est nul si l'un des deux facteurs est nul soit :
		
		$\left\{\begin{array}{l c l}
		x&=&0\\
		&\text{ou}&\\
		x - 4&=&0
		\end{array}\right.$ soit finalement $\left\{\begin{array}{l c l}
		x&=&0\\
		&\text{ou}&\\
		x &=&4
		\end{array}\right.$
		
L'équation a deux solutions  : 0 et 4.
	\end{enumerate}
\item %Dire pour chaque élève s'il a résolu l'équation $f(x) = g(x)$.

%Expliquer pourquoi.
$\bullet~$ Paul n'a pas trouvé toutes les solutions puisqu'il n'a cherché les solutions que parmi ceux  qui sont entiers, de $- 3$ à 3.

$\bullet~$Jane a trouvé la solution 4 mais pas la solution 0 : elle n'aura jamais la certitude d'avoir trouvé toutes les solutions puisqu'il lui est impossible d'introduite dans le programme tous les normes entiers.

$\bullet~$Morgane est certaine d'avoir trouvé toutes les solutions puisqu'elle a cherché les nombres solutions de l'équation $f(x) = g(x)$.
\end{enumerate}


