
\medskip

On considère la fonction $f$ définie par 
$f(x) = x^2 + 10x + 16$.

\begin{enumerate}
\item $f(6)=6^2+10\times 6 +16=36+60+16=112$, donc l'image de 6 par la fonction $f$ est $112$.
\item On utilise un tableur afin de calculer les images des entiers compris entre $-4$ et 4 par la fonction $f$.

\begin{center}
\begin{tabularx}{\linewidth}{|c|*{10}{>{\centering \arraybackslash}X|}}\hline
&A&B&C&$\blue \text{D}$ &E&F&G&H&I&J\\ \hline
1&$x$&$-4$&$-3$&$\blue -2$&$-1$&0&1&2&3&4\\ \hline
2&$f(x)$&$-8$&$-5$&$\blue 0$ &7&16&27&40&55&72\\ \hline
\end{tabularx}
\end{center}

	\begin{enumerate}
		\item Parmi les 4 formules ci-dessous, 

\begin{footnotesize}
\fbox{=B1*B1+10*B1+16}\hfill \fbox{=A1*A1+10*A1+16}\hfill  \fbox{$=(-4)*(-4)+10*(-4)+16$}\hfill \fbox{$=x*x+10*x+16$}
\end{footnotesize}

celle qui a été saisie dans la cellule B2, puis étirée vers la droite afin de calculer les images des nombres donnés par la fonction $f$est:
\fbox{=B1*B1+10*B1+16}

		\item D'après la colonne D du tableau, on peut dire que $-2$ est un antécédent de $0$ par la fonction $f$.
	\end{enumerate}
\item
	\begin{enumerate}
		\item $(x + 2)(x + 8) = x^2 + 2x + 8x + 16 = x^2+10x+16=f(x)$
		\item Pour déterminer les antécédents de 0 par la fonction $f$, on résout l'équation $f(x)=0$.
		
$\aligned
f(x)=0  & \text{ si et seulement si }  (x+2)(x+8)=0 \\
& \text{ si et seulement si } x+2=0  \text{ ou } x+8=0  \\
& \text{ si et seulement si } x=-2 \text{ ou } x=-8
\endaligned$

Le nombre $-8$ est donc un autre antécédent de 0 par la fonction $f$.
	\end{enumerate}
\end{enumerate}


