
\medskip

Un agriculteur possède un champ de blé ayant la forme
d'un triangle ABC rectangle en B représenté ci-contre.
On donne AB $= 200$ m et BC $= 150$ m.

Pour moissonner son champ, il utilise une moissonneuse batteuse qui, à chaque passage, coupe des bandes de 12 mètres de large parallèles à la droite (AB). On a donc BE~$=~12$~m.

Il commence à passer le long du côté [AB]. Le segment en pointillés [DE] représente la limite du premier passage de la moissonneuse batteuse.

Après avoir fait 5 passages, il a moissonné le quadrilatère ABGF{}.

\medskip

\begin{minipage}{0.65\linewidth}
\begin{enumerate}
\item 
	\begin{enumerate}
		\item $\text{BG}=5\text{BE}=5\times 12=60$ donc $\text{BG}=60$~m.
		
%		Montrer que BG $= 60$ m.
		\item $\text{CG} = \text{BC} - \text{BG} = 150-60=90$ donc $\text{CG}= 90$ m.
	\end{enumerate}	
\item  %Démontrer que la longueur GF est de $120$ m.
\begin{list}{\textbullet}{Dans les triangles ABC et FGC:}
\item (AB) et (FG) sont parallèles;
\item B, G et C sont alignés dans cet ordre;
\item A, F et C sont alignés dans cet ordre.
\end{list}

On peut donc appliquer le théorème de Thalès:

$\dfrac{\text{GF}}{\text{BA}}=\dfrac{\text{CG}}{\text{CB}}$
soit
$\dfrac{\text{GF}}{200}=\dfrac{90}{150}$
donc
$\text{GF}=\dfrac{90}{150}\times 200=120$

La longueur GF est donc de $120$ m.
	\end{enumerate}

\end{minipage}\hfill
\begin{minipage}{0.3\linewidth}
\scalebox{0.8}{
\psset{unit=0.7cm}
\begin{pspicture}(6.2,11)
%\psgrid
\pspolygon(0.1,0.2)(6.1,0.2)(0.1,10.3)
\uput[dr](6.1,0.2){C}
\psline(2.4,0.2)(2.4,6.4)
\psline[linestyle=dashed](0.56,0.2)(0.56,9.5)
\psline[linestyle=dashed](1.02,0.2)(1.02,8.8)
\psline[linestyle=dashed](1.48,0.2)(1.48,8)
\psline[linestyle=dashed](1.94,0.2)(1.94,7.2)
\multido{\n=0.33+0.46}{5}{\rput(\n,0.2){/}}
\uput[ul](0.1,10.3){A}\uput[dl](0.1,0.2){B}\psframe(0.1,0.2)(0.3,0.4)\psframe(2.4,0.2)(2.6,0.4)
\uput[d](0.56,0.2){E}\uput[ur](0.56,9.5){D}
\uput[d](2.4,0.2){G}\uput[ur](2.4,6.4){F}
\end{pspicture}
}%% fin du scalebox
\end{minipage}

\begin{enumerate}[start=3]
\item
	\begin{enumerate}
		\item Le triangle CGF est rectangle en G donc son aire est:
		$\dfrac{\text{GF}\times \text{CG}}{2} = \dfrac{120\times 90}{2}=\np{5400}$.
		
L'aire du triangle rectangle CGF est donc de \np{5400}~m$^2$.
		\item Le quadrilatère ABGF a une surface de \np{9600} m$^2$ qui a été moissonnée en 80 minutes, et on admet que le temps de travail de la moissonneuse batteuse est proportionnel à la surface moissonnée.

%Calculer le temps de travail qu'il faut pour moissonner la partie restante CGF de son champ.

On établit un tableau de proportionnalité: \hfill
{\renewcommand{\arraystretch}{1.5}
\begin{tabular}{|c|c|c|}
\hline
Surface (m$^2$) & \np{9600} & \np{5400}\\
\hline
Temps (min) & 80 & ?\\
\hline
\end{tabular}
}

\begin{tabularx}{\linewidth}{@{} l @{~} X}
$\dfrac{\np{5400} \times \np{80}}{\np{9600}}=45$ & donc  le temps de travail qu'il faut pour moissonner la partie restante CGF de son champ est de 45 minutes.
\end{tabularx}

	\end{enumerate}
	
\item L'année suivante, il décide de clôturer son champ ABC.
 %Quelle longueur de clôture doit-il acheter ?

La longueur de la clôture est $\text{AB} + \text{BC} + \text{AC}=200+150 + \text{AC} = 350+\text{AC}$.

Le triangle ABC est rectangle en B donc, d'après le théorème de Pythagore, on a:

$\text{AC}^2 = \text{AB}^2 +  \text{BC}^2 = 200^2 + 150^2=\np{62500}$; donc $\text{AC}=\ds\sqrt{\np{62500}}=250$

$350+250=600$ donc il faut acheter 600 mètres de clôture.
\end{enumerate}

\vspace{0.5cm}

