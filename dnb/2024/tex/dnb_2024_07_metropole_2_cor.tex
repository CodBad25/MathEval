
\medskip

%\begin{tabularx}{\linewidth}{|X|l|}\hline
%Programme A &Programme B\\ \hline
%\begin{itemize}
%\item Choisir un nombre.
%\item Prendre le carré du nombre choisi.
%\item Multiplier le résultat par 2.
%\item Ajouter le double du nombre
%de départ.
%\item Soustraire 4 au résultat.
%\end{itemize}&\begin{scratch}
%\setscratch{scale=0.7,num blocks=true,print,fill blocks, fill gray=0.9}
%	\blockinit{quand \greenflag est cliqué}
%	\blocksensing{demander \ovalnum{Choisir un nombre}~ et attendre}
%	\blockvariable{mettre \selectmenu{nombre choisi} à \ovalsensing{réponse}}
%	\blockvariable{mettre \selectmenu{Résultat 1} à \ovaloperator{\ovalvariable{Nombre choisi} + \ovalnum{2} }}
%	\blockvariable{mettre \selectmenu{Résultat 2} à \ovaloperator{\ovalvariable{Nombre choisi} - \ovalnum{1} }}
%	\blocklook{dire \ovaloperator{regrouper \ovalnum{Le résultat est} et \ovaloperator{\ovalvariable{Résultat 1} * \ovalvariable{Résultat 2} }}}
%\end{scratch}
%\\ \hline		
%\end{tabularx}

\medskip

\begin{enumerate}
\item
	\begin{enumerate}
		\item %Vérifier que, si on choisit 5 comme nombre de départ, le résultat du programme A est 56.
On obtient successivement : $5 \to 5^2 = 25  \to 2 \times 25 = 50  \to 50 + 2\times 5 = 50 + 10 = 60  \to 60 - 4 = 56$.
		\item %Quel résultat obtient-on avec le programme B si on choisit $-9$ comme nombre de départ ?
Avec $- 9$ au départ on obtient $- 9 + 2 = - 7$ en résultat 1 et $- 9 - 1 = - 10$ en résultat 2.
		
Le résultat final est $- 7 \times  (-10) = 70$.
	\end{enumerate}
\item %On choisit un nombre quelconque $x$ comme nombre de départ.
	\begin{enumerate}
		\item %Parmi les trois propositions ci-dessous, recopier l'expression qui donne le résultat obtenu par le programme B ?
		
%\[E_1 = (x + 2) - 1\qquad E_2 = (x + 2) \times (x - 1)\qquad  E_3 = x + 2 \times x - 1\]
Résultat 1 = $x + 2$ ; Résultat 2 : $x - 1$ et résultat final = $(x + 2)(x - 1)$ soit $E_2$.
		\item %Exprimer en fonction de $x$ le résultat obtenu avec le programme A.
On obtient successivement : $x  \to x^2  \to 2 \times x^2 = 2x^2  \to 2x^2 + 2x  \to 2x^2 + 2x - 4$.
	\end{enumerate}
\item %Démontrer que, quel que soit le nombre choisi au départ, le résultat du programme A est toujours le double du résultat du programme B.
Le résultat avec le programme A est :

$2x^2 + 2x - 4 = 2 \times x^2 + 2 \times x - 2 \times 2 = 2\left(x^2 + x - 2\right)$.

Or en développant $E_2 = (x + 2)(x - 1) = x^2 + 2x - x - 2 = x^2 + x - 2$
 : c'est le résultat du programme B.

Le résultat du programme A est le double du résultat du programme B.
\end{enumerate}

\medskip

\bigskip

