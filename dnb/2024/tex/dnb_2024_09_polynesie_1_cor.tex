
%\medskip
%
%\emph{Dans cet exercice, toutes les questions sont indépendantes.}

\medskip

\begin{enumerate}
\item On a décomposé ci-dessous cinq nombres en produits de facteurs premiers.

On cherche ceux qui sont divisibles par 21.

\begin{center}
\renewcommand{\arraystretch}{1.6}
\begin{tabularx}{\linewidth}{|*{5}{>{\centering \arraybackslash}X|}}\hline
Nombre 1& Nombre 2& Nombre 3& Nombre 4& Nombre 5\\ \hline
$2^2 \times 11 \times 23$ &$2^4 \times 3^4 \times 11$	&$7^3 \times 13 \times 17$
&$2 \times 3 \times 5 \times 7$&$2^3 \times 3^2 \times 7$\\ \hline
\end{tabularx}
\end{center}

\begin{list}{\textbullet}{}
\item Nombre 4: $2 \times 3 \times 5 \times 7 = 2 \times 5 \times 3 \times 7 = 2 \times 5 \times 21$ donc divisible par 21.
\item Nombre 5: $2^3 \times 3^2 \times 7 = 2^3 \times 3\times 3 \times 7 = 2^3 \times 3\times 21$ donc divisible par 21.
\end{list}

\item L'écriture scientifique du nombre \np{0,00000276} est $2,76\times 10^{-6}$.

\item La comète Hale-Bopp a atteint la vitesse de \np{2640} km/min. %Quelle est sa vitesse en m/s ? 

En km/s, sa vitesse est de $\dfrac{\np{2640}}{60}$, et en m/s, elle est de $\dfrac{\np{2640}}{60}\times \np{1000}$ soit $\np{44000}$.

\item Les solutions de l'équation $(2x - 7)(3x + 1) = 0$ sont les nombres $x$ tels que $2x-7=0$ ou $3x+1=0$.

\begin{list}{\textbullet}{}
\item $2x-7=0$ équivaut à $2x=7$ qui équivaut à $x=\dfrac{7}{2}$.
\item $3x+1=0$ équivaut à $3x=-1$ qui équivaut à $x=-\dfrac{1}{3}$.
\end{list}

L'équation $(2x - 7)(3x + 1) = 0$ a donc pour solutions $x=\dfrac{7}{2}$ et $x=-\dfrac{1}{3}$.

\item On considère la fonction $f$ définie par $f(x) = 5x^2 + 2$.

L'image de $-3$ par la fonction $f$ est: $f(-3)=5\times (-3)^2+2=5\times 9+2 = 45+2=47$.
\end{enumerate}
\begin{minipage}{0.48\linewidth}
\begin{enumerate}[resume]
\item Sur la figure ci-contre (qui n'est pas à l'échelle) :
\begin{itemize}[label=$\bullet~$]
\item les points A, E et C sont alignés;
\item les points B, D et C sont alignés;
\item les droites (AB) et (ED) sont parallèles;
\item AB $= 5$ cm, BD $= 1$ cm, CD $= 3$ cm.
\end{itemize}
\end{enumerate}
\end{minipage}\hfill
\begin{minipage}{0.48\linewidth}
\psset{unit=1cm}
\begin{pspicture}(6.6,4.4)
\pspolygon(0.4,2.2)(4.8,3.7)(6.2,0.4)%ABC
\psline(1.72,1.8)(5.1,3)
\uput[l](0.4,2.2){A} \uput[u](4.8,3.7){B} \uput[dr](6.2,0.4){C} \uput[r](5.1,3){D}
\uput[dl](1.72,1.8){E}
\uput[u](2.6,2.95){5 cm} \uput[r](4.95,3.35){1 cm} \uput[r](5.65,1.7){3 cm}
\end{pspicture}
\end{minipage}

\begin{enumerate}[label=]
\item

Les points A, E et C sont alignés,  les points B, D et C sont alignés, et les droites (AB) et (ED) sont parallèles, donc on peut appliquer le théorème de Thalès dans les triangles  CDE et CBA: $\dfrac{\text{CD}}{\text{CB}} = \dfrac{\text{DE}}{\text{BA}}$.

$\text{CD} = 3$, $\text{DB}=1$ et $\text{CB} = \text{CD} + \text{DB}$ donc $\text{CB}= 3+1=4$

De plus, $\text{BA}=5$ donc $\dfrac{3}{4} = \dfrac{\text{DE}}{5}$ donc $\text{DE} = \dfrac{5\times 3}{4}=3,75$.

Donc DE mesure $3,75$ cm.
\end{enumerate}

\bigskip


