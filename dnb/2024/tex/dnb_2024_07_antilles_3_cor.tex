
\medskip

%Sur la figure ci-après, qui n'est pas à l'échelle, on a représenté le trajet de la course que doit faire Oscar.
%
%\medskip

\begin{minipage}{0.48\linewidth}
Dans le triangle DLA rectangle en L, le point J appartient au segment [DA] et le
point K appartient au segment [DL].

On donne : 
\phantom{On donne :}
\begin{tabular}{l}
DL = 600~m ;\\
KJ = 200~m ;\\
DJ = 520 m ;\\
KL = 120 m.
\end{tabular}
\end{minipage}\hfill
\begin{minipage}{0.48\linewidth}
\psset{unit=0.85cm,arrowsize=2pt 3}
\begin{pspicture}(8,6.8)
%\psgrid
\psline[linewidth=1.25pt]{->}(1.5,0.5)(3.5,3.5)
\psline[linewidth=1.25pt]{->}(3.5,3.5)(5.2,2.3)
\psline[linewidth=1.25pt]{->}(5.2,2.3)(7.9,3.6)
\uput[d](1.5,0.5){D} \uput[ul](3.5,3.5){K} \uput[dr](5.2,2.3){J} \uput[r](7.9,3.6){A}
\uput[u](5,5.8){L}
\rput{-128}(5,5.8){\psframe(0.3,0.3)}
\psline[linestyle=dashed](3.5,3.5)(5,5.8)(7.9,3.6)
\psline[linestyle=dashed](1.5,0.5)(5.2,2.3)
\psline[linewidth=0.6pt]{<->}(0.5,1.3)(4,6.6)\rput{55}(2.1,4.1){600 m}
\rput{57}(4.5,4.65){120 m}\rput{-38}(4.55,3.1){200 m}\rput{28}(3.55,1.3){520 m}
\end{pspicture}
\end{minipage}

\medskip

\begin{enumerate}
\item %Montrer que la longueur DK est égale à 480~m.
On a DK + KL = DL soit $\text{DK} + 120 = 600$, d'où DK $= 600 - 120 = 480$~(m).
\item %Montrer que le triangle DKJ est rectangle en K.
On a $\text{DK}^2 + \text{KJ}^2 = 480^2 + 200^2 = \np{230400} + \np{40000} = \np{270400}$ et DJ$^2 = 520^2 = \np{270400}$.

On a donc $\text{DK}^2 + \text{KJ}^2 = \text{DJ}^2$ : d'après la réciproque du théorème de Pythagore le triangle DKJ est rectangle en K.
\item %Justifier que les droites (KJ) et (LA) sont parallèles.
Les droites (LA) et (KJ) sont perpendiculaires à la même droite (DL) : elles sont donc parallèles.
\item %Montrer que le segment [DA] mesure 650~m.
Les droites (LA) et (KJ) sont parallèles, les points D, K et L sont alignés et les points D, J et A le sont aussi : on a donc une configuration de Thalès  : on peut donc écrire l'égalité :

$\dfrac{\text{DK}}{\text{DL}} = \dfrac{\text{DJ}}{\text{DA}}$, soit $\dfrac{480}{600} = \dfrac{520}{\text{DA}}$, d'où $\text{DA}\times 480 = 600 \times 520$ puis $\text{DA} = \dfrac{600 \times 520}{480} = 650$~(m).
\item %Calculer la longueur du trajet DKJA, fléché sur la figure.
La longueur du trajet fléché est :

DK + KJ + JA = $480 + 200 + (650 - 520) = 810$.
\item %Un photographe place une caméra au point D. Afin de filmer l'ensemble de la course
%sans bouger la caméra, l'angle $\widehat{\text{LDA}}$ doit être inférieur à $25\degres{}$.

% Est-ce le cas ?
Dans le triangle rectangle LDA, on a DA $ = \text{DJ}  + \text{JA} = 520 + 130 = 650$ et par exemple : $\cos(\widehat{\text{LDA}}) = \dfrac{\text{long. côté adjacent}}{\text{long. hypoténuse}} = \dfrac{600}{650} = \dfrac{60}{65} = \dfrac{12}{13}$

La calculatrice donne $\widehat{\text{LDA}} \approx 22,6^\circ$.

Cette valeur est inférieure à 25 : le photographe pourra tout filmer sans bouger sa caméra.
\end{enumerate}

\bigskip

