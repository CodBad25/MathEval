
%\medskip

\begin{enumerate}
\item
\begin{multicols}{2}
 Sur la figure ci-contre, DEFGH est un pentagone régulier et le point E appartient à la demi-droite [D$y$). \\
\\
On admet que tous les angles du pentagone régulier mesurent $108$ degrés.

\columnbreak

\begin{center}
\psset{unit=1cm}
\begin{pspicture}(-2,-2)(3.5,2.1)
\pspolygon(1.7;18)(1.7;90)(1.7;162)(1.7;234)(1.7;306)%FGHDE
\uput[dl](1.7;234){D} \uput[d](1.7;306){E} \uput[r](1.7;18){F}
\uput[u](1.7;90){G} \uput[l](1.7;162){H} \uput[d](3.2,-1.39){$y$}
\psline(1.7;306)(3.3,-1.39)
\psarc(1.7;306){0.7}{0}{72}
\end{pspicture}
\end{center}
\end{multicols}

%\medskip

$\widehat{\text{DEF}}+ \widehat{\text{FE}y}=180$;
or $\widehat{\text{DEF}}=108$ donc $\widehat{\text{FE}y}=180-108 = 72$

L'angle $\widehat{\text{FE}y}$  mesure donc $72$ degrés.

\item% Dans la suite de cet exercice, aucune justification n'est attendue.
	\begin{enumerate}
		\item On complète les lignes 3 et 5 du bloc \og pentagone \fg{}  pour obtenir un pentagone régulier.
		
%La variable \og longueur \fg{} permet de modifier la longueur des côtés du pentagone.
		
\hfill
\begin{scratch}[num blocks,scale=1]
\initmoreblocks{définir \namemoreblocks{pentagone}}
\blockpen{stylo en position d'écriture}
\blockrepeat{répéter \ovalnum{\red\bf 5} fois}
{\blockmove{avancer de \ovalnum{longueur} pas}
\blockmove{tourner \turnleft{} de \ovalnum{\red\bf 72} degrés}
}
\blockpen{relever le stylo}
\end{scratch}
\hfill~

		\item Camille, Lou et Zoé ont chacun codé un programme qui trace un pentagone et son image par l'une des transformations suivantes : translation, symétrie centrale, rotation.

\begin{center}
\begin{tabularx}{\linewidth}{|*{3}{X|}}
\hline
Programme de Camille&Programme de Lou&Programme de Zoé\\ 
\hline
\begin{scratch}[num blocks,scale=0.8]
\blockinit{Quand \greenflag est cliqué}
\blockpen{effacer tout}
\blockmove{aller à x : \ovalnum{0} y : \ovalnum{0}}
\blockmove{s'orienter à ~\ovalnum{90}}
\blockvariable{mettre \selectmenu{longueur} à \ovalnum{60}}
\blockmoreblocks{pentagone}
\blockmove{avancer de \ovalnum{120} pas}
\blockmoreblocks{pentagone}
\end{scratch}
&
\begin{scratch}[num blocks,scale=0.8]
\blockinit{Quand \greenflag est cliqué}
\blockpen{effacer tout}
\blockmove{aller à x : \ovalnum{0} y : \ovalnum{0}}
\blockmove{s'orienter à ~\ovalnum{90}}
\blockvariable{mettre \selectmenu{longueur} à \ovalnum{60}}
\blockmoreblocks{pentagone}
\blockmove{tourner \turnleft{} de \ovalnum{60} degrés}
\blockmoreblocks{pentagone}
\end{scratch}
&
\begin{scratch}[num blocks,scale=0.8]
\blockinit{Quand \greenflag est cliqué}
\blockpen{effacer tout}
\blockmove{aller à x : \ovalnum{0} y : \ovalnum{0}}
\blockmove{s'orienter à ~\ovalnum{90}}
\blockvariable{mettre \selectmenu{longueur} à \ovalnum{60}}
\blockmoreblocks{pentagone}
\blockmove{tourner \turnleft{} de \ovalnum{180} degrés}
\blockmoreblocks{pentagone}
\end{scratch}\\ 
\hline
\end{tabularx}
\end{center}

%On rappelle que l'instruction \og s'orienter à 90 \fg{} signifie que l'on s'oriente vers la droite.

Les trois élèves ont effectué une copie d'écran de ce qu'ils ont obtenu sans indiquer ni leur prénom ni le nom de la transformation choisie.

\begin{center}
\begin{tabularx}{\linewidth}{|*{3}{X|}}\hline
Copie d'écran 1 &Copie d'écran 2& Copie d'écran 3\\ \hline
\begin{center}
\psset{unit=0.75cm}
\begin{pspicture}(-2,-2)(2,2)
\def\penta{\pspolygon(1.;18)(1.;90)(1.;162)(1.;234)(1.;306)}
\rput{-12}(-0.3945,0.1045){\penta}\rput(0.6,0.){\penta}
\end{pspicture}
\end{center}
&
\begin{center}
\psset{unit=0.75cm}
\begin{pspicture}(-2,-2)(2,2)
\def\penta{\pspolygon(1.;18)(1.;90)(1.;162)(1.;234)(1.;306)}
\rput(0.6,0.6){\penta}\rput{180}(-0.6,-1){\penta}
\end{pspicture}
\end{center}
&
\begin{center}
\psset{unit=0.75cm}
\begin{pspicture}(-2,-2)(2,2)
\def\penta{\pspolygon(1.;18)(1.;90)(1.;162)(1.;234)(1.;306)}
\rput(-1.2,0){\penta}\rput(1.2,0){\penta}
\end{pspicture}
\end{center}\\ \hline
\end{tabularx}
\end{center}

On complète le tableau ci-dessous, en associant le prénom de l'élève au numéro de sa copie d'écran ainsi qu'au nom de la transformation qu'il a choisie.

\begin{center}
{\renewcommand{\arraystretch}{1.5}
\begin{tabular}[t]{|c|c|c|}
\hline
Nom de l'élève&Numéro de la copie d'écran&Nom de la transformation\\ 
\hline
Camille& \textcolor{blue}{3} & \textcolor{blue}{translation}\\ 
\hline 
Lou & \textcolor{blue}{1} & \textcolor{blue}{rotation}\\ 
\hline 
Zoé& \textcolor{blue}{2} & \textcolor{blue}{symétrie centrale}\\ 
\hline 
\end{tabular}}
\end{center}

\item Sofia souhaite illustrer à l'aide d'un programme l'effet d'une homothétie sur un pentagone. 

%Le tableau donne, dans le désordre, toutes les instructions utiles pour écrire ce programme. L'ordre d'apparition dans le programme de deux instructions est précisé.

On complète le tableau, en indiquant l'ordre d'apparition de chacune des instructions dans le programme de Sofia.

\begin{center}
%\renewcommand{\arraystretch}{2.5}
\begin{tabular}{|l| >{\centering\arraybackslash}m{4cm}|}
\hline
\hfill Instruction \hfill~ &Ordre d'apparition de l'instruction dans le programme de Sofia\\ 
\hline
\begin{scratch}[scale=1]\blockpen{effacer tout} \end{scratch}& 2\ieme \\ 
\hline
\begin{scratch}[scale=1]\blockmove{s'orienter à \ovalnum{90} } \end{scratch}& 4\ieme \\ 
\hline
\begin{scratch}[scale=1]\blockmoreblocks{pentagone} \end{scratch}&6\ieme\\ 
\hline
\begin{scratch}[scale=1]\blockinit{Quand \greenflag est cliqué} \end{scratch}&\multicolumn{1}{|c|}{1\up{re}}\\ 
\hline
\begin{scratch}[scale=1]\blockvariable{mettre \selectmenu{longueur} à \ovalnum{60}} \end{scratch}& 5\ieme\\ 
\hline
\begin{scratch}[scale=1]\blockmove{aller à x : \ovalnum{0} y : \ovalnum{0}} \end{scratch}& 3\ieme\\ 
\hline
\begin{scratch}[scale=1]\blockmoreblocks{pentagone} \end{scratch}& 8\ieme \\ 
\hline
\begin{scratch}[scale=1]\blockvariable{mettre \selectmenu{longueur} à \ovaloperator{\ovalvariable{longueur} *\ovalnum{1.5}}} \end{scratch}& 7\ieme\\ 
\hline
\end{tabular}
\end{center}

Ce qui donne le programme suivant:

\begin{center}
\begin{scratch}[scale=1]
\blockinit{Quand \greenflag est cliqué} 
\blockpen{effacer tout} 
\blockmove{aller à x : \ovalnum{0} y : \ovalnum{0}} 
\blockmove{s'orienter à \ovalnum{90} } 
\blockvariable{mettre \selectmenu{longueur} à \ovalnum{60}} 
\blockmoreblocks{pentagone} 
\blockvariable{mettre \selectmenu{longueur} à \ovaloperator{\ovalvariable{longueur} *\ovalnum{1.5}}} 
\blockmoreblocks{pentagone} 
\end{scratch}

\end{center}

	\end{enumerate}
\end{enumerate}

%\bigskip


