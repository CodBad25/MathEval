
\medskip

\begin{minipage}{0.58\linewidth}
Sur la figure ci-contre, qui n'est pas à l'échelle,
\begin{itemize}[label=$\bullet~$]
\item le triangle ONM est rectangle en N,
\item le triangle OPQ est rectangle en P{},
\item le triangle ORS est rectangle en R,
\item ON $= 6$ cm et $\widehat{\text{MON}} = 32\degres{}$.
\item P est un point du segment [OM] et R est un point du segment [OQ].
\end{itemize}
\end{minipage}\hfill
\begin{minipage}{0.42\linewidth}
\psset{unit=1.1cm}
\begin{pspicture}(4.5,4.1)
%\psgrid
\pspolygon(0.3,1.3)(0.3,3.6)(4.1,3.6)%%MNO
\psline(0.82,1.62)(1.7,0.3)(4.1,3.6)%PQO
\psline(3.04,2.18)(3.65,1.7)(4.1,3.6)%RSO
\psframe(0.3,3.6)(0.5,3.4)\rput{-57}(0.82,1.62){\psframe(0.2,0.2)}
\rput{-38}(3.04,2.18){\psframe(0.2,0.2)}
\psarc(4.1,3.6){0.8}{180}{212}
\uput[d](0.3,1.3){M} \uput[ul](0.3,3.6){N} \uput[ur](4.1,3.6){O}
\uput[d](0.78,1.62){P} \uput[d](1.7,0.3){Q} \uput[d](3.09,2.18){R}
\uput[d](3.65,1.7){S} \uput[u](2.2,3.6){6 cm} \rput(3.65,3.45){\footnotesize $32\degres$}
\end{pspicture}
\end{minipage}

\medskip

\begin{enumerate}
\item% Calculer la mesure de la longueur MN. On donnera une valeur approchée au millimètre près.
Dans le triangle OMN rectangle en N, on a:

$\tan \left (\widehat{\text{MON}}\right ) = \dfrac{\text{MN}}{\text{ON}}$ donc
$\text{MN} = \text{ON}\times \tan \left (\widehat{\text{MON}}\right ) = 6 \times \tan \left (32\right ) \approx 3,7$.

\item On donne PQ $= 2,5$ cm et OQ $= 6,5$ cm. %Montrer que OP $= 6$ cm.

On applique le théorème de Pythagore dans le triangle OPQ rectangle en P:

$\text{OP}^2 + \text{PQ}^2 =\text{OQ}^2$ donc $\text{OP}^2 =\text{OQ}^2 -\text{PQ}^2 = 6,5^2-2,5^2 = 42,25 - 6,25 = 36$ \\
donc $\text{OP} = 6$~cm. 

\item% Montrer que les triangles ONM et OPQ ne sont pas des triangles égaux.
$\text{ON} = \text{OP} = 6$ mais $\text{MN}\approx 3,7$ et$\text{PQ}=2,5$ donc $\text{MN} \neq \text{PQ}$; les triangles rectangles ONM et OPQ n'ont pas leurs côtés de l'angle droit égaux, donc ce ne sont pas des triangles égaux.

\item On sait que le triangle OPQ est un agrandissement du triangle ORS et que OS $= 3,25$ cm.%, calculer l'aire du triangle ORS.

OS est l'hypoténuse du triangle ORS et OS $=3,25$~cm. 
OQ est l'hypoténuse du triangle OPQ et OQ $=6,5$~cm. 
Comme $6,5=2\times 3,25$, on peut dire que le triangle OPQ est un agrandissement du triangle ORS de facteur 2, et donc que l'aire du triangle OPQ est 4 fois plus grande que l'aire du triangle ORS.

L'aire du triangle OPQ est: $\dfrac{\text{OP}\times \text{PQ}}{2} = \dfrac{6\times 2,5}{2}= 7,5$.

L'aire du triangle OPQ est 4 fois plus grande que l'aire du triangle ORS donc l'aire du triangle ORS est 4 fois plus petite que l'aire du triangle OPQ, donc est égale à:
$\dfrac{7,5}{4}$ c'est-à-dire $\np{1,875}$~cm$^2$.
\end{enumerate}

\bigskip

