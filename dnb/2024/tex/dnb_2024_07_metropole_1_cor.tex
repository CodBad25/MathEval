
\medskip

%Au casino, la roulette est un jeu de hasard pour lequel chaque joueur mise au choix sur un ou plusieurs numéros.
%
%On lance une bille sur une roue qui tourne, numérotée de 0 à 36.
%
%La bille a la même probabilité de s'arrêter sur chaque numéro.

%\begin{center}
%\includegraphics[width=7.4cm]{roulette.eps}
%\psset{unit=1cm}
%
%\begin{pspicture}(-4,-4)(4,4)
%\pscircle(0,0){1.8}
%\pscircle(0,0){2.5}
%\pscircle(0,0){3.4}\pscircle(0,0){3.55}
%\pscircle[linewidth=2pt](0,0){3.75}
%\multido{\n=-2.125+19.46,\na=17.3345+19.46}{19}{\pscustom[fillstyle=solid,fillcolor=black]{\psline(1.8;\n)(3.4;\n)\psarc(0,0){3.4}{\n}{\na}\psline(3.4;\na)(1.8;\na)\psarc(0,0){1.8}{\na}{\n}}}
%
%\end{pspicture}
%\end{center}

\medskip

\begin{enumerate}
\item %Expliquer pourquoi la probabilité que la bille s'arrête sur le numéro 7 est $\dfrac{1}{37}$.
De 0 à 36 il y a $36 - 0 + 1 = 37$ nombres.

La bille a la même probabilité de s'arrêter sur l'une de ces 37 cases, la probabilité de chaque nombre est donc égale à $\dfrac{1}{37}$.
\item %Déterminer la probabilité que la bille s'arrête sur une case à la fois noire et paire.
Il y a 19 cases blanches donc 18 cases noires et parmi elles 2~;~4~;~6~;~8~;~10~;~20~;~22~;~24~;~26~;~28 soit 10 chances sur 37, donc la probabilité est égale à $\dfrac{10}{37}$.
\item 
	\begin{enumerate}
		\item %Déterminer la probabilité que la bille s'arrête sur un numéro inférieur ou égal à 6.
De 0 à 6 il y a 7 nombres donc la probabilité est égale à $\dfrac{7}{37}$
		\item %En déduire la probabilité que la bille s'arrête sur un numéro supérieur ou égal à 7.
La probabilité que la bille s'arrête sur un numéro supérieur ou égal à 7 est égale à $1 - \dfrac{7}{37} = \dfrac{30}{37}$.
		\item %Un joueur affirme qu'on a plus de 3 chances sur 4 d'obtenir un numéro supérieur ou égal à 7. A-t-il raison ?
On a $\dfrac{30}{37} \approx 0,81 > 0,8 > 0,75 = \dfrac34$ : le joueur a raison.

\emph{Remarque} $\dfrac34 = \dfrac{3 \times 10}{4 \times 10} = \dfrac{30}{40}$.

Or $37 < 40 \Rightarrow \dfrac{1}{40} < \dfrac{1}{37}$ et en multipliant chaque membre par 30 : $\dfrac{30}{40} < \dfrac{30}{37}$.
	\end{enumerate}
\end{enumerate}

\bigskip

