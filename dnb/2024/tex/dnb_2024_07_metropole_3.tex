
\medskip

Sur la figure ci-dessous, on a :

\medskip

\begin{itemize}
\item $\mathcal{C}$ est un cercle de centre O et de rayon $4,5$~cm ;
\item $[\text{AB}]$ est un diamètre de ce cercle et D est un point du cercle ;
\item les points B, E, A sont alignés, ainsi que les points D, F{}, A ;
\item les droites (BD) et (EF) sont parallèles ;
\item BD $= 5,4$ cm ~;~ DA $= 7,2$ cm \:et\: AE $= 2,7$ cm.
\end{itemize}

\begin{center}
\psset{unit=0.75cm}
\begin{pspicture}(-4,-4)(4,4)
%\psgrid
\pscircle(0,0){4}\uput[ur](4;50){$\mathcal{C}$}
\pspolygon(4;0)(4;100)(4;180)%ADB
\uput[r](4;0){A} \uput[l](4;180){B} \uput[ul](4;100){D}
\uput[d](1.3,0){E} \uput[ur](2.4,1.34){F} \uput[d](0,0){O}
\psline(1.3,0)(2.4,1.34)
\end{pspicture}
\end{center}

\medskip

\begin{enumerate}
\item Justifier que le diamètre [AB] mesure 9 cm.
\item Démontrer que le triangle ABD est rectangle en D.
\item Calculer AF{}.
\item 
	\begin{enumerate}
		\item Justifier que l'aire du triangle ABD est égale à $19,44~\text{cm}^2$.
		\item Calculer l'aire du disque, arrondie au centième.
	\end{enumerate}
	
\emph{Rappel} : l'aire du disque est égale à $\pi \times R^2$, où $R$ est le rayon du disque. 
\item Quel pourcentage de l'aire du disque représente l'aire du triangle ABD ?
\end{enumerate}

\bigskip

