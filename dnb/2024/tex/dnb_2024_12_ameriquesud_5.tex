
\bigskip

Dans cet exercice, aucune justification n'est attendue pour les réponses apportées aux questions 1. et 2.

À l'aide d'un logiciel de programmation, on définit un bloc \og Losange\fg{} pour construire un losange.

\begin{center}
\begin{tabularx}{\linewidth}{|*{2}{>{\centering \arraybackslash}X|}}\hline
\textbf{Bloc \og Losange \fg }\rule{0pt}{15pt}&\textbf{Losange obtenu}\\
&\\
\parbox{5cm}{
\begin{scratch}[pre text=\bf\sffamily,scale=1]
\initmoreblocks{définir \namemoreblocks{Losange}}
\blockpen{stylo en position d'écriture}
\blockrepeat{répéter \ovalnum{2} fois}
{
\blockmove{avancer de \ovalnum{20}}
\blockmove{tourner \turnright{} de \ovalnum{60} degrés}
\blockmove{avancer de \ovalnum{$a$}}
\blockmove{tourner \turnright{} de \ovalnum{$b$} degrés}
}
\blockpen{relever le stylo}
\end{scratch}
}% fin du parbox
&
\parbox{5cm}{
\psset{unit=1cm,arrowsize=2pt 3}
\begin{pspicture}(0,-5)(4.2,1.3)
\pspolygon(3.3,0.7)(4.1,0.7)(4.45,0)(3.65,0)
%\psgrid 
\psline{->}(2.6,0.7)(3.3,0.7)
\uput[r](0,1.1){Point et}
\uput[r](0,0.7){orientation de}
\uput[r](0,0.3){départ}
\end{pspicture}
}
\\ 
&\\
 \hline
\end{tabularx}
\end{center}

\medskip

\begin{enumerate}
\item Dans le bloc \og Losange \fg, par quelles valeurs faut-il remplacer $a$ et $b$ pour obtenir le losange ci-dessus ?
\item On définit ensuite un nouveau bloc nommé \og Motif A \fg{} :

\begin{center}
\begin{scratch}[pre text=\bf\sffamily,scale=1]
\initmoreblocks{définir \namemoreblocks{Motif A}}
\blockrepeat{répéter \ovalnum{3} fois}
{
\blockmove{Losange}
\blockmove{tourner \turnright{} de \ovalnum{60} degrés}
}
\end{scratch}
\end{center}

Parmi les figures suivantes, quelle est celle qui est obtenue en exécutant le bloc \og Motif A\fg{} ?

\begin{center}
\begin{tabularx}{\linewidth}{|*{3}{>{\centering \arraybackslash}X|}}\hline
Figure 1& Figure 2& Figure 3\\ \hline
\psset{unit=1cm}
\begin{pspicture}(-2,-2)(2,2)
\def\petale{\pspolygon(0,0)(1;0)(1.732;-30)(1;-60)}
\multido{\n=0+-45}{3}{\rput{\n}(0,0){\petale}}

\end{pspicture}&
\psset{unit=1cm}
\begin{pspicture}(-2,-2)(2,2)
\def\petale{\pspolygon(0,0)(1;0)(1.732;30)(1;60)}
\multido{\n=0+60}{6}{\rput{\n}(0,0){\petale}}
\end{pspicture}&\psset{unit=1cm}
\begin{pspicture}(-2,-2)(2,2)
\def\petale{\pspolygon(0,0)(1;0)(1.732;-30)(1;-60)}
\multido{\n=0+-60}{3}{\rput{\n}(0,0){\petale}}

\end{pspicture}\\ \hline
\end{tabularx}
\end{center}

\item On a défini un nouveau bloc nommé \og Motif B \fg. En l'exécutant, on a obtenu la figure ci-dessous:

\begin{center}
\psset{unit=1cm}
\begin{pspicture}(6,-1)
\def\petale{\pspolygon(0,0)(1;0)(1.732;-30)(1;-60)}
\multido{\n=0+2}{3}{\rput(\n,0){\petale}}
\end{pspicture}
\end{center}

Écrire un script du bloc \og Motif B \fg.
\end{enumerate}
