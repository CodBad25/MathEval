
\medskip

On considère le programme de calcul ci-dessous :

\begin{center}
\begin{tabular}{|l|}\hline
$\bullet~$Choisir un nombre\\
$\bullet~$Mettre ce nombre au carré\\
$\bullet~$Soustraire le triple du nombre de départ\\
$\bullet~$Soustraire 4\\ \hline
\end{tabular}
\end{center}

\smallskip

\begin{enumerate}
\item Montrer que si on choisit 5 comme nombre de départ, le résultat du programme est 6.
\item On choisit $x$ comme nombre de départ.

Exprimer le résultat du programme en fonction de $x$.
\item Vérifier que l'on peut écrire ce résultat sous la forme $(x+ 1)(x - 4)$.
\item Déterminer les nombres à choisir au départ pour que le résultat du programme soit 0.
\item Juliette a écrit le programme ci-dessous :

\begin{scratch}[num blocks]
\blockinit{quand \greenflag est cliqué}
\blockmove{demander \ovalnum{Choisir un nombre} et attendre}
\blockvariable{mettre \selectmenu{x} à \ovalmove{réponse}}
\blockvariable{mettre \selectmenu{y} à \ovaloperator{\ovalnum{\ldots}*\ovalnum{\ldots}}}
\blockvariable{mettre \selectmenu{z} à \ovaloperator{\ovalnum{3}*\ovalnum{x}}}
\blockvariable{mettre \selectmenu{Résultat} à \ovaloperator{\ovalnum{\ldots}-\ovalnum{\ldots} - 4}}
\blocklook{dire \ovalnum{Résultat } pendant \ovalnum{5}secondes}
\end{scratch}
\end{enumerate}

\medskip

Recopier et compléter sur la copie les lignes 4 et 6 du programme afin que celui-ci corresponde au programme de calcul encadré.

\bigskip

