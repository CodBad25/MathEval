
%M. et M\up{me} Martin veulent construire une terrasse en béton dans leur jardin. Ils souhaitent que leur terrasse ait une hauteur de $15 \mathrm{~cm}$. Les représentations ci-dessous ne sont pas à l'échelle.



\begin{enumerate}
\item %Montrer que FJ $= 4$~m.
EFGH est un rectangle, donc HG  = EF = 6, puis FJ = EI - EF $= 10 - 6 = 4$~(m).

\item %Afin de pouvoir couler le béton, M. et M\up{me} Martin doivent délimiter la terrasse en installant des planches tout autour. Quelle longueur de planches doivent-ils acheter au minimum ?
$\bullet~$EFGH est un rectangle, donc GF = HE = 3 et FI = 4.

Le triangle GFJ est rectangle en F ; le théorème de Pythagore s'écrit :

GJ$^2 = \text{GF}^2 + \text{FJ}^2 = 3^2 + 4^2 = 9 + 16 = 25 = 5^2$, d'où GJ = 5~(m).

$\bullet~$On a donc EF + FJ + JG + GH + HE = 6 + 4 + 5 + 6 + 3 = 24~(m).

Il faut acheter 24 m de planches.
\item M. et M\up{me} Martin souhaitent réaliser $4 \mathrm{~m}^{3}$ de béton.
	\begin{enumerate}
		\item %Montrer que le volume de la terrasse est bien inférieur à $4 \mathrm{~m}^{3}$.
La base du prisme a une aire :

$\mathcal{A}(\text{EFJGH}) = \mathcal{A}(\text{EFGH}) + \mathcal{A}(\text{FJG}) = 3 \times 6 + \dfrac{3 \times 4}{2} = 18 + 6 = 24$~(m$^2$).
		
Le volume de la terrasse est égal à : $\mathcal{V} = 24 \times 0,15 = 3,6$~(m$^3$) soit moins de 4.
		\item %Sachant que pour faire 1 m$^{3}$ de béton, il faut $250$~kg de ciment, quelle masse de ciment (en kg) doivent-ils acheter pour réaliser 4 m$^{3}$ de béton ?
Pour faire 1 m$^{3}$ de béton, il faudra acheter $250$~kg de ciment.

Pour faire 4 m$^{3}$ de béton, il faudra donc acheter $4 \times 250$~kg~$ = \np{1000}$~kg de ciment.

		\item Pour faire du béton, on ajoute de l'eau à un mélange de ciment, de gravier et de sable.
	\end{enumerate}
Dans ce mélange, les masses de ciment - gravier - sable sont dans le ratio $2~:~ 7~:~5$.

%Déterminer (en kg), la masse de gravier et la masse de sable nécessaires pour réaliser les 4 m$^{3}$ de béton.
Le ratio, peut également s'écrire par proportionnalité $1~;~3,5~;~2,5$, d'où pour faire $4 \mathrm{~m}^{3}$ de béton:

-- quantité de gravier nécessaire $\np{1000} \times 3,5 = \np{3500}$~(kg) ;

-- quantité de sable nécessaire $\np{1000} \times 2,5 = \np{2500}$~(kg).
\item M. et M\up{me} Martin souhaitent peindre la surface supérieure de leur terrasse.

À l'aide des documents 1, 2 et 3 , déterminer le type et le nombre de pots nécessaires pour effectuer ces travaux avec un coût minimum.

%\textbf{Document 1 : Pots de peinture proposés}
%
%\begin{center}
%\begin{tabular}{|c|c|c|}\cline { 2 - 3 }
%\multicolumn{1}{c|}{} 		&Pot A 	& Pot B \\ \hline
%Contenance (en litres) 	&5 		& 10 \\ \hline
%Prix (en euros) 			&79,90 	& 129,90 \\ \hline
%\end{tabular}
%\end{center}
%
%%\begin{center}
%%\includegraphics[max width=\textwidth]{2024_05_30_e3637aa9ef1bd32cf3c8g-6}
%%\end{center}
%\textbf{Document 2 :} L'offre du mois :
%
%Moins 50\,\% sur le deuxième article identique.
%
%\medskip
%
%\textbf{Document 3 :}
%
%Deux couches de peinture sont nécessaires. 1 litre de peinture permet de réaliser une couche de 5~m$^{2}$.
On a vu que l'aire de la terrasse est égale à 24~m$^2$. Passer deux couches revient à peindre 48~m$^2$.

Il faut donc $\dfrac{48}{5} = \dfrac{96}{10} = 9,6$~l de peinture.

$\bullet~$ on peut acheter deux pots A de 5 l pour un coût de  $79,90 + \dfrac{79,90}{2} = 79,90 + 39,95 = 119,85$~\euro. (le 2\up{e} pot est à 50\,\% de réduction)

$\bullet~$ ou acheter un pot B de 10 l à 129,90~\euro.

C'est la première solution qui a un  coût minimal.
\end{enumerate}

