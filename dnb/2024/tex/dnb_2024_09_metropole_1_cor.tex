
%\medskip
%
%Pour chacune des affirmations suivantes, dire si elle est vraie ou fausse.
%
%Toutes les réponses devront être justifiées.

\medskip

\begin{enumerate}
\item Affirmation 1

La décomposition en produit de facteurs premiers du nombre $260$ est $4 \times 5 \times 13$.

\begin{tabular}{@{\hspace*{0.05\linewidth}} | p{0.92\linewidth}}
Le nombre 4 n'est pas un nombre premier donc $4 \times 5 \times 13$ n'est pas une décomposition en produit de facteurs premiers.

\hfill\textbf{Affirmation 1 fausse}
\end{tabular}

\item Affirmation 2

Une urne opaque contient des boules indiscernables au toucher: 3 boules blanches, 4 boules jaunes et 8 boules rouges.
On pioche au hasard une boule dans cette urne et on note sa couleur.

Une autre urne opaque contient des boules indiscernables au toucher : 1 boule marquée de la lettre A, 1 boule marquée de la lettre B et 3 boules marquées de la lettre C.
On pioche au hasard une boule dans cette urne et on note la lettre obtenue.

La probabilité d'obtenir une boule de couleur rouge est supérieure à la probabilité d'obtenir une boule marquée de la lettre C.

\begin{tabular}{@{\hspace*{0.05\linewidth}} | p{0.92\linewidth}}
\begin{list}{\textbullet}{D'après le texte:}
\item La première urne contient $3+4+8=15$ boules dont 8 rouges.

La probabilité d'obtenir une boule de couleur rouge est donc $\frac{8}{15}$.
\item La seconde urne contient $1+1+3=5$ boules dont 3 marquées C.

La probabilité d'obtenir une boule marquée de la lettre C est donc
$\frac{3}{5}$.
\end{list}

$\frac{3}{5}=\frac{9}{15}$ et $\frac{8}{15}<\frac{9}{15}$
donc la probabilité d'obtenir une boule de couleur rouge est inférieure à la probabilité d'obtenir une boule marquée de la lettre C.

\hfill\textbf{Affirmation 2 fausse}
\end{tabular}

\item Affirmation 3

La solution de l'équation $7x + 5 = 2x - 2$ est $-1,4$.

\begin{tabular}{@{\hspace*{0.05\linewidth}} | p{0.92\linewidth}}
$7x + 5 = 2x - 2$
équivaut à
$7x-2x = -2-5$
équivaut à
$5x=-7$
équivaut à
$x=-\frac{7}{5}$
équivaut à
$x=-1,4$

\hfill\textbf{Affirmation 3 vraie}
\end{tabular}

\item Affirmation 4

On empile 10 pièces cylindriques de $1,9$~cm de diamètre et de $0,2$~cm de hauteur. Le volume du cylindre, arrondi à l'unité, formé par les $10$~pièces est de $6$~cm$^3$.

%\emph{Rappel} : le volume d'un cylindre de rayon $R$ et de hauteur $h$ est égal à $\pi \times R^2 \times h$.

\begin{tabular}{@{\hspace*{0.05\linewidth}} | p{0.92\linewidth}}
Une pièce cylindrique a un diamètre de $1,9$~cm, donc un rayon $R=0,95$. \\
Sa hauteur est de $h=0,2$.

Le volume d'une pièce est  $\pi \times R^2 \times h$ soit environ $3,14 \times 0,95^2 \times 0,2$ et donc $0,567$

Le volume des 10 pièces est environ $0,567\times 10=5,67$ soit 6~cm$^3$ en arrondissant à l'unité.

\hfill\textbf{Affirmation 4 vraie}
\end{tabular}

\item Affirmation 5

Un éléphant qui court à une vitesse de 5 m/s est plus rapide qu'un cochon qui se déplace à une vitesse de 17 km/h.

\begin{tabular}{@{\hspace*{0.05\linewidth}} | p{0.92\linewidth}}
Dans une heure, il y a \np{3600} secondes, et dans un kilomètre, il y a $\np{1000}$ mètres.

Une vitesse de 5 m/s correspond donc en m/h à $5\times \np{3600}=\np{18000}$, soit en km/h: $\frac{\np{18000}}{\np{1000}}=18$.

Donc  l'éléphant court à une vitesse de 18 km/h et le cochon se déplace à une vitesse de 17 km/h; donc l'éléphant est plus rapide que le cochon.

\hfill\textbf{Affirmation 5 vraie}
\end{tabular}
\end{enumerate}

\hspace{0.5cm}

