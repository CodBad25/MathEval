
\medskip

On considère la fonction $f$ définie par 

\[f(x) = x^2 + 10x + 16.\]

\begin{enumerate}
\item Vérifier par le calcul que l'image de 6 par la fonction $f$ est $112$.
\item On utilise un tableur afin de calculer les images des entiers compris entre $-4$ et 4 par la
fonction $f$.

\begin{center}
\begin{tabularx}{\linewidth}{|c|*{10}{>{\centering \arraybackslash}X|}}\hline
&A&B&C&D&E&F&G&H&I&J\\ \hline
1&$x$&$-4$&$-3$&$-2$&$-1$&0&1&2&3&4\\ \hline
2&$f(x)$&$-8$&$-5$&0&7&16&27&40&55&72\\ \hline
\end{tabularx}
\end{center}

	\begin{enumerate}
		\item Parmi les 4 formules ci-dessous, recopier celle qui a été saisie dans la cellule B2, puis
étirée vers la droite afin de calculer les images des nombres donnés par la fonction $f$.

\begin{footnotesize}
\fbox{=B1*B1+10*B1+16}\, \fbox{=A1*A1+10*A1+16}\, \fbox{$=(-4)*(-4)+10*(-4)+16$} \, \fbox{$=x*x+10*x+16$}
\end{footnotesize}
		\item En utilisant le tableau, déterminer un antécédent de $0$. 
	\end{enumerate}
\item
	\begin{enumerate}
		\item Démontrer que $f(x)$ peut s'écrire $(x + 2)(x + 8)$.
		\item En déduire un autre antécédent de 0 par la fonction $f$.
	\end{enumerate}
\end{enumerate}

\bigskip

