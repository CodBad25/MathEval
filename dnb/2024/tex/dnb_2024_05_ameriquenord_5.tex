
Dans cet exercice, on considère la figure codée ci-dessous.

\begin{itemize}
\item Les points A, C et E sont alignés.
\item Les points B, C et D sont alignés.
\item AB $= 240$~mm
\item CE $= 80$~mm
\end{itemize}


\begin{center}
\psset{unit=1cm,arrowsize=2pt 3}
\begin{pspicture}(-2.3,-0.4)(2.3,4.6)
\pspolygon(-1.9,0)(1.9,0)(-0.6,4.2)(0.6,4.2)%ABDE
\uput[dl](-1.9,0){A} \uput[dr](1.9,0){B} \uput[l](-0.6,4.2){D} \uput[r](0.6,4.2){E} 
\uput[r](0,3.15){C}
\psarc(-1.9,0){0.5cm}{0}{60}\psarc(1.9,0){0.5cm}{120}{180}
\psdots[dotstyle=|,dotangle=0,dotscale=2](0,4.2)
\psdots[dotstyle=|,dotangle=60,dotscale=2](0.3,3.7)
\psdots[dotstyle=|,dotangle=-60,dotscale=2](-0.3,3.7)
\rput(-1,0.35){$60\degres$}
\psdots[dotstyle=+,dotangle=-20,dotscale=2](-1.5,0.3)
\psdots[dotstyle=+,dotangle=20,dotscale=2](1.5,0.3)
\rput(0,-0.25){240 mm}
\rput(1.2,3.5){80 mm}
\end{pspicture}

Le dessin n'est pas à l'échelle.
\end{center}


\section*{Partie A}

\begin{enumerate}
\item Montrer que le triangle ABC est équilatéral.

\item Montrer que les droites (DE) et (AB) sont parallèles.

\end{enumerate}

\section*{Partie B}

On donne le programme suivant qui permet de tracer la figure précédente.

Ce programme comporte une variable nommée \og côté \fg.

Les longueurs sont données en pas : \textbf{1 pas représente 1~mm}.

On rappelle que l'instruction \begin{scratch}[scale=0.6]
\blockmove{s'orienter à \ovalnum{90} degrés}
\end{scratch}
signifie que le lutin se dirige horizontalement vers la droite.

\begin{center}
\begin{tabularx}{\linewidth}{|X|X|}\hline
Programme&Le bloc \textbf{Triangle}\\
\begin{scratch}[num blocks, scale=1]
\blockinit{Quand \greenflag est cliqué}
\blockmove{aller à x: \ovalnum{- 180}  y: \ovalnum{- 150}}
\blockmove{s'orienter à \ovalnum{90} degrés}
\blockvariable{mettre \selectmenu{côté} à \ovaloperator{\ovalnum{\ldots}}}
\blockmoreblocks{triangle}
\blockmove{tourner \turnleft{} de \ovalnum{60} degr\'es}
\blockmove{avancer de \ovalnum{240} pas}
\blockvariable{mettre \selectmenu{côté} à \ovaloperator{\ovalnum{côté} / 3}}
\blockmoreblocks{triangle}
\end{scratch}&
\begin{scratch}
\initmoreblocks{définir \namemoreblocks{triangle}}
\blockpen{stylo en position d'écriture}
\blockrepeat{répéter \ovalnum{3} fois}{
	\blockmove{avancer de \ovalnum{côté} pas}
	\blockmove{tourner \turnleft{} de \ovalnum{120} degrés}}
\blockpen{relever le stylo}
\end{scratch}\\ \hline
\end{tabularx}
\end{center}

\begin{enumerate}
\item Quelles sont les coordonnées du point de départ du lutin ? Aucune justification n'est demandée.

\item Quelle valeur doit être saisie à la ligne 4 dans le programme ? Aucune justification n'est demandée.

\item Le lutin démarre à la case D8. Dans quelle case se trouve-t-il lorsqu'il vient d'exécuter la ligne 7 du programme ? Aucune justification n'est demandée.

\begin{center}
\psset{unit=1cm,arrowsize=2pt 3}
\begin{pspicture}(11,9)
\psgrid[gridlabels=0pt,subgriddiv=1,griddots=8,gridcolor=orange,gridwidth=1.5pt]
\rput(1.5,8.5){A} \rput(2.5,8.5){B} \rput(3.5,8.5){C} \rput(4.5,8.5){D} \rput(5.5,8.5){E} 
\rput(6.5,8.5){F} \rput(7.5,8.5){G} \rput(8.5,8.5){H} \rput(9.5,8.5){I} \rput(10.5,8.5){J} 
\rput(0.5,7.5){1} \rput(0.5,6.5){2} \rput(0.5,5.5){3} \rput(0.5,4.5){4}
\rput(0.5,3.5){5} \rput(0.5,2.5){6} \rput(0.5,1.5){7} \rput(0.5,0.5){8}
\pspolygon(4.5,0.5)(10.5,0.5)(6.5,7.3)(8.5,7.3)
\rput(2,0.5){Départ du lutin}\psline[linewidth=1.15
pt]{->}(3.5,0.5)(4.4,0.5)
\end{pspicture}
\end{center}


\item Expliquer l'instruction \og côté $/ 3$ \fg{} de la ligne 8 du programme pour le tracé de la figure.
\end{enumerate}
