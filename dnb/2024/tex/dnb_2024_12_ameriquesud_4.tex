
\medskip

On dispose d'un terrain en pente sur lequel on souhaite construire une maison. Il faut pour cela enlever de la terre afin d'obtenir un terrain horizontal.
On dispose des informations suivantes :

\medskip

\begin{minipage}{0.28\linewidth}
La maison sera construite sur le terrain horizontal représenté par le segment [BC].
Le triangle ABC est rectangle en C et : 

AC $= 2,6$ m

AB $=17$ m
\end{minipage}\hfill
\begin{minipage}{0.67\linewidth}
\psset{unit=0.8cm}
\begin{pspicture}(12,5.5)
\pspolygon[fillstyle=dots,hatchcolor=lightgray,linecolor=white](0,0)(12,0)(12,1)(3.2,1)(3.2,2.4)(0,2.4)
\pspolygon[fillstyle=hlines,hatchcolor=gray,linecolor=white](8.3,1)(3.2,2.4)(3.2,1)%BAC
\psframe(3.2,1)(3.4,1.2)
\rput(6,2.8){Terre à enlever}\psline{->}(6,2.4)(5,1.6)
\pspolygon[fillstyle=solid,fillcolor=lightgray](0.5,2.4)(0.5,3.2)(1.4,4.6)(2.3,3.2)(2.3,2.4)
\pspolygon[fillstyle=solid,fillcolor=lightgray](9.7,1)(9.7,1.8)(10.6,3.2)(11.7,1.8)(11.7,1)
\uput[d](8.3,1){B} \uput[u](3.2,2.4){A} \uput[dl](3.2,1){C} 
\rput(6,5.3){\textbf{Vue en coupe du terrain}}
\psline(12,1)(3.2,1)(3.2,2.4)(0,2.4)
\psline[linestyle=dashed](8.3,1)(3.2,2.4)
\end{pspicture}
\end{minipage}

\medskip

\begin{enumerate}
\item Justifier que la longueur CB est égale à 16,8 m.
\item Le coût des travaux pour enlever la terre dépend de la mesure de l'angle $\widehat{\text{ABC}}$.
Si la mesure de l'angle $\widehat{\text{ABC}}$ est supérieure à $8,5\degres$, cela entraînera un surcoût des travaux (c'est-à-dire que les travaux pour enlever la terre coûteront plus cher).

Est-ce le cas pour ce terrain?
\item On admet que le volume de terre enlevée correspond au volume du prisme droit CBAFED de hauteur [CF] et de bases triangulaires ACB et DFE, comme représenté ci-dessous. On rappelle que les longueurs CF et AD sont égales.

% figure prisme de terre
\begin{center}
\psset{unit=1cm}
\begin{pspicture}(5.4,3.2)
\pspolygon(0.2,0.5)(2.5,0.2)(0.2,1.7)%CBA
\psline(2.5,0.2)(4.7,1.2)(2.4,2.7)(0.2,1.7)%BEDA
\psline[linestyle=dashed](0.2,0.5)(2.4,1.5)(4.7,1.2)%CFE
\psline[linestyle=dashed](2.4,1.5)(2.4,2.7)%FD
\uput[ul](0.2,1.7){A} \uput[dr](2.5,0.2){B} \uput[dl](0.2,0.5){C}
\uput[u](2.4,2.7){D} \uput[r](4.7,1.2){E} \uput[l](2.4,1.7){F}
\psframe(0.2,0.5)(0.4,0.7)\psframe(2.4,1.5)(2.6,1.7)
\rput{90}(-0.1,1.1){2,6 m}\rput{-30}(3.6,2.3){17 m}\rput{24}(1.2,2.5){30 m}
\end{pspicture}
\end{center}

Déterminer le volume de terre à enlever en m$^3$.

On rappelle la formule:

Volume d'un prisme droit = aire d'une base du prisme $\times$ hauteur du prisme.
\end{enumerate}

\bigskip

