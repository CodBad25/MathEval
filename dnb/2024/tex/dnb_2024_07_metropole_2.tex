
\medskip

\begin{tabularx}{\linewidth}{|X|l|}\hline
Programme A &Programme B\\ \hline
\begin{itemize}
\item Choisir un nombre.
\item Prendre le carré du nombre choisi.
\item Multiplier le résultat par 2.
\item Ajouter le double du nombre
de départ.
\item Soustraire 4 au résultat.
\end{itemize}&\begin{scratch}
\setscratch{scale=0.7,num blocks=true,print,fill blocks, fill gray=0.9}
	\blockinit{quand \greenflag est cliqué}
	\blocksensing{demander \ovalnum{Choisir un nombre}~ et attendre}
	\blockvariable{mettre \selectmenu{nombre choisi} à \ovalsensing{réponse}}
	\blockvariable{mettre \selectmenu{Résultat 1} à \ovaloperator{\ovalvariable{Nombre choisi} + \ovalnum{2} }}
	\blockvariable{mettre \selectmenu{Résultat 2} à \ovaloperator{\ovalvariable{Nombre choisi} - \ovalnum{1} }}
	\blocklook{dire \ovaloperator{regrouper \ovalnum{Le résultat est} et \ovaloperator{\ovalvariable{Résultat 1} * \ovalvariable{Résultat 2} }}}
\end{scratch}
\\ \hline		
\end{tabularx}

\medskip

\begin{enumerate}
\item
	\begin{enumerate}
		\item Vérifier que, si on choisit 5 comme nombre de départ, le résultat du programme A est 56.
		\item Quel résultat obtient-on avec le programme B si on choisit $-9$ comme nombre de départ ?
	\end{enumerate}
\item On choisit un nombre quelconque $x$ comme nombre de départ.
	\begin{enumerate}
		\item Parmi les trois propositions ci-dessous, recopier l'expression qui donne le résultat obtenu par le programme B ?
		
\[E_1 = (x + 2) - 1\qquad E_2 = (x + 2) \times (x - 1)\qquad  E_3 = x + 2 \times x - 1\]
 
		\item Exprimer en fonction de $x$ le résultat obtenu avec le programme A.
	\end{enumerate}
\item Démontrer que, quel que soit le nombre choisi au départ, le résultat du programme A est toujours le double du résultat du programme B.
\end{enumerate}

\medskip

\bigskip

