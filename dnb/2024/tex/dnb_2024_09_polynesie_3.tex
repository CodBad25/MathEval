
\medskip

\begin{minipage}{0.58\linewidth}
Sur la figure ci-contre, qui n'est pas à l'échelle,
\begin{itemize}[label=$\bullet~$]
\item le triangle ONM est rectangle en N,
\item le triangle OPQ est rectangle en P{},
\item le triangle ORS est rectangle en R,
\item ON $= 6$ cm et $\widehat{\text{MON}} = 32\degres{}$.
\item P est un point du segment [OM] et R est un point du segment [OQ].
\end{itemize}
\end{minipage}\hfill
\begin{minipage}{0.42\linewidth}
\psset{unit=1.1cm}
\begin{pspicture}(4.5,4.1)
%\psgrid
\pspolygon(0.3,1.3)(0.3,3.6)(4.1,3.6)%%MNO
\psline(0.82,1.62)(1.7,0.3)(4.1,3.6)%PQO
\psline(3.04,2.18)(3.65,1.7)(4.1,3.6)%RSO
\psframe(0.3,3.6)(0.5,3.4)\rput{-57}(0.82,1.62){\psframe(0.2,0.2)}
\rput{-38}(3.04,2.18){\psframe(0.2,0.2)}
\psarc(4.1,3.6){0.8}{180}{212}
\uput[d](0.3,1.3){M} \uput[ul](0.3,3.6){N} \uput[ur](4.1,3.6){O}
\uput[d](0.78,1.62){P} \uput[d](1.7,0.3){Q} \uput[d](3.09,2.18){R}
\uput[d](3.65,1.7){S} \uput[u](2.2,3.6){6 cm} \rput(3.65,3.45){\footnotesize $32\degres$}
\end{pspicture}
\end{minipage}

\medskip

\begin{enumerate}
\item Calculer la mesure de la longueur MN. On donnera une valeur approchée au millimètre près.
\item On donne PQ $= 2,5$ cm et OQ $= 6,5$ cm. Montrer que OP $= 6$ cm.
\item Montrer que les triangles ONM et OPQ ne sont pas des triangles égaux.
\item Sachant que le triangle OPQ est un agrandissement du triangle ORS et que OS $= 3,25$ cm, calculer l'aire du triangle ORS.
\end{enumerate}


\bigskip

