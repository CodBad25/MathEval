
\medskip


\begin{enumerate}
\item %Vérifier que pour une température de l'eau de $26~\degres C$, le temps de filtration est de $15$~h.
$26 \longmapsto 26 + 4 = 30  \longmapsto 30 \times 0,5 = 15$ : on obtient un temps de filtration de 15 h. 
\item %On note $x$ la température de l'eau de la piscine (en degré Celsius).

%Montrer que le temps de filtration, en heure, peut s'écrire $0,5x + 2$.
De même : $x  \longmapsto x + 4  \longmapsto 0,5 \times (x + 4) = 0,5x + 2$.
\item M%On donne ci-dessous la courbe représentative de la fonction $f$ définie par 
\[f(x) = 0,5x + 2\]
 où $x$ désigne la température de l'eau (en $\degres C$) et $f(x)$ le temps de filtration (en h).

\begin{center}
\psset{xunit=0.6cm,yunit=0.4cm,arrowsize=2pt 3,dash=1mm 1mm}
\begin{pspicture}(-1,-1.2)(18,14)
\multido{\n=0+1}{19}{\psline[linewidth=0.25pt,linestyle=dashed](\n,0)(\n,14)}
\multido{\n=0+1}{15}{\psline[linewidth=0.15pt,linestyle=dashed](0,\n)(18,\n)}
\psaxes[linewidth=1.25pt,Dy=2,labelFontSize=\scriptstyle]{->}(0,0)(0,0)(18,14)
\psplot[plotpoints=2000,linewidth=1.25pt,linecolor=blue]{0}{18}{0.5 x mul 2 add}
\uput[u](17.5,0){$x$}\uput[r](0,13.5){$f(x)$}
\end{pspicture}
\end{center}
	\begin{enumerate}
		\item %Le temps de filtration est-il proportionnel à la température de l'eau de la piscine ?
On lit sur le graphique :

$f(6) = 5$ et $f(12) = 8$ : pour une température doublée le temps de filtration ne l'est pas  : le temps de filtration n'est pas proportionnel à la température de l'eau de la piscine.

On peut aussi remarquer que la droite représentative de la fonction $f$ ne contient pas l'origine.
		\item %Quelle est l'image de 10 par la fonction ? Aucune justification n'est demandée.
$\bullet~$Sur le graphique on lit $f(10) = 7$ ;

$\bullet~$Par le calcul : $0,5 \times 10 + 2 = 5 + 2 = 7$/
	\end{enumerate}
\item %Résoudre l'équation $0,5x + 2 = 17$ et interpréter ce résultat dans le contexte du problème.
$0,5x + 2 = 17$ entraine en ajoutant $- 2$ à chaque membre : $0,5x = 15$, puis en multipliant chaque membre par 2 : $x = 30$.

Cela signifie que pour une température de $30\degres$ C, le temps de filtration doit être de 15 h.
\item %M. Durand a décidé de filtrer sa piscine 16~h par jour, tous les jours du 1\up{er} juillet au 31 août inclus.

%À l'aide des documents ci-dessous, calculer la dépense liée au fonctionnement de la filtration au cours de cette période.

%\emph{Laisser toute trace de recherche, même si elle n'a pas abouti.}
\end{enumerate}

%\bigskip
%
%\begin{minipage}{0.4\linewidth}
%\begin{tabular}{|l|}\hline
%\textbf{Document 1 : Puissance}\\
%Puissance de la pompe : 0,8 kW\\
%kW signifie kiloWatt\\ \hline
%\end{tabular}
%\end{minipage} \hfill
%\begin{minipage}{0.4\linewidth}
%\begin{tabular}{|l|}\hline
%\textbf{Document 2 : Prix}\\
%Prix d'un kWh : $0,23$~\euro\\
%kWh signifie kiloWatt-heure\\ \hline
%\end{tabular}
%\end{minipage}
%
%\medskip
%
%\begin{minipage}{\linewidth}
%\begin{tabular}{|l|}\hline
%\textbf{Document 3 : Calcul de la consommation électrique de la pompe (en kWh)}\\
%Puissance de la pompe (en kW) $\times$ nombre d'heures d'utilisation par jour $\times$ nombre de jours\\ \hline
%\end{tabular}
%\end{minipage}
La pompe a fonctionné pendant deux mois de 31 jours pendant 16 h chaque jour, soit pendant $2 \times 31 \times 16 = 992$~(h).

D'après la formule donnée la dépense est donc égale à :

$0,8 \times 992 \times 0,23 = 182,528$, soit 182,53~\euro.

\bigskip

