
\medskip

%Pour faire écouter de la musique à son enfant, Aurélie a sélectionné 22 chansons :
%
%9 chants de Noël, 6 comptines et des berceuses.
%
%Le temps d'écoute total des chansons de sa liste est de 55 minutes.

\medskip

\begin{enumerate}
\item %MCalculer le nombre de berceuses présentes dans la liste.
Le nombre de berceuses présentes dans la liste est égale à la différence :
\[22 - (9 + 6) = 22 - 15 = 7.\]

\item %Calculer la durée moyenne d'une chanson de cette liste. Le résultat sera donné en minute et seconde.
Le temps total pour écouter les 22 musiques est de 55 minutes soit un e moyenne par chanson de :

\[\dfrac{55}{22} = \dfrac{11 \times 5}{11 \times 2} = \dfrac52 = 2,5.\]

Or 2,5~min = 2~min 30 s.

En moyenne écouter une chanson dure deux minutes et demie.
\item %Aurélie écoute une chanson. Elle utilise la fonction aléatoire de son lecteur, c'est-à-dire que la chanson écoutée est choisie au hasard parmi toutes les chansons de la liste.
	\begin{enumerate}
		\item %Montrer que la probabilité que la chanson écoutée soit une comptine est égale à $\dfrac{3}{11}$.
		Il y a 6 comptines sur les 22 chansons ; la probabilité d'écouter une comptine est donc égale à :
		
		\[\dfrac{6}{22} = \dfrac{3}{11}.\]
		\item %Quelle est la probabilité que la chanson écoutée ne soit pas une berceuse ?
On a vu que 15 chansons sur 22 ne sont pas des berceuses. La probabilité de ne pas écouter une berceuse est donc égale à :
\[\dfrac{15}{22}.\]

		\item %Les chansons sont numérotées de 1 à 22. On considère l'évènement :
De 1 à 22, il y a 2~;~3~;~5~;~7~;~11~;~13~;~17~;~19 qui sont des naturels premiers, soit 8 nombres premiers.

La probabilité d'écouter une musique dont le numéro est premier est donc égale à 

$\dfrac{8}{22} = \dfrac{4}{11}$.

Comparer $\dfrac{4}{11}$ et $\dfrac13$, c'est comparer $\dfrac{4}{11} = \dfrac{4 \times 3}{11\times 3} = \dfrac{12}{33}$ avec $\dfrac13 = \dfrac{1 \times 11}{3 \times 11} = \dfrac{11}{33}$.

Comme $\dfrac{11}{33} < \dfrac{12}{33}, \qquad \dfrac13 < \dfrac{4}{11}$.

La probabilité d'écouter une musique étiquetée par un nombre premier est donc supérieure à $\dfrac13$.
%\begin{center}\og Le numéro de la chanson écoutée est un nombre premier. \fg\end{center}
%La probabilité de cet évènement est-elle supérieure à
%$\dfrac13$ ? Justifier.
	\end{enumerate}
\end{enumerate}

\bigskip

