
\medskip

Un garage propose 2 options au client:

\begin{itemize}
\item Option \emph{Achat}  : prix d'achat de la voiture \np{22400}~\euro. Assurance obligatoire $75$~\euro par mois.
\item Option \emph{Location} : 425 \euro par mois, assurance comprise.
\end{itemize}

\smallskip

L'objectif de cet exercice est de comparer ces deux options.

\textbf{Partie B}

\medskip

\begin{enumerate}
\item \np{22400}~\euro pour le prix d'achat plus le coût de l'assurance pendant 12 mois soit $12 \times 75 = 900$~\euro, soit un total de

\[\np{22400} + 900 = \np{23300}~\euro.\]

\item De la même façon l'option \emph{Achat } reviendra à : $\np{22400} + 36 \times 75= \np{2700}$, soit un total de $\np{22400} + \np{2700} = \np{25100}$.

L'option \emph{Location} reviendra à $36 \times 425 = \np{15300}$

Donc sur une durée de 36 mois la location coûtera : $\np{25100} - \np{15300} = \np{9800}$ de moins que l'achat.
\item %Afin de comparer les dépenses correspondantes à ces options le client a réalisé le tableau suivant à l'aide d'un tableur:


Dans la cellule il faut écrire =425 *B1.
\end{enumerate}

\bigskip

\textbf{Partie B}

\medskip

\[g(x) = 425x.\]

\begin{enumerate}[resume]
\item Au bout de $x$ mois on aura dépensé \np{22400}~(\euro{}) et $x \times 75 = 75x$~(\euro{}) pour l'assurance obligatoire, soit un total de :

\[f(x) = \np{22400} + 75x.\]
\end{enumerate}

\psset{xunit=0.1cm, yunit=0.02cm, arrowsize=2pt 3}
\begin{center}
\begin{pspicture*}(-35,-60)(130,460)
\psaxes[linewidth=1.25pt, Dx=10, Dy=1000, yLabels={}]{->}(0,0)(0,0)(130,460)

% Grille verticale
\multido{\n=0+2}{66}{\psline[linewidth=0.2pt](\n,0)(\n,460)}
\multido{\n=0+10}{14}{\psline[linewidth=0.4pt](\n,0)(\n,460)}

% Grille horizontale
\multido{\n=0+20}{23}{\psline[linewidth=0.2pt](0,\n)(130,\n)}
\multido{\n=0+100}{5}{\psline[linewidth=0.4pt](0,\n)(130,\n)}

% Courbes (valeurs divisées par 100)
\psplot[plotpoints=1000, linewidth=1.25pt, linecolor=red]{0}{130}{425 x mul 100 div}
\psplot[plotpoints=1000, linewidth=1.25pt, linecolor=blue]{0}{130}{75 x mul 22400 add 100 div}

% Légendes des courbes
\uput[u](125,320){\blue $\mathcal{C}_f$}
\uput[u](95,420){\red $\mathcal{C}_g$}

% Point d'intersection
\psline[linewidth=1.25pt, linestyle=dashed, ArrowInside=->](64,272)(64,0)
\psline[linewidth=1.25pt, linestyle=dashed, ArrowInside=->](64,272)(0,272)
\uput[d](64,0){64}
\uput[l](0,272){$\approx \np{27000}$}

% Légendes des axes
\rput[r](-2,440){\footnotesize dépense en euro}
\rput[t](122,-30){\footnotesize durée en mois}

% Labels verticaux manuels
\rput[r](-2,0){\scriptsize 0}
\rput[r](-2,100){\scriptsize 10\,000}
\rput[r](-2,200){\scriptsize 20\,000}
\rput[r](-2,300){\scriptsize 30\,000}
\rput[r](-2,400){\scriptsize 40\,000}
\end{pspicture*}
\end{center}

On lit sur le graphique que les deux droites sont sécantes au point d'abscisse 64 : donc à partir de  
65 mois il est préférable, financièrement de choisir l'option \emph{Achat}.

\emph{Remarque }: on peut s'interroger sur la pertinence de cette comparaison entre achat et location :

$\bullet~$ tout d'abord on laisse entendre qu'une voiture louée ne coûte rien en assurance, alors que celle-ci est obligatoire, mais que les assurances proposées par les loueurs sont souvent insuffisantes ;

$\bullet~$le concepteur du sujet semble ignorer que la plupart des locations sont proposées sur 3 et plus souvent 4 ans ;

$\bullet~$nulle part n'est signalé qu'à la fin de la location, le locataire n'a fait que payer et se retrouve sans rien ;

$\bullet~$enfin la même chose arrive en cas de vol ou d'accident grave du véhicule.

En conclusion il est très difficile de comparer location et achat.
