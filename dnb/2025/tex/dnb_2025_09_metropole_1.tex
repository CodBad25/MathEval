
\medskip

Pour faire écouter de la musique à son enfant, Aurélie a sélectionné 22 chansons :

9 chants de Noël, 6 comptines et des berceuses.

Le temps d'écoute total des chansons de sa liste est de 55 minutes.

\medskip

\begin{enumerate}
\item Calculer le nombre de berceuses présentes dans la liste.
\item Calculer la durée moyenne d'une chanson de cette liste. Le résultat sera donné en
minute et seconde.
\item Aurélie écoute une chanson. Elle utilise la fonction aléatoire de son lecteur, c'est-à-dire que la chanson écoutée est choisie au hasard parmi toutes les chansons de la liste.
	\begin{enumerate}
		\item Montrer que la probabilité que la chanson écoutée soit une comptine est égale à $\dfrac{3}{11}$.
		\item Quelle est la probabilité que la chanson écoutée ne soit pas une berceuse ?
		\item Les chansons sont numérotées de 1 à 22. On considère l'évènement :
		
\begin{center}\og Le numéro de la chanson écoutée est un nombre premier. \fg\end{center}
La probabilité de cet évènement est-elle supérieure à
$\dfrac13$ ? Justifier.
	\end{enumerate}
\end{enumerate}

\bigskip

