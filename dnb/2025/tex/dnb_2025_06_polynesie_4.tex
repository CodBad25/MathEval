
\medskip

On considère le programme de calcul suivant.

\begin{center}
\psset{unit=1cm,arrowsize=2pt 3}
\begin{pspicture}(-6.5,0)(6.5,6)
%\psgrid
\psframe(-2.4,5.3)(2.4,6)\rput(0,5.65){Nombre choisi au départ}
\psline{->}(0,5.3)(-2.1,4.5)\psline{->}(0,5.3)(2.1,4.5)
\psframe(-2.1,4.5)(-6.5,3.8)\rput(-4.3,4.15){Ajouter 4}
\psframe(2.1,4.5)(6.5,3.8)\rput(4.3,4.15){Soustraire 2}
\psline{->}(-2.1,3.8)(0,3.3)\psline{->}(2.1,3.8)(0,3.3)
\psframe(-3.6,2.6)(3.6,3.3)\rput(0,2.95){Multiplier les deux résultats}
\psline{->}(0,2.6)(0,2)
\psframe(-3.6,1.3)(3.6,2)\rput(0,1.65){Soustraire le carré du nombre de départ}
\psline{->}(0,1.3)(0,0.8)
\psframe(-3.6,0)(3.6,0.8)\rput(0,0.4){Résultat final}
\end{pspicture}
\end{center}

\begin{enumerate}
\item Montrer que si on choisit 5 comme nombre de départ, le résultat du programme est 2. 
\item On choisit $x$ comme nombre de départ.
	\begin{enumerate}
		\item Parmi les expressions suivantes, quelle est celle qui permet d'exprimer le résultat de ce programme de calcul en fonction de $x$ ? Aucune justification n'est attendue.

\begin{center}
\renewcommand{\arraystretch}{1.2}
\begin{tabularx}{\linewidth}{|*{4}{>{\centering \arraybackslash}X|}}\hline
Expression A &Expression B& Expression C& Expression D\\ \hline
$x + 4 \times x - 2- x^2$& $x + 4 \times x - 2 - 2x$ &$(x + 4) \times (x - 2) - x^2$& $(x + 4) \times (x - 2) - 2x$\\ \hline
\end{tabularx}
\end{center}
		\item  Montrer que le résultat du programme de calcul peut s'écrire sous la forme $2x - 8$.
	\end{enumerate}
\item On appelle $f$ la fonction définie par $f(x) = 2x - 8$.

Voici trois représentations graphiques:

\begin{center}
\begin{tabularx}{\linewidth}{|*{3}{>{\centering \arraybackslash}X|}}\hline
Représentation \no 1& Représentation \no 2& Représentation \no 3\\ \hline
\psset{unit=0.3cm,,arrowsize=2pt 3}
\begin{pspicture*}(-4.2,-9)(7,8)
\psaxes[linewidth=1.25pt,Dx=2,Dy=2,labelFontSize=\scriptstyle]{->}(0,0)(-4.2,-9)(7,8)
\multido{\n=-4+2}{6}{\psline[linewidth=0.2pt]( \n,-9)(\n,8)}
\multido{\n=-8+2}{9}{\psline[linewidth=0.2pt](-4,\n)(6.4,\n)}
\psplot[plotpoints=2000,linewidth=1.25pt]{-4}{6}{x 0.5 sub dup mul 8 sub}
\end{pspicture*}
&\psset{unit=0.3cm,,arrowsize=2pt 3}
\begin{pspicture*}(-8,-11)(3,4.5)
\psaxes[linewidth=1.25pt,Dx=2,Dy=2,labelFontSize=\scriptstyle]{->}(0,0)(-8,-11)(3,4.5)
\multido{\n=-8+2}{7}{\psline[linewidth=0.2pt](\n,-10)(\n,4)}
\multido{\n=-10+2}{8}{\psline[linewidth=0.2pt](-8,\n)(3,\n)}
\psplot[plotpoints=500,linewidth=1.25pt]{-8}{2}{2 neg x mul 8 sub}
\end{pspicture*}
&\psset{unit=0.3cm,,arrowsize=2pt 3}
\begin{pspicture*}(-3,-11)(8,5)
\psaxes[linewidth=1.25pt,Dx=2,Dy=2,labelFontSize=\scriptstyle]{->}(0,0)(-3,-11)(8,5)
\multido{\n=-2+2}{5}{\psline[linewidth=0.2pt](\n,-10)(\n,8)}
\multido{\n=-10+2}{8}{\psline[linewidth=0.2pt](-3,\n)(7,\n)}
\psplot[plotpoints=2000,linewidth=1.25pt]{-3}{6}{2 x mul 8 sub}
\end{pspicture*}
\\ \hline
\end{tabularx}
\end{center}
	\begin{enumerate}
		\item La représentation graphique de la fonction $f$ est la représentation \no 3. Expliquer pourquoi les représentations \no 1 et \no 2 ne conviennent pas.
		\item Déterminer l'image de 4 par la fonction $f$.
	\end{enumerate}
\item Quel nombre de départ faut-il choisir pour que le résultat du programme de calcul soit égal à
100 ?
\end{enumerate}

\bigskip

