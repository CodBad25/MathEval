
\medskip

Cet exercice est un questionnaire à choix multiple (QCM).

Pour chaque question, quatre réponses sont proposées. \textbf{Une seule réponse est exacte}.

Recopier sur la copie le numéro de la question et la réponse choisie. Aucune justification
n'est demandée.

\medskip

\textbf{Question 1}

La décomposition en produit de facteurs premiers de 120 est:

\begin{center}
\begin{tabularx}{\linewidth}{|*{4}{>{\centering \arraybackslash}X|}}\hline
\textbf{Réponse A}&\textbf{Réponse B}&\textbf{Réponse C}&\textbf{Réponse D}\\ \hline
$2~\times~3~\times~4~\times~5$& $15~\times~2~\times~2~\times~2$& $2^3 ~\times~3~\times~5$& $53 + 67$\\ \hline
\end{tabularx}
\end{center}
\medskip

\textbf{Question 2}

\begin{minipage}{0.5\linewidth}
Dans la cellule A2, la formule \og = $-$ 4 * A1 - 12 \fg{} a été saisie.

On l'étire jusqu'à la cellule B2.

La valeur obtenue dans la cellule B2 est .:
\end{minipage}\hfill
\begin{minipage}{0.45\linewidth}
\begin{tabularx}{\linewidth}{|c|*{2}{>{\centering \arraybackslash}X|}}\hline
&A &B\\ \hline
1&2&5\\ \hline
2& $-20$& \\ \hline
\end{tabularx}
\end{minipage}

\begin{center}
\begin{tabularx}{\linewidth}{|*{4}{>{\centering \arraybackslash}X|}}\hline
\textbf{Réponse A}&\textbf{Réponse B}&\textbf{Réponse C}&\textbf{Réponse D}\\ \hline
$- 32$& $- 20$& $8$& $68$\\ \hline
\end{tabularx}
\end{center}

\medskip

\textbf{Question 3}

\begin{minipage}{0.67\linewidth}
Sur la figure ci-contre, le rapport de l'homothétie de centre O qui transforme le carré A en le carré B est:
\end{minipage}\hfill
\begin{minipage}{0.31\linewidth}
\psset{unit=1cm}
\begin{pspicture}(-0.5,-1)(4,1)
\psgrid[subgriddiv=2,gridwidth=0.2pt,subgridwidth=0.2pt,gridlabels=0pt](-0.5,-1)(4,1)
\psdot[dotstyle=+,dotangle=45,dotsize=10pt](0,0)
\pspolygon[linewidth=1.25pt](1,0)(1.5,-0.5)(3,1)(4,0)(3,-1)(1.5,0.5)
\rput(1.5,0){A}\rput(3,0){B}\uput[d](0,0){O}
\end{pspicture}
\end{minipage}

\begin{center}
\begin{tabularx}{\linewidth}{|*{4}{>{\centering \arraybackslash}X|}}\hline
\textbf{Réponse A}&\textbf{Réponse B}&\textbf{Réponse C}&\textbf{Réponse D}\\ \hline
$-2$&$-0,5$&0,5&2\\ \hline
\end{tabularx}
\end{center}

\medskip

\textbf{Question 4}

Une écriture factorisée de $4x^2 - 1$ est : 

\begin{center}
\begin{tabularx}{\linewidth}{|*{4}{>{\centering \arraybackslash}X|}}\hline
\textbf{Réponse A}&\textbf{Réponse B}&\textbf{Réponse C}&\textbf{Réponse D}\\ \hline
$(2x - 1)(2x + 1)$&$(4x - 1)(4x + 1)$&$4(x - 1)(x + 1)$&$(2x - 1)^2$\\ \hline
\end{tabularx}
\end{center}

\medskip

\textbf{Question 5}

\begin{minipage}{0.5\linewidth}
Dans le triangle TER ci-contre, la mesure de la longueur RE 
arrondie au centième de cm est :
\end{minipage}\hfill
\begin{minipage}{0.47\linewidth}
\psset{unit=1cm,arrowsize=2pt 3}
\begin{pspicture}(-2.4,-0.3)(3.5,3.3)
\pspolygon(0;0)(4.1;39)(3.3;129)
\rput{39}(0,0){\psframe(0.3,0.3)}
\psline[linewidth=0.8pt]{<->}(-2.1,2.9)(3.2,2.9)
\uput[u](0.6,2.9){7,4 cm}\psarc(4.1;39){0.5}{180}{219}
\uput[l](3.3;129){T}\uput[r](4.1;39){E}\uput[d](0,0){R}\uput[dl](2.6,2.45){$39\degres$}
\end{pspicture}
\end{minipage}

\begin{center}
\begin{tabularx}{\linewidth}{|*{4}{>{\centering \arraybackslash}X|}}\hline
\textbf{Réponse A}&\textbf{Réponse B}&\textbf{Réponse C}&\textbf{Réponse D}\\ \hline
4,66 cm&5,75 cm&9,52 cm&11,76 cm\\ \hline
\end{tabularx}
\end{center}

\bigskip

