
	\begin{enumerate}
		\item Dans le triangle CDE, rectangle en D, on applique le théorème de Pythagore :


		$\mathrm{CE}^2 = \mathrm{CD}^2 + \mathrm{DE}^2$

		Soit, en remplaçant les longueurs connues :
		\quad		$29,1^2 = 21,6^2 + \mathrm{DE}^2$

		Et donc :\quad $\mathrm{DE}^2 = 29,1^2 - 21,6^2 = 846,81 - 466,56 = 380,25$

		DE est une longueur, donc c'est un nombre positif :\quad $\mathrm{DE} = \sqrt{380,25} = \np[cm]{19,5}$.

		On trouve bien la longueur annoncée.

		\item Puisque le triangle CDE est rectangle en D, on va choisir comme base le côté [CD], et la hauteur correspondante est donc [DE].

		$\mathcal{A}_{\mathrm{CDE}} = \dfrac{\mathrm{CD}\times \mathrm{DE}}{2} = \dfrac{21,6 \times 19,5}{2} = \np[cm^2]{210,6}$.

		\item On sait que :
		\begin{itemize}
			\item Les points G, C et E sont alignés;
			\item Les points F, C et D sont alignés;
			\item Les droites (GF) et (DE) sont parallèles.
		\end{itemize}

		Le théorème de Thalès, appliqué dans cette configuration nous donne :

		$\dfrac{\mathrm{GF}}{\mathrm{DE}} = \dfrac{\mathrm{FC}}{\mathrm{CD}} = \dfrac{\mathrm{CG}}{\mathrm{CE}}$.

		En particulier :\quad $\dfrac{\mathrm{GF}}{\mathrm{DE}} = \dfrac{\mathrm{FC}}{\mathrm{CD}}$.

		En remplaçant par les valeurs connues :\quad $\dfrac{\mathrm{GF}}{19,5} = \dfrac{17,2}{21,6}$.

		On en déduit :\quad $\mathrm{GF} = 19,5 \times \dfrac{17,2}{21,6} = \dfrac{559}{36} \approx 15,53$.

		Au millimètre près, donc a donc :\quad $\mathrm{GF}\approx \np[cm]{15,5}$.

		\item \begin{enumerate}
			\item On a calculé : \quad $\mathcal{A}_{\mathrm{CDE}} = \np[cm^2]{210,6}$.

			Donc :\quad $\dfrac{1}{9}\times \mathcal{A}_{\mathrm{CDE}} = \dfrac{1}{9}\times 210,6 = \np[cm^2]{23,4}$.

			On a effectivement l'aire de ABC qui est $\dfrac{1}{9}$ de l'aire de CDE.

			\item \emph{Remarque :} cette première partie n'était pas demandée : on admet que les triangles sont semblables.

			Les triangles ABC et CDE sont effectivement semblables, car comme (AB) est perpendiculaires à (FC), et que (DE) l'est aussi, ces deux droites sont parallèles. Les deux triangles CBA et CDE forment donc une configuration où l'on peut appliquer le le théorème de Thalès, et donc les deux triangles sont semblables.

			Appelons $k$ le rapport de proportionnalité entre les longueurs du triangle CDE et celles de CBA. On sait donc que les aires sont proportionnelles, avec le rapport $k^2$.

			Comme on a calculé à la question précédente que le rapport de proportionnalité des aires est de $\dfrac{1}{9}$, cela signifie que $k^2 = \dfrac{1}{9}$.

			Comme $k$ est un rapport entre des longueurs, qui sont positives, $k$ sera positif aussi, donc :\quad $k = \sqrt{\dfrac{1}{9}} = \dfrac{1}{3}$.

			Ainsi, la longueur homologue de AB dans CDE étant DE : \quad $\begin{aligned}[t]
				\mathrm{AB} &= \dfrac{1}{3}\mathrm{DE} = \dfrac{1}{3} 19,5 \\
				&= \np[cm]{6,5}
			\end{aligned}$
		\end{enumerate}
	\end{enumerate}

