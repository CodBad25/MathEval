
\medskip

Le jardin botanique d'une ville peut être représenté par le quadrilatère ABCD ci-dessous.

\begin{minipage}{0.55\linewidth}
\psset{unit=1cm}
\begin{pspicture}(8.8,5.6)
%\psgrid
\pspolygon[fillstyle=solid,fillcolor=lightgray,linewidth=0pt](0.3,5)(2.9,5)(2.25,3.87)(0.3,5)%%ABE
\pspolygon[fillstyle=solid,fillcolor=lightgray,linewidth=0pt](8.4,0.3)(0.3,0.3)(2.25,3.87)(8.4,0.3)%%% CDE
\pspolygon(0.3,5)(2.9,5)(8.4,0.3)(0.3,0.3)%ABCD
\psline[linewidth=1.75pt,linestyle=dotted](0.3,0.3)(2.9,5)
\psline[linewidth=1.75pt,linestyle=dotted](0.3,5)(8.4,0.3)
\psframe(0.3,5)(0.5,4.8)\psframe(0.3,0.3)(0.5,0.5)
\rput{-28}(2.25,3.87){\psframe(0.2,0.2)}
\uput[ul](0.3,5){A} \uput[ur](2.9,5){B} \uput[dr](8.4,0.3){C} \uput[dl](0.3,0.3){D}
\uput{10pt}[190](2.25,3.87){E}
\end{pspicture}
\end{minipage}
\hfill
\begin{minipage}{0.45\linewidth}
On sait que :
\smallskip

$\bullet~$ AB $= 500$ m, BE $= 250$ m et DE $= 750$ m ;

\smallskip

$\bullet~$ les segments [AC] et [BD] se coupent au point E.

\smallskip

\emph{La figure ci-contre n'est pas à l'échelle.}
\end{minipage}

\begin{enumerate}
\item Quelle est la longueur du segment [DB] ?
\item En raisonnant dans le triangle rectangle ABD, montrer que la longueur du segment [AD],
arrondie au mètre, est égale à environ $866$~m.
\item 
	\begin{enumerate}
		\item Calculer le sinus de l'angle $\widehat{\text{EAB}}$.
		\item En déduire la mesure en degrés de l'angle $\widehat{\text{EAB}}$.
	\end{enumerate}
\item 
	\begin{enumerate}
		\item Montrer que les droites (AB) et (DC) sont parallèles.
		\item Montrer que la longueur du segment [CD] est égale à \np{1500}~m.
	\end{enumerate}
\item Un piéton fait le tour du jardin botanique en marchant à la vitesse moyenne de 1,1~m/s.

Il lit sur son plan que la longueur du segment [BC] est environ égale à \np{1323}~m.

Le temps mis par le piéton pour faire le tour du jardin botanique est-il inférieur à une heure?

\end{enumerate}


