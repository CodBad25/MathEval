
\emph{La figure ci-dessous n'est pas en vraie grandeur}.

\begin{center}
\psset{unit=0.8cm}
\begin{pspicture}(0,-3.4)(10,4.5)
\pspolygon(0,0)(9,0)(9,-2.7)(0,4)%EBCD
\psline(0,4)(2.3,0)%DM
\psframe(0.4,0.4)\psframe(9,0)(8.6,-0.4)
\psarc(2.3,0){0.6}{120}{180}
\uput[ur](5.3,0){A}\uput[u](9,0){B}\uput[d](9,-2.7){C}
\uput[u](0,4){D}\uput[d](0,0){E}\uput[d](2.3,0){M}
\uput[r](9,-1.35){30 m}\uput[dl](7.2,-1.6){50 m}\uput[ur](2.6,2.1){70 m}
\rput(1.3,0.5){$60\,\degres$}
\end{pspicture}
\end{center}

On a les données suivantes:

\medskip

\begin{itemize}[label= \small $\bullet~~$]
\item Les points A, B, E et M sont alignés
\item Les points A, C et D sont alignés
\item ADE est un triangle rectangle en E
\item ABC est un triangle rectangle en B
\item AD $= 70$ m
\item BC $= 30$ m
\item AC $= 50$ m 
\item $\widehat{\text{DME}} = 60\degres$
\end{itemize}

\medskip

\begin{enumerate}
\item Calculer la longueur AB.
\item Montrer que les droites (DE) et (BC) sont parallèles.
\item Montrer que la longueur DE est égale à $42$~m.
\item Montrer que la longueur EM est environ égale à $24,2$~m.
\item En déduire l'aire du triangle AMD.
\end{enumerate}


