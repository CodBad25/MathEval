
\medskip

\begin{enumerate}
\item B3 ou C9 sont des codes possibles
\item  On peut choisir entre 3 lettres puis entre 10 chiffres : il y a donc $3 \times 10 = 30$ codes possibles différents.

Il y a 10 codes commençant par C : la probabilité que le code commence par la lettre C est donc : $\dfrac{10}{30} = \dfrac 13$.
\item Il y a trois codes se finissant par 7 : A7,\: B7 et C7.

La probabilité que le code se finisse par 7 est égale à $\dfrac{3}{30} = \dfrac{1}{10} = 0,1$.
\item 2, \:3,\:5 et 7 sont premiers : il y a 3 codes finissant par l'un de ces 4 nombres, soit $ \times 4 = 12$ codes contenant un nombre premier.

La probabilité que le code contienne un nombre premier est donc égale à $\dfrac{12}{30} = \dfrac{6 \times 2}{6 \times 5} = \dfrac 25 = \dfrac{4}{10} = 0,4$.
\item
	\begin{enumerate}
		\item Avec 30 codes différents il lui faudra au maximum : $30 \times 5 = 150$~s soit $120 + 30$~s ou 2 min 30~s, donc en moins de 3~min.
		\item N'importe qui peut trouver le code en 2 min 30 s maximum : c'est insuffisant.

En prenant l'une des 26 lettres de l'alphabet, il faudra $26 \times 5 \times 3  = 390$~s soit 6~min 30~s soit en plus de deux fois plus de temps.
	\end{enumerate}
\item
	\begin{enumerate}
		\item B5 n'est pas le code attendu ; le programme affiche Code faux.
		\item Le code attendu est B7.
	\end{enumerate}
\end{enumerate}
