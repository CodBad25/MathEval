
%On considère les deux programmes de calcul suivants :
%
%\begin{center}
%\begin{tabularx}{\linewidth}{m{6.cm}|X}
%\textbf{Programme A}&\textbf{Programme B}\\
%\begin{itemize}[label= $\bullet~~$]
%\item Choisir un nombre
%\item Multiplier par 3
%\item Ajouter 15
%\item Diviser par 3
%\item Soustraire le nombre de départ
%\end{itemize}&\psset{unit=0.825cm,arrowsize=2pt 3}
%\begin{pspicture}(-4,0)(4,6)
%%\psgrid
%\rput(0,5){Choisir un nombre}\psframe(-1.9,5.4)(1.9,4.6)
%\psline{->}(-1.9,5)(-2.6,5)(-2.6,4.1)
%\rput(-2.6,3.7){Soustraire 1}\psframe(-3.8,4.1)(-1.4,3.3)\psline{->}(1.9,5)(2.6,5)(2.6,4.1)
%\rput(2.6,3.7){Soustraire 6}\psframe(1.4,4.1)(3.8,3.3)
%\psline{->}(-1.4,3.7)(-0.6,3.7)(-0.6,2.3)\psline{->}(1.4,3.7)(0.6,3.7)(0.6,2.3)
%\rput(0,1.9){\small Multiplier les deux résultats obtenus}\psframe(-3.3,2.3)(3.3,1.5)
%\psline{->}(0,1.5)(0,0.8)
%\rput(0,0.5){Ajouter 5}\psframe(-1,0.8)(1,0.2)
%%\psframe
%\end{pspicture}
%\end{tabularx}
%\end{center}

\begin{enumerate}
\item %Montrer que, lorsque le nombre choisi est 4, le résultat obtenu avec le programme A est 5.
On obtient successivement :

{\Large $4 \overset{\times 3}{\longmapsto} \quad12 \quad \overset{+ 15}{\longmapsto} \quad27 \quad \overset{\div 3}{\longmapsto} \quad 9 \quad \overset{- 4}{\longmapsto} \quad 5$}

\item %Montrer que, lorsque le nombre choisi est $- 2$, le résultat obtenu avec le programme A est 5.
{\Large $-2 \overset{\times 3}{\longmapsto} \quad - 6 \quad \overset{+ 15}{\longmapsto} \quad 9 \quad \overset{\div 3}{\longmapsto} \quad 3 \quad \overset{- (- 2)}{\longmapsto} \quad 5$}
\item %Justifier que l'affirmation suivante est vraie :

\begin{center}\og Le programme A donne toujours le même résultat. \fg\end{center}

En effet {\Large $a \overset{\times 3}{\longmapsto} \quad 3a \quad \overset{+ 15}{\longmapsto} \quad 3a + 15 = 3(a + 5) \quad \overset{\div 3}{\longmapsto} \quad a + 5\quad \overset{- a}{\longmapsto} \quad 5$}.

Quel que soit le nombre de départ $a$, le nombre trouvé à la fin est 5.
\item %Lorsque le nombre choisi est 10, quel résultat obtient-on avec le programme B ?
On calcule d'une part $10 - 1 = 9$, de l'autre $10 - 6 = 4$ ; le produit de ces deux nombres est égal à $9 \times 4 = 36$ et enfin $36 + 5 = 41$.
\item %Il existe exactement deux nombres pour lesquels les programmes A et B fournissent à chaque fois des résultats identiques.
En partant de $x$ le programme A donne le résultat 5 et avec le programme B, on obtient le nombre $(x - 1)(x - 6) + 5$.
Les résultats sont identiques si :

$5 = (x - 1)(x - 6) + 5$ autrement dit si $(x - 1)(x - 6) = 0$ cettez équation produit a pour solution 1 et 6

%Quels sont ces deux nombres?
1 et 6 sont bien les deux seuls nombres qui donnent comme résultat 5 par les deux programmes.
\end{enumerate}

