
Dans cet exercice, toutes les longueurs sont exprimées en centimètres.

On considère :
\begin{itemize}[label=$\bullet$]
\item le rectangle ABCD tel que AD = $x$ et AB = $16 - 2x$ ;
\item le carré EFGH tel que EF $= 2x$.
\end{itemize}

\psset{unit=0.85cm,arrowsize=2pt 3}
\begin{center}
\begin{pspicture}(0,0)(15,5)
%\psgrid
\psframe(0,2)(10,4)
\psframe(11,0)(15,4)
\psline[linewidth=0.8pt]{<->}(-0.5,2)(-0.5,4)
\psline[linewidth=0.8pt]{<->}(0,1.8)(10,1.8)
\psline[linewidth=0.8pt]{<->}(11,-0.5)(15,-0.5)
\uput[l](-0.5,3){$x$}\uput[d](4.5,1.8){$16 - 2x$}\uput[d](13,-0.5){$2x$}
\uput[l](0,2){A}\uput[ul](0,4){D}\uput[dr](10,2){B}\uput[ur](10,4){C}
\uput[dl](11,0){E}\uput[l](11,4){H}\uput[dr](15,0){F}\uput[r](15,4){G}
\psframe(0,2)(0.2,2.2)\psframe(0,4)(0.2,3.8)
\psframe(10,4)(9.8,3.8) \psframe(10,2)(9.8,2.2)
\psframe(11,0)(11.2,0.2) \psframe(15,0)(14.8,0.2)
\psframe(15,4)(14.8,3.8) \psframe(11,4)(11.2,3.8)
\end{pspicture}
\end{center}

\bigskip

\textbf{PARTIE A :} Dans cette partie, $x = \np[cm]{1,5}$.

\medskip

\begin{enumerate}
\item Calculer le périmètre du carré EFGH.
\item Calculer AB.
\item Construire en vraie grandeur le rectangle ABCD.
\item Les périmètres de ABCD et EFGH sont-ils égaux ?
\end{enumerate}

\bigskip

\textbf{PARTIE B :} Dans cette partie, on cherche pour quelle(s) valeur(s) de $x$, le périmètre du rectangle est égal au périmètre du carré.

\medskip

\begin{enumerate}
\item Pour essayer de répondre au problème, on utilise la feuille de calcul suivante:

\smallskip

\renewcommand \arraystretch {1}
\begin{tabularx}{\linewidth}{|c|m{3.85cm}|*{6}{>{\centering \arraybackslash}X|}}
\cline{2-8}
\multicolumn{1}{c|}{}&\centering A&B&C&D&E&F&G\\\hline
1&Valeur de $x$&1&2&3&4&5&6\\\hline
2&Périmètre du carré&8&16&24&32&40&48\\\hline
3&Périmètre du rectangle&30&28&26&24&22&20\\\hline
\end{tabularx}

\begin{enumerate}
\item Quel formule a-t-on pu saisir dans la cellule B2 avant de l'étirer jusqu'à G2?
\item Ce tableau nous permet-il de trouver une valeur de $x$ pour laquelle les deux périmètres sont égaux ?
\end{enumerate}

\item 
	\begin{enumerate}
		\item Montrer que le périmètre du rectangle peut s'écrire $-2x + 32$.
		\item Déterminer la solution au problème par la résolution d'une équation.
	\end{enumerate}
\end{enumerate}

