
\medskip

\textbf{Partie A}

\medskip

\begin{enumerate}
\item Dans le triangle ABC rectangle en B, le théorème de Pythagore permet d'écrire $\text{AC}^2 = \text{AB}^2 + \text{BC}^2 = 600^2 + 450^2 = \np{360000} + \np{202500} = \np{562500} = 750^2$, d'où AC $= 750$~(m).
\item 
	\begin{enumerate}
		\item Les droites (DE) et (AB) étant perpendiculaires à la même droite (BC) sont parallèles.
		\item D'après le résultat précédent et les points A, E d'une part, B, D, C de l'autre sont alignés : le théorème de Thalès permet d'écrire l'égalité des rapports :
		
$\dfrac{\text{CD}}{\text{CB}} = \dfrac{\text{CE}}{\text{CA}} = \dfrac{\text{ED}}{\text{AB}}$.

En particulier $\dfrac{\text{CD}}{\text{CB}} =  \dfrac{\text{ED}}{\text{AB}}$ soit $\dfrac{270}{450} = \dfrac{\text{ED}}{600}$.

Or $\dfrac{270}{450} = \dfrac{90 \times 3}{90 \times 5} = \dfrac35$.

Ob a donc $\dfrac35 = \dfrac{\text{ED}}{600}$, d'où en multipliant par 600 :

ED $ = \dfrac35 \times 600 = \dfrac{3 \times 5 120}{5} = 3 \times 120 = 360$~(m).
	\end{enumerate}
\item L'aire du triangle CDE est égale à $\dfrac{\text{DE} \times \text{DC}}{2} = \dfrac{360 \times 270}{2} = 180 \times 270 = \np{48600}~\left(\text{m}^2\right)$.
\end{enumerate}

\medskip

\textbf{Partie B}

\medskip

\begin{enumerate}
\item On a $\dfrac{80}{16} = 5, \: \dfrac{60}{12} = 5$ et $\dfrac{50}{8}  = 6,25$ : le ratio n'est pas respecté.
\item Il faut 80 kg de blé pour \np{10000}~m$^2$, soit $\dfrac{80}{\np{10000}}$ ~kg pour 1 m$^2$ et enfin 

$\dfrac{80}{\np{10000}} \times \np{48600} = 80 \times 4,86 = 388,8$~(kg) pour le terrain CDE.
\item Pour le seigle il aura besoin  de la même façon de : $\dfrac{60}{\np{10000}} \times \np{48600} = 60 \times 4,86 = 291,6$~(kg)

Pour les pois il lui faudra acheter : $\dfrac{50}{\np{10000}} \times \np{48600} = 50 \times 4,86 = 243$~(kg).

Tout ceci lui coûtera :

$388,8 \times 1,4 + 291,6 \times 1,3 + 243 \times 2,1 = \np{1433,7}$, soit \np{1433,70}~\euro : son budget est suffisant.
\end{enumerate}

\bigskip

