
\medskip

Un agriculteur souhaite cultiver un champ représenté par le triangle ABC ci-contre.

Sur la figure qui n'est pas à l'échelle, on a les informations suivantes :

\begin{minipage}{0.48\linewidth}
\begin{itemize}[label=$\bullet~$]
\item le triangle ABC est rectangle en B ;
\item les points C, E et A sont alignés ;
\item les points C, D et B sont alignés ;
\item AB $= 600$ m ; BC $= 450$ m ; CD $= 270$ m.
\end{itemize}

\emph{Les parties {\rm A} et {\rm B} sont indépendantes}
\end{minipage}\hfill
\begin{minipage}{0.5\linewidth}
\psset{unit=1cm,arrowsize=2pt 3}
\begin{pspicture}(9,5.6)
\pspolygon(0.4,1)(5.9,1)(5.9,5.3)%ABC
\psline(5.9,2.5)(2.3,2.5)%DE
\psframe(5.9,2.5)(5.7,2.7)\psframe(5.9,1)(5.7,1.2)
\uput[l](0.4,1){A}\uput[r](5.9,1){B}\uput[u](5.9,5.3){C}\uput[dr](5.9,2.5){D}\uput[ul](2.3,2.5){E}
\psline[linewidth=0.8pt]{<->}(0.4,0.8)(5.9,0.8)\uput[d](3.15,0.8){600 m}
\psline[linewidth=0.8pt]{<->}(6.3,2.5)(6.3,5.3)\rput{90}(6.5,3.9){270 m}
\psline[linewidth=0.8pt]{<->}(6.7,1)(6.7,5.3)\rput{90}(6.9,3.15){450 m}
\end{pspicture}
\end{minipage}

\textbf{Partie A : étude géométrique du terrain}

\medskip

\begin{enumerate}
\item Montrer que le segment [AC] mesure 750 mètres.
\item 
	\begin{enumerate}
		\item Montrer que les droites (ED) et (AB) sont parallèles.
		\item Montrer que le segment [DE] mesure 360 mètres.
	\end{enumerate}
\item Montrer que l'aire du triangle CDE est \np{48600}~m$^2$.
\end{enumerate}

\medskip

\textbf{Partie B : étude du prix du mélange de graines}

\medskip

L'agriculteur souhaite semer un mélange de graines (blé, seigle et pois) en respectant les indications suivantes.

\medskip

\begin{tabularx}{\linewidth}{Xm{0.5cm}X}
Indication 1 : prix au kilo pour chaque type de graine&&Indication 2 : répartition du type de graines pour une surface de \np{10000}~m$^2$\\
\end{tabularx}
\begin{tabularx}{\linewidth}{|X|m{0.5cm}|X|}\cline{1-1}\cline{3-3}
$\bullet~$ Blé: 1,40 \euro/kg		&	&$\bullet~$ Blé : 80 kg\\ 
$\bullet~$ Seigle: 1,30 \euro/kg	&	&$\bullet~$ Seigle : 60 kg\\
$\bullet~$ Pois: 2,10 \euro/kg		&	&$\bullet~$ Pois: 50 kg\\ \cline{1-1}\cline{3-3}
\end{tabularx}

\medskip

\begin{enumerate}
\item Un vendeur lui propose des sacs contenant un mélange de blé, seigle, et pois selon le ratio 16 : 12 : 8. Montrer que la composition de ce sac ne respecte pas l'indication 2.
\item L'agriculteur souhaite semer le mélange de graines sur la partie du champ représentée par le triangle CDE dont l'aire mesure \np{48600}~m$^2$. Il a calculé qu'il doit prévoir 388,80~kg de blé pour respecter la répartition indiquée dans l'énoncé.

Justifier le calcul de l'agriculteur.
\item L'agriculteur dispose d'un budget de \np{1500}~\euro{} pour semer le mélange de graines sur la totalité des \np{48600} m$^2$ de terrain.

Il a calculé qu'il doit acheter 388,80 kg de blé, 291,6 kg de seigle et 243 kg de pois pour respecter la répartition indiquée dans l'énoncé.

L'agriculteur dispose-t-il d'un budget suffisant?
\end{enumerate}

\bigskip

