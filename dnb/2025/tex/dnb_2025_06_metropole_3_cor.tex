
\medskip

Cet exercice est un questionnaire à choix multiples (QCM). Aucune justification  n'est demandée.

Pour chaque question, quatre réponses (A, B, C ou D) sont proposées.

Une seule réponse est exacte. Recopier sur la copie le numéro de la question et la lettre
correspondant à la réponse exacte. 

\medskip

\textbf{Question 1}


$8,4 = 3 \times 2,8$, donc un melon coûte 2,80~\euro{} et 5 melons coûtent $5 \times 2,80 = \dfrac{2,8 \times 10}{2} = \dfrac{28}{2} = 14$~(\euro)

\textbf{Question 2}

Une symétrie autour de la droite perpendiculaire au segment ayant pour extrémités les deux points les plus proches des deux figures, perpendiculaire au milieu de ce segment.


\textbf{Question 3}

Augmenter de 20\,\% c'est multiplier par $1 + \dfrac{20}{100} = 1 + 0,20 = 1,20$.

Donc $350 \times 1,2 = 420$~\euro.
\textbf{Question 4}

En prenant comme base [AB] et [BC] comme hauteur, l'aire est égale à $\dfrac{6 \times 4,5}{2} = 3 \times 4,5 = 13,5$~cm$^2$.

\textbf{Question 5}

$(2x + 3)(x - 4) = 2x \times x - 2x \times 4 + 3 \times x 3 \times (- 4) = 2x^2 - 8x + 3x - 12 = 2x^2 - 5x - 12$.

\textbf{Question 6}


Avec la base rectangulaire d'aire $\mathcal{B} = 7 \times 4 = 28$~cm$^2$ et la hauteur $h = 12$~(cm), on a :

\[V = \dfrac{\mathcal{B} \times h}{3} = \dfrac{28 \times 12}{3} = 28 \times 4 = 112~\text{cm}^3.\]

bigskip

