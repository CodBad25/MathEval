
\medskip

\begin{enumerate}
\item On obtient : $1 \longmapsto -2 \longmapsto 2 \longmapsto 8$.
\item $- 2 \longmapsto 4 \longmapsto 8 \longmapsto 32$.
\item En partant du nombre $x$ :

$x \longmapsto - 2x \longmapsto - 2x + 4 \longmapsto 4(- 2x + 4) = - 8x + 16$.
\item
	\begin{enumerate}
		\item De $- 8x + 16 = 4$ en ajoutant $8x$ à chaque membre, on obtient :
		
$16 = 4 + 8x$ puis en ajoutant $- 4$ à chaque membre :

$12 = 8x$ ou $4 \times 3 = 4 \times 2x$ d'où $3 = 2x$ et en multipliant chaque membre par $\dfrac 12$

$3 \times \dfrac 12 = x$ et enfin $x = \dfrac 32 = 1,5$. Donc l'équation a une solution $S = \{1,5\}$.
\item Le nombre de départ est 1,5.
	\end{enumerate}
\item $\bullet~$L'ordonnée à l'origine est égale à 16, donc le graphe 2 est disqualifié ;

$\bullet~$ Le coefficient directeur de la droite est égal à $- 8$ ; on doit donc en partant du point sur la droite de coordonnées (0~;~16) se déplacer horizontalement à droite de 1 puis verticalement de 8 vers le bas ou de 2 à droite et 16 vers le bas pour retrouver un point de la représentation : c'est ce que l'on peut faire sur la représentation graphique 3.
\end{enumerate}


\bigskip

