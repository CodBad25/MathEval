
\medskip


\parbox{0.5\linewidth}{À la kermesse du village, il y a un jeu de grande
roue. Le joueur lance la roue et gagne le lot indiqué.

On suppose que la roue est bien équilibrée et que les
secteurs sont superposables.

Les lots sont de deux sortes : les jouets (petite
voiture, poupée et ballon) et les sucreries (chocolat,
sucette et bonbons).

\medskip

\begin{enumerate}
\item Gilda lance la roue une fois. Quelle est la
probabilité qu'elle gagne un ballon ?
\item Marie lance la roue une fois. Quelle est la
probabilité qu'elle gagne une des sucreries ?
\item Roméo lance la roue deux fois. Quelle est la
probabilité qu'il gagne du chocolat puis une
petite voiture ?\end{enumerate}}\hfill\parbox{0.48\linewidth}{\psset{unit=0.5cm}
\begin{tikzpicture}
  % Définir les paramètres du cercle
  \def\radius{3} % Rayon du cercle
  \filldraw[black!80] (-0.2,3.2) -- (0.2,3.2) -- (0,2.8) -- cycle;
  \draw[-latex] (150:3.3) arc (150:120:3.3) ; 
  % Dessiner les secteurs et ajouter du texte le long de chaque secteur
  \foreach \nom [count=\i from 1] in {sucette,petite voiture,chocolat,poup{é}e,bonbons,ballon}  {
    % Calculer les angles de début et de fin pour chaque secteur
    \pgfmathsetmacro{\startangle}{60 * (\i - 1)};
    \pgfmathsetmacro{\endangle}{60 * \i};
    
    % Dessiner le secteur
    \draw (0,0) -- (\startangle:\radius) arc (\startangle:\endangle:\radius) -- cycle;
    
    
    % Ajouter du texte le long de l'arc
       % \path[
       %     postaction={decorate, decoration={text along path, text align=center, text={|\Huge|Texte %\i}, reverse path=true}}
       % ] (\startangle:\radius) arc (\startangle:\endangle:\radius);
         \draw (\startangle:\radius) -- ($(\startangle:\radius) + (0.6, 0.2)$);
         \filldraw[color=black!60, fill=black!95] ($(\startangle:\radius) + (0.6, 0.2)$) circle(0.08);
         
    % Ajouter du texte le long de l'arc
        \path[
            postaction={decorate, decoration={text along path, text align=center, raise=-4ex,  text={|\large | \nom }, reverse path=true}}
        ] (\startangle:\radius) arc (\startangle:\endangle:\radius);
        
        
  }
 %psarc{<-}(0,0){3.2}{120}{150}
  %\draw[line width=1.5pt] (0,0) -- (0.75,0.2);
  %\filldraw[color=black!60, fill=black!95] (30:3) circle(0.08pt)
  %\filldraw[color=black!60, fill=black!95, very thick] ({3 cos(\startangle)}, 0.2) circle (0.08);
  %\def\clou{\psline[linewidth=1.5pt](0,0)(0.75,0.2)\pscircle*(0.75,0.2){0.08}}
\end{tikzpicture}
}



\vspace{0,5cm}

