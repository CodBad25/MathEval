
\smallskip

Les parents de Teiki lui ont légué un terrain divisé en deux parcelles dont on peut assimiler la forme à la figure ci-dessous:

\begin{minipage}{0.4\linewidth}
DC = 10 m 

AC = 25 m 

DE = 22 m 

(AB) // (DE)

La figure n'est pas à l'échelle
\end{minipage}\hfill
\begin{minipage}{0.58\linewidth}
\psset{unit=1cm,arrowsize=2pt 3}
\begin{pspicture}(8.5,5.8)
\pspolygon(1.6,0.4)(8.4,0.4)(1.6,5)%ABC
\psline(1.6,3)(4.6,3)%DE
\psframe(1.6,0.4)(1.85,0.65)
\psframe(1.6,3)(1.85,3.25)
\psline[linewidth=0.8pt]{<->}(1.4,3)(1.4,5)\rput{90}(1.2,4){10 m}
\psline[linewidth=0.8pt]{<->}(1.6,2.8)(4.6,2.8)\uput[d](3.1,2.9){22 m}
\psline[linewidth=0.8pt]{<->}(0.6,0.4)(0.6,5)\rput{90}(0.4,2.7){25 m}
\uput[ur](4.6,3){E}\uput[dl](1.6,0.4){A}\uput[dr](8.4,0.4){B}\uput[u](1.6,5){C}\uput[dl](1.6,3){D}
\uput[r](1.5,3.7){Parcelle 1}\uput[r](1.5,1.1){Parcelle 2}
\end{pspicture}
\end{minipage}

\textbf{Partie A}

\medskip

\begin{enumerate}
\item L'aire de la parcelle 1 correspond à l'aire du triangle CDE.
	\begin{enumerate}
		\item Justifier, par un calcul, que l'aire de la parcelle 1 vaut $11 0$~m$^2$.
		\item Teiki veut planter des arbres sur la parcelle 1, il a besoin de $90$~m$^2$ pour le
faire.

Indiquer s'il pourra planter ces arbres.

Justifier la réponse.
	\end{enumerate}
\item Le triangle ABC est constitué des parcelles 1 et 2.
	\begin{enumerate}
		\item Vérifier, en utilisant le théorème de Thalès, que AB $= 55$~m. Détailler les calculs sur la copie.
		\item Calculer l'aire du triangle ABC. Exprimer le résultat en m$^2$.
		\item Calculer l'aire de la parcelle 2.

Détailler le calcul sur la copie. Exprimer le résultat en $^2$.
		\item Pour construire une maison, il faut au minimum une parcelle de $550^2$.

Indiquer si Teiki pourra construire sa maison sur la parcelle 2.

Justifier la réponse.
	\end{enumerate}
\end{enumerate}

\medskip

\textbf{Partie B}

\medskip

Teiki souhaite clôturer son terrain.

\medskip

\begin{enumerate}
\item En utilisant le théorème de Pythagore, vérifier que la longueur BC, arrondie à l'unité, est égale à $60$~m.
\item Calculer le périmètre du terrain ABC. Exprimer le résultat en m. Arrondir à l'unité.
\end{enumerate}
\medskip

Teiki souhaite installer une clôture autour de son terrain ABC.

Il hésite entre deux types de clôture:

\begin{description}
\item[ ] Type A : Canisses en Osier à \np{2200}~F le mètre avec livraison gratuite;
\item[ ] Type B : Canisses en roseaux fendus à \np{1920}~F le mètre avec un forfait livraison.
\end{description}

Le graphique ci-dessous représente le prix en F pour chacun des deux types de clôture en fonction de la longueur en m.

\begin{landscape}
\psset{xunit=0.125cm,yunit=0.035cm,arrowsize=2pt 3}
\begin{pspicture}(-30,-2)(150,340)
\psaxes[linewidth=1.25pt,Dx=5,Dy=20,labelFontSize=\scriptstyle]{->}(0,0)(0,0)(150,340)
\multido{\n=0+5}{31}{\psline[linewidth=0.2pt](\n,0)(\n,340)}
\multido{\n=0+20}{18}{\psline[linewidth=0.2pt](0,\n)(150,\n)}
\psplot[plotpoints=2000,linewidth=1.25pt,linecolor=red]{0}{145}{x 2.207 mul}
\psplot[plotpoints=2000,linewidth=1.25pt,linecolor=blue]{0}{145}{x 1.931 mul 20 add}
\uput[r](0,330){Prix en milliers de F $y$}
\uput[u](135,0){Longueur en m $x$}
\end{pspicture}
\end{landscape}


\begin{enumerate}[resume]
\item Écrire sur le graphique ci-dessus, pour chacune des deux représentations graphiques, celle correspondant au « type A » et celle correspondant au
« type B ».
\item Déterminer, à l'aide de la représentation graphique correspondant au type B, le montant du forfait livraison. Exprimer le résultat en F.
\item Indiquer, à l'aide du graphique, la clôture de type A ou de type B, qui coûtera le moins cher pour une longueur de $140$~m.

Justifier la réponse et laisser les traits de lecture apparents sur le graphique.
\end{enumerate}

\medskip

