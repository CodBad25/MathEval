
\medskip

%\begin{minipage}{0.6\linewidth}
%Lors d'une promotion, une agence de voyage propose un tirage au sort permettant de gagner une journée de vacances. 
%
%Chaque client fait tourner la roue ci-contre, partagée en 8 secteurs de même mesure.
%
%Dans cet exemple le client gagne une journée de vacances à la campagne.
%\end{minipage}\hfill
%\begin{minipage}{0.35\linewidth}
%\psset{unit=1cm,arrowsize=2pt 3}
%\begin{pspicture}(-2.7,-2.7)(2.7,2.7)
%\pscircle*(0,0){2.5}
%\pscircle[linewidth=2.5pt,linecolor=white](0,0){0.4}
%\pspolygon[linecolor=white](0.2;0)(0.2;120)(0.2;240)
%\multido{\n=20+45}{8}{\psline[linewidth=1.5pt,linecolor=white](0.4;\n)(2.5;\n)}
%\rput{-2.5}(1.7;-2.5){\white Mer}\rput{-177.5}(1.7;-177.5){\white Mer}\rput{-92.5}(1.7;-92.5){\white Mer}
%\rput{42.5}(1.6;42.5){\white Montagne}\rput{222.5}(1.6;222.5){\white Montagne}
%\rput{87.5}(1.7;87.5){\white Ville}\rput{-47.5}(1.7;-47.5){\white Ville}
%\rput{132.5}(1.6;132.5){\white Campagne}\psline[linewidth=2.5pt]{->}(3.3;132.5)(2.5;132.5)
%\end{pspicture}
%\end{minipage}
%
%\medskip
%
%L'agence présente les résultats des tirages au sort, effectués sur une semaine, dans le diagramme ci-dessous.
%
%\psset{xunit=1cm,yunit=0.1cm}
%\begin{center}
%\begin{pspicture}(-1,-10)(12,82)
%\multido{\n=0+2}{42}{\psline[linewidth=0.3pt](0,\n)(12,\n)}
%\multido{\n=0+10,\na=0+10}{9}{\psline[linewidth=0.9pt](0,\n)(12,\n)\uput[l](0,\n){\na}}
%\psframe*(1.5,0)(2.5,41)\psframe*(4.5,0)(5.5,55)\psframe*(7.5,0)(8.5,68)\psframe*(10.5,0)(11.5,35)
%\uput[d](2,0){Ville}\uput[d](5,0){Montagne}\uput[d](8,0){Mer}\uput[d](11,0){Campagne}
%\uput[u](5,55){55}\uput[u](11,35){35}\rput{90}(-0.8,40){Effectif}
%\end{pspicture}
%\end{center}

\begin{enumerate}
\item %Indiquer le nombre de fois où la roue s'est arrêtée sur un secteur « Ville ».
On lit que qu'il y a eu 42 sorties \og Ville \fg.
\item %Montrer que le nombre total de tirages au sort effectués cette semaine-là est 200.
Le nombre total de tirages cette semaine est égale à :
\[42 + 55 + 68 + 35 = (42 + 68) + (55 + 35) = 110 + 90 = 200.\]
\item %Calculer la fréquence d'apparition du secteur \og Montagne \fg.
Dans la semaine il y a eu 55 sorties de la montagne sur un total de 200 tirages, soit une fréquence de $\dfrac{55}{200} = \dfrac{27,5}{100} = 0,275$.
\end{enumerate}

On fait tourner la roue.
\begin{enumerate}[resume]
\item %À l'aide des informations sur la roue, calculer la probabilité de s'arrêter sur \og Montagne \fg.

La roue est partagée en 8 secteurs, dont 2 sont marqués Montagne : la fréquence de tirage de ce lieu est donc égal à $\dfrac 28 = \dfrac 14 = 0,25$.
\item %Elsa participe au tirage au sort. En examinant la roue elle pense qu'elle a plus de chance de gagner une journée à la montagne qu'à la ville. Indiquer si elle a raison ou tort. Justifier la réponse.
De la même façon, il y a 2 secteurs marqués Ville sur 8 secteurs : la probabilité de tirer Ville est la même que celle de tirer Montagne : Elsa a tort.
\end{enumerate}

\bigskip

