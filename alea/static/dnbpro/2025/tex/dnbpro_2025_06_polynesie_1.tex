
\smallskip

Cet exercice est un questionnaire à choix multiples (QCM).
Pour chaque question, trois réponses sont proposées mais une seule est exacte.
Une réponse juste rapporte 4 points, une réponse fausse ou l'absence de réponse ne rapporte aucun point.
Pour chaque question, recopier sur la copie, sans justifier, la réponse choisie :

\begin{center} Réponse A \quad ou \quad Réponse B \quad ou\quad  Réponse C.

\medskip

\begin{tabularx}{\linewidth}{|c|m{7cm}|*{3}{>{\centering \arraybackslash}X|}}\hline
\multicolumn{2}{|c|}{Questions}&\multicolumn{3}{|c|}{Réponses proposées}\\ \cline{3-5}
\multicolumn{2}{|c|}{~}&Réponse A &Réponse B& Réponse C\\ \hline
1&Soit la fonction $f$ définie par :
\[f(x) = 5x + 3.\]

L'image de 1 par la fonction $f$ est:	&5,3	&8	&54\\ \hline
2&Le volume V, en cm$^3$, d'un cube de
4 cm de côté est:						&12		&16	&64\\ \hline
3&\centering $\dfrac 75 + \dfrac 25 = $&$\dfrac{9}{10}$&$\dfrac 95$&$\dfrac{14}{25}$\rule[-4mm]{0mm}{10mm}\\ \hline
4&L'équation $8x-5= 19$ a pour solution:&$-1$&0&3\\ \hline
5&Voici les notes de Vaitiare :

\[12\quad 9\quad 14\quad 15\quad  19\quad  15\]

La moyenne des notes de Vaitiare est:&15&14&9\\ \hline
\end{tabularx}
\end{center}

\medskip

