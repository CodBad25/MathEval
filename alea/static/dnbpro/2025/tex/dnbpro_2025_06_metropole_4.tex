
\medskip

Paul prévoit de faire de la randonnée pendant les vacances. Il utilise habituellement un bâton
de marche de longueur réglable.

La longueur minimale du bâton est 48 cm.

Il doit le placer dans la valise représentée ci-dessous.

\psset{unit=1cm,arrowsize=2pt 3}
\begin{center}
\begin{pspicture}(8,3.6)
%\psgrid
\psframe(1.6,0.2)(3.4,3)
\psline[linewidth=1.25pt](2.2,3)(2.2,3.8)\psline[linewidth=1.25pt](2.8,3.)(2.8,3.8)
\psline[linewidth=3pt](2,3.8)(3,3.8)
\psline[linewidth=0.8pt]{<->}(1.4,0.2)(1.4,3)\rput{90}(1.2,1.6){45 cm}
\rput(4.8,1.6){Valise de Paul}
\end{pspicture}
\end{center}

\begin{enumerate}
\item Indiquer s'il est possible de mettre le bâton à la verticale dans la valise.

Justifier la réponse.

\medskip

On schématise le fond de la valise par le rectangle ABCD ci-dessous (le dessin n'est pas à l'échelle).

\psset{unit=1cm,arrowsize=2pt 3}
\begin{center}
\begin{pspicture}(1,0)(12,3.8)

\psframe(5.7,0.8)(12.8,3.6)
\psline(5.7,0.8)(12.8,3.6)
\psframe(5.7,3.6)(5.9,3.4)\psframe(5.7,0.8)(5.9,1)
\psframe(12.8,3.6)(12.6,3.4)\psframe(12.8,0.8)(12.6,1)
\uput[r](1,2.5){Longueur $L$ : 45 cm}\uput[r](1,1.4){Largeur $l$ : 32 cm}
\psline{<->}(5.7,0.6)(12.8,0.6)\uput[d](9.25,0.6){$L$}
\psline{<->}(5.5,0.8)(5.5,3.6)\uput[l](5.5,2.2){$l$}
\uput[dl](5.7,0.8){A} \uput[dr](12.8,0.8){B} \uput[ur](12.8,3.6){C} \uput[ul](5.7,3.6){D} 
\end{pspicture}
\end{center}

\item  Parmi les propositions suivantes, recopier sur la copie celle qui est exacte.
Le triangle ABC est:

\begin{tabularx}{\linewidth}{*{4}{X}}
$\bullet~$ isocèle &$\bullet~$ rectangle &$\bullet~$ isocèle rectangle& $\bullet~$ équilatéral\\
\end{tabularx}

\item Calculer, à l'aide du théorème de Pythagore, la longueur AC en centimètre.

Arrondir le résultat à l'unité.
\item Justifier que Paul peut placer son bâton dans le fond de la valise.
\end{enumerate}

\bigskip

