
\medskip

Dans un avion, les valises sont placées en cabine ou en soute. Les compagnies aériennes appliquent un surcoût aux valises en cabine qui pèsent plus de 15 kilogrammes.
Le programme scratch ci-dessous permet de calculer la masse en trop et le montant du surcoût en euros demandé aux clients pour conserver leur valise de plus de 15 kilogrammes en cabine.

\medskip

\begin{enumerate}
\item %À l'aide de la ligne 5, donner le prix en euros (\euro) du kilogramme supplémentaire.
On lit à la ligne 5 que tout kilo au dessus des 15 premiers kilos est facturé 12,30~\euro. 
\item %Indiquer ce que permet de calculer la ligne 4.
La ligne 4 calcule le montant à payer pour les kilos au delà de 15.
\item %Calculer la valeur du montant affiché par le programme pour une valise de $17,50$~kg.
On a $(17,50 - 15) \times 12,30 = 30,75$~(\euro)
\end{enumerate}

%En soute, un surcoût est appliqué aux valises qui pèsent plus de $23$~kg.
%
%Chaque kilogramme supplémentaire coûte $10,70$~\euro.

\begin{enumerate}[resume]
\item %Indiquer, sur la copie, le numéro des deux lignes à modifier pour adapter le programme à une valise en soute.
Il faut modifier les lignes 3 et 4.
\item %Écrire, sur la copie, les deux lignes avec les valeurs modifiées pour obtenir le prix à payer pour une valise en soute pesant $23$ kg.

\begin{scratch}[numblocks,scale=0.8,start=3]
\blockvariable{mettre \selectmenu{$x$} à \ovaloperator{\ovalvariable{masse} -\ovalnum{23}}}
\blockvariable{mettre \selectmenu{résultat} à \ovaloperator{\ovalvariable{x} *\ovalnum{10,70}}}
\end{scratch}

\end{enumerate}




