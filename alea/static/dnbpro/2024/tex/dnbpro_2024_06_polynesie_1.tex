
\smallskip

Cet exercice est un questionnaire à choix multiples (QCM). \\
Pour chaque question, une seule des trois réponses proposées est exacte.\\
Pour chaque question, recopier sur la copie, sans justifier, la réponse choisie : Réponse A,
Réponse B ou Réponse C.\\

\begin{tabularx}{\linewidth}{|m{6cm}|*{3}{>{\centering \arraybackslash}X|}}
\hline
\centering Questions & \multicolumn{3}{ c|}{Réponses proposées}\\
\cline{2-4}
 & Réponse A & Réponse B & Réponse C\\
\hline
\begin{enumerate}
\item Pauro possède 50~F{}. Rai a le double de Pauro. Rai dépense 2 fois 15~F{}.
\end{enumerate}
Combien reste-t-il à Rai?
& 70 F & 30 F & 20 F\\ 
\hline
\begin{enumerate}[start=2]
\item Voici les notes de Haiata:\newline
8 \quad 18 \quad  14 \quad 16 \quad 12 \quad 16
\end{enumerate}
La moyenne des notes de Haiata est:
& 84 & 14 & 6\\
\hline
\begin{enumerate}[start=3]
\item Dans une urne, il y a 15 boules : 5 boules bleues, 7 boules vertes et 3 boules rouges. Léo tire une boule au hasard.
\end{enumerate}
La probabilité de tirer une boule rouge est:
& $\dfrac{5}{15}$ & $\dfrac{3}{15}$ & $\dfrac{12}{15}$\\
\hline
\begin{enumerate}[start=4]
\item Un pavé droit a une longueur $L$ de 12 m, une largeur $\ell$ de 6 m et une hauteur $h$ de $2,2$0 m.

\scalebox{0.8}{
\psset{xunit=0.5cm,yunit=0.2cm,radius=0pt,arrowsize=3pt 2}
\def\xmin {-1}   \def\xmax {9} \def\ymin {-2}   \def\ymax {10}
\begin{pspicture}(\xmin,\ymin)(\xmax,\ymax)
%\psgrid[subgriddiv=1, gridlabels=0, gridcolor=lightgray] 
\Cnode*(0,0){A}  \Cnode*(2,2){D}\Cnode*(6,0){B} \Cnode*(8,2){C} 
\Cnode*(6,6){F}    \Cnode*(8,8){G} \Cnode*(0,6){E} \Cnode*(2,8){H} 
\pspolygon[linecolor=white,fillstyle=solid,fillcolor=lightgray!30](A)(B)(C)(G)(H)(E)
\psline(G)(C)(B)(A)(E)(F)(G)(H)(E) \psline(B)(F)
{\psset{linestyle=dashed}
\psline(A)(D)(C) \psline(D)(H)}
\psline{<->}(0,-1)(6,-1)\uput[d](3,-1){$L$} 
\psline{<->}(-0.3,0)(-0.3,6) \uput[l](-0.3,3){$h$} 
\psline{<->}(7,0)(9,2) \uput[dr](8,1){$\ell$}
\end{pspicture}
}% fin du scalebox
\end{enumerate}
Le volume du pavé droit est de:
& $20,2$ m$^3$ & $132$ m$^3$ & $158,4$ m$^3$\\
\hline
%\begin{minipage}[t][2cm][t]{6cm}
%\vspace{0pt}
\begin{enumerate}[start=5]
\item Soit la fonction $f$ définie par:\newline
$f(x)=-2x+10$
\end{enumerate}
La représentation graphique de $f$ est:\newline
%\end{minipage}
&
\parbox{3cm}{
\psset{unit=0.15cm,radius=0pt,arrowsize=3pt 3}
\def\xmin{-7}   \def\xmax{7} \def\ymin{-9}   \def\ymax {12}
\begin{pspicture*}(\xmin,\ymin)(\xmax,\ymax)
\psgrid[unit=0.75cm,subgriddiv=1, gridlabels=0, gridcolor=lightgray]
\psaxes[labelFontSize=\scriptstyle, ticksize=-0pt 0pt, Dx=5, Dy=5](0,0)(\xmin,\ymin)(\xmax,\ymax)
\uput[dl](0,0){\footnotesize 0}
\psplot{\xmin}{\xmax}{-2 x mul 10 add}
\end{pspicture*}
}
&
\parbox{3cm}{
\psset{unit=0.1cm,radius=0pt,arrowsize=3pt 3}
\def\xmin{-12}   \def\xmax{12} \def\ymin{-18}   \def\ymax {18}
\begin{pspicture*}(\xmin,\ymin)(\xmax,\ymax)
\psgrid[unit=1cm,subgriddiv=1, gridlabels=0, gridcolor=lightgray]
\psaxes[labelFontSize=\scriptstyle, ticksize=-0pt 0pt, Dx=10, Dy=10](0,0)(\xmin,\ymin)(\xmax,\ymax)
\uput[dl](0,0){\footnotesize 0}
\psplot{\xmin}{\xmax}{2 x mul 10 sub}
\end{pspicture*}
}
&
\parbox{3cm}{
\psset{unit=0.5cm}
\def\xmin{-1.5}   \def\xmax{3.5} \def\ymin{-1.5}   \def\ymax {6.5}
\begin{pspicture*}(\xmin,\ymin)(\xmax,\ymax)
\psgrid[subgriddiv=1, gridlabels=0, gridcolor=lightgray]
\psaxes[labelFontSize=\scriptstyle, ticksize=-0pt 0pt, Dx=1, Dy=1](0,0)(\xmin,\ymin)(\xmax,\ymax)
\uput[dl](0,0){\footnotesize 0}
\psplot{\xmin}{\xmax}{x  -2 mul 5 add}
\end{pspicture*}
}\\
\hline
\end{tabularx}

\bigskip


