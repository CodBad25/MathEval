
\medskip

Comme chaque dimanche, Maui se rend au marché de Papeete pour faire quelques achats.

\begin{tabular}{@{} l !{\textbullet} l}
Il achète& une pièce de  Pua'a roti  à \np{1760} F le morceau,\\
& deux paquets de  Firi-firi  à 500 F le paquet,\\
& deux poissons perroquet à \np{1200} F l'unité,\\
& un paquet de  Taro  à 800 F le Paquet,\\
& un tas de  Fe'i  à 400 F le tas,\\
& une bouteille de  Miti haari  à 500 F la bouteille.
\end{tabular}

La facture incomplète des achats de Maui au marché de Papeete est réalisée sur un tableur.

\begin{enumerate}
\item  Compléter cette facture.

\begin{center}
\renewcommand{\arraystretch}{1.3}
\begin{tabular}{|*5{c|}}
\hline
& A & B & C & D\\
\hline
1 & \textbf{Aliment} & \textbf{Quantité} & \textbf{Prix unitaire en F} & \textbf{Prix en F}\\
\hline
2 & Pièce de Pua'a roti & 1 & $\cdots$ & $\cdots$ \\
\hline
3 & Paquet de Fri fri & $\cdots$ & 500 & $\cdots$ \\
\hline
4 & Poisson perroquet &  $\cdots$ & \np{1200} & $\cdots$ \\
\hline
5 & Paquet de Taro &  $\cdots$ & \np{800} & $\cdots$ \\
\hline
6 & Tas de Fe'i & 1 & $\cdots$ & $\cdots$ \\
\hline
7 & Bouteille de \og Miti haari \fg{} & 1 & $\cdots$ & $\cdots$  \\
\hline
8 & \multicolumn{2}{c|}{~~} &  \textbf{PRIX TOTAL en F} & $\cdots$ \\
\hline
\end{tabular}
\end{center}

\item Recopier sur la copie la formule à insérer dans la cellule D3, parmi celles proposées
ci-dessous:

\begin{center}
{\ttfamily\Large
\begin{tabular}{| p{3cm} | c | p{3cm} | c | p{3cm} |}
\cline{1-1} \cline{3-3} \cline{5-5} 
= 1 * 500 & & = B3 * C3 & & = B3 + C3\\
\cline{1-1} \cline{3-3} \cline{5-5}
\end{tabular} }
\end{center}
\end{enumerate}

On admet que le montant total de la facture s'élève à \np{6860} F{}.\\
Une remise de 15\,\% est accordée à Maui.

\begin{enumerate}[resume]
\item  Calculer le montant de cette remise. Exprimer le résultat en F{}.
\item Calculer le prix payé par Maui. Exprimer le résultat en F{}.
\end{enumerate}

\bigskip

