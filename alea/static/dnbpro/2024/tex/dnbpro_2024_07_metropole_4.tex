
\medskip

\begin{minipage}[t]{9cm}
\vspace{0pt}
Le bateau de type 470, utilisé aux JO, comporte une grande voile triangulaire appelée grand-voile.

Cette grande voile est renforcée par 2 lattes parallèles.

Pour pouvoir participer aux JO, un bateau 470 doit respecter les caractéristiques suivantes :

\begin{center}
\begin{tabular}{|c|c|}
\hline
\textbf{Aire maximale de la grand-voile} & $8,75$ m$^2$\rule{0pt}{13pt}\\
\hline
\textbf{Longueur minimale de la grande latte} & $1,7$ m\\
\hline
\end{tabular}
\end{center}

Sur un des bateaux en compétition, on étudie ces caractéristiques pour voir s'il peut participer aux JO.
\end{minipage}
\hfill
\begin{minipage}[t]{6cm}
\vspace{0pt}
\includegraphics[width=5cm]{voilier_470}
\end{minipage}

Sa grand-voile a la forme d'un triangle ABC comme dans le dessin ci-dessous.

La petite latte est représentée par le segment [DE] et la grande latte par le segment [FG].
Les dimensions sont indiquées en mètres.

\begin{minipage}[t]{6.2cm}
\vspace{0pt}
\psset{unit=1.2cm,radius=2pt,arrowsize=3pt 3}
\def\xmin{-4}   \def\xmax{1} \def\ymin{-1}   \def\ymax {7}
\begin{pspicture}(\xmin,\ymin)(\xmax,\ymax)
\Cnode*(0,0){B} \Cnode*(0,6.2){C} \Cnode*(-2.8,0){A} 
\Cnode*(-0.99,4.01){D} \Cnode*(0,3.53){E} \Cnode*(-1.61,2.65){F} \Cnode*(0,1.86){G} 
\pspolygon[fillstyle=solid,fillcolor=lightgray!30](A)(B)(C)
\psline(D)(E) \psline(F)(G) 
\uput[dl](A){A} \uput[dr](B){B} \uput[u](C){C} \uput[ul](D){D} 
\uput[r](E){E} \uput[l](F){F} \uput[r](G){G} 
\psframe(0,0)(-0.3,0.3)
{\footnotesize
\uput[d](-1.4,0){AB = $2,8$} \uput[l](-2.2,1.3){AF = $2,9$} 
\uput[l](-1.3,3.3){FD = $1,5$} \uput[l](-0.5,5.1){DC = $2,4$} 
\uput*[d]{-26.1}(-0.75,3.7){\scalebox{0.8}{DE = $1,1$}} 
}
%\psgrid[subgriddiv=5, gridlabels=0, gridcolor=gray, subgridcolor=lightgray]
\end{pspicture}
\end{minipage}
\hfill
\begin{minipage}[t]{8cm}
\vspace{0pt}
\begin{enumerate}
\item  Calculer la longueur du côté AC.\\

\item Montrer que la longueur BC, arrondie au dixième, est $6,2$ m.\\

\item Calculer l'aire en mètre carré (m$^2$) de la voile ABC.
Arrondir au dixième.\\

\item En déduire si l'aire de la voile respecte la caractéristique permettant de participer aux JO.\\

\item Les droites (DE) et (FG) sont parallèles.\\
Montrer que la longueur de la grande latte [FG] arrondie au dixième est 1,8 m.\\

\item En déduire si ce bateau peut participer aux JO. Justifier.
\end{enumerate}
\end{minipage}


