
\vspace{0.25cm}

Pour chaque question, quatre réponses sont proposées mais \textbf{une seule est exacte}.\\
Cocher la bonne réponse \textbf{sans justification}.\\
Une réponse juste rapporte 4 points, une réponse fausse ou l'absence de réponses ne rapporte aucun point.

\medskip

\begin{enumerate}
\item Une urne contient:

\begin {center}
\begin{tabularx}{0.5\linewidth}{|*{2}{>{\centering \arraybackslash}X|}}\hline
8 boules rouges &8 boules bleues\\ \hline
8 boules vertes &8 boules jaunes\\ \hline
\end{tabularx}
\end{center}

La probabilité de tirer une boule jaune est égale à :

\begin {center}
\begin{tabularx}{0.7\linewidth}{*{4}{>{\centering \arraybackslash}X}}
$\square~$ $\dfrac14$&$\square~$ $\dfrac18$&$\square~$ $\dfrac{1}{16}$&$\square~$ $\dfrac{1}{32}$
\end{tabularx}
\end{center}

\item On relève le prix d'une même paire de baskets dans différents magasins:

\begin {center}
\begin{tabularx}{\linewidth}{|c|*{5}{>{\centering \arraybackslash}X|}}\cline{2-6}
\multicolumn{1}{c|}{~}	&Magasin 1	& Magasin 2 &Magasin 3 	&Magasin 4 	&Magasin 5\\ \hline
Prix (en euros) 		&40 		&45			&25			& 30 		&60\\ \hline
\end{tabularx}
\end{center}

Le prix moyen de cette paire de baskets est:

\begin{center}
\begin{tabularx}{0.7\linewidth}{*{4}{>{\centering \arraybackslash}X}}
$\square~$ 20 \euro & $\square~$ 26 \euro&$\square~$ 40 \euro &$\square~$ 48 \euro
\end{tabularx}
\end{center}
\end{enumerate}

\begin{minipage}{0.48\linewidth}
\begin{enumerate}[resume]
\item Les coordonnées des points A et B sont:

$\square~$A (3~;~2) et B $(- 3~;~1)$

$\square~$A (2~;~3) et B $(- 3~;~1)$

$\square~$A (3~;~2) et B $(1~;~- 3)$

$\square~$A (2~;~3) et B $(1~;~- 3)$
\end{enumerate}
\end{minipage}\hfill
\begin{minipage}{0.5\linewidth}
\psset{unit=0.75cm,arrowsize=2pt 3}
\begin{pspicture*}(-3.5,-0.7)(3.5,4)
\psgrid[gridlabels=0pt,subgriddiv=1,gridwidth=0.155pt]
\psaxes[linewidth=1.25pt,labelFontSize=\scriptstyle]{->}(0,0)(-3.5,-0.4)(3.5,4)
\psdots[dotstyle=x,dotscale=1.5](-3,1)(2,3)
\uput[l](-3,1){B}\uput[r](2,3){A}
\end{pspicture*}
\end{minipage}

\begin{enumerate}[resume]
\item La solution de l'équation $4x - 3 = x - 2$ est 

\begin{tabularx}{\linewidth}{*{4}{>{\centering \arraybackslash}X}}
$\square~$ $x = -1$ & $\square~$ $x = 0$ & $\square~$ $x = 1$& $\square~$ $x = \frac13$\\ 
\end{tabularx}
\end{enumerate}
\begin{minipage}{0.69\linewidth}
\begin{enumerate}[resume]
\item La piscine représentée ci-contre est un pavé droit qui a pour longueur 10 m, largeur 3 m et profondeur 2 m.

Si on multiplie chacune de ses dimensions par deux, alors son volume est multiplié par:

\begin {center}
\begin{tabularx}{0.6\linewidth}{*{4}{>{\centering \arraybackslash}X}}
$\square~$ 2&$\square~$ 4&$\square~$ 6&$\square~$8\\
\end{tabularx}
\end{center}
\end{enumerate}
\end{minipage}
\begin{minipage}{0.3\linewidth}
\psset{unit=1cm,arrowsize=2pt 3}
\begin{pspicture*}(3.6,2.3)
%\psgrid
\pspolygon(0,1)(1.4,0)(3.6,1.7)(2.7,2.3)
\pspolygon[fillstyle=solid,fillcolor=lightgray](0.3,1)(2.7,2.15)(2.7,1.6)(0.8,0.66)
\pspolygon[fillstyle=solid,fillcolor=lightgray](2.7,2.15)(2.7,1.6)(2.94,1.4)(3.35,1.75)
\pspolygon[fillstyle=solid,fillcolor=lightgray](0.8,0.66)(1.45,0.28)(2.97,1.44)(2.7,1.6)
\end{pspicture*}
\end{minipage}

\vspace{0.5cm}

