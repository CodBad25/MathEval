
\smallskip

Pour sa préparation physique sur une plage, une athlète effectue, en courant, un circuit dont le plan est représenté par la figure ci-dessous.

\begin{center}
\begin{tabular}{|m{7.6cm}|m{5.7cm}|}
\hline
\multicolumn{1}{|c|}{Figure}	& \multicolumn{1}{c|}{Informations et données}\\ 
\hline
\psset{unit=1cm,arrowsize=2pt 3}
\begin{pspicture}(0,-0.5)(7.5,4)
\psline[ArrowInside=->,linewidth=1.25pt](0.3,0.2)(2.5,0.2)(2.5,3)(6.9,3)(0.3,0.2)%DEBCD
\uput[ul](2.5,1.1){A} \uput[ul](2.5,3){B} \uput[r](6.9,3){C} \uput[dl](0.3,0.2){D} \uput[dr](2.5,0.2){E} 
\end{pspicture}&Le départ se fait en E.

\begin{itemize}[label=$\bullet~$]
\item Les droites (DC) et (BE)\newline se coupent en A.
\item Les droites (BC) et (DE) sont\newline parallèles.
\item ABC est un triangle rectangle\newline en B.
\end{itemize}

AE $= 6$ m ; AB $= 10$ m ; BC $= 24$ m et\newline AD $=15,6$ m.\\ \hline
\end{tabular}
\end{center}

\begin{enumerate}
\item À l'aide du théorème de Pythagore appliqué au triangle rectangle ABC, calculer AC.
\textbf{Exprimer} le résultat en m.
\item Voici deux propositions de méthodes permettant de calculer DE. Une seule méthode est correcte.

\textbf{Indiquer} sur la copie le numéro de cette méthode. Justifier la réponse.

\begin{center}
\begin{tabularx}{\linewidth}{|X|X|}\hline
\multicolumn{1}{|c|}{\textbf{Méthode \no 1}} &\multicolumn{1}{c|}{\textbf{Méthode \no 2}} \\
\begin{itemize}[label=$\bullet~$]
\item Les droites (DC) et (BE) se coupent en A. 
\item  (BC) // (DE)
\end{itemize}

On a $\dfrac{\text{AE}}{\text{AB}} = \dfrac{\text{DE}}{\text{BC}}$\rule[-10pt]{0pt}{0pt},

donc $\dfrac{6}{10} = \dfrac{\text{DE}}{24}$\rule[-10pt]{0pt}{0pt}

et DE $ = \dfrac{6 \times 24}{10} = 14,4$~m\rule[-10pt]{0pt}{0pt}
&
\vspace{0.4cm}
ADE est un triangle rectangle en E.

D'après le théorème de Pythagore,

on a : DE$^2 = \text{AD}^2 + \text{AE}^2$ 

DE$^2 = 15,6^2 + 6^2$

DE$^2 = 243,36 + 36$

DE$^2 = 279,36$

Donc DE $=\displaystyle \sqrt{279,36} \approx 16,7$~m\\ \hline
\end{tabularx}
\end{center}

\item \textbf{Calculer} la longueur du parcours EBCDE. \textbf{Exprimer} le résultat en m.
\end{enumerate}

\bigskip

