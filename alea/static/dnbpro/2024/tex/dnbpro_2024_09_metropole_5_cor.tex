
\medskip

Le programme suivant permet de calculer l'aire d'un trapèze.

\begin{minipage}{0.6\linewidth}
\begin{scratch}[scale=0.85]
\blockinit{quand \greenflag est cliqué}
\blocksensing{demander \ovalnum{Quelle est la longueur B de la grande base du trapèze ?}}
\blockvariable{mettre \ovalvariable{B} à \ovalvariable{réponse}}
\blocksensing{demander \ovalnum{Quelle est la longueur b de la grande base du trapèze ?}}
\blockvariable{mettre \ovalvariable{b} à \ovalvariable{réponse}}
\blocksensing{demander \ovalnum{Quelle est la longueur h du trapèze ?}}
\blockvariable{mettre \ovalvariable{h} à \ovalvariable{réponse}}
\blocklook{dire \ovalvariable{ regrouper \ovalnum{ l'aire du trapèze est égale à } et \ovalcontrol{\ovalcontrol{\ovalvariable{B} + \ovalvariable{b}} * \ovalvariable{h}} / \ovalnum{2}}}
\end{scratch}
\end{minipage}\hfill
\begin{minipage}{0.37\linewidth}
\begin{center}
\begin{pspicture}(5,4.4)
\psset{arrowsize=2pt 3}
\pspolygon(0,0.6)(4.6,0.6)(3.4,3.6)(1.2,3.6)
\psline{<->}(1.2,3.6)(1.2,0.6) \uput[r](1.2,2.1){Hauteur $h$}
\psline{<->}(3.4,3.8)(1.2,3.8) \uput[u](2.3,3.8){Petite base $b$}
\psline{<->}(0,0.4)(4.6,0.4) \uput[d](2.3,0.4){Grande base $B$}
\end{pspicture}
\end{center}
\end{minipage}

\begin{enumerate}
\item En s'aidant de la dernière instruction du programme,  la formule de l'aire d'un trapèze est: $\dfrac{(B + b) \times h}{2}$.

\item Si $B = 12$, $b = 8$ et $h = 6$, le résultat affiché par le programme est : 
$\dfrac{(12+8)\times 6}{2}=60$.

\item  On complète le programme de calcul d'aire d'un rectangle avec les lignes numérotées dans le script ci-dessous.

\medskip

\begin{scratch}
\blockinit{quand \greenflag est cliqué}
\blocksensing{ demander \ovalnum{Quelle est la longueur  du rectangle ?} et attendre }
 \setscratch{num blocks}
\renewcommand*\numblock[1]{\No 5~~} 
\blockvariable{mettre \ovalvariable{longueur} à \ovalvariable{réponse}}
\renewcommand*\numblock[1]{\No 4~~}
\blocksensing{demander \ovalnum{Quelle est la largeur du rectangle ?} et attendre}
\renewcommand*\numblock[1]{\No 6~~}
\blockvariable{mettre \ovalvariable{largeur} à \ovalvariable{réponse}}
\renewcommand*\numblock[1]{\No 3~~}
\blocklook{Dire \ovaloperator{regrouper \ovalnum{L'aire du rectangle est égale à } et \ovaloperator{\ovalcontrol{longueur} * \ovalcontrol{largeur}}}}
\end{scratch}
\end{enumerate}

