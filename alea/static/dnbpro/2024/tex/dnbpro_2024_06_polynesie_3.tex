
\medskip

Sur un arbre, la plateforme de départ d'une tyrolienne (point K) est située à 8 m de
hauteur. La plateforme d'arrivée de la tyrolienne (point L) est située à un niveau plus
bas, accrochée à un second arbre.

Les deux arbres supportant les plateformes sont perpendiculaires au sol et situés à
40 m l'un de l'autre.

Une tyrolienne va de la plateforme K et à la plateforme L sur une distance totale de
$40,3$ m.

\begin{center}
\psset{xunit=0.8cm,yunit=0.6cm,radius=2pt,arrowsize=3pt 3}
\def\xmin{-2}   \def\xmax{15} \def\ymin{-2}   \def\ymax {8}
\begin{pspicture}(\xmin,\ymin)(\xmax,\ymax)
%\psgrid[subgriddiv=1, gridlabels=0, gridcolor=gray, subgridcolor=lightgray]
\psframe[fillstyle=solid,fillcolor=lightgray!20](0,0)(-1,8) 
\psframe[fillstyle=solid,fillcolor=lightgray!20](14,0)(13,8) 
\psline[linewidth=1.2pt](-3,0)(16,0) 
\psline[linewidth=1.2pt](0,7)(13,5)% \uput[u](6.5,6){$\cdots$ m}
\psline{<->}(0,1)(13,1) \uput[u](6.5,1){40 m} \uput[d](6.5,0){SOL}
\psline[linestyle=dashed](0,5)(14,5)
\uput[d](13,0){S} \uput[ur](0,7){K} \uput[dr](0,5){O} \uput[ul](13,5){L}
\uput[d](11,-1){\emph{\textbf{Le schéma n'est pas à l'échelle}}}
\uput[r](1,7.5){Départ} \uput[l](13,4){Arrivée}
\psframe(0,0)(0.3,0.4) \psframe(13,0)(12.7,0.4) \psframe(0,5)(0.3,5.4) 
%\psline{<->}(-1.5,0)(-1.5,7) \uput[l](-1.5,3.5){$\cdots$ m}
\end{pspicture}
\end{center}

\begin{enumerate}
\item Compléter le schéma ci-dessous  avec les données ci-dessus.

\begin{center}
\psset{xunit=0.8cm,yunit=0.6cm,radius=2pt,arrowsize=3pt 3}
\def\xmin{-3}   \def\xmax{16} \def\ymin{-2}   \def\ymax {9}
\begin{pspicture}(\xmin,\ymin)(\xmax,\ymax)
%\psgrid[subgriddiv=1, gridlabels=0, gridcolor=gray, subgridcolor=lightgray]
\psframe[fillstyle=solid,fillcolor=lightgray!20](0,0)(-1,8) 
\psframe[fillstyle=solid,fillcolor=lightgray!20](14,0)(13,8) 
\psline[linewidth=1.2pt](-3,0)(16,0) 
\psline[linewidth=1.2pt](0,7)(13,5) \uput[u](6.5,6){$\cdots$ m}
\psline{<->}(0,1)(13,1) \uput[u](6.5,1){40 m} \uput[d](6.5,0){SOL}
\psline[linestyle=dashed](0,5)(14,5)
\uput[d](13,0){S} \uput[ur](0,7){K} \uput[dr](0,5){O} \uput[ul](13,5){L}
\uput[d](11,-1){\emph{\textbf{Le schéma n'est pas à l'échelle}}}
\uput[r](1,7.5){Départ} \uput[l](13,4){Arrivée}
\psframe(0,0)(0.3,0.4) \psframe(13,0)(12.7,0.4)  \psframe(0,5)(0.3,5.4)
\psline{<->}(-1.5,0)(-1.5,7) \uput[l](-1.5,3.5){$\cdots$ m}
\end{pspicture}
\end{center}

\item Calculer la hauteur KO dans le triangle KOL rectangle en O.\\
Arrondir le résultat au dixième.
\end{enumerate}

On admet que KO = $4,9$ m.

\begin{enumerate}[resume]
\item Calculer à quelle hauteur LS se trouve la plateforme d'arrivée.
\end{enumerate}

Pour des raisons de sécurité, la pente d'une tyrolienne ne doit pas dépasser 8\,\%. Le calcul de cette pente pour la tyrolienne représentée sur le schéma ci-dessus se fait à l'aide de la formule:

\[\dfrac{\text{KO}}{\text{OL}}\times 100\]

\hfill{} avec $\text{KO}=4,9$ m et $\text{OL}=40$ m\hfill~

\begin{enumerate}[resume]
\item Calculer la pente de la tyrolienne.
\item Indiquer si les normes de sécurité sont respectées pour l'utilisation de la tyrolienne.\\
 Justifier la réponse.
\end{enumerate}

\bigskip

