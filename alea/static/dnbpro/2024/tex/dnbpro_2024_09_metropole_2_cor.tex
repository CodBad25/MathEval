
\medskip

Gabin souhaite réduire son impact sur l'environnement. Il a réalisé auprès d'un organisme spécialisé une estimation de la quantité de dioxyde de carbone (CO$_2$) qu'il émet en une année.

Les résultats sont donnés dans le tableau ci-dessous.

\begin{center}
\begin{tabularx}{0.8\linewidth}{|X|c|} \hline
\textbf{Domaine}&\multicolumn{1}{|m{5cm}|}{\textbf{Masse de dioxyde de carbone}\newline\hspace*{2cm} \textbf{(en tonne)}}\\ \hline
Transport &3,6\\ \hline
Logement& 2,2 \\ \hline
Vie quotidienne &1,4\\ \hline
Alimentation&\ldots\\ \hline
Émissions indirectes &2,9\\ \hline
Total&12,1\\ \hline
\end{tabularx}
\end{center}

\medskip

\begin{enumerate}
\item La masse totale de dioxyde de carbone (CO$_2$) émis par Gabin en une année est de $12,1$~tonnes.

\item $12,1- \left ( 3,6 + 2,2 +1,4 +  2,9\strut \right ) = 12,1 - 10,1 = 2 $

Donc la masse de CO$_2$ du domaine Alimentation est de 2 tonnes.

\item $\dfrac{2}{12,1}\times 100 \approx 16,53$

Donc le pourcentage de CO$_2$, arrondi au dixième, correspondant à l'alimentation par rapport au total des émissions est de $16,5$\,\%.
\end{enumerate}

L'objectif de Gabin est d'émettre moins de 10 tonnes de CO$_2$ par an. Pour atteindre cet objectif, il effectue des travaux d'isolation et change son mode de chauffage. Ses émissions de CO$_2$ dues au logement diminuent de 50\,\%.

\begin{enumerate}[resume]
\item Les émissions de CO$_2$ dues au logement avant travaux sont de $2,2$ tonnes. Si elles diminuent de 50\,\%, elles sont divisées par 2.

Donc la masse de CO$_2$ émis par an pour le domaine Logement après les travaux réalisés et le changement de mode de chauffage est de $1,1$~tonne.

\item La nouvelle masse totale de CO$_2$ émis par Gabin en une année a baissé de $1,1$ tonne donc s'établit à 11~tonnes.

\item $11>10$ donc  l'objectif de Gabin n'est pas atteint.
\end{enumerate}

\vspace{0.5cm}

