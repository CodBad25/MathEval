
\smallskip

Mathis pratique le surf à Teahupo'o. Il pèse $70$~kg et est de niveau intermédiaire.

Son surfboard a les dimensions suivantes :

\begin{itemize}[label=$\bullet~$]
\item Longueur : 5 pieds et 11 pouces
\item Largeur : 19 pouces
\item Épaisseur : 2,6 pouces $ = 6,5$ cm
\end{itemize}

\medskip

À l'aide des données et de l'exemple de calcul suivants:

\begin{center}
\begin{tabularx}{0.9\linewidth}{|X|m{1cm}|X|}
\cline{1-1}\cline{3-3}
&&\\
\begin{list}{\textbullet}{Données:}
\item 1 pied $= 30,5$ cm
\item 1 pouce $= 2,5$ cm
\end{list}
&&
\begin{list}{\quad}{Exemple de calcul:}
\item 3 pieds et 5 pouces
\item $3 \times 30,5 + 5 \times 2,5$
\item $= 104$ cm
\end{list}\\
\cline{1-1}\cline{3-3}
\end{tabularx}
\end{center}

\begin{enumerate}
\item \textbf{Calculer} la largeur du surfboard. \textbf{Détailler} le calcul. \textbf{Exprimer} le résultat en cm.
\item \textbf{Calculer} la longueur du surfboard. \textbf{Détailler} les calculs. \textbf{Exprimer} le résultat en cm.

Le volume du surfboard de Mathis, en cm$^3$, en fonction de l'épaisseur, en cm, est donné par: 
\begin{center}Volume $= \np{4523} \times$  épaisseur.\end{center}

\item \textbf{Calculer} le volume de ce surfboard. \textbf{Détailler} le calcul. \textbf{Exprimer} le résultat en cm$^3$.
\item \textbf{Exprimer} ce volume en litres, arrondi au dixième (donnée : 1L $= \np{1000}~$cm$^3$).

Le volume d'un surfboard détermine la flottabilité. Le choix d'une planche adaptée est essentiel pour obtenir de bons résultats. Le choix du volume d'un surfboard est lié à la masse et au niveau du surfeur.

\textbf{Données :} Volume (en L) d'un surfboard, suivant la masse du surfeur et de son niveau (Confirmé, Intermédiaire et Débutant).

\begin{center}
\begin{tabularx}{\linewidth}{|*{4}{>{\centering \arraybackslash}X|}}\hline
Masse en kg&\textbf{Confirmé} &\textbf{Intermédiaire} &\textbf{Débutant}\\
&Volume en L&Volume en L&Volume en L\\ \hline
55&19,80&23,65&37,95\\ \hline
60& 21,00&25,20&40,80\\ \hline
65&22,75&27,30&44,20\\ \hline
70&24,50&29,40&47,60\\ \hline
75&26,25&31,50&51,00\\ \hline
80&28,00&33,60&54,40\\ \hline
85&29,75&35,70&57,80\\ \hline
\end{tabularx}
\end{center}

Le volume du surfboard de Mathis est de $29,40$ L.
\item  Indiquer si ce surfboard est adapté pour Mathis. \textbf{Justifier} la réponse.

\end{enumerate}

\bigskip

