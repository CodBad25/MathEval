
\medskip

Pour chaque question, quatre réponses sont proposées mais une \textbf{seule est exacte}.\\
Cocher la bonne réponse \textbf{sans justification}.\\
Une réponse juste rapporte 4 points, une réponse fausse ou l'absence ne rapporte aucun point.

\begin{enumerate}
\item Un million peut s'écrire:

\medskip

\begin{tabularx}{\linewidth}{XXXX}
\scalebox{1.2}{$\square$~} $10^{3}$ & \scalebox{1.2}{$\square$~}  $10^{4}$& \scalebox{1.2}{$\square$~}  $10^{6}$ & \scalebox{1.2}{$\square$~}  $10^{9}$
\end{tabularx}

\bigskip

\item Sur un plan de maison à l'échelle 1/100, si une chambre mesure $3,4$ cm de largeur sur le plan, sa largeur réelle est de:

\medskip

\begin{tabularx}{\linewidth}{XXXX}
\scalebox{1.2}{$\square~$} $3,4$ cm & \scalebox{1.2}{$\square$~} $34$ m & \scalebox{1.2}{$\square$~} $3,4$ m & \scalebox{1.2}{$\square$~} $34$ cm
\end{tabularx}

\bigskip

\item Si on lance un dé équilibré à 6 faces, la probabilité d'obtenir un 6 est de:

\medskip

\begin{tabularx}{\linewidth}{XXXX}
\scalebox{1.2}{$\square$~} $\dfrac{1}{2}$ & \scalebox{1.2}{$\square$~} $\dfrac{1}{3}$ & \scalebox{1.2}{$\square$~} $\dfrac{1}{4}$ & \scalebox{1.2}{$\square$~} $\dfrac{1}{6}$ 
\end{tabularx}

\bigskip

\item Une barre énergétique de masse totale 80 g contient 70\,\% de sucre, la masse de
sucre dans cette barre est de:

\medskip

\begin{tabularx}{\linewidth}{XXXX}
\scalebox{1.2}{$\square$~} 48 g & \scalebox{1.2}{$\square$~} 72 g & \scalebox{1.2}{$\square$~} 15 g & \scalebox{1.2}{$\square$~} 56 g
\end{tabularx}

\bigskip

\item Si on multiple par 2 les dimensions du cube ci-dessous, son volume sera de:

\begin{center}
\psset{unit=0.3cm,radius=0pt,arrowsize=3pt 2}
\def\xmin {-1}   \def\xmax {11} \def\ymin {-1}   \def\ymax {9}
\begin{pspicture}(\xmin,\ymin)(\xmax,\ymax)
%\psgrid[subgriddiv=1, gridlabels=0, gridcolor=lightgray] 
\Cnode*(0,0){A}  \Cnode*(2,2){D}\Cnode*(6,0){B} \Cnode*(8,2){C} 
\Cnode*(6,6){F}    \Cnode*(8,8){G} \Cnode*(0,6){E} \Cnode*(2,8){H} 
\psline(G)(C)(B)(A)(E)(F)(G)(H)(E) \psline(B)(F)
\psset{linestyle=dashed}
\psline(A)(D)(C) \psline(D)(H)
{\footnotesize
\uput[d](3,0){1 cm} \uput[r](7,1){1 cm} \uput[r](8,5){1 cm}}
\end{pspicture}
\end{center}

\medskip

\begin{tabularx}{\linewidth}{XXXX}
\scalebox{1.2}{$\square$~} 3 cm$^3$ & \scalebox{1.2}{$\square$~} 6 cm$^3$ & \scalebox{1.2}{$\square$~} 8 cm$^3$ & \scalebox{1.2}{$\square$~} 12 cm$^3$ 
\end{tabularx}

\end{enumerate}

\bigskip

