
\vspace{0.25cm}

%\emph{Les calculs seront détaillés sur la copie.}
%
%\smallskip

Pour limiter les bouteilles en plastique, une association souhaite offrir une gourde en métal à chaque élève de sixième.\\
Les prix que propose un commerçant sont dans le tableau page suivante.

\begin{figure}[t!]
\centering
Facture\rule[-15pt]{0pt}{0pt}

\renewcommand{\arraystretch}{1.4}
\begin{tabular}{|>{\cellcolor{lightgray}} c |c|c|>{\raggedleft\arraybackslash} m{3.5cm}|>{\raggedleft\arraybackslash} m{3cm}|}
\hline
\rowcolor{lightgray} & A & B & \centering C & \centering\arraybackslash  D\\
\hline
1 & Articles & Quantités & \centering Prix unitaire en F (prix d'une gourde) & \centering\arraybackslash Montant total \newline{} (en F)\\
\hline
2 & Gourdes vertes & 45 & \np{1500,00} & \np{67500,00}\\
\hline
3 & Gourdes bleues & 29 & \np{1200,00} & \np{34800,00}\\
\hline
4 & Gourdes rouges & 36 & \np{1300,00} & $\blue\np{46800,00}$\\
\hline
5 & Gourdes grises &$\blue 70$ & \np{1125,00} & $\blue\np{78750,00}$\\
\hline
6 & \multicolumn{2}{c|}{~} & \centering Total HT & \np{227850,00}\\
\cline{1-1} \cline{4-5} 
7 & \multicolumn{2}{|c|}{~} & \centering TVA 13\,\% & $\blue\np{29620,50}$\\
\cline{1-1} \cline{4-5} 
8 & \multicolumn{2}{|c|}{~} & \centering Total TTC & $\blue\np{257470,50}$\\
\cline{1-1} \cline{4-5} 
\end{tabular}
\end{figure}

\begin{flushleft}
\textbf{Partie A}
\end{flushleft}

\begin{enumerate}
\item % \textbf{Justifier} par un calcul le montant total pour les gourdes vertes.
Il y a 45 gourdes vertes à \np[F]{1500,00} pièce. \\
$45\times \np{1500,00}=\np{67500,00}$

Le montant total pour les gourdes vertes est donc bien de \np[F]{67500,00}.

\item 	
\begin{list}{\textbullet}{On détaille les calculs permettant de compléter le tableau.}
\item Montant total (en F) des gourdes rouges:
$36 \times \np{1300,00} = \np{46800,00}$.
\item Montant total (en F) des gourdes grises:\\
$\np{227850,00}  - ( \np{67500,00} + \np{34800,00} +  \np{46800,00}) = \np{78750,00}$.
\item Nombre de gourdes grises:
$\dfrac{\np{78750,00}}{\np{1125,00}}=70$.
\item Montant (en F) de la TVA:
$\np{227850,00} \times  \dfrac{13}{100} = \np{29620,50}$.
\item Montant total TTC (en F):
$\np{227850,00} + \np{29620,50} = \np{257470,50}$.
\end{list}

\item La formule que l'on doit mettre dans la cellule D8 est: \quad
\fbox{\texttt{= D6 + D7}}

\end{enumerate}

\begin{flushleft}
\textbf{Partie B}
\end{flushleft}

Les gourdes sont toutes distribuées aux élèves. Parmi les 180 élèves de sixième, on choisit un élève au hasard.

\begin{enumerate}[resume]
\item  Il y a 36 gourdes rouges et 180 gourdes au total.\\
La probabilité que l'élève ait une gourde rouge est $\dfrac{36}{180}$ soit $0,2$.

\item La probabilité qu'il ait une gourde d'une autre couleur est donc $1-0,2=0,8$.
\end{enumerate}

