
\vspace{0.25cm}

L'association souhaite installer un poulailler.

\medskip

%\includegraphics[width=\linewidth]{MetroPro23_3}

\fbox{\parbox{\linewidth}{
\begin{flushleft}
Dimensions du terrain: longueur = 7 mètres, largeur = 4 mètres
\end{flushleft}

\begin{center}
\psset{unit=0.9cm,radius=0pt,arrowsize=3pt 2,linewidth=1.2pt}
\def\xmin{0}   \def\xmax{4} \def\ymin{0}   \def\ymax{6}
\begin{pspicture}(\xmin,\ymin)(\xmax,\ymax)
%\psgrid[subgriddiv=5,gridlabels=0, gridcolor=lightgray, subgridcolor=lightgray!40] 
\Cnode*(0,2.2){A} \Cnode*(1,2){B }\Cnode*(4,3){C} \Cnode*(3,3.2){D}
\Cnode*(0.4,3.6){E} \Cnode*(3.4,4.6){F}
\psline(A)(B)(C)(D)(A)(E)(B)
\psline(C)(F)(D) \psline(E)(F)
\rput{18.43}(2.7,2.3){\scriptsize Longueur : 2,50 mètres}
\rput{-11.4}(0.5,1.8){\scriptsize Largeur}
\uput[d](2,6){\bf Figure 1} 
\uput[d](2,5.4){\footnotesize Squelette minimal du poulailler}
\end{pspicture}
\def\xmin{0.3}   \def\xmax{6.8} \def\ymin{0}   \def\ymax{6}
\begin{pspicture}(\xmin,\ymin)(\xmax,\ymax)
%\psgrid[subgriddiv=5,gridlabels=0, gridcolor=lightgray, subgridcolor=lightgray!40] 
\psline(0,1.6)(1,1.4)(0.4,3)(0,1.6)
\psline(6.6,3.3)(5.6,3.5)(6,4.9)(6.6,3.3)
\psline(1.4,1)(2.4,0.8)(5.4,1.8)(4.4,2)(1.4,1)
\psline(2.4,1.8)(5.4,2.8)(4.8,4.4)(1.8,3.4)(2.4,1.8)
\psline(1,2.4)(1.4,3.8)(4.4,4.8)(4,3.4)(1,2.4)
\uput[d](3,6){\bf Figure 2} 
\uput[d](3,5.4){\footnotesize Vue éclatée du poulailler}
\psline{->}(0,1.2)(0.4,1.8) \uput[d](0,1.2){\footnotesize Cadre latéral}
\psline{->}(5,0.6)(3.9,1.3) \uput[d](5,0.6){\footnotesize Cadre posé au sol}
\end{pspicture}
\def\xmin{0.3}   \def\xmax{3} \def\ymin{0}   \def\ymax{6}
\begin{pspicture}(\xmin,\ymin)(\xmax,\ymax)
%\psgrid[subgriddiv=5,gridlabels=0, gridcolor=lightgray, subgridcolor=lightgray!40] 
\Cnode*(0,2){T} \Cnode*(3,2){L }\Cnode*(1.5,5){P}
\Cnode*(1.5,2){R}
\uput[d](T){T} \uput[d](L){L} \uput[u](P){P} \uput[d](R){R} 
\rput(0.75,2){$\boldsymbol\slash\slash$} \rput(2.25,2){$\boldsymbol\slash\slash$}
\psline(T)(L)(P)(T) \psline[linestyle = dashed](P)(R)
\psframe(R)(1.8,2.3)
\uput[d](1.5,1.5){\bf Figure 3} 
\uput[d](1.5,0.9){\footnotesize Dimension du cadre}
\uput[d](1.5,0.5){\footnotesize latéral du poulailler}
\uput[l]{90}(1.5,3){\footnotesize $1,6$ m}
\uput[ul]{63.4}(0.75,3.5){\footnotesize $1,75$ m}
\uput[ur]{-63.4}(2.25,3.5){\footnotesize $1,75$ m}
\end{pspicture}
\end{center}}}


%\medskip

\begin{enumerate}
\item  Les figures planes qui composent la vue éclatée du poulailler de la figure 2 sont trois rectangles et deux triangles isocèles.

\item La figure 3 ci-dessus représente le cadre latéral du poulailler.

\begin{enumerate}
\item  On utilise la relation de Pythagore dans le triangle PRL, rectangle en R:
%montrer que la longueur RL arrondie au centième vaut $0,71$ m.

$\text{PL}^2=\text{PR}^2+\text{RL}^2$
donc 
$\text{RL}^2 = \text{PL}^2-\text{PR}^2 = 1,75^2-1,6^2= \np{0,5025}$

$\ds\sqrt{\np{0,5025}}\approx \np{0,7089}$
donc  la longueur RL arrondie au centième vaut $0,71$ m.

\item La largeur TL du cadre du poulailler posé au sol est, en m:
$\text{TL} = 2\times \text{RL} \approx 1,42$.

\item $2,50\times 1,42 = 3,55$ donc  l'aire de la surface du sol délimitée par le cadre du poulailler est d'environ $3,55$~m$^2$.
\end{enumerate}

\item L'association achète un modèle dont les dimensions au sol sont :

~\hfill Longueur = $2,50$ m\qquad\qquad Largeur = $1,42$ m\hfill~

Un membre de l'association affirme qu'il est possible de placer six poulaillers sur le terrain. Voici un schéma prouvant qu'il a raison.

%\smallskip
%
%\textbf{Indication}: on pourra utiliser la figure d'aide à la résolution ci-dessous sachant que chaque poulailler peut être disposé dans le sens de la longueur ou de la largeur.


\begin{center}
\psset{unit=1cm,arrowsize=3pt 2}
\def\xmin{-1}   \def\xmax{8} \def\ymin{-1}   \def\ymax{5}
\begin{pspicture}(\xmin,\ymin)(\xmax,\ymax)
\psframe[fillstyle=solid,fillcolor=lightgray,linecolor=gray!30](0,0)(7,4)
\psline{<->}(0,-0.3)(7,-0.3) \uput[d](3.5,-0.3){7 m}
\psline{<->}(7.3,0)(7.3,4) \uput[r]{-90}(7.4,2){4 m}
\psframe(0,4)(2.5,2.58)
\psline{<->}(0,4.3)(2.5,4.3) \uput[u](1.25,4.3){$2,50$ m}
\psline{<->}(-0.3,4)(-0.3,2.58) \uput[l]{90}(-0.3,3.29){$1,42$ m}
{\footnotesize\uput[u](1.25,3.2){Cadre posé} \uput[d](1.25,3.35){au sol} }
%%%%%%%
\psset{linecolor=blue}
\psframe(0,0)(1.42,2.5)
\psframe(1.86,0)(3.28,2.5)%% décalé de 0,44
\psframe(3.72,0)(5.14,2.5)
\psframe(5.58,0)(7,2.5)
\psframe(7,4)(4.5,2.58)
\psline{<->}(-0.3,0)(-0.3,2.5) \uput[l]{90}(-0.3,1.25){\blue $2,50$ m}
\psline{<->}(4.5,4.3)(7,4.3) \uput[u](5.75,4.3){\blue $2,50$ m}
\psline{<->}(0,-0.6)(1.42,-0.6) \uput[d](0.71,-0.6){\blue $1,42$ m}
\end{pspicture}

\end{center}

\end{enumerate}


