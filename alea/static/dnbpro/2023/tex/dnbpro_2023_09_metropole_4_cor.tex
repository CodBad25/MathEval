
\vspace{0.25cm}

Dans un jeu vidéo réalisé avec le logiciel Scratch, l'avatar d'un joueur au maillot foncé prend le ballon à la sortie d'une mêlée puis se déplace vers la zone grisée. 
La partie est gagnée lorsque l'avatar entre dans la zone grisée en évitant les autres joueurs.

%Les programmes 1, 2 et 3 ci-dessous simulent différents chemins du joueur qui se déplace.

\begin{enumerate}
\item On trace en tirets rouges sur le graphique ci-dessous le chemin parcouru par l'avatar du joueur lorsqu'on utilise le programme 3.

%La ressource d'aide ci-dessous précise les commandes  \og s'orienter \fg{}  et \og tourner \fg{}.

\smallskip

\begin{tabularx}{\linewidth}{ |*3{>{\centering\arraybackslash} X|}}
\hline
\textbf{Programme 1} & $\blue\text{\bf Programme 2}$ & $\red\text{\bf Programme 3}$\\
\hline
&&\\
\scalebox{0.8}{
\begin{scratch}
\blockinit{quand \greenflag est cliqué}
\blockmove{aller à x: \ovalnum{20} y: \ovalnum{20}}
\blockmove{s'orienter à \ovalnum{90}}
\blockmove{avancer de \ovalnum{120} pas}
\blockmove{tourner \turnleft{} de \ovalnum{90} degrés}
\blockmove{avancer de \ovalnum{40} pas}
\blockmove{tourner \turnright{} de \ovalnum{90} degrés}
\blockmove{avancer de \ovalnum{20} pas}
\blockmove{tourner \turnleft{} de \ovalnum{90} degrés}
\blockmove{avancer de \ovalnum{20} pas}
\end{scratch}}
&
\scalebox{0.8}{
\begin{scratch}
\blockinit{quand \greenflag est cliqué}
\blockmove{aller à x: \ovalnum{20} y: \ovalnum{20}}
\blockmove{s'orienter à \ovalnum{90}}
\blockmove{avancer de \ovalnum{40} pas}
\blockmove{tourner \turnleft{} de \ovalnum{90} degrés}
\blockmove{avancer de \ovalnum{80} pas}
\blockmove{tourner \turnright{} de \ovalnum{90} degrés}
\blockmove{avancer de \ovalnum{40} pas}
\blockmove{tourner \turnleft{} de \ovalnum{90} degrés}
\blockmove{avancer de \ovalnum{90} pas}
\end{scratch}}
&
\scalebox{0.8}{
\begin{scratch}
\blockinit{quand \greenflag est cliqué}
\blockmove{aller à x: \ovalnum{20} y: \ovalnum{20}}
\blockmove{s'orienter à \ovalnum{90}}
\blockmove{avancer de \ovalnum{80} pas}
\blockmove{tourner \turnleft{} de \ovalnum{90} degrés}
\blockmove{avancer de \ovalnum{80} pas}
\blockmove{tourner \turnright{} de \ovalnum{90} degrés}
\blockmove{avancer de \ovalnum{100} pas}
\blockmove{tourner \turnleft{} de \ovalnum{90} degrés}
\blockmove{avancer de \ovalnum{20} pas}
\end{scratch}}\\
&&\\
\hline
\end{tabularx}

\smallskip

%\begin{center}
%\textbf{Ressources}
%\end{center}
%
%\setlength{\columnseprule}{1pt}
%\begin{multicols}{2}
%\begin{center}
%\begin{scratch}\blockmove{s'orienter à \ovalnum{90}}\end{scratch}
%\parbox{3cm}{\psset{unit=0.7cm,arrowsize=2pt 1}
%\begin{pspicture}(-2,-2)(2,2)
%%\psgrid
%\psframe[fillstyle=solid,fillcolor=gray!70,linecolor=gray!70](-1.7,-1.7)(1.7,1.7)
%\pspolygon[fillstyle=solid,fillcolor=gray!70,linecolor=gray!70](-0.3,1.7)(0,2)(0.3,1.7)
%\pscircle[fillstyle=solid,fillcolor=gray,linecolor=gray](0,0){1.2}
%\pswedge[fillstyle=solid,fillcolor=gray!50,linecolor=gray!50](0,0){1.2}{0}{90}
%\multido{\i=0+15}{24}{\psline[linecolor=white](0.9;\i)(1.1;\i)}
%\psline[linecolor=white,linewidth=1.5pt](0,1.2)(0,0)(1.2,0)
%\psdots[dotscale=1.5,linecolor=white](0,0)
%\pscircle[fillstyle=solid,fillcolor=white,linecolor=white](1.2,0){0.3}
%\psline[linewidth=1.5pt]{->}(1,0)(1.4,0)
%\end{pspicture}}
%
%\smallskip
%
%\emph{Le joueur s'oriente pour courir\\ dans le sens de la flèche}
%\end{center}
%
%\columnbreak
%
%\begin{center}
%
%\vspace*{0.48cm}
%
%\begin{scratch}\blockmove{tourner \turnleft{} de \ovalnum{90} degrés}\end{scratch}
%
%\smallskip
%
%\emph{Le joueur tourne de 90\degres{} \\
%dans le sens de la flèche}
%\end{center}
%\end{multicols}

\item On trace en bleu sur le graphique le chemin du programme 2 permettant à l'avatar de gagner.

%\textbf{Remarque} : Les chemins des 3 programmes se superposent en début de parcours.

{\psset{unit=0.05cm}
\begin{pspicture*}(-20,-40)(230,220)
\psframe[fillstyle=solid,fillcolor=lightgray!30](0,180)(220,200)
\psgrid[unit=1cm,subgriddiv=5,  gridlabels=0, subgridcolor=lightgray!40,gridcolor=lightgray](0,0)(11,10)
\psaxes[labelFontSize=\scriptstyle,arrowsize=3pt 3, ticksize=-2pt 2pt,Dx=20,Dy=20]{->}(0,0)(-19,-10)(220,200)
\uput[d](220,0){\scriptsize $x$} \uput[l](0,200){\scriptsize $y$}
\uput[r](-10,210){\scriptsize\textbf{Nombre de pas : déplacement vertical}}
\uput[l](215,-15){\scriptsize\textbf{Nombre de pas : déplacement horizontal}}
\psset{dotstyle=+,dotscale=1,linecolor=black,linewidth=5pt}
\psdots(39,38)(108,30)(150,24)(190,12)
\psset{dotstyle=+,linecolor=gray}
\psdots(39,53)(80,60)(120,60)(160,80)(20,100)(60,120)(200,120)
\psset{dotstyle=x,linecolor=red}
\psdots(20,20)
\uput[u](160,184){\textbf{ZONE GRISÉE}}  
%%%%%%%%%%%%%%
\psline[linestyle=dashed,linewidth=2pt](20,20)(100,20)(100,100)(200,100)(200,120)
\uput[u](150,100){\red\scriptsize programme 3}
\psline[linecolor=blue,linewidth=1.2pt](20,20)(60,20)(60,100)(100,100)(100,190)
\uput[r](100,150){\blue\scriptsize programme 2}
\uput[d](20,18){\red\scriptsize départ}
\end{pspicture*}}

\end{enumerate}


