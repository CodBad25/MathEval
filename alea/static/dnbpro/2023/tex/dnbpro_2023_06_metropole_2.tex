
\vspace{0.25cm}

Les photographies ci-dessous représentent deux pots de fleurs cylindriques.

Le grand pot est un agrandissement de coefficient 3 du petit pot. Ce qui signifie que le diamètre et la hauteur du grand pot sont 3 fois plus grands que le diamètre et la hauteur du petit pot.

\begin{center}
\includegraphics[width=0.9\linewidth]{MetroPro23_1}\\
\textbf{Le schéma n'est pas à l'échelle.}
\end{center}

\textbf{Volume du petit pot}

\begin{enumerate}
\item  Calculer le rayon $R_1$ du pot 1.
\item Montrer par un calcul détaillé que le volume $V_1$ du pot 1 est égal à \np{1177,5}~cm$^3$.

Rappel: $V_{\text{Cylindre}} = \pi \times R^2 \times h$, on prendra $\pi=3,14$.
\end{enumerate}

\textbf{Volume du grand pot}

\begin{enumerate}[resume]
\item  Calculer le rayon $R_2$ du pot 2.
\item Calculer la hauteur $h_2$ du pot 2.
\item À l'aide de la formule, calculer le volume $V_2$ du pot 2.
\item Affirmation : \og Quand on réalise un agrandissement avec un coefficient multiplicateur de 3, le volume d'un cylindre est multiplié par 27. \fg

Cette affirmation est-elle exacte ? Justifier la réponse.

\end{enumerate}


