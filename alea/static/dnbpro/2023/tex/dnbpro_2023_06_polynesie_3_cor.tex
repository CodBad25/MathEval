
\medskip
 
\textbf{PARTIE A}

\medskip

Terii vend les produits de sa ferme au marché de Papeete sur Tahiti. Il a relevé et classé, par ordre croissant, les masses de gingembre (en kg) vendues au mois de mai.

Voici les relevés statistiques de 19 ventes réalisées au mois de mai :

\smallskip

\begin{tabularx}{\linewidth}{|*{19}{>{\centering\arraybackslash}X|}}
\hline
3 & 3 & 4 & 4 & 4 & 5 & 5 & 5 & 6 & 6 & 7 & 7 & 7 & 8 & 8 & 9 & 10 & 11 & 12\\
\hline
\end{tabularx}


\begin{enumerate}
\item L'étendue de cette série statistique est $12-3=9$.

\item %\textbf{Déterminer} la médiane de cette série statique.
Il y a 19 nombres rangés en ordre croissant, donc la médiane est la 10\ieme{} valeur, soit 6. 

\item La masse moyenne de ces ventes est, en kg:

$\dfrac{3 + 3 + 4 + 4 + 4 + 5 + 5 + 5 + 6 + 6 + 7 + 7 + 7 + 8 + 8 + 9 + 10 + 11 + 12}{19} = \dfrac{124}{19} \approx 6,5$.

\item Terii estime que la vente sur un mois est rentable lorsque les masses médiane et moyenne des ventes sont supérieures ou égales à 6 kg. 
%Est-ce le cas pour le mois de mai ?

La médiane est de 6 kg et $6\geqslant 6$; la moyenne est d'environ $6,5$~kg et $6,5\geqslant 6$. Donc la vente sur un mois est rentable.
% \textbf{Justifier} la réponse.
\end{enumerate}

\medskip

\textbf{PARTIE B}

\medskip

Terii vend 500 g de gingembre pour \np{1270} F, et on sait que le prix est proportionnel à la masse de gingembre.

\begin{enumerate}
\item  $\np{1000}=2\times 500$

Donc le prix pour \np{1000} g de gingembre est de $2\times \np{1270}$ soit \np[F]{2540}.

\item On complète le tableau des prix.

\begin{center}
\renewcommand{\arraystretch}{1.6}
\begin{tabularx}{0.8\linewidth}{| >{\centering\arraybackslash}m{4cm} | *5{>{\centering\arraybackslash} X |}}
\hline
Masse de gingembre\newline (en grammes) & 100 & 500 & 900 & \np{1000} & $\blue \np{3900}$\\
\hline
Prix (en F) & $\blue 254$ & \np{1270} & $\blue \np{2286}$ & $\blue \np{2540}$ & \np{9906}\\
\hline
\end{tabularx}
\end{center}

\item %\textbf{Calculer} la masse de gingembre qu'un client peut acheter pour \np{15500} F.

% \textbf{Arrondir} le résultat au gramme.
Avec \np[F]{1270}, on peut acheter 500~g de gingembre donc avec \np[F]{15500} on peut acheter $\dfrac{\np{15500}\times 500}{\np{1270}}$ soit environ \np[g]{6102}.

\end{enumerate}

\vspace{0.5cm} 
 
