
\vspace{0.25cm}

Les deux programmes ci-dessous sont réalisés à l'aide du logiciel Scratch.

\medskip

\textbf{Programme A~~~~}
\begin{scratch}%[num blocks,baseline=10]
\blockinit{quand \greenflag est cliqué}
\blockmove{demander \ovalnum{Saisir une valeur} et attendre}
\blockvariable{mettre \selectmenu{x} à \ovalmove{réponse}}
\blockvariable{mettre \selectmenu{x} à \ovaloperator{\ovalmove{x} * \ovalnum{6}}}
\blocklook{Dire \ovaloperator{regrouper \ovalnum{Le résultat est} et \ovalmove{x}} pendant \ovalnum{3} secondes}
\end{scratch}

\bigskip

\textbf{Programme B~~~~}
\begin{scratch}[num blocks]
\renewcommand*\numblock[1]{Ligne \no \the\numexpr#1~~}
\blockinit{quand \greenflag est cliqué}
\renewcommand*\numblock[1]{\the\numexpr#1~~}
\blockmove{demander \ovalnum{Saisir une valeur} et attendre}
\blockvariable{mettre \selectmenu{x} à \ovalmove{réponse}}
\blockvariable{mettre \selectmenu{x} à \ovaloperator{\ovalmove{x} * \ovalnum{2}}}
\blockvariable{mettre \selectmenu{x} à \ovaloperator{\ovalmove{x} + \ovalnum{26}}}
\blocklook{Dire \ovaloperator{regrouper \ovalnum{Le résultat est} et \ovalmove{x}} pendant \ovalnum{3} secondes}
\end{scratch}

\begin{enumerate}
\item  Le résultat affiché par le programme A si la valeur saisie est 5 est 30.

\item La valeur 4 est saisie dans le programme B.

$4\times 2 = 8$ (ligne 4)

$8+26=34$ (ligne 5)

Donc le résultat affiché par ce programme est 34.

\item Les instructions des lignes 4 et 5 du programme B peuvent être remplacées par la ligne

\begin{center}
{\ovaloperator{\ovaloperator{\ovalmove{x} * \ovalnum{2}} + \ovalnum{26}}}
\end{center}

\item On note $x$ le nombre saisi. \\
L'expression algébrique qui traduit le programme B est $2x+26$.\\
L'expression algébrique qui traduit le programme A est $6x$.

\item Un seul nombre conduit les deux programmes à afficher le même résultat; c'est le nombre $x$ tel que $6x=2x+26$.

Cela équivaut à
$6x-2x=26$ soit
$4x=26$ soit
$x=\dfrac{26}{4}$ soit
$x=6,5$

Le nombre $6,5$ conduit les deux programmes à afficher le même résultat.
\end{enumerate}

