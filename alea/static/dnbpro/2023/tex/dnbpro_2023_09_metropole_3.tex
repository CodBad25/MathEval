
\vspace{0.25cm}

Deux fournisseurs de maillots imprimés proposent les tarifs suivants:

\smallskip

{\renewcommand{\arraystretch}{1.5}
\begin{tabular}{|c| *2{>{\centering\arraybackslash} m{3.5cm}|}}
\cline{2-3}
\multicolumn{1}{c|}{~} & Prix d'un maillot & Frais d'impression\\
\hline
Entreprise A & 80 € & Gratuit\\
\hline
Entreprise B & 50 € & 900~€ pour l'ensemble de la commande\\
\hline
\end{tabular}}
\hfill
\parbox{4cm}{\includegraphics[width=4cm]{MetroProSept23_2}}

\begin{enumerate}
\item Calculer le prix, en euros, de 20 maillots imprimés fournis par l'entreprise A.
\item La représentation graphique du tarif de l'entreprise B en fonction du nombre de maillots commandés est donnée dans le repère suivant.

\begin{center}
\psset{xunit=0.12cm,yunit=0.002cm,arrowsize=3pt 2}
\def\xmin{-9}   \def\xmax{94} \def\ymin{-490}   \def\ymax{5400}
\begin{pspicture*}(\xmin,\ymin)(\xmax,\ymax)
\psgrid[xunit=1.2cm,yunit=1cm,subgriddiv=5,  gridlabels=0, subgridcolor=lightgray!50,gridcolor=gray!50](-1,-1)(10,11)
\psaxes[labelFontSize=\scriptstyle,ticks=none, Dx=10,Dy=500]{->}(0,0)(\xmin,\ymin)(\xmax,\ymax) 
\uput{9pt}[dl](0,0){\scriptsize 0}
\psplot[linewidth=1.2pt,linecolor=blue]{0}{80}{900 50 x mul add}
\small
\uput[r](58,3700){\textbf{\blue{}Tarif de l'entreprise B}} 
\uput[r](2,5200){\textbf{Prix en \euro}} \uput[d](70,-250){\textbf{Nombre de maillots}} 
\end{pspicture*}
\end{center}


Déterminer graphiquement le prix de 20 maillots imprimés fournis par l'entreprise B.

Laisser apparents les traits de lecture.
\item Indiquer l'entreprise qui propose le tarif le plus avantageux pour 20 maillots.
\item Représentation graphique du tarif de l'entreprise A.
\begin{enumerate}
\item  Le nombre de maillots commandés est noté $x$. Inscrire sur la copie l'expression, du tableau ci-dessous, qui correspond au tarif de l'entreprise A, en fonction de $x$.

\begin{center}
\begin{tabular}{|c|c|c|}
\hline
Expression 1 & Expression 2 & Expression 3\\
\hline
\np{3200} &  $80 + x$ &  $80 \times x$\\
\hline
\end{tabular}
\end{center}

\item Compléter le tableau de valeurs ci-dessous.

\textbf{Tableau de valeurs du Tarif de l'entreprise A}

\begin{center}
\begin{tabular}{|*5{>{\centering\arraybackslash}m{2.2cm}|}}
\hline
Nombre de maillots\newline{}$x$ & 0 & 20 & 30 & 60\\
\hline
Prix en \euro \rule[-5pt]{0pt}{20pt} & $\ldots$ & $\ldots$ & $\ldots$ & \np{4800}\\
\hline
Coordonnées des points & K\,$(0\;;\; \ldots)$ & L\,$(20\;;\; \ldots)$ & M\,$(30\;;\; \ldots)$ & N\,$(60\;;\; \np{4800})$\\
\hline
\end{tabular}
\end{center}

\item Sur le graphique précédent, placer les points K, L, M et N et tracer la droite passant par ces quatre points. Cette droite représente le tarif de l'entreprise A en fonction du nombre de maillots commandés.
\end{enumerate}

\item Pour une commande de 35 à 50 maillots, indiquer l'entreprise qui propose le tarif le plus avantageux. Justifier la réponse.
\end{enumerate}

\vspace{0.5cm}

