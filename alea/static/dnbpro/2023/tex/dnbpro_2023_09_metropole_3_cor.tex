
\vspace{0.25cm}

Deux fournisseurs de maillots imprimés proposent les tarifs suivants:

\begin{center}
{\renewcommand{\arraystretch}{1.5}
\begin{tabular}{|c| *2{>{\centering\arraybackslash} m{3.5cm}|}}
\cline{2-3}
\multicolumn{1}{c|}{~} & Prix d'un maillot & Frais d'impression\\
\hline
Entreprise A & 80 € & Gratuit\\
\hline
Entreprise B & 50 € & 900~€ pour l'ensemble de la commande\\
\hline
\end{tabular}}
\end{center}

\begin{enumerate}
\item $20\times 80 = \np{1600}$ donc  le prix de 20 maillots imprimés fournis par l'entreprise A est de \np{1600}~\euro.

\item La représentation graphique du tarif de l'entreprise B en fonction du nombre de maillots commandés est donnée dans le repère ci-après.

\begin{figure}[t!]
\centering
\psset{xunit=0.12cm,yunit=0.002cm,arrowsize=3pt 2}
\def\xmin{-9}   \def\xmax{94} \def\ymin{-490}   \def\ymax{5400}
\begin{pspicture*}(\xmin,\ymin)(\xmax,\ymax)
\psgrid[xunit=1.2cm,yunit=1cm,subgriddiv=5,  gridlabels=0, subgridcolor=lightgray!50,gridcolor=gray!50](-1,-1)(10,11)
\psaxes[labelFontSize=\scriptstyle,ticks=none, Dx=10,Dy=500]{->}(0,0)(\xmin,\ymin)(\xmax,\ymax) 
\uput{9pt}[dl](0,0){\scriptsize 0}
\psplot[linewidth=1.2pt,linecolor=blue]{0}{80}{900 50 x mul add}
\small
\uput[r](58,3700){\textbf{\blue{}Tarif de l'entreprise B}} 
\uput[r](2,5200){\textbf{Prix en \euro}} \uput[d](70,-250){\textbf{Nombre de maillots}} 
%%%%%%%%%%%%%%
\psset{linecolor=red,linestyle=dashed,linewidth=1.2pt}
\psline[ArrowInside=->](20,0)(20,1900)(0,1900)
\uput*[l](0,1900){\red\small \np{1900}}
\uput*{8pt}[d](20,0){\red\small 20}
\psdots(0,0)(20,1600)(30,2400)(60,4800)
\uput[dr](0,0){\red K} \uput[dr](20,1600){\red L} 
\uput[dr](30,2400){\red M} \uput[dr](60,4800){\red N}
\psplot{0}{80}{80 x mul}
\uput[ul](50,4000){\textbf{\red{}Tarif de l'entreprise A}} 
\end{pspicture*}
\end{figure}

Graphiquement le prix de 20 maillots imprimés fournis par l'entreprise B est de \np{1900}~\euro.

%Laisser apparents les traits de lecture.
\item Pour 20 maillots,  l'entreprise qui propose le tarif le plus avantageux est l'entreprise A.

\item Représentation graphique du tarif de l'entreprise A.

\begin{enumerate}
\item  Le nombre de maillots commandés est noté $x$. L'expression, du tableau ci-dessous, qui correspond au tarif de l'entreprise A, en fonction de $x$ est $80\times x$.

\begin{center}
\begin{tabular}{|c|c|c|}
\hline
Expression 1 & Expression 2 & $\blue \text{Expression 3}$\\
\hline
\np{3200} &  $80 + x$ &  $\blue 80 \times x$\\
\hline
\end{tabular}
\end{center}

\item On complète le tableau de valeurs.

\begin{center}
\begin{tabular}{|*5{>{\centering\arraybackslash}m{2.2cm}|}}
\hline
Nombre de maillots $x$ & 0 & 20 & 30 & 60\\
\hline
Prix en \euro \rule[-5pt]{0pt}{20pt} & $\blue 0$ & $\blue \np{1600}$ & $\blue \np{2400}$ & \np{4800}\\
\hline
Coordonnées des points & K\,$(0\;;\; {\blue 0})$ & L\,$(20\;;\; {\blue\np{1600}})$ & M\,$(30\;;\; {\blue\np{2400}})$ & N\,$(60\;;\; \np{4800})$\\
\hline
\end{tabular}
\end{center}

\item On place les points K, L, M et N et on tracer la droite passant par ces quatre points. Cette droite représente le tarif de l'entreprise A en fonction du nombre de maillots commandés.
\end{enumerate}

\item Pour une commande de 35 à 50 maillots,  l'entreprise qui propose le tarif le plus avantageux est l'entreprise B car sa représentation graphique est \textbf{en-dessous} de la représentation graphique de l'entreprise A.
\end{enumerate}

\vspace{0.5cm}

