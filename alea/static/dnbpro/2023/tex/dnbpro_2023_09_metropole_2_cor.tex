
\vspace{0.25cm}

Le nombre de licenciés dans les fédérations des sept principaux sports pratiqués en France en 2021 est donné ci-dessous.

%\begin{center}
%\includegraphics[width=\linewidth]{MetroProSept23_1}
%\end{center}

\begin{tabularx}{\linewidth}{| *7{>{\centering\arraybackslash} X|}}
\hline
Football & Tennis & Équitation & Basketball & Judo & Handball & Rugby\\
\hline
\np{1900000} & \np{950000} & \np{664000} & \np{515000} & \np{512000} & \np{335000} & \np{286000}\\
\hline
\end{tabularx}

\begin{enumerate}
\item  Le nombre de licenciés de la fédération d'équitation est \np{664000}.

\item %Vérifier par un calcul que le nombre total de licenciés des sept fédérations est 
$\np{1900000} + \np{950000} + \np{664000} + \np{515000} + \np{512000} + \np{335000} + \np{286000} = \np{5162000}$

\item Affirmation : \og Le nombre moyen de licenciés des sept fédérations est de \np{737429}. \fg

$\dfrac{\np{5162000}}{7}\approx \np{737428,57}$

Donc Le nombre moyen de licenciés des sept fédérations, arrondi à l'unité, est de \np{737429}.

%Cette affirmation est-elle exacte ? Justifier la réponse par un calcul.
\item Le pourcentage de licenciés de la fédération de rugby par rapport à l'ensemble des licenciés des sept fédérations est:
$\dfrac{\np{286000}}{\np{5162000}}\times 100 \approx 5,54$.

\item On suppose que le nombre total de licenciés des sept fédérations reste constant.\\
La fédération de rugby se fixe comme objectif de porter à 8\,\% la part de ses licenciés par rapport à l'ensemble des licenciés.

On calcule 8\,\% de \np{5162000}: cela donne $\np{5162000}\times \dfrac{8}{100}=\np{412860}$.

Le nombre de nouveaux licenciés que la fédération de rugby doit accueillir pour atteindre son objectif est donc de $\np{412860}- \np{286000}$ soit $\np{126960}$

%Calculer le nombre de nouveaux licenciés que la fédération de rugby doit accueillir pour atteindre son objectif.
\end{enumerate}

\vspace{0.5cm}

