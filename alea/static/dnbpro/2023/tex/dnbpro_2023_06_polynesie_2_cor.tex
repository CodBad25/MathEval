
%\textbf{Les calculs seront détaillés sur la copie.}

\smallskip

Hiro vient d'acheter un terrain dont on peut assimiler la forme à la figure ci-dessous.

\medskip

\textbf{Partie A - Semage du terrain}

\medskip

\begin{multicols}{2}
\begin{enumerate}
\item  %\textbf{Calculer} l'aire de la figure 1
$60\times 80=\np{4800}$ donc l'aire de la figure 1 est de $\np{4800}$~m$^2$.

\item %\textbf{Indiquer} la longueur DC et \textbf{calculer} l'aire de la figure 2.
$\text{DC}=\text{FD}=60$\\[8pt]
$\dfrac{\text{BD}\times \text{DC}}{2}=\dfrac{80\times 60}{2}=\np{2400}$

Donc l'aire de la figure 2 est de $\np{2400}$~m$^2$.

\item% \textbf{Calculer} la longueur DE et \textbf{calculer} l'aire de la figure 3.
$\text{DE} = \text{BE} - \text{BD} = 150-80=70$\\[8pt]
$\dfrac{\text{FD}\times \text{DE}}{2}=\dfrac{60\times 70}{2}=\np{2100}$

Donc l'aire de la figure 3 est de $\np{2100}$~m$^2$.

\item $\np{4800} + \np{2400} + \np{2100} = \np{9300}$ donc l'aire du terrain qu'il vient d'acheter est de \np{9300}~m$^2$

\end{enumerate}

\columnbreak

\scalebox{0.9}{
\psset{unit=0.4cm,radius=0pt,arrowsize=3pt 2}
\def\xmin{-1}   \def\xmax{19} \def\ymin{-1}   \def\ymax{18}
\begin{pspicture}(\xmin,\ymin)(\xmax,\ymax)
%\psgrid[subgriddiv=1, gridlabels=0, gridcolor=lightgray]
%\psaxes[labelFontSize=\scriptstyle, ticksize=-0pt 0pt](0,0)(\xmin,\ymin)(\xmax,\ymax)
\Cnode*(12,0){E} \Cnode*(12,7){F} \Cnode*(18,7){C}  \Cnode*(12,7){D} 
\Cnode*(12,15){B} \Cnode*(6,15){A} \Cnode*(6,7){F}   
\uput[r](E){E} \uput[dl](F){F} \uput[dr](C){C} \uput[dr](D){D} 
\uput[ur](B){B} \uput[ul](A){A} \uput[dr](C){C} 
\pspolygon[fillstyle=solid,fillcolor=lightgray!50](A)(B)(C)(D)(E)(F)
\psframe[fillstyle=solid,fillcolor=gray](B)(11.5,14.5)
\psframe[fillstyle=solid,fillcolor=gray](D)(11.5,7.5)
\psline[linestyle=dashed](B)(D)(F)
\psline{<->}(6,17)(12,17) \rput*(9,17){60 m}
\psline{<->}(4,15)(4,7) \rput*(4,11){80 m}
\psline{<->}(1,15)(1,0) \rput*(1,7.5){150 m}
\psline[linestyle=dotted](1,15)(A)
\psline[linestyle=dotted](1,0)(E)
\rput(15,7){\pmb{//}} \rput(9,7){\pmb{//}}
\rput(9,11){\Huge\ding{172}} \rput(14,9){\Huge\ding{173}}
\rput(10,5){\Huge\ding{174}}
\end{pspicture}}
\end{multicols}

\begin{enumerate}
\setcounter{enumi}{4}
\item Il voudrait semer de l'herbe de prairie sur la totalité du terrain. Les semences se vendent par sac de 12 kg, ce qui permet d'ensemencer \np{1200}~m$^2$.
\begin{enumerate}
\item %\textbf{Calculer} la masse de semences nécessaire pour ensemencer les \np{9300}~m$^2$.
12 kg permettent d'ensemencer \np{1200}~m$^2$, soit 1~kg pour 100~m$^2$. \\
Il faudra donc 93~kg pour ensemencer les \np{9300}~m$^2$.

\item $93\div 12=7,75$ donc il devra acheter 8  sacs de semences pour ensemencer l'ensemble de son terrain.
\end{enumerate}
\end{enumerate}

\medskip

\textbf{Partie B - Clôture du terrain}

\medskip

\begin{enumerate}
\item Dans le triangle rectangle BDC rectangle en D, on veut déterminer la longueur BC
\begin{enumerate}
\item On utilise le théorème de Pythagore pour calculer la longueur BC.

\item %\textbf{Vérifier} que la longueur BC est égale à 100 m.
Dans le triangle BDC rectangle en D, on a:

$\text{BC}^2=\text{BD}^2+\text{CD}^2$ donc $\text{BC}^2 = 80^2+60^2=\np{6400} + \np{3600} = \np{10000}$. On en déduit que $\text{BC}=100$.

\end{enumerate}
\end{enumerate}

On admet que EF = 92 m.

\begin{enumerate}[resume]
\item AB + BC + CD + DE + EF + FA = 60 + 100 + 60 + 70 + 92 + 80 = 462

Donc le périmètre du terrain est de 462 m.

\item Il souhaiterait grillager le contour de son terrain.\\
Il dispose de 460 m de grillage. 

$460 <462$ donc la longueur de grillage de 460~m n'est pas suffisante.
\end{enumerate}
 
 
