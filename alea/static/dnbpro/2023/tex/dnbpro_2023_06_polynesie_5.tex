
\medskip

Les tableaux ci-dessous présentent trois figures et trois programmes de construction.

\medskip

\begin{tabularx}{\linewidth}{| *3{>{\centering\arraybackslash}X|}}
\hline
\textbf{Figure 1} & \textbf{Figure 2}  & \textbf{Figure 3} \\
\hline
\psset{unit=1cm}
\def\xmin{0}   \def\xmax{3} \def\ymin{-0.5}   \def\ymax{5.5}
\begin{pspicture}(\xmin,\ymin)(\xmax,\ymax)
%\psgrid[subgriddiv=1, gridlabels=0, gridcolor=lightgray]
%\psaxes[labelFontSize=\scriptstyle, ticksize=-0pt 0pt](0,0)(\xmin,\ymin)(\xmax,\ymax)
\pspolygon(0,4.5)(2,4.5)(2,0)
\end{pspicture}
&
\psset{unit=1cm}
\def\xmin{0}   \def\xmax{3} \def\ymin{-0.5}   \def\ymax{5.5}
\begin{pspicture}(\xmin,\ymin)(\xmax,\ymax)
%\psgrid[subgriddiv=1, gridlabels=0, gridcolor=lightgray]
%\psaxes[labelFontSize=\scriptstyle, ticksize=-0pt 0pt](0,0)(\xmin,\ymin)(\xmax,\ymax)
\psframe(0,0)(3,5)
\end{pspicture}
&
\psset{unit=1cm}
\def\xmin{0}   \def\xmax{3} \def\ymin{-0.5}   \def\ymax{5.5}
\begin{pspicture}(\xmin,\ymin)(\xmax,\ymax)
%\psgrid[subgriddiv=1, gridlabels=0, gridcolor=lightgray]
%\psaxes[labelFontSize=\scriptstyle, ticksize=-0pt 0pt](0,0)(\xmin,\ymin)(\xmax,\ymax)
\psframe(0,1)(3,4)
\end{pspicture}\\
\hline
\multicolumn{3}{c}{~}\\
\hline
\textbf{Programme 1} & \textbf{Programme 2} & \textbf{Programme 3}\\
\hline
~\newline
\scalebox{0.7}{
\begin{scratch}
\blockinit{quand \greenflag est cliqué}
\blockmove{s'orienter à \ovalnum{90}}
\blockmove{aller à x: \ovalnum{0} y: \ovalnum{0}}
\blockpen{stylo en position d'écriture}
\blockrepeat{répéter \ovalnum{4} fois}
{\blockmove{avancer de \ovalnum{60} pas}
\blockmove{tourner \turnright{} de \ovalnum{90} degrés}
}
\end{scratch}}
&
~\newline
\scalebox{0.7}{
\begin{scratch}
\blockinit{quand \greenflag est cliqué}
\blockmove{s'orienter à \ovalnum{90}}
\blockmove{aller à x: \ovalnum{0} y: \ovalnum{0}}
\blockpen{stylo en position d'écriture}
\blockmove{avancer de \ovalnum{50} pas}
\blockmove{tourner \turnright{} de \ovalnum{90} degrés}
\blockmove{avancer de \ovalnum{90} pas}
\blockmove{aller à x: \ovalnum{0} y: \ovalnum{0}}
\end{scratch}}
&
~\newline
\scalebox{0.7}{
\begin{scratch}
\blockinit{quand \greenflag est cliqué}
\blockmove{s'orienter à \ovalnum{90}}
\blockmove{aller à x: \ovalnum{0} y: \ovalnum{0}}
\blockpen{stylo en position d'écriture}
\blockrepeat{répéter \ovalnum{2} fois}
{\blockmove{avancer de \ovalnum{60} pas}
\blockmove{tourner \turnright{} de \ovalnum{90} degrés}
\blockmove{avancer de \ovalnum{90} pas}
\blockmove{tourner \turnright{} de \ovalnum{90} degrés}
}
\end{scratch}}
~\newline\\
\hline
\end{tabularx}

\begin{enumerate}
\item  \textbf{Associer} chacune des figures à son programme correspondant.
\item \textbf{Compléter} le programme suivant pour obtenir la figure ci-dessous constituée de trois carrés identiques.

\begin{center}
\psset{unit=0.8cm,arrowsize=3pt 2}
\def\xmin{0}   \def\xmax{9} \def\ymin{-1}   \def\ymax{3}
\begin{pspicture}(\xmin,\ymin)(\xmax,\ymax)
\psframe(0,0)(3,3) \psframe(3,0)(6,3) \psframe(6,0)(9,3) 
\psline{<->}(0,-0.5)(3,-0.5) \uput[d](1.5,-0.5){60 pas}
\end{pspicture}
\end{center}



Le bloc \og Tracer carré \fg{} permet de tracer un carré de 60 pas de côté.

\begin{center}
\begin{scratch}
\blockinit{quand \greenflag est cliqué}
\blockrepeat{répéter \ovalnum{\hspace*{0.5cm}} fois}
{\blocklist{Tracer carré}
\blockpen{relever le stylo}
\blockmove{avancer de \ovalnum{\hspace*{0.5cm}} pas}
}
\end{scratch}
\end{center}


\end{enumerate}

