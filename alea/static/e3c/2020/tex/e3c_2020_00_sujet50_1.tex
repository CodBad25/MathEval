
\medskip

Ce QCM comprend 5 questions. Pour chacune des questions, une seule des quatre réponses proposées est correcte. Les questions sont \textbf{indépendantes}. 

Pour chaque question, indiquer le numéro de la question et recopier sur la copie la lettre correspondante à la réponse choisie. Aucune justification n'est demandée, cependant des traces de recherche au brouillon peuvent aider à trouver la bonne réponse. 

Chaque réponse correcte rapporte $1$ point. Une réponse incorrecte ou une question sans réponse n'apporte ni ne retire de point.

\medskip

\textbf{Question 1}

Pour tout réel $x$, l'expression $\text{e}^x\times \text{e}^{x+2}$ est égale à :

\begin{center}
\begin{tabularx}{\linewidth}{|*{4}{X|}}\hline
\textbf{a.~~}$\text{e}^{2x+2}$&\textbf{b.~~}$\text{e}^{x^2+2}$&\textbf{c.~~}$\text{e}^{\frac{x}{x+2}}$&\textbf{d.~~}$\text{e}^{x^2 + 2x}$\\ \hline
\end{tabularx}
\end{center}
\textbf{Question 2}

Soit $g$ une fonction définie et dérivable en 1. Dans un repère du plan, une équation de la
tangente à la courbe de la fonction $g$ au point d'abscisse 1 est : 

\begin{center}
\begin{tabularx}{\linewidth}{|*{2}{X|}}\hline
\textbf{a.~~}$y=g(1) \times (x - 1)- g'(1)$&\textbf{b.~~}$y= g'(1) \times (x - 1) +g(1)$\\
\textbf{c.~~}$y = g'(1) \times (x + 1) - g(1)$&\textbf{d.~~}$y = g(1) \times (x + 1) + g'(1)$\\ \hline
\end{tabularx}
\end{center}

\textbf{Question 3}

Le plan est muni d'un repère \Oij. On considère la droite $(d)$ de vecteur directeur $\vect{u}(4~;~7)$ et passant par le point A$(-2~;~3)$. Une équation cartésienne de la droite $(d)$ est:

\begin{center}
\begin{tabularx}{\linewidth}{|*{4}{X|}}\hline
\textbf{a.~~}$- 7x + 4y- 26= 0$&\textbf{b.~~}$4x + 7y - 13 = 0$&\textbf{c.~~}$- 7x+ 4y+ 26= 0$&\textbf{d.~~}$4x - 7y + 29 = 0$\\ \hline
\end{tabularx}
\end{center}

\textbf{Question 4}

$t$ est un réel. On sait que $\cos(t)= \dfrac{2}{3}$. Alors $\cos(t+4\pi) + \cos (-t)$ est égal à : 

\begin{center}
\begin{tabularx}{\linewidth}{|*{4}{X|}}\hline
\textbf{a.~~}$- \dfrac{4}{3}$&\textbf{b.~~}$0$&\textbf{c.~~}$\dfrac{4}{3}$&\textbf{d.~~}$\dfrac{2}{3}$\rule[-3mm]{0mm}{9mm}\\ \hline
\end{tabularx}
\end{center}

\medskip

\textbf{Question 5}

On considère, dans un repère du plan, la parabole $(P)$ d'équation : $y= - x^2 +6x - 9$. La parabole $(P)$ n'admet :

\begin{center}
\begin{tabularx}{\linewidth}{|*{4}{X|}}\hline
\textbf{a.~~}aucun point d'intersection avec l'axe des abscisses&\textbf{b.~~}un seul point d'intersection avec l'axe des abscisses&\textbf{c.~~}deux points d'intersection avec l'axe des abscisses&\textbf{d.~~}trois points d'intersection avec l'axe des abscisses\\ \hline
\end{tabularx}
\end{center}
\bigskip

