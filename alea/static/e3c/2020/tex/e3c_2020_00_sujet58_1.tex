
\medskip

\emph{Ce QCM comprend $5$ questions indépendantes.}

\emph{Pour chacune d'elles, une seule des réponses proposées est exacte.}
 
\emph{Indiquer pour chaque question sur la copie la lettre correspondant à la réponse
choisie. Aucune justification n'est demandée.}

\emph{Chaque réponse correcte rapporte $1$ point. Une réponse incorrecte ou une absence de
réponse n'apporte, ni ne retire de point.}

\medskip

\begin{enumerate}
\item Pour tout réel $x$, $\e^{2x}+\e^{4x}$ est égal à

\begin{tabularx}{\linewidth}{*{4}{X}}
\textbf{a.~~} $\e^{6x}$ &\textbf{b.~~} $\e^{2x}(1+\e^{2})$&\textbf{c.~~}$\e^{3x}(\e^{x}+\e^{-x})$& \textbf{d.~~} $\e^{8x^2} $.
\end{tabularx}

\item Dans le plan muni d'un repère orthonormé \Oij, on considère les vecteurs $\vv{u}(-5~;~2)$ et \mbox {$\vv{v}(4~;~10)$} et la droite (d) d'équation : $5x + 2y + 3 = 0$.

\begin{tabularx}{\linewidth}{*{4}{X}}
\textbf{a.~~} $\vv{u}$ et $\vv{v}$  sont colinéaires &\textbf{b.~~} $\vv{u}$ est un vecteur normal à la droite (d) &\textbf{c.~~} $\vv{u}$ et $\vv{v}$  sont orthogonaux& \textbf{d.~~} $\vv{u}$ est un vecteur directeur de (d) .
\end{tabularx}

\item La dérivée $f'$ de la fonction $f$ définie sur $\R$ par $f(x)=(2x-1)\e^{-x}$  est : 

\begin{tabularx}{\linewidth}{*{4}{X}}
\textbf{a.~~}  $2x\e^{-x}$ &\textbf{b.~~}$-2x\e^{-x}$ &\textbf{c.~~}$(-2x+3)\e^{-x}$& \textbf{d.~~}$2\e^{-x}+(2x-1) \e^{-x}$  .
\end{tabularx}

\item Pour tout réel $x$, on a $\sin(\pi+x)=$

\begin{tabularx}{\linewidth}{*{4}{X}}
\textbf{a.~~} $-\sin (x)$ &\textbf{b.~~} $\cos (x)$&\textbf{c.~~}$\sin (x)$& \textbf{d.~~} $-\cos (x)$ .
\end{tabularx}
\smallskip

\begin{multicols}{2}
 \item Soit $f$ une fonction définie et dérivable sur $\R$ dont la courbe représentative est donnée ci-contre. La tangente à la courbe au point A est la droite T.


\psset{unit=0.95cm,labelFontSize=\scriptstyle,labelsep=0.1pt,showorigin=false}
\begin{pspicture}(-1.5,-2.8)(3.5,3.5)
\multido{\n=-1+1}{8}{\psline[linewidth=0.75pt,linecolor=lightgray](\n,-2.1)(\n,3.2)}
\multido{\n=-2+1}{6}{\psline[linewidth=0.75pt,linecolor=lightgray](-1.1,\n)(6.1,\n)}
\psaxes[linewidth=0.95pt,]{->}(0,0)(-1.2,-2.2)(6.1,3.3)
\psdots[dotstyle=+,dotscale =1.4,dotangle=45](0,3) 
\uput[dl](0,0){O} \uput[dr](0.2,3.2){A} \uput[u](1.7,-2){\blue $\mathcal{C}_f$}\uput[d](0.5,-1.1){\red T}
%\pscurve[linewidth=0.45pt,linecolor=blue,plotpoints=5000](0,3)(0.9,0)(1,-0.2)(2,-1.6)(3,-1.7)(4,-1.5)(5,-1.3)(5.9,-1)
\psplot[linewidth=1pt,linecolor=red,plotpoints=3000]{0}{1}{x 5 neg mul 3 add}
\psplot[linewidth=1.25pt,linecolor=blue,plotpoints=2000]{-0.1}{6}{0.9 x sub 2.71828 x 0.5 mul exp div 3.333 mul}
\end{pspicture}
\end{multicols}

\begin{tabularx}{\linewidth}{*{4}{X}}
\textbf{a.~~}$f'(0)=3$  &\textbf{b.~~}$f'(0)=\frac{1}{5}$  &\textbf{c.~~}$f'(0)=5$& \textbf{d.~~}$f'(0)=-5$.
\end{tabularx}
 
\end{enumerate}

\vspace{0.75cm}

