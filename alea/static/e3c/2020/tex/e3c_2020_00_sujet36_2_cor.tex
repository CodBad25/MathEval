
\subsection*{1.}

\paragraph{a.} Avec \(\overrightarrow{AB} = \begin{pmatrix} -2 \\ 4 \end{pmatrix}\) et \(\overrightarrow{AC} = \begin{pmatrix} 1 \\ 2 \end{pmatrix}\), on a :
\[
\overrightarrow{AB} \cdot \overrightarrow{AC} = -2 \times 1 + 4 \times 2 = -2 + 8 = 6.
\]

\paragraph{b.} On a :
\[
\|\overrightarrow{AB}\|^2 = \overrightarrow{AB} \cdot \overrightarrow{AB} = 4 + 16 = 20, \text{ d'où } \|\overrightarrow{AB}\| = \sqrt{20} = \sqrt{4 \times 5} = 2 \sqrt{5},
\]
\[
\|\overrightarrow{AC}\|^2 = \overrightarrow{AC} \cdot \overrightarrow{AC} = 1 + 4 = 5, \text{ d'où } \|\overrightarrow{AC}\| = \sqrt{5}.
\]

\paragraph{c.} On peut également écrire le produit scalaire :
\[
\overrightarrow{AB} \cdot \overrightarrow{AC} = \|\overrightarrow{AB}\| \times \|\overrightarrow{AC}\| \times \cos(\widehat{BAC}),
\]
soit :
\[
6 = 2 \sqrt{5} \times \sqrt{5} \times \cos(\widehat{BAC}) \iff \cos(\widehat{BAC}) = \dfrac{6}{2 \sqrt{5} \times \sqrt{5}} = \dfrac{6}{10} = \dfrac{3}{5} = 0{,}6.
\]
La calculatrice donne \(\widehat{BAC} \approx 53{,}1^\circ\), soit \(53^\circ\) au degré près.

\subsection*{2.}

\paragraph{a.} Une équation de la droite \((AB)\) étant \(y = ax + b\), alors on a le système :
\[
\begin{cases} 
-1 = 2a + b \\ 
3 = 0 \times a + b 
\end{cases}
\]
d'où \(b = 3\) et, par substitution, on obtient \(-1 = 2a + 3\) ou \(a = -2\).

Donc :
\begin{align*}
&M(x\,;\,y) \in (AB) \\
\iff &y = -2x + 3 \\
\iff &2x + y - 3 = 0.
\end{align*}

\paragraph{b.} Si \(H(x\,;\,y)\), on a \(\overrightarrow{CH} \begin{pmatrix} x - 3 \\ y - 1 \end{pmatrix}\), et les coordonnées vérifient :
\begin{align*}
&\begin{cases} 
H \in (AB) \\ 
\overrightarrow{CH} \cdot \overrightarrow{AB} = 0
\end{cases} \\
\iff &\begin{cases} 
2x + y - 3 = 0 \\ 
-2(x - 3) + 4(y - 1) = 0 
\end{cases} \\
\iff &\begin{cases}
2x + y - 3 = 0 \\
-2x + 6 + 4y - 4 = 0
\end{cases} \\
\iff &\begin{cases}
2x + y - 3 = 0 \\
-2x + 4y + 2 = 0
\end{cases}
\end{align*}

Ce qui donne par résolution :
\[
5y - 1 = 0 \iff y = \dfrac{1}{5},
\]
et :
\begin{align*}
&2x + \dfrac{1}{5} - 3 = 0 \\
\iff &2x - \dfrac{14}{5} = 0 \\
\iff &2x = \dfrac{14}{5} \\
\iff &x = \dfrac{7}{5}.
\end{align*}
Le point \(H\) a donc pour coordonnées \(H\left( \dfrac{7}{5}\,;\,\dfrac{1}{5} \right)\).

