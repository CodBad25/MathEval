
\medskip

Une entreprise fabrique chaque jour $x$ tonnes d'un produit. Le coût total mensuel, en milliers d'euros, pour produire chaque jour $x$ tonnes de ce produit est modélisé par la fonction $C$ définie sur l'intervalle [0~;~10] par:

\[C(x) = (5x - 2)\text{e}^{-0,2x} + 2.\]

On a représenté ci-dessous la courbe $\mathcal{C}_C$ de la fonction $C$ dans un repère.

\begin{center}
\psset{unit=0.5cm,arrowsize=2pt 3}
\begin{pspicture*}(-1.2,-1.2)(10.5,12.5)
\psgrid[gridlabels=0pt,subgriddiv=1](0,0)(10.5,12.5)
\psaxes[linewidth=1.25pt,Dx=2,Dy=2]{->}(0,0)(0,0)(10.5,12.5)
\psaxes[linewidth=1.25pt,Dx=2,Dy=2](0,0)(0,0)(10.5,12.5)
\psplot[plotpoints=2000,linewidth=1.25pt,linecolor=blue]{0}{10}{5 x mul 2 sub 2.71828 0.2 x mul exp div 2 add}
\uput[ur](9,9){\blue $\mathcal{C}_C$}
\end{pspicture*}
\end{center}
\medskip

\begin{enumerate}
\item Par lecture graphique, donner une estimation de la quantité journalière de produit pour laquelle le coût total mensuel est maximal.
\item Le \textbf{coût marginal} $C_m$ qui correspond au supplément de coût total pour la production d'une unité de valeur supplémentaire, est assimilé à la \textbf{dérivée} de la fonction coût total.
	\begin{enumerate}
		\item Démontrer que le coût marginal $C_m$ est défini sur l'intervalle [0~;~10] par: 
		
\[C_m(x) = (-x + 5,4)\text{e}^{-0,2x}.\]
		
		\item Pour quelle quantité de produit fabriqué par jour le coût marginal est-il négatif ? 
		\item Donner le tableau de variations de la fonction $C$ sur l'intervalle [0~;~10].
		\item Déterminer le coût total mensuel maximal sur l'intervalle considéré. On donnera la valeur arrondie à l'euro près.
	\end{enumerate}
\end{enumerate}

\bigskip

