
\medskip

Camille et Dominique ont été embauchés au même moment dans une entreprise et ont négocié leur contrat à des conditions différentes:

\setlength\parindent{10mm}
\begin{itemize}
\item Camille a commencé en 2010 avec un salaire annuel de \np{14400}~\euro, alors que le salaire de Dominique était, cette même année, de \np{13200}~\euro.
\item Le salaire de Camille augmente de $600$~\euro{} par an alors que celui de Dominique augmente de 4 \,\% par an.
\end{itemize}
\setlength\parindent{0mm}

\medskip

\begin{enumerate}
\item Quels étaient les salaires annuels de Camille et de Dominique en 2012 ? 
\item  On modélise les salaires de Camille et de Dominique à l'aide de suites.
	\begin{enumerate}
		\item On note $u_n$ le salaire de Camille en l'année $2010 +n$. On a donc $u_0 = \np{14400}$.
		
Quelle est la nature de la suite $\left(u_n\right)$ ?
		\item Déterminer en quelle année le salaire de Camille dépassera \np{20000}~ \euro.
		\item On note $v_n$ le salaire de Dominique en l'année $2010 +n$.
		
Exprimer $v_{n+1}$ en fonction de $v_n$.
		\item Calculer le salaire de Dominique en 2020. On arrondira le résultat à l'euro.
	\end{enumerate}
\item On veut déterminer à partir de quelle année le salaire de Dominique dépassera celui de Camille. 

Pour cela, on dispose du programme incomplet ci-dessous écrit en langage Python.

Recopier et compléter les quatre parties en pointillé du programme ci-dessous:

\begin{center}
\begin{tabular}{|l|}\hline
\texttt{def algo ( ) :}\\
\qquad \texttt{A= \np{14400}}\\
\qquad \texttt{B= \np{13200}}\\
\qquad \texttt{n = 0}\\
\qquad \texttt{while \ldots}\\
\qquad \qquad \texttt{A= \ldots}\\
\qquad \qquad \texttt{B= \ldots}\\
\qquad \qquad \texttt{n= \ldots}\\
\qquad \texttt{return(n)}\\ \hline
\end{tabular}
\end{center}
\end{enumerate}

\bigskip

