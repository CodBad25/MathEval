  
\medskip
Ce QCM comprend 5 questions.

Pour chacune des questions, une seule des quatre réponses proposées est correcte. Les
questions sont indépendantes.

Pour chaque question, indiquer le numéro de la question et recopier sur la copie la lettre
correspondant à la réponse choisie.

Aucune justification n'est demandée mais il peut être nécessaire d'effectuer des recherches
au brouillon pour déterminer votre réponse.

Chaque réponse correcte rapporte $1$ point. Une réponse incorrecte ou une question sans
réponse n'apporte ni ne retire de point.

\medskip
\textbf{Question 1}
\medskip 


Soient $\vv{u}$ et $\vv{v}$ deux vecteurs de coordonnées respectives $(-1~;~0)$ et $(-3~;~4)$ dans un repère orthonormé du plan. Alors $ \left\|\vv{u}-\vv{v}\right\|$ est égale à :

\medskip

\begin{tabularx}{\linewidth}{*{4}{X}}
\textbf{a.~~} $4\sqrt{2} $ &\textbf{b.~~} $\sqrt{32} $&\textbf{c.~~}$20 $& \textbf{d.~~} $2\sqrt{5}  $.
\end{tabularx}

\medskip
\textbf{Question 2}
\medskip 

Le tableau de signes de la fonction polynôme définie sur $\R$ par $f(x) = x^2 + 2x + 5$ est :

\medskip

\begin{tabularx}{\linewidth}{*{4}{X}}
\textbf{a.~~} \begin{tabular}[]{|c|lcccccr|}
\hline
$x$ &$-\infty$& &$-3$& &1& &$+\infty$\\
\hline
$f(x)$& &+ &0& $-$ &0 &+&\\\hline
\end{tabular} &\textbf{b.~~}
\begin{tabular}[]{|c|lcccr|}\hline
$x$& $-\infty$& &$- 16$& &$+\infty$\\\hline
$f(x)$&        &+& 0  & + & \\\hline
\end{tabular}\\[1.2cm]

\textbf{c.~~}
\begin{tabular}[]{|c|l m{1.2cm}r|}
\hline
 $x$& $-\infty$& &$+\infty$\\\hline
$f(x)$&& \centering +&\\\hline
\end{tabular}
& \textbf{d.~~} 
\begin{tabular}[]{|c|l m{1.2cm}r|}
\hline
 $x$& $-\infty$& &$+\infty$\\\hline
$f(x$)&& \centering $-$&\\\hline
\end{tabular}.
\end{tabularx}

\medskip
\textbf{Question 3}
\medskip 

Sur l'intervalle $] -\pi~;~\pi]$, l'équation $\sin(x) = \dfrac{1}{2}$
a pour solution(s)

\medskip
\begin{tabularx}{\linewidth}{*{4}{X}}
\textbf{a.~~} $\dfrac{\pi}{6} $ &\textbf{b.~~} $\dfrac{\pi}{3}$ et $\dfrac{2\pi}{3} $&\textbf{c.~~}$-\dfrac{\pi}{6}$ et $\dfrac{\pi}{6} $& \textbf{d.~~} $\dfrac{\pi}{6}$ et $\dfrac{5\pi}{6} $.
\end{tabularx}

\medskip

\textbf{Question 4}

\medskip 

On considère la suite $\left(u_n\right)$ définie par $u_0 = 15$ et pour tout entier naturel $n$ :

\[u_{n+1} = 0,8 u_n + 1.\]

On a écrit la fonction suite() ci-contre en langage Python.

\begin{python}
def suite():
	n=0
	u=15
	while u>6:
		n=n+1
		u=0.8*u+1
	return n
\end{python}

L'appel de cette fonction renvoie :

\medskip

\begin{tabularx}{\linewidth}{*{2}{X}}
\textbf{a.~~} Le plus petit entier $n$ tel que $u_n > 6$ &\textbf{b.~~}Le plus petit entier $n$ tel que $u_n \leqslant 6$ \\\textbf{c.~~} Le premier terme de la suite tel que $u_n > 6$ & \textbf{d.~~}Le premier terme de la suite tel que $u_n \leqslant 6$ .
\end{tabularx}

\medskip

\textbf{Question 5}

\medskip 

Pour tout réel $x$,\, $\e^{3x-5} \times \e^{4 - 3x}$ est égal à :

\begin{tabularx}{\linewidth}{*{4}{X}}
\textbf{a.~~} $\dfrac{1}{\text{e}}$ &\textbf{b.~~} $ \e^{(3x-5)\times(4-3x)}$&\textbf{c.~~}$ \e $& \textbf{d.~~} $\e^{-9x^2+27x-20} $.
\end{tabularx}

\vspace{0,5cm}

