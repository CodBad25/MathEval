	
	\subsection*{Question 1}
	
	Soit \(ABC\) un triangle tel que \(AB = 6\), \(AC = 3\) et \(\widehat{ BAC} = \dfrac{\pi}{3}\).
	
	On a \(\overrightarrow{AB} \cdot \overrightarrow{AC} = AB \times AC \times \cos(\widehat{BAC}) = 6 \times 3 \times \dfrac{1}{2} = 3 \times 3 = 9\).
	
	\subsection*{Question 2}
	
	\[
	\dfrac{f(1+h) - f(1)}{h} = h^2 + 3h - 1 \quad \text{ou} \quad \dfrac{f(1+h) - f(1)}{(1+h)-1} = h^2 + 3h - 1.
	\]
	
	On a par définition
	\[
	\lim_{h \to 0} \dfrac{f(1+h) - f(1)}{(1+h)-1} = f'(1) \quad \text{si cette limite existe}.
	\]
	
	Or,
	\[
	\lim_{h \to 0} h^2 + 3h - 1 = -1.
	\]
	
	Donc \(f'(1) = -1\) (nombre dérivé de la fonction \(f\) en 1).
	
	\subsection*{Question 3}
	
	Soit \(f\) la fonction définie sur \(\mathbb{R}\) par \(f(x) = (x + 2)e^x\).
	
	\(f\) est un produit de fonctions dérivables sur \(\mathbb{R}\), donc sur \(\mathbb{R}\),
	\[
	f'(x) = 1e^x + (x + 2)e^x = e^x (1 + x + 2) = (x + 3)e^x.
	\]
	
	\subsection*{Question 4}
	
	Soit \(f\) une fonction telle que \(f(2) = 5\) et \(f'(2) = -1\). Dans un repère, la tangente à la courbe représentative de \(f\) au point d’abscisse 2 a pour équation :
	
	On sait qu’une équation de la tangente à la courbe représentative de \(f\) au point d’abscisse 2 est :
$y  = f'(2)(x - 2)+f(2)$, soit  $y - 5 = -1(x - 2)$ ou encore $y = -x + 7$.
	
	
	\subsection*{Question 5}
	
	
	Pour l’équation réduite de la tangente à la courbe \(C_f\) au point \(A\), l’ordonnée à l’origine est celle de \(B\) soit \(-\dfrac{5}{3}\) et son coefficient directeur est celui de la droite \((AB)\), soit
	\[
	\dfrac{-\dfrac{5}{3} - \dfrac{4}{3}}{0 - 1} = \dfrac{-3}{-1} = 3.
	\]
	
	On sait que ce coefficient directeur est égal au nombre dérivé \(f'(1) = 3\).
	
