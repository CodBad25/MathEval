
\medskip

Ce QCM comprend 5 questions. Pour chacune des questions, une seule des quatre réponses proposées est correcte. Les questions sont indépendantes.

Pour chaque question, indiquer le numéro de la question et recopier sur la copie la lettre correspondante à la réponse choisie.

Aucune justification n'est demandée mais il peut être nécessaire d'effectuer des recherches au brouillon pour aider à déterminer votre réponse.

Chaque réponse correcte rapporte 1 point. Une réponse incorrecte ou une question sans réponse n'apporte, ni ne retire de point

\medskip

\textbf{Question 1}

\medskip

\begin{minipage}{8cm}
{La courbe ci-contre $\mathcal{C}_f$ est la représentation graphique, dans un repère orthonormé, d'une fonction  $f$. Les droites $d$ et $d'$ sont respectivement les tangentes à la courbe  $\mathcal{C}_f$ aux points d'abscisses 1 et 2.

Les équations réduites de $d$ et $d'$  sont respectivement : $d : y = 2x - 2$ et $d' : y = - x + 2$.}
\end{minipage}\hspace{0,5cm}
\begin{minipage}{8.75cm}
{\psset{unit=1.2cm}
\begin{pspicture*}(-0.6,-2.2)(4,2.2)
\psgrid[gridlabels=0pt,subgriddiv=1,gridwidth=0.15pt]
\psaxes[linewidth=1.25pt,labelFontSize=\scriptstyle]{->}(0,0)(-0.6,-2.2)(4,2.2)
\psplot[plotpoints=2000,linewidth=1.25pt,linecolor=red]{0.2}{4}{x 3 exp x dup mul 6 mul sub 11 x mul add 6 sub}
\uput[ul](1,0){\small A} \uput[ur](2,0){\small B}
\psline(-0.6,2.6)(4,-2)\psline(-0.6,-3.2)(4,6)
\uput[dr](2,2){$d$}\uput[ur](0.2,1.8){$d'$}\uput[r](3.5,1.5){$\red \mathcal{C}_f$}
\end{pspicture*}
}\end{minipage}

\medskip

\begin{tabularx}{\linewidth}{*{4}X}
\textbf{a.~~}$f'(1) = 0$ &\textbf{b.~~}$f'(2) = 2$ &\textbf{c.~~}$f'(2) = -1$ &\textbf{d.~~}$f'(1) = -2$ 
\end{tabularx}

\medskip

\textbf{Question 2}

\medskip

Soit $x \in \left[\dfrac{\pi}{2}~;~\dfrac{\pi}{2}\right]$ tel que $\sin x = \dfrac{1}{2}$.

Parmi les propositions suivantes, laquelle est juste ?

\begin{tabularx}{\linewidth}{*{4}X}
\textbf{a.~~}$\cos x = - \dfrac{\sqrt{3}}{2}$ &\textbf{b.~~}$x = \dfrac{\pi}{6}$ &\textbf{c.~~}$\cos x =  \dfrac{\sqrt{3}}{2}$ &\textbf{d.~~}$x = -\dfrac{7\pi}{6}$ 
\end{tabularx}
\medskip

\textbf{Question 3}

\medskip


\begin{minipage}{8.75cm}
{Soit (O, I, J) un repère orthonormé du plan.

Soit  A et B deux points de coordonnées respectives (3~;~4~) et (4~;~0).

Parmi les propositions suivantes, laquelle est
juste?}
\end{minipage}\hspace{0.5cm}\begin{minipage}{8.75cm}
{\psset{unit=1cm,arrowsize=2pt 3}
\begin{pspicture}(-0.5,-0.5)(4.2,4.2)
\psgrid[gridlabels=0pt,subgriddiv=1,gridwidth=0.15pt]
\psaxes[linewidth=1.25pt,labelFontSize=\scriptstyle](0,0)(-0.5,-0.5)(4.2,4.2)
\psaxes{->}(1,1)
\pspolygon(3,4)(4,0)\psline(3,0)(3,4)
\uput[ul](0,0){O}\uput[u](3,4){A}\uput[ur](4,0){B} \uput[ur](3,0){H}
\psframe(3,0)(2.8,0.2)\uput[ul](1.5,2){5}\uput[u](1.5,0){4}\uput[ur](3.5,2){$\sqrt{17}$}
\uput[dr](1,0){I}\uput[ul](0,1){J}
\end{pspicture}
}
\end{minipage}

\medskip

\begin{tabularx}{\linewidth}{*{2}X}
\textbf{Proposition A }: $\vect{\text{OA}} \cdot \vect{\text{OB}} = 20$&\textbf{Proposition B }: $\sin \widehat{\text{AOB}} = \dfrac{\sqrt{17}}{5}$\\
\textbf{Proposition C }: $\cos \widehat{\text{AOB}} = \dfrac{4}{5}$&\textbf{Proposition D } : $\sin \widehat{\text{AOB}} = \dfrac{4}{5}$
\end{tabularx}
\medskip

\textbf{Question 4}

\medskip

Soit (0, I, J) un repère orthonormé du plan.

Soit $d$ une droite dont une équation cartésienne est : $3x + 2y - 10 = 0$.

Une équation cartésienne de la droite $d'$ perpendiculaire à la droite $d$ et passant par le point A de coordonnées (1~;~2) est :

\begin{center}
\begin{tabularx}{\linewidth}{*{2}X}
\textbf{Proposition A }: $ 3x+2y- 7 = 0$&\textbf{Proposition B }: $2x +3y - 8= 0 $ \\
\textbf{Proposition C }: $2x- 3y+4 = 0$	& \textbf{Proposition D }: $3x - 2y +1 = 0$
\end{tabularx}
\end{center}

\medskip

\textbf{Question 5}

\medskip

Soit (O, I, J) un repère orthonormé du plan.

Soit A et B deux points de coordonnées respectives (1~;~2) et $(5~;~-2)$. 

Une équation cartésienne du cercle $C$ de diamètre [AB] est :

\begin{center}
\begin{tabularx}{\linewidth}{*{2}X}
\textbf{Proposition A }: $ x^2+y^2 - 8x- 2y + 7 = 0$ 	&\textbf{Proposition B }: $(x- 1)^2+(y - 2)^2 = 32$\\ \textbf{Proposition C }: $x^2 + y^2 - 4x +2y - 5 = 0$	& \textbf{Proposition D }: $x^2+y^2 -6x + 1 = 0$
\end{tabularx}
\end{center}
%\end{minipage}

\vspace{0,5cm}

