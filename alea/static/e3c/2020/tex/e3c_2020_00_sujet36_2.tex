  
\medskip

Dans un repère orthonormé \Oij{} du plan, on considère les points A$(2~;~-1)$,
B$(0~;~3)$ et C$(3~;~1)$.
\begin{enumerate}
\item \begin{enumerate}
		\item Vérifier que $\vv{\text{AB}}\cdot\vv{\text{AC}} = 6$.
%$\vect{\text{AB}}\binom{- 2}{4}$ et $\vect{\text{AC}}\binom{1}{2}$, d'où 
%$\vv{\text{AB}}\cdot\vv{\text{AC}} = - 2 + 8 = 6$.
		\item Calculer $\left\|\vv{\text{AB}}\right\|$ et $\left\|\vv{\text{AC}}\right\|$, on donnera les valeurs exactes.
%$\left\|\vv{\text{AB}}\right\|^2 = \vv{\text{AB}}\cdot\vv{\text{AB}} = 4 + 16 = 20$, d'où $\left\|\vv{\text{AB}}\right\| = \sqrt{20} = 2\sqrt{5}$ ;
%
%$\left\|\vv{\text{AC}}\right\|^2 = \vv{\text{AC}}\cdot\vv{\text{AC}} = 1 + 4  = 5$, d'où $\left\|\vv{\text{AC}}\right\| = \sqrt{5}$.
		\item Vérifier que $\cos(\widehat{\text{BAC}}) = 0,6$ et en déduire la mesure de l'angle $\widehat{\text{BAC}}$ au degré près.
%On a aussi :

%$\vv{\text{AB}}\cdot\vv{\text{AC}}  = \left\|\vv{\text{AB}}\right\|  \times \left\|\vv{\text{AC}}\right\| \cos  (\widehat{\text{BAC}}) $, soit :
%
%$6 = \sqrt{5} \times 2\sqrt{5} \cos  (\widehat{\text{BAC}}) $ ; on en déduit que 
%
%$ \cos  (\widehat{\text{BAC}}) = \dfrac{6}{2 \times 5} = \dfrac{3}{5} = 0,6$.
		\end{enumerate}
\item 
	\begin{enumerate}
		\item Vérifier qu'une équation cartésienne de la droite (AB) est $2x + y - 3 = 0$.
%Soit $y = ax + b$ est l'équation de la droite (AB) :
%
%$\left\{\begin{array}{l c l}
%-1&=&2a + b\\
%3&=&0a + b
%\end{array}\right. \iff \left\{\begin{array}{l c l}
%-1&=&2a + 3\\
%3&=& b
%\end{array}\right. \iff \left\{\begin{array}{l c l}
%-4&=&2a\\
%3&=&b
%\end{array}\right. \iff \left\{\begin{array}{l c l}
%-2&=&a\\
%3&=&b
%\end{array}\right.$
%
%Une équation de la droite (AB) est donc $y = - 2x + 3$ ou $2x + y - 3 = 0$.
		\item On note H le pied la hauteur du triangle ABC issue du sommet C.

Déterminer les coordonnées du point H.
%		On a avec H$(x~;~y)$,  $\vect{\text{CH}}\binom{x - 3}{y - 1}$ donc 
%		
%$\left\{\begin{array}{l }
%\text{H} \in (\text{AB})\\
%\vect{\text{CH}} \cdot \vect{\text{AB}} = 0
%\end{array}\right. \iff
%\left\{\begin{array}{l c l}
%2x + y - 3&=&0\\
%- 2(x - 3) + 4(y - 1)&=&0
%\end{array}\right.\iff 
%\left\{\begin{array}{l c l}
%2x + y - 3&=&0\\
%- 2x + 6 + 4y - 4&=&0
%\end{array}\right. \iff
%\left\{\begin{array}{l c l}
%2x + y - 3&=&0\\
%- 2x + 4y + 2&=&0
%\end{array}\right. \Rightarrow 5y - 1 = 0 \iff y = \dfrac{1}{5}$, d'où en utilisant la première équation :
%
%$y = - 2x + 3 = - \dfrac{2}{5} + 3 = \dfrac{13}{5}$.
%
%Donc H$\left( \dfrac{1}{5}~;~\dfrac{13}{5}\right)$.
	\end{enumerate}
\end{enumerate}

\vspace{0,5cm}

