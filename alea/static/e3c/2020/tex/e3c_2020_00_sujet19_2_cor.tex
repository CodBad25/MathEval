	
	\subsection*{1.}
	
	On retranche 20 \%, donc on multiplie par \(1 - \dfrac{20}{100} = 1 - 0,2 = 0,8\).
	
	Donc \(10 \times 0,8 + 1 = 8 + 1 = 9 = U_1\).
	
	\subsection*{2.}
	
Le terme précédent de la suite est multiplié par 0,8 puis on ajoute  1.\\
On a donc quel que soit \(n \in \mathbb{N}\), \(u_{n+1} = 0,8 u_n + 1\).
	
	\subsection*{3.}
	
	On peut conjecturer que la limite de la suite est égale à 5.
	
	\subsection*{4.}
	
	Cet algorithme permet de trouver le rang \(N\) tel que \(u_N \leq 5,1\).
	
	\subsection*{5.}
	
	L’algorithme donnera \(N = 18\).
	
