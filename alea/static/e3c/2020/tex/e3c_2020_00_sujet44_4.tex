
\medskip

En 2012, un artisan batelier a transporté $300$ tonnes de marchandises sur sa péniche.

Il augmente sa cargaison chaque année de 11\,\% par rapport à l'année précédente.

\smallskip

On modélise alors la quantité en tonnes de marchandises transportées par l'artisan batelier par une suite $\left(u_n\right)$ où pour tout entier naturel $n$, $u_n$ est la quantité en tonnes de marchandises transportées en $(2012 + n)$. Ainsi $u_0 = 300$.

\medskip

\begin{enumerate}
\item 
	\begin{enumerate}
		\item Donner la nature de la suite $\left(u_n\right)$ et préciser sa raison. 
		\item Pour tout entier naturel $n$, exprimer $u_n$ en fonction de $n$.
	\end{enumerate}
\item Le batelier décide qu'à partir de \np{1000} tonnes transportées dans l'année, il achètera une péniche plus grande.
	\begin{enumerate}
		\item Recopier et compléter l'algorithme suivant, écrit en langage Python, afin de déterminer en quelle année il devra changer de péniche :
		
\begin{center}
\begin{tabular}{|l|}\hline
\texttt{u = 300}\\
\texttt{n = 0}\\
\texttt{while \ldots :}\\
\qquad \texttt{u = \ldots}\\
\qquad  \texttt{n = n+1}\\ \hline
\end{tabular}
\end{center}

		\item En quelle année changera-t-il de péniche ?
	\end{enumerate}
\item Une tonne transportée est payée au batelier $15$~\euro.

La proposition: \og Le chiffre d'affaires total entre 2012 et 2019 de l'artisan batelier sera
supérieur à \np{70000}~\euro{} \fg{} est-elle vraie ? Justifier la réponse.
\end{enumerate}

