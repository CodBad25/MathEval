
\medskip

On considère la fonction dérivable $f$ définie sur $\R$ par 

\[f(x) = 8x^3 - 6x^2 - 2.\]

On note $f'$ la fonction dérivée de la fonction $f$.

Soit $\mathcal{C}$ la courbe représentative de $f$ dans un plan muni d'un repère orthogonal.

\medskip

\begin{enumerate}
\item
\begin{enumerate}
\item Justifier que pour tout réel $x$, \[f(x) = (x - 1)\left( 8x^2 + 2x + 2\right).\]
\item En déduire que la courbe $\mathcal{C}$ coupe l'axe des abscisses en un seul point A dont on donnera
les coordonnées.
\end{enumerate}
\item 
	\begin{enumerate}
		\item  Justifier que pour tout réel $x$, $f'(x) = 12x(2x - 1)$.
		\item  En déduire le tableau de variations de la fonction $f$.
	\end{enumerate}
\item Le point B de coordonnées $\left(0~;~-\dfrac{5}{2}\right)$ appartient-il à la tangente T à la courbe $\mathcal{C}$ au point $\psCancel[cancelType=s,linewidth=0.07pt]{B}$
d'abscisse $x = \dfrac{1}{2}$ ? Justifier.
\end{enumerate}

\vspace{0,5cm}

