
\medskip

Cet exercice est un questionnaire à choix multiples (QCM). 

Les cinq questions sont indépendantes.

 Pour chacune des questions, une seule des quatre réponses est exacte.
 
Indiquer sur la copie le numéro de la question ainsi que la réponse choisie.

 Aucune justification n’est attendue.
 
Une réponse juste rapporte un point, une réponse fausse ou l’absence de réponse n’enlèvent pas de point.

\medskip

\textbf{Question 1} 

\medskip 

Dans un repère orthonormé, un vecteur normal à la droite d’équation $4x+5y-32=0$ est le vecteur :

\medskip

\begin{tabularx}{\linewidth}{*{4}{X}}
\textbf{a.~~} $\vv{v}\ \binom{-5}{4} $ &\textbf{b.~~} $\vv{v}\ \binom{-4}{5} $&\textbf{c.~~}$\vv{v}\ \binom{4}{5} $& \textbf{d.~~} $\vv{v}\ \binom{5}{4}  $.
\end{tabularx}

\medskip

\textbf{Question 2}

\medskip 

Dans un repère orthonormé, le projeté orthogonal du point A(7~;~9) sur la droite d’équation 

$4x+5y-32=0$ est le point :

\medskip

\begin{tabularx}{\linewidth}{*{4}{X}}
\textbf{a.~~} H(7~;~0,8)  &\textbf{b.~~} H(3~;~4)&\textbf{c.~~} H(4~;~3,2) & \textbf{d.~~}  H(4~;~5).
\end{tabularx}

\medskip

\textbf{Question 3}

\medskip 

Dans un repère orthonormé, une équation du cercle de centre A$(-1~;~3)$ et de rayon 2 est :

\medskip

\begin{tabularx}{\linewidth}{*{2}{X}}
\textbf{a.~~} $ x^2-1+y^2=2^2 $ &\textbf{b.~~} $x^2+2x+1+y^2-6y+9=2 $\\\textbf{c.~~}$(x+1)^2+(y-3)^2=2^2 $& \textbf{d.~~} $(x-1)^2+(y+3)^2=2^2  $.
\end{tabularx}

\medskip

\textbf{Question 4}

\medskip 

Dans un repère orthonormé, la parabole d’équation $y=3x^2-9x+5$ a pour sommet le point S et pour axe de symétrie la droite $\Delta$. Les coordonnées de S et l’équation de $\Delta$ sont :

\medskip
\begin{tabularx}{\linewidth}{*{2}{X}}
\textbf{a.~~} $ S\left(\dfrac{3}{2}~;~-\dfrac{7}{4}\right)$ et $\Delta : x=\dfrac{3}{2}$ &\textbf{b.~~} $S\left(\dfrac{3}{2}~;~-\dfrac{7}{4}\right)$ et $\Delta : y=-\dfrac{7}{4}$\\\textbf{c.~~}$ S(3~;~5) $ et $\Delta : x = 3$ & \textbf{d.~~} $S(3 ;5)$ et $\Delta : y = 5$.
\end{tabularx}

\medskip

\textbf{Question 5}

\medskip 

On considère l’inéquation $- 3x^2 + 9x-5 > 0$. L’ensemble $\mathscr{S}$ des solutions de cette inéquation est ($x_1$ et $x_2$ sont deux réels tels que $x_1< x_2 $ pour les propositions \textbf{b.} et \textbf{d.}) :

\medskip

\begin{tabularx}{\linewidth}{*{2}{X}}
\textbf{a.~~} $ \emptyset$ &\textbf{b.~~}de la forme $]-\infty~;~x_1 [ \cup ] x_2~;~+\infty [ $\\\textbf{c.~~}$\R $& \textbf{d.~~}de la forme $  ] x_1~;~x_2 [  $.
\end{tabularx}

\vspace{0.5cm}

