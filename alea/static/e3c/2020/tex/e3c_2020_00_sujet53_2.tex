
\medskip

Une entreprise fabrique des jeux en bois. Avant sa commercialisation, chaque jeu est soumis
à deux contrôles : un contrôle de peinture et un contrôle de solidité.

Après un très grand nombre de vérifications, on constate que :

\begin{itemize}
\item  8\,\% des jeux ont un défaut de peinture,
\item parmi les jeux qui n'ont pas de défaut de peinture, 5\,\% ont un défaut de solidité,
\item 2\,\% des jeux présentent les deux défauts.
\end{itemize}
On choisit au hasard un jeu parmi ceux fabriqués par l'entreprise. On note :
\begin{itemize}
\item $T$ l'évènement : \og le jeu a un défaut de peinture. \fg
\item $S$ l'évènement : \og le jeu a un défaut de solidité. \fg
\end{itemize}

\medskip

\begin{enumerate}
\item Démontrer que $p_T(S) = 0,25$.
\item Recopier et compléter l'arbre pondéré de probabilité ci-dessous traduisant les données de
l'énoncé.

\begin{center}
\psset{nodesepA=0pt,nodesepB=3pt,treesep=0.75,labelsep=0.1pt,levelsep=2.5cm}
\pstree[treemode=R]{\TR{}}
{\pstree{\TR{$T$~~}\taput{}}
	{
	\TR{$S$}\taput{$\np{0.25}$}
	\TR{$\overline{S}$}\tbput{}
	}
\pstree{\TR{$\overline{T}$~~}\tbput{}}
	{\TR{$S$}\taput{}
	\TR{$\overline{S}$}\tbput{}
	}
}
\end{center}

\item Démontrer que la probabilité que le jeu choisi au hasard n'ait pas de défaut de solidité est
égale $0,934$.
\item Les jeux qui présentent un défaut de solidité sont détruits. Dans cette question, on leur
attribuera un prix de vente de 0 \euro.

Les jeux ne présentant aucun défaut sont vendus $14$ \euro{} chacun.

Les autres jeux sont vendus 9 \euro{} chacun.

On note $X$ la variable aléatoire qui donne le prix de vente, en euros, d'un jeu.

	\begin{enumerate}
		\item  Recopier et compléter le tableau ci-dessous donnant, pour chaque valeur $x_i$ de $X$,
la probabilité de l'évènement $ \{X=x_i\}$.

\begin{center}
\begin{tabularx}{0.5\linewidth}{|m{1.4cm}|*{3}{>{\centering \arraybackslash}X|}}
\hline
\centering $x_i$& 0 &9	& 14\\\hline
$P(X=x_i)$		&	&	&\\\hline
\end{tabularx}
\end{center}
	\item Quel est le prix de vente moyen d'un jeu fabriqué par cette entreprise ?

\emph{On arrondira le résultat au centime d'euro.}
	\end{enumerate}
\end{enumerate}

\vspace{0,5cm}

