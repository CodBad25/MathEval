
\medskip

On s'intéresse à la consommation d'essence d'un véhicule en fonction de sa vitesse.

\medskip

\textbf{Lecture graphique}

Le graphique ci-dessous représente la consommation d'essence en litres pour $100$~km en fonction de la vitesse en km.h$^{-1}$ du véhicule.

\begin{center}
\psset{xunit=0.08cm,yunit=0.4cm}
\begin{pspicture}(-5,-1)(135,16)
\multido{\n=0+5}{28}{\psline[linestyle=dotted](\n,0)(\n,14)}
\multido{\n=0+1}{15}{\psline[linestyle=dotted](0,\n)(135,\n)}
\psaxes[linewidth=1.25pt,Dx=10,Dy=2]{->}(0,0)(0,0)(135,15)
\uput[r](0,15){consommation en litres pour 100 km}
\uput[u](120,0){vitesse en km.h$^{-1}$}
%\pscurve[linewidth=1.25pt,linecolor=blue](30,11)(35,7)(40,5)(45,4.1)(50,4)(55,4.1)(60,4.4)(65,4.85)(70,5.15)(75,5.85)(80,6.1)(85,6.4)(90,7.06)(95,7.6)(100,8)(105,8.3)(110,8.9)(115,9)(120,9.25)(125,9.85)(130,10)
\psplot[plotpoints=2000,linewidth=1.25pt,linecolor=red]{30}{130}{x dup mul 20 mul 1600 x mul sub 40000 add x dup mul div}
\end{pspicture}
\end{center}

Avec la précision permise par le graphique, répondre aux questions suivantes:

\medskip

\begin{enumerate}
\item Quelle est la consommation du véhicule lorsque celui-ci roule à 40 km.h$^{-1}$ ?
\item Pour quelle(s) vitesse(s) le véhicule consomme-t-il 8 litres pour 100 km ?
\item Pour quelle vitesse la consommation du véhicule semble-t-elle minimale ?
\end{enumerate}

\textbf{Modélisation}

\medskip

Si on note $x$  la vitesse du véhicule en km.h$^{-1}$, avec $30 \leqslant x \leqslant 130$, la consommation d'essence en litres pour 100 km est modélisée par la fonction $f$ d'expression :

\[f(x) = \dfrac{20x^2  - \np{1600}x + \np{40000}}{x^2}.\]

On désigne par $f'$ la fonction dérivée de la fonction $f$ sur l'intervalle [30~;~130].

\begin{enumerate}[resume]
\item Montrer que pour tout $x \in [30~;~130],$

\[f'(x) = \dfrac{800\left (2x - 100\right )}{x^3}.\]

\item Démontrer la conjoncture de la question 3.
\end{enumerate}

\bigskip

