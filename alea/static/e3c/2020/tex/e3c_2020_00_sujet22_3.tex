
\medskip

Dans un repère orthonormé on considère le point $A(-3~;~5)$ et la droite $(d)$ dont une équation cartésienne est $-x+3y+2=0$.

\medskip

\begin{enumerate}
\item Tracer la droite $(d)$ dans le repère suivant.

\begin{center}
\psset{labelFontSize=\scriptstyle,showorigin=false}
\begin{pspicture}(-6,-6)(6,6)
\multido{\n=-5+1}{11}{\psline[linewidth=0.75pt,linecolor=lightgray](\n,-5.2)(\n,5.2)}
\multido{\n=-5+1}{11}{\psline[linewidth=0.75pt,linecolor=lightgray](-5.2,\n)(5.3,\n)}
\psaxes[linewidth=0.95pt,]{->}(0,0)(-5.3,-5.3)(5.3,5.3)
\uput[dl](0,0){O}

\end{pspicture}
\end{center}

\item Déterminer les coordonnées d’un vecteur normal à la droite $(d)$.
\item Déterminer une équation cartésienne de la droite perpendiculaire à $(d)$ et passant par A.
\item En déduire que le point H, projeté orthogonal de A sur la droite $(d)$, a pour coordonnées $(-1;~~-1)$.
\item En déduire la distance entre le point A et la droite $(d)$.
\end{enumerate}

\vspace{0,5cm}

