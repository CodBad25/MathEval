
\medskip

Ce QCM comprend $5$ questions indépendantes. Pour chacune d'elles, une seule des réponses proposées est exacte.

Indiquer pour chaque question sur la copie la lettre correspondant à la réponse choisie. Aucune justification n'est demandée.

Chaque réponse correcte rapporte $1$ point. Une réponse incorrecte ou une absence de réponse n'apporte ni ne retire de point.

\medskip

\textbf{Question 1}

\medskip

L'inéquation $x^2 +x + 2 > 0$ :

\begin{center}
\begin{tabularx}{\linewidth}{|*{4}{X|}}\hline
\textbf{a.~~}n'a pas de solution&\textbf{b.~~}a une seule solution&\textbf{c.~~}a pour ensemble de solutions l'intervalle [1~;~2]&\textbf{d.~~}a pour solution l'ensemble des
 nombres réels\\ \hline
\end{tabularx}
\end{center}

\textbf{Question 2}

\medskip

Soient $\vect{u}$ et $\vect{v}$ deux vecteurs tels que $\|\vect{u}\|= 3,\,\|\vect{v}\|= 2$ et $\vect{u} \cdot \vect{v} = - 1$ alors $\|\vect{u} + \vect{v}\|^2$ est égal à:

\begin{center}
\begin{tabularx}{\linewidth}{|*{4}{X|}}\hline
\textbf{a.~~}11&\textbf{b.~~}13&\textbf{c.~~}15&\textbf{d.~~}25\\ \hline
\end{tabularx}
\end{center}

\textbf{Question 3}

\medskip

Soient $A$ et $B$ deux évènements d'un univers tels que $P_A(B) = 0,2$ et $P(A) = 0,5$.

 Alors la probabilité $P(A \cap B)$ est égale à :
 
\begin{center}
\begin{tabularx}{\linewidth}{|*{4}{X|}}\hline
\textbf{a.~~}0,4&\textbf{b.~~}0,1&\textbf{c.~~}0,25&\textbf{d.~~}0,7\\ \hline
\end{tabularx}
\end{center} 

\textbf{Question 4}

\medskip

Soit $\left(u_n\right)$ une suite arithmétique de terme initial $u_0 = 2$ et de raison $3$.

La somme S définie par $S = u_0 + u_1 +\ldots + u_{12}$ est égale à :

\begin{center}
\begin{tabularx}{\linewidth}{|*{4}{X|}}\hline
\textbf{a.~~}45&\textbf{b.~~}222&\textbf{c.~~}260&\textbf{d.~~}301\\ \hline
\end{tabularx}
\end{center} 

\textbf{Question 5}

\medskip

Soit $f$ la fonction définie sur l'ensemble des nombres réels par $f(x) = (2x - 5)^3$.

Une expression de la dérivée de $f$ est :

\begin{center}
\begin{tabularx}{\linewidth}{|*{4}{X|}}\hline
\textbf{a.~~}$3(2x- 5)^2$&\textbf{b.~~}$6(2x- 5)^2$&\textbf{c.~~}$2(2x- 5)^2$&\textbf{d.~~}$2^3$\\ \hline
\end{tabularx}
\end{center} 

\medskip

