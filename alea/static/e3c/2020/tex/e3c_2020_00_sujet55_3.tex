 
\medskip

Le plan est muni d'un repère orthonormé \Oij.

On considère les points A, B et C de coordonnées : A$(7~;~-2)$, B (7~;~4) et C (1~;~1).
\begin{enumerate}
\item Montrer que $y=1$ est une équation de la droite $(d_1)$ passant par C et perpendiculaire à (AB).
\item Que représente cette droite pour le triangle ABC ?
\item Donner une équation de la droite $(d_2)$, hauteur du triangle ABC issue du sommet B.
\item On appelle H le point d'intersection des droites $(d_1)$ et $(d_2)$.
Donner en justifiant la valeur du produit scalaire : $\vv{AH}\cdot \vv{CB}$.
\end{enumerate}


\begin{center}
\psset{xunit=0.5cm,yunit=0.5cm,labelsep=0.1pt,labelFontSize=\scriptstyle,showorigin=false}
\begin{pspicture}(-2,-4.5)(11,8)
 \multido{\n=-1+1}{13}{\psline[linewidth=0.75pt,linecolor=lightgray](\n,-4)(\n,7.5)}
 \multido{\n=-4+1}{12}{\psline[linewidth=0.75pt,linecolor=lightgray](-1.5,\n)(11.2,\n)}
 \psaxes[linewidth=0.95pt]{->}(0,0)(-1.5,-4)(11,7.5)
\uput[dl](0,-0.2){\footnotesize O}\uput[l](6.8,-2){A}\uput[l](6.8,4){B}\uput[l](0.9,0.9){C}
\psdots[dotstyle=+,dotscale =1.4,dotangle=45](1,1)(7,4)(7,-2)
\end{pspicture}
\end{center}

\vspace{0.75cm}

