
\medskip

Une étude statistique menée lors des entraînements montre que, pour un tir au but, Karim marque avec une probabilité de $0,7$.

Karim effectue une série de 3 tirs au but. Les deux issues possibles après chaque tir sont les évènements:

\setlength\parindent{1cm}
\begin{itemize}
\item[$\bullet~~$]$M$: \og Karim marque un but \fg{} ;
\item[$\bullet~~$]$R$: \og Karim rate le tir au but \fg.
\end{itemize}
\setlength\parindent{0cm}

On admet que les tirs au but de Karim sont indépendants.

\medskip

\begin{enumerate}
\item On note $X$ la variable aléatoire qui prend pour valeur le nombre total de buts marqués à l'issue de cette série de tirs par Karim.
	\begin{enumerate}
		\item Réaliser un arbre pondéré permettant de décrire toutes les issues possibles.	
		\item Déterminer la loi de probabilité de $X$.
		\item Calculer l'espérance $E(X)$ de la variable aléatoire $X$.
	\end{enumerate}
\item  On propose à un spectateur le jeu suivant: il mise $15$~\euro{} avant la série de tirs au but de Karim ; chaque but marqué par Karim lui rapporte $6$~\euro, et chaque but manqué par Karim ne lui rapporte rien.

On note $Y$ la variable aléatoire qui prend pour valeur le gain algébrique du spectateur, c'est- à-dire la différence entre le gain total obtenu et la mise engagée.
	\begin{enumerate}
		\item Exprimer $Y$ en fonction de $X$.
		\item Calculer l'espérance $E(Y)$ de la variable aléatoire $Y$. 

Interpréter ce résultat dans le contexte de l'énoncé.
	\end{enumerate}
\end{enumerate}
