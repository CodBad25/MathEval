
\medskip

Une enquête a été réalisée auprès des élèves d'un lycée afin de connaître leur point de vue sur la durée de la pause méridienne et sur les rythmes scolaires.

L'enquête révèle que 55\,\% des élèves sont favorables à une pause méridienne plus longue.

Parmi ceux qui sont favorables à une pause méridienne plus longue, 95\,\% souhaitent une répartition des cours plus étalée sur l'année scolaire.

Parmi ceux qui ne sont pas favorables à une pause méridienne plus longue, seulement 10\,\% souhaitent une répartition des cours plus étalée sur l'année scolaire.

\smallskip

On tire au hasard le nom d'un élève du lycée. On considère les évènements suivants:

\smallskip

$\bullet~~$ $L$: \og L'élève concerné est favorable à une pause méridienne plus longue. \fg

$\bullet~~$ $C$ : \og L'élève concerné souhaite une répartition des cours plus étalée sur l'année scolaire. \fg.

\medskip

\begin{enumerate}
\item Recopier et compléter l'arbre pondéré ci-dessous décrivant la situation.

\begin{center}
\pstree[treemode=R,nodesepA=0pt,nodesepB=3pt]{\TR{}}
{\pstree{\TR{$L$~~}\naput{\ldots}}
	{\TR{$C$}\naput{\ldots}
	\TR{$\overline{C}$}\nbput{\ldots}
	}
\pstree{\TR{$\overline{L}$~~}\nbput{\ldots}}
	{\TR{$C$}\naput{\ldots}
	\TR{$\overline{C}$}\nbput{\ldots}
	}
}
\end{center}

\item Calculer la probabilité que l'élève concerné soit favorable à une pause méridienne plus longue et souhaite une répartition des cours plus étalée sur l'année scolaire.
\item Montrer que $P(C) = \np{0,5675}$.
\item Calculer la probabilité que l'élève concerné soit favorable à une pause méridienne plus longue sachant qu'il souhaite une répartition des cours plus étalée sur l'année scolaire.

En donner une valeur arrondie à $10^{-4}$.
\item Les évènements $L$ et $C$ sont-ils indépendants ? Justifier la réponse.
\end{enumerate}

\bigskip

