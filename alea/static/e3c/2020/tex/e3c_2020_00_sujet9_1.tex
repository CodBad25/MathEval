
\medskip

Ce QCM comprend 5 questions indépendantes. Pour chacune d'elles, une seule des affirmations proposées est exacte.

Indiquer pour chaque question sur la copie la lettre correspondant à la réponse choisie. Aucune justification n'est demandée.

Chaque réponse correcte rapporte $1$ point. Une réponse incorrecte ou une absence de réponse n'apporte ni ne retire de point.

\medskip

\textbf{Question 1}

On choisit au hasard un individu parmi les passagers en transit dans un aéroport. On a représenté ci-dessous un arbre de probabilités lié à certains évènements dont certains éléments ont été effacés.

On considère les évènements suivants:
\begin{itemize}
\item $A$ : \og le passager parle anglais \fg
\item $B$ : \og le passager ne parle pas anglais \fg
\item $E$ : \og le passager est un membre de l'Union Européenne \fg
\end{itemize}

\begin{center}
\pstree[treemode=R,nodesepA=0pt,nodesepB=3pt]{\TR{}}
{\pstree{\TR{$A$~~}\naput{0,6}}
	{\TR{$E$}
	\TR{$\overline{E}$}\tbput{0,5}
	}
\pstree{\TR{$B$~~}}
	{\TR{$E$}\taput{0,3}
	\TR{$\overline{E}$}
	}
}
\end{center}
\begin{center}
\begin{tabularx}{\linewidth}{|*{4}{X|}}\hline
\textbf{a.~~}$P_B(E) = 0,12$&\textbf{b.~~}$p(E) = 0,42$&\textbf{c.~~}\small La probabilité que le passager choisi soit européen et ne parle pas anglais est 0,3&\textbf{d.~~}$P(A \cup B) = 1,1$\\ \hline
\end{tabularx}
\end{center}

\medskip

\textbf{Question 2}

Le plan est muni d'un repère orthonormé. Soit $D$ la droite d'équation $3x + y - 2 = 0$.

\begin{center}
\begin{tabularx}{\linewidth}{|*{4}{X|}}\hline
\textbf{a.~~}Le point de coordonnées $(6~;~-15)$ appartient à $D$&\textbf{b.~~}$D$ est perpendiculaire à la droite d'équation $12x + 4y = 0$&\textbf{c.~~}Le vecteur de coordonnées
(1~;~3) est un vecteur directeur de $D$.&\textbf{d.~~}\small Le vecteur de coordonnées
(3~;~1) est un vecteur directeur des droites perpendiculaires à $D$.\\ \hline
\end{tabularx}
\end{center}

\medskip

\textbf{Question 3} On considère dans l'ensemble des réels l'équation trigonométrique $\sin x = 1$.

\begin{center}
\begin{tabularx}{\linewidth}{|*{4}{X|}}\hline
\textbf{a.~~}Cette équation admet une unique solution dans l'ensemble des réels&\textbf{b.~~}Cette équation admet une infinité de solutions dans l'ensemble des réels&\textbf{c.~~}$2\pi$ est une solution de cette équation&\textbf{d.~~}$- \dfrac{57\pi\rule{0pt}{11pt}}{2\rule[-3pt]{0pt}{0pt}}$ est une solution de cette équation\\ \hline
\end{tabularx}
\end{center}

\medskip

\textbf{Question 4}

Soit $f$ la fonction définie sur l'ensemble des nombres réels par $f(x) = \dfrac{2x}{x^2 + 1}$
et $\mathcal{C}$ sa courbe représentative dans un repère du plan.

\begin{center}
\begin{tabularx}{\linewidth}{|*{4}{X|}}\hline
\textbf{a.~~} La courbe $\mathcal{C}$ n'admet pas de tangente au point d'abscisse $0$&\textbf{b.~~}La tangente à $\mathcal{C}$ au point d'abscisse $0$ pour équation $y = 2x$&\textbf{c.~~} La tangente à $\mathcal{C}$ au point d'abscisse $0$ a pour coefficient directeur 1&\textbf{d.~~}La tangente à $\mathcal{C}$ au point d'abscisse $0$ est parallèle à l'axe des abscisses\\ \hline
\end{tabularx}
\end{center}

\medskip

\textbf{Question 5}

Soit la fonction $f$ définie sur l'intervalle $]-2~;~ +\infty[$ par:

\[f(x) = \dfrac{x - 3}{x+2}\]

$f$ est dérivable sur l'intervalle $]-2~;~ +\infty[$ et pour tout réel $x$ de $]-2~;~ +\infty[$, on a :

\begin{center}
\begin{tabularx}{\linewidth}{|*{4}{X|}}\hline
\textbf{a.~~}$f'(x) = 1$&\textbf{b.~~}$f'(x) = \dfrac{2x - 1}{(x + 2)^2\rule[-5pt]{0pt}{0pt}}$&\textbf{c.~~}$f'(x) = \dfrac{5}{(x + 2)^2}$&\textbf{d.~~}$f'(x) = 2x - 1$\rule[-3mm]{0mm}{8mm}\\ \hline
\end{tabularx}
\end{center}

\bigskip

