
\subsection*{1.}

Augmenter de 5 \% c'est multiplier par \(1 + \dfrac{5}{100} = 1 + 0{,}05 = 1{,}05\).

\begin{itemize}
    \item \(u_1 = u_0 \times 1{,}05 = 416 \times 1{,}05 = 436{,}8 \approx 437\).
    \item \(u_2 = u_1 \times 1{,}05 = 436{,}8 \times 1{,}05 = 458{,}64 \approx 459\).
\end{itemize}

\subsection*{2.}

Quel que soit le naturel \(n\), \(u_{n+1} = u_n \times 1{,}05\) : cette égalité montre que la suite \((u_n)\) est une suite géométrique de raison \(q = 1{,}05\) et de premier terme \(u_0 = 416\).

\subsection*{3.}

On sait qu'alors, quel que soit le naturel \(n\), \(u_n = u_0 \times 1{,}05^n = 416 \times 1{,}05^n\).

\subsection*{4.}

On a donc en particulier : \(u_7 = 416 \times 1{,}05^7 \approx 585{,}4 \approx 585\).

\subsection*{5.}

Avec la calculatrice : on saisit \(416\) + Entrée puis \(\times 1{,}05\) + Entrée, séquence que l'on répète jusqu'à ce que l'on obtienne un nombre supérieur à 700 : il faut répéter la séquence 11 fois pour obtenir environ 711{,}5.

Le nombre d'adhérents dépassera les 700 adhérents en suivant ce modèle en 2018 + 11 = 2029.

