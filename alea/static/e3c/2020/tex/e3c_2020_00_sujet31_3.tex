
\medskip

On injecte dans le sang d'un malade $\np[cm^3]{2}$  d'un médicament. On admet que le processus
d'élimination du médicament peut être modélisé par une suite $\left(U_n\right)$, dont le terme général $U_n$ représente le volume en cm$^3$ de médicament présent dans le sang au bout de $n$ heures,
$n$ étant un entier naturel. Dans ce modèle, on considère que le volume de médicament
contenu dans le sang diminue de 8\,\% chaque heure.

\medskip

\begin{enumerate}
\item Vérifier que $U_1 = 1,84$ et en donner une interprétation dans le contexte de l'exercice.
\item 
	\begin{enumerate}
		\item Pour tout entier naturel $n$ , exprimer $U_{n+1}$ en fonction de $U_n$.

		\item En déduire la nature de la suite $\left(U_n\right)$. Préciser sa raison et son premier terme.

	\end{enumerate}
\item Pour que le médicament soit actif, le volume de médicament présent dans le sang du malade doit rester supérieur à un certain seuil $S$ ; ce seuil dépend du malade.
	\begin{enumerate}
		\item À l'aide d'une fonction écrite en langage Python, on se propose de déterminer, en fonction de $S$, le nombre maximal d'heures durant lesquelles le médicament reste actif.
Compléter le programme suivant, écrit en Python.

\begin{center}
\begin{tabular}{|l|}
\hline
def volMedicament(S) :\\
\hspace{2em}u=2\\
\hspace{2em}n=0\\
\hspace{2em}while u > $S$ :\\
\hspace{2em}	u = u*$\dots$\\
\hspace{2em}	n = n+1\\
return n\\\hline
\end{tabular}
\end{center}

		\item On s'intéresse au cas d'un malade pour qui ce seuil est estimé à $S = \np[cm^3]{1,5}$.
Que doit-on saisir pour exécuter la fonction volMedicament afin qu'elle renvoie 
le nombre maximal d'heures durant lesquelles le médicament reste actif chez ce malade ?


Quel est alors ce nombre d'heures ?

	\end{enumerate}
\end{enumerate}

\vspace{0,5cm}

