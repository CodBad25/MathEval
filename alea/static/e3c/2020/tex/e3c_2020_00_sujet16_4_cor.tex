	

	\subsection*{1.}
	a.	
	\[
	\begin{array}{|c|c|c|c|c|c|}
		\hline
		C & 300 & 326 & 350 & 372 & 392 \\
		\hline
		\text{C < 400 ?} & \text{oui} & \text{oui} & \text{oui} & \text{oui} & \text{oui} \\
		\hline
	\end{array}
	\]
	
	b. Le programme s’arrête avec \(N = 5\). Ceci signifie que le nombre de colonies dépassera 400 au cours de la 5e année.
	
	\[
	C_{n+1} = 0,92 C_n + 50.
	\]
	
	\subsection*{2.}
	

		On a $C_0 = 300$, \quad $C_1 = 326$, \quad $C_2 = 349,92$. \\
Donc  $C_1 - C_0 = 26$  et $C_2 - C_1 = 23,92$ :\\
 la différence entre deux termes consécutifs de la suite n’est pas constante : la suite n’est pas arithmétique. \\
		$\dfrac{C_1}{C_0} \approx 1,087$  et $\dfrac{C_2}{C_1} \approx 1,073$ :\\
		 le quotient de termes consécutifs de la suite n’est pas constant : la suite n’est pas géométrique.

	
	\subsection*{3.}
	
	On admet que \(C_n = 625 - 325 \times 0,92^n\) pour tout entier naturel \(n\).\\
	On sait que, comme \(0 < 0,92 < 1\), alors \(\lim\limits{n \to +\infty} 0,92^n = 0\)\\
	 et on a aussi $\lim\limits_{n \to +\infty} 325 \times 0,92^n = 0$ \\
	 et par conséquent $\lim\limits_{n \to +\infty} C_n = 625$.\\
	 On part de 300 et on peut augmenter jusqu’à 625 colonies :\\ on ne peut donc atteindre, si cette modélisation est correcte, 700 colonies.
	
