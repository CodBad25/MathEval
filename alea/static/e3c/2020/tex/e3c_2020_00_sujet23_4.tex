
\medskip

Un jeu vidéo fait évoluer un personnage sur un parcours semé d’obstacles.

Au début du parcours, ce personnage est doté de \np{1000} pions noirs dans son sac et il
n’a pas de pion blanc.

Le nombre de pions noirs diminue au cours du jeu.

Le personnage gagne 10 pions blancs par minute jouée.

Chaque partie est chronométrée et dure 45 minutes. Au bout des 45 minutes, la
partie s’arrête et le joueur a gagné si le nombre de pions blancs gagnés est supérieur
ou égal au nombre de pions noirs du sac.

\medskip

\begin{enumerate}
\item Étude de l’évolution du nombre de pions blancs

On note $u_n$ le nombre de pions blancs obtenus au bout de $n$ minutes de jeu.

Ainsi $u_0 = 0$.

Déterminer la nature de la suite $\left(u_n\right)$ et en déduire, pour tout entier $n$, l’expression
de $u_n$ en fonction de $n$.
\item Étude de l’évolution du nombre de pions noirs

Lucas estime qu’au cours d’une partie, le nombre de ses pions noirs diminue de 2\,\%
par minute. Il voudrait savoir si cette évolution est suffisante pour gagner, ou s’il doit
poursuivre son entraînement.

On note $v_n$ le nombre de pions noirs restant à la $n$-ième minute.

Ainsi $v_0 = \np{1000}$.
\begin{enumerate}
\item Justifier que $v_1 = 980$.
\item Déterminer la nature de la suite $\left(v_n\right)$ et en déduire, pour tout entier $n$,
l’expression de $v_n$ en fonction de $n$.
\end{enumerate}
\item On a calculé les premiers termes des suites $\left(u_n\right)$ et $\left(v_n\right)$ à l’aide d’un tableur. La feuille de calcul est donnée ci-dessous.
Les termes de la suite $\left(v_n\right)$ ont été arrondis à l’unité.
Lucas peut-il gagner la partie ?

\begin{tabularx}{0.5\linewidth}{|>{\columncolor{lightgray}}m{0.35cm}|*{3}{>{\centering \arraybackslash}X|}}
\hline
\rowcolor{lightgray}&A&B&C\\\hline
1&$n$&$u_n$&$v_n$\\\hline
2&0&0&\np{1000}\\\hline
3&1&10&980\\\hline
4&2&20&960\\\hline
5&3&30&941\\\hline
6&4&40&922\\\hline
7&5&50&904\\\hline
8&6&60&886\\\hline
9&7&70&868\\\hline
10&8&80&851\\\hline
\multicolumn{4}{|c|}{\dotfill}\\\hline
\multicolumn{4}{|c|}{\dotfill}\\\hline
\multicolumn{4}{|c|}{\dotfill}\\\hline
41&39&390&455\\\hline
42&40&400&446\\\hline
43&41&410&437\\\hline
44&42&420&428\\\hline
45&43&430&419\\\hline
46&44&440&411\\\hline
47&45&450&403\\\hline
48&46&460&395\\\hline
49&47&470&387\\\hline
50&48&480&379\\\hline
51&49&490&372\\\hline
52&50&500&364\\\hline
\end{tabularx}
\end{enumerate}
