	
	\subsection*{Question 1}
	
	On a \(u_{n+1} = u_n - \dfrac{13}{100} u_n = u_n \left(1 - \dfrac{13}{100}\right) = u_n (1 - 0,13) = 0,87 u_n\).
	
	La relation \(u_{n+1} = 0,87 u_n\) montre que la suite \((u_n)\) est une suite géométrique de raison 0,87 et de premier terme \(u_0 = 100\).
	
	\subsection*{Question 2}
	
	La loi de probabilité incomplète de la variable aléatoire \(X\) est donnée ci-dessous :
	
	\[
	E(X) = 0,2 \times (-6) + 0,1 \times (-3) + 0,2 \times 0 + 0,4 \times 3 + 0,1 \times x_5 = 0,7.
	\]
	
	Soit \(-1,2 - 0,3 + 1,2 + 0,1 \times x_5 = 0,7\),\\ d’où \(0,1 \times x_5 = 1\) et \(x_5 = 10\).
	
	\subsection*{Question 3}
	
	Soit \(f\) la fonction dérivable définie sur \(\left] -\dfrac{7}{3}, +\infty \right[\) par \(f(x) = \dfrac{2x + 3}{3x + 7}\) et \(f'\) sa fonction dérivée.
	
	Comme \(x \neq -\dfrac{7}{3}\), \(f(x)\) existe et est dérivable en tant que quotient de fonctions dérivables sur \(\left] -\dfrac{7}{3}, +\infty \right[\) :
	
	\[
	f'(x) = \dfrac{2(3x + 7) - 3(2x + 3)}{(3x + 7)^2} = \dfrac{6x + 14 - 6x - 9}{(3x + 7)^2} = \dfrac{5}{(3x + 7)^2}.
	\]
	
	\subsection*{Question 4}
	
	Soit \(a\) le prix initial de l’article. L’augmenter de 10 \% c’est le multiplier par 1,10.
	
	On a donc \(a \times 1,10 \times b = a\), soit en supposant le prix non nul \(1,10b = 1\),\\ d’où \(b = \dfrac{1}{1,10} \approx 0,909\).
	
	Or multiplier un prix par 0,909 c’est le baisser de \(1 - 0,909 = 0,091\) soit environ 9 \%.
	
	\subsection*{Question 5}
	
	Réponse : C.
	
