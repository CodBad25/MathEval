  
\medskip

Le dépistage d'une maladie particulière que l'on appelle M s'effectue par un test basé sur le dosage d'une hormone particulière. D'après une étude, cette maladie M touche 1,5\,\% de la population.

Si une personne est atteinte par la maladie M, le test sera positif dans 95\,\% des cas ; alors que si la personne n'est pas atteinte par la maladie M, le test sera négatif dans 99\,\% des cas.

On soumet à ce test une personne prise au hasard dans la population.

On note :

\begin{itemize}
\item $A$ l'évènement \og La personne est atteinte par la maladie M.\fg{} ;
\item $T$ l'évènement \og Le test est positif\fg.
\end{itemize}
\begin{enumerate}
\item Déterminer la probabilité pour que le test soit positif et que la personne choisie ne soit pas malade.
\item Déterminer la probabilité pour que le test soit positif.
\item Calculer $P_T\left(\overline{A}\right)$. (Arrondir à $10^{-3}$ près). Interpréter ce résultat dans le contexte de l'exercice.
\end{enumerate}

\vspace{0,5cm}

