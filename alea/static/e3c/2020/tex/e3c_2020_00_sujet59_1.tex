  
\medskip

\emph{Ce QCM comprend 5 questions indépendantes.}

\emph{Pour chacune d'elles, une seule des réponses proposées est exacte.}
 
\emph{Indiquer pour chaque question sur la copie la lettre correspondant à la réponse
choisie. Aucune justification n'est demandée.}

\emph{Chaque réponse correcte rapporte $1$ point. Une réponse incorrecte ou une absence de
réponse n'apporte, ni ne retire de point.}

\medskip

 \textbf{Question 1}

 \medskip

 Soit la suite arithmétique $\left(u_n\right)$ de premier terme $u_0 = 2$ et de raison $0,9$. On a :
\begin{center}
\begin{tabularx}{\linewidth}{*{4}{X}}
\textbf{a.~~} $u_{50}= 47$ &\textbf{b.~~} $u_{50}= 100,9$&\textbf{c.~~}$u_{50}= -47$& \textbf{d.~~} $u_{50}= -100,9$.
\end{tabularx}
\end{center}

\medskip

 \textbf{Question 2}

\medskip

Soit la suite géométrique $\left(v_n\right)$ de premier terme $v_0 = 2$ et de raison 0,9.
La somme des 37 premiers termes de la suite $\left(v_n\right)$ est :

\medskip

\begin{tabularx}{\linewidth}{*{4}{X}}
\textbf{a.~~} $2 \times \dfrac{1-0,9^{38}}{1-0,9}$ &\textbf{b.~~} $2\times \frac{1-0,9^{37}}{1- 0,9}$&\textbf{c.~~}$ 0,9\times \dfrac{1-2^{38}}{1-2}$& \textbf{d.~~} $0,9 \times \dfrac{1-2^{37}}{1-2}$.
\end{tabularx}

\medskip

\textbf{Question 3}

 \medskip
 Un programme en langage Python qui retourne la somme des entiers de 1 à 100 est :
 
 \medskip

\begin{center}
 \begin{tabularx}{\linewidth}{*{4}{X}}
 \begin{minipage}[]{3cm}\textbf{a.~~}Def Somme() :\\s=0 \\While s<100\\ \phantom{defdefw}s=s+1 \\return (s)\end{minipage}  &
\begin{minipage}[]{3cm} \textbf{b.~~} Def Somme() :\\s=0\\ While s<100 \\\phantom{defdefw}s=2*s+1\\ return (s)\end{minipage} & 
\begin{minipage}[]{3cm}\textbf{c.~~} Def Somme() :\\ s=0\\ for k in range 101\\\phantom{defdefw} s=s+k \\return(s)\end{minipage}& 
\begin{minipage}[]{3cm}\textbf{d.~~}Def Somme() : \\s=0\\ for k in range 100\\\phantom{defdefw} s=s+k \\return(s)\end{minipage} .
\end{tabularx}
\end{center}

\medskip

\textbf{Question 4}

 \medskip


 On a $ x\in \left[-\dfrac{\pi}{2}~;~0\right]$ et $\cos x=0,8$ alors :

\begin{center}
\begin{tabularx}{\linewidth}{*{4}{X}}
\textbf{a.~~} $\sin x=0,6$ &\textbf{b.~~} $\sin x=-0,6$&\textbf{c.~~}$\sin x=-0,2$& \textbf{d.~~} $ \sin x=0,2$.
\end{tabularx}
\end{center}

\medskip

\textbf{Question 5}

\medskip

Le nombre réel $\frac{13 \pi}{4}$ est associé au même point du cercle trigonométrique que le réel :

\begin{center}
\begin{tabularx}{\linewidth}{*{4}{X}}
\textbf{a.~~} $-\dfrac{14 \pi}{4}$ &\textbf{b.~~}$-\dfrac{3\pi}{4}$ &\textbf{c.~~}$\dfrac{7 \pi}{4}$& \textbf{d.~~}$\dfrac{19\pi}{4}$ .
\end{tabularx}
\end{center}

\vspace{0,5cm}

