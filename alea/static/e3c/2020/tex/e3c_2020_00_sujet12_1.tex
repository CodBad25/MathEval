
\medskip

Cet exercice est un questionnaire à choix multiple (QCM). Pour chaque question, une seule des quatre réponses proposées est exacte. Une bonne réponse rapporte un point. Une mauvaise réponse, une réponse multiple ou l'absence de réponse ne rapporte ni n'enlève aucun point.

Relevez sur votre copie le numéro de la question ainsi que la lettre correspondant à la réponse choisie.Aucune justification n'est demandée.

\bigskip

\textbf{Question 1}

\medskip

Quelle est la forme factorisée de $f(x) = 0,5(x - 2)^2 - 8$ ?

\begin{center}
\begin{tabularx}{\linewidth}{|*{2}{X|}}\hline
\textbf{a.~~}$0,5x^2 - 2x - 6$&\textbf{b.~~}$0,5(x + 10)(x - 6)$\\
\textbf{c.~~}$0,5(x - 6)(x + 2)$&\textbf{d.~~}$0,5(x - 10)(x + 6)$\\ \hline
\end{tabularx}
\end{center}

\medskip

\textbf{Question 2}

\medskip

$\left(u_n\right)$ est une suite arithmétique de raison $r = 0,5$ telle que $u_{10} = -4$. Quelle est la valeur du terme $u_2$ ?

\begin{center}
\begin{tabularx}{\linewidth}{|*{4}{X|}}\hline
\textbf{a.~~}$8$&\textbf{b.~~}$0$&\textbf{c.~~}$- 10$&\textbf{d.~~}$- 8$\\ \hline
\end{tabularx}
\end{center}

\medskip

\textbf{Question 3}

\medskip

Soit la fonction $f$ définie pour tout $x \ne -2$ par : $f(x) = \dfrac{2x - 1}{x+2}$.

Parmi les expressions  suivantes, laquelle définit la dérivée $f'$ de la fonction $f$ sur 
$\R\backslash \{-2\}$ ?

\begin{center}
\begin{tabularx}{\linewidth}{|*{4}{X|}}\hline
\textbf{a.~~}$f'(x) = - \dfrac{5}{(x + 2)^2}$&\textbf{b.~~}$f'(x) =  \dfrac{3}{(x + 2)^2}$&\textbf{c.~~}$f'(x) = \dfrac{5}{(x + 2)^2}$&\textbf{d.~~}$f'(x) = 2$\rule[-3pt]{0mm}{7mm}\\ \hline
\end{tabularx}
\end{center}

\medskip

\textbf{Question 4}

\medskip

On se place dans un repère orthonormé \Oij. Laquelle de ces équations est une
équation cartésienne de la droite $\Delta$, de vecteur directeur $\vect{u}\begin{pmatrix}-1\\2\end{pmatrix}$ et passant par le point
A$(-1~;~3)$ ?

\begin{center}
\begin{tabularx}{\linewidth}{|*{4}{X|}}\hline
\textbf{a.~~}$2x - y + 1 = 0$&\textbf{b.~~}$x + 2y + 1 = 0$&\textbf{c.~~}$-x + 2y- 7= 0$&\textbf{d.~~}$-2x - y + 1 = 0$\\ \hline
\end{tabularx}
\end{center}

\medskip

\textbf{Question 5}

\medskip

On se place dans un repère orthonormé \Oij. Parmi ces propositions, quelle est l'équation cartésienne du cercle de centre A(2~;~4) et de rayon 3 ?

\begin{center}
\begin{tabularx}{\linewidth}{|*{2}{X|}}\hline
\textbf{a.~~}$(x- 2)^2+(y- 4)^2 = 3$&\textbf{b.~~}$(x+2)^2 + (y+4)^2 = 9$\\
\textbf{c.~~}$x^2 + y^2 - 4x - 8y+ 11 = 0$&\textbf{d.~~}$x^2 + y^2 + 11 = 0$\\ \hline
\end{tabularx}
\end{center}

\bigskip

