
\medskip

On considère la suite $\left(u_n\right)$ définie pour tout entier naturel $n$, par $u_n = \dfrac{n+2}{n+1}$.

\medskip

\begin{enumerate}
\item Calculer $u_0$,\, $u_1$,\,$u_2$ puis $u_{99}$.
\item 
	\begin{enumerate}
		\item Exprimer, pour tout entier naturel $n$, $u_n - 1$ en fonction de $n$.
		\item Montrer que, pour tout entier naturel $n$, on a :
		
\[u_{n+1} - u_n = \dfrac{- 1}{(n+1)(n+2)}.\]

		\item En déduire le sens de variation de la suite $\left(u_n\right)$.
	\end{enumerate}
\item Soit $a$ un nombre réel dans l'intervalle ]1 ~;~2].

Recopier et compléter sur la copie le programme Python suivant pour qu'il permette de déterminer le plus petit entier naturel $n$ tel que $u_n \leqslant a$, où $a$ est un nombre de l'intervalle ]1~;~2].

\begin{center}
\begin{python}
Def seuil(a) :
	n = 0
	while (n+2) / (n+1) ... a : 
		n = ...
	return ... 
\end{python}
\end{center}
\end{enumerate}
