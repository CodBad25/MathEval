
\medskip

On considère la fonction $f$ définie sur l'intervalle $[0~;~+\infty[$ par 

\[f(x)=3x\e^{-0,4x}.\]

La fonction dérivée de la fonction $f$ est notée $f'$.

\textbf{On admet} que la fonction $f'$ a pour expression $f'(x)=(- 1,2x + 3)\e^{-0,4x}$

\medskip

\begin{enumerate}
\item Déterminer le signe de $f'(x)$ sur l'intervalle $[0~;~+\infty[$.
\item En déduire le tableau de variation de la fonction $f$ sur l'intervalle $[0~;~+\infty[$.
\item Un sportif a pris un produit dopant. La fonction $f$ modélise la quantité, en mg/L, de
ce produit dopant présent dans le sang du sportif $x$ heures après la prise.
	\begin{enumerate}
		\item  Pourquoi peut-on affirmer que ce produit dopant n'est pas naturellement présent
dans l'organisme du sportif ?
		\item Combien de temps après son absorption, ce produit dopant sera-t-il présent en
quantité maximale dans le sang du sportif ?
		\item Le sportif absorbe ce produit dopant au début d'une séance d'entraînement.

Le même jour, 6 heures après le début de cette séance d'entraînement, il est soumis à un contrôle anti-dopage.

Celui-ci se révélera positif si la quantité de produit dopant présent dans l'organisme de ce sportif dépasse \np[mg/L]{1,4}.

Ce contrôle anti-dopage sera-t-il positif ? Justifier. 
	\end{enumerate}
\end{enumerate}
