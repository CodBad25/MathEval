
\medskip

Un snack propose deux types de plats : des sandwichs et des pizzas.

Le snack propose également plusieurs desserts.

La gérante constate que 80\,\% des clients qui achètent un plat choisissent un sandwich et que parmi
ceux-ci seulement 30\,\% prennent également un dessert.

Elle constate aussi que 45\,\% des clients qui ont choisi une pizza comme plat ne prennent pas de
dessert.

On choisit au hasard un client ayant acheté un plat dans ce snack.

On considère les évènements suivants :

\begin{itemize}
\item $S$ : \og Le client interrogé a choisi un sandwich \fg.
\item $T$ : \og Le client interrogé a choisi un dessert \fg.
\end{itemize}

\medskip

\begin{enumerate}
\item Sans justifier, recopier puis compléter l'arbre pondéré suivant :

\begin{center}
\psset{nodesepA=0pt,nodesepB=3pt,treesep=0.75,labelsep=0.1pt,levelsep=2.75cm}
\pstree[treemode=R]{\TR{}}
{\pstree{\TR{$S$~}\taput{$\np{0.8}$}}
	{
	\TR{$T$}\taput{$\dots$}
	\TR{$\overline{T}$}\tbput{$\dots$}
	}
\pstree{\TR{$\overline{S}$~}\tbput{$\dots$}}
	{\TR{$T$}\taput{$\dots$}
	\TR{$\overline{T}$}\tbput{$\dots$}
	}
}
\end{center}
\item Calculer la probabilité que le client ait choisi un sandwich et un dessert.
\item Démontrer que $P(T) = 0,35$.
\item Sachant que le client a acheté un dessert, quelle est la probabilité, arrondie à \np{0.01} près, qu'il ait acheté une pizza ?
\item Les évènements $S$ et $T$ sont-ils indépendants ?
\end{enumerate}

\vspace{0,5cm}

