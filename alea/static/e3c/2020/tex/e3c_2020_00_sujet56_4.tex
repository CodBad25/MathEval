  
\medskip

En 2016, a été lancée une plate-forme de streaming par abonnement.

Le tableau suivant donne le nombre d'abonnés (en million) au 31 décembre de
chaque année de 2016 jusqu'en 2019.
\begin{center}
\begin{tabular}[]{|m{4.85cm}|*{4}{c|}}
\hline
Rang de l'année &1& 2& 3 &4\\\hline
31 décembre de l'année :& 2016& 2017& 2018& 2019\\\hline
Nombre d'abonnés (en millions) &12& 13,7& 15,8& 18,2\\ \hline
\end{tabular}
\end{center}

Les responsables de cette plate-forme étudient l'évolution du nombre d'abonnés afin
d'adapter leurs investissements.

\medskip

\begin{enumerate}
\item  Quelle a été en pourcentage l'évolution du nombre d'abonnés entre 2016 et
2017 ?
\item Expliquer pourquoi le taux moyen d'évolution par an entre 2016 et 2019,
arrondi au centième, est de 14,89\,\%.
\item On considère que le nombre d'abonnés a augmenté de 15\,\% par an à partir de
2016. On décide de modéliser ce nombre d'abonnés (en millions) par une
suite de premier terme 12.

Préciser la nature de cette suite et sa raison.
\item Quel sera selon ce modèle, le nombre d'abonnés au 31 décembre 2020 ?
\item Pour déterminer en quelle année, selon ce modèle, sera obtenu l'objectif de
40 millions d'abonnés, on a défini en langage Python la fonction Seuil ci-dessous.

\begin{center}
\begin{tabular}[]{|c|l|}
\hline
1& def Seuil():\\
2&  \hspace{1em}n=2016\\
3&  \hspace{1em}A=12\\
4&  \hspace{1em}while \ldots :\\
5&  \hspace{1em}\phantom{w}A= \dots\\
6&  \hspace{1em}\phantom{w}n=n+1\\
7&  \hspace{1em}return n\\
\hline
\end{tabular}
\end{center}
Recopier et compléter les instructions 4 et 5 afin que ce programme fournisse
l'année où cet objectif sera atteint.
\end{enumerate}


