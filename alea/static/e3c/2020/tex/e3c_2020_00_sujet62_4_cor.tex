
\( f(x) = 35 \e^{-0{,}22x} \).

\subsection*{1.}

On a :
\[
f(0) = 35 \times \e^{-0{,}22 \times 0} = 35 \times 1 = 35.
\]
À sa mise sur le marché la voiture est vendue 35 000 €.

\subsection*{2.}

5 ans et 6 mois correspondent à la valeur \( x = 5{,}5 \).

On a :
\[
f(5{,}5) = 35 \times \e^{-0{,}22 \times 5{,}5} = 35 \times \e^{-1{,}21} \approx 10{,}4369,
\]
soit environ 10 437 €.

\subsection*{3.}

La fonction \( f \) est dérivable sur \(\mathbb{R}\), et sur cet intervalle :
\[
f'(x) = 35 \times (-0{,}22) \e^{-0{,}22x} = -7{,}7 \e^{-0{,}22x}.
\]

\subsection*{4.}

On sait que, quel que soit le réel \( x \), \( \e^{-0{,}22x} > 0 \), donc \( f'(x) < 0 \) : la fonction \( f \) est donc décroissante sur \(\mathbb{R}\), donc en particulier sur \([0 \,;\, 10]\) de :

\[
f(0) = 35 \text{ à } f(10) = 35 \times \e^{-0{,}22 \times 10} = 35 \times \e^{-2{,}2} \approx 3{,}8781 \text{ (3 878 €)}.
\]

\subsection*{5.}

En programmant sur la calculatrice les valeurs de la fonction \( x \longmapsto f(x) = 35\e^{-0{,}22x} \), on obtient :

\[
f(5{,}6) \approx 10{,}21 \quad \text{et} \quad f(5{,}7) \approx 9{,}9876.
\]

Or \(5{,}6\) ans \( \approx \) 5 ans et \(7{,}2\) mois et \(5{,}7\) ans \( \approx \) 5 ans et \(8{,}4\) mois.

Or 5 ans et 8 mois valent \( 5 + \dfrac{8}{12} = 5 + \dfrac{2}{3} = \dfrac{17}{3} \).

\( f\left(\dfrac{17}{3}\right) \approx 10{,}0611 \), donc le client vendra sa voiture au bout de 5 ans et 9 mois.

