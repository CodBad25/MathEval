
\subsection*{Question 1}

On a \( f(x) = (x + 1)^2 + c - 1 \).

Comme \( c > 1 \Rightarrow c - 1 > 0 \) et que, quel que soit le réel \( x \), \( (x + 1)^2 \geqslant 0 \), il en résulte que :
\[
(x + 1)^2 + c - 1 > 0.
\]
Ce trinôme est donc strictement positif.

\subsection*{Question 2}

On a :
\begin{align*}
&\cos^2(x) + \sin^2(x) = 1 \\
\iff &\sin^2(x) = 1 - \cos^2(x) \\
\iff &\sin^2(x) = 1 - \dfrac{9}{25} \\
\iff &\sin^2(x) = \dfrac{16}{25}.
\end{align*}

Or, sur \( x \in [-\pi \,;\, 0], \text{ on a } \sin(x) < 0 \), donc \( \sin(x) = -\sqrt{\dfrac{16}{25}} = -\dfrac{4}{5} \).

\subsection*{Question 3}

Les côtés \([AB]\) et \([AD]\) sont perpendiculaires, donc les vecteurs \(\overrightarrow{AB}\) et \(\overrightarrow{AD}\) sont orthogonaux, d'où \(\overrightarrow{AB} \cdot \overrightarrow{AD} = 0\).

\subsection*{Question 4}

La droite coupe l'axe des abscisses en un point d'ordonnée nulle, donc en remplaçant \(y\) par 0 dans l'équation de la droite, on a :
\[
2x + 1 = 0 \iff x = -\dfrac{1}{2}.
\]
Réponse \(\textbf{d.}\)

\subsection*{Question 5}

\( \dfrac{\e^x}{\e^{-x}} = \e^{x + x} = \e^{2x} = (\e^x)^2 \).

