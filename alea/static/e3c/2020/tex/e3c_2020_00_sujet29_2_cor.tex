
\subsection*{1.}

Il y a au total : \(30 + 75 + 30 + 15 = 150\) adhérents.

La probabilité que l'adhérent choisi ait 17 ans est égale à \(\dfrac{30}{150} = \dfrac{1}{5} = 0,2\).

\subsection*{2.}

Sur les 15 adhérents de 18 ans, il y a 10 filles. La probabilité est donc égale à \(\dfrac{10}{15} = \dfrac{2}{3}\).

\subsection*{3.}

On note \(X\) la variable aléatoire donnant l'âge de l'adhérent choisi.
\[
\begin{array}{|c|c|c|c|c|}
\hline
Age \, x_i & 15 \, \text{ans} & 16 \, \text{ans} & 17 \, \text{ans} & 18 \, \text{ans} \\
\hline
Total & 30 & 75 & 30 & 15 \\
\hline
p(X = x_i) & \dfrac{1}{5} & \dfrac{1}{2} & \dfrac{1}{5} & \dfrac{1}{10} \\
\hline
\end{array}
\]

\subsection*{4.}

Il y a \(150 - 30 = 120\) adhérents de plus de 15 ans, donc :
\[
p(X \geqslant 16) = \dfrac{120}{150} = \dfrac{40}{50} = \dfrac{80}{100} = 0,8.
\]
80\% des adhérents ont plus de 15 ans.

\subsection*{5.}

\(
E(X) = 15 \times \dfrac{1}{5} + 16 \times \dfrac{1}{2} + 17 \times \dfrac{1}{5} + 18 \times \dfrac{1}{10} = 3 + 8 + 3,4 + 1,8 = 16,2.
\)

L'âge moyen d'un adhérent est donc de 16,2 ans.

