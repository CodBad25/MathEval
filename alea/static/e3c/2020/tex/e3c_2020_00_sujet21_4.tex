
\medskip

Une entreprise de \np{1000} employés est organisée en 3 services \og A \fg, \og B \fg{} et \og C \fg{} d’effectifs
respectifs $450$, $230$ et $320$ employés. Une enquête effectuée auprès de tous les employés
sur leur temps de parcours quotidien entre leur domicile et l’entreprise a montré que :

\begin{itemize}
\item 40\,\% des employés du service \og A \fg{} résident à moins de 30 minutes de l’entreprise ;
\item 20\,\% des employés du service \og B \fg{} résident à moins de 30 minutes de l’entreprise ;
\item 80\,\% des employés du service \og C \fg{} résident à moins de 30 minutes de l’entreprise.
\end{itemize}

On choisit au hasard un employé de cette entreprise et on considère les évènements
suivants :

\begin{itemize}
\item $A$ : l’employé fait partie du service \og A \fg{} ;
\item $B$ : l’employé fait partie du service \og B \fg{} ;
\item $C$ : l’employé fait partie du service \og C \fg{} ;
\item $T$ : l’employé réside à moins de $30$~minutes de l’entreprise.
\end{itemize}
On rappelle que si $E$ et $F$ sont deux évènements, la probabilité d’un évènement $E$ est notée
$p(E)$ et celle de $E$ sachant $F$ est notée $p_F(E)$.

\medskip

\begin{enumerate}
\item Justifier que $p(A) = 0,45$ puis donner $p_A(T)$.
\item Compléter l’arbre pondéré ci-dessous.
\psset{nodesepA=0pt,nodesepB=3pt,treesep=0.75,labelsep=0.1pt,levelsep=2.75cm}
\pstree[treemode=R]{\TR{}}
{\pstree{\TR{$A$~}\taput{$\np{0.45}$}}
	{
	\TR{$T$}\taput{$\dots$}
	\TR{$\overline{T}$}\tbput{$\dots$}
	}
\pstree{\TR{$B$~}\tbput{$\dots$}}
	{\TR{$T$}\taput{$\dots$}
	\TR{$\overline{T}$}\tbput{$\dots$}
	}
\pstree{\TR{$C$~}\tbput{$\dots$}}
	{\TR{$T$}\taput{$\dots$}
	\TR{$\overline{T}$}\tbput{$\dots$}
	}	
}
\item Déterminer la probabilité que l'employé choisi soit du service \og A \fg et qu’il réside à moins de
30 minutes de son lieu de travail.
\item Montrer que $p(T) = 0,482$.
\item Sachant qu'un employé de l’entreprise réside à moins de 30 minutes de son lieu de travail,
déterminer la probabilité qu'il fasse partie du service \og C \fg. Arrondir à $10^{-3}$ près.

\end{enumerate}

