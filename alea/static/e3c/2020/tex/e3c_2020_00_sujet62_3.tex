
\medskip

Claire joue régulièrement à un jeu de simulation de tournois de judo en ligne. Les
adversaires qu'elle combat sont générés automatiquement de manière aléatoire selon le
niveau atteint dans le jeu.

Elle a atteint le niveau le plus élevé, celui de la ceinture noire. Les scores relevés par le jeu
montrent qu'elle gagne dans 45\,\% des cas si son adversaire est ceinture noire et dans 70\,\%
si son adversaire n'est pas ceinture noire.

Claire commence un tournoi et un premier adversaire est généré par le jeu. À ce niveau la
probabilité d'affronter un adversaire ayant une ceinture noire est 0,6.

On note :

\begin{itemize}
\item $N$ l'évènement : \og l'adversaire est ceinture noire \fg{} ;
\item $G$ l'évènement : \og Claire gagne le combat \fg.
\end{itemize}

\medskip

\begin{enumerate}
\item  Recopier et compléter l'arbre pondéré ci-dessous modélisant cette situation.

\begin{center}
\psset{nodesepA=0pt,nodesepB=3pt,treesep=0.5,labelsep=0.1pt,levelsep=2.5cm}
\pstree[treemode=R]{\TR{}}
{\pstree{\TR{$N$~}\taput{}}
	{
	\TR{$G$}\taput{}
	\TR{$\overline{G}$}\tbput{}
	}
\pstree{\TR{$\overline{N}$~}\tbput{}}
	{\TR{$G$}\taput{}
	\TR{$\overline{G}$}\tbput{}
	}
}
\end{center}

\item Calculer la probabilité que l'adversaire soit ceinture noire et que Claire gagne son
tournoi.
\item Montrer que la probabilité que Claire gagne son combat est $0,55$.
\item Claire vient de perdre un combat. Quelle est la probabilité que le combat ait été
contre une ceinture noire ?
\item On considère dans cette question que la probabilité que Claire gagne est $0,55$. Elle
fait deux combats successifs.

On note $X$ la variable qui compte le nombre de victoires.
Donner la loi de probabilité de $X$.
\end{enumerate}

\vspace{0,5cm}

