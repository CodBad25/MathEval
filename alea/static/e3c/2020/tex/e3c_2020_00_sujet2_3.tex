
\medskip

\textbf{Partie A}

\medskip 

Soit $\left(u_n\right)$ une suite géométrique de raison 2 de premier terme $u_0 = 0,2$.

\medskip

\begin{enumerate}
\item Calculer $u_{18}$ puis $u_{50}$.
\item Calculer $u_0 + u_1 + u_2 + u_3 + u_4 + \ldots + u_{18}$, c'est-à-dire la somme des 19
premiers termes de la suite $\left(u_n\right)$.
\item Recopier et compléter les trois parties en pointillé de l'algorithme suivant permettant de déterminer le plus petit entier $n$ tel que la somme des $n + 1$ premiers termes de la suite $u$ dépasse \np{100000}.
\end{enumerate}

\begin{center}
$\begin{array}{|l|}\hline
U \gets  0,2\\
S \gets 0,2\\
N \gets 0\\
~\\
\text{Tant que \ldots \ldots}\\
\quad U \gets \ldots\\
\quad S \gets \ldots\\
\quad N \gets N + 1\\
~\\

\text{Fin tant que}~~~~~~~\\
\text{Afficher}\: N\\ \hline
\end{array}$
\end{center}

\medskip

\textbf{Partie B}

\medskip

Claude a donné $20$ centimes d'euros (soit 0,20 ~\euro) à son petit-enfant Camille pour sa naissance. 

Ensuite, Claude a doublé le montant offert d'une année sur l'autre pour chaque anniversaire jusqu'aux 18 ans de Camille.

La somme totale versée par Claude à Camille permet-elle de payer un appartement à Angers d'une valeur de \np{100000}~\euro ?

\bigskip

