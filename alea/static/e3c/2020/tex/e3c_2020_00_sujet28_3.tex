
\medskip

En 2019, les déchets d'une entreprise sont évalués à \np{6 000} tonnes.

Cette entreprise s'engage à réduire ses déchets de 5\,\% chaque année.

\medskip

\begin{enumerate}
\item  Avec cette politique, quelle quantité de déchets peut envisager l'entreprise pour l'année 2020 ?
\item Pour tout entier naturel $n$, on note $d_n$ la quantité de déchets produits en tonne par cette entreprise l'année $2019+n$.

Avec cette notation, on a alors $d_0 =\np{6000}$.
\begin{enumerate}
\item  Exprimer $d_{n+1}$ en fonction de $d_n$ pour tout entier naturel $n$.
\item Quelle est la nature de la suite $\left(d_n\right)$ ?
\item Déterminer la quantité totale de déchets produits par l'entreprise entre 2019 et 2023.

\emph{On arrondira le résultat à la tonne près.}
\end{enumerate}
\item L'entreprise souhaite savoir au bout de combien d'années d'application de cette politique de réduction des déchets la quantité annuelle produite aura diminué de 40\,\% par rapport à la quantité produite en 2019.

Recopier et compléter l'algorithme ci-dessous sur la copie afin qu'il permette de répondre à la question posée :

\begin{center}
\begin{tabular}[]{|m{4cm}|}
\hline
$D\longleftarrow \np{6000}$\\
 $N \longleftarrow 0$\\
Tant que $D \dotfill $ \\
\hspace{1.5em}$D \longleftarrow \dotfill $\\
\hspace{1.5em}$N\longleftarrow N+1$\\
Fin Tant que \\
\hline
\end{tabular}
\end{center}
\end{enumerate}

\vspace{0,5cm}

