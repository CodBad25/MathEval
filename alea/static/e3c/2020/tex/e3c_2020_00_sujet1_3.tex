
\medskip

Lors du lancement d'un hebdomadaire, \np{1200} exemplaires ont été vendus.

Une étude de marché prévoit une progression des ventes de 2\,\% chaque semaine.

On modélise le nombre d'hebdomadaires vendus par une suite $\left(u_n\right)$ où $u_n$ représente le nombre de journaux vendus durant la $n$-ième semaine après le début de l'opération.

On a donc $u_0 = \np{1200}$.

\medskip

\begin{enumerate}
\item Calculer le nombre $u_2$. Interpréter ce résultat dans le contexte de l'exercice. 
\item Écrire, pour tout entier naturel $n$, l'expression de $u_n$ en fonction de $n$.
\item Voici un programme rédigé en langage Python :

\begin{center}
\begin{tabularx}{0.45\linewidth}{|X|}\hline
\texttt{def suite ( ) :}\\
\quad \texttt{u = 1200} \\
\quad \texttt{S = 1200}\\
\quad \texttt{n=0}\\
\quad \texttt{while S < 30000}\\
\qquad \texttt{n = n+1}\\
\qquad  \texttt{u = u*1,02}\\
\qquad  \texttt{S=S+u}\\
\quad \texttt{return(n)}\\ \hline
\end{tabularx}
\end{center}

Le programme retourne la valeur $20$.

Interpréter ce résultat dans le contexte de l'exercice.
\item Déterminer le nombre total d'hebdomadaires vendus au bout d'un an.
\end{enumerate}

\bigskip

