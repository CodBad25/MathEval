
\subsection*{Question 1}

Puisque \( x > 2{,}5 \), alors \( -2x + 5 \neq 0 \), donc \( f \) est dérivable sur \( ]2{,}5 \,;\, +\infty[ \) et on a :
\begin{align*}
f'(x) &= \dfrac{3(-2x + 5) - (-2)(3x + 1)}{(-2x + 5)^2} \\
&= \dfrac{-6x + 15 + 6x + 2}{(-2x + 5)^2} \\
&= \dfrac{17}{(-2x + 5)^2}.
\end{align*}

\subsection*{Question 2}

La courbe coupe l'axe des abscisses aux points d'abscisse 1 et 3. L'expression de \( f(x) \) contient donc les facteurs \( x - 1 \) (ou \( 1 - x \)) et \( x - 2 \) (ou \( 2 - x \)).

D'autre part, on doit avoir \( f(2) = 2 \) : seule la réponse \textbf{b.} convient.

\subsection*{Question 3}

Le nombre dérivé \( f'(0) \) est égal au coefficient directeur de la tangente au point d'abscisse 0. En utilisant les points \( (0\,;\,2) \) et \( (2\,;\,4) \), on trouve que :
\[
f'(0) = \dfrac{4 - 2}{2 - 0} = \dfrac{2}{2} = 1.
\]

\subsection*{Question 4}

Une équation réduite de la droite \( (GH) \) est : \( y = ax + b \).

On a :
\[
\begin{cases}
G \in (GH) \\
H \in (GH)
\end{cases}
\iff 
\begin{cases}
-2 = a + b \\
4 = 6a + b
\end{cases}
\Rightarrow 6 = 5a \iff a = \dfrac{6}{5}.
\]
On en déduit que : \( b = -2 - \dfrac{6}{5} = -\dfrac{16}{5} \).

Une équation de la droite \( (GH) \) est : \( y = \dfrac{6}{5}x - \dfrac{16}{5} \).

On a :
\[
D(-14 \,;\, -20) \in (GH) \iff -20 = \dfrac{6}{5} \times (-14) - \dfrac{16}{5},
\]
qui est vraie.


\subsection*{Question 5}

De la relation \( \sin^2 x + \cos^2 x = 1 \), on en déduit que :
\[
\sin^2 x = 1 - \left( -\dfrac{\sqrt{3}}{2} \right)^2 = 1 - \dfrac{3}{4} = \dfrac{1}{4}.
\]
Or, sur l'intervalle \( \left[ \pi \,;\, \dfrac{3\pi}{2} \right] \), on sait que \( \sin x < 0 \), donc \( \sin x = -\dfrac{1}{2} \).

