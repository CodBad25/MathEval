	
	\textbf{Question 1}
	
	Retrancher 15\% signifie multiplier par $1 - \dfrac{1,5}{100} = 0,985$.\\
	 Ainsi, pour tout naturel $n$, on a :
	\[
	d_{n+1} = 0,985 d_n.
	\]
	En particulier :
	\[
	d_1 = 0,985 \times 537 \approx 529.
	\]
	
	\textbf{Question 2}
	
	La suite $(d_n)$ est une suite géométrique de premier terme $d_0 = 537$ et de raison $q = 0,985$.
	
	\textbf{Question 3}
	
	Le programme Python pour trouver au bout de combien d'années la quantité de déchets est inférieure à 513 kg est :
	
	\begin{verbatim}
		def annee():
		n = 0
		d = 537
		while d > 513:
		    n = n + 1
		    d = 0.985 * d
		return n
	\end{verbatim}
	
	On trouve qu'au bout de 4 ans, soit en 2023, la quantité de déchets devient inférieure à 513 kg pour la première fois.
	
