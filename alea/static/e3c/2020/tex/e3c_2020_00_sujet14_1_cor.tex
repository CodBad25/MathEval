	\section*{Exercice 1 (5 points)}
	
	\begin{enumerate}
		\item 	L’inéquation \(x^2 + x + 2 > 0\) :
		
		\[
		x^2 + x + 2 > 0 \iff \left( x + \dfrac{1}{2} \right)^2 - \dfrac{1}{4} + 2 > 0 \iff \left( x + \dfrac{1}{2} \right)^2 + \dfrac{7}{4} > 0.
		\]
		
		Comme \(\dfrac{7}{4} > 0\), ceci est vrai quel que soit le réel \(x\) : \(S = \mathbb{R}\).
		
		\item 
		
		Soient \(\overrightarrow{u}\) et \(\overrightarrow{v}\) deux vecteurs tels que \(\|\overrightarrow{u} \| = 3\), \(\|\overrightarrow{v} \| = 2\) et \(\overrightarrow{u} \cdot \overrightarrow{v} = -1\),\\
		alors \(\|\overrightarrow{u} + \overrightarrow{v} \|^2\) est égal à :
		
		\[
		\|\overrightarrow{u} + \overrightarrow{v} \|^2 = (\overrightarrow{u} + \overrightarrow{v}) \cdot (\overrightarrow{u} + \overrightarrow{v}) = \overrightarrow{u} \cdot \overrightarrow{u} + \overrightarrow{v} \cdot \overrightarrow{v} + 2 \overrightarrow{u} \cdot \overrightarrow{v} = 3^2 + 2^2 + 2 \times (-1) = 9 + 4 - 2 = 11.
		\]
		
		\item 
		
		Soient \(A\) et \(B\) deux évènements d’un univers tels que \(P_A(B) = 0,2\) et \(P(A) = 0,5\).\\ Alors la probabilité \(P(A \cap B)\) est égale à :
		
		\[
		P_A(B) = \dfrac{P(A \cap B)}{P(A)} \iff 0,2 = \dfrac{P(A \cap B)}{0,5} \Rightarrow P(A \cap B) = 0,2 \times 0,5 = 0,1.
		\]
		
		\item 
		
		Soit \((u_n)\) une suite arithmétique de terme initial \(u_0 = 2\) et de raison $3$.\\ 
		La somme \(S\) définie par \(S = u_0 + u_1 + \ldots + u_{12}\) est égale à :
		
		\[
		S = 2 + 5 + 8 + 11 + \ldots + 35 + 38
		\]
		
		ou
		
		\[
		S = 38 + 35 + 32 + \ldots + 8 + 5 + 2.
		\]
		
		En faisant la somme, on obtient :
		
		\[
		2S = 40 + 40 + 40 + \ldots + 40 = 13 \times 40 = 520 \Rightarrow S = 260.
		\]
		
		\item 
		
		On a \(f'(x) = 3(2x - 5)^2 \times (2x - 5)' = 3 \times 2(2x - 5)^2 = 6(2x - 5)^2\).
	\end{enumerate}
	

	
