
\subsection*{1.}

Retrancher 4 \% revient à multiplier par \(1 - \dfrac{4}{100} = 1 - 0{,}04 = 0{,}96\).

Donc une semaine après le remplissage, il reste : \(80 \times 0{,}96 = 76{,}8 \text{ m}^3\).

\subsection*{2.}

\paragraph{a.} On vu que le volume au bout d'une semaine est celui de la semaine précédente multiplié par \(0{,}96\).

Donc pour tout naturel \(n\), \(V_{n+1} = 0{,}96V_n\), égalité qui montre que la suite \((V_n)\) est une suite géométrique de premier terme \(V_0 = 80\) et de raison \(q = 0{,}96\).

\paragraph{b.} On sait qu'alors, pour tout naturel \(n\), \(V_n = V_0 \times q^n = 80 \times 0{,}96^n\).

\paragraph{c.} On a donc \(V_7 = 80 \times 0{,}96^7 \approx 60{,}1158 \approx 60{,}116\) m\(^3\) au litre près.

\subsection*{3.}

\begin{center}
\begin{python}
def nombreJour(U) :
    N = 1
    V = 76.8
    while V >= 70 :
        N = N + 1
        V = (V + 2) * 0.96
    return N
\end{python}
\end{center}

Il s'arrête la \(8^{\text{ème}}\) semaine.

