
\medskip

\textbf{Les résultats seront donnés sous forme de fractions irréductibles.}

Une enquête a été menée auprès de lycéens pour estimer la proportion de ceux qui ont déjà consommé du cannabis. Pour encourager les réponses sincères, on met en place le protocole suivant :

Chaque adolescent lance d’abord un dé équilibré à 6 faces et l’enquêteur qui va l’interroger ne connaît pas le résultat du lancer. À la question « Avez-vous déjà consommé du cannabis ? », l’adolescent doit répondre :

\begin{itemize}
\item [\textbullet] \og non \fg{} si le résultat du lancer est 5, qu’il ait ou non déjà consommé du cannabis ;
\item [\textbullet] \og oui \fg{} si le résultat du lancer est 6, qu’il ait ou non déjà consommé du cannabis ;
\item [\textbullet] \og oui \fg{} ou « non » dans les autres cas, mais de façon sincère.
\end{itemize}

On note :

\begin{itemize}
\item [\textbullet] $N$ : l’évènement l’adolescent a répondu \og non \fg{} ;
\item [\textbullet] $O$ : l’évènement l’adolescent a répondu \og oui \fg{} ;
\item [\textbullet] $C$ : l’évènement l’adolescent a déjà consommé effectivement du cannabis ;
\item [\textbullet] $\overline{C}$ : l’évènement l’adolescent n'a jamais consommé du cannabis.
\end{itemize} 

Sur les lycéens qui ont participé à cette enquête on constate que la probabilité qu’un adolescent ait répondu \og oui \fg{} est de $\frac{3}{5}$ , soit $p(0) = \frac{3}{5}$.

On veut déterminer la probabilité, notée $p$, qu’un adolescent ait déjà consommé du cannabis.

On a donc $p(C)= p$.

\medskip

\begin{enumerate}
\item Justifier que la probabilité qu’un adolescent ait répondu « oui » sachant qu’il n’a jamais consommé de cannabis est $\frac{1}{6}$.
\item On a représenté ci-dessous l’arbre de probabilités représentant la situation. Compléter cet arbre.

\begin{center}
\psset{nodesepA=0pt,nodesepB=3pt,treesep=2cm,levelsep=3cm}
\pstree[treemode=R]{\TR{}}
{\pstree{\TR{$C$~~}\taput{$p$}}
	{
	\TR{$O$}\taput{$\dots$}
	\TR{$N$}\tbput{$\frac{1}{6}$}
	}
\pstree{\TR{$\overline{C}$~~}\tbput{$\dots$}}
	{\TR{$O$}\taput{$\frac{1}{6}$}
	\TR{$N$}\tbput{$\dots$}
	}
}
\end{center}
\item \begin{enumerate}
\item Démontrer que la probabilité $p$ qu’un adolescent ait déjà consommé du cannabis vérifie l’équation :

\[\frac{2}{3}p+\frac{1}{6}=\frac{3}{5}.\]

\item En déduire la valeur de $p$.
\end{enumerate}
\item Sachant qu’un adolescent a répondu « non » pendant l’enquête, quelle est la probabilité qu’il n’ait jamais consommé de cannabis ?
\end{enumerate}

