
\medskip

Ce QCM comprend 5 questions.

Pour chacune des questions, une seule des quatre réponses proposées est correcte.

Les questions sont indépendantes.

Pour chaque question, indiquer le numéro de la question et recopier sur la copie la lettre correspondante à la réponse choisie.

Aucune justification n'est demandée mais il peut être nécessaire d'effectuer des recherches au brouillon pour aider à déterminer votre réponse.

Chaque réponse correcte rapporte $1$ point. Une réponse incorrecte ou une question sans réponse n'apporte ni ne retire de point.

\medskip

\textbf{Question 1 :}

\medskip

Dans un repère orthonormé, on considère les points A(4~;~2), B(2~;~6). Une équation cartésienne de la médiatrice du segment [AB] est :

\begin{center}
\begin{tabularx}{\linewidth}{|*{4}{X|}}\hline
\textbf{a.~~}$x = 3$&\textbf{b.~~}$x - 2y + 5 = 0$&\textbf{c.~~}$x + 2y - 11  = 0$&\textbf{d.~~}$y = 0,5x + 3$\\ \hline
\end{tabularx}
\end{center}

\medskip

\textbf{Question 2 :}

\medskip

On donne deux points P et N tels PN $= 6$.

L'ensemble des points M tels que $\vect{\text{MP}} \cdot \vect{\text{MN}} = 0$ est:

\begin{center}
\begin{tabularx}{\linewidth}{|*{4}{X|}}\hline
\textbf{a.~~}la droite (PN).&\textbf{b.~~}le cercle de diamètre [PN].&\textbf{c.~~}un cercle de rayon 6&\textbf{d.~~}le milieu du segment [PN].\\ \hline
\end{tabularx}
\end{center}

\medskip

\textbf{Question 3 :}

\medskip
Soit $g$ la fonction définie sur $\R$ par $g(x) = x^3 - 4x + 5$.

Une équation de la tangente à la courbe représentative de $g$ dans un repère orthonormé au point d'abscisse $- 1$ est :

\begin{center}
\begin{tabularx}{\linewidth}{|*{4}{X|}}\hline
\textbf{a.~~}$y=8x+7$&\textbf{b.~~}$y=-7 x+1$&\textbf{c.~~}$y= -x +7$&\textbf{d.~~}$x = - 0,5$\\ \hline
\end{tabularx}
\end{center}

\medskip

\textbf{Question 4 :}

\medskip

L'axe de symétrie de la parabole d'équation $y = x^2 + x + 3$ est:

\begin{center}
\begin{tabularx}{\linewidth}{|*{4}{X|}}\hline
\textbf{a.~~}$y =x$&\textbf{b.~~}$y=- 0,5x+1$&\textbf{c.~~}$y= -0,5$&\textbf{d.~~}$x = - 0,5$\\ \hline
\end{tabularx}
\end{center}

\medskip

\textbf{Question 5 :}

\medskip 

L'inéquation $-3\text{e}^{x+2} > - 3\text{e}^{4}$ d'inconnue $x$, a pour ensemble de solutions :

\begin{center}
\begin{tabularx}{\linewidth}{|*{4}{X|}}\hline
\textbf{a.~~}$]- 2~;~+ \infty[$&\textbf{b.~~}$]2~;~+ \infty[$&\textbf{c.~~}$]- \infty~;~2[$&\textbf{d.~~}$]- \infty~;~- 2[$\\ \hline
\end{tabularx}
\end{center}

\bigskip

