
\subsection*{Question 1}

Pour tout réel \(x\), on a :
\[
\sin(7\pi - x) = \sin(7\pi - x - 6\pi) = \sin(\pi - x) = \sin(x).
\]
Réponse : \(\textbf{a.}\)

\subsection*{Question 2}

Avec \( A(-3\,;\,5) \), \( B(2\,;\,-2) \) et \( C(1\,;\,7) \), on a :
\[
\overrightarrow{AB} \begin{pmatrix} 5 \\ -7 \end{pmatrix} \quad \text{et} \quad \overrightarrow{AC} \begin{pmatrix} 4 \\ 2 \end{pmatrix}.
\]
D'où :
\[
\overrightarrow{AB} \cdot \overrightarrow{AC} = 5 \times 4 + (-7) \times 2 = 20 - 14 = 6.
\]
Réponse : \(\textbf{b.}\)

\subsection*{Question 3}

On considère la fonction \(f(x) = \dfrac{3x + 4}{x^2 + 1}\).

Sa dérivée est :
\[
f'(x) = \dfrac{3(x^2 + 1) - 2x(3x + 4)}{(x^2 + 1)^2} = \dfrac{3x^2 + 3 - 6x^2 - 8x}{(x^2 + 1)^2} = \dfrac{-3x^2 - 8x + 3}{(x^2 + 1)^2}.
\]
Réponse : \(\textbf{c.}\)

\subsection*{Question 4}

L'ensemble des points \( M(x\,;\,y) \) tels que \( x^2 + y^2 - 10x + 6y + 30 = 0 \) est :

\begin{align*}
&x^2 + y^2 - 10x + 6y + 30 = 0 \\
\iff &(x - 5)^2 - 25 + (y + 3)^2 - 9 + 30 = 0 \\
\iff &(x - 5)^2 + (y + 3)^2 - 4 = 0 \\
\iff &(x - 5)^2 + (y + 3)^2 = 2^2,
\end{align*}
on reconnaît une équation du cercle de centre \( (5\,;\,-3) \) et de rayon 2.

Réponse : \(\textbf{c.}\)

\subsection*{Question 5}

La somme \( 1 + 5 + 5^2 + \dots + 5^{30} \) est égale à :

\[
S= 1 + 5 + 5^2 + 5^3 + \dots + 5^{30},
\]
donc :
\[
5S = 5 + 5^2 + 5^3 + \dots + 5^{30} + 5^{31},
\]
et par différence :
\[
4S = 5^{31} - 1 \iff S = \dfrac{5^{31} - 1}{4}.
\]

Réponse : \(\textbf{d.}\)

