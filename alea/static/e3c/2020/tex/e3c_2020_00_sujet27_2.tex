
\medskip

Dans le plan muni d'un repère, on a tracé la courbe représentative $\mathcal{C}_f$ d'une fonction $f$ définie et dérivable sur $\R$. On note $f'$ la dérivée de $f$. 

On sait que la courbe $\mathcal{C}_f$ admet exactement deux tangentes horizontales :

\begin{itemize}
\item l'axe des abscisses comme tangente à la courbe $\mathcal{C}_f$ au point $A(-1~;~0)$ ;
\item la droite $T_B$ comme tangente à la courbe $\mathcal{C}_f$ au point $B\left(\dfrac{1}{3}~;~-\dfrac{32}{27}\right)$.
\end{itemize}

\begin{center}
\psset{xunit=2.5cm,yunit=1.5cm,labelFontSize=\scriptstyle,showorigin=false,comma=true}
\begin{pspicture}(-3,-3)(3,2.6)
\multido{\n=-2.5+0.1}{50}{\psline[linewidth=0.35pt,linecolor=lightgray](\n,-2.9)(\n,2.4)}
\multido{\n=-2.9+0.1}{54}{\psline[linewidth=0.35pt,linecolor=lightgray](-2.5,\n)(2.4,\n)}
\multido{\n=-2.5+0.5}{10}{\psline[linewidth=0.45pt](\n,-2.9)(\n,2.4)}
\multido{\n=-2.5+0.5}{10}{\psline[linewidth=0.45pt](-2.5,\n)(2.4,\n)}
\psaxes[linewidth=0.95pt,Dx=0.5,Dy=0.5]{->}(0,0)(-2.5,-2.9)(2.5,2.4)
\def\Func{x x x 1 add mul 1 sub mul 1 sub }
\psplot[plotpoints=2000,linewidth=1.25pt,linecolor=red]{-2}{1.4}{\Func}
\psline[linewidth=1.25pt](-2.5,-1.185185)(2.4,-1.185185)
\psdots[dotstyle=Mul,dotscale=2.06,linecolor=blue](-1,0)(0.333333,-1.185185)
\uput[u](-1,0){$A$}\uput[d](0.333333,-1.185185){$B$}\uput[dl](0,0){O}\uput[r](1.01,1.3){\red $\mathcal{C}_f$}
\uput[d](2.2,-1.25){$T_B$}
\end{pspicture}
\end{center}

\begin{enumerate}
\item  Par lecture graphique, donner les solutions de l'équation $f(x)=0$.
\end{enumerate}

La fonction $f$ est définie sur $\R$ par $f(x)=x^3+x^2-x-1$. On note $f'$ la dérivée de $f$.

\begin{enumerate}[resume]
\item Déterminer $f'(x)$ pour tout réel $x$.
\item En déduire le tableau de variations de $f$.
\item En utilisant ce qui précède, déterminer la position relative de la courbe $\mathcal{C}_g$ de la fonction $g$ définie sur $\R$ par $g(x)=x^3+x^2$ et de la droite $D$ d'équation $y=x+1$.
\end{enumerate}

\vspace{0,5cm}

