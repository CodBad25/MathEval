
\medskip

Ce QCM comprend 5 questions.

Pour chacune des questions, une seule des quatre réponses proposées est correcte.

Les questions sont indépendantes.

Pour chaque question, indiquer le numéro de la question et recopier sur la copie la lettre correspondante à la réponse choisie.

Aucune justification n'est demandée mais il peut être nécessaire d'effectuer des recherches au brouillon pour aider à déterminer votre réponse.

Chaque réponse correcte rapporte $1$ point. Une réponse incorrecte ou une question sans réponse n'apporte ni ne retire de point.

\medskip

\textbf{Question 1}

\medskip

Dans un repère orthonormé, on a : $\vect{\text{AB}}\binom{-4}{3}$  et $\vect{\text{CB}}\binom{- 1}{5}$ . Le produit scalaire $\vect{\text{AB}} \cdot \vect{\text{CB}}$ vaut :

\begin{center}
\begin{tabularx}{\linewidth}{|*{4}{X|}}\hline
\textbf{a.~~}$- 23$ &\textbf{b.~~}$- 17$&\textbf{c.~~}$19$&\textbf{d.~~}$23$\\ \hline
\end{tabularx}
\end{center}

\medskip

\textbf{Question 2}

\medskip

Dans un repère orthonormé, on a $\vect{\text{CB}}\binom{-1}{5}$. Alors la longueur CB est égale à :

\begin{center}
\begin{tabularx}{\linewidth}{|*{4}{X|}}\hline
\textbf{a.~~}$24$ &\textbf{b.~~}$\sqrt{24}$&\textbf{c.~~}$26$&\textbf{d.~~}$\sqrt{26}$\rule[-3mm]{0mm}{9mm}\\ \hline
\end{tabularx}
\end{center}

\medskip

\textbf{Question 3}

\medskip

\parbox{0.68\linewidth}{ABC est un triangle équilatéral de côté 3.

I et H sont les milieux respectifs de [CB] et de [AB].

D est le projeté orthogonal de I sur (CH).}\hfill \parbox{0.3\linewidth}{
\psset{unit=1cm}
\begin{pspicture}(-2,0)(2,3.5)
\pspolygon(-2,0)(2,0)(0,3.464)
\uput[dl](-2,0){A}\uput[dr](2,0){B}\uput[u](0,3.464){C}\uput[r](1,1.732){I}\uput[r](0,1.732){D}
\uput[d](0,0){H}
\psdots(-2,0)(2,0)(0,3.464)(1,1.732)(0,1.732)(0,0)
\psline(0,3.464)(0,0)
\end{pspicture}}

On a :

\begin{center}
\begin{tabularx}{\linewidth}{|*{4}{X|}}\hline
\textbf{a.~~}$\vect{\text{HB}} \cdot \vect{\text{HC}} = 0$ &\textbf{b.~~}$\vect{\text{AH}} \cdot \vect{\text{DI}} = 0$&\textbf{c.~~}$\vect{\text{AH}} \cdot \vect{\text{AI}} = 0$&\textbf{d.~~}$\vect{\text{BH}} \cdot \vect{\text{DI}} = 0$\rule[-3mm]{0mm}{9mm}\\ \hline
\end{tabularx}
\end{center}

\medskip

\textbf{Question 4}

\medskip

Soit un réel $x$ tel que $\cos (x) = \dfrac{\sqrt{3}}{2}$. On a :

\begin{center}
\begin{tabularx}{\linewidth}{|*{4}{X|}}\hline
\textbf{a.~~}$\cos (-x) = \dfrac{\sqrt{3}}{2}$ &\textbf{b.~~}$\sin (- x) = - \dfrac{\sqrt{3}}{2}$&\textbf{c.~~}$\sin (x) = \dfrac{\sqrt{3}}{2}$&\textbf{d.~~}$\cos (- x) = - \dfrac{\sqrt{3}}{2}$\rule[-3mm]{0mm}{9mm}\\ \hline
\end{tabularx}
\end{center}

\medskip

\textbf{Question 5}

\medskip

Le plan est muni d'un repère orthonormé.

On considère l'équation de cercle $x^2 - 2x + (y + 3)^2 = 3$. 

Son centre a pour coordonnées :

\begin{center}
\begin{tabularx}{\linewidth}{|*{4}{X|}}\hline
\textbf{a.~~}$(-1~;~-3)$ &\textbf{b.~~}$(1~;~-3)$&\textbf{c.~~}$(-2~;~3)$&\textbf{d.~~}$(-2~;~-3)$\\ \hline
\end{tabularx}
\end{center}

\bigskip

