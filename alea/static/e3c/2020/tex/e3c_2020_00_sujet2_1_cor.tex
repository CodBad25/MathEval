	\section*{Exercice 1 (5 points)}
	
	Ce QCM comprend cinq questions. Pour chacune des questions, une seule des quatre réponses proposées est correcte. Les questions sont indépendantes. Pour chaque question, indiquer le numéro de la question et recopier sur la copie la lettre correspondante à la réponse choisie. Aucune justification n’est demandée mais il peut être nécessaire d’effectuer des recherches au brouillon pour aider à déterminer votre réponse. Chaque réponse correcte rapporte 1 point. Une réponse incorrecte ou une question sans réponse n’apporte ni ne retire de point.
	
	\subsection*{Question 1}
	L’inéquation $e^{-2x} > 0$ d’inconnue $x$ a pour ensemble de solutions :
	
	\begin{itemize}
		\item[a.] $\mathbb{R}$
		\item[b.] $]0 ; +\infty[$
		\item[c.] $]-\infty ; 0[$
		\item[d.] $\emptyset$
	\end{itemize}
	
	On sait que quel que soit $a \in \mathbb{R}$, $e^a > 0$. L’ensemble des solutions est donc $\mathbb{R}$.
	
	\subsection*{Question 2}
	Pour tout réel $x$, $(e^x - 1)^2$ est égal à :
	
	\begin{itemize}
		\item[a.] $e^{2x} - 1$
		\item[b.] $e^{2x} + 1$
		\item[c.] $e^{2x} - 2e^x + 1$
		\item[d.] $e^{(x^2)} - 1$
	\end{itemize}
	
	D’après l’identité $(a - b)^2 = a^2 + b^2 - 2ab$, $(e^x - 1)^2 = e^{2x} + 1 - 2e^x$.
	
	\subsection*{Question 3}
	Soit $f$ la fonction définie sur $\mathbb{R}$ par : $f(x) = e^{5x-1}$. Pour tout réel $x$, $f'(x)$ est égal à :
	
	\begin{itemize}
		\item[a.] $e^{5x-1}$
		\item[b.] $5e^{5x}$
		\item[c.] $5e^{5x-1}$
		\item[d.] $5xe^{5x-1}$
	\end{itemize}
	
	On sait que $(e^u)' = u' \times e^u$. Avec $u(x) = 5x - 1$, $u'(x) = 5$, donc $f'(x) = 5e^{5x-1}$.
	
	\subsection*{Question 4}
	Dans un repère orthonormé, la droite passant par $A(4; 7)$ et de vecteur normal $\vec{n}\left( \begin{array}{c}
		-1 \\
		3 \end{array} \right)$ a pour équation :
	
	\begin{itemize}
		\item[a.] $3x + y - 19 = 0$
		\item[b.] $3x + y + 19 = 0$
		\item[c.] $-x + 3y + 17 = 0$
		\item[d.] $-x + 3y - 17 = 0$
	\end{itemize}
	
	Soit $d$ la droite dont on cherche l’équation. $M(x ; y) \in d \Leftrightarrow \overrightarrow{AM}$ et $\vec{n}$ sont orthogonaux, donc $\overrightarrow{AM} \cdot \vec{n} = 0$. Avec $\overrightarrow{AM} \left( \begin{array}{c}
		x-4 \\
		y-7 \end{array} \right)$ et $\vec{n} \left( \begin{array}{c}
		-1 \\
		3 \end{array} \right)$, on obtient :
	\[ M(x ; y) \in d \Leftrightarrow -(x - 4) + 3(y - 7) = 0 \Leftrightarrow -x + 3y - 17 = 0. \]
	
	\subsection*{Question 5}
	Le plan est muni d’un repère orthonormé. On considère l’équation de cercle $x^2 - 4x + (y + 3)^2 = 3$. Son centre a pour coordonnées :
	
	\begin{itemize}
		\item[a.] $(-2 ; -3)$
		\item[b.] $(2 ; -3)$
		\item[c.] $(-4 ; 3)$
		\item[d.] $(4 ; -3)$
	\end{itemize}
	
	$x^2 - 4x + (y + 3)^2 = 3 \Leftrightarrow (x - 2)^2 - 4 + (y + 3)^2 = 3 \Leftrightarrow (x - 2)^2 + (y - (-3))^2 = 7$. On a donc $CM^2 = 7$, avec $C(2 ; -3)$.
	
