	\section*{Exercice 1 (5 points)}
	

	\subsection*{Question 1}
Il y a 200 jetons. Avec $R$ l’évènement : « le jeton tiré est rouge » et $G$ l’évènement : « le jeton est gagnant », on a alors $P(R) = \dfrac{150}{200} = \dfrac{3}{4}$ et $P_R(G) = 0,20$.\\
Donc $P(R \cap G) = P(R) \times P_R(G) = \dfrac{3}{4} \times 0,2 = 0,75 \times 0,2 = 0,15$.\\ La réponse correcte est \textbf{c.}
	
	\subsection*{Question 2}
	
D’après la loi des probabilités totales :
	\[
	P(G) = P(G \cap R) + P(G \cap \overline{R})
	\]
Or $P(G \cap \overline{R}) = P(\overline{R}) \times P_{\overline{R}}(G) = 0,25 \times 0,4 = 0,1$. Donc $P(G) = 0,15 + 0,1 = 0,25$.\\ La réponse correcte est \textbf{c.}
	
	\subsection*{Question 3}
La probabilité de tirer deux jetons rouges est $\left(\dfrac{3}{4}\right)^2 = \dfrac{9}{16} = 0,5625$.\\ La réponse correcte est \textbf{a.}
	

\subsection*{Question 4}
	
On a $P(X > 0) = P(X = 10) = 0,25$.\\ La réponse correcte est \textbf{d.}
	
	\subsection*{Question 5}

Le gain algébrique moyen en euros que peut espérer un joueur est égal à l’espérance mathématique de la variable aléatoire $X$ :
	\[
	E(X) = -5 \times 0,6 + 0 \times 0,15 + 10 \times 0,25 = -3 + 2,5 = -0,5
	\]
La réponse correcte est \textbf{b.}
	
