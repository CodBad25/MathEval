
\medskip

Cet exercice est un questionnaire à choix multiples. Pour chacune des questions suivantes,
une seule des quatre réponses proposées est exacte. Aucune justification n'est demandée.
Une bonne réponse rapporte un point. Une mauvaise réponse, une réponse multiple ou
l'absence de réponse ne rapporte ni n'enlève aucun point.

Indiquer sur la copie le numéro de la question et la réponse correspondante.

\medskip

\begin{enumerate}
\item  Dans le plan rapporté à un repère orthonormé, on considère la droite $\mathcal{D}$ d'équation
cartésienne $4x + 5y - 7 = 0$.

Un vecteur normal à $\mathcal{D}$ a pour coordonnées :

\medskip

\begin{tabularx}{\linewidth}{*{4}{X}}
\textbf{a.~~} $(5~;~4) $ &\textbf{b.~~} $(-5~;~4) $&\textbf{c.~~}$(4~;~5) $& \textbf{d.~~} $(4~;~-5)  $.
\end{tabularx}

\medskip

 \item Dans le plan rapporté à un repère orthonormé, l'ensemble E des points M de coordonnées
$(x~;~y)$ vérifiant : $x^2 - 2x +y^2 = 3$ est un cercle :

\medskip

\begin{tabularx}{\linewidth}{*{2}{X}}
\textbf{a.~~} de centre $A(1~;~0)$ et de rayon 2.  &\textbf{b.~~}  de centre $ A(1~;~0)$ et de rayon 4.\\
\textbf{c.~~}de centre $ A(-1~;~0)$ et de rayon 2.& \textbf{d.~~} de centre $A(-1~;~0)$ et de rayon 4. 
\end{tabularx}

\item La somme $15 + 16 + 17 + \dots + 243$ est égale à :

\medskip

\begin{tabularx}{\linewidth}{*{4}{X}}
\textbf{a.~~} \np{29403} &\textbf{b.~~} \np{29412}&\textbf{c.~~}$\np{29541} $& \textbf{d.~~} $\np{29646}  $.
\end{tabularx}

\item On considère la fonction $f$ dérivable définie sur $\R$ par $f(x) = (x + 1)\e^x$.

La fonction dérivée $f'$ de $f$ est définie sur $\R$ par :

\medskip

 \begin{tabularx}{\linewidth}{*{4}{X}}
\textbf{a.~~} $f'(x) = (x + 2)\e^x  $ &\textbf{b.~~} $f'(x) = (x + 1)\e^x $&\textbf{c.~~}$f'(x) = x\e^x $& \textbf{d.~~} $f'(x) = \e^x $.
\end{tabularx}
 
\item En utilisant l'arbre de probabilité pondéré ci-dessous, on obtient :

\medskip
	
\begin{minipage}[]{0.25\linewidth}
\renewcommand\arraystretch{2}
\begin{tabular}[]{l}
\textbf{a.~~} $p(B) =\dfrac{1}{4}$\\
\textbf{b.~~} $p(B) =\dfrac{2}{5}$\\
\textbf{c.~~} $p(B) =\dfrac{13}{20}$\\
\textbf{d.~~} $p(B) =\dfrac{3}{10}$\\
\end{tabular}
\end{minipage}\hfill
\begin{minipage}[]{0.6\linewidth}
\psset{nodesepA=0pt,nodesepB=3pt,treesep=0.75,labelsep=0.1pt,levelsep=2.75cm}
\pstree[treemode=R]{\TR{}}
{\pstree{\TR{$A$~~}\taput{$\frac{1}{3}$}}
	{
	\TR{$B$}\taput{$\frac{2}{5}$}
	\TR{$\overline{B}$}\tbput{$\frac{3}{5}$}
	}
\pstree{\TR{$\overline{A}$~~}\tbput{$\frac{2}{3}$}}
	{\TR{$B$}\taput{$\frac{1}{4}$}
	\TR{$\overline{B}$}\tbput{$\frac{3}{4}$}
	}
}
\end{minipage}
\end{enumerate}

\vspace{0,5cm}

