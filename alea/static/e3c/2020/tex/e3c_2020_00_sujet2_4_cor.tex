	\section*{Exercice 4 (5 points)}
	

\begin{center}
	\psset{unit=0.75cm,arrowsize=2pt 4}
	\begin{pspicture}(17.5,5.8)
		\psframe[fillstyle=solid,fillcolor=lightgray](17.5,5.8)
		\psline[linecolor=white,linewidth=1.5pt](0,1)(14.3,5.8)
		\psline[linecolor=white](1.35,1.4)(2.5,0.8)(9,3)(7.7,3.5)
		\psline(3.3,2.1)(3.3,3.2)(5.85,4.1)(5.85,2.9)(9.7,4.2)(16.7,1)(5.85,2.9)
		\psline(16.7,1)(3.3,2.1)
		\psline[linecolor=white,linestyle=dashed](3.3,2.1)(2.5,2.5)(5,3.4)(5.85,2.9)
		\psline[linestyle=dashed](3.3,3.2)(2.5,3.7)(5,4.5)(5.85,4.1)
		\psline[linestyle=dashed](2.5,2.5)(2.5,3.7)
		\psline[linestyle=dashed](5,3.4)(5,4.5)
		\psline{<->}(5.6,3)(9.6,4.4)\uput[u](7.6,3.7){9 m}
		\psline{<->}(10,4.3)(16.7,1.2)\uput[u](13.35,2.75){18 m}
		\psline(9.45,4.07)(9.7,3.95)(10,4.1)
		\uput[d](3.3,2){A}\uput[d](5.85,2.9){B}\uput[u](9.7,4.2){D}\uput[dr](16.7,1){T}
	\end{pspicture}
\end{center}
	
	\subsection*{1.}
	Les droites $(TD)$ et $(AB)$ sont perpendiculaires, les vecteurs $\overrightarrow{TD}$ et $\overrightarrow{DB}$ sont donc orthogonaux : leur produit scalaire est donc nul.
	
	\subsection*{2.}
	D’après la relation de Chasles : $\overrightarrow{TA} \cdot \overrightarrow{TB} = (\overrightarrow{TD} + \overrightarrow{DA}) \cdot \overrightarrow{TB} = \overrightarrow{TD} \cdot \overrightarrow{TB} + \overrightarrow{DA} \cdot \overrightarrow{TB}$.
	
	Or $\overrightarrow{TD} \cdot \overrightarrow{TB} = \overrightarrow{TD} \cdot \overrightarrow{TD} = TD^2 = 324$.
	
	D’autre part $\overrightarrow{DA} \cdot \overrightarrow{TB} = \overrightarrow{DA} \cdot \overrightarrow{DB} = (7,32 + 9) \times 9 = 146,88$.
	
	Finalement :
	\[
	\overrightarrow{TA} \cdot \overrightarrow{TB} = 324 + 146,88 = 470,88
	\]
	
	\subsection*{3. Méthode 1:}
	On sait que $\overrightarrow{TA} \cdot \overrightarrow{TB} = TA \times TB \times \cos (\widehat{ATB}) \quad(1)$.
	
	Or d’après le théorème de Pythagore appliqué au triangle $TDB$ rectangle en $D$, on a :
	\[
	AT^2 = AD^2 + TD^2 = 16,322 + 182 = 590,3424.
	\]
	L’égalité $(1)$ s’écrit donc :
	\[
	470,88 = \sqrt{405} \times \sqrt{590,3424} \times \cos (\widehat{ATB}), \text{ d’où }
	\cos (\widehat{ATB}) = \frac{470,88}{\sqrt{405} \times \sqrt{590,3424}} \approx 0,96301.
	\]
	
	La calculatrice donne $\widehat{ATB} \approx 15,63^{\circ}, \text{ soit } \widehat{ATB} \approx 15,6^{\circ} \text{ au dixième près.}$
	
	\subsection*{3. Méthode 2:}
	Dans le triangle $ATD$ rectangle en $D$ :\\
	$ \tan (\widehat{ATD}) = \frac{AD}{TD} = \frac{9+7,32}{18} \approx 0,90667, \text{ d’où } \widehat{ATD} \approx 42,1975°.$\\
	Dans le triangle $BTD$ rectangle en $D$ :\\
	$ \tan (\widehat{BTD}) = \frac{BD}{TD} = \frac{9}{18} = \frac{1}{2}, \text{ d’où } \widehat{BTD} \approx 26,5651^{\circ}.$\\
	On a donc par différence $\widehat{ATB} \approx 42,1975^{\circ} - 26,5651^{\circ} \approx 15,6324^{\circ} \text{ soit } 15,6^{\circ} \text{ au dixième près.}$
	
