
\medskip

Une commune compte $800$ habitants au début de l'année 2019. Le maire prévoit une baisse de 2\,\% par an du nombre d'habitants à partir de 2019.

Pour tout entier naturel $n$, on note $u_n$ le nombre d'habitants $n$ années après 2019. 

Ainsi, $u_0 = 800$ et pour tout entier naturel, $u_{n+1}= 0,98u_n$.

\medskip

\begin{enumerate}
\item Calculer $u_1$ et préciser ce que cette valeur représente dans le contexte de l'exercice.
\item Préciser la nature de la suite $\left(u_n.\right)$ ainsi que sa raison.
\item Déterminer, pour tout entier naturel $n$, l'expression de $u_n$ en fonction de $n$.
\item Calculer une valeur approchée, à l'entier près, du nombre d'habitants dans cette commune en 2025.
\item Recopier et compléter sur la copie la fonction écrite en langage Python ci-dessous, afin qu'elle permette de calculer, pour tout entier naturel $n$, le terme $u_n$.

\begin{center}
\begin{tabular}[]{|l|}
\hline
def u(n):\\ 
\hspace{1.3em}	u = $\dots$\\
\hspace{1.3em}	for i in range(1, $\dots$):\\
\hspace{2.8em}  u = \dots \\
\hspace{1.3em}return \dots\\
\hline
\end{tabular}
\end{center}
\end{enumerate}

\vspace{0,5cm}

