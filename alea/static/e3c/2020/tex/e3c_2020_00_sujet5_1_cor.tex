	

	\subsection*{Question 1}
	

	On a \(f(-x) = \sin(-x) - (-x) = -\sin(x) + x = -(\sin(x) - x) = -f(x)\)\\
	La fonction est impaire. Réponse b.
	
	\subsection*{Question 2}
	

	\(2\cos(x) - \sqrt{3} = 0 \iff \cos(x) = \dfrac{\sqrt{3}}{2}\) \\
	Il y a deux solutions : \(\dfrac{\pi}{6}\) et \(-\dfrac{\pi}{6}\).\\
	 Réponse a.
	
	\subsection*{Question 3}
	

	On a \(\overrightarrow{DA} \cdot \overrightarrow{DC} = DA \times DC \times \cos(\widehat{ \overrightarrow{DA}, \overrightarrow{DC}}) = 3 \times 4 \times \cos\left(\dfrac{2\pi}{3}\right) = 12 \times \left(-\dfrac{1}{2}\right) = -6\). Réponse d.
	
	\subsection*{Question 4}
	

	La droite \((d_1)\) a pour vecteur normal \(\overrightarrow{n_1}\begin{pmatrix}3\\ -4\end{pmatrix}\), donc un vecteur normal à \((d_2)\) est par exemple \(\overrightarrow{n_2}\begin{pmatrix}-4\\ -3\end{pmatrix}\).\\
	 Une équation de \((d_2)\) est donc \(4x + 3y + d = 0\) et comme \(A(1, 1) \in (d_2)\),\\ on a \(4 + 3 + d = 0 \iff d = -7\).\\ Une équation de \((d_2)\) est donc \(4x + 3y - 7 = 0\). \\Réponse b.
	
	\subsection*{Question 5}
	

	
	En écrivant l'équation de \((d')\) sous la forme \(2x - y - \dfrac{7}{2} = 0\), on voit que \((d)\) et \((d')\) sont parallèles et distinctes. \\
	Réponse c.
	
