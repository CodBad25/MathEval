
\medskip

Le directeur d'une maternité en milieu rural a enregistré $900$ accouchements entre le 1\up{er}
janvier 2019 et le 31 décembre 2019.

Depuis déjà 10 ans, il constate que le nombre d'accouchements baisse d'environ 4\,\% chaque
année par rapport à l'année précédente.

En supposant que cette diminution se poursuive avec ce même taux les prochaines années, il
modélise le nombre d'accouchements de cette maternité pour l'année $2019 + n$ à l'aide du
$n$-ième terme d'une suite $\left(u_n\right)$. Il a ainsi $u_0 = 900$.

\medskip

\begin{enumerate}
\item Montrer que la suite $\left(u_n\right)$ est une suite géométrique dont on précisera la raison.
\item On considère la fonction Suite définie ci-dessous en langage Python.
\begin{python}
1 def Suite(n):
2 	  u=900
3   for i in range(1,n+1):
4 	  u=0.96*u
5   return u 
\end{python}

Quelle sera la valeur obtenue pour Suite(5) ?
\item Pour tout entier naturel $n$, exprimer $u_n$ en fonction de $n$.
\item Le directeur sait que la maternité devra fermer dès le nombre d'accouchements
deviendra inférieur à 600.

Avec ce modèle, la maternité sera-t-elle fermée en 2030 ? Justifier.
\item Selon ce modèle, en quelle année la maternité fermera-t-elle ses portes ?
\end{enumerate}

\vspace{0,5cm}

