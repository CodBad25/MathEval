
\subsection*{1.}

Retrancher 25 \%, c'est multiplier par \(1 - \dfrac{25}{100} = 0{,}75\).

Donc :
\begin{align*}
&u_1 = u_0 \times 0{,}75 = 3 \times 0{,}75 = 2{,}25 \, \text{(m)}, \\
&u_2 = u_1 \times 0{,}75 = 2{,}25 \times 0{,}75 = 1{,}6875 \, \text{(m)}.
\end{align*}

\subsection*{2.}

Puisque pour tout \(n\), \(h_{n+1} = 0{,}75 h_n\), la suite n'est pas arithmétique mais géométrique de raison \(q = 0{,}75\) et de premier terme \(h_0 = 3\).

\subsection*{3.}

Voir ci-dessus.

\subsection*{4.}

On obtient \(h_6 = 0{,}5340\) soit \(0{,}53 \, \text{(m)}\) au cm près.

\subsection*{5.}

\begin{center}
\begin{python}
def seuil() :
    h = 3
    n = 0
    while h >= 0.1 :
        h = h * 0.75
        n = n + 1
    return n
\end{python}
\end{center}

Le script renverra la valeur \(n = 12\).

