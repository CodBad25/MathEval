
\medskip

\emph{Ce QCM comprend 5 questions.}

\emph{Pour chacune des questions, une seule des quatre réponses proposées est correcte.}

\emph{Les questions sont indépendantes.}

\emph{Pour chaque question, indiquer le numéro de la question et recopier sur la copie la
lettre correspondante à la réponse choisie.}

\emph{Aucune justification n’est demandée mais il peut être nécessaire d’effectuer des
recherches au brouillon pour aider à déterminer votre réponse.}

\emph{Chaque réponse correcte rapporte $1$ point. Une réponse incorrecte ou une question
sans réponse n’apporte ni ne retire de point.}

\medskip

\textbf{Question 1}

\medskip 

EFG est un triangle tel que EF = 8, FG = 5 et $\widehat{\text{EFG}} = \dfrac{3\pi}{4}$. 

Alors $\vv{\text{FE}}\cdot \vv{\text{FG}}$ est égal à :

\medskip

\begin{tabularx}{\linewidth}{*{4}{X}}
\textbf{a.~~} $20\sqrt{2}$ &\textbf{b.~~} $-20\sqrt{2}$&\textbf{c.~~}$20\sqrt{3}$& \textbf{d.~~}  $-20\sqrt{3}$
\end{tabularx}

\medskip

\textbf{Question 2}

\medskip 

Dans un repère orthonormé, on a tracé la courbe représentative d’une fonction $f$ et
sa tangente au point A d’abscisse 0.

\begin{center}\psset{unit=0.8cm,labelFontSize=\scriptstyle,showorigin=false}
\begin{pspicture}(-3,-2)(4.7,4.3)
\multido{\n=-2+1}{7}{\psline[linewidth=0.45pt](\n,-1.6)(\n,4.1)}
\multido{\n=-1+1}{6}{\psline[linewidth=0.45pt](-2.8,\n)(4.2,\n)}
\psaxes[linewidth=0.95pt]{->}(0,0)(-2.9,-1.9)(4.2,4.3)
\def\Func{x neg 2 add }
\psplot[plotpoints=1000,linewidth=1.25pt,linecolor=red]{-2}{3.4}{\Func}
\psplot[plotpoints=2000,linewidth=1.25pt,linecolor=blue]{-2.19}{4}{ x 2 add 2.71828 x exp div}
\psdot[dotstyle=+,dotscale=1.9,linecolor=green, dotangle=45](0,2)
\uput[ur](0,2){A}
\end{pspicture}
\end{center}

On note $f'$ la dérivée de la fonction $f$. On a :

\medskip

\begin{tabularx}{\linewidth}{*{4}{X}}
\textbf{a.~~} $f'(0) = 2$ &\textbf{b.~~} $f'(0) = -1$&\textbf{c.~~}$f'(2) = -1$& \textbf{d.~~} $f'(-2) = 0$
\end{tabularx}

\medskip

\textbf{Question 3}

\medskip 

On se place dans un repère orthonormé. Une équation du cercle de centre B$(2~;~3)$
et de rayon 4 est :

\medskip

\begin{tabularx}{\linewidth}{*{2}{X}}
\textbf{a.~~} $(x + 2)^2 + (y + 3)^2 = 4$ &\textbf{c.~~} $(x -2)^2 + (y - 3)^2 = 16$\\
\textbf{b.~~}$(x- 2)^2 + (y - 3)^2 = 4$& \textbf{d.~~} $(x + 2)^2 + (y + 3)^2 = 16$\\
\end{tabularx}

\medskip

\textbf{Question 4}

\medskip 

On se place dans un repère orthonormé du plan. On a tracé ci-dessous la courbe
représentative d’une fonction $f$ définie sur $\R$.
\begin{center}
\psset{xunit=0.9cm,yunit=0.9cm,showorigin=false}
\begin{pspicture}(-3.9,-3.75)(4.7,4.6)
\multido{\n=-3+1}{8}{\psline[linewidth=0.45pt](\n,-3.5)(\n,3.7)}
\multido{\n=-3+1}{7}{\psline[linewidth=0.45pt](-3.4,\n)(4.5,\n)}
\psaxes[linewidth=0.95pt]{->}(0,0)(-3.4,-3.5)(4.4,3.7)
\def\Func{x 2 add x 3 sub mul 0.5 mul }
\psplot[plotpoints=1000,linewidth=0.85pt,linecolor=blue]{-3.2}{4.1}{\Func}
\end{pspicture}
\end{center}

L’équation $f(x) = -3$ a pour solution(s) :

\medskip

\begin{tabularx}{\linewidth}{*{4}{X}}
\textbf{a.~~} $3$ &\textbf{b.~~} $0$&\textbf{c.~~}$-3$& \textbf{d.~~} $0$ et $1$
\end{tabularx}

\medskip

\textbf{Question 5}

\medskip 


Un vecteur normal à la droite d’équation cartésienne $-3x -2y + 5 = 0$ est :

\medskip

\begin{tabularx}{\linewidth}{*{4}{X}}
\textbf{a.~~} $ \vv{n}\ \dbinom{2}{-3}$ &\textbf{b.~~} $\vv{n}\ \dbinom{3}{-2}$&\textbf{c.~~}$\vv{n}\ \dbinom{-3}{2} $& \textbf{d.~~} $\vv{n}\ \dbinom{3}{2}$
\end{tabularx}

\vspace{0,5cm}

