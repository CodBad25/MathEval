	
	\subsection*{1.}
	
	On a \(P(A \cup B) = P(A) + P(B) - P(A \cap B)\). \\
	Comme \(A\) et \(B\) sont indépendants, \(P(A \cap B) = P(A) \times P(B) = 0,2 \times 0,5 = 0,1\), donc
	\[
	P(A \cup B) = P(A) + P(B) - P(A \cap B) = 0,5 + 0,2 - 0,1 = 0,6.
	\]
	
	\subsection*{2.}
	
	Les termes de la somme sont les onze premiers termes de la suite géométrique de premier terme 1 et de raison 1,2 :
	\[
	S = 1 + 1,2 + 1,2^2 + 1,2^3 + \cdots + 1,2^{10}
	\]
	
	On a donc :
	\[
	1,2S = 1,2 + 1,2^2 + \cdots + 1,2^{11}
	\]
	
	Par différence :
	\[
	0,2S = 1,2^{11} - 1 \quad \text{et par conséquent} \quad S = \dfrac{1,2^{11} - 1}{0,2} \approx 32,1504
	\]
	
	Soit environ 32,15 au centième près.
	
	\subsection*{3.}
	
	La fonction est \(f(x) = xe^{-x}\).
	
	On a \(g'(x) = 2e^x + (2x - 5)e^x = e^x(2 + 2x - 5) = e^x(2x - 3)\).
	
	\subsection*{4.}
	
	\[
	\dfrac{e^3 \times e^{-5}}{e^2} = \dfrac{e^{3-5}}{e^2} = \dfrac{e^{-2}}{e^2} = e^{-2-2} = e^{-4} = \dfrac{1}{e^4}
	\]
	
