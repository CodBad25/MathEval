
\subsection*{1.}

Avec \(x = 1\), on obtient :
\[
N(1) = 100\e^{-2} \approx 13{,}534,
\]
soit \(13{,}534\) millions de smartphones à mille près.

\subsection*{2.}

Avec \(x = 1\), on a vu que \(N(1) \approx 13{,}534\) et \(R(1) = 1 \times N(1) = N(1) \approx 13{,}534\).

Puis \(C(1) = 0{,}4 \times N(1) \approx 5{,}413\).

On a donc :
\[
B(1) = R(1) - C(1) \approx 13{,}534 - 5{,}413 = 8{,}121 \, \text{milliards d'euros}.
\]

\subsection*{3.}

On a :
\begin{align*}
B(x) &= R(x) - C(x) \\
&= xN(x) - 0{,}4N(x) \\
&= (x - 0{,}4)N(x) \\
&= (x - 0{,}4) \times 100\e^{-2x} \\
&= (100x - 40)\e^{-2x}.
\end{align*}

\subsection*{4.}

On sait que \(\e^{-2x} > 0\), quel que soit le réel \(x\), donc le signe de \(B'(x)\) est celui de \(180x - 200\).

\begin{align*}
&180x - 200 > 0 \\
\iff &180x > 200 \\
\iff &x > \dfrac{200}{180} \\
\iff &x > \dfrac{10}{9}.
\end{align*}

Conclusion : \(B'(x) > 0\) sur \(\left[0{,}4 \,;\, \dfrac{10}{9}\right]\), la fonction \(B\) est croissante sur cet intervalle, et \(B'(x) < 0\) sur \(\left[\dfrac{10}{9} \,;\, 4\right]\), la fonction \(B\) est décroissante sur cet intervalle.

\[
B\left(\dfrac{10}{9}\right) = (100 \times \dfrac{10}{9} - 40)\e^{-2 \times \dfrac{10}{9}} = 210\e^{-5} \approx 7{,}706,
\]
est le maximum de la fonction \(B\) sur \([0{,}4 \,;\, 2]\).

\subsection*{5.}

D'après la question précédente, le bénéfice est maximal pour \( x = \dfrac{10}{9} \approx 1{,}11111 \) (soit \(1111{,}11\) €) et ce bénéfice est d'environ \(7{,}7\) milliards d'euros.

