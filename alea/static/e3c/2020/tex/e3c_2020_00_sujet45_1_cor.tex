
\subsection*{Question 1}

\(\vec{u} \cdot \vec{v} = -2 \times 3 + 4 \times (-6) = -6 - 24 = -30\).

\subsection*{Question 2}

On a, d’après le théorème d’Al Kashi :
\begin{align*}
BC^2 &= AB^2 + AC^2 - 2 \times AB \times AC \times \cos(\widehat{BAC}) \\
&= 5^2 + 7^2 - 2 \times 5 \times 7 \times \dfrac{1}{2} \\
&= 25 + 49 - 35 \\
&= 39.
\end{align*}
Donc \(BC = \sqrt{39}\).

\subsection*{Question 3}

\begin{align*}
&M(x\,;\,y) \in \mathcal{C} \\
\iff &AM^2 = R^2 \\
\iff &(x - 2)^2 + (y - 3)^2 = 4^2 \\
\iff &x^2 + 4 - 4x + y^2 + 9 - 6y = 0 \\
\iff &x^2 + y^2 - 4x - 6y - 3 = 0.
\end{align*}

\subsection*{Question 4}

Le réel \(-\dfrac{23\pi}{3}\) a le même point image sur le cercle trigonométrique que le réel :
\[
\dfrac{-23\pi}{3} + 8\pi = \dfrac{-23\pi}{3} + \dfrac{24\pi}{3} = \dfrac{\pi}{3}.
\]

\subsection*{Question 5}

L’algorithme donne la liste : \([1, 5, 13, 29]\).

