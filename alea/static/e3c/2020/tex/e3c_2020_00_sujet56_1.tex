
\medskip

\emph{Ce QCM comprend $5$ questions indépendantes.}

\emph{Pour chacune d'elles, une seule des réponses proposées est exacte.}

\emph{Indiquer pour chaque question sur la copie la lettre correspondant à la réponse
choisie. Aucune justification n'est demandée.}

\emph{Chaque réponse correcte rapporte 1 point. Une réponse incorrecte ou une absence de réponse n'apporte ni ne retire de point.}

\medskip

\textbf{QUESTION 1}

\medskip

\begin{tabularx}{\linewidth}{*{4}{X}}
\textbf{a.~~} Si le discriminant
d'un polynôme
du second degré
est strictement
positif, alors ce
polynôme admet
2 racines
positives.&
\textbf{b.~~} Si le discriminant
d'un polynôme
du second degré
est strictement
négatif, alors ce
polynôme admet
2 racines
négatives.&
\textbf{c.~~} Si un polynôme
du second degré
est toujours
strictement
positif, alors ce
polynôme
n'admet pas de
racine.&
\textbf{d.~~}Si le discriminant
d'un polynôme
du second degré
est nul, alors ce
polynôme admet
le nombre 0 pour
racine.
\end{tabularx}

\medskip

\textbf{QUESTION 2}

\medskip

\begin{tabularx}{\linewidth}{*{4}{X}}
\begin{minipage}[t]{3.5cm}\textbf{a.~~}
L'équation

 $\cos x =-\frac{1}{2}$
admet 2 solutions dans l'intervalle
 $\left]-\frac{\pi}{2};\frac{\pi}{2}\right]$.
\end{minipage}&
\begin{minipage}[t]{3.5cm}\textbf{b.~~}
 L'équation
 
$\cos x= -\frac{1}{2}$
admet 1
solution dans
l'intervalle $[0 ; \pi[$.
\end{minipage}&
 \begin{minipage}[t]{3.5cm}\textbf{c.~~}
L'équation

$\sin x = -\frac{1}{2}$ admet 1
solution dans
l'intervalle \mbox{$[0 ; \pi[$}.
\end{minipage}&
 \begin{minipage}[t]{3.5cm}\textbf{d.~~}
\L'équation

$\sin x= -\frac{1}{2}$ admet 2
solutions dans
l'intervalle $\left]-\frac{\pi}{2} ;\frac{\pi}{2}\right]$.
\end{minipage}
\end{tabularx}

\medskip

\textbf{QUESTION 3}

\medskip

La courbe représentative d'une fonction $f$, définie et dérivable sur l'ensemble
des nombres réels, est donnée ci-dessous avec ses tangentes, aux points A
et B d'abscisses respectives 2 et 4. On note $f'$ la fonction dérivée de $f$.

\begin{center}
\psset{unit=0.5cm,labelFontSize=\scriptstyle,labelsep=0.1pt,showorigin=false}
\begin{pspicture}(-3.5,-5.8)(9,7)
\multido{\n=-3+1}{13}{\psline[linewidth=0.75pt,linecolor=lightgray](\n,-4.5)(\n,6.2)}
\multido{\n=-4+1}{11}{\psline[linewidth=0.75pt,linecolor=lightgray](-3.5,\n)(9.2,\n)}
\psaxes[linewidth=0.95pt,]{->}(0,0)(-3.5,-4.5)(9.5,6.4)
\psdots[dotstyle=+,dotscale =1.4,dotangle=45](2,-2) \psdots[dotstyle=+,dotscale =1.4,dotangle=45](4,-1)
\uput[dl](3.8,-1){B} \uput[dr](2,-2.1){A} 
\psplot[linewidth=1pt,linecolor=blue,plotpoints=5000]{-2.7}{5.7}{ x x x 0.125 neg mul 0.75 add mul 0.5 sub mul 3 sub }
\psplot[linewidth=1pt,linecolor=orange,plotpoints=3000]{-1}{9}{x  4 sub}
\psplot[linewidth=1pt,linecolor=red,plotpoints=3000]{-3}{9}{x 0.5 neg mul 1  add}
\end{pspicture}
\end{center}

\begin{tabularx}{\linewidth}{*{4}{X}}
\textbf{a.~~} $f(0) = 1$ &\textbf{b.~~} $f'(2) = 1$&\textbf{c.~~}$f'(2) = -2 $& \textbf{d.~~} $f'(4) = 0,5$.
\end{tabularx}
\medskip

\textbf{QUESTION 4}

\medskip

On considère la fonction $g$ définie sur l'ensemble des nombres réels $\R$ par :
$g(x)= x^3 - \np{0,0012}x + 1$.

\medskip

\begin{tabularx}{\linewidth}{*{4}{X}}
\textbf{a.~~} $g$ est strictement
croissante sur~$\R$.&
\textbf{b.~~} $g$ est croissante
sur~$R$.&
\textbf{c.~~} $g$ est constante
sur l'intervalle
\mbox{$[- 0,02~;~0,02]$}.&
\textbf{d.~~} $g$ est
décroissante sur
l'intervalle
\mbox{$[- 0,02~;~0,02]$}.
\end{tabularx}

\medskip

\textbf{QUESTION 5}

\medskip
\begin{tabularx}{\linewidth}{*{4}{X}}

\textbf{a.~~} L'équation
\mbox{$(\e^x)^2 = 1$} 
admet deux solutions
dans $\R$.

&\textbf{b.~~} L'ensemble de
définition de la
fonction
exponentielle est
\mbox{$]0~;~+ \infty[$}.

&\textbf{c.~~} La fonction
dérivée de la
fonction
$x\mapsto \e^{-x}$ est la
fonction $x \mapsto \e^{-x}$.

& \textbf{d.~~} L'ensemble de
définition de la
fonction
exponentielle est
$\R$.
\end{tabularx}

\vspace{0,5cm}

