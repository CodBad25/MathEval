
\subsection*{Question 1}

Le trinôme \(-3(x - 2)(x + 1)\) a pour racines \(-1\) et \(2\). Son coefficient \(a = -3 < 0\), donc il est négatif sauf entre les racines, d'où :
\[
S = ]-1 \,;\, 2[.
\]

\subsection*{Question 2}

Quel que soit le réel \(x\), on a :
\[
\cos(x + 3\pi) = \cos(x + \pi) = -\cos(x).
\]

\subsection*{Question 3}

\begin{align*}
&M(x\,;\,y) \in \textit{d} \\
\iff &\overrightarrow{AM} \cdot \vec{v} = 0 \\
\iff &2(x - 1) - 3(y - 2) = 0 \\
\iff &2x - 3y + 4 = 0 \\
\iff &3y = 2x + 4 \\
\iff &y = \frac{2}{3}x + \frac{4}{3}.
\end{align*}

\subsection*{Question 4}

On a :
\[
f'(x) = \frac{2x(x + 1) - 1 \times x^2}{(x + 1)^2} = \frac{2x^2 + 2x - x^2}{(x + 1)^2} = \frac{x^2 + 2x}{(x + 1)^2}.
\]
Le coefficient directeur de la tangente à \(\mathcal{C}\) au point d'abscisse 1 est :
\[
f'(1) = \frac{1^2 + 2 \times 1}{(1 + 1)^2} = \frac{3}{4}.
\]

\subsection*{Question 5}

\begin{align*}
&x^2 - 2x + y^2 + 4y = 4 \\
\iff &(x - 1)^2 - 1^2 + (y + 2)^2 - 4 = 4 \\
\iff &(x - 1)^2 + (y + 2)^2 = 9 \\
\iff &AM^2 = 3^2.
\end{align*}
Cela montre que les points \( M \) appartiennent au cercle de centre \( A(1\,;\,-2) \) et de rayon 3.

