
\subsection*{1.}

Baisser de 8\% revient à multiplier par :
\(1 - \dfrac{8}{100} = 1 - 0,08 = 0,92\).

Donc : \(u_1 = 2 \times 0,92 = 1,84\).

\subsection*{2.}

\paragraph{a.} On a vu que l'on passe d'un terme au suivant en le multipliant par \(0,92\).  

On a donc, quel que soit \(n \in \mathbb{N}\), \(u_{n+1} = 0,92 u_n\).

\paragraph{b.} Le résultat précédent montre que la suite \((u_n)\) est une suite géométrique de raison \(0,92\) et de premier terme \(u_0 = 2\).

\subsection*{3.}

\paragraph{a.}
\begin{center}
\begin{python}
def volMedicament(S) :
	u = 2
	n = 0
	while u > S :
		u = u * 0.92
		n = n + 1
return n
\end{python}
\end{center}

\paragraph{b.} Il faut saisir \texttt{volMedicament(1,5)}.

On obtient \(n = 5\) : au bout de ce temps, il ne reste qu'environ \(1,43 \, \text{cm}^3\) de médicament.

