
\medskip

Le président d'un club de handball a constaté une augmentation du nombre d'adhérents dans
son club depuis 2016 (toutes catégories confondues). En effet en 2016, il y avait $377$ adhérents,
$396$ en 2017 et $416$ en 2018. Ce qui correspond à une hausse chaque année d'environ 5\,\%.

Il souhaite faire une estimation pour les années à venir, en supposant que cette hausse de 5\,\%
par an se poursuit.

On modélise le nombre d'adhérents l'année 2018 + $n$ par la suite de terme général $u_n$.

On a donc $u_0 = 416$.

\medskip

\begin{enumerate}
\item Calculer $u_1$ et $u_2$. Arrondir les résultats à l'unité.
\item Quelle est la nature de la suite $\left(u_n\right)$ ? Préciser son premier terme et sa raison.
\item Exprimer $u_n$ en fonction de $n$, pour tout entier naturel $n$.
\item Calculer $u_7$. Interpréter ce résultat par rapport aux données de l'énoncé.
\item À partir de quelle année le président du club peut-il espérer dépasser les $700$ adhérents ?
\end{enumerate}

\vspace{0,5cm}

