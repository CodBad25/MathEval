
\medskip
 
À l'issue d'une étude conduite pendant plusieurs années, on modélise l'évolution du prix du m$^2$ d'un appartement neuf dans une ville française de la manière suivante:

À partir d'un prix de \np{4200}~\euro{} le m$^2$ en 2019, on applique chaque année une augmentation annuelle de 3\,\%.

\medskip

\begin{enumerate}
\item Avec ce modèle, montrer que le prix du m$^2$ d'un appartement neuf dans cette ville en 2021 serait de \np{4455,78}~\euro.
\item  On considère la suite de terme général $u_n$ qui permet d'estimer, avec ce modèle, le prix en euro du m$^2$ d'un appartement neuf l'année $2019 + n$. On a donc $u_0 = \np{4200}$.
	\begin{enumerate}
		\item Quelle est la nature de la suite $\left(u_n\right)$ ? En préciser la raison.
		\item En déduire l'expression du terme général $u_n$ en fonction de $n$, pour tout entier naturel~$n$.
		\item Selon ce modèle, pourra-t-on acheter en 2024, un appartement de $40$~m$^2$ si l'on dispose d'une somme de \np{200000}~\euro ?
	\end{enumerate}
\item  On définit, en langage Python, la fonction seuil ci-dessous.

\begin{center}
\fbox{
\begin{tabularx}{0.5\linewidth}{c X}%\hline
1& \texttt{def seuil():}\\
2&\qquad \texttt{u = 4200} \\
3&\qquad \texttt{n = 0 }\\
4&\qquad \texttt{while u <= 8000 : }\\
5&\qquad \qquad \texttt{u = \ldots} \\
6&\qquad \qquad \texttt{n = n+1 }\\
7&\qquad \texttt{return \ldots}\\% \hline
\end{tabularx}
}
\end{center}
 
Recopier et compléter les lignes 5 et 7 de sorte que cette fonction renvoie le nombre d'années nécessaires pour que, selon ce modèle, le prix du m$^2$ d'un appartement neuf dépasse \np{8000}~\euro.

\end{enumerate}

\bigskip

