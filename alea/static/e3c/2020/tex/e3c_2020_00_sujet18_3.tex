  
\medskip

\emph{Dans cet exercice toutes les probabilités seront données sous forme décimale, arrondie au
millième.}
\medskip


Une entreprise récupère des smartphones endommagés, les répare et les reconditionne afin de
les revendre à prix réduit.

\begin{itemize}
\item  45\,\% des smartphones qu’elle récupère ont un écran cassé ; 
\item  parmi les smartphones ayant un écran cassé, 30\,\% ont également une batterie
défectueuse ;
 \item par contre, seulement 20\,\% des smartphones ayant un écran non cassé ont une batterie
défectueuse.
\end{itemize}
\begin{enumerate}
\item Un technicien chargé de réparer et reconditionner les smartphones de l’entreprise prend un
smartphone au hasard dans le stock. On note :

\begin{itemize}
\item $E$ l’évènement : \og Le smartphone choisi a un écran cassé \fg.
\item $B$ l’évènement : \og Le smartphone choisi a une batterie défectueuse \fg.
\end{itemize}

\medskip

\begin{enumerate}
\item Représenter la situation décrite ci-dessus par un arbre pondéré.
\item Démontrer que la probabilité que le smartphone choisi ait une batterie défectueuse est
égale à $0,245$.
\item Sachant que le smartphone choisi a une batterie défectueuse, quelle est la probabilité
qu’il ait un écran cassé ?
\end{enumerate}
\item L’entreprise dépense 20 \euro{} pour réparer et reconditionner chaque smartphone qu’elle
 récupère. Si l’écran est cassé, elle dépense 30 \euro{} supplémentaires, et si la batterie est
défectueuse, elle dépense 40 \euro{} supplémentaires.

On note $X$ la variable aléatoire égale au coût total de réparation et reconditionnement d’un
smartphone choisi au hasard dans le stock.
\begin{enumerate}
\item Recopier et compléter sur la copie (aucune justification n’est attendue) le tableau
suivant pour donner la loi de probabilité de la variable aléatoire $X$.

\begin{center}
\begin{tabularx}{\linewidth}{|*{5}{>{\centering \arraybackslash} X|}} \hline
 $x_i$		&20		&50			&$\cdots$	&$\cdots$\\ \hline
 $p(X=x_i)$	&0,44	&$\cdots$  	&$\cdots$ 	&$\cdots$ \\ \hline
\end{tabularx}
\end{center}
\item L’entreprise doit réparer et reconditionner $500$ smartphones. Combien doit-elle
s’attendre à dépenser ?
\end{enumerate}
\end{enumerate}

\vspace{0,5cm}

