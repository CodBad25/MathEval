
\medskip

Un biologiste étudie une population de bactéries dans un milieu fermé. À l'instant initial, il y a \np{10 000} bactéries et la population augmente de 15\,\% par heure.

On modélise la situation par une suite $(u_n)$ pour laquelle, pour tout entier naturel $n$, $u_n$ représente une estimation du nombre de bactéries au bout de $n$ heures.
On a donc $u_0= \np{10000}$.
\begin{enumerate}
\item  Expliquer pourquoi la suite $\left(u_n\right)$ vérifie pour tout entier naturel $n$ :

$u_n=\np{10 000} \times 1,15^n$.
\item Quelle est la nature de la suite $\left(u_n\right)$. On précisera le premier terme et la raison.
\item Combien y aura-t-il de bactéries au bout de 10 heures ?
\item On considère la fonction suivante définie en langage Python.

\begin{python}
def bacteries(N) :
u=10000
	for i in range(N) :
u=u*1.15
	return u
\end{python}

On a appelé cette fonction en donnant différentes valeurs au paramètre $n$ et l'on a dressé le tableau suivant.

\begin{center}
\begin{tabular}[]{|m{7em}|*{4}{c|}}\hline
$n$ &10&  100& \np{1 000}& \np{10 000}\\\hline
Bactéries (N)& \np{40 455} &$1,2 \times 10^{10}$ &  $4,99 \times 10^{64}$ & $3,052 \times 10^{307}$\\
\hline
\end{tabular}
\end{center}

Quelle interprétation peut-on donner de ces résultats dans le contexte de l'exercice ?
\item Lorsque la population atteint \np{200000} bactéries, le biologiste répand un désinfectant afin de tester son efficacité. Une heure plus tard, il reste \np{4000}  bactéries. Quel est le pourcentage de diminution du nombre de bactéries ?
\end{enumerate}

\vspace{0,5cm}

