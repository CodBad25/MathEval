
\medskip

Un service de vidéos à la demande réfléchit au lancement d'une nouvelle série mise en ligne chaque semaine et qui aurait comme sujet le quotidien de jeunes gens favorisés.

Le nombre de visionnages estimé la première semaine est de \np{120000}. Ce nombre augmenterait ensuite de 2\,\% chaque semaine.

Les dirigeants souhaiteraient obtenir au moins \np{400000} visionnages par semaine.

On modélise cette situation par une suite $\left(u_n\right)$ où $u_n$ représente le nombre de visionnages $n$ semaines après le début de la diffusion. On a donc $u_0 = \np{120000}$.

\medskip

\begin{enumerate}
\item Calculer le nombre $u_1$ de visionnages une semaine après le début de la diffusion. 
\item Justifier que pour tout entier naturel $n$ : $u_n = \np{120000} \times  1,02^n$.
\item À partir de combien de semaines le nombre de visionnages hebdomadaire sera-t-il supérieur à \np{150000} ?
\item Voici un algorithme écrit en langage Python:

\begin{center}
\begin{tabularx}{0.55\linewidth}{X}
\texttt{def seuil():}\\
\quad \texttt{u = 120000}\\
\quad \texttt{n = 0}\\
\quad \texttt{while u < 400000:}\\
\qquad \texttt{n = n+1}\\
\qquad \texttt{u = 1.02*u}\\
\quad \texttt{return n}\\
\end{tabularx}
\end{center}

Déterminer la valeur affichée par cet algorithme et interpréter le résultat précédent
dans le contexte de l'exercice. 

\item On pose pour tout entier naturel $n$ : $S_n = u_0 + \ldots + u_n$. Montrer que l'on a :

\[S_n = \np{6000000} \times \left(1,02^{n+1} - 1\right).\]

En déduire le nombre total de visionnages au bout de $52$ semaines (arrondir à
l'unité).

\end{enumerate}
