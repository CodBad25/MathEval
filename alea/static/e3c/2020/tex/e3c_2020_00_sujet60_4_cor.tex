
\subsection*{1.}

\paragraph{a.} Le point \( A' \), milieu de \([BC]\), a pour coordonnées :
\[
\left( \dfrac{6 + 0}{2} \,;\, \dfrac{0 + 6}{2} \right) \text{ soit } (3 ; 3).
\]
Ainsi, le rayon du cercle de centre \( I \) et passant par \( A' \) est :
\begin{align*}
R &= \sqrt{(3 - 1)^2 + (3 - 2)^2} \\
&= \sqrt{2^2 + 1^2} \\
&= \sqrt{5}.
\end{align*}

\paragraph{b.} Une équation du cercle \( \Gamma \) est donc :
\[
(x - 1)^2 + (y - 2)^2 = (\sqrt{5})^2 \text{ soit } (x - 1)^2 + (y - 2)^2 = 5.
\]

\subsection*{2.}

\paragraph{a.} On a : 
\[
(0 - 1)^2 + (0 - 2)^2 = 1 + 4 = 5
\]
Donc \( O(0\,;\,0) \) appartient à \( \Gamma \).

\paragraph{b.} Montrons tout d'abord que le point \( H \) de coordonnées \( (2\,;\,4) \) appartient à la droite \( (BC) \).  

On a \( \overrightarrow{BC} \begin{pmatrix} -6 \\ 6 \end{pmatrix} \) et \( \overrightarrow{BH} \begin{pmatrix} -4 \\ 4 \end{pmatrix} \).

Ainsi :
\[
\overrightarrow{BH} = \dfrac{2}{3} \overrightarrow{BC}.
\]
Ces deux vecteurs sont colinéaires et donc le point \( H \) appartient à la droite \( (BC) \).

\[
\overrightarrow{AH} \begin{pmatrix} 4 \\ 4 \end{pmatrix} \quad \text{et} \quad \overrightarrow{AH} \cdot \overrightarrow{BC} = -6 \times 4 + 6 \times 4 = 0.
\]
Ainsi \( (AH) \) et \( (BC) \) sont perpendiculaires.  
Par conséquent, \( H \) est le pied de la hauteur issue de \( A \).  
Donc \( H_A \) a bien pour coordonnées \( (2\,;\,4) \).

\paragraph{c.} On a :
\[
(2 - 1)^2 + (4 - 2)^2 = 1 + 4 = 5
\]
Donc \( H_A \) est sur le cercle \( \Gamma \).

