
\medskip

Ce QCM comprend 5 questions indépendantes. Pour chacune d'elles, une seule des réponses
proposées est exacte.

Indiquer pour chaque question sur la copie la lettre correspondant à la réponse choisie.
Aucune justification n'est demandée.

Chaque réponse correcte rapporte 1 point. Une réponse incorrecte ou une absence de
réponse n'apporte ni ne retire de point.

\medskip

\textbf{Question 1}

\medskip 

Pour tout entier naturel $n$, on définit la suite $\left(u_n\right)$ par : $u_n= 3 \times \dfrac{10^n}{2^{n+1}}$

La suite $\left(u_n\right)$ est une suite :

\begin{tabularx}{\linewidth}{*{2}{X}}
\textbf{A.~~} arithmétique de
raison 3.&\textbf{B.~~} géométrique de
raison 3.\\\textbf{C.~~}arithmétique de
raison 5.& \textbf{D.~~}géométrique de
raison 5. .
\end{tabularx}

\medskip

\textbf{Question 2}

\medskip 

Dans un repère orthonormé \Oij{} du plan, on considère les points $A(-2~;~ 1)$ et $B(2; 4)$.

La droite $\Delta$ passe par le point $C(-1~;~1)$ et admet le vecteur $\vv{AB}$ pour vecteur normal.

La droite $\Delta$ admet pour équation cartésienne :

\medskip

\begin{tabularx}{\linewidth}{*{4}{X}}
\textbf{A.~~} $3x - 4y + 7 = 0$ &\textbf{B.~~} $ 4x + 3y + 1 = 0$&\textbf{C.~~}$ 3x - 4y - 1 = 0$& \textbf{D.~~} $4x + 3y + 7 = 0$.
\end{tabularx}

\medskip

\textbf{Question 3}

\medskip 

Dans l'intervalle $\left[0~;~\dfrac{\pi}{2}\right]$, l'unique solution de l'équation : $2\cos (x + \pi) + 1 = 0$ est :

\begin{tabularx}{\linewidth}{*{4}{X}}
\textbf{A.~~} $\dfrac{\pi}{3}$ &\textbf{B.~~} $-\dfrac{5\pi}{3}$&\textbf{C.~~}$\dfrac{\pi}{6}$& \textbf{ D.~~} $\dfrac{2\pi}{3}$.
\end{tabularx}

\medskip

\textbf{Question 4}

\medskip 

On considère la fonction $f$ définie et dérivable sur $\R$ par : $f(x)=\dfrac{\e^x}{ 1+\e^x}$

La fonction dérivée $f'$ de la fonction $f$ est définie par :

\medskip

\begin{tabularx}{\linewidth}{*{4}{X}}
\textbf{A.~~} $f'(x)=\dfrac{\e}{1+\e}$ &\textbf{B.~~} $f'(x)=\dfrac{\e^x}{(1+\e^x)^2}$&\textbf{C.~~}$f'(x)=1$& \textbf{D.~~} $f'(x)=\dfrac{-\e^x}{(1+\e^x)^2}$.
\end{tabularx}

\medskip

\textbf{Question 5}

\medskip 

On considère la fonction $f$ définie sur $R$ par : $f(x) = -0,5(x + 2)^2 + 4,5$.

On peut affirmer que :

\begin{center}
A.
\end{center}

Le tableau de variations de la fonction $f$ est donné ci-dessous :

\begin{center}
\begin{pspicture}(6.75,1.8)
\psframe(6.65,1.7)
\psline(0,1.24)(6.65,1.24)  \psline(1.2,0)(1.2,1.7)
\uput[u](0.7,1.2){$x$} \uput[u](1.58,1.2){$-\infty$}\uput[u](4.15,1.2){$2$}\uput[u](6.3,1.2){$+\infty$}
\uput[d](0.7,0.9){$f(x)$}\uput[d](4.1,1.2){4,5}
\psline{->}(2,0.2)(3.4,0.85)\psline{->}(4.65,0.85)(5.8,0.2)
\end{pspicture}
\end{center}

\begin{center}
B.
\end{center}

La courbe représentative de la fonction $f$ admet un sommet de coordonnées $(4,5~;~-2)$.

\begin{center}
C.
\end{center}
Le signe de $f(x)$ est donné ci-dessous :

\smallskip
\begin{center}
\begin{tabular}{|c|l m{0.5cm}c m{0.5cm}c m{0.5cm} r|}
\hline
$x$&$-\infty$& &$-5$&&1&&$+\infty$\\ \hline
$f(x)$&&\centering $-$&&\centering +&&\centering $-$&\\
\hline
\end{tabular}
\end{center}

\begin{center}
D.
\end{center}
La fonction $f$ admet un minimum en $-2$ égal à 4,5.

\vspace{0,5cm}

