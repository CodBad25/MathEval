
\medskip

Une entreprise produit du tissu.

Le coût total de production (en \euro) de l'entreprise est modélisé par la fonction

\[C(x)= 15x^3 - 120x^2 + 500x + 750\] 

où $x$ est la longueur de tissu fabriqué exprimée en kilomètre, $x$ étant compris entre 0 et 10.

Chaque kilomètre de tissu est vendu $680$ \euro.

On note $B(x)$ le résultat de l'entreprise, c’est-à-dire la différence entre la recette et le coût de production, pour la vente de $x$ kilomètres de tissu.

\medskip

\begin{enumerate}
\item Quel est le résultat de l’entreprise pour la vente de 3 kilomètres de tissu ?
\item Montrer que : $B(x)=-15x^3+120x^2+180x-750$.
\item Donner une expression de $B'(x)$, où $B'$ est la fonction dérivée de la fonction $B$.
\item Dresser le tableau de signes de $B'(x)$ sur $[0~;~10]$ puis le tableau de variations de la fonction $B$.
\item Combien de kilomètres de tissu l’entreprise doit-elle produire afin d’obtenir un résultat maximal ?
\end{enumerate}

