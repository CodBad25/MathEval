
\medskip

Dans le repère ci-dessous, on note $\mathcal{C}_f$ la courbe représentative d'une fonction $f$ définie sur l'intervalle $[-10~;~2]$.

On a placé dans ce repère les points $A~(0~;~2$), $B~(2~;~0)$ et $C~(-2~;~0)$.

On dispose des renseignements suivants :
\begin{itemize}
\item Le point B appartient à la courbe $\mathcal{C}_f$.
\item La droite (AC) est tangente en A à la courbe $\mathcal{C}_f$.
\item La tangente à la courbe $\mathcal{C}_f$ au point d'abscisse 1 est une droite parallèle à l'axe des abscisses.
\end{itemize}


\psset{unit=1.05cm,labelFontSize=\scriptstyle, showorigin=false}
\begin{pspicture}(-10.35,-1.3)(2.7,3)
\multido{\n=-10.33333333+0.33333333}{40}{\psline[linewidth=0.3pt,linecolor=lightgray](\n,-1.33)(\n,3)}
 \multido{\n=-1.3333333+0.3333333}{14}{\psline[linewidth=0.3pt,linecolor=lightgray](-10.4,\n)(2.7,\n)}
% \multido{\n=-10+1}{11}{\psline[linewidth=0.6pt](\n,-1.33)(\n,3)}
 %\multido{\n=-1+1}{5}{\psline[linewidth=0.6pt](-10.4,\n)(2.7,\n)}
 \psaxes[linewidth=1.25pt]{->}(0,0)(-10.33,-1.33)(2.7,3)
\def\Func{2.71828 x exp 2 x sub mul}
 \psplot[plotpoints=3000,linewidth=1.25pt,linecolor=green]{-10}{2}{\Func}
 \psplot[plotpoints=3000,linewidth=0.85pt,linecolor=red]{-10.333}{2.3}{2.71828}
 \psplot[plotpoints=3000,linewidth=0.85pt,linecolor=blue]{-3.4}{1.02}{x 2 add}
 \psdots[dotstyle=Mul,dotscale=2.6](0,2)(2,0)(-2,0)
 \uput[ur](2,0){B} \uput[r](0,2){A} \uput[u](-2,0){C} \uput[dl](0,0){O} 
 \uput[dr](1.8,1.32){\blue $\displaystyle \mathcal{C}_f$}
\end{pspicture}

\begin{enumerate}
\item Déterminer la valeur de $f'(1)$.
\item Donner une équation de la tangente à la courbe $\mathcal{C}_f$ au point A.
\end{enumerate}
On admet que cette fonction $f$ est définie sur $[-10~;~2]$ par \[f(x)=(2-x)\e^x.\]
\begin{enumerate}[resume]
\item Montrer que pour tout réel $x$ appartenant à l'intervalle $[-10~;~2]$,

\[f'(x)=(-x+1)e^x.\]

\item En déduire le tableau de variations de la fonction $f$ sur l'intervalle $[-10 ;2]$.
\item Déterminer une équation de la tangente à la courbe $\mathcal{C}_f$ au point B.
\end{enumerate}

\vspace{0,5cm}

