
\medskip

Un jeu est organisé à partir d'un sac contenant 6 jetons rouges et 4 jetons noirs. Les jetons
sont indiscernables au toucher.

Un joueur prend deux jetons au hasard dans le sac selon le déroulé suivant :
\begin{itemize}
\item le joueur prend un premier jeton au hasard dans le sac et il met le jeton de côté ;
\item le joueur prend un second jeton au hasard dans le sac et il met le jeton de côté.
\end{itemize}

On note :
\begin{itemize}
\item $R_1$ l'évènement \og le premier jeton tiré est de couleur rouge \fg;
\item $R_2$ l'évènement \og le second jeton tiré est de couleur rouge \fg.
\end{itemize}

\medskip

\begin{enumerate}
\item  Recopier sur la copie et compléter l'arbre ci-dessous :

\begin{center}

\psset{nodesepA=0pt,nodesepB=3pt,treesep=0.75,labelsep=0.1pt,levelsep=2.75cm}
\pstree[treemode=R]{\TR{}}
{\pstree{\TR{$R_1$~~}\taput{$\dots$}}
	{
	\TR{$R_2$}\taput{$\dots$}
	\TR{$\overline{R_2}$}\tbput{$\dots$}
	}
\pstree{\TR{$\overline{R_1}$~~}\tbput{$\dots$}}
	{\TR{$R_2$}\taput{$\dots$}
	\TR{$\overline{R_2}$}\tbput{$\dots$}
	}
}
\end{center}

\item On considère l'évènement $A$  \og le joueur obtient deux jetons de couleur rouge \fg.
	\begin{enumerate}
		\item Déterminer la probabilité $p(A)$.
		\item Décrire l'évènement contraire de l'évènement $A$ par une phrase de la forme

\hspace{7em}\og le joueur obtient $\ldots$ \fg.
		\item Montrer que la probabilité que le second jeton tiré soit de couleur rouge est égale à $0,6$.
		\item Le second jeton tiré est de couleur noire. Que peut-on alors penser de l'affirmation suivante :

\og il y a plus de 50\,\% de chance que le premier jeton tiré ait été de couleur rouge \fg{} ?

Justifier la réponse.
	\end{enumerate}
\end{enumerate}







