\section*{Exercice 2 (5 points)}

\subsection*{1. Justifier que pour tout réel $x$ appartenant à $[0 ; 9]$ : $V(x) = 4x^3 - 84x^2 + 432x$.}
La longueur est égale à $24 - 2x$ et la largeur $18 - 2x$. La hauteur valant $x$, le volume de la boîte sans couvercle est égal à :
\[V(x) = L \times l \times h = (24 - 2x)(18 - 2x) \times x = x(432 - 48x - 36x + 4x^2) = 4x^3 - 84x^2 + 432x	\]
	
\subsection*{2. On note $V'$ la fonction dérivée de $V$ sur $[0 ; 9]$. Donner l’expression de $V'(x)$ en fonction de $x$.}
La fonction polynôme $V$ est dérivable sur $\mathbb{R}$ et donc sur $[0 ; 9]$,
	\[	V'(x) = 12x^2 - 168x + 432 = 12(x^2 - 14x + 36)	\]
	
\subsection*{3. Dresser alors le tableau de variations de $V$ en détaillant la démarche.}
Avec $\Delta = 14^2 - 4 \times 36 = 196 - 144 = 52$, le trinôme a deux racines :
	\[
	x_1 = \dfrac{14 + \sqrt{52}}{2} = 7 + \sqrt{13} \quad \text{et} \quad x_2 = \dfrac{14 - \sqrt{52}}{2} = 7 - \sqrt{13}
	\]
On sait que ce trinôme est positif sauf sur $]7 - \sqrt{13} ; 7 + \sqrt{13}[$. Comme $\sqrt{13} \approx 3,6$, $7 + \sqrt{13} \approx 10,6$. Donc :
	\begin{itemize}
		\item sur $]7 - \sqrt{13} ; 7 + \sqrt{13}[$, $V'(x) < 0$, donc $V$ est décroissante ;
		\item sur $[0 ; 7 - \sqrt{13}]$ et sur $[7 + \sqrt{13} ; 9]$, $V'(x) > 0$, donc $V$ est croissante.
	\end{itemize}
	
	\subsection*{4. Pour quelle(s) valeur(s) de $x$ la contenance de la boîte est-elle maximale ?}
	
	D’après la question précédente, $V(7 - \sqrt{13}) \approx 654,98 \approx 655 \text{ cm}^3$ est la contenance maximale de la boîte.
	
	\subsection*{5. L’industriel peut-il construire ainsi une boîte dont la contenance est supérieure ou égale à 650 cm$^3$ ? Justifier.}
	
	Oui, puisque la capacité maximale est proche de $655$.\\
	 Pour avoir une capacité de $650$ cm$^3$, il faut prendre $x \approx 3,06$ cm.
	
