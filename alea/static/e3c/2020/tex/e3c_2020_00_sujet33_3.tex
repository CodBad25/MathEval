
\medskip
Un libraire dispose d'un stock de magazines. On sait que \qty{40}{\percent} des magazines provient d'un fournisseur
A et le reste d'un fournisseur B.

Il constate que \qty{91}{\percent} des magazines reçus sont vendus dans la semaine.

Il constate également que \qty{85}{\percent} des magazines provenant du fournisseur A sont vendus dans la semaine.

Le responsable des achats prend au hasard un magazine dans le stock. On considère les évènements
suivants:
\begin{itemize}
\item $A$: «le magazine provient du fournisseur A».
\item $B$: «le magazine provient du fournisseur B».
\item $S$: «le magazine est vendu dans la semaine».
\end{itemize}
Pour tout évènement $E$, on note $\overline{E}$ l'évènement contraire de $E$.

Pour tout évènement $E$ et $F$ où $F$ est un évènement de probabilité non nulle, la probabilité de $E$ sachant
$F$ est notée $p_F(E)$.
\begin{enumerate}
\item Quelle est la probabilité que le magazine provienne du fournisseur B?
\item On note $p_B(S)=x$, $x\in\interval{0}{1}$. Recopier et compléter sur la copie avec les trois valeurs demandées
l'arbre pondéré ci-dessous traduisant la situation:

\begin{center}
\psset{nodesepA=0pt,nodesepB=3pt,treesep=0.75,labelsep=0.1pt,levelsep=2.5cm}
\pstree[treemode=R]{\TR{}}
{\pstree{\TR{$A$~~}\naput{$\cdots$}}
	{
	\TR{$S$}\naput{\num{0.85}}
	\TR{$\overline{S}$}\nbput{$\dots\mathstrut$}
	}
\pstree{\TR{$B$~~}\nbput{$\dots\mathstrut$}}
	{\TR{$S$}\naput{$x\mathstrut$}
	\TR{$\overline{S}$}\tbput{}
	}
}
\end{center}

\item Calculer la probabilité que le magazine choisi au hasard provienne du fournisseur A et qu'il soit
vendu dans la semaine.
\item Démontrer que $\num{0,34}+\num{0,6}x=\num{0,91}$. En déduire que $(B\cap S)=\num{0,57}$.
\item Le magazine choisi est vendu dans la semaine. Calculer la probabilité qu'il provienne du fournisseur
B. En donner sa valeur arrondie à $10^{-3}$.
\end{enumerate}

\medskip
