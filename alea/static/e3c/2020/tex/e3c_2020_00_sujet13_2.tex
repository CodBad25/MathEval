
\medskip

En 2000, la production mondiale de plastique était de $187$ millions de tonnes. On suppose que depuis 2000, cette production augmente de 3,7\,\% chaque année.

On modélise la production mondiale de plastique, en millions de tonnes, produite en l'année $2000 + n$ par la suite de terme général $u_n$ où $n$ désigne le nombre d'années à partir de l'an 2000. 

Ainsi, $u_0 = 187$.

\medskip

\begin{enumerate}
\item Montrer que la suite $\left(u_n\right)$ est une suite géométrique dont on donnera la raison.
\item Pour tout $n \in \N$,exprimer $u_n$ en fonction de $n$.
\item Étudier le sens de variation de la suite $\left(u_n\right)$.
\item  Selon cette estimation, calculer la production mondiale de plastique en 2019. Arrondir au million de tonnes.
\item Des études montrent que 20\,\% de la quantité totale de plastique se retrouve dans les océans, et que 70\,\% de ces déchets finissent par couler.

Montrer que la quantité totale, arrondie au million de tonnes, de déchets flottants sur l'océan dus à la production de plastique de 2000 à 2019 compris est de $324$ millions de tonnes.
\end{enumerate}

\bigskip

