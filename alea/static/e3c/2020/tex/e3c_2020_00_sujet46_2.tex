
\medskip

Maxime participe à un jeu qui se déroule en deux parties :
\begin{itemize}
\item La probabilité qu'il gagne la première partie est de 0,2.
\item S'il gagne la première partie, il gagne la deuxième avec une probabilité de 0,9. 
\item S'il perd la première partie, il perd la suivante avec une probabilité de 0,6.
\end{itemize}

On note :
\begin{itemize}[label=\textbullet]
\item $G_1$ l'évènement \og Maxime gagne la première partie \fg
\item $G_2$ l'évènement \og Maxime gagne la première partie \fg
\end{itemize}

\bigskip

\textbf{Partie A}

\medskip

\begin{enumerate}
\item Construire un arbre pondéré illustrant la situation.
\item Calculer la probabilité que Maxime gagne les deux parties du jeu.
\item Montrer que la probabilité que Maxime gagne la deuxième partie du jeu est $0,5$.
\end{enumerate}

\bigskip

\textbf{Partie B}

\medskip

On sait de plus que :

\begin{itemize}
\item à chaque partie gagnée, le joueur gagne $1,50$~\euro.
\item à chaque partie perdue, il perd $1$~\euro.
\end{itemize}

\smallskip

On note $X$ la variable aléatoire qui correspond au gain algébrique en euros de Maxime
à l'issue des deux parties.

\medskip

\begin{enumerate}
\item Recopier sur la copie et compléter le tableau ci-dessous donnant la loi de probabilité de la variable aléatoire $X$.

\begin{center}
\begin{tabularx}{\linewidth}{|*{5}{>{\centering \arraybackslash}X|}}\hline
Valeurs de $X$&&&3&Total\\ \hline
Probabilité&&& 0,18&\\ \hline
\end{tabularx}
\end{center}
\item Déterminer si ce jeu est équitable. Justifier.
\end{enumerate}

\bigskip

