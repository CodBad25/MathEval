
\medskip

On dispose d'un paquet de cartes contenant un nombre identique de cartes de la catégorie \og Sciences\fg{} et de la catégorie \og Économie \fg. Une question liée à un de ces deux thèmes figure sur chaque carte.

Les cartes sont mélangées et on en tire une au hasard dans le paquet. Ensuite, on essaye de répondre à la question posée.

Un groupe de copains participe à ce jeu. Connaissant leurs points forts et leurs faiblesses, on estime qu'il a :

$\bullet~~$ 3 chances sur 4 de donner la bonne réponse lorsqu'il est interrogé en sciences;

$\bullet~~$ 1 chance sur 8 de donner la bonne réponse lorsqu'il est interrogé en économie.

\smallskip

On note $S$ l'évènement \og La question est dans la catégorie Sciences\fg{} et $B$ l'évènement \og La réponse donnée par le groupe est bonne \fg.

\bigskip

\textbf{Partie A :}

\medskip

\begin{enumerate}
\item Calculer $P(B \cap S)$.
\item Déterminer la probabilité que le groupe de copains réponde correctement à la
question posée.
\item Les évènements $S$ et $B$ sont-ils indépendants?
\end{enumerate}

\bigskip

\textbf{Partie B :}

\medskip

Pour participer à ce jeu, on doit payer 5~\euro{} de droit d'inscription. On recevra :

\setlength\parindent{1cm}
\begin{itemize}
\item[$\bullet$~~] 10 \euro{} si on est interrogé en sciences et que la réponse est correcte ;
\item[$\bullet$~~] 30 \euro{} si on est interrogé en économie et que la réponse est correcte ;
\item[$\bullet$~~] rien si la réponse donnée est fausse.
\end{itemize}
\setlength\parindent{0cm}

\smallskip

Soit $X$ la variable aléatoire qui, à chaque partie jouée, associe son gain. On appelle gain la différence en euros entre ce qui est reçu et les $5$~\euro{} de droit d'inscription.

\medskip

\begin{enumerate}
\item Déterminer la loi de probabilité de $X$.
\item Que retourne la fonction Jeu écrite ci-dessous en langage Python
avec les listes : 

L =[-5~;~5~;~25] et G = [0,5625~;~0,375~;~\np{0,0625}] ?

\begin{center}

\begin{tabularx}{0.4\linewidth}{|X|}\hline
def \texttt{Jeu(L,G):}\\
\quad  \texttt{n = len(L)}\\
\quad \texttt{E = 0}\\
\quad \texttt{for i in range(n):}\\
\qquad \texttt{E =E + L[i]*G[i]} \\
\quad \texttt{return(E)}\\ \hline
\end{tabularx}
\end{center}
\end{enumerate}

\bigskip

