
\subsection*{1.}

\paragraph{a.} 
\(h_1 = 2 \times \dfrac{4}{5} = \dfrac{8}{5} = 1{,}6 \text{ (m)}\)

\(h_2 = 1{,}6 \times \dfrac{4}{5} = \dfrac{6,4}{5} = 1{,}28 \text{ (m)}\)

\paragraph{b.} On passe d'une hauteur de rebond à la suivante en la multipliant par \(\dfrac{4}{5} = 0{,}8\).

Donc \(h_{n+1} = 0{,}8 \times h_n\).

\paragraph{c.} L'égalité précédente montre que la suite \((h_n)\) est une suite géométrique de raison \(q = 0{,}8\) et de premier terme \(u_0 = 2\).

\paragraph{d.} La raison de la suite étant comprise entre 0 et 1, la suite est décroissante.

\subsection*{2.}

On entre la valeur initiale : 2,

puis on tape \(\times 0{,}8\) ENTRÉE on obtient \(h_1\)

et à chaque ENTRÉE les termes suivants de la suite.

On obtient \(h_{10} \approx 0{,}21\) et \(h_{11} \approx 0{,}17\). C'est donc au \(10^{\text{ème}}\) rebond que celui-ci est inférieur à 20 cm.

