
\subsection*{Question 1}

On considère la droite \(d\) dont une équation cartésienne dans un repère orthonormé est :
\[
2x - 3y + 4 = 0.
\]

Un vecteur directeur de \(d\) est \(\vec{u} \begin{pmatrix} 3 \\ 2 \end{pmatrix}\).

On a \(\vec{u} \cdot \overrightarrow{n} = -36 + 36 = 0\) : la réponse \textbf{b.} est vraie.

\subsection*{Question 2}

Dans un repère orthonormé :
\begin{itemize}
    \item le cercle \(\mathcal{C}\) a pour équation \(x^2 - 2x + y^2 + y = 3\) ;
    \item la droite \(D\) a pour équation \(y = 1\).
\end{itemize}

Un point \(M(x\,;\,y) \in \mathcal{C} \cap D \iff \left\{
\begin{array}{l}
x^2 - 2x + y^2 + y = 3 \\
y = 1
\end{array} \right.\)

\[
\Rightarrow x^2 - 2x + 1 + 1 = 3 \iff x^2 - 2x - 1 = 0.
\]

Pour cette équation du second degré :
\[
\Delta = (-2)^2 - 4 \times (-1) = 4 + 4 = 8 > 0,
\]
donc l'équation a deux solutions distinctes. C'est l'affirmation \textbf{c.} qui est vraie.

\subsection*{Question 3}

La fonction \(f\) est définie sur l'ensemble des réels par \(f(x) = \cos(2x)\).

On a \(f(-x) = \cos(-2x) = \cos(2x)\). L'affirmation \textbf{a.} est vraie.

\subsection*{Question 4}

Soit la suite \((u_n)\) définie par \(u_0 = 1\) et, pour tout entier naturel \(n\) :
\[
u_{n+1} = \dfrac{1}{2} \left( u_n + \dfrac{2}{u_n} \right).
\]

Le bon algorithme est le \textbf{d.}.

\subsection*{Question 5}

Pour l'équation \(\e^x = 1\), on a bien \(\e^0 = 1\). L'affirmation \textbf{c.} est vraie.

