
\subsection*{1.}

Salaires annuels de Camille et de Dominique en 2012 :
\begin{itemize}
    \item Camille : \(14400 + 600 + 600 = 15600\) (€) ;
    \item Dominique : \(13200 \times 1{,}04^2 = 14277{,}12\) (€).
\end{itemize}

\subsection*{2.}

\paragraph{a.} On a, pour tout entier naturel \( n \), \( u_{n+1} = u_n + 600 \) : la suite \( (u_n) \) est donc une suite arithmétique de raison \(r = 600\).

\paragraph{b.} On sait que, pour tout entier naturel \( n \), \( u_n = u_0 + 600n \).

Il faut donc résoudre dans \( \mathbb{N} \) l'inéquation :
\begin{align*}
&14400 + 600n > 20000 \\
\iff &600n > 5600 \\
\iff &n > \dfrac{5600}{600} \text{ avec } \dfrac{5600}{600} \approx 9{,}3.
\end{align*}

Le salaire annuel de Camille dépassera les 20000 € pour la première fois après \(n = 10\) ans, soit en 2020.

\paragraph{c.} D'une année à la suivante, le salaire est multiplié par 1{,}04.

Donc, pour tout entier naturel \( n \), \( v_{n+1} = v_n \times 1{,}04 \), ce qui montre que la suite \( (v_n) \) est une suite géométrique de raison \(q = 1{,}04\) et de premier terme \( v_0 = 13200 \).

\paragraph{d.} On sait que, pour tout entier naturel \( n \), \( v_n = 13200 \times 1{,}04^n \).

Donc, pour \( n = 10 \), \( v_{10} = 13200 \times 1{,}04^{10} \approx 19539{,}22 \), soit environ 19539 €.

\subsection*{3.}

\begin{center}
\begin{python}
def algo() :
    A = 14400
    B = 13200
    n = 0
    while A >= B :
        A = A + 600
        B = B * 1.04
        n = n + 1
    return n
\end{python}
\end{center}

Ce programme retourne \( n = 14 \), donc le salaire de Dominique dépassera celui de Camille en 2024.

