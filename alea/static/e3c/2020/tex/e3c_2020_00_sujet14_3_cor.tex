	\section*{Exercice 3 (5 points)}
	
\begin{enumerate}
	\item  \(T_1 = 0,82 \times 1000 + 3,6 = 820 + 3,6 = 823,6\) (°C). Soit environ 824 °C.
	
\item La formule est donnée dans l’algorithme : \(T_{n+1} = 0,82T_n + 3,6\).
	
	\item On obtient successivement :
	
	\[
	T_2 \approx 678,95, \quad T_3 \approx 560,34, \quad T_4 \approx 463,08.
	\]
	
	Au bout de 4 heures, la température du four est de 463 °C à l’unité près.
	
\item \begin{minipage}{7cm}	\begin{python}
		1 def froid():
		2     T = 1000
		3     n = 0
		4     while T > 70:
		5         T = 0.82 * T + 3.6
		6         n = n + 1
		7     return n
		\end{python} \end{minipage}
		
\item Au bout de 15 heures, la température est à peu près égale à 69,9 °C.
\end{enumerate}
	
