
\medskip

On considère qu'en 2019, \np{3300000}~personnes étaient atteintes de diabète en France.

Pour étudier l'évolution de la maladie, des chercheurs appliquent un modèle selon lequel le nombre de personnes atteintes augmente de $2$\,\% par an.

On note $u_n$ le nombre de personnes atteintes de diabète en France selon ce modèle durant l'année $(2019 + n)$. On a donc $u_0 = \np{3300000}$.

\medskip

\begin{enumerate}
\item Justifier que, selon ce modèle, le nombre de personnes atteintes de diabète en France sera de \np{3433320} en 2021.
\item Quelle est la nature de la suite $\left(u_n\right)$ ?
\item Donner l'expression de $u_n$ en fonction de $n$.
\item En déduire le nombre de personnes qui, selon ce modèle, seront atteintes de diabète en France en 2025.
\item On définit en langage Python la fonction suivante.

\begin{center}
\begin{python}
def seuil(S): 
    u=33000
    n=0
    while u<S :
       u=u*1,02
       n=n+1
    return n
\end{python}
\end{center}

Après exécution dans la console on obtient l'affichage suivant :

\begin{center}
\begin{tabular}{|l|}\hline
$>>>$ seuil(5000000)\\
21\\ \hline
\end{tabular}
\end{center}

Interpréter ce résultat dans le contexte de l'exercice.
\end{enumerate}

\bigskip

