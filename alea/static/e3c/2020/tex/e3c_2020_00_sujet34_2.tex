
\medskip

Un pépiniériste stocke un grand nombre d'arbustes de la famille des \emph{viburnum} en vue de les
vendre. Ceux-ci sont de deux espèces différentes : les \emph{viburnum tinus} (nom commun : laurier
tin) et les \emph{viburnum opulus} (nom commun : boule de neige). Il constate que :
\begin{itemize}
\item 80\,\% de ses arbustes sont des lauriers tins, les autres sont des boules de neige.
\item Parmi les lauriers tins, 41\,\% mesurent \np[m]{1.10} ou plus.
\item Parmi les boules de neige, 32\,\% mesurent \np[m]{1.10} ou plus.
\end{itemize}

\medskip

\begin{enumerate}
	\item Est-il vrai que moins de 15\,\% des \emph{viburnum} de ce pépiniériste sont des boules de neige de moins de \np[m]{1.10} ?
\end{enumerate}

On choisit au hasard un \emph{viburnum} chez ce pépiniériste et on considère les évènements
suivants :

\begin{itemize}
\item $L$ : \og le \emph{viburnum} choisi est un laurier tin \fg
\item $T$ : \og le \emph{viburnum} mesure plus de \np[m]{1.10}\fg.
\end{itemize}

\begin{enumerate}[resume]
\item Décrire par une phrase la probabilité $p_L\left(\overline{T}\right)$.

 Décrire également par une phrase l'évènement $\overline{L}\cap T$

\item Recopier et compléter sur la copie l'arbre de probabilité ci-dessous traduisant les données de l'énoncé.

\vspace{0,5cm}
\begin{center}

\psset{nodesepA=0pt,nodesepB=3pt,treesep=0.75,labelsep=0.1pt,levelsep=2.75cm}
\pstree[treemode=R]{\TR{}}
{\pstree{\TR{$L$~~}\taput{$\dots$}}
	{
	\TR{$T$}\taput{$\dots$}
	\TR{$\overline{T}$}\tbput{$\dots$}
	}
\pstree{\TR{$\overline{L}$~~}\tbput{$\dots$}}
	{\TR{$T$}\taput{$\dots$}
	\TR{$\overline{T}$}\tbput{$\dots$}
	}
}
\end{center}

\medskip

\item Montrer que la probabilité que le \emph{viburnum} mesure \np[m]{1.10} ou plus est égale à 0,392.
\item Le \emph{viburnum} choisi a une taille inférieure à \np[m]{1.10}. Quelle est la probabilité que ce soit un boule de neige ? On arrondira le résultat à  $10^{-3}$.
\end{enumerate}

\vspace{0,5cm}

