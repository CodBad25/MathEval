
\medskip

Dans cet exercice, les distances sont exprimées en mètres.

On considère un rectangle ABCD d'aire $49$~m$^2$ tel que DC $= x$ et BC $= y$. 

On admet que les nombres $x$ et $y$ sont strictement positifs.

\begin{center}
\psset{unit=0.9cm}
\begin{pspicture}(4,3.5)
\psframe(4,3)
\psline[linewidth=0.4pt]{<->}(0,3.3)(4,3.3)
\psline[linewidth=0.4pt]{<->}(4.3,0)(4.3,3)
\uput[u](2,3.3){$x$}\uput[r](4.3,1.5){$y$}
\uput[dl](0,0){A}\uput[dr](4,0){B}
\uput[ur](4,3){C}\uput[ul](0,3){D}
\end{pspicture}
\end{center}

On souhaite déterminer les dimensions $x$ et $y$ pour que le périmètre de ce rectangle soit
minimal.

\medskip

\begin{enumerate}
\item 
	\begin{enumerate}
		\item Montrer que le périmètre, en mètres, du rectangle ABCD est égal à $2x + \dfrac{98}{x}$.
		\item Calculer ce périmètre pour $x = 10$.

Soit $f$ la fonction définie sur $]0~;~+ \infty[$ par $f(x) = 2x + \dfrac{98}{x}$.

On admet que $f$ est dérivable sur $]0~;~+ \infty[$ et on note $f'$ sa fonction dérivée.
	\end{enumerate}
\item Montrer que, pour tout $x > 0$,

\[f'(x) = \dfrac{2x^2 - 98}{x^2}.\]

\item Déterminer le tableau de variations de la fonction $f$ sur $]0~;~+ \infty[$.
\item En déduire les dimensions du rectangle d'aire $49$~m$^2$ dont le périmètre est minimal.
\end{enumerate}

\bigskip

