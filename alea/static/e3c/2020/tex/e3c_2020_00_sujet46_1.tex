
\emph{Cet exercice est un questionnaire à choix multiple (QCM).\\
Pour chacune des questions, une seule des quatre réponses proposées est correcte.\\
Aucune justification n'est demandée.\\
Une bonne  réponse  rapporte $1$ point. Une mauvaise réponse, plusieurs réponses  ou ou l'absence de réponse ne rapportent ni n'enlèvent aucun point.\\
Une bonne réponse rapporte un point. Une mauvaise réponse, plusieurs réponses 
Pour chaque question, indiquer le numéro de la question et recopier sur la copie la lettre qui correspond à la réponse choisie.}


\medskip

\begin{enumerate}
\item On munit le plan du repère orthonormé \Oij.

On considère trois points du plan A, B et C tels que AB $= 2$, AC $= \sqrt{3}$ et $\widehat{\text{BAC}} = \dfrac{5\pi}{3}$.

Alors $\vect{\text{AB}} \cdot \vect{\text{AC}} = $

\begin{center}
\begin{tabularx}{\linewidth}{|*{4}{X|}}\hline
\textbf{a.~~}$2\sqrt{3}$ &\textbf{b.~~}$3$&\textbf{c.~~}$- 2\sqrt{3}$  &\textbf{d.~~}$ - 3$\\ \hline
\end{tabularx}
\end{center}

\item Soit $a$ un nombre réel. On munit le plan du repère orthonormé \Oij.

On considère les vecteurs $\vect{u}\binom{\sin (a)}{\cos (a)}$ et $\vect{v}\binom{-\cos (a)}{\sin (a)}$. Alors $\vect{u} \cdot \vect{v}$ est égal à :

\begin{center}
\begin{tabularx}{\linewidth}{|*{4}{X|}}\hline
\textbf{a.~~}$\sin^2 (a)+\cos^2 (a)$ &\textbf{b.~~}$1$&\textbf{c.~~}$\sin^2 (a)-\cos^2 (a)$  &\textbf{d.~~}$0$\\ \hline
\end{tabularx}
\end{center}

\item On munit le plan du repère orthonormé \Oij.

On considère les points A(2~;~8), B(35~;~0), C(7~;~-5) et D(3~;~0). Alors, les droites (AB) et (CD) sont:

\begin{center}
\begin{tabularx}{\linewidth}{|*{4}{X|}}\hline
\textbf{a.~~}parallèles&\textbf{b.~}perpendiculaires&\textbf{c.~~}sécantes&\textbf{d.~~}confondues\\ \hline
\end{tabularx}
\end{center}

\item On munit le plan du repère orthonormé \Oij.

On considère la fonction $f$ définie pour tout réel $x$ non nul par $f(x) = \dfrac{3}{x}$. On note $C$ sa courbe représentative dans ce repère. 

L'équation réduite de la tangente à $C$ au point d'abscisse 1 est :

\begin{center}
\begin{tabularx}{\linewidth}{|*{4}{X|}}\hline
\textbf{a.~~}$y = -3x + 6$ &\textbf{b.~~}$y = -3x$&\textbf{c.~~}$ y = 3x$  &
\textbf{d.~~}$y = 3x + 6$\\ \hline
\end{tabularx}
\end{center}

\item L'ensemble des solutions dans $\R$ de l'équation $x^2 = 6x - 5$ est :

\begin{center}
\begin{tabularx}{\linewidth}{|*{4}{X|}}\hline
\textbf{a.~~}$S =\{1~;~5\}$ &\textbf{b.~~}$S = \{1\}$&\textbf{c.~~}$S = \emptyset$  &\textbf{d.~~}$S = \{- 5~;~-1\}$\\ \hline
\end{tabularx}
\end{center}
\end{enumerate}

\bigskip

