 
\medskip

On considère la fonction $f$ définie pour tout nombre réel $x$  de l'intervalle $[-1~;~5]$ par : \[f(x)=x^3-6x^2+9x+1.\]

\begin{enumerate}
\item  Soit $f'$ la fonction dérivée de $f$. Déterminer, pour tout nombre réel $x$ de $[-1~;~5]$, l'expression de $f'(x)$.
\item Montrer que pour tout nombre réel $x$ de $[-1~;~5]$, \[f'(x) = 3(x - 1)(x - 3).\]
\item Dresser le tableau de signe de $f'(x$) sur $[-1~;~5]$ et en déduire le tableau de variation de la fonction $f$ sur ce même intervalle.
\item Déterminer l'équation de la tangente $T$ à la courbe de la fonction $f$ au point d'abscisse $0$.
\item Déterminer l'autre point de la courbe de $f$ en lequel la tangente est parallèle à $T$.
\end{enumerate} 
