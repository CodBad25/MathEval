
\subsection*{Question 1}

Le paramètre \( \lambda \) est négatif pour les fonctions \(f\) et \(g\), positif pour les fonctions \(h\) et \(k\).

Choisissons une abscisse négative : le point correspondant de la courbe \(\mathcal{C}_g\) a une ordonnée supérieure à celle du point de la courbe \(\mathcal{C}_f\).

Réponse \textbf{b.}

\subsection*{Question 2}

On a :
\begin{align*}
P(X \geqslant 1) &= 1 - P(X = 0) \\
&= 1 - 0{,}5^2 \\
&= 0{,}75.
\end{align*}

Réponse \textbf{d.}

\subsection*{Question 3}

La fonction sinus est une fonction périodique de période \(2\pi\).

Seuls les points \(A_0\) et \(A_3\) ont des abscisses séparées de \(2\pi\).

Réponse \textbf{c.}

\subsection*{Question 4}

\( f(x) \) est un polynôme du second degré dont le discriminant est :
\[
\Delta = (-2)^2 - 4 \times 0{,}5 \times 1 = 2
\]

Les racines sont :
\[
x_1 = \dfrac{2 - \sqrt{2}}{1} = 2 - \sqrt{2} \quad \text{et} \quad x_2 = \dfrac{2 + \sqrt{2}}{1} = 2 + \sqrt{2}.
\]

Réponse \textbf{b.}

\subsection*{Question 5}

D'après la formule d'Al-Kashi, on a :
\begin{align*}
AC^2 &= AB^2 + BC^2 - 2AB \times BC \cos \widehat{ABC} \\
&= 25 + 4 - 20 \cos(60) \\
&= 19
\end{align*}

Réponse \textbf{a.}

