
\subsection*{Question 1}

\(\overrightarrow{AB} \cdot \overrightarrow{AC} = AB \times AC \times \cos(\widehat{BAC}) = 2 \times \sqrt{3} \times \left(-\dfrac{\sqrt{3}}{2}\right) = -3\).

\subsection*{Question 2}

\(\vec{u} \cdot \vec{v} = -\sin(a)\cos(a) + \cos(a)\sin(a) = 0\).

\subsection*{Question 3}

On a \(\overrightarrow{AB}\begin{pmatrix} \dfrac{19}{3} \\ -8 \end{pmatrix}\) et \(\overrightarrow{CD}\begin{pmatrix} -4 \\ 5 \end{pmatrix}\).

Or :
\begin{align*}
\det\left(\vec{u}, \vec{v}\right) &= \dfrac{19}{3} \times 5 - (-4) \times (-5) \\
&= \dfrac{95}{3} - 20 \\
&= \dfrac{35}{3} \neq 0,
\end{align*}
les vecteurs ne sont pas colinéaires, les droites \((AB)\) et \((CD)\) ne sont pas parallèles.

\(\vec{u} \cdot \vec{v} = -\dfrac{76}{3} - 40 < 0\), donc les vecteurs ne sont pas orthogonaux : les droites ne sont pas perpendiculaires.

\subsection*{Question 4}

Une équation réduite de la tangente à \(\mathcal{C}\) au point d'abscisse 1 est :
\[
y - f(1) = f'(1)(x - 1).
\]
Or, pour \(x \) non nul, on a : \(f'(x) = -\dfrac{3}{x^2}\).

Donc \(f'(1) = -3\) et avec \(f(1) = 3\), l'équation devient :
\[
y - 3 = -3(x - 1) \quad \text{ou} \quad y = -3x + 3 + 3 \quad \text{et enfin} \quad y = -3x + 6.
\]

\subsection*{Question 5}

L'équation s'écrit : \(x^2 - 6x + 5 = 0\).

On a :
\[
\Delta = 36 - 20 = 16 = 4^2 > 0.
\]
Le trinôme a deux racines :
\[
x_1 = \dfrac{6 + 4}{2} = 5 \quad \text{et} \quad x_2 = \dfrac{6 - 4}{2} = 1.
\]

