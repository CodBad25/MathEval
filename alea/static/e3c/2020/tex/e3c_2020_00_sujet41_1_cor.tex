
\subsection*{Question 1}

\(\overrightarrow{AB} \cdot \overrightarrow{CB} = 4 + 15 = 19\).

\subsection*{Question 2}

\(CB^2 = \overrightarrow{CB} \cdot \overrightarrow{CB} = 1 + 25 = 26\), donc \(CB = \sqrt{26}\).

\subsection*{Question 3}

Puisque \( H \) est le milieu de \( [AB] \), la droite \( (CH) \) est la médiane mais aussi la hauteur issue de \( C \), donc les droites \( (CH) \) et \( (AB) \) sont perpendiculaires, et les vecteurs \( \overrightarrow{HB} \) et \( \overrightarrow{HC} \) sont orthogonaux, donc leur produit scalaire est nul.

\subsection*{Question 4}

La fonction cosinus est paire, donc \( \cos(-x) = \cos(x) = \dfrac{\sqrt{3}}{2} \).

\subsection*{Question 5}

L'équation peut s'écrire :
\begin{align*}
&x^2 - 2x + (y + 3)^2 = 3 \\
\iff &(x - 1)^2 - 1 + (y + 3)^2 = 3 \\
\iff &(x - 1)^2 + (y - (-3))^2 = 2^2.
\end{align*}

C'est donc l'équation du cercle de centre \( (1\,;\,-3) \) et de rayon 2.

