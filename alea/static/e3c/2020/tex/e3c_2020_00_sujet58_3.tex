  
\medskip

Soit $h$ la fonction définie sur $[-6~;~26]$ par :
\[h(x)=-x^3+30x^2-108x-490.\]

\begin{enumerate}
\item  Soit $h'$ la fonction dérivée de $h$. Exprimer $h'(x)$ en fonction de $x$.
\item On note $\mathcal{C}$ la courbe représentative de $h$ et $\mathcal{C}'$ celle de $ h'$.
\begin{enumerate}
\item  Identifier $\mathcal{C}$ et $\mathcal{C}'$ sur le graphique orthogonal ci-dessous parmi les trois courbes $\mathcal{C}_1$,$\mathcal{C}_2$ et $\mathcal{C}_3$ proposées.
\item Justifier le choix pour $\mathcal{C}'$.
\end{enumerate}

\psset{xunit=0.3cm,yunit=0.003cm,labelFontSize=\scriptstyle,labelsep=0.1pt}
\begin{pspicture}(-11,-800)(35,1625)
\psaxes[linewidth=0.95pt,Dx=2,Dy=200]{->}(0,0)(-10,-800)(31.5,1625)
\def\Func{ x x x neg 30 add mul  108 sub  mul 490 sub }
\psplot[plotpoints=1000,linewidth=1.25pt,linecolor=blue]{-6}{26}{\Func}
\psplot[plotpoints=1000,linewidth=1.25pt,linecolor=blue,linestyle=dashed]{-6}{26}{x x 3 neg mul 60 add mul 108 sub}
\psplot[plotpoints=1000,linewidth=2pt,linecolor=blue, linestyle=dotted]{-6}{26}{x x 3  mul 60 sub mul 108 add}
\uput[ul](26.5,640){$\mathcal{C}_3$} \uput[u](12.,940){$\mathcal{C}_2$} \uput[ur](-6,-450){$\mathcal{C}_1$}
\end{pspicture}
\item Soit ($\mathcal{T}$) la tangente à $\mathcal{C}$ au point A d'abscisse 0. Déterminer son équation réduite.
\item Étudier le signe de $h'(x)$ puis dresser le tableau de variation de la fonction $h$ sur $[-6~;~26]$. 
\end{enumerate}

\emph{\footnotesize L'intervalle d'études a été changé  car les graphiques ne correspondaient pas aux intervalles donnés $[0~;~26]$ ou $[0~;~30]$}


