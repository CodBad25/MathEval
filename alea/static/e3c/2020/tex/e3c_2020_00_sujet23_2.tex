
\medskip

On considère la fonction $f$ définie sur $[0~;~+\infty[$ par \[f(x) =\dfrac{\e^x}{1+x}.\]

On note $\mathcal{C}_f$ la représentation graphique de $f$ dans un repère du plan.

\medskip

\begin{enumerate}
\item Déterminer les coordonnées du point A, point d’intersection de la courbe $\mathcal{C}_f$
avec l’axe des ordonnées.
\item La courbe $\mathcal{C}_f$ coupe-t-elle l’axe des abscisses ? Justifier la réponse.
\item On note $f'$ la dérivée de la fonction $f$ sur $[0~;~+\infty[$. Montrer que, pour tout
réel $x$ de l’intervalle $[0; +\infty[$, \[f'(x) =\dfrac{x\e^x}{(1 + x)^2}.\]
\item Étudier le signe de $f'(x)$ sur $[0~;~+\infty[$. En déduire le sens de variation de $f$ sur
$[0~;~+\infty[$.

\item On note $\mathcal{T}$ la tangente à $\mathcal{C}_f$ au point B d’abscisse 1,6. La tangente $\mathcal{T}$ passe-t-elle par l’origine du repère ? Justifier la réponse.

\emph{Remarque : \footnotesize {le texte donnait A mais A était déjà défini autrement}}
\end{enumerate}

\vspace{0,5cm}

