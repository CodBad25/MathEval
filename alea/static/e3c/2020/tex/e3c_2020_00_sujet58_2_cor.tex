
\subsection*{1.}

Augmenter de 2 \% revient à multiplier par \(1 + \dfrac{2}{100} = 1 + 0{,}02 = 1{,}02\).

Donc :
\[
u_1 = 4600 \times 1{,}02 = 4692 \quad \text{(habitants en 2011 dans la ville A)},
\]
\[
v_1 = 5100 + 110 = 5210 \quad \text{(habitants en 2011 dans la ville B)}.
\]

\subsection*{2.}

On a pour tout naturel \(n\), \(u_{n+1} = 1{,}02u_n\) : la suite \((u_n)\) est donc une suite géométrique de raison \(q = 1{,}02\) et de premier terme \(u_0 = 4600\).

On a pour tout naturel \(n\), \(v_{n+1} = v_n + 110\) : la suite \((v_n)\) est donc une suite arithmétique de raison \(r = 110\) et de premier terme \(v_0 = 5100\).

\subsection*{3.}

On sait que, pour tout naturel \(n\), \(u_n = 4600 \times 1{,}02^{n - 1}\).

Pour \(n = 10\), \(u_{10} = 4600 \times 1{,}02^{10 - 1} = 4600 \times 1{,}02^9 \approx 5497\).

\subsection*{4.}

On sait que, pour tout naturel \(n\), \(v_n = 5100 + 110n\).

Pour \(n = 10\), \(v_{10} = 5100 + 110 \times 10 = 5100 + 1100 = 6210\).

\subsection*{5.}

\begin{center}
\begin{python}
def année() :
    u = 4600
    v = 5100
    n = 0
    while u < v :
        u = u * 1.02
        v = v + 110
        n = n + 1
    return n
\end{python}
\end{center}

