
\medskip

On considère deux élevages de chatons sacrés de Birmanie :
\begin{itemize}
\item  Dans le premier élevage 75\,\% des chatons deviennent couleur Chocolat et 25\,\% deviennent couleur Blue.
\item Dans le second élevage 30\,\% des chatons deviennent couleur Chocolat et 70\,\% deviennent couleur Blue.
\end{itemize}

Une animalerie se fournit dans ces deux élevages. Elle achète 40\,\% de ses chatons au premier élevage et 60\,\% au deuxième.

On choisit au hasard un chaton de l'animalerie.

On note $A$ l'évènement \og Le chaton provient du premier élevage \fg{} et $B$ l'évènement \og Le chaton est de couleur Blue \fg.

On note $\overline{A}$ l'évènement contraire de $A$ et $\overline{B}$ l'évènement contraire de $B$.

\medskip

\begin{enumerate}
\item
	\begin{enumerate}
		\item Recopier sur la copie et compléter l'arbre de probabilité ci-dessous :

\begin{center}
\psset{nodesepA=0pt,nodesepB=3pt,treesep=0.75,labelsep=0.1pt,levelsep=2.5cm}
\pstree[treemode=R]{\TR{}}
{\pstree{\TR{$A$~~}\taput{$\dots$}}
	{
	\TR{$B$}\taput{$\dots$}
	\TR{$\overline{B}$}\tbput{$\dots$}
	}
\pstree{\TR{$\overline{A}$~~}\tbput{$\dots$}}
	{\TR{$B$}\taput{$\dots$}
	\TR{$\overline{B}$}\tbput{$\dots$}
	}
}
\end{center}

		\item Calculer $p\left(\overline{A}\cap \overline{B}\right)$ et interpréter ce résultat.
		\item Montrer que la probabilité que le chaton soit de couleur Chocolat est $0,48$.
		\item Sachant que Jules a choisi un chaton couleur Blue dans cette animalerie, quelle est la probabilité que le chaton provienne du deuxième élevage ? 	On donnera le résultat à $10^{-2}$ près.
	\end{enumerate}
\item Le responsable du rayon fixe à $100$~\euro{} le prix de vente d'un chaton couleur Blue et à 75~\euro{} le prix d'un chaton couleur Chocolat.

On choisit au hasard un chaton de l'animalerie et on désigne par $X$ la variable aléatoire égale au prix en euros du chaton acheté. Déterminer la loi de probabilité de $X$.
\end{enumerate}
