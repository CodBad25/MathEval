
\medskip

Le plan est rapporté à un repère orthonormé.

On considère les points A$(-3~;~1)$, B$(3~;~5)$ et C$(7~;~1)$ dans ce repère.

Le but de cet exercice est de déterminer les coordonnées du centre du cercle circonscrit au triangle ABC et le rayon de ce cercle.

On rappelle que le cercle circonscrit à un triangle est le cercle passant par les trois sommets de ce triangle.

\medskip

\begin{enumerate}
\item Placer les points A, B et C dans le plan puis construire le cercle circonscrit au triangle ABC.
\item Vérifier que la droite $\Delta$ d'équation $3x + 2y - 6 = 0$ est la médiatrice du segment [AB]. 
\item Déterminer les coordonnées du point B$'$, milieu du segment [AC].
\item Déterminer les coordonnées du point I, centre du cercle circonscrit au triangle ABC.
\item Calculer une valeur exacte du rayon du cercle circonscrit au triangle ABC.
\end{enumerate}
