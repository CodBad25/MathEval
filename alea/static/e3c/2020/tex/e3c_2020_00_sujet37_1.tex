
\medskip

Cet exercice est un questionnaire à choix multiple (QCM) comportant cinq questions.

Pour chacune des questions, une seule des quatre réponses proposées est correcte. Les questions
sont indépendantes.

Pour chaque question, indiquer le numéro de la question et recopier sur la copie la lettre
correspondant à la réponse choisie.

Aucune justification n'est demandée mais il peut être nécessaire d'effectuer des recherches au
brouillon pour déterminer la réponse.

Chaque réponse correcte rapporte 1 point. Une réponse incorrecte ou une question sans réponse
n'apporte ni ne retire de point.

\medskip

\textbf{Question 1}

\medskip 

Soit $x$ un nombre réel. On peut affirmer que :

\medskip 
 
\begin{tabularx}{\linewidth}{*{2}{X}}
\textbf{a.~~} $\cos(x)=\sin(x)$ & \textbf{b.~~}$\cos(\pi-x)=\cos(\pi+x)$ \\\textbf{c.~~}$\sin(\pi+x)=\sin(\pi-x) $& \textbf{d.~~}$ \cos \left(\dfrac{\pi}{2}+ x\right) = \cos \left(\dfrac{\pi}{2}- x\right) $.
\end{tabularx}


\medskip

\textbf{Question 2}

\medskip 

Les solutions dans l'intervalle $[0~;~2\pi[$ de l'équation $\sin(x) = -\dfrac{\sqrt{3}}{2}$
 sont :
 
 \medskip
 
\begin{tabularx}{\linewidth}{*{4}{X}}
\textbf{a.~~} $\dfrac{4\pi}{3}$ et $\dfrac{5\pi}{3} $ &\textbf{b.~~} $\dfrac{2\pi}{3}$ et $\dfrac{4\pi}{3} $&\textbf{c.~~}$ \dfrac{\pi}{3}$ et $\dfrac{2\pi}{3}$& \textbf{d.~~} $-\dfrac{2\pi}{3}$ et $- \dfrac{\pi}{3}  $.
\end{tabularx}


\medskip

\textbf{Question 3}

\medskip 

On considère ABCD un carré direct dans lequel on construit un
triangle ABE équilatéral direct.

On note AB $= a$.

On peut alors affirmer que :

\begin{minipage}[t]{0.7\linewidth}

\begin{tabularx}{\linewidth}{*{2}{X}}
\textbf{a.~~} $ \vv{\text{AB}} \cdot \vv{\text{AC}} =\dfrac{1}{2}a^2$\rule[-3mm]{0mm}{9mm} &\textbf{b.~~} $\vv{\text{AB}}\cdot\vv{\text{AD}} = a^2 $\\\textbf{c.~~}$\vv{\text{AB}}\cdot \vv{\text{AE}}=\dfrac{1}{2}a^2$\rule[-3mm]{0mm}{9mm}& \textbf{d.~~} $ \vv{\text{AD}}\cdot \vv{\text{DC}} = -a^2 $.
\end{tabularx}
\end{minipage}
\begin{minipage}[]{0.25\linewidth}
\begin{pspicture}(3,2.2)
\psframe(2,2)
\psdot[dotstyle=Mul,dotscale=1.15](1,1.732)
\psline(0,0)(1,1.732)\psline(1,1.732)(2,0)
\uput{0.01}[dl](0,0){A}\uput{0.01}[dr](2,0){B}\uput{0.01}[ur]{0.01}(2,2){C}\uput{0.01}[ul](0,2){D}\uput{0.1}[r](1,1.732){E}
\end{pspicture}
\end{minipage}

\medskip

\textbf{Question 4}

\medskip 

Soient $\vv{u}$ et $\vv{v}$ deux vecteurs. On peut affirmer que :

\begin{tabularx}{\linewidth}{*{2}{X}}
\textbf{a.~~} $\vv{u}\cdot\vv{v}=0 $ &\textbf{b.~~} $\vv{u}\cdot\vv{v}=-\vv{v}\cdot\vv{u}$\\
\textbf{c.~~}$\vv{u}\cdot\vv{u}= \dfrac{1}{2}\left\|\vv{u}\right\|^2$. & \textbf{d.~~} $\left\|\vv{u} +\vv{v}\right\|^2= \left\|\vv{u}\right\|^2+ \left\|\vv{v}\right\|^2+ 2 \vv{u}\cdot\vv{v}$.
\end{tabularx}

\medskip

\textbf{Question 5}

\medskip 

Soit $n$ un entier naturel.

On cherche à exprimer en fonction de $n$ la somme suivante :

$\mathscr{S}=1-2+4-8+16-32+\dots+(-2)^n$.

On peut affirmer que :

\medskip

\begin{tabularx}{\linewidth}{*{2}{X}}
\textbf{a.~~} $\mathscr{S}=\dfrac{1+(-2^n)}{2}\times(n+1) $ &\textbf{b.~~} $\mathscr{S}$ est la somme des termes d'une suite arithmétique de raison $(-2)$\\
\textbf{c.~~}$\mathscr{S}=\dfrac{1-(-2)^n}{1-2} $& \textbf{d.~~} $\mathscr{S}=\dfrac{1}{3}\left(1-(-2)^{n+1}\right)  $.
\end{tabularx}

\vspace{0,5cm}

