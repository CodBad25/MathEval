	\section*{Exercice 3 (5 points)}
	
\begin{enumerate}
\item 
\begin{enumerate}
	\item  	Baisser de 14\%, c’est multiplier par $1 - \dfrac{14}{100} = 1 - 0,14 = 0,86$. La suite $(P_n)$ est donc une suite géométrique de premier terme $P_0 = 10500$ et de raison 0,86.
		
		\item 	La valeur en 2010 est $P_8 = 10500 \times 0,86^8 \approx 3141,79$ €.
		\end{enumerate}
		\item 	\begin{enumerate}
				\item 			
			\begin{minipage}{7cm}
				\begin{python}
				def algo():
					P = 10500
					n = 2002
					while P > 1500:
						P = 0,86 * P
					n = n + 1
					return n
			\end{python}
			\end{minipage}
			\item	La valeur renvoyée est 2015.
		\end{enumerate}
	\end{enumerate}
			
