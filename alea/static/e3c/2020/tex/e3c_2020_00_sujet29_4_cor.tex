
\subsection*{1.}

Ajouter 3\%, c'est multiplier par \(1 + \dfrac{3}{100} = 1 + 0,03 = 1,03\).

Donc :

\(u_1 = u_0 \times 1,03 = 600 \times 1,03 = 618\) (€),

\(u_2 = u_1 \times 1,03 = 618 \times 1,03 = 636,54\) (€).

\subsection*{2.}

On a, quel que soit le naturel \(n\), \(u_{n+1} = 1,03 u_n\). Ceci montre que la suite \((u_n)\) est une suite géométrique de raison \(1,03\) et de premier terme \(u_0 = 600\).

\subsection*{3.}

On sait que, quel que soit le naturel \(n\), \(u_n = u_0 \times 1,03^n = 600 \times 1,03^n\).

\subsection*{4.}

\begin{python}
	def nombreAnnees() :
	n = 0
	u = 600
	while u <= 1000 :
		n = n + 1
		u = u * 1.03
	returnn
\end{python}

\subsection*{5.}

Pour dépasser \(1000\) euros, il faut attendre 18 ans (prix : \(\approx 1021,46\) ).

