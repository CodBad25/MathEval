
\subsection*{Question 1}

Pour le trinôme \(-3x^2 + 2x + 1\), le réel \(1\) est une racine évidente, et comme le produit des racines est égal à \(\dfrac{c}{a} = \dfrac{1}{-3}\), l'autre racine est \(-\dfrac{1}{3}\).  

On sait que ce trinôme est du signe de \(a = -3 < 0\), donc négatif, sauf (ce que nous cherchons) sur l'intervalle \(\left]-\dfrac{1}{3}\,;\,1\right[\).

\subsection*{Question 2}

On a \(M(x\,;\,y) \in \textit{d} \iff \overrightarrow{AM}\) et \(\vec{v}\) sont colinéaires, donc si :
\begin{align*}
&-2(x - (-1)) = 3(y - 5) \\
\iff &-2x - 2 = 3y - 15 \\
\iff &-2x - 3y + 13 = 0.
\end{align*}

\subsection*{Question 3}

Sur l'intervalle \(]-\infty\,;\,2[ \cup ]2\,;\,+\infty[\), on a :
\begin{align*}
f'(x) &= \dfrac{2(x - 2) - 1(2x + 1)}{(x - 2)^2} \\
&= \dfrac{2x - 4 - 2x - 1}{(x - 2)^2} \\
&= \dfrac{-5}{(x - 2)^2}.
\end{align*}

\subsection*{Question 4}

\[
\dfrac{(\e^x)^2 \times \e^{-x+1}}{\e^{5x}} = \e^{2x - x + 1 - 5x} = \e^{-4x + 1}.
\]

\subsection*{Question 5}

Sur \(\mathbb{R}\), on a :
\begin{align*}
f'(x) &= \e^x(3\e^x - 1) + \e^x \times 3\e^x \\
&= \e^x(3\e^x - 1 + 3\e^x) \\
&= \e^x(6\e^x - 1) \\
&= 6\e^{2x} - \e^x.
\end{align*}

