
\medskip

Ce QCM comprend 5 questions.

Pour chacune des questions, une seule des quatre réponses proposées est correcte.

Les questions sont indépendantes.

Pour chaque question, indiquer le numéro de la question et recopier sur la copie la lettre correspondante à la réponse choisie.

Aucune justification n’est demandée mais il peut être nécessaire d’effectuer des recherches au brouillon pour aider à déterminer votre réponse.

Chaque réponse correcte rapporte 1 point. Une réponse incorrecte ou une question sans réponse n’apporte ni ne retire de point.

\medskip

\textbf{Question 1}

\medskip 

On considère la loi de probabilité de la variable aléatoire $X$ donnée par le tableau ci-dessous :

\begin{center}
\begin{tabularx}{\linewidth}{|*{6}{>{\centering \arraybackslash} X|}} \hline
 $k$&$-5$&0&10&20&50\\
 \hline
 $p(X=k)$&0,71& 0,03 &0,01 &0,05& 0,2\\
 \hline
 \end{tabularx}
\end{center}

L’espérance de $X$ est :

\begin{center}
\begin{tabularx}{\linewidth}{*{4}{X}}
\textbf{a.~~} $15$ &\textbf{b.~~} $0,2$&\textbf{c.~~}$7,55$& \textbf{d.~~} $17$.
\end{tabularx}
\end{center}

\medskip

\textbf{Question 2}

\medskip 

On se place dans un repère orthonormé.

Le cercle de centre A$(-2~;~4)$ et de rayon 9 a pour équation :

\begin{center}
\begin{tabularx}{\linewidth}{*{4}{X}}
\textbf{a.~~} $(x+2)^2+(y-4)^2=81$ &\textbf{b.~~} $(x-2)^2+(y+4)^2=81$&\textbf{c.~~}$(x+2)^2+(y-4)^2=9$& \textbf{d.~~} $(x-2)^2+(y+4)^2=9$.
\end{tabularx}
\end{center}

\medskip

\textbf{Question 3}

\medskip 

\begin{minipage}[t]{8.64cm}
Soit $f$ la fonction définie par

$f(x) = ax^2+bx+c$ où $a$, $b$ et $c$ sont des réels.

On considère dans un repère la courbe représentative de $f$ tracée ci-contre.

On appelle $\Delta$ son discriminant.

On peut affirmer que :

\end{minipage}
\begin{minipage}[]{6.3cm}
\psset{showorigin=false,arrowsize=2pt 3}
\begin{pspicture*}(-2.75,-0.8)(3.4,4.26)
\psaxes[linewidth=0.75pt,labels=none,ticks=none]{->}(0,0)(-2.2,-0.8)(3.3,3.7)\uput[dl](0,0){O}
\def\Func{x 1.35 add x 2.4 sub mul 0.957 neg mul }
\psplot[plotpoints=2000,linewidth=1.25pt,linecolor=blue]{-1.8}{2.6}{\Func}
\end{pspicture*}
\end{minipage}

\medskip

\begin{tabularx}{\linewidth}{*{2}{X}}
\textbf{a.~~} $a > 0$ ou $c < 0$ &\textbf{c.~~}$a < 0$ et $c < 0$.\\
\textbf{b.~~} $c$ et $\Delta$ sont du même signe&\textbf{d.~~} $a < 0$ et $\Delta< 0$. 
\end{tabularx}

\medskip

\textbf{Question 4}

\medskip 

On considère la suite $\left(u_n\right)$ définie par $u_0 = - 2$ et $u_{n+1} = 2u_n - 5$.

Un algorithme permettant de calculer la somme $S = u_0 + u_1 + \dots+ u_{36}$ est :

\vspace{0.5cm}
\begin{tabularx}{\linewidth}{*{4}{X}}

\textbf{a.~~} 
\begin{minipage}[t]{2.9cm}
$U=-2$\\
$S=0$\\
Pour i de 1 à 37\\
$U \leftarrow 2U-5$\\
$S\leftarrow  S+U$\\
Fin Pour 
\end{minipage}
&

\textbf{b.~~}
 \begin{minipage}[t]{2.9cm}
$U=-2$\\
$S=0$\\
Pour i de 1 à 36\\
$U \leftarrow 2U-5$\\
$S \leftarrow  S+U$\\
Fin Pour
\end{minipage}
&

\textbf{c.~~}
 \begin{minipage}[t]{2.9cm}
$U=-2$\\
$S=-2$\\
Pour i de 1 à 37\\
$S \leftarrow  S+U$\\
$U \leftarrow 2U-5$\\
Fin Pour
\end{minipage}
&

\textbf{d.~~} 
\begin{minipage}[t]{2.9cm}
$U=-2$\\
$S=-2$\\
Pour i de 1 à 36\\
$U\leftarrow 2U-5$\\
$S \leftarrow S+U$\\
Fin Pour
\end{minipage}

\end{tabularx}

\medskip

\textbf{Question 5}

\medskip 


La suite $\left(u_n\right)$ définie par $u_0 =-2$ et $u_{n+1} = 2u_n - 5$ est :

\begin{center}
\begin{tabularx}{\linewidth}{*{4}{X}}
\textbf{a.~~} arithmétique mais pas géométrique &\textbf{b.~~}géométrique mais pas arithmétique &\textbf{c.~~}ni arithmétique, ni géométrique& \textbf{d.~~}à la fois arithmétique et géométrique.
\end{tabularx}
\end{center}

\vspace{0,5cm}

