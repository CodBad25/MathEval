 
\medskip

Un restaurant propose à sa carte deux desserts différents :
\begin{itemize}
\item  le premier dessert est un assortiment de macarons, et est choisi par 40\,\% des clients,
\item le second dessert est une part de tarte, et est choisie par 30 \,\% des clients.
\end{itemize}
Les autres clients ne prennent pas de dessert. Aucun client ne prend plusieurs desserts.

Le restaurateur a remarqué 
que parmi les clients ayant pris comme dessert un assortiment de macarons, 70\,\% prennent un café,
que parmi les clients ayant pris comme dessert une part de tarte 40\,\% prennent un café et,
que parmi les clients n’ayant pas pris de dessert 90\,\% prennent un café. 
 
On interroge au hasard un client de ce restaurant. On note :

\begin{itemize}
\item [*] $M$ l’évènement : « Le client prend un assortiment de macarons. »
\item [*] $T$ l’évènement : « Le client prend une part de tarte. »
\item [*] $N$ l’évènement : « Le client ne prend pas de dessert. »
\item [*]$C$ l’évènement : « Le client prend un café. »
\end{itemize}

\medskip

\begin{enumerate}
\item Construire un arbre de probabilités décrivant la situation.
\item Calculer $P(T\cap C)$  puis $P(C)$.
\item On rencontre un client ayant pris un café. Quelle est la probabilité qu’il ait pris une part de tarte ? On donnera le résultat sous forme d’une fraction irréductible.
\end{enumerate}

\vspace{0,5cm}

