
\medskip

On dispose d'un dé équilibré à six faces et de deux urnes U et V contenant des boules blanches ou rouges, indiscernables au toucher.

L'urne U contient $40$ boules blanches et $60$ boules rouges.

L'urne V contient $70$ boules blanches et $30$ boules rouges.

Un jeu consiste à lancer le dé puis tirer une boule dans l'une des urnes. Si on obtient 1 ou 6 sur le dé, le tirage s'effectue dans l'urne U. Si on obtient 2, 3, 4 ou 5 sur le dé, le tirage s'effectue dans l'urne V.

On considère les évènements :
\begin{itemize}
\item $U$ : \og le tirage s'effectue dans l'urne U \fg
\item $V$ : \og le tirage s'effectue dans l'urne V \fg
\item $B$ : \og la boule tirée est blanche \fg
\item $R$ : \og la boule tirée est rouge \fg.
\end{itemize}

Sauf indication contraire, les probabilités seront arrondies au millième.

\medskip

\begin{enumerate}
\item Représenter la situation à l'aide d'un arbre pondéré.
\item Déterminer la probabilité de l'évènement \og la boule tirée est rouge \fg.
\item On tire une boule rouge. Quelle est la probabilité qu'elle ait été tirée dans l'urne U ?
\item Pour jouer, il faut miser 1 \euro. Le joueur gagne 3 \euro{} s'il tire une boule rouge et il ne gagne rien s'il tire une boule blanche. On note $\mathcal{G}$  la variable aléatoire donnant le gain du joueur.
	\begin{enumerate}
		\item Déterminer la loi de probabilité de la variable aléatoire $\mathcal{G}$.

\emph{On donnera le tableau de la loi de probabilité, mais aucune justification n'est demandée.}
		\item Calculer l'espérance mathématique de $\mathcal{G}$. Interpréter ce résultat.
	\end{enumerate}
\end{enumerate}
