
\medskip

\emph{Dans tout l'exercice, on notera $p(E)$ la probabilité d'un évènement $E$.}

\bigskip

La répartition des $150$ adhérents d'un club de sport est donnée dans le tableau ci-dessous :

\medskip
\begin{center}
\begin{tabular}[]{|m{2.95cm}|*{4}{c|}}
\hline
\text{Âge} 				&15 \text{ans}	&16 \text{ans} 	&17 \text{ans} 	&18 \text{ans}\\\hline
\text{Nombre de filles}	&17				&39				&22				&10\\\hline
\text{Nombre de garçons}&13 			&36				&8 				&5\\\hline
\text{Total} 			&30				&75				&30				&15\\\hline
\end{tabular}
\end{center}

\medskip

On choisit un adhérent au hasard.

\medskip

\begin{enumerate}
\item Quelle est la probabilité que l'adhérent choisi ait 17 ans ?
\item L'adhérent choisi a 18 ans. Quelle est la probabilité que ce soit une fille ?
\end{enumerate}

On note $X$ la variable aléatoire donnant l'âge de l'adhérent choisi. 

\begin{enumerate}[resume]
\item  Déterminer la loi de probabilité de $X$.
\item Calculer $p(X\geqslant 16)$ et interpréter le résultat.
\item Calculer l'espérance de $X$. Interpréter le résultat.
\end{enumerate}

\vspace{0,5cm}

