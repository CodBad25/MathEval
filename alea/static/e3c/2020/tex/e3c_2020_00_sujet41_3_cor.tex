
\subsection*{Partie A}

\paragraph{1.} On sait que quel que soit \(n\), \( u_n = u_0 \times q^n \).

Donc : \(u_{19} = 82695 \times 0{,}999^{19} \approx 81137{,}9\).

\paragraph{2.}
\[
S = u_0 + u_1 + \dots + u_{19} \quad (1).
\]
Donc
\[
0{,}999S = u_1 + \dots + u_{19} + u_{20} \quad (2).
\]
Par différence \((1) - (2)\), on obtient :
\begin{align*}
&0{,}001S = u_0 - u_{20} \\
\iff &S = 1000(u_0 - u_{20}) \\
\iff &S = 1000(82695 - 82695 \times 0{,}999^{20}) \approx 1638281{,}8.
\end{align*}

\subsection*{Partie B}

Retrancher 0{,}1 \% revient à multiplier par \(1 - \dfrac{0{,}1}{100} = 1 - 0{,}001 = 0{,}999\).

Le terme \(u_0\) de la partie A est la population (en milliers) au 1er janvier 2016.

La population en 2035 est donc en milliers le terme \(u_19 \approx 81138 \) soit environ \(81138000\) habitants.

\subsection*{Partie C}

\paragraph{1.} Population (2) va donner la population en 2017, soit 826471005 puis celle de 2018 soit \( \approx 82647035 \).

\paragraph{2.} Population (19) donne la population 19 ans après 2016, soit en 2035. 
    
Donc au 1\(^\text{er}\) janvier 2035, il y aura environ 82243175 habitants.

