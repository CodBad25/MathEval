
\medskip

Une petite entreprise de textile commercialise des nappes et des lots de serviettes assorties. Un client achète au plus une nappe et au plus un lot de serviettes.

En consultant le fichier des ventes de l'entreprise, on constate que :

\setlength\parindent{9mm}
\begin{itemize}
\item[$\bullet~~$] 20\,\% des clients achètent une nappe ;
\item[$\bullet~~$] Parmi les clients ayant acheté une nappe, 70\,\% ont acheté un lot de serviettes;
\item[$\bullet~~$] Parmi les clients n'ayant pas acheté de nappe, 10\,\% ont tout de même acheté un lot de serviettes.
\end{itemize}
\setlength\parindent{0mm}

\smallskip

On choisit au hasard un client de cette entreprise.

 
Pour tout évènement $A$,on note $\overline{A}$ l'évènement contraire de $A$ et $P(A)$ la probabilité de l'évènement $A$.
 
On note les évènements suivants:

\setlength\parindent{9mm}
\begin{itemize}
\item[$\bullet~~$]$N$ \og le client achète une nappe \fg{} ;
\item[$\bullet~~$]$S$ \og le client achète un lot de serviettes \fg.
\end{itemize}
\setlength\parindent{0mm}

\medskip

\begin{enumerate}
\item Reproduire sur la copie et compléter l'arbre pondéré ci-dessous décrivant la situation:

\begin{center}
\pstree[treemode=R,nodesepA=0pt,nodesepB=3pt]{\TR{}}
{\pstree{\TR{$N$~~} \ncput*[nrot=:U]{\ldots}}
	{\TR{$S$} \ncput*[nrot=:U]{\ldots}
	\TR{$\overline{S}$} \ncput*[nrot=:U]{\ldots}
	}
\pstree{\TR{$\overline{N}~~$} \ncput*[nrot=:U]{\ldots}}
	{\TR{$S$} \ncput*[nrot=:U]{\ldots}
	\TR{$\overline{S}$} \ncput*[nrot=:U]{\ldots}
	}
}
\end{center}

\item  Calculer la probabilité que le client achète une nappe et un lot de serviettes.
\item  Montrer que la probabilité de l'évènement $S$ est égale à $0,22$.
\item  Calculer la probabilité que le client achète une nappe sachant qu'il a acheté une serviette.
\item  Une nappe est vendue $45$~\euro{} et un lot de serviettes $25$~\euro.

On appelle $D$ la variable aléatoire donnant la dépense effectuée par un client.

Calculer l'espérance mathématique de $D$ et donner une interprétation de ce nombre dans le contexte de l'exercice.
\end{enumerate}

\bigskip

