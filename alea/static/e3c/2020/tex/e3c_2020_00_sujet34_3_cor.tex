
\subsection*{Partie A.}

\paragraph{1.} La relation \(v_{n+1} = \dfrac{2}{3} v_n\), vraie pour tout entier naturel \(n\), montre que la suite \((v_n)\) est une suite géométrique de raison \(q = \dfrac{2}{3}\) et de premier terme \(v_0 = 1\).

\paragraph{2.} On sait que \(v_n = 1 \times \left(\dfrac{2}{3}\right)^n\), pour tout entier naturel \(n\).

\paragraph{3.}
\[
S_{10} = 1 + \dfrac{2}{3} + \left(\dfrac{2}{3}\right)^2 + \dots + \left(\dfrac{2}{3}\right)^9 (1).
\]
En multipliant (1) par \(\dfrac{2}{3}\), on obtient :
\[
\dfrac{2}{3} S_{10} = \dfrac{2}{3} + \left(\dfrac{2}{3}\right)^2 + \dots + \left(\dfrac{2}{3}\right)^{10} (2).
\]
En faisant la différence \((1) - (2)\), on obtient :
\[
\dfrac{1}{3} S_{10} = 1 - \left(\dfrac{2}{3}\right)^{10},
\]
d'où, en multipliant par \(3\), on obtient :
\[
S_{10} = 3 - 3\left(\dfrac{2}{3}\right)^{10} = \dfrac{58025}{19683} \approx 2{,}94798.
\]

\subsection*{Partie B.}

\paragraph{4.} \texttt{terme(5)} donne le sixième terme de la suite \((w_n)\) définie par \(w_1 = 4\) et \(w_{n+1} = 2w_n - 3\).

On obtient les termes successifs : \(4\,;\,5\,;\,7\,;\,11\,;\,19\,;\,35\).

\paragraph{5.}

\begin{center}
\begin{python}
def somme_termes(n) :	
	w = 4
	S = 4
	longueur = 2
	for i in range(n) :
		w = 2*w - 3
		S = S + w
	return S
\end{python}
\end{center}

