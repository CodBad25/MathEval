
\( f(x) = 3x^3 - 5x^2 + 2 \).

\subsection*{1.}

On a sur \( \mathbb{R} \) :
\begin{align*}
f'(x) &= 3 \times 3x^2 - 2 \times 5x \\
&= 9x^2 - 10x \\
&= x(9x - 10).
\end{align*}

\subsection*{2.}

\[
M(x \,;\, y) \in T \iff y - f(-1) = f'(-1)(x - (-1))
\]

Avec :
\begin{align*}
f(-1) &= 3 \times (-1)^3 - 5 \times (-1)^2 + 2 \\
&= -3 - 5 + 2 \\
&= -6, 
\end{align*}
et :
\begin{align*}
f'(-1) &= -1 \times (9 \times (-1) - 10) \\
&= -1 \times (-19) \\
&= 19,
\end{align*}
l'équation réduite devient :
\[
y = 19(x + 1) - 6 \quad \text{ou} \quad y = 19x + 13.
\]

\subsection*{3.}

\( g(x) = 3x^3 - 4x + 1 \).

\paragraph{a.} Pour tout réel \(x\) :
\begin{align*}
f(x) - g(x) &= 3x^3 - 5x^2 + 2 - (3x^3 - 4x + 1) \\
&= 3x^3 - 5x^2 + 2 - 3x^3 + 4x - 1 \\
&= -5x^2 + 4x + 1.
\end{align*}

\paragraph{b.} Pour le trinôme du second degré \( f(x) - g(x) \), on a :
\[
\Delta = 4^2 - 4 \times (-5) \times 1 = 16 + 20 = 36 = 6^2 > 0.
\]
Il y a donc deux racines :
\[
x_1 = \dfrac{-4 + 6}{2 \times (-5)} = -\dfrac{2}{-10} = -\dfrac{1}{5} \quad \text{et} \quad x_2  = \dfrac{-4 - 6}{2 \times (-5)} = \dfrac{-10}{-10} = 1.
\]

\paragraph{c.} On sait que le trinôme est négatif sauf entre les racines.

Donc \( f(x) - g(x) > 0\) sur l'intervalle \(\left]-\dfrac{1}{5}\,;\,1\right[\).

La courbe \( \mathcal{C}_f \) est au-dessus de la courbe \( \mathcal{C}_g \) pour \( -\dfrac{1}{5} < x < 1 \).

