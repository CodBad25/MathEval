
\medskip

Cet exercice est un questionnaire à choix multiple (QCM).
 Pour chaque question, une seule des quatre réponses proposées est exacte.

Une bonne réponse rapporte un point. Une mauvaise réponse, une réponse multiple ou l'absence de réponse ne rapporte ni n'enlève
aucun point.

Relevez sur votre copie le numéro de la question ainsi que la lettre correspondant à la
réponse choisie. 

Aucune justification n'est demandée.

\medskip

\textbf{Question 1}

\medskip 

\begin{minipage}[]{0.47\linewidth}
On considère une fonction $f$ définie sur $\R$ par :
$f(x) = ax^2+bx+c$ où $a$, $b$ et $c$  sont des nombres réels.

 $\Delta$  désigne la quantité $b^2-4ac$.

Parmi les affirmations suivantes, laquelle est cohérente avec la
représentation graphique, ci-contre, de cette fonction ?
\end{minipage}
\hfill
\begin{minipage}[]{0.49\linewidth}
\psset{xunit=0.7cm,yunit=0.35cm,labelFontSize=\scriptstyle,showorigin=false}
\begin{pspicture}(-6,-4.75)(3.5,11.96)
\psaxes[linewidth=0.95pt,Dx=2,Dy=5]{->}(0,0)(-4.5,-4.1)(3.5,11.4)
\def\Func{x 2.732 add x 0.732 sub mul  }
\psplot[plotpoints=2000,linewidth=1.25pt,linecolor=red]{-4.5}{2.6}{\Func}
\end{pspicture}

\end{minipage}
\vspace{0,5cm}

\begin{tabularx}{\linewidth}{*{4}{X}}
\textbf{a.~~} $a > 0$ et $\Delta > 0 $ &\textbf{b.~~} $a < 0$ et $\Delta < 0 $&\textbf{c.~~}$a > 0$ et $\Delta < 0 $& \textbf{d.~~} $a < 0$ et $\Delta > 0  $.
\end{tabularx}

\medskip

\textbf{Question 2}

\medskip 

Lors d'un jeu, on mise 1 euro et on tire une carte au hasard parmi 30 cartes numérotées
de 1 à 30.

On gagne 3 euros si le nombre porté sur la carte est premier, sinon, on ne gagne rien.

On détermine le gain algébrique en déduisant le montant de la mise de celui du gain.

On note $X$ la variable aléatoire qui prend pour valeur le gain algébrique.

Que vaut l'espérance $E(X)$ de la variable aléatoire $X$ ?

\medskip

\begin{tabularx}{\linewidth}{*{4}{X}}
\textbf{a.~~} $\dfrac{1}{3} $ &\textbf{b.~~} $\dfrac{1}{10} $&\textbf{c.~~}$ 0$& \textbf{d.~~} $\dfrac{2}{3}$.
\end{tabularx}

\medskip

\textbf{Question 3}

\medskip

Quelle est la valeur exacte de $\dfrac{\e^6\times\e^3}{\e^2}$ ?

\medskip

\begin{tabularx}{\linewidth}{*{4}{X}}
\textbf{a.~~} $\e^{11}$ &\textbf{b.~~} $\e^{9}  $&\textbf{c.~~}$\e^{7} $& \textbf{d.~~} $ \e^{-7} $.
\end{tabularx}

\medskip

\textbf{Question 4}

\medskip 

On considère la suite arithmétique $\left(u_n\right)$ de raison $-5$ et telle que $u_1 = 2$. Quelle est, pour tout entier naturel $n$, l'expression du terme général $u_n$ de cette suite ?

\medskip

\begin{tabularx}{\linewidth}{*{4}{X}}
\textbf{a.~~} $u_n = 2- 5n $ &\textbf{b.~~} $u_n = -5 + 2n  $&\textbf{c.~~}$ u_n = 7 - 5n  $& \textbf{d.~~} $u_n = 2 \times(-5)^n $.
\end{tabularx}

\medskip

\textbf{Question 5}

\medskip 

Les équations cartésiennes ci-dessous sont celles de droites données du plan. Le vecteur
$\vv{u}\ \dbinom{-1}{2}$  est un vecteur normal à l'une de ces droites.

Quelle est l'équation de cette droite ?

\medskip

\begin{tabularx}{\linewidth}{*{4}{X}}
\textbf{a.~~} $2x + y+ 5 = 0 $ &\textbf{b.~~} $x + 2y + 3 = 0  $&\textbf{c.~~}$ -x + 0,5y+ 2 = 0$& \textbf{d.~~} $ -4x + 8y = 0 $.
\end{tabularx}

\vspace{0,5cm}

