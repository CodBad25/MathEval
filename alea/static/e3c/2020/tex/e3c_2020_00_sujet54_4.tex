
\medskip

$150$ élèves d'un établissement sont inscrits aux activités du midi :
\begin{itemize}
\item $30$ sont inscrits en musique.
\item $45$ sont inscrits en sport.
\item $75$ sont inscrits en cinéma.
\end{itemize} 

Chaque élève pratique une et une seule activité. 
\begin{itemize}
\item Parmi les élèves inscrits en musique, 30\,\% sont des filles.
\item Parmi les élèves inscrits en sport, 60\,\% sont des filles.
\item Parmi les élèves inscrits en cinéma, 72\,\% sont des filles.
\end{itemize}
On choisit au hasard un élève inscrit aux activités du midi.

On note : 
\begin{itemize}
\item $F$ l'évènement : \og l'élève choisi est une fille \fg,
\item $M$ l'évènement : \og l'élève choisi est inscrit en musique \fg,
\item $S$ l'évènement : \og l'élève choisi est inscrit en sport \fg, 
\item $C$ l'évènement : \og l'élève choisi est inscrit en cinéma \fg.
\end{itemize}

\medskip

\begin{enumerate}
\item  Recopier et compléter l'arbre pondéré représentant la situation.
\begin{center}
\psset{nodesepA=0pt,nodesepB=3pt,treesep=0.75,labelsep=0.1pt,levelsep=2.5cm}
\pstree[treemode=R]{\TR{}}
{\pstree{\TR{$M$~~}\taput{$\np{0.2}$}}
	{
	\TR{$F$}\taput{}
	\TR{$\overline{F}$}\tbput{}
	}
\pstree{\TR{$S$~~}\tbput{}}
	{\TR{$F$}\taput{}
	\TR{$\overline{F}$}\tbput{}
	}
\pstree{\TR{$C$~~}\tbput{}}
	{\TR{$F$}\taput{}
	\TR{$\overline{F}$}\tbput{}
	}
}
\end{center}
\item Calculer la probabilité que l'élève choisi soit une fille inscrite en musique.
\item Montrer que la probabilité que l'élève choisi soit une fille est égale à $0,6$.
\item Les évènements $M$ et $F$ sont-ils indépendants ?
\item Sachant que l'élève choisi est un garçon, calculer la probabilité qu'il soit inscrit en cinéma.
\end{enumerate}
