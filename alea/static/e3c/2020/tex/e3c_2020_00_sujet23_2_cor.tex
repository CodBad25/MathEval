	
	\textbf{Question 1}
	
	Les points de l’axe des ordonnées sont caractérisés par $x = 0$, donc :
	\[
	f(0) = \dfrac{e^0}{1+0} = \dfrac{1}{1} = 1.
	\]
	Donc $A(0;1)$.
	
	\textbf{Question 2}
	
	Les points de l’axe des abscisses sont caractérisés par $y = 0$, donc :
	\[
	\dfrac{e^x}{1 + x} = 0.
	\]
	Or, sur $[0; +\infty[$, $e^x > 0$ et $1 + x > 1 > 0$, donc $f(x) > 0$.\\
	 La courbe $C_f$ ne coupe pas l'axe des abscisses.
	
	\textbf{Question 3}
	
	$f$ est dérivable sur $[0; +\infty[$ car c’est un quotient de fonctions dérivables sur cet intervalle :
	\[
	f'(x) = \dfrac{e^x(1 + x) - e^x}{(1 + x)^2} = \dfrac{x e^x}{(1 + x)^2}.
	\]
	
	\textbf{Question 4}
	
	Tous les termes de $f'(x)$ sont positifs, donc $f'(x) > 0$. La fonction $f$ est donc croissante sur $[0; +\infty[$.
	
	\textbf{Question 5}
	
	Une équation de la tangente à $C_f$ au point $B$ d’abscisse $1,6$ est :
	\[
	y - f(1,6) = f'(1,6)(x - 1,6).
	\]
	Avec $f(1,6) = \dfrac{e^{1,6}}{2,6}$ et $f'(1,6) = \dfrac{1,6 e^{1,6}}{2,6^2}$, l'équation devient :
	\[
	y - \dfrac{e^{1,6}}{2,6} = \dfrac{1,6 e^{1,6}}{2,6^2}(x - 1,6).
	\]
Le couple $(0;0)$ ne vérifie pas cette équation donc le point $O(0; 0)$ n'appartient pas à $C_f$.
	
