	
	\subsection*{Question 1}
	
	On sait que \(u_4 = u_0 + 4a = 3\) et \(u_{10} = u_0 + 10a = 18\) (a étant la raison de la suite). Par différence, on obtient :
	\[
	u_{10} - u_4 = 18 - 3 = 6a \quad \text{ou} \quad 6a = 15 \quad \text{ou encore} \quad 2a = 5 \quad \text{et} \quad a = \dfrac{5}{2}.
	\]
	
	On en déduit que \(u_0 = u_4 - 4a = 3 - 4 \times \dfrac{5}{2} = 3 - 10 = -7\).
	
	Donc en particulier :
	\[
	u_{12} = u_0 + 12a = -7 + 12 \times \dfrac{5}{2} = -7 + 30 = 23.
	\]
	
	\subsection*{Question 2}
	
	\[
	S = 2 + 3 + 4 + \ldots + 999 + 1000 \quad \text{et} \quad S = 1000 + 999 + 998 + \ldots + 3 + 2.
	\]
	
	En sommant par colonne :
	\[
	2S = 1002 + 1002 + \ldots + 1002 = 999 \times 1002 \quad \text{d’où} \quad S = 999 \times 501 = 500499.
	\]
	
	\subsection*{Question 3}
	
	On sait que quel que soit le naturel \(n\), \(v_n = -3 \times 0,3^n\).
	
	Or,
	\[
	\lim_{n \to +\infty} 0,3^n = 0 \quad \text{et} \quad \lim_{n \to +\infty} -3 \times 0,3^n = 0.
	\]
	
	\subsection*{Question 4}
	
	Le coefficient \(a = -2 < 0\), donc la parabole est tournée vers le bas : elle est donc croissante sur \(\left]-\infty ; -2\right[\), le maximum étant \(f(-2) = -3\), puis décroissante sur \(\left[-2 ; +\infty\right[\).
	
	\subsection*{Question 5}
	
	Pour le trinôme \(x^2 - 5x + 6 = (x - 2)(x - 3)\), les racines sont 2 et 3. On sait que ce trinôme est positif sauf sur l’intervalle \(\left]2 ; 3\right[\) où \(f(x) < 0\).
	
