
\subsection*{Question 1}

Le coefficient directeur de la tangente \(\mathcal{T}\) est égal à 5 et son ordonnée à l'origine est 18.

L'équation réduite de \(\mathcal{T}\) est donc :
\[
y = 5x + 18.
\]

\subsection*{Question 2}

La seule fonction positive sur \( \,]-\infty \,;\, -2[ \) et sur \( \,]2 \,;\, +\infty[ \), négative ailleurs est la fonction de \textbf{b.}

\subsection*{Question 3}

On a : \(\cos(x + \pi) + \sin\left(x + \dfrac{\pi}{2}\right) = -\cos x + \cos x = 0\).

\subsection*{Question 4}

On cherche les racines du trinôme :
\[
\Delta = 16 - 4 \times (-2) \times 6 = 16 + 64 = 64 > 0,
\]
il y a donc deux racines :
\[
x_1 = \frac{-4 + \sqrt{64}}{-4} = -1 \quad \text{et} \quad x_2 = \frac{-4 - \sqrt{64}}{-4} = 3.
\]
On sait que le trinôme est négatif sauf entre les racines où il est positif.

Réponse : \( \,]-1 \,;\, 3[\).

\subsection*{Question 5}

\(h\) est une fonction produit de deux fonctions dérivables sur \(\mathbb{R}\). Sur cet intervalle, on a donc :
\[
h'(x) = 2\e^x + (2x - 1)\e^x = \e^x (2 + 2x - 1) = \e^x (2x + 1).
\]

