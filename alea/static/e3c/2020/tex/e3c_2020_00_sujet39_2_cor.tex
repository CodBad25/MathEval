
\subsection*{1.}

La fonction \( f(x) = x^2 - 3x + 4 \) est un polynôme dérivable sur \( \mathbb{R} \), donc sur \( [0\,;\,+\infty[ \). Sur cet intervalle :
\[
f'(x) = 2x - 3.
\]

\(2x - 3 < 0 \iff x < \dfrac{3}{2}\), donc \(f\) est décroissante sur \( \left]0\,;\,\dfrac{3}{2} \right[\) puis croissante sur \( \left]\dfrac{3}{2}\,;\,+\infty \right[ \).

Le minimum est :
\[
f\left(\dfrac{3}{2}\right) = \dfrac{9}{4} - \dfrac{9}{2} + 4 = \dfrac{7}{4}.
\]

\subsection*{2.}

\paragraph{a.} \(\mathcal{C}\) est la représentation graphique de la fonction \(x \longmapsto \sqrt{x}\), donc \( M(x \,;\, \sqrt{x}) \).

\paragraph{b.} On a donc :
\begin{align*}
AM^2 &= (x - 2)^2 + (\sqrt{x} - 0)^2 \\
&= x^2 + 4 - 4x + x \\
&= x^2 - 3x + 4.
\end{align*}

\paragraph{c.} Pour le trinôme \(x^2 - 3x + 4\), on a :
\[
\Delta = 9 - 16 = -7 < 0.
\]
Ce trinôme n'a pas de racines et on sait que son minimum est atteint pour :
\[
x = -\dfrac{b}{2a} = -\dfrac{-3}{2} = \dfrac{3}{2}.
\]
Donc le point correspondant au point de \(C\) le plus proche de \(A\) a pour coordonnées \( \left( \dfrac{3}{2} \,;\, \sqrt{\dfrac{3}{2}} \right).  \textit{ Ce point est noté } B \textit{ pour la suite}\).

\paragraph{d.} On a pour \(x \neq 0\), \(f'(x) = \dfrac{1}{2\sqrt{x}}\).

La tangente en \(B\) a pour coefficient directeur :
\[
f'\left( \dfrac{3}{2} \right) = \dfrac{1}{2 \sqrt{\dfrac{3}{2}}} = \dfrac{1}{\sqrt{2} \times \sqrt{3}} = \dfrac{1}{\sqrt{6}}.
\]
La droite \((AB)\) a pour coefficient directeur :
\[
\dfrac{\sqrt{\dfrac{3}{2}} - 0}{\dfrac{3}{2} - 2} = \dfrac{\sqrt{\dfrac{3}{2}}}{-\dfrac{1}{2}} = -2 \sqrt{\dfrac{3}{2}}.
\]
Le produit des deux coefficients directeurs est :
\[
\dfrac{1}{\sqrt{6}} \times \left( -2 \sqrt{\dfrac{3}{2}} \right) = \dfrac{-2\sqrt{3}}{\sqrt{2} \times \sqrt{3} \times \sqrt{2}} = \dfrac{-2}{2} = -1.
\]
L'élève a raison, les droites sont bien perpendiculaires.

