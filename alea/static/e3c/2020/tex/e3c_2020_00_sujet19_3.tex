
\medskip

On considère la fonction $f$ définie sur l’intervalle $[-2~;~2]$ par 

\[f(x) = 2x^3 + 2x^2 - 2x + 3\]

Soit $\mathcal{C}$ sa représentation graphique dans le repère suivant.

\psset{xunit=2.75cm,yunit=0.4cm,labelFontSize=\scriptstyle,showorigin=false}
\begin{pspicture}(-2.5,-4)(2.4,24)
\multido{\n=-2+1}{5}{\psline[linewidth=0.25pt,linecolor=lightgray](\n,-3)(\n,22)}
\multido{\n=-2+2}{13}{\psline[linewidth=0.25pt,linecolor=lightgray](-2.2,\n)(2.3,\n)}
\psaxes[linewidth=0.95pt,Dy=2]{->}(0,0)(-2.2,-3)(2.5,23.4)
\def\Func{x x x 1 add mul  1 sub 2 mul mul 3 add }
\psplot[plotpoints=1000,linewidth=1.25pt,linecolor=red]{-2}{2}{\Func}
\end{pspicture}
\begin{enumerate}
\item  On considère la droite $d$ d’équation $y = 2x+3$.
\begin{enumerate}
\item  Montrer que déterminer les abscisses des points d’intersection entre la droite $d$ et la courbe $\mathcal{C}$ revient à résoudre l’équation $2x(x^2+x-2)=0$ sur l’intervalle $[-2~;~2]$.
\item Déterminer les coordonnées des points d’intersection entre $d$ et $\mathcal{C}$.
\end{enumerate}
\item On considère la droite $d'$ d'équation $y=2x+a$ où $a$ est un nombre réel.

À l’aide du graphique, donner une valeur de $a$ pour laquelle la droite $d'$ et la courbe $\mathcal{C}$ ont un seul point d’intersection.
\item On note $f'$ la fonction dérivée de $f$.
\begin{enumerate}
\item  Démontrer que, pour tout nombre réel $x$ appartenant à l’intervalle $[-2~;~2]$, \[f'(x)=6\left(x+1\right)\left(x-\frac{1}{3}\right).\]
\item Étudier les variations de $f$ sur l’intervalle $[- 2~;~2]$. 
\end{enumerate}
\end{enumerate}

\vspace{0,5cm}

