
\medskip

Le 1\up{er} janvier 2019, le propriétaire d'un appartement a fixé à 650 euros le montant des loyers mensuels pour l'année 2019. Chaque 1\up{er} janvier, le propriétaire augmente de 1,52\,\% le loyer mensuel.

On modélise l'évolution du montant des loyers mensuels par une suite $\left(u_n\right)$. L'arrondi à l'unité du terme $u_n$ représente le montant, en euros, du loyer mensuel fixé le 1\up{er} janvier de l'année $(2019+n)$, pour $n$ entier naturel. Ainsi $u_0 = 650$ euros.

\medskip

\begin{enumerate}
\item 
	\begin{enumerate}
		\item Calculer le montant du loyer mensuel fixé le 1\up{er} janvier 2020.
		\item Quelle est la nature de la suite $\left(u_n\right)$ ? Préciser sa raison et son premier terme.
		\item Calculer le montant du loyer mensuel qui, selon ce modèle, sera fixé pour l'année 2027.
	\end{enumerate}
\item Pour calculer la somme totale des loyers perçus par le propriétaire durant les années 2019
à 2019+A, on utilise la fonction ci-dessous, écrite en langage Python.

\begin{python}
1 def somme(A):
2 	S=0
3 	n=0
4 	while n<=A:
5		S=S+7800*1.0152**n
6 		n = n + 1
7 	return S
\end{python}

L'exécution de ce programme pour quelques valeurs de A donne les résultats ci-dessous :

\begin{center}
%\rowcolors{1}{}{aliceblue}
\begin{tabular}[]{|l}
>\! >\! > somme(\textcolor{blue}{0})\\
\textcolor{blue}{7800.0}\\
>\! >\! > somme(\textcolor{blue}{1})\\
\textcolor{blue}{15718.560000000001}\\
>\! >\! > somme(\textcolor{blue}{2})\\
\textcolor{blue}{23757.482112000005}\\
>\! >\! > somme(\textcolor{blue}{3})\\
\textcolor{blue}{31918.595440102407}\\
>\! >\! > somme(\textcolor{blue}{8})\\
\textcolor{blue}{74623.04180934158}\\
\end{tabular}
\end{center}
	\begin{enumerate}
		\item Interpréter, dans le contexte de l'exercice, le résultat obtenu lors de l'appel somme(1).
		\item Déterminer la somme totale des loyers perçus par le propriétaire durant les années
2022 à 2027 incluses. On arrondira le résultat à l'unité.
	\end{enumerate}
\end{enumerate}
