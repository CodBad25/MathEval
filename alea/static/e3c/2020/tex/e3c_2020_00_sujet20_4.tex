
\medskip

Pierre joue à un jeu dont une partie est constituée d'un lancer d'une fléchette sur une cible suivi d'un tirage au sort dans deux urnes contenant des tickets marqués \og gagnant\fg ou
\og perdant\fg{} indiscernables.

\setlength\parindent{8mm}
\begin{itemize}
\item[$\bullet~~$] S'il tire un ticket marqué \og gagnant \fg, il pourra recommencer une partie.
\item[$\bullet~~$] S'il atteint le centre de la cible, Pierre tire un ticket dans l'urne $U_1$ contenant exactement neuf tickets marqués \og gagnant\fg et un ticket marqué \og perdant \fg.
\item[$\bullet~~$] S'il n'atteint pas le centre de la cible (donc même s'il n'atteint pas la cible), Pierre tire un ticket dans l'urne $U_2$ contenant exactement quatre tickets marqués \og gagnant\fg et six tickets marqués \og perdant \fg.
\end{itemize}
\setlength\parindent{0mm}

\smallskip

Pierre atteint le centre de la cible avec une probabilité de 0,3.

\smallskip
On note les évènements suivants:

\qquad $C$ : \og Pierre atteint le centre de la cible\fg{} ;

\qquad $G$ : \og Pierre tire un ticket lui offrant une autre partie \fg.

\medskip

\begin{enumerate}
\item Recopier l'arbre pondéré ci-dessous et justifier la valeur $0,9$.

\begin{center}
\pstree[treemode=R,nodesepA=0pt,nodesepB=3pt]{\TR{}}
{\pstree{\TR{$C$~~}\naput{0,3}}
	{\TR{$G$} \naput{0,9}
	\TR{$\overline{G}$}\nbput{\ldots}
	}
\pstree{\TR{$\overline{C}$~~}\nbput{\ldots}}
	{\TR{$G$} \naput{\ldots}
	\TR{$\overline{G}$}\nbput{\ldots}
	}
}
\end{center}

\item Compléter sur la copie l'arbre pondéré en traduisant les données de l'exercice.
\item Calculer la probabilité de l'évènement $\overline{C} \cap G$.
\item Montrer que la probabilité qu'à l'issue d'une partie Pierre en gagne une nouvelle est égale à $0,55$.
\item Sachant que Pierre a gagné une nouvelle partie, quelle est la probabilité qu'il ait atteint le centre de la cible ? Arrondir le résultat à $10^{-3}$.
\end{enumerate}
