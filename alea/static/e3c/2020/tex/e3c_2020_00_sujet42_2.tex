
\medskip

Un fromager fait l'inventaire des produits qu'il a en cave. 

Le graphique ci-dessous indique la répartition de ses trois types de fromages : au lait de chèvre, au lait de vache ou au lait de brebis.

\begin{center}
\psset{unit=1cm}
\begin{pspicture}(-2,-2)(2,3)
\rput(0,2.8){\large Répartition des types de fromage}
\pscircle(0,0){2}
\pswedge(0,0){2}{0}{54}\rput(1.6;35){\footnotesize fromage au }\rput(1.5;20){\footnotesize lait de brebis}
\pswedge(0,0){2}{54}{144}\rput(1.6;90){\footnotesize fromage au lait}\rput(1.3;85){\footnotesize de chèvre}
\pswedge(0,0){2}{144}{360}\rput(1.3;270){fromage au lait}\rput(1.6;270){de vache}
\end{pspicture}
\end{center}

Chacun de ses trois types de fromages se partage en deux catégories: frais ou affiné. Le tableau suivant donne la répartition des fromages de chaque catégorie suivant leur affinage:

\begin{center}
\begin{tabularx}{\linewidth}{|m{3cm}|*{2}{>{\centering \arraybackslash}X|}}\hline
&frais &affiné\\ \hline
Lait de vache & 20\,\% &80\,\% \\ \hline
Lait de chèvre&40\,\%& 60\,\%\\ \hline
Lait de brebis&70\,\% &30\,\%\\ \hline
\end{tabularx}
\end{center}

Le fromager prend un fromage au hasard. On note les évènements suivants :

\setlength\parindent{1cm}
\begin{itemize}[label=\textbullet]
\item $V$ : \og le fromage est fait avec du lait de vache \fg{};
\item $C$ : \og le fromage est fait avec du lait de chèvre \fg{} ;
\item $B$ : \og le fromage est fait avec du lait de brebis \fg{} ;
\item $F$ : \og le fromage est frais \fg{} ;
\item $A$ : \og le fromage est affiné \fg.
\end{itemize}
\setlength\parindent{0cm}

\medskip

\begin{enumerate}
\item Donner les probabilités $P_C(A)$ et $P(B)$.
\item Démontrer que $P(A) = 0,675$.
\item Le fromager prend au hasard un fromage affiné. Quelle est la probabilité qu'il
s'agisse d'un fromage au lait de vache? On donnera le résultat à $10^{-3}$ près.
\end{enumerate}

\bigskip

