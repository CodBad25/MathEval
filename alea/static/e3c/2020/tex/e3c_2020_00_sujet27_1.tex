
\medskip

Ce QCM comprend 5 questions.

 Pour chacune des questions, une seule des quatre réponses proposées est correcte. 

Les questions sont indépendantes.

Pour chaque question, indiquer le numéro de la question et recopier sur la copie la lettre
correspondante à la réponse choisie.

 Aucune justification n'est demandée mais il peut être
nécessaire d'effectuer des recherches au brouillon pour aider à déterminer votre réponse.

Chaque réponse correcte rapporte 1 point. Une réponse incorrecte ou une question sans
réponse n'apporte ni ne retire de point.

\medskip

\textbf{Question 1 }

\medskip 

Soit $g$ la fonction définie sur $\R$ par $g(x) = \e^{100x}$ . Alors :

\medskip 

\begin{tabularx}{\linewidth}{*{2}{X}}
\textbf{a.~~} $g$ est croissante sur $\R$. &\textbf{b.~~} $g $ est décroissante sur $\R$. \\\textbf{c.~~}$g$ change de sens de variation sur $\R$. & \textbf{d.~~} aucune des propositions \textbf{a.~}, \textbf{b.~} et \textbf{c.~} n'est correcte.
\end{tabularx}

\medskip

\textbf{Question 2}

\medskip 


Soit $f$ la fonction définie sur $\R$ par $f(x) = 100x^2 + 10x + 1$. Dans le plan muni d'un repère
orthogonal, la courbe représentative de la fonction $f$ est une parabole dont l'axe de
symétrie a pour équation :

\medskip 

\begin{tabularx}{\linewidth}{*{4}{X}}
\textbf{a.~~} $x = 10 $ &\textbf{b.~~} $x = -10 $&\textbf{c.~~}$x = 0,05 $& \textbf{d.~~} $ x = -0,05 $.
\end{tabularx}

\medskip

\textbf{Question 3}

\medskip 

Soit $a$ et $b$ les fonctions définies sur $\R$ par $a(x) = 3x^2 + 15x+ 1$ et
$b(x) = 25x^2 + 5x - 100$. Dans le plan muni d'un repère orthonormé les courbes
représentatives des fonctions $a$ et $b$ ont :

\medskip 

\begin{tabularx}{\linewidth}{*{2}{X}}
\textbf{a.~~} 0 point d'intersection &\textbf{b.~~} 1 point d'intersection\\\textbf{c.~~}2 points d'intersection& \textbf{d.~~}4 points d'intersection.
\end{tabularx}

\medskip

\textbf{Question 4}

\medskip 

La somme $1 + 5 + 5^2 + \dots + 5^{10}$ est égale à :

\medskip 
\begin{tabularx}{\linewidth}{*{4}{X}}
\textbf{a.~~} $ \np{2 441 406}$ &\textbf{b.~~} $271 $&\textbf{c.~~}$ 5^{55}$& \textbf{d.~~} $ \np{12 207 031} $.
\end{tabularx}

\medskip

\textbf{Question 5}

\medskip 

Soit $f$ la fonction définie sur $\R$ dont la représentation graphique $\mathcal{C}_f$ est donnée ci-dessous.
On sait de plus que la courbe $\mathcal{C}_f$ admet deux tangentes horizontales : une au point d'abscisse
$-1$ et l'autre au point d'abscisse 3.

\begin{center}
\psset{xunit=1.2cm,yunit=1.2cm,labelFontSize=\scriptstyle,showorigin=false}
\begin{pspicture}(-4,-3.5)(4.7,1.6)

\psaxes[linewidth=0.95pt]{->}(0,0)(-3.5,-3.5)(4.3,1.4)
\def\Func{x x x 12 div 0.25 sub mul 0.75 sub mul 0.17 add}
\psplot[plotpoints=1000,linewidth=1.25pt,linecolor=red]{-3.3}{4.3}{\Func}
\psline[linewidth=0.5pt,linestyle=dashed](-1,0)(-1,0.58333)
\psline[linewidth=0.5pt,linestyle=dashed](3,0)(3,-2.08333)
\psline[linewidth=0.75pt,]{<->}(-1.5,0.58333)(-0.5,0.58333)
\psline[linewidth=0.75pt]{<->}(3.5,-2.08333)(2.5,-2.08333)
\uput[ur](-2.5,0.3){\red $\mathcal{C}_f$}\uput[dl](0,0){O}
\end{pspicture}
\end{center}


Alors le réel $f'(-1)\times f'(3)$  est :

\smallskip

\begin{tabularx}{\linewidth}{*{4}{X}}
\textbf{a.~~}strictement positif &\textbf{b.~~}strictement négatif &\textbf{c.~~} égal à 0& \textbf{d.~~}égal à $f'(-3)  $.
\end{tabularx}

\vspace{1cm}

