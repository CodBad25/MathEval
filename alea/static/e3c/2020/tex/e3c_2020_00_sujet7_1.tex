
\medskip

Ce QCM comprend 5 questions.

Pour chacune des questions, une seule des quatre réponses proposées est correcte.

Les questions sont indépendantes.

Pour chaque question, indiquer le numéro de la question et recopier sur la copie la lettre correspondante à la réponse choisie.

Aucune justification n'est demandée mais il peut être nécessaire d'effectuer des recherches au brouillon pour aider à déterminer votre réponse.

Chaque réponse correcte rapporte $1$ point. Une réponse incorrecte ou une question sans réponse n'apporte ni ne retire de point.

\medskip

\textbf{Question 1} 

On considère les points E$(3~;~-4)$ et F(7~;~2).

La droite (EF) passe par le point:

\begin{center}
\begin{tabularx}{\linewidth}{|*{4}{X|}}\hline
\textbf{a.~~}A(0~;~8)&\textbf{b.~~}B(5,5~;~0)&\textbf{c.~~}C(13~;~11)&\textbf{d.~~}D$(-25~;~45)$\\ \hline
\end{tabularx}
\end{center}


\textbf{Question 2}

On considère la droite $D$ qui a pour équation réduite $y = - 2x + 4$.

Parmi les vecteurs suivants, déterminer celui qui est un vecteur normal de la droite $D$ :

\begin{center}
\begin{tabularx}{\linewidth}{|*{4}{X|}}\hline
\textbf{a.~~}$\vect{n_1}(2~;~1)$\rule{0pt}{12pt}&\textbf{b.~~}$\vect{n_2}(-1~;~2)$&\textbf{c.~~}$\vect{n_3}(1~;~-2)$&\textbf{d.~~}$\vect{n_4}(-2~;~1)$\\ \hline
\end{tabularx}
\end{center}

\textbf{Question 3}

Soit ABCD un carré de côté 6 et I le milieu de [BC]. Alors le produit scalaire $\vect{\text{AD}} \cdot \vect{\text{AI}}$ vaut :

\begin{center}
\begin{tabularx}{\linewidth}{|*{4}{X|}}\hline
\textbf{a.~~}$- 18$&\textbf{b.~~}$18$&\textbf{c.~~}$36$&\textbf{d.~~}$9\sqrt{5}$\\ \hline
\end{tabularx}
\end{center}

\textbf{Question 4}

Sur le cercle trigonométrique ci-dessous, le nombre $\dfrac{14\pi}{3}$ a pour image le point :

\begin{center}
\psset{unit=1cm}
\begin{pspicture}(-2.1,-2.1)(2.1,2.1)
\pscircle(0,0){2}
\psdots(2;60)(2;120)(2;-60)(2;-120)
\uput[ur](2;60){E}\uput[ul](2;120){F}\uput[dl](2;-120){G}\uput[dr](2;-60){H}
\psline(-2.1,0)(2.1,0)\psline(0,-2.1)(0,2.1)
\end{pspicture}
\end{center}

\begin{center}
\begin{tabularx}{\linewidth}{|*{4}{X|}}\hline
\textbf{a.~~}E&\textbf{b.~~}F&\textbf{c.~~}G&\textbf{d.~~}H\\ \hline
\end{tabularx}
\end{center}

\textbf{Question 5}
Soit le réel $x$ appartenant à l'intervalle $\left[\dfrac{\pi}{2}~;~\pi\right]$ tel que $\sin x = 0,8$. Alors :

\begin{center}
\begin{tabularx}{\linewidth}{|*{4}{X|}}\hline
\textbf{a.~~}$\cos(x)= 0,6 $&\textbf{b.~~}$\cos(x)=- 0,6$&\textbf{c.~~}$\cos(x)=0,2$&\textbf{d.~~}$\cos(x)=- 0,2$\\ \hline
\end{tabularx}
\end{center}

\bigskip

