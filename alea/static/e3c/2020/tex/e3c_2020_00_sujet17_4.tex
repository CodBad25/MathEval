
\medskip

\psset{labelFontSize=\scriptstyle,showorigin=false}
\begin{pspicture}(-7,-3)(4,4)
 \psline[linewidth=0.5pt](-2,0)(3,0)
\psline[linewidth=0.5pt](0,-2)(0,3)
\psline[linewidth=0.5pt](3,0)(3,3)
\psline[linewidth=0.5pt](0,3)(3,3)
 \psline[linewidth=0.5pt](-2,0)(-2,-2)
\psline[linewidth=0.5pt](-2,-2)(0,-2)
\psline[linewidth=0.5pt,linestyle=dashed](3,0)(3,-2)
\psline[linewidth=0.5pt,linestyle=dashed](0,-2)(3,-2)
\uput[ul](0,0){O}\uput[ul](0,3){C}\uput[ur](3,3){B}\uput[r](3,0){A}\uput[dr](3,-2){M}
\uput[ul](-2,0){D}\uput[dl](-2,-2){E}\uput[d](0,-2){F}
\psplot[plotpoints=1000, linewidth=0.5pt, linecolor=lightgray]{-3.3333}{1.24}{x 1.5 mul 3 add}
\psplot[plotpoints=1000, linewidth=0.5pt, linecolor=lightgray]{-4.5}{3.4}{x 0.6667 neg  mul }
\end{pspicture}

OABC et ODEF sont des carrés de côtés respectifs 3 et 2. OAMF est un rectangle.
On note H le projeté orthogonal du point M sur la droite (DC).

Dans cet exercice, on pourra, si on le souhaite, se placer dans le repère
$(\text{O},\frac{1}{3}\,\vv{\text{OA}} ,\frac{1}{3}\,\vv{\text{OC}} )$.

\medskip

\begin{enumerate}
\item La droite (OM) est-elle perpendiculaire à la droite (DC) ?
\item Calculer $\vv{\text{CD}}\cdot \vv{\text{CM}}$.
\item Déterminer la longueur CH.
\end{enumerate}
