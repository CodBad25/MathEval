
\medskip

Pour placer un capital de \np{5000} euros, une banque propose un placement à taux fixe de 5\,\%
par an. Avec ce placement, le capital augmente de 5\,\% chaque année par rapport à l'année
précédente. Pour bénéficier de ce taux avantageux, il ne faut effectuer aucun retrait
d'argent durant les quinze premières années.

On modélise l'évolution du capital disponible par une suite $\left(u_n\right)$. On note $u_n$ le capital
disponible après $n$ années de placement.

On dépose \np{5000} euros le 1\up{er} janvier 2020. Ainsi $u_0 = \np{5000} $.

\medskip

\begin{enumerate}
\item Montrer que $u_2 = \np{5512,5}$. Interpréter ce résultat dans le contexte de l'exercice.
\item Exprimer $u_{n+1}$ en fonction de $u_n$.
\item Quelle est la nature de la suite $\left(u_n\right)$ ? Préciser son premier terme et sa raison.
\item Exprimer $u_n$ en fonction de $n$.
\item Justifier que le capital aura doublé après $15$ années de placement.
\end{enumerate}

\vspace{0,5cm}

