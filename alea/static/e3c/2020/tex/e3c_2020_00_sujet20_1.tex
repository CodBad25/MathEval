
\medskip

Cet exercice est un questionnaire à choix multiple (QCM) comportant cinq questions. Pour chacune des questions, une seule des quatre réponses proposées est correcte.

Les questions sont indépendantes.

Pour chaque question, indiquer le numéro de la question et recopier sur la copie la lettre correspondant à la réponse choisie.

Aucune justification n'est demandée mais il peut être nécessaire d'effectuer des recherches au brouillon pour aider à déterminer la réponse.

Chaque réponse correcte rapporte 1 point. Une réponse incorrecte ou une question sans réponse n'apporte ni ne retire de point.

\medskip

\textbf{Question 1}

\medskip

Soit ABC un triangle tel que AB $= 6$, AC $= 3$ et $\widehat{\text{BAC}} = \dfrac{\pi}{3}$.

\textbf{a.~~} $\vect{\text{AB}} \cdot  \vect{\text{AC}} = 9$

\textbf{b.~~} $\vect{\text{AB}} \cdot  \vect{\text{AC}} = 18$

\textbf{c.~~} $\vect{\text{AB}} \cdot  \vect{\text{AC}}= 9\sqrt{3}$

\textbf{d.~~} les données sont insuffisantes pour calculer $\vect{\text{AB}} \cdot \vect{\text{AC}}$
.
\medskip

\textbf{Question 2}

\medskip

Soit $f$ une fonction telle que, pour tout nombre réel $h$ non nul,

\[\dfrac{f(1 + h) - f(1)}{h} =  h^2 + 3h - 1.\]

Alors $f'(1)$ est égal à:

\textbf{a.~~} $h^2 + 3h - 1$

\textbf{b.~~} $- 1$

\textbf{c.~~} 3

\textbf{d.~~} les données sont insuffisantes pour déterminer $f'(1)$.

\medskip

\textbf{Question 3}

\medskip

Soit $f$ la fonction définie sur $\R$ par $f(x) = (x + 2)\text{e}^x$.

Alors, la fonction $f'$ dérivée de $f$  est donnée sur $\R$ par :

\textbf{a.~~} $f'(x) = \text{e}^x$

\textbf{b.~~} $f'(x)=(x+3)\text{e}^x$

\textbf{c.~~} $f'(x)=(-x-1)\text{e}^x$

\textbf{d.~~} $f'(x) = \dfrac{(- x - 1)\text{e}^x}{\text{e}^{2x}}$ 

\medskip

\textbf{Question 4}

\medskip

Soit $f$ une fonction telle que $f(2) = 5$ et $f'(2) = -1$. 

Dans un repère, la tangente à la courbe représentative de $f$ au point d'abscisse $2$ a pour équation :

\textbf{a.~~} $y = - x - 3$

\textbf{b.~~} $y =-x + 3$

\textbf{c.~~} $y= - x + 7$

\textbf{d.~~} $y = 5x - 11$

\medskip

\textbf{Question 5}

\medskip

Soit $f$ une fonction définie et dérivable sur $\R$ dont la courbe représentative $\mathcal{C}_f$ dans un
repère est la courbe ci-dessous.

\begin{center}
\psset{xunit=1.5cm,yunit=0.75cm,comma=true}
\begin{pspicture*}(-1.2,-3)(2.1,5)
\psgrid[gridlabels=0pt,subgriddiv=2,griddots=8]
\psaxes[linewidth=1.25pt,Dx=1,Dy=1]{->}(0,0)(-1.2,-2.95)(2.1,5)
\psplot[plotpoints=2000,linewidth=1.25pt,linecolor=blue]{-1.2}{2.1}{x dup mul 5 mul x sub 3 div}
\psplot[plotpoints=2000,linewidth=1.25pt]{-1.2}{2.1}{x 3 mul  5 3 div sub}
\psdots(1,1.333)(0,-1.6667)
\uput[dr](1,1.333){A}\uput[dr](0,-1.6667){B}\uput[ul](1.7,4.25){\blue $\mathcal{C}_f$}
\end{pspicture*}
\end{center}

La tangente à la courbe $\mathcal{C}_f$ au point A$\left(1~;~\dfrac{4}{3}\right)$ passe par le point B$\left(0~;~- \dfrac{5}{3}\right)$.

Alors :

\textbf{a.~~} $f'(1)= \dfrac{1}{3}$

\textbf{b.~~} $f'(1) = \dfrac{4}{3}$

\textbf{c.~~} $f'(1) = - \dfrac{5}{3}$

\textbf{d.~~} $f'(1) = 3$

\bigskip

