
\medskip

\emph{Les deux parties suivantes sont indépendantes.}

\medskip

\textbf{Partie A.} 

\medskip

On considère la suite $\left(v_n\right)$ définie par $v_0=1$ et $v_{n+1} = \frac{2}{3}v_n$ pour tout entier naturel $n$.

\medskip

\begin{enumerate}
\item Quelle est la nature de la suite $\left(v_n\right)$ ? En préciser les éléments caractéristiques.
\item Donner, pour tout entier naturel $n$, une expression de $v_n$ en fonction de $n$.
\item Calculer la somme $\mathscr{S}$ des dix premiers termes de la suite $\left(v_n\right)$.
\end{enumerate}
\medskip

\textbf{Partie B.}

\medskip

On modélise une suite $\left(w_n\right)$ à l'aide de la fonction suivante écrite en langage Python :

\begin{python}
def terme(n) :
	 w=4
     for i in range(n) :
         w=2*w- 3
     return w
   \end{python}

\begin{enumerate}[resume]
\item Que renvoie l'exécution de \verb|terme(5)| ?
\item En s'inspirant de la fonction \verb|terme(n)|, proposer une fonction \verb|somme_termes(n)|, écrite en langage Python, qui renvoie la somme des $n$ premiers termes de la suite 
$\left(w_n\right)$.
\end{enumerate}

\vspace{0,5cm}

