
\subsection*{Question 1}

\(f(x) = 3(x + 2)^2 + 5\). On a :
\[
(x+2)^2 \geqslant 0 \Rightarrow 3(x+2)^2 \geqslant 0 \Rightarrow 3(x+2)^2 + 5 \geqslant 5 > 0.
\]
Ce trinôme ne s'annule pas. Le discriminant est strictement négatif.

\subsection*{Question 2}

Un vecteur directeur de la droite d'équation \(2x + 3y + 5 = 0\) est : \(\vec{u} \begin{pmatrix} -3 \\ 2 \end{pmatrix}\).

\subsection*{Question 3}

\[
\overrightarrow{AB} \begin{pmatrix} 1 \\ 3 \end{pmatrix} \quad \text{et} \quad \overrightarrow{AC} \begin{pmatrix} -2 \\ 2 \end{pmatrix}.
\]
D'où :
\[
\overrightarrow{AB} \cdot \overrightarrow{AC} = 1 \times (-2) + 3 \times 2 = -2 + 6 = 4.
\]

\subsection*{Question 4}

\(g\) est un produit de fonctions dérivables sur \(\mathbb{R}\) et sur cet intervalle :
\begin{align*}
g'(x) &= 2\e^x + (2x + 1)\e^x \\
&= \e^x(2 + 2x + 1) \\
&= \e^x(2x + 3) \\
&= (2x + 3)\e^x.
\end{align*}

\subsection*{Question 5}

Pour tout réel \(x\), \(\sin(x + \pi) = -\sin(x)\).

