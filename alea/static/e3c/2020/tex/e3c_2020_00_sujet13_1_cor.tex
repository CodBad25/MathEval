	\section*{Exercice 1 (5 points)}
	
	\subsection*{Question 1}
La parabole a sa concavité vers le bas donc \(a < 0\) et l’équation \(g(x) = 0\) a deux solutions distinctes donc \(\Delta > 0\). \\
La réponse correcte est \textbf{c.}
	
	\subsection*{Question 2}
	
La dérivée est négative sauf sur l’intervalle \([1, 5]\) ; la fonction est donc décroissante puis croissante sur \([1, 5]\) et ensuite décroissante.\\
 La réponse correcte est \textbf{c.}
	
	\subsection*{Question 3}
	
La tangente au point d’abscisse 3 a pour équation :
\[ y - f(3) = f'(3)(x - 3). \]
Avec \(f(3) = 7\) et \(f'(3) = g(3) = 4\), une équation de la tangente est :
\[ y - 7 = 4(x - 3) \quad \text{soit} \quad y = 4x - 5. \]
La réponse correcte est \textbf{d.}
	
	\subsection*{Question 4}
	
Le vecteur \(\overrightarrow{AB} \left( \begin{array}{c} -2 \\ 3 \end{array} \right)\) est normal à tout vecteur directeur de la droite perpendiculaire. Une équation de cette droite est donc :
\[-2x + 3y + c = 0 \]
et comme le couple \((1, -3)\) vérifie cette équation, on a \(-2 - 9 + c = 0\) soit \(c = 11\).\\
 On a finalement \(-2x + 3y + 11 = 0\).\\
  La réponse correcte est \textbf{a.}
	
\subsection*{Question 5}
Par définition du produit scalaire :
	\[ \overrightarrow{BA} \cdot \overrightarrow{BC} = BA \times BC \times \cos(\widehat{BA, BC}), \]
	ce qui donne avec \(\overrightarrow{BA} \left( \begin{array}{c} 2 \\ -3 \end{array} \right)\) et \(\overrightarrow{BC} \left( \begin{array}{c} -2 \\ -5 \end{array} \right)\),
	\[ BA^2 = 4 + 9 = 13 \quad \text{et} \quad BC^2 = 4 + 25 = 29, \]
donc \(BA = \sqrt{13}\) et \(BC = \sqrt{29}\).
	
\[ \cos(\widehat{BA, BC}) = \dfrac{\overrightarrow{BA} \cdot \overrightarrow{BC}}{BA \times BC} = \dfrac{-4 + 15}{\sqrt{13} \times \sqrt{29}} = \dfrac{11}{\sqrt{13 \times 29}}. \]
La calculatrice donne \(\widehat{BA, BC} \approx 55,49^\circ\), soit au degré près $55^\circ$.\\
 La réponse correcte est \textbf{c.}
	
