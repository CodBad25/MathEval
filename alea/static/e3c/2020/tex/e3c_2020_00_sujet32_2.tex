
\medskip

Une entreprise vend des téléviseurs. Une étude a montré que ces téléviseurs peuvent rencontrer deux types de défauts : un défaut sur la dalle, un défaut sur le condensateur.
L'étude indique que :

\begin{itemize}
\item 3\,\% des téléviseurs présentent un défaut sur la dalle et que parmi ceux-ci, 2\% ont également un défaut sur le condensateur.
\item 5\,\% des téléviseurs ont un défaut sur le condensateur.
\end{itemize}
On choisit un téléviseur au hasard et on considère les évènements suivants :
\begin{itemize}
\item $D$ : \og le téléviseur a un défaut sur la dalle \fg{} ;
\item $C$ : \og le téléviseur a un défaut sur le condensateur \fg.
\end{itemize}

Pour tout évènement $E$, on note $p(E)$ sa probabilité et $\overline{E}$ l'évènement contraire de $E$. Pour tout évènement $F$ de probabilité non nulle, on note $p_F(E)$ la probabilité de $E$ sachant que $F$ est réalisé.

Les résultats seront approchés si nécessaire à $10^{-4}$ près.

\medskip

\begin{enumerate}
\item  Justifier que $p(D)= 0,03$ puis donner $p_D(C)$.
\item Recopier l'arbre ci-dessous et compléter uniquement les pointillés par les probabilités associées :

\begin{center}

\psset{nodesepA=0pt,nodesepB=3pt,treesep=0.75,labelsep=0.1pt,levelsep=2.75cm}
\pstree[treemode=R]{\TR{}}
{\pstree{\TR{$D$~~}\taput{$\dots $}}
	{
	\TR{$C$}\taput{$\dots $}
	\TR{$\overline{C}$}\tbput{$\dots $}
	}
\pstree{\TR{$\overline{D}$~~}\tbput{$\dots $}}
	{\TR{$C$}\taput{$ $}
	\TR{$\overline{C}$}\tbput{$ $}
	}
}
\end{center}

\item Calculer la probabilité $p(D\cap C)$ de l'évènement $D \cap C$.
\item Le téléviseur choisi a un défaut sur le condensateur. Quelle est alors la probabilité qu'il ait un défaut sur la dalle ?
\item Montrer que la probabilité que le téléviseur choisi ait un défaut sur le condensateur et n'ait pas de défaut sur la dalle est égale à 0,0494.
\end{enumerate}

\vspace{0,5cm}

