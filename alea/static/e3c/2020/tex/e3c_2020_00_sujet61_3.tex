
\medskip

Un restaurant propose à sa carte deux types de dessert : un assortiment de macarons et une part de tarte tatin. Des études statistiques montrent que:

\setlength\parindent{1cm}
\begin{itemize}
\item l'assortiment de macarons est choisi par 50\,\% des clients;
\item la part de tarte tatin, est choisie par 30\,\% des clients;
\item 20\,\% des clients ne prennent pas de dessert;
\item aucun client ne prend plusieurs desserts.
\end{itemize}
\setlength\parindent{0cm}

Le restaurateur a remarqué que:

\setlength\parindent{1cm}
\begin{itemize}
\item parmi les clients ayant pris un assortiment de macarons, 80\,\% prennent un café;
\item parmi les clients ayant pris une part de tarte tatin, 60\,\% prennent un café;
\item parmi les clients n'ayant pas pris de dessert, 90\,\% prennent un café.
\end{itemize}
\setlength\parindent{0cm}

On interroge au hasard un client de ce restaurant. 

On note les évènements suivants:

\setlength\parindent{1cm}
\begin{itemize}
\item $M$ : \og Le client prend un assortiment de macarons \fg{};
\item $T$ : \og Le client prend une part de tarte tatin \fg{} ;
\item $P$ :\og Le client ne prend pas de dessert\fg{} ;
\item $C$ : \og Le client prend un café\fg{} et $\overline{C}$ l'évènement contraire de $C$.
\end{itemize}
\setlength\parindent{0cm}

\medskip

\begin{enumerate}
\item En utilisant les données de l'énoncé, préciser la valeur de $P(T)$ probabilité de $T$ et celle de $P_T(C)$ probabilité de l'évènement $C$ sachant que $T$ est réalisé.
\item  Recopier et compléter l'arbre ci-dessous :

\begin{center}
\pstree[treemode=R,nodesepB=3pt,levelsep=2.5cm]{\TR{}}
{\pstree{\TR{$M$~~}\taput{0,5}}
	{\TR{$C$} \taput{}
	\TR{$\overline{C}$} \tbput{}
	}
\pstree{\TR{$T$~~}\taput{}}
	{\TR{$C$} \taput{}
	\TR{$\overline{C}$} \tbput{}
	}
\pstree{\TR{$P$~~}\tbput{}}
	{\TR{$C$} \taput{}
	\TR{$\overline{C}$} \tbput{}
	}
}
\end{center}

\item  
	\begin{enumerate}
		\item Exprimer par une phrase ce que représente l'évènement $M \cap C$ puis calculer $P(M \cap C)$.
		\item Montrer que $P(C) = 0,76$.
	\end{enumerate}
\item  Quelle est la probabilité que le client prenne un assortiment de macarons sachant qu'il prend un café ? (On donnera le résultat arrondi au centième).
\end{enumerate}

\bigskip

