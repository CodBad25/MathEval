
\medskip

Lors des journées classées \og rouges \fg{} selon Bison Futé, l'autoroute qui relie Paris à Limoges
en passant par Orléans est surchargée.

Lors de ces journées classées \og rouges \fg{}, on a pu observer le comportement des
automobilistes faisant le trajet de Paris à Limoges en passant par Orléans.
\begin{itemize}
\item Pour le trajet de Paris à Orléans, 30\,\% d'entre eux prennent la route nationale, les
autres prennent l'autoroute.

\item Pour le trajet d'Orléans à Limoges :

\begin{itemize}
\item  parmi les automobilistes ayant pris la route nationale entre Paris et Orléans,

40\,\% prennent la route départementale, les autres prennent l'autoroute ;

 \item parmi les automobilistes n'ayant pas pris la route nationale entre Paris et Orléans, 
 
45\,\% prennent la route départementale, les autres prennent l'autoroute.
\end{itemize}
\end{itemize}
On choisit un automobiliste au hasard parmi ceux effectuant, en journée classée rouge, le
trajet Paris – Limoges en passant par Orléans.

On note $N$ l'événement \og l'automobiliste prend la route nationale entre Paris et Orléans \fg{}
et $D$ l'événement \og l'automobiliste prend la route départementale entre Orléans et
Limoges \fg.

Si $A$ est un évènement, on note $\overline{A}$ l'évènement contraire de A.

\medskip

\begin{enumerate}
\item  Recopier sur la copie et compléter l'arbre ci-dessous.
\begin{center}
\psset{nodesepA=0pt,nodesepB=3pt,treesep=0.75,labelsep=0.1pt}
\pstree[treemode=R]{\TR{}}
{\pstree{\TR{$N$~~}\taput{$\dots$}}
	{
	\TR{$D$}\taput{$\dots$}
	\TR{$\overline{D}$}\tbput{$\dots$}
	}
\pstree{\TR{$\overline{N}$~~}\tbput{$\dots$}}
	{\TR{$D$}\taput{$\dots$}
	\TR{$\overline{D}$}\tbput{$\dots$}
	}
}
\end{center}



\item Calculer $p\left(\overline{N}\cap \overline{D}\right)$  et interpréter le résultat.
\item Montrer que la probabilité que l'automobiliste ne choisisse pas la Route Départementale
entre Orléans et Limoges est 0,565.
\end{enumerate}
Lors de ces journées classées \og rouges \fg, on donne les temps de parcours suivants :

Paris – Orléans, par autoroute : 3 heures ;

Paris – Orléans, par nationale : 2 heures ;

Orléans – Limoges, par autoroute : 4 heures ;

Orléans – Limoges, par départementale : 3 heures et demie.
\begin{enumerate}[resume]
\item Recopier et compléter le tableau ci-dessous, qui donne pour chaque trajet, le temps en
heure et la probabilité :

\begin{center}
\begin{tabularx}{\linewidth}{|m{3cm}|*{4}{>{\centering \arraybackslash}X|}} \hline
Évènement 		&$N\cap D$ &$N\cap \overline{D}$ &$\overline{N}\cap D$& $\overline{N}\cap\overline{D}$\\\hline
Temps en heure 	& 5,5 &&&\\\hline
Probabilité 	&0,12&&&\\\hline
\end{tabularx}
\end{center}

\item Calculer l'espérance de la variable aléatoire qui donne la durée du trajet en heure et en
donner une interprétation.
\end{enumerate}% …
