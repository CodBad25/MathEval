
\medskip

On considère les suites $\left(u_n\right)_{n\geqslant 0}$ et $\left(v_n\right)_{n\geqslant 0}$ définies par $u_0 = 7$ et, pour tout entier naturel $n$,

\[u_{n+1} = 0,5u_n + 3\quad \text{et}\quad v_n = u_n - 6.\]

\medskip

\begin{enumerate}
\item Montrer que la suite $\left(v_n\right)_{n\geqslant 0}$ est une suite géométrique de raison $0,5$ et de premier terme~$1$.
\item Pour tout entier naturel $n$, exprimer $v_n$ en fonction de $n$.
\item En déduire, pour tout entier naturel $n$, une expression de $u_n$n en fonction de $n$.
\item On note $S = v_0 + v_1 + \ldots + v_{100}$ la somme des $101$ premiers termes de la suite $\left(v_n\right)_{n\geqslant 0}$.
	\begin{enumerate}
		\item Déterminer la valeur de $S$.
		\item En déduire la valeur de la somme des $101$ premiers termes de la suite $\left(u_n\right)_{n\geqslant 0}$.
	\end{enumerate}
\end{enumerate}

\bigskip

