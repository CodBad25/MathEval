
\medskip

Un journal hebdomadaire est sur le point d'être créé.

Une étude de marché aboutit à deux estimations différentes concernant le nombre de journaux vendus :
\begin{itemize}
\item  1\up{re} estimation : \np{1000} journaux vendus lors du lancement, puis une progression des ventes de 3\,\% chaque semaine.
\item 2\up{e} estimation : \np{1000} journaux vendus lors du lancement, puis une progression régulière de 40 journaux supplémentaires vendus chaque semaine.
 \end{itemize}
On considère les suites $\left(u_n\right)$ et $\left(v_n\right)$ telles que, pour tout entier naturel $n\geqslant 1$, $u_n$ représente le nombre de journaux vendus la $n$-ième semaine selon la première estimation et $v_n$ représente le nombre de journaux vendus la $n$-ième semaine selon la deuxième estimation. 

Ainsi, $u_1 = v_1 = \np{1000}$.

\medskip

\begin{enumerate}
\item On considère la feuille de calcul ci-dessous :

\begin{center}
\begin{tabularx}{0.5\linewidth}{|>{\columncolor{blue}}m{0.35cm}|*{3}{>{\centering \arraybackslash}X|}}
\hline
\rowcolor{blue}&A&B&\cellcolor{yellow}C\\\hline
1&$n$&$u_n$&$v_n$\\\hline
2&1&\np{1000}&\np{1000}\\\hline
3&2&\np{1030}&\np{1040}\\\hline
4&3&\np{1060.9}&\np{1080}\\\hline
5&4&\np{1092.727}&\np{1120}\\\hline
\end{tabularx}
\end{center}

Quelle formule, saisie en B3 et recopiée vers le bas, permet d'obtenir les termes de la suite $\left(u_n\right)$ ?

\medskip

\item 
	\begin{enumerate}
		\item  Donner la nature de la suite $(u_n)$ puis celle de la suite $(v_n)$. Justifier.
		\item Montrer que pour tout entier naturel $n\geqslant1$, $v_n=960+40n$.
		\item Écrire, pour tout entier naturel $n\geqslant1$, l'expression de $u_n$ en fonction de $n$.
	\end{enumerate}
\item On définit, pour tout entier $n\geqslant 1$, la suite $\left(w_n\right)$ par $w_n=v_n-u_n$.

On donne ci-dessous un extrait de son tableau de valeurs :

\begin{center}
\begin{tabular}[]{|*{3}{c|}m{1.2cm}|*{4}{c|}}\hline
$n$		&1&2	&	&19&20&21		&22\\\hline
$w_n$	&0&10	&	&18&6&$-6$	&$-20$\\\hline
\end{tabular}
\end{center}

À partir de quelle semaine le nombre de journaux vendus d'après la première estimation devient-il supérieur au nombre de journaux vendus d'après la deuxième estimation ?
\end{enumerate}

