
\medskip

À la naissance de Lisa, sa grand-mère a placé la somme de \np{5000}~euros sur un compte et cet argent est resté bloqué pendant $18$ ans.

Lisa retrouve dans les papiers de sa grand-mère l'offre de la banque :

\begin{center}
\begin{tabularx}{\linewidth}{|X|}\hline
\textbf{Offre}\\ \hline
Intérêts composés au taux annuel constant de 3\,\%.\\
À la fin de chaque année le capital produit 3\,\% d'intérêts qui sont intégrés au capital.\\ \hline
\end{tabularx}
\end{center}

On considère que l'évolution du capital acquis, en euro, peut être modélisée par une suite 
$\left(u_n\right)$ dans laquelle, pour tout entier naturel $n$,\, $u_n$ est le capital acquis, en euro, $n$ années après la naissance de Lisa.

On a ainsi $u_0 = \np{5000}$.

\medskip

\begin{enumerate}
\item Montrer que $u_1 = \np{5150}$ et $u_2 = \np{5304,5}$. 
\item 
	\begin{enumerate}
		\item Pour tout entier naturel $n$, exprimer $u_{n+1}$ en fonction de $u_n$. 
		
En déduire la nature de la suite $\left(u_n\right)$  en précisant sa raison et son premier terme.
		\item Pour tout entier naturel $n$, exprimer $u_n$ en fonction de $n$.
	\end{enumerate}
\item  Calculer le capital acquis par Lisa à l'âge de $18$ ans. Arrondir au centième.
\item  Si Lisa n'utilise pas le capital dès ses $18$ ans, quel âge aura-t-elle quand celui-ci dépassera \np{10000} euros ?
\end{enumerate}

\bigskip

