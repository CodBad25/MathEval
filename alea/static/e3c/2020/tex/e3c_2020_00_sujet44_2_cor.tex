
\( p(5) = -x^3 + 3x^2 + 9x + 5\).

\subsection*{Partie A}

\paragraph{1.} On a :
\begin{align*}
p(5) &= -5^3 + 3 \times 5^2 + 9 \times 5 + 5 \\
&= -125 + 75 + 45 + 5 \\
&= 0.
\end{align*}

\paragraph{2.} 
\begin{align*}
(5 - x)(x^2 + 2x + 1) &= 5x^2 + 10x + 5 - x^3 - 2x^2 - x \\
&= -x^3 + 3x^2 + 9x + 5 \\
&= p(x).
\end{align*}

\paragraph{3.} Le signe de \( p(x) \) dépend donc de celui de \( 5 - x \) et de celui du trinôme :
\[
x^2 + 2x + 1 = (x + 1)^2.
\]

On sait que, quel que soit \(a \in \mathbb{R}, a^2 \geqslant 0\). Le signe de \( p(x) \) est donc celui de \( 5 - x \) :
\begin{itemize}
    \item \( p(x) > 0 \) sur \( ]-\infty \,;\, 5[ \) ;
    \item \( p(x) < 0 \) sur \( ]5 \,;\, +\infty[ \) ;
    \item \( p(-1) = p(5) = 0 \).
\end{itemize}

\subsection*{Partie B}

\paragraph{1.} Sur \( \mathbb{R} \), on a :
\begin{align*}
p'(x) &= -3x^2 + 6x + 9 \\
&= 3(-x^2 + 2x + 3).
\end{align*}

\paragraph{2.} Pour le trinôme \( -x^2 + 2x + 3 \), on a :
\[
\Delta = 4 + 4 \times 3 = 16 = 4^2 > 0,
\]
il a donc deux racines :
\[
x_1 = \dfrac{-2 + 4}{-2} = -1 \quad \text{et} \quad x_2 = \dfrac{-2 - 4}{-2} = 3.
\]

On sait que ce trinôme est négatif sauf sur l'intervalle \( ]-1 \,;\, 3[ \).

La fonction est décroissante sauf sur l'intervalle \( ]-1 \,;\, 3[ \) où elle est croissante de \( p(-1) = 0 \) à \( p(3) = 32 \).

Le maximum de la fonction \( p \) sur l'intervalle \( [0 \,;\, 5] \) est donc \( p(3) = 32 \).

