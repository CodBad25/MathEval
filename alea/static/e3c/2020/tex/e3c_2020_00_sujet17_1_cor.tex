	
	\subsection*{Question 1}
	
	\begin{align*}
E(X) &= -5 \times 0,71 + 0 \times 0,03 + 10 \times 0,01 + 20 \times 0,05 + 50 \times 0,2\\
& = -3,55 + 0,1 + 1 + 10\\
& = 7,55.
	\end{align*}
	
	\subsection*{Question 2}
	
	Si \(C\) est le cercle, on a :
	\[
	M(x ; y) \in C \Leftrightarrow AM^2 = 9^2 \Leftrightarrow (x - (-2))^2 + (y - 4)^2 = 81 \Leftrightarrow (x + 2)^2 + (y - 4)^2 = 81.
	\]
	
	\subsection*{Question 3}
	
	La parabole est tournée vers le bas, donc \(a < 0\). \(f(0) = c > 0\).\\
	 Le trinôme a deux racines, donc \(\Delta > 0\).
	
	\subsection*{Question 4}
	
	Réponse : algorithme d.
	
	\subsection*{Question 5}
	
	La suite \((u_n)\) définie par \(u_0 = -2\) et \(u_{n+1} = 2u_n - 5\) est :
	\[
	u_0 = -2, \quad u_1 = -9, \quad u_2 = -23.
	\]
	
$u_1 - u_0 = -7 \quad \text{et} \quad u_2 - u_1 = -14$:\\
la différence de deux termes consécutifs n'est pas constante : ce n'est pas une suite arithmétique.
	
	
$\dfrac{u_1}{u_0} = \dfrac{-9}{-2} = 4,5 $ et  $\dfrac{u_2}{u_1} = \dfrac{-23}{-9} \approx 2,56$ :\\  le quotient de deux termes consécutifs n'est pas constant : ce n'est pas une suite géométrique.	
