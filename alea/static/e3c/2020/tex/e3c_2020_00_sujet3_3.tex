
\medskip

Un propriétaire souhaite construire un enclos rectangulaire sur son terrain.

Celui-ci est représenté ci-dessous dans un repère orthonormé, d'unité le mètre. Il est délimité par l'axe des abscisses, l'axe des ordonnées, la droite d'équation $x = 5$ et la courbe $\mathcal{C}_f$ représentative de la fonction $f$ définie sur [0~;~5] par 

\[f(x) = 4\text{e}^{-0,5x}.\]

\begin{center}
\psset{unit=1.5cm}
\begin{pspicture}(-1,-1)(5.5,4)
\psaxes[linewidth=1.25pt,labelFontSize=\scriptstyle]{->}(0,0)(0,0)(5.5,4)
\psplot[plotpoints=2000,linewidth=1.25pt,linecolor=red]{0}{5}{4  2.71828 0.5 x mul exp div}
\uput[u](0.5,3.5){\red $\mathcal{C}_f$}
\psframe[fillstyle=hlines](0,0)(2.8,0.986)
\uput[ul](0,0){O}\uput[ur](2.8,0.986){B}\uput[d](2.8,0){A}\uput[ur](0,0.986){C}
\uput[ur](5,0){D}
\end{pspicture}
\end{center}

L'enclos est représenté par le rectangle OABC où O est l'origine du repère et B un point de $\mathcal{C}_f$, A et C étant respectivement sur l'axe des abscisses et l'axe des ordonnées. On note $x$ l'abscisse du point A et D le point de coordonnées (5~;~0). 

Le but de l'exercice est de déterminer la position du point A sur le segment [OD] permettant d'obtenir un enclos de superficie maximale.
\begin{itemize}
\item 
\end{itemize}
\medskip

\begin{enumerate}
\item Justifier que la superficie de l'enclos, en m$^2$, est donnée en fonction de $x$ par $g(x) = 4x\text{e}^{-0,5x}$ pour $x$ dans l'intervalle [0~;~5].
\item La fonction $g$ est dérivable sur [0~;~5]. Montrer que, pour tout réel $x$ de l'intervalle [0~;~5], on a $g'(x) = (4 - 2x)\text{e}^{-0,5x}$.
\item En déduire le tableau de variations de la fonction $g$ sur [0~;~5].
\item Où doit-on placer le point A sur [OD] pour obtenir une superficie d'enclos maximale ?

Donner la superficie maximale possible en arrondissant le résultat au dm$^2$.


\end{enumerate}


