
\subsection*{1.}

Augmenter de 15 \% revient à multiplier par \(1 + \dfrac{15}{100} = 1 + 0{,}15 = 1{,}15\).

Donc on passe de la population à l'instant \(n\) à celle à l'instant \(n+1\) en la multipliant par \(1{,}15^n\), soit \(u_{n+1} = u_n \times 1{,}15\).

En généraliant on à donc, quel que soit le naturel \(n\), \(u_n = 10000 \times 1{,}15^n\).

\subsection*{2.}

La suite \((u_n)\) est donc une suite géométrique de premier terme \(u_0 = 10000\) et de raison \(q = 1{,}15\).

\subsection*{3.}

Au bout de 10 heures, il y aura :
\[
u_{10} = 10000 \times 1{,}15^{10} \approx 40455{,}6,
\]
soit environ 40 456 bactéries.

\subsection*{4.}

Ces résultats montrent que la modélisation choisie ne veut rien dire : en un peu plus d'un an, les bactéries auraient envahi le laboratoire...

\subsection*{5.}

La diminution en pourcentage est :
\[
\dfrac{200000 - 4000}{200000} \times 100 = \dfrac{200 - 4}{200} \times 100 = \dfrac{196}{200} \times 100 = \dfrac{196}{2} = 98 \%.
\]

