
\medskip

Un téléphone coûte $600$ euros lors de son lancement. Tous les ans, le fabricant sort une
nouvelle version de ce téléphone. Le prix de ce téléphone augmente de 3\,\% chaque année.

On note $u_n$ le prix du téléphone en euros $n$ années après son lancement. On a donc
$u_0 = 600$.

\medskip

\begin{enumerate}
\item Calculer $u_1$ et $u_2$. Interpréter les résultats.
\item Exprimer $u_{n+1}$ en fonction de $u_n$, pour tout entier naturel $n$ et en déduire la nature de
la suite $\left(u_n\right)$. Préciser sa raison et son premier terme.
\item Exprimer, pour tout entier $n$, $u_n$ en fonction de $n$.
\item Recopier et compléter sur la copie la fonction Python ci-dessous pour qu'elle détermine
le nombre minimum d'années nécessaires afin que le prix du téléphone dépasse \np{1000}
euros.

\begin{center}
\begin{tabular}[]{|m{12.4em}|}
\hline
def nombreAnnees():\\
n = 0\\
u = 600\\
while \dots :\\
\hspace{1em}n = \dots\\
\hspace{1em}u =\dots\\
return n\\
\hline
\end{tabular}
\end{center}
\item Quelle est la valeur de $n$ renvoyée par cette fonction Python ?
\end{enumerate}
