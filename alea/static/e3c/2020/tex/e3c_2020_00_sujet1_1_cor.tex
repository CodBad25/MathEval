	
	Ce QCM comprend cinq questions. Pour chacune des questions, une seule des quatre réponses proposées est correcte. Les questions sont indépendantes. Pour chaque question, indiquer le numéro de la question et recopier sur la copie la lettre correspondante à la réponse choisie. Aucune justification n’est demandée mais il peut être nécessaire d’effectuer des recherches au brouillon pour aider à déterminer votre réponse. Chaque réponse correcte rapporte 1 point. Une réponse incorrecte ou une question sans réponse n’apporte ni ne retire de point.
	
\textbf{	Question 1}\\
	ABC est un triangle tel que $AB = 5$, $AC = 6$ et $\widehat{ BAC} = \dfrac{\pi}{4}$.\\
	 Alors $\overrightarrow{AB} \cdot \overrightarrow{AC}$ est égal à :
	
	\begin{itemize}
		\item[a.] $15\sqrt{2}$
		\item[b.] $15\sqrt{3}$
		\item[c.] $15/2$
		\item[d.] $15$
	\end{itemize}
	
	On a $\overrightarrow{AB} \cdot \overrightarrow{AC} = AB \times AC \times \cos(\widehat{ BAC}) = 5 \times 6 \times \dfrac{\sqrt{2}}{2} = 15\sqrt{2}$.
	
\textbf{	Question 2}\\
	ABCD est un carré de centre $O$ tel que $AB = 1$. Alors $\overrightarrow{AB} \cdot \overrightarrow{OB}$ est égal à :
	
	\begin{itemize}
		\item[a.] $1$
		\item[b.] $0$
		\item[c.] $0,5$
		\item[d.] $-1$
	\end{itemize}
	
	En projetant sur $(OB)$, on a $\overrightarrow{AB} \cdot \overrightarrow{OB} = \overrightarrow{OB} \cdot \overrightarrow{OB}$. \\
	La diagonale du carré a pour longueur $\sqrt{2}$, donc $\overrightarrow{AB} \cdot \overrightarrow{OB} = \dfrac{\sqrt{2}}{2} \times \dfrac{\sqrt{2}}{2} = \dfrac{2}{4} = \dfrac{1}{2}$.
	
	\textbf{{Question 3}}\\
	$\overrightarrow{u}$ et $\overrightarrow{v}$ sont deux vecteurs orthogonaux tels que $\|\overrightarrow{u}\| = 2$ et $\|\overrightarrow{v}\| = 1$. $(\overrightarrow{u} + \overrightarrow{v}) \cdot (2\overrightarrow{u} - \overrightarrow{v})$ est égal à :
	
	\begin{itemize}
		\item[a.] $6$
		\item[b.] $9$
		\item[c.] $13$
		\item[d.] $7$
	\end{itemize}
	
	$(\overrightarrow{u} + \overrightarrow{v}) \cdot (2\overrightarrow{u} - \overrightarrow{v}) = 2 \overrightarrow{u} \cdot \overrightarrow{u} - \overrightarrow{u} \cdot \overrightarrow{v} + 2 \overrightarrow{u} \cdot \overrightarrow{v} - \overrightarrow{v} \cdot \overrightarrow{v} = 2 \times 2^2 - 0 + 0 - 1^2 = 7$.
	
\textbf{	Question 4}\\
	On se place dans un repère orthonormé du plan. Sur la figure ci-dessous, on a tracé la courbe représentative notée $C$ d’une fonction $f$ définie sur $\mathbb{R}$. La droite $D$ est tangente à la courbe $C$ au point $A(5, 0)$.
	

	\begin{center}
		\psset{unit=0.8cm}
		\begin{pspicture}(-6,-2)(8,5.25)
			\psgrid[gridlabels=0pt,gridwidth=0.4pt,subgriddiv=1]
			\psaxes[linewidth=1.25pt]{->}(0,0)(-6,-2)(8,5)
			%\psplot[plotpoints=2000,linewidth=1.25pt,linecolor=red]{-5}{8}{x 4 add x 6 sub mul x 4 exp 1 add div 0.25 sub}
			%\psplot[plotpoints=2000,linewidth=1.25pt,linecolor=orange]{-4.9}{2}{4 x 3 exp 3 div x dup mul 2.5 mul add 4 x mul add 3 mul 26 div sub}
			%\psplot[plotpoints=2000,linewidth=1.25pt,linecolor=red]{-5}{8}{x 4 add x 5 sub mul x dup mul  9 x mul add 0.25 sub mul 2.7182 0.0001 x mul exp div}%   5 div neg 
			\psplot[plotpoints=2000,linewidth=1.25pt]{-5}{8}{5 3 div x 3  div sub}
			%\pscurve[linewidth=1.25pt,linecolor=blue](-4.85,5)(-4.82,4)(-4.7,3)(-4.5,1)(-4,0)(-3,1.3)(-2,3.75)(-1,4.4)(0,4)(1,3.1)(2,2)(3,1.1)(4,0.45)(5,0)(6,-0.25)(7,-0.35)(8,-0.4)
			\psbezier[linewidth=1.25pt,linecolor=red](-4.85,5)(-4.5,1)(-4.6,0.4)(-4,0)
			\psbezier[linewidth=1.25pt,linecolor=red](-4,0)(-2.5,0.1)(-3,4)(-1,4.35)
			\psbezier[linewidth=1.25pt,linecolor=red](-1,4.35)(2,3.8)(2.25,0.6)(5,0)
			\psbezier[linewidth=1.25pt,linecolor=red](5,0)(7,-0.6)(8,-1)(8,-1)
		\end{pspicture}
	\end{center}

	
	On note $f'$ la dérivée de la fonction $f$, alors $f'(5)$ est égal à :
	
	\begin{itemize}
		\item[a.] $3$
		\item[b.] $-3$
		\item[c.] $\dfrac{1}{3}$
		\item[d.] $-\dfrac{1}{3}$
	\end{itemize}
	
	$f'(5)$ est égal à la pente de la tangente au point d’abscisse 5, qui est égale à $-\dfrac{3}{9} = -\dfrac{1}{3}$.
	
	\subsection*{Question 5}
	Pour tout réel $x$ de l’intervalle $]-\infty, 0]$, on a :
	
	\begin{itemize}
		\item[a.] $f'(x) \leq 0$
		\item[b.] $f'(x) > 0$
		\item[c.] $f(x) > 0$
		\item[d.] $f(x) \leq 0$
	\end{itemize}
	
	Sur l’intervalle $]-\infty, 0]$, on voit que la fonction est positive, donc $f(x) > 0$.
	\medskip
	
