
\medskip

Au sein d'un lycée, parmi les élèves de première ayant choisi la spécialité mathématique, il y a $110$ filles dont $5$ ne poursuivent pas la spécialité en terminale et $90$ garçons dont $8$ ne poursuivent pas la spécialité.

On interroge au hasard un élève et on définit les évènements suivants :

\setlength\parindent{9mm}
\begin{itemize}[label=\textbullet]
\item $F$ l'évènement: \og L'élève interrogé est une fille \fg,
\item $G$ l'évènement: \og L'élève interrogé est un garçon \fg,
\item $S$ l'évènement: \og L'élève interrogé poursuit la spécialité \fg.
\end{itemize}
\setlength\parindent{0mm}

\emph{On donnera les valeurs exactes pour chacune des questions.}

\medskip

\begin{enumerate}
\item Calculer $p(G)$, $p\left(G \cap \overline{S}\right)$ et $p\left(\overline{S}\right)$.
\item L'élève interrogé ne poursuit pas la spécialité. Calculer la probabilité que ce soit
un garçon.
\item  Les évènements $G$ et $S$ sont-ils indépendants ?
\end{enumerate}

\bigskip

