	
\subsection*{1. Montrer que la suite \((u_n)\) est une suite géométrique dont on donnera la raison.}
Ajouter 3,7 \% c'est multiplier par \(1 + \dfrac{3,7}{100} = 1 + 0,037 = 1,037\).\\
 On a donc pour tout naturel \(n\), \(u_{n+1} = 1,037u_n\) : la suite \((u_n)\) est une suite géométrique de premier terme \(u_0 = 187\) et de raison 1,037.
	
\subsection*{2. Pour tout \(n \in \mathbb{N}\), exprimer \(u_n\) en fonction de \(n\).}
On sait qu'alors avec \(q\) comme raison \(u_n = u_0 \times q^n\), quel que soit le naturel \(n\). Ici \(u_n = 187 \times 1,037^n\).
	
\subsection*{3. Étudier le sens de variation de la suite \((u_n)\).}
	
	%\[ u_{n+1} = 1,037u_n \quad \iff \quad \dfrac{u_{n+1}}{u_n} = 1,037 > 1 \]
	%La suite est donc croissante.
	La suite \((u_n)\) est géométrique de premier terme \(u_0=187\) et de raison \(q=1,037\).
	On a \(u_0 > 0\) et \(q>1\), la suite \((u_n)\) est donc strictement croissante.
	
	\subsection*{4. Selon cette estimation, calculer la production mondiale de plastique en 2019. Arrondir au million de tonnes.}
	
	2019 correspond à \(n = 19\), \\
	donc \(u_{19} = 187 \times 1,037^{19} \approx 372,9\), donc environ 373 millions au million près.
	
	\subsection*{5. Des études montrent que 20 \% de la quantité totale de plastique se retrouve dans les océans, et que 70 \% de ces déchets finissent par couler. Montrer que la quantité totale, arrondie au million de tonnes, de déchets flottants sur l'océan dus à la production de plastique de 2000 à 2019 compris est de 324 millions de tonnes.}
	
	Sur les 20 \% de plastiques allant à la mer, 30 \% flottent en 2000, soit 6 \%. La production totale de plastiques de 2000 à 2019 est :
	\[ S_{2019} = u_0 + 1,037u_0 + \ldots + 1,037^{19}u_0 \]
	\[ 1,037S_{2019} = 1,037u_0 + 1,037^2u_0 + \ldots + 1,037^{20}u_0 \]
	Par différence :
	\[ 0,037S_{2019} = 1,037^{20}u_0 - u_0 \]
	\[ S_{2019} = u_0 \dfrac{1,037^{20} - 1}{0,037} \approx 5398,32 \]
	Restent en surface :
	\[ 5398,32 \times 0,20 \times 0,30 \approx 323,89, \]
	soit 324 millions au million près.
	
