
\subsection*{Question 1}

On sait que, pour tout naturel \(n\) :
\[
u_n = u_0 + n \times r = 2 + 0{,}9n,
\]
donc en particulier :
\[
u_{50} = 2 + 50 \times 0{,}9 = 2 + 45 = 47.
\]

\subsection*{Question 2}

On sait que, pour tout naturel \(n\), \(v_n = v_0 \times q^n = 2 \times 0{,}9^n\).

Donc :
\begin{align*}
S_{36} &= v_0 + v_1 + \ldots + v_{36} \\
&= 2 + 2 \times 0{,}9 + 2 \times 0{,}9^2 + \ldots + 2 \times 0{,}9^{36}.
\end{align*}

D'où :
\[
0{,}9 S_{36} = 2 \times 0{,}9 + 2 \times 0{,}9^2 + \ldots + 2 \times 0{,}9^{37}.
\]

On en déduit par différence des deux lignes précédentes que :
\[
0{,}9 S_{36} - S_{36} = 2 \times 0{,}9^{37} - 2, \text{ d'où } -0{,}1 S_{36} = 2 \left(0{,}9^{37} - 1\right),
\]
et finalement :
\[
S_{36} = \dfrac{2 \left(0{,}9^{37} - 1\right)}{-0{,}1} = 20 \times (1 - 0{,}9^{37}) = 2 \times \dfrac{1 - 0{,}9^{37}}{1 - 0{,}9}.
\]

\subsection*{Question 3}

Le programme correct est le troisième.

\subsection*{Question 4}

De \( \cos^2 x + \sin^2 x = 1 \) ou \( \sin^2 x = 1 - \cos^2 x = 1 - 0{,}64 = 0{,}36 \).

Comme \( x \in \left[-\dfrac{\pi}{2} \,;\, 0\right] \), on sait que \( \sin x < 0 \), donc \( \sin x = -0{,}6 \).

\subsection*{Question 5}

\(\dfrac{13\pi}{4} = \dfrac{16\pi}{4} - \dfrac{3\pi}{4}\), donc le point associé au réel \( \dfrac{13\pi}{4} \) est le même que celui qui est associé à \( -\dfrac{3\pi}{4} \).

