
\medskip


\emph{Pour chacune des questions, une seule des quatre réponses proposées est correcte.}

\emph{Les questions sont indépendantes.}

\emph{Pour chaque question, indiquer le numéro de la question et recopier sur la copie la lettre correspondant à la réponse choisie.}

\emph{Aucune justification n'est demandée, mais il peut être nécessaire d'effectuer des recherches au brouillon pour aider à déterminer votre réponse.}

\emph{Chaque réponse correcte rapporte un point. Une réponse incorrecte ou une question sans réponse n'apporte ni ne retire aucun point.}

\medskip

\textbf{Question 1}

\medskip 

Dans un repère orthonormé, le cercle de centre A$(2~;~-1)$ et de rayon 4 a comme équation :

\medskip

\begin{tabularx}{\linewidth}{*{4}{X}}
\textbf{a.~~} $(x+2)^2+(y-1)^2=16$ &\textbf{b.~~} $(x-2)^2+(y+1)^2=4$&\textbf{c.~~}$(x-2)^2+(y+1)^2=16$& \textbf{d.~~} $(x+2)^2+(y-1)^2=4$.
\end{tabularx}

\medskip
\textbf{Question 2}
\medskip 

Soit la droite (d) d'équation cartésienne $2x-y+1 = 0$.
Sachant que la droite $(d_1)$ est perpendiculaire à la droite (d), une équation de $(d_1)$
 peut être :

\begin{tabularx}{\linewidth}{*{4}{X}}
\textbf{a.~~} $x-2y+2 = 0$ &\textbf{b.~~} $x+2y-1 = 0$&\textbf{c.~~}$-2x+y-1 = 0$& \textbf{d.~~} $x-y+2 = 0$.
\end{tabularx}

\medskip
\textbf{Question 3}
\medskip 
L'expression de $\sin(\pi- x)+\cos \left(x+ \frac{\pi}{2}\right)$ est égale à :


\begin{tabularx}{\linewidth}{*{4}{X}}
\textbf{a.~~} $-2\sin(x)$ &\textbf{b.~~} $0$&\textbf{c.~~}$2\sin(x)$& \textbf{d.~~} $\cos(x)-\sin (x)$.
\end{tabularx}

\medskip
\textbf{Question 4}
\medskip 

On considère la fonction polynôme du second degré $f$ définie sur $\R$ par $f(x)= -3x^2+x-5$

Le tableau de variations de cette fonction est :

\begin{tabularx}{\linewidth}{cXcX}
\textbf{a.~~}&
\begin{pspicture}(6.75,2.8)
\psframe(6.65,2.7)
\psline(0,1.7)(6.65,1.7)  \psline(1.2,0)(1.2,2.7)
\uput[u](0.7,2){$x$} \uput[u](1.8,2){$-\infty$}\uput[u](3.65,1.73){$-\dfrac{1}{6}$}\uput[u](6.3,2){$+\infty$}
\uput[d](0.7,1.2){$f$}
\psline{->}(2,0.2)(3.4,1.25)\psline{->}(4.65,1.25)(5.8,0.2)
\end{pspicture}
&\textbf{b.~~}&
\begin{pspicture}(6.75,2.8)
\psframe(6.65,2.7)
\psline(0,1.7)(6.65,1.7)  \psline(1.2,0)(1.2,2.7)
\uput[u](0.7,2){$x$} \uput[u](1.8,2){$-\infty$}\uput[u](3.65,1.73){$\dfrac{1}{3}$}\uput[u](6.3,2){$+\infty$}
\uput[d](0.7,1.2){$f$}
\psline{->}(2,0.2)(3.4,1.25)\psline{->}(4.65,1.25)(5.8,0.2)
\end{pspicture} \\
\textbf{c.~~}&
\begin{pspicture}(6.75,2.8)
\psframe(6.65,2.7)
\psline(0,1.7)(6.65,1.7)  \psline(1.2,0)(1.2,2.7)
\uput[u](0.7,2){$x$} \uput[u](1.8,2){$-\infty$}\uput[u](3.65,1.73){$-\dfrac{1}{3}$}\uput[u](6.3,2){$+\infty$}
\uput[d](0.7,1.2){$f$}
\psline{->}(2,1.25)(3.4,0.2)\psline{->}(4.65,0.2)(5.8,1.25)
\end{pspicture}&
\textbf{d.~~}&
\begin{pspicture}(6.75,2.8)
\psframe(6.65,2.7)
\psline(0,1.7)(6.65,1.7)  \psline(1.2,0)(1.2,2.7)
\uput[u](0.7,2){$x$} \uput[u](1.8,2){$-\infty$}\uput[u](3.65,1.73){$\dfrac{1}{6}$}\uput[u](6.3,2){$+\infty$}
\uput[d](0.7,1.2){$f$}
\psline{->}(2,0.2)(3.4,1.25)\psline{->}(4.65,1.25)(5.8,0.2)
\end{pspicture}.
\end{tabularx}

\medskip

\textbf{Question 5}

\medskip

À un jeu, la variable aléatoire donnant le gain algébrique $G$ suit la loi de probabilité suivante (en euros) :

\begin{center}
\begin{tabular}[]{|m{2cm}|*{4}{c|}}
\hline
Valeurs de $G$&$-25$&$-3$&$x$&$100$\\\hline
Probabilité&$\dfrac{1}{3}$&$\dfrac{1}{6}$&0,3&0,2\rule[-3mm]{0mm}{9mm}\\\hline
\end{tabular}
\end{center}

Sachant que l'espérance de $G$ est égale à $\dfrac{38}{3}$, la valeur de $x$ est :

\begin{tabularx}{\linewidth}{*{4}{X}}
\textbf{a.~~} $0$ &\textbf{b.~~} $5$&\textbf{c.~~}$20$& \textbf{d.~~} $25$.
\end{tabularx}

\vspace{0,5cm}

