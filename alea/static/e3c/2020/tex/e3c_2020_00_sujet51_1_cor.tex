
\subsection*{Question 1}

\begin{align*}
&M(x\,;\,y) \in \mathcal{C} (A, R = 4) \\
\iff &AM^2 = 4^2 \\
\iff &(x - 2)^2 + (y + 1)^2 = 16.
\end{align*}

\subsection*{Question 2}

La droite \((d)\) a pour vecteur directeur \( \vec{u} \begin{pmatrix} 1 \\ 2 \end{pmatrix} \), la droite d'équation \( x + 2y - 1 = 0 \) a pour vecteur directeur \( \vec{v} \begin{pmatrix} -2 \\ 1 \end{pmatrix} \), et \(\vec{u} \cdot \vec{v} = -2 + 2 = 0\).

\subsection*{Question 3}

\(\sin(\pi - x) + \cos\left(x + \dfrac{\pi}{2}\right) = \sin(x) - \sin(x) = 0\).

\subsection*{Question 4}

Comme :
\[
\Delta = 1^2 - (-3) \times (-5) = 1 - 15 = -14 < 0,
\]
le trinôme n'a pas de racines.

De plus, comme \(a = -3 < 0\), la fonction est croissante sur \(\, \left] -\infty \,;\, -\dfrac{b}{2a} \right[ = \left] -\infty \,;\, \dfrac{1}{6} \right[\), pui décroissante sur \(\, \left] \dfrac{1}{6} \,;\, +\infty \right[\).

Donc réponse \textbf{d.}

\subsection*{Question 5}

On a :
\begin{align*}
E(X) &= \dfrac{38}{3} \\
-\dfrac{25}{3} - \dfrac{3}{6} + 0,3x + 20 &= \dfrac{38}{3} \\
0,3x &= \dfrac{38}{3} + \dfrac{25}{3} + \dfrac{3}{6} - 20 \\
&= \dfrac{63}{3} + \dfrac{3}{6} - 20 \\
&= 21 - 20 + \dfrac{1}{2} \\
&= \dfrac{3}{2},
\end{align*}
donc :
\[
x = \dfrac{3}{2} \times \dfrac{1}{0,3} = \dfrac{10}{2} = 5.
\]

