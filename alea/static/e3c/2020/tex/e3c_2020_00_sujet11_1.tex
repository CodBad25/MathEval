
\medskip

Ce QCM comprend 5 questions. 

Pour chacune des questions, une seule des quatre réponses proposées est correcte. 

Les questions sont indépendantes.

Pour chaque question, indiquer le numéro de la question et recopier sur la copie la lettre correspondant à la réponse choisie. 

Aucune justification n'est demandée mais il peut être nécessaire d'effectuer des recherches au brouillon pour aider à déterminer votre réponse.

Chaque réponse correcte rapporte $1$ point. Une réponse incorrecte ou une question sans réponse n'apporte ni ne retire de point.

\bigskip

Une urne contient $150$ jetons rouges et $50$ jetons bleus, tous indiscernables au toucher. 20\,\% des jetons rouges sont gagnants et 40\,\% des jetons bleus sont gagnants.

 Un joueur tire au hasard un jeton de l'urne.

\medskip

\textbf{Question 1}

\medskip

La probabilité que le jeton soit rouge et gagnant est :

\begin{center}
\begin{tabularx}{\linewidth}{|*{4}{X|}}\hline
\textbf{a.~~}$0,2$&\textbf{b.~~}$0,45$&\textbf{c.~~}$0,15$&\textbf{d.~~}$0,95$\\ \hline
\end{tabularx}
\end{center}

\textbf{Question 2}

\medskip

La probabilité que le jeton soit gagnant est :

\begin{center}
\begin{tabularx}{\linewidth}{|*{4}{X|}}\hline
\textbf{a.~~}$0,2$&\textbf{b.~~}$0,6$&\textbf{c.~~}$0,25$&\textbf{d.~~}$0,95$\\ \hline
\end{tabularx}
\end{center}

\textbf{Question 3}

\medskip

Un joueur tire successivement et avec remise deux jetons de l'urne. La probabilité
qu'il tire deux jetons rouges est :

\begin{center}
\begin{tabularx}{\linewidth}{|*{4}{X|}}\hline
\textbf{a.~~}$\np{0,5625}$&\textbf{b.~~}$0,75$&\textbf{c.~~}$0,30$&\textbf{d.~~}$0,15$\\ \hline
\end{tabularx}
\end{center}

On note $X$ la variable aléatoire qui représente le gain algébrique en euros d'un joueur.

 La loi de probabilité de $X$ est donnée par le tableau suivant:
 
\begin{center}
\begin{tabularx}{\linewidth}{|c|*{3}{>{\centering \arraybackslash}X|}}\hline
Valeurs $a$ prises par $X$	& $-5$	&$0$	&$10$\\ \hline
$P(X= a)$					& 0,6	&0,15	&0,25\\ \hline
\end{tabularx}
\end{center}

\textbf{Question 4}

\medskip

La probabilité $P(X > 0)$ est égale à :

\begin{center}
\begin{tabularx}{\linewidth}{|*{4}{X|}}\hline
\textbf{a.~~}$0,15$&\textbf{b.~~}$0,6$&\textbf{c.~~}$10$&\textbf{d.~~}$0,25$\\ \hline
\end{tabularx}
\end{center}

\textbf{Question 5}

\medskip

Le gain algébrique moyen en euros que peut espérer un joueur est égale à :

\begin{center}
\begin{tabularx}{\linewidth}{|*{4}{X|}}\hline
\textbf{a.~~}$0$&\textbf{b.~~}$-0,5$&\textbf{c.~~}$\dfrac{5}{3}$&\textbf{d.~~}$5$\rule[-3mm]{0mm}{9mm}\\ \hline
\end{tabularx}
\end{center}

\bigskip

