  
\medskip

Pour chacune des cinq affirmations suivantes, dire si elle est vraie ou fausse. Chaque réponse devra être justifiée.

\emph{Toute démarche de justification même non aboutie sera prise en compte.}

\medskip

\begin{enumerate}
\item  Dans le plan muni d'un repère orthonormé, on donne les points :
\[\text{A}(2~;~-2), \hspace{1.5em}\text{B}(4~;~0),\hspace{1.5em} \text{C}(0~;~-5),\hspace{1.5em} \text{D}(-7~;~1).\]
\begin{description}
\item[Affirmation 1 :] Les droites (AB) et (CD) sont perpendiculaires.
\item[Affirmation 2 :] Une équation de la droite perpendiculaire à (AB) passant par C est : \[y = x-5\]
\item [Affirmation 3 :] Une équation du cercle de centre A passant par B est : \[(x - 2)^2+(y + 2)^2 = 8\]
\end{description}
\item Soit $f$ la fonction définie pour tout $x\in]0~;~+\infty[$ par : \[f(x)=\frac{\e^x}{x}.\]
 On note $f'$ sa fonction dérivée.
\begin{description}
\item[Affirmation 4 :] $f'(1) = 0$
\end{description}
\item On donne : $\cos\left(\dfrac{2\pi}{5}\right)=\dfrac{-1+\sqrt{5}}{4}$
\begin{description}
\item[Affirmation 5 :] $\sin\left(\dfrac{2\pi}{5}\right)< 0$
\end{description}
\end{enumerate}

\vspace{1cm}

