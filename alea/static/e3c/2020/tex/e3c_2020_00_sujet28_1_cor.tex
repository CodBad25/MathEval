	
	\subsection*{Question 1}
	Une équation de \(D\) est de la forme \(M (x\,;\,y) \in D \iff x + 3y + c = 0\) et avec :
	\begin{align*}
		A(-1\,;\,2) \in D &\iff -1 + 6 + c = 0 \\
		&\iff c = -5,
	\end{align*}
	on obtient :
	\[
	M (x\,;\,y) \in D \iff x + 3y - 5 = 0.
	\]
	
	\subsection*{Question 2}
	Un vecteur directeur de \(D\) est par exemple \(\vec{u} \begin{pmatrix} 8 \\ 5 \end{pmatrix}\) ou \(\vec{u}\begin{pmatrix} 4 \\ 2,5 \end{pmatrix}\), soit un vecteur directeur de \(D'\). Donc \(D\) est parallèle à \(D'\).
	
	\subsection*{Question 3}
	Le coefficient directeur de la droite \(D\) est celui de la droite \((AB)\), soit :
	\[
	\dfrac{1 - (-1)}{1 - 0} = \dfrac{2}{1}.
	\]
	Donc le nombre dérivé est égal à \( \dfrac{2}{1} = 2 \).
	
	\subsection*{Question 4}
	Les expressions sont toutes des trinômes du second degré. Leurs racines sont \(-1\) et \(2\). Donc leur somme est égale à \(1 = -\dfrac{b}{a}\) et leur produit \(-2 = \dfrac{c}{a}\). Le seul trinôme ayant ces propriétés est :
	\[
	-x^2 + x + 2.
	\]
	
	\subsection*{Question 5}
	\(f\), produit de fonctions dérivables sur \(\mathbb{R}\), est dérivable sur cet intervalle et :
	\[
	f'(x) = e^x + x e^x = e^x (1 + x) = (x + 1) e^x.
	\]
	
