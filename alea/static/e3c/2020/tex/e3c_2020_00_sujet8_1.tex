
\medskip

Ce QCM comprend cinq questions indépendantes.

Pour chacune des questions, une seule des quatre réponses proposées est correcte.

Pour chaque question, indiquer le numéro de la question et recopier sur la copie la lettre correspondante à la réponse choisie.

\textbf{Aucune justification n'est demandée mais il peut être nécessaire d'effectuer des recherches au brouillon pour aider à déterminer votre réponse.}

Chaque réponse correcte rapporte $1$ point. Une réponse incorrecte ou une question \textbf{sans réponse n'apporte ni ne retire de point.}

\medskip

\textbf{Question 1}

$\dfrac{\text{e}^{5x}}{\text{e}^{2x - 2}} = $

\begin{center}
\begin{tabularx}{\linewidth}{|*{4}{X|}}\hline
\textbf{a.~~}$\text{e}^{3x + 2}$&\textbf{b.~~}$\text{e}^{3x - 2}$&\textbf{c.~~}$\text{e}^{2,5x - 2,5}$&\textbf{d.~~}$\text{e}^{7x - 2}$\\ \hline
\end{tabularx}
\end{center}

\medskip

\textbf{Question 2}

Soit la suite définie par : $\left\{\begin{array}{l c l}
u_0&=&2\\u_{n+1} &=& 3u_n - 2
\end{array}\right.$ pour $n \in \N$.

\begin{center}
\begin{tabularx}{\linewidth}{|*{4}{X|}}\hline
\textbf{a.~~}$u_3 = 7$&\textbf{b.~~}$u_3 = 10$&\textbf{c.~~}$u_3 = 28$&\textbf{d.~~}$u_3 = 4$\\ \hline
\end{tabularx}
\end{center}
 
\medskip

\textbf{Question 3}

Dans un atelier 3\,\% des pièces produites sont défectueuses. On constate qu'au cours du
contrôle qualité, si la pièce est bonne, elle est acceptée dans 95\,\% des cas, et que si elle est défectueuse, elle est refusée dans 98\,\% des cas.

La probabilité qu'une pièce soit refusée est égale à :

\begin{center}
\begin{tabularx}{\linewidth}{|*{4}{X|}}\hline
\textbf{a.~~}$\np{0,0779}$&\textbf{b.~~}$\np{0,0294}$&\textbf{c.~~}$\np{0,0485}$&\textbf{d.~~}$0,98$\\ \hline
\end{tabularx}
\end{center}
   
\medskip

\textbf{Question 4}

Sachant que $\cos x = \dfrac{5}{13}$ et que $x$ est compris entre $- \dfrac{\pi}{2}$ et $0$, la valeur de $\sin x$ est: 

\begin{center}
\begin{tabularx}{\linewidth}{|*{4}{X|}}\hline
\textbf{a.~~}$\dfrac{8}{13}$&\textbf{b.~~}$- \dfrac{8}{13}$&\textbf{c.~~}$\dfrac{12}{13}$&\textbf{d.~~}$- \dfrac{12}{13}$\rule[-3mm]{0mm}{8mm}\\ \hline
\end{tabularx}
\end{center}

\medskip

\textbf{Question 5}

La loi de probabilité d'une
variable aléatoire $X$ est donnée par le tableau ci-dessous :

\begin{center}
\begin{tabularx}{\linewidth}{|l|*{3}{>{\centering \arraybackslash}X|}}\hline
Valeurs $x_i$&$-2$ &$0$& $5$\\ \hline
$p_i = P\left(X = x_i\right)$&0,3 &0,5 &0,2\\ \hline
\end{tabularx}
\end{center}

L'espérance $E(X)$ de la variable aléatoire $X$ est égale à :

\begin{center}
\begin{tabularx}{\linewidth}{|*{4}{X|}}\hline
\textbf{a.~~}$3$&\textbf{b.~~}$0,9$&\textbf{c.~~}$0,4$&\textbf{d.~~}$0,5$\\ \hline
\end{tabularx}
\end{center}

\bigskip

