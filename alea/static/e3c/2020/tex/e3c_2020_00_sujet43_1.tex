
\medskip

Ce QCM comprend 5 questions.

Pour chacune des questions, une seule des quatre réponses proposées est correcte.

Les questions sont indépendantes.

Pour chaque question, indiquer le numéro de la question et recopier sur la copie la lettre correspondante à la réponse choisie.

Aucune justification n'est demandée mais il peut être nécessaire d'effectuer des recherches au brouillon pour aider à déterminer votre réponse.

Chaque réponse correcte rapporte $1$ point. Une réponse incorrecte ou une question sans réponse n'apporte ni ne retire de point.

\medskip

\textbf{Question 1}

\medskip

On définit la fonction $f$ sur $]2,5~;~+\infty[$ par :

\[f(x)= \dfrac{3x+1}{- 2x + 5}.\]

Alors pour tout $x \in ]2,5~;~+\infty[$, ~ b)
2
$f'(x)$ est donnée par l'expression :

\begin{center}
\begin{tabularx}{\linewidth}{|*{4}{X|}}\hline
\textbf{a.~~}$- \dfrac{3}{2}$ &\textbf{b.~~}$\dfrac{17}{(- 2x + 5)^2}$&\textbf{c.~~}$\dfrac{17}{(- 2x + 5)^2}$&\textbf{d.~~}$-\dfrac{13}{(- 2x + 5)^2}$\rule[-3mm]{0mm}{9mm}\\ \hline
\end{tabularx}
\end{center}

\medskip

\textbf{Question 2}

\medskip
On considère une fonction $f$ polynôme de degré 2 dont une représentation graphique est donnée ci-dessous dans un repère orthonormé.

\begin{center}
\psset{unit=1cm}
\begin{pspicture*}(-1,-2)(5,3)
\psgrid[gridlabels=0pt,subgriddiv=1]
\psaxes[linewidth=1.25pt,labelFontSize=\scriptstyle]{->}(0,0)(-0.95,-1.95)(5,3)
\psplot[plotpoints=2000,linewidth=1.25pt,linecolor=blue]{0.5}{3.5}{2 2 x mul sub x 3 sub mul}
\end{pspicture*}
\end{center}

Par lecture graphique, on peut affirmer qu'une forme factorisée de $f$ est:

\begin{center}
\begin{tabularx}{\linewidth}{|*{4}{X|}}\hline
\textbf{a.~~}$- 2(x + 1)(x + 3)$ &\textbf{b.~~}$- 2(x - 1)(x - 3)$&\textbf{c.~~}$2(x- 1)(x - 3)$ &\textbf{d.~~}$2(x + 1)(x + 3)$\\ \hline
\end{tabularx}
\end{center}

\medskip

\textbf{Question 3}

\medskip

On se place dans un repère orthogonal. On a tracé ci-dessous la courbe représentative d'une fonction $f$ ainsi que sa tangente au point A.

\begin{center}
\psset{xunit=1cm,yunit=0.5cm}
\begin{pspicture*}(-1.5,-1)(3,6)
\psgrid[gridlabels=0pt,subgriddiv=1]
\multido{\n=-0.5+1.0}{4}{\psline[linewidth=0.4pt](\n,-1)(\n,6)}
\psaxes[linewidth=1.25pt,labelFontSize=\scriptstyle]{->}(0,0)(-1.5,-1)(3,6)
\pscurve[linewidth=1.25pt,linecolor=blue](-1.5,1.8)(-1,1.82)(0,2)(0.5,2.7)(1,4)(1.5,5.1)(1.65,5.2)(2,4)(2.5,1)(2.7,-1)
\psplot[plotpoints=200,linewidth=1.25pt]{-1}{2.5}{x  2 add}
\uput[ul](0,2){A}
\end{pspicture*}
\end{center}

On a alors:

\begin{center}
\begin{tabularx}{\linewidth}{|*{4}{X|}}\hline
\textbf{a.~~}$f'(0) = 0$ &\textbf{b.~~}$f'(0) = 2$&\textbf{c.~~}$f'(0) = 1$ &\textbf{d.~~}$f'(0) = 0,5$\\ \hline
\end{tabularx}
\end{center}

\medskip

\textbf{Question 4}

\medskip

Le plan est rapporté à un repère orthonormé. 

On considère les points G$(1~;~-2)$ et H(6~;~4). 

La droite (GH) passe par le point :

\begin{center}
\begin{tabularx}{\linewidth}{|*{4}{X|}}\hline
\textbf{a.~~}A$(-3~;~2)$ &\textbf{b.~~}B(2,5~;~0)&\textbf{c.~~}C$(10~;~12)$  &\textbf{d.~~}D$(-14~;~-20)$\\ \hline
\end{tabularx}
\end{center}

\medskip

\textbf{Question 5}

\medskip

On considère un nombre réel $x$ appartenant à l'intervalle $\left[\pi~;~\dfrac{3\pi}{2}\right]$ tel que $\cos x = - \dfrac{\sqrt{3}}{2}$.

Alors $\sin (x)$ est égal à :

\begin{center}
\begin{tabularx}{\linewidth}{|*{4}{X|}}\hline
\textbf{a.~~}$\dfrac{\sqrt{3}}{2}$ &\textbf{b.~~}$-\dfrac{\sqrt{3}}{2}$&\textbf{c.~~}$-\dfrac{1}{2}$&\textbf{d.~~}$\dfrac{1}{2}$\rule[-3mm]{0mm}{9mm}\\ \hline
\end{tabularx}
\end{center}

\bigskip

