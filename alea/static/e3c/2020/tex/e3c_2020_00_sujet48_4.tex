
\medskip

Dans un aéroport, les portiques de sécurité servent à détecter les objets métalliques que peuvent emporter les voyageurs.

On choisit au hasard un voyageur franchissant un portique. 

On note:

\setlength\parindent{9mm}
\begin{itemize}[label=\textbullet]
\item $S$ l'évènement \og le voyageur fait sonner le portique \fg{} ;
\item $M$ l'évènement \og le voyageur porte un objet métallique \fg.
\end{itemize}
\setlength\parindent{0mm}

On note $\overline{S}$ et $\overline{M}$ les évènements contraires des évènements $S$ et $M$.

On considère qu'un voyageur sur $500$ porte sur lui un objet métallique. 

On admet que :

\setlength\parindent{9mm}
\begin{itemize}[label=\textbullet]
\item Lorsqu'un voyageur franchit le portique avec un objet métallique, la probabilité que le portique sonne est égale à 0,95.
\item Lorsqu'un voyageur franchit le portique sans objet métallique, la probabilité que le portique ne sonne pas est de 0,96.
\end{itemize}
\setlength\parindent{0mm}

\medskip

\begin{enumerate}
\item À l'aide des données de l'énoncé, préciser les valeurs de $P(M)$, $P_M(S)$ et $P_{\overline{M}}\left(\overline{S}\right)$. 
\item Recopier et compléter l'arbre pondéré ci-dessous, modélisant cette situation :

\begin{center}
\pstree[treemode=R,nodesepA=0pt,nodesepB=3pt]{\TR{}}
{
\pstree{\TR{$M$~~}\taput{\ldots}}
	{\TR{$S$} \taput{\ldots}
	\TR{$\overline{S}$} \tbput{\ldots}
	}
\pstree{\TR{$\overline{M}$~~}\tbput{\ldots}}
	{\TR{$S$}\taput{\ldots}
	\TR{$\overline{S}$}\tbput{\ldots}
	}
}
\end{center}

\item Montrer que $P(S) = \np{0,04182}$.
\item En déduire la probabilité qu'un voyageur porte un objet métallique sachant qu'il a fait
sonner le portique en passant. On arrondira le résultat à $10^{-3}$.
\item Les évènements $M$ et $S$ sont-ils indépendants?
\end{enumerate}

