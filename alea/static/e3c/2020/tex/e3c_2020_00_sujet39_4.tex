
\medskip

Une enquête réalisée dans un camping a donné les résultats suivants :

\begin{itemize}
\item 60\,\% des campeurs viennent en famille, les autres viennent entre amis ;
\item parmi ceux venant en famille, 35\,\% profitent des activités du camping ;
\item parmi ceux venant entre amis, 70\,\% ne profitent pas des activités du camping.
\end{itemize}

On choisit au hasard un client de ce camping et on considère les évènements suivants :

\begin{itemize}
\item $F$ : \og le campeur choisi est venu en famille \fg,
\item $A$ : \og le campeur choisi profite des activités du camping \fg.
\end{itemize}

\medskip

\begin{enumerate}
\item Recopier et compléter l'arbre de probabilités donné ci-dessous :

\begin{center}
\psset{nodesepA=0pt,nodesepB=3pt,treesep=0.75,labelsep=0.1pt,levelsep=2.75cm}
\pstree[treemode=R]{\TR{}}
{\pstree{\TR{$F$~~}\taput{$\np{0.6}$}}
	{
	\TR{$A$}\taput{}
	\TR{$\overline{A}$}\tbput{}
	}
\pstree{\TR{$\overline{F}$~~}\tbput{}}
	{\TR{$A$}\taput{}
	\TR{$\overline{A}$}\tbput{}
	}
}
\end{center}

\medskip

\item 
	\begin{enumerate}
		\item Calculer $p(F\cap\overline{A})$.
		\item Interpréter ce résultat dans le contexte de l'exercice.
	\end{enumerate}
\item Montrer que $p(A) = 0,33$.
\item Sachant que le campeur choisi a profité des activités du camping, calculer la probabilité qu'il soit venu en famille. Arrondir le résultat au centième.
\end{enumerate}
