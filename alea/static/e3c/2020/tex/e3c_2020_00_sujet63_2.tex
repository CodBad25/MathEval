
\medskip

Durant l'été, une piscine extérieure perd chaque semaine 4\,\% de son volume d'eau par évaporation. On étudie ici un bassin qui contient 80 m$^3$ après son remplissage.

\medskip

\begin{enumerate}
\item Montrer par un calcul que ce bassin contient $76,8$~m$^3$ d'eau une semaine après son remplissage.
\item  On ne rajoute pas d'eau dans le bassinet l'eau continue à s'évaporer. On modélise le volume d'eau contenue dans la piscine par une suite $\left(V_n\right)$ : pour tout entier naturel $n$, on note $V_n$ la quantité d'eau en m$^3$ contenue dans la piscine $n$ semaines après son remplissage. Ainsi
$V_0 = 80$.
	\begin{enumerate}
		\item Justifier que pour tout entier naturel $n$,\: $V_{n+1} = 0,96 V_n$ et préciser la nature de la suite $\left(V_n\right)$ ainsi définie.
		\item Donner une expression de $V_n$ en fonction de $n$.
		\item Quelle quantité d'eau contient le bassin au bout de $7$ semaines ?
	\end{enumerate}
\item  Pour compenser en partie les pertes d'eau provoquées par l'évaporation, on décide de rajouter 2~m$^3$ d'eau chaque semaine dans le bassin. 

On souhaite déterminer au bout de combien de semaines, le volume d'eau contenu dans la piscine devient inférieur à $70$~m$^3$. 

Compléter la fonction Python suivante afin que l'appel nombreJour(70) renvoie le nombre de semaines à partir duquel le volume d'eau de la piscine sera inférieur à $70$~m$^3$.

\begin{center}
\begin{tabularx}{0.4\linewidth}{|X|}\hline
\texttt{def nombreJour(U) :}\\
\quad \texttt{N=0}\\
\quad \texttt{V= 80}\\
\quad \texttt{while \ldots  $\geqslant$ \ldots}\\
\qquad \texttt{N=N+1}\\
\qquad \texttt{V = \ldots}\\
\quad return \ldots\\ \hline
\end{tabularx}
\end{center}
\end{enumerate}

\bigskip

