
\medskip

Ce QCM comprend 5 questions. 

Pour chacune des questions, une seule des quatre réponses proposées est correcte. 

Les questions sont indépendantes.

Pour chaque question, indiquer le numéro de la question et recopier sur la copie la lettre correspondante à la réponse choisie.

Aucune justification n'est demandée mais il peut être nécessaire d'effectuer des recherches au brouillon pour aider à déterminer votre réponse.

Chaque réponse correcte rapporte 1 point. Une réponse incorrecte ou une question sans réponse n'apporte ni ne retire de point.

\medskip
\textbf{Question 1}
\medskip 

Pour tout réel $x,\dfrac{\e^{2x}}{\e^{x+1}}$ est égale à :

\medskip 
\begin{tabularx}{\linewidth}{*{4}{X}}
\textbf{a.~~} $\e^{x-1} $ &\textbf{b.~~} $\e^{3x+1} $&\textbf{c.~~}$\dfrac{2x}{x+1} $& \textbf{d.~~} $ \e $.
\end{tabularx}

\medskip
\textbf{Question 2}
\medskip 

Dans le plan muni d'un repère, les courbes représentatives des fonctions
\[x \mapsto 15x^2 + 10x - 1 \ \text{et  }\ x\mapsto  19x^2 - 22x + 10\] ont :

\medskip 
\begin{tabularx}{\linewidth}{*{4}{X}}
\textbf{a.~~}  aucun point
d'intersection &\textbf{b.~~} un seul point d'intersection &\textbf{c.~~}deux points d'intersection& \textbf{d.~~}quatre points
d'intersection.

\end{tabularx}

\medskip
\textbf{Question 3}
\medskip 

Le plan est rapporté à un repère orthonormé. Le cercle de centre A de
coordonnées $(3~;~- 1)$ et de rayon 5 a pour équation cartésienne :

\medskip 
\begin{tabularx}{\linewidth}{*{4}{X}}
\textbf{a.~~} $(x + 3)^2 + ( y - 1)^2 = 25$&\textbf{b.~~} $( x - 3)^2 + ( y + 1)^2 = 5 $\\\textbf{c.~~}$(x + 3)^2 + ( y - 1)^2 = 5 $& \textbf{d.~~} $( x - 3)^2 + ( y + 1)^2 = 25  $.
\end{tabularx}

\medskip
\textbf{Question 4}
\medskip 

Dans un repère orthonormé, la droite $d$ d'équation cartésienne
$3 x + 2 y + 4 = 0$ admet un vecteur normal de coordonnées :

\medskip 
\begin{tabularx}{\linewidth}{*{4}{X}}
\textbf{a.~~} $\binom{\phantom{-}2}{-3}$ &\textbf{b.~~}$\binom{-3}{\phantom{-}2}$&\textbf{c.~~}$\binom{3}{2}$& \textbf{d.~~}$\binom{2}{3}$ .
\end{tabularx}

\medskip
\textbf{Question 5}
\medskip 

Le plus petit entier naturel $n$ tel que la somme $1 + 2 + 3 + 4 + \dots + n$ soit
supérieure à \np{5000} est égal à :

\medskip 

\begin{tabularx}{\linewidth}{*{4}{X}}
\textbf{a.~~} $\np{1000} $ &\textbf{b.~~} $ 500 $&\textbf{c.~~}$ 200$& \textbf{d.~~} $100  $.
\end{tabularx}

\vspace{0,5cm}

