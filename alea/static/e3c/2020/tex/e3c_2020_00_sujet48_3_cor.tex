
\subsection*{1.}

Augmenter de 2 \%, c'est multiplier par \( 1 + \dfrac{2}{100} = 1 + 0{,}02 = 1{,}02 \).

À partir de \(u_0 = 3\ 300\ 000\), on a donc \(u_1 = u_0 \times 1{,}02\), puis :
\begin{align*}
u_2 &= 1{,}02 u_1 \\
&= 1{,}02^2 u_0 \\
&= 1{,}02^2 \times 3\ 300\ 000 \\
&= 3\ 433\ 320.
\end{align*}
En 2021, le nombre de personnes atteintes de diabète en France sera de \(3\ 433\ 320\).

\subsection*{2.}

Quel que soit \( n \), \( u_{n+1} = 1{,}02 \times u_n \) : La suite \( (u_n) \) est doncune suite géométrique de raison \(q = 1{,}02\) et de premier terme \( u_0 = 3\ 300\ 000 \).

\subsection*{3.}

On sait que, pour tout naturel \( n \), \( u_n = 3\ 300\ 000 \times 1{,}02^n \).

\subsection*{4.}

2025 correspond à \(n = 6\) et \(u_6 = 3\ 300\ 000 \times 1{,}02^6 \approx 3\ 716\ 335{,}9\), soit environ \(3\ 716\ 336\) personnes seront atteintes de diabète en France en 2025.

\subsection*{5.}

L'algorithme calcule le nombre de personnes atteintes tant que leur nombre est inférieur à \(S\).

Donc pour un seuil de \(5\ 000\ 000\) il faut dépasser \(2019 + 21 = 2040\).

