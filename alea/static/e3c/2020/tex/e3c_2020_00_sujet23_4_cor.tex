	
	\textbf{Question 1}
	
	On a $u_n = 0 + 10n$. La suite $(u_n)$ est arithmétique de premier terme $u_0 = 0$ et de raison $10$.
	
	\textbf{Question 2}
	
	a. Retrancher 2\% signifie multiplier par $1 - \dfrac{2}{100} = 0,98$. On a donc :
	\[
	u_1 = u_0 \times 0,98 = 1000 \times 0,98 = 980.
	\]
	
	b. Si le nombre de pions noirs diminue de 2\% chaque minute, la suite $(v_n)$ est géométrique de raison $0,98$ et de premier terme $10 000$. On a :
	\[
	v_n = 1000 \times 0,98^n.
	\]
	Lucas peut gagner la partie si elle dure au moins 45 minutes. À la fin, Lucas aura $430$ pions blancs et $419$ pions noirs.
	
