
\subsection*{1.}

\paragraph{a.} On a \( u_1 = 3600 + 200 = 3800 \) (€).

\paragraph{b.} On a, pour tout naturel \( n \), \( u_n = 3600 + 200n \).

2030 correspond à \( n = 10 \), donc :
\begin{align*}
u_{10} &= 3600 + 200 \times 10 \\
&= 3600 + 2000 \\
&= 5600 \text{ (€)}.
\end{align*}

\subsection*{2.}

\paragraph{a.} Augmenter de 5 \%, c'est multiplier par \( 1 + \dfrac{5}{100} = 1 + 0,05 = 1{,}05 \), donc :
\[
v_1 = 3600 \times 1{,}05 = 3780 \text{ (€)}.
\]

\paragraph{b.} On a, pour tout naturel \( n \), \( v_n = 3600 \times 1{,}05^n \),

donc en particulier :
\[
v_{10} = 3600 \times 1{,}05^{10} \approx 5864{,}02 \text{ (€)}.
\]

\subsection*{3.}

Le script fonctionne tant que le loyer avec le contrat n° 1 est supérieur ou égal à celui du contrat n° 2.

Cela signifie qu'à partir de \( n = 7 \), soit en 2017, le loyer avec le contrat n° 2 sera plus onéreux que l'autre \( (u_7 = 5000 \text{ et } v_7 \approx 5065{,}56) \).

