
\subsection*{1.}

\paragraph{a.} Augmenter de 11 \%, c'est multiplier par \( 1 + \dfrac{11}{100} = 1 + 0{,}11 = 1,11 \).

On a donc, pour tout naturel \( n \), \( u_{n+1} = u_n \times 1,11 \) : la suite \( (u_n) \) est donc une suite géométrique de raison \(q = 1{,}11\) et de premier terme \( u_0 = 300 \).

\paragraph{b.} On sait que, pour tout naturel \( n \), \( u_n = u_0 \times q^n \), \( q \) étant la raison de la suite.

Donc \( u_n = 300 \times 1,11^n \), quel que soit le naturel \( n \).

\subsection*{2.}

\paragraph{a.}
\begin{center}
\begin{python}
u = 300
n = 0
while u < 1000 :
    u = u * 1.11
    n = n + 1
\end{python}
\end{center}

\paragraph{b.} On a \( u_{11} \approx 945{,}5 \) et \( u_{12} \approx 1049{,}5 \).

Il changera de péniche en \( 2012 + 12 = 2024 \).

\subsection*{3.}

On a :
\[
S = u_0 + u_1 + u_2 + \dots + u_7 \quad (1),
\]
et par conséquent :
\[
1,11S = u_1 + u_2 + \dots + u_7 + u_8 \quad (2),
\]
d'où par différence \((2) - (1)\) :
\[
0,11S = u_8 - u_0 \quad \text{d'où} \quad S = \dfrac{300 \times 1,11^8 - 300}{0,11} \approx 3558 (t).
\]

Il aura donc transporté 3558 tonnes à 15 € la tonne.

Son chiffre d'affaires total entre 2012 et 2019 est donc \( 3558 \times 15 = 53370 \), soit beaucoup moins que les 70000 € prévus.

