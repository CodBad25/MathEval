
\medskip

Dans un aéroport, les portiques de sécurité servent à détecter les objets métalliques que pourraient emporter certains voyageurs.

On choisit au hasard un voyageur franchissant un portique. On note :

\begin{itemize}
\item $S$ l'évènement \og le voyageur fait sonner le portique\fg.
\item $M$ l'évènement \og le voyageur porte un objet métallique\fg.
\end{itemize}

On considère qu'un voyageur sur $500$ porte sur lui un objet métallique.

On remarque que :
\begin{itemize}
\item Lorsqu'un voyageur franchit le portique avec un objet métallique, la probabilité que le portique sonne est égale à $0,98$.
\item Lorsqu'un voyageur franchit le portique sans objet métallique, la probabilité que le portique ne sonne pas est aussi égale à $0,98$.
\end{itemize}

\medskip

\begin{enumerate}
\item  Recopier et compléter l'arbre de probabilités ci-dessous illustrant cette situation :

\begin{center}
\psset{nodesepA=0pt,nodesepB=3pt,treesep=0.75,labelsep=0.1pt,levelsep=2.75cm}
\pstree[treemode=R]{\TR{}}
{\pstree{\TR{$M$~}\taput{$\dots$}}
	{
	\TR{$S$}\taput{$\dots$}
	\TR{$\overline{S}$}\tbput{$\dots$}
	}
\pstree{\TR{$\overline{M}$~}\tbput{$\dots$}}
	{\TR{$S$}\taput{$\dots$}
	\TR{$\overline{S}$}\tbput{$\dots$}
	}
}
\end{center}

\medskip

\item Montrer que : $p(S)= \np{0.021 92}$.
\end{enumerate}

 On suppose qu'à chaque fois qu'un voyageur franchit le portique, la probabilité que ce portique sonne est égale à \np{0.02192}, et ce de façon indépendante des éventuels déclenchements de sonnerie lors des passages des autres voyageurs.

Deux personnes passent successivement le portique de sécurité. On note $X$ la variable aléatoire donnant le nombre de fois où le portique sonne.

\begin{enumerate}[resume]
\item 
	\begin{enumerate}
		\item  Justifier qu'on peut modéliser la loi de $X$ par une loi binomiale $\mathcal{B}(n~;~p)$ dont on précisera les paramètres $n$ et $p$.
		\item  Reprendre et compléter le tableau donnant la loi de $X$ :

\begin{center}
\begin{tabular}[]{|c|*{3}{>{\centering \arraybackslash}m{2cm}|}}
\hline
$k$ &0&1&2\\\hline
$p(X=k)$&&&\\
\hline
\end{tabular}

\end{center}

		\item Calculer et interpréter l'espérance de $X$ dans le contexte de l'exercice.
	\end{enumerate}
\end{enumerate}

\vspace{0,5cm}

