
\medskip

Cet exercice est un questionnaire à choix multiples (QCM). 

Les  questions sont indépendantes.

Pour chacune des cinq questions, une seule des quatre réponses proposées est exacte.

Le candidat indiquera sur sa copie le numéro de la question et la lettre correspondant à la
réponse choisie.

Aucune justification n’est demandée mais il peut être nécessaire d’effectuer des recherches
au brouillon pour aider à déterminer la réponse.

Une réponse exacte rapporte un point, une réponse fausse ou une absence de réponse ne rapporte ni n’enlève aucun point.

\medskip

\begin{enumerate}
\item Soit $p$ une probabilité sur un univers $\Omega$ et $A$ et $B$ deux évènements indépendants tels que
$p(A) = 0,5$ et $p(B) = 0,2$.

Alors $p(A\cup B)$ est égal à :

\begin{center}
\begin{tabularx}{\linewidth}{*{4}{X}}
\textbf{a.~~} $0,1$ &\textbf{b.~~} $0,7$&\textbf{c.~~}$0,6$& \textbf{d.~~}On ne peut pas savoir.
\end{tabularx}
\end{center}

\item La valeur arrondie au centième de $1 + 1,2 + 1,2^2 + 1,2^3 + \dots + 1,2^{10}$ est :

\begin{center}
\begin{tabularx}{\linewidth}{*{4}{X}}
\textbf{a.~~} $3,27$ &\textbf{b.~~} $25,96$&\textbf{c.~~}$26,96$& \textbf{d.~~} $32,15$.
\end{tabularx}
\end{center}

\item Soit $f$ la fonction définie sur $R$ par $f(x) =\dfrac{x}{\e^x}$

Pour tout réel $x$, $f(x)$ est égal à :

\begin{center}
\begin{tabularx}{\linewidth}{*{4}{X}}
\textbf{a.~~} $f(x) = \dfrac{\e^{-x}}{-x}$ &\textbf{b.~~} $f(x) = x\e^{-x}$&\textbf{c.~~}$f(x) = -x\e^{-x}$& \textbf{d.~~} $f(x) = \dfrac{\e^{-x}}{x}$.
\end{tabularx}
\end{center}

\item Soit $g$ la fonction définie sur $\R$ par $g(x) = (2x - 5)\e^x$. On admet que $g$ est dérivable sur $\R$
et on note $g'$ sa fonction dérivée.

Alors pour tout réel $x$, $g'(x)$ est égal à :

\begin{center}
\begin{tabularx}{\linewidth}{*{4}{X}}
\textbf{a.~~} $(2x -3)\e^x$ &\textbf{b.~~} $(-2x + 7)\e^x$&\textbf{c.~~}$2\e^x$& \textbf{d.~~} $-5\e^x$.
\end{tabularx}
\end{center}

\item  Le nombre $\dfrac{\e^3\times \e^{-5}}{\e^2}$ est égal à :

\begin{center}
\begin{tabularx}{\linewidth}{*{4}{X}}
\textbf{a.~~} $ -1 $ &\textbf{b.~~} $\e^{-\dfrac{15}{2}}$&\textbf{c.~~}$ \dfrac{1}{\e^{4}}$& \textbf{d.~~} $\dfrac{3 \e^{-5}}{2} $.
\end{tabularx}
\end{center}

\end{enumerate}

\vspace{0,5cm}

