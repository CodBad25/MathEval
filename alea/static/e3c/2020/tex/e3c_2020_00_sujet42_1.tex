
\medskip

Ce QCM comprend 5 questions.

Pour chacune des questions, une seule des quatre réponses proposées est correcte.

Les questions sont indépendantes.

Pour chaque question, indiquer le numéro de la question et recopier sur la copie la lettre correspondante à la réponse choisie.

Aucune justification n'est demandée mais il peut être nécessaire d'effectuer des recherches au brouillon pour aider à déterminer votre réponse.

Chaque réponse correcte rapporte $1$ point. Une réponse incorrecte ou une question sans réponse n'apporte ni ne retire de point.

\medskip

\textbf{Question 1}

\medskip

On considère la fonction $g$ définie sur $\R$ par $g(x) = 2x^2 + 5x - 4$.

La tangente à la courbe représentative de $g$ au point d'abscisse $2$ a pour équation:

\begin{center}
\begin{tabularx}{\linewidth}{|*{4}{X|}}\hline
\textbf{a.~~}$y = 14x + 14$ &\textbf{b.~~}$y = 14x -14$&\textbf{c.~~}$y = 13x - 15$&\textbf{d.~~}$y = 13x - 12$\\ \hline
\end{tabularx}
\end{center}

\medskip

\textbf{Question 2}

\medskip

On se place dans un plan muni d'un repère orthonormé. On considère les points A(4~;~8), 
B(9~;~6) et D(2~;~11). Alors $\vect{\text{AD}} \cdot \vect{\text{BD}}$ est égal à :

\begin{center}
\begin{tabularx}{\linewidth}{|*{4}{X|}}\hline
\textbf{a.~~}$-1$ &\textbf{b.~~}$11$&\textbf{c.~~}$-31$&\textbf{d.~~}$29$\\ \hline
\end{tabularx}
\end{center}

\medskip

\textbf{Question 3}

\medskip

Dans un plan muni d'un repère orthonormé, on considère la droite $D$ d'équation
$3x - 4y + 5 = 0$. La droite parallèle à $D$ et passant par A(4~;~8) a pour équation :

\begin{center}
\begin{tabularx}{\linewidth}{|*{4}{X|}}\hline
\textbf{a.~~}$4x +3y - 40 = 0$ &\textbf{b.~~}$3x - 4y - 5 = 0$&\textbf{c.~~}$3x - 4y +20 = 0$&\textbf{d.~~}$4x +3y +6 = 0$\\ \hline
\end{tabularx}
\end{center}

\medskip

\textbf{Question 4}

\medskip

Soit $\left(u_n\right)$ la suite géométrique de raison $q = - 1,2$ et de terme initial $u_0 = 10$. 

Alors :

\begin{center}
\begin{tabularx}{\linewidth}{|*{4}{X|}}\hline
\textbf{a.~~}$0< u_{\np{3000}} < \np{1000}$ &\textbf{b.~~}$u_{\np{3000}} = \np{- 3590}$&\textbf{c.~~}$u_{\np{3000}} > \np{1000}$&\textbf{d.~~}$u_{\np{3000}} = - \np{36000}$\\ \hline
\end{tabularx}
\end{center}

\medskip

\textbf{Question 5}

\medskip

Soit $\left(v_n\right)$ la suite définie par : $v_0 = 1$ et $v_{n+1} = 4v_n +2$ pour tout entier $n$.

On veut déterminer la plus petite valeur de $n$ telle que $v_n$ est supérieur ou égal
à \np{100000}. On réalise pour cela le programme incomplet ci-dessous écrit en langage
Python :

\begin{center}
\begin{tabular}{|l|}\hline
\texttt{def algo( ) : }\\
\qquad \texttt{V = 1}\\
\qquad \texttt{n = 0}\\
\qquad \texttt{while \ldots \ldots :}\\
\qquad \qquad \texttt{n = n + 1}\\
\qquad \qquad \texttt{V =4 * V + 2}\\
\qquad  \texttt{return(n)}\\ \hline
\end{tabular}
\end{center}

Pour que le programme retourne la valeur demandée, il faut compléter la partie en pointillé par:

\begin{center}
\begin{tabularx}{\linewidth}{|*{4}{X|}}\hline
\textbf{a.~~}$V = = \np{100000}$ &\textbf{b.~~}$V!= \np{100000}$&\textbf{c.~~}$V > \np{100000}$&\textbf{d.~~}$ V < \np{100000}$\\ \hline
\end{tabularx}
\end{center}

\bigskip

