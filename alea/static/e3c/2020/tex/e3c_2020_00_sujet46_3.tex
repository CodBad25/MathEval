
\medskip

Une personne souhaite louer une maison à partir du 1\up{er} janvier 2020 et a le choix entre deux formules de contrat :

\begin{itemize}[label=\textbullet]
\item Contrat \no 1 : le loyer augmente chaque année de 200 €.
\item Contrat \no 2 : le loyer augmente chaque année de 5\,\%.
\end{itemize}

Pour tout entier naturel $n$, on note :

\begin{itemize}[label=\textbullet]
\item $u_n$ le loyer annuel de l'année $2020 + n$ pour le contrat \no 1.
\item $v_n$ le loyer annuel de l'année $2020 + n$ pour le contrat \no 2.
\end{itemize}

Dans les deux cas, le loyer annuel initial est de \np{3600}~\euro. 

On a donc $u_0 = v_0 = \np{3600}$.

\medskip

\begin{enumerate}
\item Étude de la suite $\left(u_n\right)$
	\begin{enumerate}
		\item Déterminer le loyer annuel de l'année 2021 pour le contrat n°l.
		\item Déterminer l'expression de $u_n$ en fonction de $n$ puis en déduire le loyer annuel de
l'année 2030.
	\end{enumerate}
\item Étude de la suite $\left(v_n\right)$
	\begin{enumerate}
		\item Déterminer le loyer annuel de l'année 2021 pour le contrat \no 2.
		\item Déterminer l'expression de $v_n$ en fonction de $n$ puis en déduire le loyer annuel de l'année 2030.
	\end{enumerate}
\item On considère le script suivant, écrit en langage Python :

\begin{center}
\begin{tabular}{|l|}\hline
\texttt{u = \np{3600}}\\
\texttt{v = \np{3600}}\\
\texttt{n = 0}\\
\texttt{while u > = v}\\
\qquad \texttt{u = u + 200}\\
\qquad \texttt{v = 1,05*v}\\
\qquad \texttt{n = n + 1}\\ \hline
\end{tabular}
\end{center}

Après exécution, la variable $n$ contient la valeur 6. Donner une interprétation de ce résultat dans le contexte de l'exercice.
\end{enumerate}

\bigskip

