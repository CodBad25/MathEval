
\medskip

Dans un jeu, Jeanne doit trouver la bonne réponse à une question posée.

Les questions sont classées en trois catégories : sport, cinéma et musique.

Jeanne, fervente supportrice de ce jeu, est consciente qu'elle a :

$\bullet~~$1 chance sur 2 de donner la bonne réponse sachant qu'elle est interrogée en sport ;

$\bullet~~$3 chances sur 4 de donner la bonne réponse sachant qu'elle est interrogée en cinéma ;
 
$\bullet~~$1 chance sur 4 de donner la bonne réponse sachant qu'elle est interrogée en musique.

On note :

\quad $S$ l'évènement: \og Jeanne est interrogée en sport\fg{} ;

\quad $C$ l'évènement: \og Jeanne est interrogée en cinéma\fg{} ; 

\quad $M$ l'évènement: \og Jeanne est interrogée en musique\fg{} ; 

\quad $B$ l'évènement: \og Jeanne donne une bonne réponse \fg


\emph{Rappel de notation: la probabilité d'un évènement $A$ est notée $P(A)$.}

Dans chaque catégorie, il y a le même nombre de questions. On admet donc que
$P(S) = P(C) = P(M) = \dfrac{1}{3}$.

\begin{enumerate}
\item Construire un arbre pondéré décrivant la situation.
\item Jeanne tire au hasard une question. Montrer que $P(B) = \dfrac{1}{2}$.
\end{enumerate}

Pour participer à ce jeu, Jeanne doit payer 10~\euro{} de droit d'inscription. Elle recevra:

$\bullet~~$ 10 \euro{} si elle est interrogée en sport et que sa réponse est bonne ;

$\bullet~~$ 20 \euro{} si elle est interrogée en cinéma et que sa réponse est bonne ;

$\bullet~~$ 50 \euro{} si elle est interrogée en musique et que sa réponse est bonne ;

$\bullet~~$ rien si la réponse qu'elle donne est fausse.

On note $X$ la variable aléatoire qui, à chaque partie jouée par Jeanne associe son gain algébrique, c'est-à-dire la différence en euros entre ce qu'elle reçoit et les $10$~\euro{} de droit d'inscription.

\begin{enumerate}[resume]
\item Montrer que $P(X = 40) = \dfrac{1}{12}$.
\item Déterminer la loi de probabilité de $X$.
\item Calculer l'espérance mathématique de $X$. Jeanne a-t-elle intérêt à jouer ?
\end{enumerate}
