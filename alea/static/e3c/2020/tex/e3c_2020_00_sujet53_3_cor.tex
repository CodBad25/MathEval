
\subsection*{1.}

\paragraph{a.} On a :
\[
P_1 = (1 + 0{,}2) \times 500 + 70 = 500 \times 1{,}2 + 70 = 670,
\]
puis :
\[
P_2 = 1{,}2P_1 + 70 = 1{,}2 \times 670 + 70 = 874.
\]

\paragraph{b.}
\begin{center}
\begin{python}
def Nombrebacteries(N) :
    P = 500
    for i in range(0, N) :
        P = P * 1.2 + 70
    return P
\end{python}
\end{center}

\subsection*{2.}

\paragraph{a.} On a donc \(\alpha = 0{,}09\) et \(\beta = 0\).

\paragraph{b.} Puisque chaque jour la population augmente de 9 \%, elle est multipliée par \(1{,}09\) : c'est donc une suite géométrique de raison \(1{,}09\) et de premier terme 500.

\paragraph{c.} On sait que, pour tout naturel \(n\), \(P_n = 500 \times 1{,}09^n\).

Donc avec \(n = 9\), \(P_9 = 500 \times 1{,}09^9 \approx 1085{,}95\) : la population a plus que doublé.

