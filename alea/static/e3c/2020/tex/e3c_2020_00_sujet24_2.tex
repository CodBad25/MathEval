
\medskip

On considère une fonction $f$ définie et dérivable sur l’intervalle $[-4~;~2]$.

La fonction dérivée de $f$ est notée $f'$.

Dans le repère orthonormé ci-dessous, la courbe $\mathcal{C}$ est la courbe représentative de $f$ sur l’intervalle $[-4~;~2]$.

Le point A est le point de la courbe $\mathcal{C}$ d’abscisse $- 1$.

La droite $\mathcal{T}$ est la tangente à la courbe $\mathcal{C}$ en A.

\begin{center}
\psset{xunit=1.8cm,yunit=1.8cm,labelFontSize=\scriptstyle,showorigin=false,comma=true}
\begin{pspicture}(-5,-2.1)(3,3.15)
\multido{\n=-4.3+0.1}{69}{\psline[linewidth=0.3pt,linecolor=lightgray](\n,-2.1)(\n,3)}
\multido{\n=-2.1+0.1}{51}{\psline[linewidth=0.3pt,linecolor=lightgray](-4.3,\n)(2.5,\n)}
\multido{\n=-4+0.5}{14}{\psline[linewidth=0.55pt](\n,-2.1)(\n,3)}
\multido{\n=-2+0.5}{11}{\psline[linewidth=0.55pt](-4.3,\n)(2.5,\n)}
\psaxes[linewidth=0.85pt,,Dy=0.5,]{->}(0,0)(-4.3,-2.1)(2.5,3.1)
 \def\Func{2.71828 x exp x x neg 2.5 add mul 1 sub mul}
 \psplot[plotpoints=2000,linewidth=1.25pt,linecolor=blue]{-4}{2}{\Func}
 \psplot[plotpoints=2000,linewidth=0.75pt,linecolor=green]{-4.3}{2.5}{1.6654575 neg}
\uput[ur](-4,-0.85){\Large\blue $\displaystyle \mathcal{C}$}
\uput[dl](-4,-1.6){\Large\textcolor{green} {$\displaystyle \mathcal{T}$}}
\psdot[dotstyle=+,dotscale=1.8,linecolor=red](-1,-1.6654575)
\uput[d](-1,-1.6654575){A}
\end{pspicture}
\end{center}

\begin{enumerate}
\item Par lecture graphique, donner la valeur de $f'(-1)$.
\item Résoudre, graphiquement, l’inéquation $f'(x)\leqslant 0$.
\end{enumerate}

On admet que la fonction $f$ est définie sur $[-4~;~2]$ par $f(x)=(-x^2 + 2,5x-1)\e^x$.
\begin{enumerate}[resume]
\item Vérifier que, pour tout réel $x$ de l’intervalle $[-4~;~2]$,

\[f'(x)= (- x^2 + 0,5x + 1,5)\e^x.\]

\item Étudier le signe de la fonction $f'$ sur l’intervalle $[-4~;~2]$.
\item En déduire les variations de $f$ sur l’intervalle $[-4~;~2]$.
\end{enumerate}

\vspace{0,5cm}

