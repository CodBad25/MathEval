

\begin{center}
\textbf{Concert musical}
\end{center}

Lors d'un concert de musique rock organisé dans la ville de Venise, une scène flottante était placée à 120~m au large de la côte et donc des spectateurs du premier rang. Cette configuration particulière a posé des problèmes d'acoustique liés à l'atténuation différentielle du son émis par les différents instruments, notamment du fait de l'influence de la fréquence du son sur la directivité de l'émission par les haut-parleurs.\\
L'exercice propose de modéliser cette situation à partir de données expérimentales.

\begin{list}{\textbullet}{\textbf{Données:}}
\item Fréquences correspondant à certaines notes de musique:

\begin{center}
\begin{tabularx}{0.9\linewidth}{|>{\centering \arraybackslash}m{2cm}|*{7}{>{\centering \arraybackslash}X|}c|}
\hline
\cellcolor{lightgray} \textbf{Note} & Do1 & La1 & Mi2 & Ré3 & Do4 & Fa4 & Si4\\
\hline
\cellcolor{lightgray} \textbf{Fréquence}\newline \textbf{(Hz)} & $65,4$ & $110$ & $165$ & $294$ & $523$ & $698$ & $988$\\
 \hline
\end{tabularx}
\end{center}

\item Le niveau sonore $L$ (en dB) d'une onde sonore est relié à son intensité acoustique $I$ (en W.m$^{-2}$) par la relation:

\[L=10 \,\log \dfrac{I}{I_0},\]

où $I_0=10^{-12}$~W.m$^2$  et $\log$ désigne le logarithme décimal.
\end{list}

\medskip

[...]

\medskip
On étudie mathématiquement le modèle obtenu en introduisant les fonctions $f$ et $g$ définies sur $[1\;;\;+\infty[$ par:

\[f(x) = 125 -10 \ln(x) \text{ et } g(x) = 117 - 7,5 \ln(x).\]

Ces fonctions modélisent respectivement les niveaux sonores du La1 et du Fa4 en fonction de la distance.

\medskip

\begin{enumerate}[start=5]
\item Déterminer une expression de $f '(x)$ où $f '$ est la fonction dérivée de $f$ sur $[1\;;\;+\infty[$.
\end{enumerate}

On modifie désormais les réglages d'émission pour améliorer la qualité du son. Les expressions des nouvelles fonctions décrivant la dépendance de $L_1$ et $L_2$ avec la distance sont alors :

\[f_m(x) = 148 - 10 \ln(x) \text{ et } g_m(x) = 136 - 7,5 \ln(x),\]

respectivement, pour les notes La1 et Fa4.

\begin{enumerate}[start=6]
\item Résoudre l'équation $f_m(x) = g_m(x)$ correspondant à $148 - 10 \ln(x) = 136 - 7,5 \ln(x)$ (arrondir le résultat à $10^{-1}$).

En déduire la distance $d_m$ des enceintes à laquelle doit se trouver le public pour que les deux notes aient le même niveau sonore.

\item Pour les réglages modifiés, calculer le niveau sonore du son reçu par les
spectateurs à la distance $d_m$ des enceintes pour chacune des notes.
\end{enumerate}

\bigskip


