
\medskip

Un filtre dans un circuit électrique permet de transmettre sélectivement certaines composantes du spectre en fréquence d'un signal.

On considère un filtre composé d'une résistance et d'un condensateur.

On appelle fonction de transfert de ce filtre, la fonction $H$ définie par: 
\[H\left(\omega_c\right) = \dfrac{1 - \text{i}}{2},\]
où $\omega_{\text{C}}$ est la pulsation de coupure du filtre.

On pose en cascade deux filtres identiques de même pulsation de coupure tel que leur fonction de transfert, notée $H_T\left(\omega_c\right)$, est égale au produit des fonctions de transfert de chacun des deux filtres. Ainsi :
\[H_T\left(\omega_c\right) = H\left(w_c\right) \times H\left(\omega_c\right).\]

Donner le module et un argument de $H_T\left(\omega_c\right)$.

