
\medskip

Un filtre dans un circuit électrique permet de transmettre sélectivement certaines composantes du spectre en fréquence d'un signal.

On considère le filtre, composé d'une résistance $R$ et d'un condensateur $C$.

On appelle fonction de transfert de ce filtre, la fonction $H$ définie par: 
\[H(\omega) = \dfrac{1}{1 + RC \omega \cdot \text{i}}\]
où :
\begin{itemize}
\item i est le nombre complexe de module 1 et d'argument $\dfrac{\pi}{2}$ vérifiant i$^2 = -1$ ;
\item $R$ est la résistance, exprimée en Ohm, ayant pour valeur $10^6 \Omega$ ;
\item $C$ est la capacité du condensateur, exprimée en Farad, ayant pour valeur $10^{- 6}$ F ;
\item $\omega$ est la pulsation du signal aux bornes du circuit, exprimée en rad.s$^{- 1}$.
\end{itemize}

\medskip

La pulsation de coupure du filtre est définie par $\omega_{\text{C}} = \dfrac{1}{RC}$.

Calculer $\omega_{c}$, puis montrer que $H\left(\omega_{c}\right) = \dfrac12 - \dfrac12\text{i}$.

\bigskip

