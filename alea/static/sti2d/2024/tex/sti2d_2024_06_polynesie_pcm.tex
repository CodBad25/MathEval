
\begin{center}
\textbf{Étude de l'évolution de la température d'un soda}
\end{center}

On verse, dans une tasse en porcelaine, du soda tout juste sorti du réfrigérateur. La tasse est ensuite posée sur une table. La température de l'air ambiant est supposée constante et égale à $21\:\degres$ C.

On admet que la fonction $f$ qui modélise l'évolution de la température (en degré Celsius) du contenu de la tasse en fonction du temps $t$ écoulé (en minute) depuis la première mesure vérifie l'équation différentielle :
\[y' = - \dfrac{1}{90}y + \dfrac{7}{30}\]

\begin{enumerate}[start=2]
\item Sachant que $g(0) = 7$, démontrer que, pour tout réel $t$ positif ou nul :
\[f(t) = - 14 \e^{- \frac{1}{90}t} + 21.\]

\item Calculer $\displaystyle\lim_{t \to + \infty} f(t)$.
Interpréter ce résultat dans le contexte de l'expérience.

\item Déterminer, à partir de ce modèle, la valeur du temps $t$ pour lequel la boisson atteint la température de $20~\degres$ C. Arrondir le résultat (en minute) à l'unité.
\end{enumerate}

\bigskip


