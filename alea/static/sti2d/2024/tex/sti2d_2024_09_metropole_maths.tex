
\medskip

\textbf{Les questions 1, 2, 3 et 4 sont indépendantes les unes des autres.}

\bigskip

\textbf{Question 1}

\medskip

\emph{Pour cette question, indiquer la lettre de la réponse exacte. Aucune justification n'est demandée.}

\medskip

Pour tout nombre réel $x > 0$, l'expression $3\ln (2x) - \ln (8)$ est égale à :

\begin{center}
\begin{tabularx}{\linewidth}{|*{4}{>{\centering \arraybackslash}X|}}\hline
A&B&C&D\\ \hline
$\ln \left(\dfrac2 x \right)$&$3\ln (x)$& $3\ln \left(\dfrac x4\right)$&$3\ln (2x - 8)$\\ \hline
\end{tabularx}
\end{center}

\bigskip

\textbf{Question 2}

\medskip

\emph{Pour cette question, indiquer la lettre de la réponse exacte. Aucune justification n'est demandée.}

\medskip

Soit la fonction $g$ définie sur $\R$ par :
\[g(x) = x^2\e^{-2x}.\]

On admet que $g$ est dérivable sur $\R$ et on note $g'$ la fonction dérivée de $g$. Pour tout nombre réel $x$, on a :

\begin{center}
\begin{tabularx}{\linewidth}{|*{4}{>{\centering \arraybackslash}X|}}\hline
A&B&C&D\\ \hline
$g'(x) = 2x\e^{-2x}(1 - x)$& $g'(x) = -4x\e^{-2x}$&$g'(x) = 2x\e^{-2x}(1 + x)$&$g'(x) = - 2x^2\e^{-2x}$\\ \hline
\end{tabularx}
\end{center}

\bigskip

\textbf{Question 3}

\medskip

Le plan complexe est rapporté à un repère orthonormé \Ouv. On désigne par i le nombre complexe de module 1 et d'argument $\dfrac{\pi}{2}$.

Soient les points A et B d'affixes respectives $z_{\text{A}} = 2\e^{\text{i}\frac{\pi}{3}}$ et $z_{\text{B}} = -\sqrt 3 + \text{i}$.

Donner la forme algébrique de $z_{\text{A}}$ ainsi que la forme exponentielle de $z_{\text{B}}$.

\bigskip

\textbf{Question 4}

\medskip

En faisant apparaître les étapes de calcul, calculer :
\[\displaystyle\int_0^{\frac{\pi}{2}} \cos (2x) \:\text{d}x.\]

\bigskip


