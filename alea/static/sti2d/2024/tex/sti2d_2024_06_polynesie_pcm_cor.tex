
\bigskip

\begin{enumerate}[start=2]
\item La fonction proposée est :
$y = f(t) = -14\e^{-\frac{1}{90}t} + 21$.

Sa dérivée est :
$y' = f'(t) = -14 \times \left(-\dfrac{1}{90}\right)\e^{-\frac{1}{90}t} + 0 = \dfrac{14}{90}\e^{-\frac{1}{90}t}$.

Substituons $y$ dans l'équation différentielle :
\begin{align*}
y' &= -\dfrac{1}{90}y + \dfrac{7}{30} \\
&= -\dfrac{1}{90}\left(-14\e^{-\dfrac{1}{90}t} + 21\right) + \dfrac{7}{30} \\
&= -\dfrac{1}{90}\left(-14\e^{-\frac{1}{90}t}\right) - \dfrac{21}{90} + \dfrac{7}{30} \\
&= \dfrac{14}{90}\e^{-\frac{1}{90}t}.
\end{align*}

Le membre de droite correspond à la valeur de $f'(t)$. Ainsi, $f(t) = -14\e^{-\frac{1}{90}t} + 21$ est une solution de l'équation différentielle.

\item Lorsque $t \to +\infty$, le terme exponentiel $\e^{-\frac{1}{90}t}$ tend vers 0.

Donc : $\displaystyle\lim_{t \to +\infty} f(t) = -14 \times 0 + 21 = 21.$

Cela signifie que la température du contenu de la tasse tendra vers $21\degres$C, la température de l'air ambiant.

\item Déterminer le temps $t$ pour lequel $f(t) = 20$.
\begin{align*}
&-14\e^{-\frac{1}{90}t} + 21 = 20 \\
\iff &-14\e^{-\frac{1}{90}t} = 20 - 21 \\
\iff &-14\e^{-\frac{1}{90}t} = -1 \\
\iff &\e^{-\frac{1}{90}t} = \dfrac{1}{14} \\
\iff &-\dfrac{1}{90}t = \ln\left(\dfrac{1}{14}\right) \\
\iff &t = -90 \ln\left(\frac{1}{14}\right) \\
\iff &t = 90 \ln(14) \\
\iff &t \approx 238.
\end{align*}
Le temps pour que la boisson atteigne $20\degres$C est d’environ $238$ minutes.
\end{enumerate}

\bigskip


