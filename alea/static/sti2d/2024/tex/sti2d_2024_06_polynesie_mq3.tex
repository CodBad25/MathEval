
\medskip

Un filtre dans un circuit électrique permet de transmettre sélectivement certaines composantes du spectre en fréquence d'un signal.

On considère un filtre composé d'une résistance et d'un condensateur.

On appelle fonction de transfert de ce filtre, la fonction $H(\omega_{\text{C}})$ où $\omega_{\text{C}}$ est la pulsation de coupure du filtre.

La réponse en gain du circuit, notée $G_{dB}$ et exprimée en décibel, vaut pour cette fréquence de coupure :
\[G_{dB} = 20 \log \left(\left|H\left(\omega_{c}\right)\right|\right),\]
où $\left|H\left(\omega_{c}\right)\right| = \dfrac{\sqrt{2}}{2}$ est le module de $H\left(\omega_{c}\right)$.

Montrer que $G_{dB} = -10 \log (2)$.

\bigskip

