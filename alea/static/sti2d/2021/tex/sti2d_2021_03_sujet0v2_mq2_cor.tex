
\medskip

\begin{enumerate}
\item Forme exponentielle de $z_2$ :

\begin{list}{\textbullet}{}
\item $\left | z_2 \right | = \left | -2-2\text{i} \right | = \ds\sqrt{(-2)^2+(-2)^2} = \ds\sqrt{8} = 2\sqrt{2}$

\item $z_2 = - 2 - 2\text{i} = 2\sqrt{2}\left ( \dfrac{-2}{2\sqrt{2}} - \dfrac{2}{2\sqrt{2}}\text{i}\right )
= 2\sqrt{2} \left ( -\dfrac{\sqrt{2}}{2} - \dfrac{\sqrt{2}}{2}\text{i} \right )$

On cherche $\theta$ tel que $\cos \theta = -\dfrac{\sqrt{2}}{2}$ et $\sin \theta = - \dfrac{\sqrt{2}}{2}$.

Une valeur de $\theta$ est $-\dfrac{3\pi}{4}$.
\end{list}

$z_2$ a pour forme exponentielle $2\sqrt{2} \e^{-\frac{3\pi}{4}\text{i}}$.

\item 
$z_2^4 = \left (2\sqrt{2} \e^{-\frac{3\pi}{4}\text{i}}\right )^4
= \left (2\sqrt{2}\right )^4 \e^{4\times\left (-\frac{3\pi}{4}\text{i}\right )}
= 64 \e^{-3\pi\text{i}}$

$\e^{-3\pi\text{i}} = \e^{\pi\text{i}}=-1$ donc $z_2^4=-64$ et donc $z_2^4$ est un nombre réel.
\end{enumerate}

\bigskip

