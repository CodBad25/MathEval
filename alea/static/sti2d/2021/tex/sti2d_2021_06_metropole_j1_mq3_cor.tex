
\medskip

\begin{enumerate}
\item Pour tout $x$ appartenant à $]0~;~+ \infty[$, $f'(x) = 2 -0 - \dfrac{1}{x} = \dfrac{2x-1}{x}$.
 
\item 
\begin{list}{\textbullet}{}
\item Sur l'intervalle $]0~;~+ \infty[$, $x>0$ donc $f'(x)$ est du signe de $2x-1$ qui s'annule et change de signe pour $x=0,5$.

\item $f(0,5) = 2\times 0,5 - 1 - \ln(0,5) = - \ln\left ( \frac{1}{2}\right ) = \ln(2)$

\item $\ds\lim_{x\to 0} (2x-1) = -1$ et $\ds\lim_{x\to 0 \atop x>0} \ln(x)=-\infty$ donc
$\ds\lim_{x\to 0\atop x>0} f(x)=+\infty$

\item Sur l'intervalle $]0~;~+ \infty[$, $x\neq 0$ donc on a $f(x)=2x-1-\ln(x) = x\left (2-\dfrac{\ln(x)}{x}\right )-1$.

$\ds\lim_{x\to +\infty} \dfrac{\ln(x)}{x}=0$ donc $\ds\lim_{x\to +\infty}\left (2- \dfrac{\ln(x)}{x}\right )=2$ et donc $\ds\lim_{x\to +\infty} f(x)=+\infty$ 

On dresse le tableau de variations de la fonction $f$ sur $]0~;~+ \infty[$:

\begin{center}
{\renewcommand{\arraystretch}{1.3}
\psset{nodesep=3pt,arrowsize=2pt 3}  % paramètres
\def\esp{\hspace*{1.5cm}}% pour modifier la largeur du tableau
\def\hauteur{0pt}% mettre au moins 20pt pour augmenter la hauteur
$\begin{array}{|c| l *3{c} c|}
\hline
 x & 0 & \esp & 0,5 & \esp & +\infty \\
  \hline
2x-1 &  &  \pmb{-} & \vline\hspace{-2.7pt}0 & \pmb{+} & \\  
\hline
f'(x) & \vline\;\vline &  \pmb{-} & \vline\hspace{-2.7pt}0 & \pmb{+} & \\  
\hline
  & \vline\;\vline \Rnode{max1}{\: +\infty}  &  &  &  & \Rnode{max2}{+\infty}   \\
f(x) &\vline\;\vline &  & & &  \rule{0pt}{\hauteur} \\
 & \vline\;\vline & &   \Rnode{min}{\ln(2)} & & \rule{0pt}{\hauteur}
\ncline{->}{max1}{min} \ncline{->}{min}{max2}\\
\hline
\end{array}$
}
\end{center}	
\end{list}
\end{enumerate}

\bigskip

