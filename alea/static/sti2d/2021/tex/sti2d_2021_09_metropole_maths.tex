
\medskip

\textbf{Dans cet exercice, seulement 4 questions au choix parmi les 6 questions proposées sont à traiter. Toutes ces questions sont indépendantes les unes des autres.}

\medskip

\emph{Pour chaque question, préciser si l'affirmation est vraie ou fausse et justifier la réponse choisie.}

\bigskip

\textbf{Question 1}

\medskip

Soit $f$ la fonction définie sur $\R$ par $f(x) = x^2 \e^x$.

\smallskip

\begin{tabularx}{\linewidth}{|X|}\hline
Affirmation 1 :\\
\og La fonction $f$ est croissante sur $\R$ .\fg\\ \hline
\end{tabularx}

\bigskip

\textbf{Question 2}

\medskip

On considère la fonction $h$ définie sur $]0~;~+\infty[$ par $h(x) = \ln (2x + 1)$.

On désigne par $\mathcal{C}_h$ sa courbe représentative dans un repère orthonormé d'origine O et d'unité graphique $1$~cm.

On note $M(x~;~y)$ un point de la courbe $\mathcal{C}_h$. On suppose que l'ordonnée $y$ du point $M$
est supérieure à $15$~cm.

\smallskip

\begin{tabularx}{\linewidth}{|X|}\hline
Affirmation 2 :\\
\og L'abscisse $x$ du point $M$ se situe à plus de $16$~km du point O. \fg\\ \hline
\end{tabularx}

\bigskip

\textbf{Question 3}

\medskip

Le thorium 231 est un élément radioactif qui se désintègre selon la loi:

$N(t) = N(0)\e^{- 0,027 t}$ où $N(0)$ est le nombre de noyaux au début de l'observation et
$N(t)$ le nombre de noyaux à l'instant $t$ exprimé en heure. 

La demi-vie d'un élément radioactif est le temps au bout duquel la moitié de ses noyaux se sont désintégrés.

\smallskip

\begin{tabularx}{\linewidth}{|X|}\hline
Affirmation 3 :\\
\og La demi-vie du thorium 231 est d'environ 11 heures. \fg\\ \hline
\end{tabularx}

\bigskip

\textbf{Question 4}

\medskip

Soit $f$ la fonction définie sur $\R$ par $f(t) = \cos(t) + 2 \sin(t)$.

On considère l'équation différentielle $(E)\,  : y'' + y = 0$.

\smallskip

\begin{tabularx}{\linewidth}{|X|}\hline
Affirmation 4 :\\
\og La fonction $f$ est solution sur $\R$ de l'équation différentielle $(E)$ et vérifie les
conditions initiales $y(0) = 1$ et $y'(0) = 2$. \fg\\ \hline
\end{tabularx}

\bigskip

\textbf{Question 5}

\medskip

On considère le nombre complexe $z = \dfrac{2 - \text{i}}{1 - 3\text{i}}$.

\smallskip

\begin{tabularx}{\linewidth}{|X|}\hline
Affirmation 5 :\\
\og Le nombre complexe $z^4$ est un nombre réel négatif. \fg\\ \hline
\end{tabularx}

\bigskip

\textbf{Question 6}

\medskip

Le plan complexe est muni d'un repère orthonormé direct \Ouv. On considère
les points A, B et C d'affixes respectives:

\[z_{\text{A}} = - 1 + \text{i},\quad z_{\text{B}} = 4 + 2\text{i}\quad \text{et}\quad z_{\text{C}} = - 4\text{i}.\]

\smallskip

\begin{tabularx}{\linewidth}{|X|}\hline
Affirmation 6 :\\
\og Le triangle ABC est rectangle et isocèle. \fg\\ \hline
\end{tabularx}

\bigskip


