
\bigskip

\textbf{Question 1}

\medskip

\begin{enumerate}
\item $\text B \,(1~;~1)\in \mathcal{C}_f$ donc $f(1)=1$.
\[f(1)=1 \iff a+b-\ln(1)=1 \iff a+b=1.\]

\item La courbe admet en A une tangente parallèle à l'axe des abscisses donc de coefficient directeur égal à 0; on a donc $f'(x_{\text A})=0$ c'est-à-dire $f'(0,5)=0$.
\[f'(x)=a-\dfrac{1}{x} \text{ ; } f'(0,5)=0 \iff a-\dfrac{1}{0,5}=0 \iff a-2=0 \iff a=2.\]

\item $a+b=1$ et $a=2$ donc $b=-1$; on en déduit que $f(x)=2x-1-\ln(x)$. 
\end{enumerate}

\bigskip

\textbf{Question 2}

\medskip

La machine aura perdu la moitié de sa valeur pour $t$ tel que :
\begin{align*}
&f(t)=\np{150000} \\
\iff &\np{300000}\left(1 - \e^{- 0,09 t}\right) = \np{150000} \\
\iff &1 - \e^{- 0,09 t} = 0,5 \\
\iff &0,5 = \e^{-0,09t} \\
\iff &\ln(0,5)=-0,09t \\
\iff &t=-\dfrac{\ln(0,5)}{0,09} \\
\iff &t\approx 7,7,
\end{align*}
donc au bout de 8 ans, la machine aura perdu plus de la moitié de sa valeur.

\bigskip

\textbf{Question 3}

\medskip

\begin{enumerate}
\item Pour tout $x$ appartenant à $]0~;~+ \infty[$, $f'(x) = 2 -0 - \dfrac{1}{x} = \dfrac{2x-1}{x}$.
 
\item 
\begin{list}{\textbullet}{}
\item Sur l'intervalle $]0~;~+ \infty[$, $x>0$ donc $f'(x)$ est du signe de $2x-1$ qui s'annule et change de signe pour $x=0,5$.

\item $f(0,5) = 2\times 0,5 - 1 - \ln(0,5) = - \ln\left ( \frac{1}{2}\right ) = \ln(2)$

\item $\ds\lim_{x\to 0} (2x-1) = -1$ et $\ds\lim_{x\to 0 \atop x>0} \ln(x)=-\infty$ donc
$\ds\lim_{x\to 0\atop x>0} f(x)=+\infty$

\item Sur l'intervalle $]0~;~+ \infty[$, $x\neq 0$ donc on a $f(x)=2x-1-\ln(x) = x\left (2-\dfrac{\ln(x)}{x}\right )-1$.

$\ds\lim_{x\to +\infty} \dfrac{\ln(x)}{x}=0$ donc $\ds\lim_{x\to +\infty}\left (2- \dfrac{\ln(x)}{x}\right )=2$ et donc $\ds\lim_{x\to +\infty} f(x)=+\infty$ 

On dresse le tableau de variations de la fonction $f$ sur $]0~;~+ \infty[$:

\begin{center}
{\renewcommand{\arraystretch}{1.3}
\psset{nodesep=3pt,arrowsize=2pt 3}  % paramètres
\def\esp{\hspace*{1.5cm}}% pour modifier la largeur du tableau
\def\hauteur{0pt}% mettre au moins 20pt pour augmenter la hauteur
$\begin{array}{|c| l *3{c} c|}
\hline
 x & 0 & \esp & 0,5 & \esp & +\infty \\
  \hline
2x-1 &  &  \pmb{-} & \vline\hspace{-2.7pt}0 & \pmb{+} & \\  
\hline
f'(x) & \vline\;\vline &  \pmb{-} & \vline\hspace{-2.7pt}0 & \pmb{+} & \\  
\hline
  & \vline\;\vline \Rnode{max1}{\: +\infty}  &  &  &  & \Rnode{max2}{+\infty}   \\
f(x) &\vline\;\vline &  & & &  \rule{0pt}{\hauteur} \\
 & \vline\;\vline & &   \Rnode{min}{\ln(2)} & & \rule{0pt}{\hauteur}
\ncline{->}{max1}{min} \ncline{->}{min}{max2}\\
\hline
\end{array}$
}
\end{center}	
\end{list}
\end{enumerate}

\bigskip

\textbf{Question 4}

\medskip

\begin{enumerate}
\item D'après le cours, l'équation différentielle $a y' + b y = 0$ pour $a\neq 0$ a pour solutions les fonctions $f$ définies sur $\R$ par $f(x)=k\e^{-\frac{b}{a}x}$ avec $k\in\R$.

Donc l'équation différentielle $(E)$ a pour solution dans $[0~;+ \infty[$ les fonctions $P$ définies par $P(x)=k \e^{-0,0434 x}$ avec $k\in\R$. 

$P(0) = 6,75 \iff k\e^{0}= 6,75 \iff k=6,75$.

Sur $[0~;+ \infty[$ la solution $P$ de cette équation différentielle qui vérifie la condition initiale $P(0) = 6,75$ est définie par $P(x)=6,75\e^{-0,0434x}$.

\item La puissance du signal au bout de 1~km est $P(1)\approx \np{6,4633}$, donc la perte de puissance une fois que le signal a parcouru 1~km depuis l'entrée est, en mW, $P(0)-P(1) \approx \np{0,2867}$ soit environ $287~ \mu$W.
\end{enumerate}

\bigskip

\textbf{Question 5}

\medskip

\begin{enumerate}
\item Pour tout $x$ appartenant à $\R$,\\
  $F'(x) = \left (2x+3\right )\times \e^{x} +  \left(x^2 + 3x + 1\right) \times \e^{x}
  = \left (2x+3 + x^2 +3x+1\right ) \e^{x}
  = \left (x^2+5x+4\right ) \e^{x}\\
\phantom{F'(x)} 
 = f(x)$.
  
Donc la fonction $F$ est une primitive de la fonction $f$ sur $\R$.
  
\item $\displaystyle\int_0^1 f(x)\d x
= \left [ F(x)\strut\right ]_{0}^{1} 
= F(1)-F(0)
= \left ( (1+3+1)\e^{1}\right ) - \left ( (0+0+1)\e^{0} \right )
= 5\e - 1$

\end{enumerate}

\bigskip

\textbf{Question 6}

\medskip

\begin{enumerate}
\item Pour tout $t$ appartenant à $[0~;~+ \infty[$,

$240 \sqrt{2} \cos \left(50t + \dfrac{\pi}{4}\right)
= 240 \sqrt{2} 
\left (
 \cos \left (50t\right ) \cos \left (\frac{\pi}{4}\right )
  -
   \sin\left (50t\right ) \sin \left (\frac{\pi}{4}\right )
   \right )\\
   \phantom{240 \sqrt{2} \cos \left(50t + \dfrac{\pi}{4}\right)}
= 240 \sqrt{2} \left ( \cos \left (50t\right ) \times \dfrac{\sqrt{2}}{2}  - \sin\left (50t\right ) \times \dfrac{\sqrt{2}}{2}\right )\\
   \phantom{240 \sqrt{2} \cos \left(50t + \dfrac{\pi}{4}\right)}
=  240 \cos (50t) - 240 \sin (50t) = u(t)
$

\item On déduit que la pulsation  $\omega$ vaut $50$ et donc que la fréquence $f$ est égale à 
$\dfrac{\omega}{2\pi}=\dfrac{50}{2\pi}\approx 8$~Hz.
\end{enumerate}

\bigskip


