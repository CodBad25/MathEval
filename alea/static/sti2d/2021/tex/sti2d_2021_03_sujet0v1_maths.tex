
\medskip

\textbf{Les questions 1, 2, 3 et 4 sont indépendantes les unes des autres.}

\bigskip

\textbf{Question 1}

\medskip

\begin{enumerate}
\item On considère l'équation différentielle :
\[(E) : \quad  y'+ 100y = 8.\]

Déterminer la solution $v$ définie sur $[0~;~ +\infty[$ de cette équation différentielle, qui vérifie la condition initiale $v(0) = 0$.
\item La fonction $v$ déterminée à la question précédente modélise la vitesse (exprimée en
m.s$^{-1}$) de chute d'une bille dans un liquide visqueux en fonction du temps $t$ écoulé depuis le début de la chute (exprimé en s).

Déterminer la vitesse, arrondie à $0,001$ m.s$^{-1}$, de la bille après $0,01$ seconde de chute.
\end{enumerate}

\bigskip

\textbf{Question 2}

\medskip

\begin{center}
\fbox{
\begin{minipage}{0.9\textwidth}
\textbf{Rappel :} pour $a$ et $b$ deux réels, on a les formules suivantes :
\[\bullet~~\cos(a + b) = \cos (a) \cos(b) - \sin(a) \sin(b)\]
\[\bullet~~\cos(a - b) = \cos (a) \cos(b) + \sin(a) \sin(b)\]
\[\bullet~~\sin(a + b) = \sin(a) \cos(b) + \cos (a) \sin(b)\]
\[\bullet~~\sin(a - b) = \sin(a) \cos(b) - \cos (a) \sin(b)\]
\end{minipage}}
\end{center}

La tension $u$ (exprimée en volt) aux bornes d'un dipôle en fonction du temps $t$ (exprimé en
seconde) est donnée par :

\[u(t) = \dfrac{7\sqrt{3}}{4}\cos (100t) - \dfrac{7}{4}\sin (100t).\]

\begin{enumerate}
\item Transformer l'écriture de $u$ sous la forme $u(t) = U_{\text{max}}\cos(\omega t + \varphi)$ où :

\setlength\parindent{1cm}
\begin{itemize}
\item[$\bullet~~$]$U_{\text{max}}$ représente la tension maximale (exprimée en V) ;
\item[$\bullet~~$]$\omega$ représente la pulsation (exprimée en rad$\cdot \text{s}^{-1}$) ;
\item[$\bullet~~$]$\varphi$ représente le déphasage (exprimé en rad).
\end{itemize}
\setlength\parindent{0cm}

\item En déduire la valeur du déphasage $\varphi$ de $u(t)$.
\end{enumerate}

\bigskip

\textbf{Question 3}

\medskip

On considère les deux fonctions $f$ et $g$ définies et continues sur [0~;~9] respectivement par:

\[f(x) = x^2 - 2x + 4  \quad \text{et} \quad g(x) = 7x+4.\]

Les représentations graphiques des deux fonctions sont données ci-dessous.

\begin{center}
\psset{unit=1cm,yunit=0.1cm}
\begin{pspicture}(-1,-1)(10,75)
\psaxes[linewidth=1.25pt,Dy=5,labelFontSize=\scriptstyle]{->}(0,0)(0,0)(10,75)[$x$,-90][$y$,180]

\pscustom[fillstyle=solid,fillcolor=lightgray]{
\psplot[plotpoints=2000,linewidth=1.25pt,linecolor=red]{0}{9}{x dup mul x 2 mul sub 4 add}
\psplot[plotpoints=2000,linewidth=1.25pt,linecolor=blue]{9}{0}{x 7 mul  4 add}}
\psplot[plotpoints=2000,linewidth=1.25pt,linecolor=red]{0}{9}{x dup mul x 2 mul sub 4 add}
\psplot[plotpoints=2000,linewidth=1.25pt,linecolor=blue]{0}{9}{x 7 mul  4 add}
\uput[dr](8.5,56){\red $\mathcal{C}_f$}
\uput[ul](8.5,63){\blue $\mathcal{C}_g$}
\end{pspicture}
\end{center}

Déterminer la valeur exacte de l'aire, exprimée en unité d'aire, située entre les courbes représentatives de ces deux fonctions.

\bigskip

\textbf{Question 4}

\medskip

La tension $u_c(t)$ (exprimée en voit), aux bornes d'un condensateur lors de sa charge, est modélisée par:

\[u_c(t) = E\left(1 - \e^{-\frac{t}{RC}}\right) \, \text{ où }\,  t\,  \text{désigne le temps, exprimé en seconde.}\]

Les caractéristiques du condensateur utilisé sont :

\setlength\parindent{1cm}
\begin{itemize}
\item[$\bullet~~$] Tension maximale : $E = 4$ V 
\item[$\bullet~~$] Résistance : $R = 10^3$~$\Omega$ 
\item[$\bullet~~$] Capacité : $C = 2 \times 10^{-3}$ F
\end{itemize}
\setlength\parindent{0cm}

Déterminer le temps de charge $t$ (exprimé en seconde, arrondi à $0,1$~s près) nécessaire pour obtenir une tension aux bornes du condensateur égale à la moitié de sa tension maximale.

\bigskip


