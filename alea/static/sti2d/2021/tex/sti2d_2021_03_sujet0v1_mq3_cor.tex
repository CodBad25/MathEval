
\medskip

$f(0)=g(0)=4$ et $f(9)=g(9)=67$ et sur l'intervalle $[0~;~9]$, la courbe $\mathcal{C}_g$ est au-dessus de la courbe $\mathcal{C}_f$ ; donc la valeur  de l'aire, exprimée en unité d'aire, située entre les courbes représentatives de ces deux fonctions est :
$\ds\int_{0}^{9} \left ( g(x)-f(x)\right ) \d x$.

\[g(x)-f(x) = \left (7x+4\right ) - \left ( x^2-2	x+4\right ) = 7x+4-x^2+2x-4 = -x^2+9x.\]

On cherche une fonction $H$ primitive de $g-f$: la fonction définie par $H(x)=-\dfrac{x^3}{3}+9\dfrac{x^2}{2}$ convient.

\[\ds\int_{0}^{9} \left ( g(x)-f(x)\right ) \d x
= \left [ H(x) \strut\right ]_{0}^{9}
= H(9)-H(0)
= \left (-\dfrac{9^3}{3}+9\dfrac{9^2}{2} \right ) - \left ( -\dfrac{0^3}{3}+9\dfrac{0^2}{2} \right )
= \dfrac{243}{2}.\]

La valeur  de l'aire située entre les courbes représentatives de ces deux fonctions est $121,5$~U.A.

\bigskip

