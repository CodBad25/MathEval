
\medskip

On considère la fonction $f$ définie sur l'intervalle $[0,5~;~10]$ par :
$f(x) = x^2 - x - 2- 3 \ln (x).$

On note $f'$ la fonction dérivée de $f$.

\begin{enumerate}
\item Pour tout $x$ appartenant à l'intervalle $[0,5~;~10]$ :
$f'(x)=2x-1 -0 - 3\dfrac{1}{x} = \dfrac{2x^2 - x -3}{x}$

Or $(x+1)(2x-3)=2x^2 + 2x -3x - 3 = 2x^2 -x-3$ donc $f'(x) = \dfrac{(x + 1)(2x - 3)}{x}$.

\item On va déterminer les variations de la fonction $f$ et, pour cela, étudier le signe de $f'(x)$.

\begin{center}
{\renewcommand{\arraystretch}{1.3}
\psset{nodesep=3pt,arrowsize=2pt 3}  % paramètres
\def\esp{\hspace*{1.5cm}}% pour modifier la largeur du tableau
\def\hauteur{0pt}% mettre au moins 20pt pour augmenter la hauteur
$\begin{array}{|c| *4{c} c|}
\hline
 x & 0,5 & \esp & 1,5 & \esp & +\infty \\
 \hline
x+1 &  &  \pmb{+} & \vline\hspace{-2.7pt}0 & \pmb{+} & \\  
 \hline
2x-3 &  &  \pmb{-} & \vline\hspace{-2.7pt}0 & \pmb{+} & \\  
 \hline
x &  &  \pmb{+} & \vline\hspace{-2.7pt}0 & \pmb{+} & \\  
 \hline
f'(x) &  &  \pmb{-} & \vline\hspace{-2.7pt}0 & \pmb{+} & \\  
\hline
  & \Rnode{max1}{\phantom A}  &  &  &  & \Rnode{max2}{\phantom C}   \\
f(x) & &  & & &  \rule{0pt}{\hauteur} \\
 &  & &   \Rnode{min}{\phantom B} & & \rule{0pt}{\hauteur}
\ncline{->}{max1}{min} \ncline{->}{min}{max2}\\
\hline
\end{array}$
}
\end{center}

La fonction $f$ admet donc un minimum en $x=1,5$ qui vaut :
\[f(1,5) = 1,5^2 - 1,5 - 2 -3\ln(1,5) = -1,25-3\ln(1,5) \approx -2,47.\]
\end{enumerate}

\bigskip

