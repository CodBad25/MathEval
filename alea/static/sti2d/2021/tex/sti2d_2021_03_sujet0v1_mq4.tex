
\medskip

La tension $u_c(t)$ (exprimée en voit), aux bornes d'un condensateur lors de sa charge, est modélisée par:

\[u_c(t) = E\left(1 - \e^{-\frac{t}{RC}}\right) \, \text{ où }\,  t\,  \text{désigne le temps, exprimé en seconde.}\]

Les caractéristiques du condensateur utilisé sont :

\setlength\parindent{1cm}
\begin{itemize}
\item[$\bullet~~$] Tension maximale : $E = 4$ V 
\item[$\bullet~~$] Résistance : $R = 10^3$~$\Omega$ 
\item[$\bullet~~$] Capacité : $C = 2 \times 10^{-3}$ F
\end{itemize}
\setlength\parindent{0cm}

Déterminer le temps de charge $t$ (exprimé en seconde, arrondi à $0,1$~s près) nécessaire pour obtenir une tension aux bornes du condensateur égale à la moitié de sa tension maximale.

