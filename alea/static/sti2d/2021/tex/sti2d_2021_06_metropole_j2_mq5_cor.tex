
\medskip

\parbox{0.71\linewidth}{

\begin{enumerate}
\item $\mathcal A = \displaystyle\int_1^2 f(x)\: \text{d}x$ est l'aire de la région du plan comprise entre la courbe $\mathcal{C}_f$, l'axe des abscisses et les droites d'équations $x=1$ et $x=2$.

$\mathcal A$ est compris entre l'aire $\mathcal A_1$ du trapèze ABDC et l'aire $\mathcal A_2$ du trapèze ABFE.

$\mathcal A_1 = \dfrac{(\text{AC} + \text{BD}) \times \text{AB}}{2} = \dfrac{(2+8)\times 1}{2}=5$

$\mathcal A_2 = \dfrac{(\text{AE} + \text{BF}) \times \text{AB}}{2} = \dfrac{(3+9)\times 1}{2}=6$

Donc : $5\leqslant  \displaystyle\int_1^2 f(x)\: \text{d}x \leqslant 6$.

\item Une primitive de la fonction $f$ sur $[1~;~2]$ :
\begin{list}{\textbullet}{}
\item La fonction $x\longmapsto x$, a pour primitive : $\dfrac{x^2}{2}$.
\item La fonction $x\longmapsto \e^{x}$, a pour primitive : $\e^{x}$.
\item La fonction $x\longmapsto \dfrac{1}{x}$, a pour primitive : $\ln(x)$.
\end{list}
Donc la fonction $f$ a pour primitive la fonction $F$ définie par :
\[F(x) = \dfrac{x^2}{2} + \e^{x}-\ln(x).\]
\end{enumerate}}
\hfill
\parbox{0.26\linewidth}{
\psset{unit=1cm}
\begin{pspicture*}(-0.5,-0.7)(3,9.2)
\psaxes[linewidth=1.25pt,labelFontSize=\scriptstyle]{->}(0,0)(0,0)(3,9.2)
\def\f{2.71828 x exp x add 1 x div sub}
\pscustom[fillstyle=solid,fillcolor=blue!10]
{\psplot[plotpoints=4000]{1}{2}{\f}
\psplot[plotpoints=4000]{2}{1}{0}
\closepath
}
\psgrid[gridlabels=0pt,subgriddiv=1,gridwidth=0.1pt]
\psplot[plotpoints=2000,linewidth=1.25pt,linecolor=blue]{1}{2}{\f}
\psline[linestyle=dashed,linewidth=1.25pt](1,3)(2,9)
\psline[linestyle=dashed,linewidth=1.25pt](1,2)(2,8)
\psdots(1,0)(2,0)(2,8)(1,2)(2,9)(1,3)
\uput[ul](1,0){A}\uput[ur](2,0){B}\uput[dl](1,2){C}
\uput[ur](2,8){D}\uput[ul](1,3){E}\uput[l](2,9){F}
\uput[l](1,2.75){\blue $\mathcal{C}_f$}
\end{pspicture*}}

\begin{align*}
\displaystyle\int_1^2 f(x)\: \text{d}x &= \left [ F(x)\strut\right ]_{1}^{2} = F(2)-F(1) \\
&= \left (\dfrac{4}{2} + \e^{2} - \ln(2)  \right ) - \left ( \dfrac{1}{2} +\e^{1} - \ln(1) \right ) \\
&= 2+\e^{2}-\ln(2) -\dfrac{1}{2} - \e \\
&= \dfrac{3}{2}+ \e^{2}-\ln(2) - \e
\end{align*}

\begin{center}
\textbf{Remarque :} $ \dfrac{3}{2}+ \e^{2}-\ln(2) - \e \approx 5,48$ qui est compris entre 5 et 6.
\end{center}

\bigskip

