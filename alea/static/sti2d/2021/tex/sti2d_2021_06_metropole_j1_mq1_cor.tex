
\medskip

\begin{enumerate}
\item $\text B \,(1~;~1)\in \mathcal{C}_f$ donc $f(1)=1$.
\[f(1)=1 \iff a+b-\ln(1)=1 \iff a+b=1.\]

\item La courbe admet en A une tangente parallèle à l'axe des abscisses donc de coefficient directeur égal à 0; on a donc $f'(x_{\text A})=0$ c'est-à-dire $f'(0,5)=0$.
\[f'(x)=a-\dfrac{1}{x} \text{ ; } f'(0,5)=0 \iff a-\dfrac{1}{0,5}=0 \iff a-2=0 \iff a=2.\]

\item $a+b=1$ et $a=2$ donc $b=-1$; on en déduit que $f(x)=2x-1-\ln(x)$. 
\end{enumerate}

\bigskip

