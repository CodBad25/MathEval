
\medskip

\begin{enumerate}
\item D'après le cours on sait que les solutions de l'équation différentielle $y'+ay=b$ (avec $a\neq 0$), sont les fonctions $v$ définies sur $\R$ par: $v(t)=k \e^{-at} + \dfrac{b}{a}$ où $k\in\R$.

Les solutions de l'équation différentielle $y'+100y=8$ sont donc les fonctions $v$ définies par: $v(t)=k \e^{-100t} + \dfrac{8}{100}$ où $k\in\R$.

$v(0)=0 \iff k \e^{0} + \dfrac{8}{100}=0 \iff k=-\dfrac{8}{100}$

La solution $v$ définie sur $[0~;~ +\infty[$ de cette équation différentielle, qui vérifie la condition initiale $v(0) = 0$, est définie par $v(t)=0,08 -0,08\e^{-100t}$.

\item $v(0,01) =0,08 - 0,08\e^{-1}\approx \np{0,05057}$

La vitesse de la bille après $0,01$ seconde de chute est de $0,051$~m.s$^{-1}$. 
\end{enumerate}

\bigskip

