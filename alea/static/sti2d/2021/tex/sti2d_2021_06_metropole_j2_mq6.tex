
\medskip

\begin{center}
\fbox{
\begin{minipage}{0.9\textwidth}
\textbf{Rappel :} pour $a$ et $b$ deux réels, on a les formules suivantes :
\[\bullet~~\cos(a + b) = \cos (a) \cos(b) - \sin(a) \sin(b)\]
\[\bullet~~\cos(a - b) = \cos (a) \cos(b) + \sin(a) \sin(b)\]
\[\bullet~~\sin(a + b) = \sin(a) \cos(b) + \cos (a) \sin(b)\]
\[\bullet~~\sin(a - b) = \sin(a) \cos(b) - \cos (a) \sin(b)\]
\end{minipage}}
\end{center}

La tension $u$ aux bornes d'un générateur, exprimée en volt, dépendant du temps $t$, exprimé en seconde, est donnée à l'instant $t$ par : 

\[u(t) = 120 \cos(70t) - 120 \sin(70t).\]

\begin{enumerate}
\item Montrer que, pour tout $t$ de l'intervalle $[0~;~+\infty[$, \, $u(t) = 120\sqrt{2} \cos \left(70t + \dfrac{\pi}{4}\right)$.

\item En déduire la fréquence $f = \dfrac{\omega}{2\pi}$, exprimée en Hz, délivrée par le générateur, où $\omega$ désigne la pulsation. 
On arrondira le résultat à l'unité.
\end{enumerate}

