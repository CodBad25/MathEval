
\medskip

On considère les deux fonctions $f$ et $g$ définies et continues sur [0~;~9] respectivement par:

\[f(x) = x^2 - 2x + 4  \quad \text{et} \quad g(x) = 7x+4.\]

Les représentations graphiques des deux fonctions sont données ci-dessous.

\begin{center}
\psset{unit=1cm,yunit=0.1cm}
\begin{pspicture}(-1,-1)(10,75)
\psaxes[linewidth=1.25pt,Dy=5,labelFontSize=\scriptstyle]{->}(0,0)(0,0)(10,75)[$x$,-90][$y$,180]

\pscustom[fillstyle=solid,fillcolor=lightgray]{
\psplot[plotpoints=2000,linewidth=1.25pt,linecolor=red]{0}{9}{x dup mul x 2 mul sub 4 add}
\psplot[plotpoints=2000,linewidth=1.25pt,linecolor=blue]{9}{0}{x 7 mul  4 add}}
\psplot[plotpoints=2000,linewidth=1.25pt,linecolor=red]{0}{9}{x dup mul x 2 mul sub 4 add}
\psplot[plotpoints=2000,linewidth=1.25pt,linecolor=blue]{0}{9}{x 7 mul  4 add}
\uput[dr](8.5,56){\red $\mathcal{C}_f$}
\uput[ul](8.5,63){\blue $\mathcal{C}_g$}
\end{pspicture}
\end{center}

Déterminer la valeur exacte de l'aire, exprimée en unité d'aire, située entre les courbes représentatives de ces deux fonctions.

\bigskip

