
\medskip

\begin{center}
\fbox{
\begin{minipage}{0.9\textwidth}
\textbf{Rappel :} pour $a$ et $b$ deux réels, on a les formules suivantes :
\[\bullet~~\cos(a + b) = \cos (a) \cos(b) - \sin(a) \sin(b)\]
\[\bullet~~\cos(a - b) = \cos (a) \cos(b) + \sin(a) \sin(b)\]
\[\bullet~~\sin(a + b) = \sin(a) \cos(b) + \cos (a) \sin(b)\]
\[\bullet~~\sin(a - b) = \sin(a) \cos(b) - \cos (a) \sin(b)\]
\end{minipage}}
\end{center}

La tension $u$ (exprimée en volt) aux bornes d'un dipôle en fonction du temps $t$ (exprimé en
seconde) est donnée par :

\[u(t) = \dfrac{7\sqrt{3}}{4}\cos (100t) - \dfrac{7}{4}\sin (100t).\]

\begin{enumerate}
\item Transformer l'écriture de $u$ sous la forme $u(t) = U_{\text{max}}\cos(\omega t + \varphi)$ où :

\setlength\parindent{1cm}
\begin{itemize}
\item[$\bullet~~$]$U_{\text{max}}$ représente la tension maximale (exprimée en V) ;
\item[$\bullet~~$]$\omega$ représente la pulsation (exprimée en rad$\cdot \text{s}^{-1}$) ;
\item[$\bullet~~$]$\varphi$ représente le déphasage (exprimé en rad).
\end{itemize}
\setlength\parindent{0cm}

\item En déduire la valeur du déphasage $\varphi$ de $u(t)$.
\end{enumerate}

\bigskip

