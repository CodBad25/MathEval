
\medskip

\parbox{0.48\linewidth}{On considère la fonction $f$ définie sur l'intervalle
[1~;~2] par:
\[f(x) = x + \e^x - \dfrac{1}{x}.\]

On a tracé dans le repère orthonormé ci-contre la courbe représentative $\mathcal{C}_f$ de la fonction $f$.

On considère les points A(1~;~0) ; B(2~;~0) ; C(1~;~2); D(2~;~8) ; E(1~;~3) et F(2~;~9).

On admet que la courbe $\mathcal{C}_f$ est au-dessus du segment [CD] et en dessous du segment [EF].

\begin{enumerate}
\item À l'aide du graphique, donner un encadrement d'amplitude 1 de l'intégrale : 

\[\displaystyle\int_1^2 f(x)\: \text{d}x.\]

\item Calculer la valeur exacte de $\displaystyle\int_1^2 f(x)\: \text{d}x$.
\end{enumerate}} \hfill
\parbox{0.48\linewidth}{\psset{yunit=1.5cm,xunit=1.5cm}
\begin{pspicture*}(-0.5,-0.4)(3,9.2)
\psgrid[gridlabels=0pt,subgriddiv=1,gridwidth=0.1pt]
\psaxes[linewidth=1.25pt,labelFontSize=\scriptstyle]{->}(0,0)(0,0)(3,9.2)
\psplot[plotpoints=2000,linewidth=1.25pt,linecolor=blue]{1}{2}{2.71828 x exp x add 1 x div sub}
\psline[linestyle=dashed,linewidth=1.25pt](1,3)(2,9)
\psline[linestyle=dashed,linewidth=1.25pt](1,2)(2,8)
\uput[ur](1,0){A}\uput[ur](2,0){B}\uput[dr](1,2){C}
\uput[ur](2,8){D}\uput[ul](1,3){E}\uput[l](2,9){F}
\uput[l](1,2.75){\blue $\mathcal{C}_f$}
\end{pspicture*}}

\bigskip

