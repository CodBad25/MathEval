
\medskip

On a $RC = 10^3 \times 2 \cdot 10^{-3} = 2$, ce qui donne : $u_c(t) = 4\left(1 - \e^{-\frac{t}{2}}\right)$.

Le temps de charge $t$  nécessaire pour obtenir une tension aux bornes du condensateur égale à la moitié de sa tension maximale est tel que $u_c(t)=\dfrac{E}{2}$ c'est-à-dire $u_c(t)=2$.

On résout cette équation :
\begin{align*}
&u_c(t) = 2 \\
\iff &4\left(1 - \e^{-\frac{t}{2}}\right) = 2 \\
\iff &1 - \e^{-\frac{t}{2}}=\dfrac{1}{2} \\
\iff &\dfrac{1}{2} = \e^{-\frac{t}{2}} \\
\iff &\ln\left (\dfrac{1}{2} \right ) = -\dfrac{t}{2} \\
\iff &t = -2\ln \left (\dfrac{1}{2}\right),
\end{align*}
soit environ $1,4$ seconde.

