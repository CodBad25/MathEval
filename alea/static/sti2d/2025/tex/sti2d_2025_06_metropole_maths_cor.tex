
\bigskip

\textbf{Question 1}

\medskip

\begin{itemize}[label=\textbullet]
\item Le module est toujours positif et vaut ici $5$ (distance à l'origine),
\item L'angle est de $45^\circ$ dans le sens horaire, soit $-\frac{\pi}{4}$ (position dans le 4ème quadrant).
\end{itemize}

Donc l'affixe du point M est :
$5\e^{-\i\frac{\pi}{4}} \quad \longrightarrow \quad \text{réponse \textbf{C}.}$

\bigskip

\textbf{Question 2}

\medskip

Soit l'équation différentielle $y' = 2y - 0,5$.

\begin{enumerate}
    \item Il s'agit d'une équation différentielle linéaire du premier ordre de la forme :
\[y' + ay = b \quad \text{avec } a = -2 \text{ et } b = -0,5.\]

La solution générale est :
\begin{align*}
f(t) &= k\e^{-at} + \dfrac{b}{a} \\
&= k\e^{2t} + \dfrac{-0,5}{-2} \\
&= k\e^{2t} + 0,25 \text{ où } k \in \mathbb{R}.
\end{align*}

    \item On a :
\[f(t) = k\e^{2t} + 0,25.\]
\[f'(t) = 2k\e^{2t}.\]

Avec la condition initiale :
\[f'(0) = 2k = -3 \Rightarrow k = -\dfrac{3}{2} = -1,5.\]

La solution est donc :
\[f(t) = -1,5\e^{2t} + 0,25.\]
\end{enumerate}

\bigskip

\textbf{Question 3}

\begin{align*}
f(x) &= 0 \\
\e^{-0,016x} - 2 &= 0 \\
\e^{-0,016x} &= 2 \\
-0,016x &= \ln(2) \\
x &= -\dfrac{\ln(2)}{0,016} \\
x &\approx -43,32 \; \text{à} \; 10^{-2} \; \text{près.}
\end{align*}

\bigskip

\textbf{Question 4}

\begin{align*}
\ln\left(\dfrac{x^4}{9}\right) - 3 \ln(x) + \ln\left(\dfrac{9}{x}\right) 	&= \ln(x^4) - \ln(9) - 3 \ln(x) + \ln(9) - \ln(x) \\
&= 4\ln(x) - 3 \ln(x) - \ln(x) \\
&= 0.
\end{align*}
Ce qui prouve l'égalité.

\bigskip


