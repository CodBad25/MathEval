
\bigskip

\textbf{Question 1}

\medskip

\begin{enumerate}
\item L'équation différentielle $y'=ay+b$ a pour solutions les fonctions $f$ définies par $f(t)=k\e^{at}-\dfrac{b}{a}$ où $k$ est un réel quelconque.

Donc l'équation différentielle $y'=-2y+40$ a pour solutions les fonctions $f$ définies par $f(t)=k\e^{-2t}-\dfrac{40	}{-2}$ où $k$ est un réel quelconque, soit les fonctions $f$ définies par $f(t)=k\e^{-2t}+20$.

\item $f(0)=200 \iff k\e^{0}+20=200 \iff k=180$

La solution $f$ de l'équation différentielle $(E)$ qui vérifie $f(0) = 200$ est définie par\\
$f(t)=180\e^{-2t}+40$.
\end{enumerate}

\bigskip

\textbf{Question 2}

\medskip

$f'(x)=1\times \e^{x} + (x-1)\times \e^{x}=x\e^{x}$

Pour tout réel $x$, on sait que $\e^{x}>0$ donc $f'(x)$ est du signe de $x$. Ce qui justifie le tableau donné.

\bigskip

\textbf{Question 3}

\medskip

\begin{enumerate}
\item $z_1  \times z_2 = \left ( 2\e^{\text{i}\frac{\pi}{3}}\right ) \times \left ( \sqrt 2 \e^{\text{i}\frac{\pi}{4}}\right )
=2\sqrt{2} \e^{\left (\text{i} \frac{\pi}{3} + \text{i}\frac{\pi}{4}\right )}
=2\sqrt{2} \e^{\text{i} \frac{7\pi}{12}}$

\item On en déduit que 
$z_1  \times z_2 = 2\sqrt{2} \left ( \cos \dfrac{7\pi}{12} +  \text{i} \sin \dfrac{7\pi}{12}\right )$.
\end{enumerate}

\bigskip

\textbf{Question 4}

\medskip

\begin{enumerate}
\item Le 1\ier{} janvier 2020 correspond à $t=0$; $f(0)=10\e^{0}=10$.

Donc le nombre d'habitants donné par ce modèle au 1\up{er} janvier 2020 est de 10 millions.

Le 1\ier{} janvier 2021 correspond à $t=1$; $f(1)=10\e^{0,02} \approx 10,2$.

Donc le nombre d'habitants donné par ce modèle au 1\up{er} janvier 2021 est d'environ $10,2$ millions.

\item Pour déterminer l'année durant laquelle l'effectif de la population dépassera 20 millions d'habitants, on cherche la plus petite valeur de $t$ pour laquelle $f(t)>20$.

$f(t)>20
\iff 10\e^{0,02t}>20
\iff \e^{0,02t}>2
\iff 0,02t>\ln(2)
\iff t>\dfrac{\ln(2)}{0,02}$

$\dfrac{\ln(2)}{0,02} \approx 34,66$ donc c'est au cours de la 34\ieme{} année, c'est-à-dire en $2054$ que l'effectif dépassera 20 millions.
\end{enumerate}

\bigskip


