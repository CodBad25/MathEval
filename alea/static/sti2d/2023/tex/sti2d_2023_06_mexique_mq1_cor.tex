
\medskip

\begin{enumerate}
\item L'équation différentielle $y'=ay+b$ a pour solutions les fonctions $f$ définies par $f(t)=k\e^{at}-\dfrac{b}{a}$ où $k$ est un réel quelconque.

Donc l'équation différentielle $y'=-2y+40$ a pour solutions les fonctions $f$ définies par $f(t)=k\e^{-2t}-\dfrac{40	}{-2}$ où $k$ est un réel quelconque, soit les fonctions $f$ définies par $f(t)=k\e^{-2t}+20$.

\item $f(0)=200 \iff k\e^{0}+20=200 \iff k=180$

La solution $f$ de l'équation différentielle $(E)$ qui vérifie $f(0) = 200$ est définie par\\
$f(t)=180\e^{-2t}+40$.
\end{enumerate}

\bigskip

