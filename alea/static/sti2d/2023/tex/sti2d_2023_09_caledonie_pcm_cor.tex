
\medskip

\begin{enumerate}[start=7]
\item Déterminons les solutions sur $[0~;~+\infty[$ de cette équation différentielle.

Les solutions de  l'équation différentielle $y'+ay=b$ sur  $\R$ sont les fonctions $y$ définies par  

$y(x)=C\e^{-ax}+\dfrac{b}{a}$ où $C$ est une constante quelconque.

$a=\dfrac{1}{180}\ b=\dfrac{4}{9}$  par conséquent sur $[0~;~+\infty[$ $y(t)=C \e^{-\frac{1}{180}t}+\dfrac{\frac{4}{9}}{\frac{1}{180}}$ 

c'est-à-dire $\theta(t)=C \e^{-\frac{1}{180}t}+80$ où $C$ est une constante quelconque.

À $t = 0$, la température du moteur est de $20\degres$ C.

\item Déterminons $C$ pour que la condition initiale soit vérifiée.

$\theta(0)=20$ \:ou \: $C\e^0+80 = 20$ d'où $C=20 - 80 = - 60$.

La fonction $\theta$ est définie sur $[0~;~+\infty[$ par :
\[\theta (t) = 80 - 60\e^{-\frac{1}{180}t}.\]

\item Résolvons sur $[0~;~+\infty[$ l'équation $\theta(t) = 79$.

\begin{align*}
80 - 60\e^{-\frac{1}{180}t}&=79\\
- 60\e^{-\frac{1}{180}t}&=-1\\
\e^{-\frac{1}{180}t}&=\dfrac{1}{60}\\
-\frac{1}{180}t&=-\ln 60\\
t&=\ln 60\times 180\\
\end{align*}
L'ensemble solution de l'équation est $\{180 \ln 60\}$

\item La condition est respectée,
car $180\times \ln 60\approx 737$, or à 20 minutes correspondent $\np[s]{1200}$.

Nous avons bien $737<\np{1200}$
\end{enumerate}

\bigskip


