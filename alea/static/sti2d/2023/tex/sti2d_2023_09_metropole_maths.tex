
\medskip

\textbf{Les questions 1, 2, 3 et 4 sont indépendantes les unes des autres.}

\bigskip

\textbf{Question 1}

\medskip

\emph{Pour cette question, indiquer la lettre de la réponse exacte. Aucune justification n'est demandée.}

\medskip

On considère un réel $x$, strictement positif et on note $\log(x) = \dfrac{\ln(x)}{\ln (10)}$.

Pour tout réel $x$, strictement positif, $\log(100x)$ est égal à :

\medskip

\begin{tabularx}{\linewidth}{|*{4}{>{\centering \arraybackslash}X|}}
\hline
A& B& C& D\\
\hline
$10x$& $100 \log (x)$ & $2 + \log (x)$ &$10 + \log (x)$ \\
\hline
\end{tabularx}

\bigskip

\textbf{Question 2}

\medskip

On considère la fonction $f$ définie sur $\R$ par $f(x) = 2\e^{3x} - 2$.

Déterminer la limite de la fonction $f$ en $-\infty$.

\bigskip

\textbf{Question 3}

\medskip

On désigne par i le nombre complexe de module $1$ et d'argument $\dfrac{\pi}{2}$.

Le plan est muni d'un repère orthonormé \Ouv.

Sur le graphique suivant, on considère le point E dont l'affixe est notée: $Z_E$.

\begin{center}
\psset{xunit=1.5cm,yunit=1.5cm,labelFontSize=\scriptstyle,showorigin=false}
\begin{pspicture}(-2.8,-2.8)(2.8,2.8)
\multido{\n=-2+1}{5}{\psline[linewidth=0.45pt,linecolor=lightgray](\n,-2.5)(\n,2.5)}
\multido{\n=-2+1}{5}{\psline[linewidth=0.45pt,linecolor=lightgray](-2.5,\n)(2.5,\n)}
\psaxes[linewidth=0.75pt]{-}(0,0)(-2.5,-2.5)(2.5,2.5)
\pscircle[linewidth=1pt,linecolor=blue](0,0){3}
\uput[ur](0,0){O}\uput[ur](1,1.732){$E$}\psdots[dotstyle=bullet,dotscale=1.2](0,0)(1,1.732)
\psline[linewidth=0.95pt]{->}(0,0)(1,0)\psline[linewidth=0.95pt]{->}(0,0)(0,1)
\uput[d](0.5,0){$\vect{u}$}\uput[l](0,0.5){$\vect{v}$}
\end{pspicture}
\end{center}

Par lecture graphique, donner l'écriture exponentielle de $Z_E$.

\bigskip

\textbf{Question 4}

\medskip

On considère l'équation différentielle $(E)$ : \[y' = 2y + 0,5\]
où $y$ est une fonction de la variable $x$, définie et dérivable sur $\R$ et $y'$ la fonction dérivée de $y$.

Déterminer les solutions sur $\R$ de l'équation différentielle $(E)$.

\bigskip


