
\begin{center}
\textbf{Contrôle de la température dans un lave-linge.}
\end{center}

\medskip

Lors d'un cycle de lavage d'une machine à laver le linge, la phase qui consomme le plus d'énergie est le chauffage de l'eau utilisée en phase de lavage.

\medskip

La température de l'eau est contrôlée par une thermistance CTN, qui est un composant dont la valeur de la résistance électrique $R$ varie en fonction de la température.

\medskip

Il est possible, au laboratoire, d'étudier les variations de la résistance d'une thermistance CTN en fonction de la température à l'aide d'un montage adéquat.

\medskip

Les valeurs obtenues pour une thermistance donnée permettent de tracer la courbe suivante.

\begin{center}
\psset{xunit=0.15cm,yunit=0.0003cm,}
\begin{pspicture}(-3,-4000)(70,20000)
\multido{\n=0+10}{8}{\psline[linewidth=0.55pt](\n,-50)(\n,20000)}
\multido{\n=0+2}{36}{\psline[linewidth=0.3pt,linecolor=lightgray ](\n,-50)(\n,20000)}
\multido{\n=0+1000}{21}{\psline[linewidth=0.3pt,linecolor=lightgray](0,\n)(70,\n)}
\multido{\n=0+5000}{5}{\psline[linewidth=0.55pt](0,\n)(70,\n)}
\psaxes[linewidth=0.95pt,Dx=10,Dy=5000,labelFontSize=\scriptstyle]{-}(0,0)(-0.1,-50)(71,19999)
\psplot[plotpoints=2000,linewidth=1.25pt,linecolor=blue,linestyle=dashed]{10}{60}{2.71828 x 0.042 neg mul exp 28785 mul}
\psdots[dotstyle=+,dotscale=1.8,linecolor=red](10,20000)(15,15750)(20,12500)(25,10000)(30,8200)(35,6500)(40,5500)(45,4500)(50,3500)(55,3000)(60,2500)
\uput[d](32,-2000){Température \textcelsius}\uput[l](-5,12500){R($\Omega$)}
\end{pspicture}
\end{center}

\medskip

Cette résistance (en $\Omega$), en fonction de la température $T$ (en \textcelsius), peut être modélisée par la fonction $R$ définie sur $[0~;~100]$ :
\[R(T) = \np{28785} \times \e^{-0,042\times T}.\]

\medskip

\begin{enumerate}[start=4]
\item À l'aide du graphique, déterminer à partir de quelle température la résistance devient inférieure à $\np[ k\Omega]{10}$.

\item Résoudre sur $[0~;~100]$ l'équation $R(T) = \np{10000}$. Comparer avec la valeur lue sur le graphique à la question \textbf{4}.

\item On note $R'$ la fonction dérivée de $R$ sur $[0~;~100]$. Déterminer une expression de $R'(T)$ en $\Omega \cdot ^{\text{o}} C^{-1}$.
\end{enumerate}

\medskip

La sensibilité de la thermistance CTN est donnée par la fonction $S$ définie sur $[0~;~100]$ par:
\[S = -\dfrac{\mathrm{d}R}{\mathrm{d}T}.\]

Le dispositif de régulation de la température sera d'autant plus performant que la valeur de la sensibilité de la thermistance sera grande.

\begin{enumerate}[start=7]
\item Montrer que la sensibilité de la thermistance CTN est environ 12 fois plus grande à 30~\textcelsius{} qu'à 90~\textcelsius.
\end{enumerate}

\bigskip


