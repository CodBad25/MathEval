
\medskip

\textbf{Les questions 1, 2, 3 et 4 sont indépendantes les unes des autres.}

\bigskip

\textbf{Question 1}

\medskip

\emph{Pour cette question, indiquer la lettre de la réponse exacte. Aucune justification n'est demandée.}

\medskip

L'expression $\dfrac{\left(\text{e}^{-3x}\right)^2 \times \left(\text{e}^{2x}\right)^{-3}}{\text{e}^{5x} \times \text{e}^{6x}}$ vaut:

\begin{center}
\renewcommand\arraystretch{2.1}
\begin{tabularx}{\linewidth}{|*{4}{>{\centering \arraybackslash}X|}}\hline
A				&B						&C					&D\\ \hline
$\text{e}^{-1}$	&$\dfrac{2}{5}x^{-3}$	&$\text{e}^{- x}$	&$\text{e}^{-23x}$\\ \hline
\end{tabularx}
\end{center}

\bigskip

\textbf{Question 2}

\medskip

Soit $f$ la fonction définie sur $\R$ par :
\[f(x) = \text{e}^{2x}(-3x + 1).\]

 On admet que la fonction $f$ est dérivable sur $\R$ et on note $f'$ la fonction dérivée de $f$ sur $\R$.

Montrer que :
\[f'(x) = \text{e}^{2x}(-6x - 1).\]

\bigskip

\textbf{Question 3}

\medskip

On désigne par i le nombre complexe de module 1 et d'argument $\dfrac{\pi}{2}$.

Mettre le nombre complexe $\sqrt 3 + \text{i}$ sous forme exponentielle en détaillant les calculs.

\bigskip

\textbf{Question 4}

\medskip

Résoudre sur l'intervalle $]0~;~+\infty[$ l'équation :

\[\dfrac{2}{3\ln (10)} \ln (x) - 2,88 = 4.\]

\bigskip


