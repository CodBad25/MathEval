
\begin{center}
\textbf{Étude en laboratoire de la décharge d'un supercondensateur}
\end{center}

On réalise un montage qui permet de charger un supercondensateur de  capacité égale à 372~F avec un générateur, puis de le décharger dans un conducteur ohmique de résistance $R$.

\medskip

Le graphique ci-dessous représente l'enregistrement de l'évolution de la tension aux bornes du supercondensateur au cours de sa décharge.

\begin{center}
\psset{xunit=0.036cm,yunit=4.55cm,comma=true}
\begin{pspicture}(-20,-0.2)(360,2.7)
\multido{\n=0+10}{37}{\psline[linewidth=0.025pt](\n,0)(\n,2.5)}
\multido{\n=0.0+0.1}{26}{\psline[linewidth=0.025pt](0,\n)(360,\n)}
\psaxes[linewidth=1.25pt,Dx=50,Dy=0.5](0,0)(360,2.5)
\uput[u](360,0){$t$ (en s)}\uput[r](0,2.6){$u$ (en V)}
\psplot[plotpoints=2000,linewidth=1.25pt,linecolor=red]{0}{360}{2.3 2.71828 0.0112 x mul exp div}
\psplot[linestyle=dashed,linewidth=1.25pt]{0}{89}{2.3 0.02576 x mul sub}
\end{pspicture}
\end{center}

\medskip

L'évolution de la tension aux bornes du supercondensateur est modélisée par la fonction $f$ définie sur l'intervalle $[0~;~+\infty[$ par:
\[f(x) = 2,3\text{e}^{-\np{0,0112}x},\]

où $x$ représente le temps en seconde.

\begin{enumerate}[start=5]
\item Montrer qu'une équation de la tangente à la courbe représentative de la fonction $f$ au point d'abscisse 0 est : 
\[y = -\np{0,02576}x + 2,3.\]

On rappelle qu'une équation de la tangente à la courbe représentative d'une fonction $f$ au point d'abscisse $a$ est
\begin{center} $y = f'(a)(x - a) + f(a)$ où $f'$ est la fonction dérivée de $f$.
\end{center}

\item Déterminer l'abscisse $\tau$ du point d'intersection de cette tangente avec
l'axe des abscisses.

On donnera une valeur approchée à $10^{-1}$ près.

\item Déterminer la capacité $C$ du supercondensateur sachant que
$\tau = R \times C$ et $R = 0,235~\Omega$.

Comparer la valeur obtenue à partir de ce modèle avec les données du constructeur.
\end{enumerate}

\bigskip


