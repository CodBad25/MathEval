
\bigskip

\textbf{Question 1}

\medskip

\begin{enumerate}
\item Écrivons $z_1$ sous forme exponentielle, en détaillant les calculs.

Une forme exponentielle d'un nombre complexe z non nul est $z=\vert z\vert \e^{\text{i}\theta}$.

$\vert z\vert = \sqrt{\sqrt{2}^2+\sqrt{2}^2}=\sqrt{2+2}=2$

$\cos \theta =\dfrac{\sqrt{2}}{2}$ et $\sin \theta =\dfrac{\sqrt{2}}{2}$. Il en résulte $\theta= \dfrac{\pi}{4}$.

Une écriture sous forme exponentielle de $z$ est , par conséquent $z=2\e^{\text{i}\frac{\pi}{4}}$

\item Montrons que $2 z_2^3 = z_1$.

En utilisant la formule de Moivre $\left(\e^{\text{i}\theta}\right)^n=\left(\e^{n\text{i}\theta}\right)$, nous avons :

\[2 z_2^3=2\left(\e^{\text{i}\frac{\pi}{12}}\right)^3 =2\left(\e^{3\text{i}\frac{\pi}{12}}\right)=2\left(\e^{\text{i}\frac{\pi}{4}}\right)=z_1.\]
\end{enumerate}

\bigskip

\textbf{Question 2}

\medskip

\begin{enumerate}
\item Calculons la valeur exacte de l'ordonnée du point A.

Pour ce faire, calculons $f(1)$ :

$f(1) = (10\times 1-4)\e^{-1}=6\e^{-1} \approx 2,21$.

\item Calculons $f'(1)$ :

$f'(1) = (-10\times 1 + 14)\e^{-1}=4\e^{-1}\approx 1,47$.

Cette valeur est le coefficient directeur de $T_1$, tangente à la courbe $\mathcal{C}_f$ au point A d'abscisse 1.

\item La courbe représentative de la fonction $f$ suggère l'existence d'un maximum sur l'intervalle $[1~;~2]$.

Déterminons la valeur exacte de ce maximum.

Les maximums sont à rechercher parmi les points où la dérivée s'annule.

Résolvons donc $f'(x)=0$ soit $(-10x+14)\e^{- x}=0$.

Pour tout $x, \: \e^{-x}\not= 0$, nous sommes amenés à résoudre :

$-10x+14 =0 \iff 14 = 10x \iff 7 = 5x$ d'où $x=\dfrac{7}{5} = 1,4$.

La fonction admet un maximum pour $x = \dfrac{7}{5}$.
\end{enumerate}

\bigskip


