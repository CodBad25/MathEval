
\medskip

\begin{enumerate}
\item D'après le texte, $z_{\text M}=x_{\text M} + 3 \text{i}$, et $\text{OM} = \left |z_{\text M}\right |= 6$.

$z_{\text M}= \left |z_{\text M}\right | \left (\cos(\theta) + \text{i}\sin(\theta)\right )$ donc  $z_{\text M}= 6 \left (\cos(\theta) + \text{i}\sin(\theta)\right )$.

En égalisant les parties imaginaires, on déduit $3= 6 \sin(\theta)$ donc $\sin(\theta)=\dfrac{1}{2}$.

\item $\theta=\dfrac{\pi}{6}$ ou $\theta=\dfrac{5\pi}{6}$.

La partie réelle de $z_{\text M}$ vaut alors $6\cos \left (\frac{\pi}{6}\right )$ qui est positive, ou $6\cos \left (\frac{5\pi}{6}\right )$ qui est négative.

Comme on sait que cette partie réelle est négative, on a donc: $\theta=\frac{5\pi}{6}$.

\item L'écriture exponentielle de $z_{\text M}$ est donc: $6\e^{\frac{5\pi}{6}\text{i}}$.
\end{enumerate}

