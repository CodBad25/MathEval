
\medskip

\begin{enumerate}
\item Calculons la valeur exacte de l'ordonnée du point A.

Pour ce faire, calculons $f(1)$ :

$f(1) = (10\times 1-4)\e^{-1}=6\e^{-1} \approx 2,21$.

\item Calculons $f'(1)$ :

$f'(1) = (-10\times 1 + 14)\e^{-1}=4\e^{-1}\approx 1,47$.

Cette valeur est le coefficient directeur de $T_1$, tangente à la courbe $\mathcal{C}_f$ au point A d'abscisse 1.

\item La courbe représentative de la fonction $f$ suggère l'existence d'un maximum sur l'intervalle $[1~;~2]$.

Déterminons la valeur exacte de ce maximum.

Les maximums sont à rechercher parmi les points où la dérivée s'annule.

Résolvons donc $f'(x)=0$ soit $(-10x+14)\e^{- x}=0$.

Pour tout $x, \: \e^{-x}\not= 0$, nous sommes amenés à résoudre :

$-10x+14 =0 \iff 14 = 10x \iff 7 = 5x$ d'où $x=\dfrac{7}{5} = 1,4$.

La fonction admet un maximum pour $x = \dfrac{7}{5}$.
\end{enumerate}


