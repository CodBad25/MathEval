
\medskip

\textbf{Les questions $1$ et $2$ sont indépendantes.}

\bigskip

\textbf{Question 1}

\medskip

On désigne par i le nombre complexe de module 1 et d'argument $\dfrac{\pi}{2}$.

Soient $z_1$ et $z_2$ les nombres complexes définis par :
\begin{center}$z_1 = \sqrt 2 + \text{i}\sqrt 2$ \quad et  \quad $z_2 = \e^{\text{i}\frac{\pi}{12}}$.\end{center}

\begin{enumerate}
\item Écrire $z_1$ sous forme exponentielle, en détaillant les calculs.
\item Montrer que $2z_2^3 = z_1$.
\end{enumerate}

\bigskip

\textbf{Question 2}

\medskip

Soit la fonction $f$ définie pour tout réel $x$ par 
\[f(x) = (10x - 4)\e^{-x}.\]

On nomme $\mathcal{C}_f$ la courbe représentative de la fonction $f$ donnée dans le repère ci- dessous.

La droite $T_1$ est la tangente à la courbe $\mathcal{C}_f$ au point A d'abscisse 1 et on admet que la dérivée de $f$ est définie pour tout réel $x$ par 
\[f'(x) = (-10x + 14)\e^{-x}.\]

\begin{center}
\psset{unit=1.25cm,arrowsize=2pt 3}
\begin{pspicture*}(-1,-1)(8,4)
\psgrid[gridlabels=0pt,subgriddiv=1,gridcolor=lightgray]
\psaxes[linewidth=1.25pt]{->}(0,0)(-1,-0.95)(8,4)
\psplot[plotpoints=2000,linewidth=1.25pt,linecolor=red,labelFontSize=\scripstyle]{-3}{8.5}{10 x mul 4 sub 2.71828 x  exp div}
\psplotTangent{1.}{4}{10 x mul 4 sub 2.71828 x  exp div}
\uput[ul](1,2.2){\small A}\uput[ul](1.6,3.2){\small $T_1$}
\uput[d](7.8,0){$x$} \uput[r](0,3.9){$y$}\uput[r](3,1.3){\red $\mathcal{C}_f$}
\psdots(1,2.2)
\end{pspicture*}
\end{center}

\begin{enumerate}
\item Calculer la valeur exacte de l'ordonnée du point A.
\item Calculer $f'(1)$.

Interpréter graphiquement cette valeur.
\item La courbe représentative de la fonction $f$ suggère l'existence d'un maximum
sur l'intervalle [1~;~2].

Quelle est la valeur exacte de ce maximum ?
\end{enumerate}

\bigskip


