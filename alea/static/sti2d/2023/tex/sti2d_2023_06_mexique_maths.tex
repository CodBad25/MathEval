
\medskip

\textbf{Les questions 1, 2, 3 et 4 sont indépendantes les unes des autres.}

\bigskip

\textbf{Question 1}

\medskip

On considère l'équation différentielle 
\[(E) :\qquad   y' = - 2y + 40.\]

\begin{enumerate}
\item Déterminer l'ensemble des solutions de l'équation différentielle $(E)$.
\item En déduire la solution $f$ de l'équation différentielle $(E)$ qui vérifie
$f(0) = 200.$
\end{enumerate}

\bigskip

\textbf{Question 2}

\medskip

Soit $f$ la fonction définie sur $\R$ par 
\[f(x) = (x - 1)\text{e}^x.\]

$f$ est dérivable et sa dérivée est notée $f'$.

Justifier le signe de $f'(x)$ établi dans le tableau ci-dessous:

\begin{center}
\renewcommand{\arraystretch}{1.5}
\def\esp{\hspace*{1.5cm}}
$\begin{array}{|c | *{5}{c} |} 
\hline
x  & -\infty & \esp & 0 & \esp  & +\infty \\
\hline
f'(x) &  & \pmb{-} &  \vline\hspace{-2.7pt}{0} & \pmb{+} &    \\
\hline
\end{array}$
\renewcommand{\arraystretch}{1}
\end{center}

\bigskip

\textbf{Question 3}

\medskip

On considère les nombres complexes 
\begin{center}$z_1 = 2\text{e}^{\text{i}\frac{\pi}{3}}$ \quad et \quad  $z_2 = \sqrt 2 \text{e}^{\text{i}\frac{\pi}{4}}$\end{center}

\begin{enumerate}
\item Exprimer sous forme exponentielle le produit $z_1  \times z_2$.
\item En déduire une forme trigonométrique de $z_1  \times z_2$.
\end{enumerate}

\bigskip

\textbf{Question 4}

\medskip

L'évolution de l'effectif de la population d'un pays, exprimé en millions d'habitants, est modélisée par la fonction $f$ définie sur [0~;~40] comme suit : 
\[f(t) = 10\text{e}^{0,02t}, \]
où $t$ correspond au nombre d'années écoulées depuis le 1\up{er} janvier 2020.

\medskip

\begin{enumerate}
\item Estimer le nombre d'habitants donné par ce modèle au 1\up{er} janvier 2020 et au 1\up{er} janvier 2021.
\item D'après ce modèle, déterminer l'année durant laquelle l'effectif de la population dépassera 20 millions d'habitants.
\end{enumerate}

\bigskip


