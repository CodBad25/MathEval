
\begin{center}
\textbf{Bouteille isotherme avec indicateur de température interne}
\end{center}

Une bouteille isotherme est conçue pour conserver des boissons chaudes ou froides. Une entreprise a développé une bouteille isotherme avec un afficheur de la température de la boisson. À l’issue de la réalisation d’un prototype, les techniciens réalisent une série de tests pour contrôler la qualité du produit. Le critère ci-dessous reste à valider pour respecter le cahier des charges de l’entreprise et ainsi passer à la mise en place de la chaîne de production industrielle.

\textbf{Critère :} la variation de température d'une boisson doit être inférieure ou égale à $5$ \degres C avec une tolérance de $0,5$ \degres C au bout de 8 heures pour une température extérieure de $\theta_{\text{ext}} = 20,0$ \degres C.

On souhaite vérifier ce critère dans le cas d'une boisson chaude.

\medskip

L'évolution de la température (en \degres C ) de la boisson en fonction du temps (en heure) est modélisée par la fonction $f$ solution de l'équation différentielle suivante :
\[(E) :\quad  y' = - 0,044y + 0,88\]

où $y$ est une fonction définie sur $\R$ et $y'$ sa dérivée.

\begin{enumerate}[start=6]
\item Déterminer l'ensemble des solutions de l'équation différentielle (E).

\item Sachant que la température initiale de la boisson est de $60 \degres $C, montrer que $f$ est définie sur l'intervalle $[0~;~ +\infty[$ par :
\[f(t) = 40\text{e}^{-0,044t} + 20.\]

\item En déduire la température de la boisson au bout de $8$ heures. 

Indiquer si le critère est vérifié.
\end{enumerate}

\bigskip


