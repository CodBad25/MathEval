
\medskip

Avec $z = \sqrt 3 + \text{i}$, on a :
\[|z| = \sqrt{\left(\sqrt 3  \right)^2 + 1^2} = \sqrt{3 + 1} = 2.\]

On peut alors en factorisant 2, écrire :
\[z = 2\left( \dfrac{\sqrt 3}{2} + \dfrac{1}{2}\text{i} \right).\]

Or on sait que $\cos \dfrac{\pi}{6} = \dfrac{\sqrt 3}{2}$ et $\sin \dfrac{\pi}{6} = \dfrac{1}{2}$, donc :
\[z = 2\left(\cos \dfrac{\pi}{6} + \text{i}\sin \dfrac{\pi}{6}  \right) = 2 \text{e}^{\frac{\pi}{6}}.\]

\bigskip

