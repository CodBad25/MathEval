
\medskip

\begin{enumerate}
\item $z_2 = - \sqrt 3 + \text{i}$.

Donc $\left|z_2\right|^2 = \left(- \sqrt 3\right)^2 + 1^2 = 3 + 1 = 4 = 2^2$, donc $\left|z_2\right| = 2$.

On peut en factorisant ce module 2 dans l'écriture de $z_2$, écrire :

$z_2 = 2\left(- \dfrac{\sqrt{3}}{2} + \text{i}\dfrac12\right)$.

Or $- \dfrac{\sqrt{3}}{2} = \cos \dfrac{5\pi}{6}$ et $\dfrac12 = \sin \dfrac{5\pi}{6}$, donc :

$z_2 = 2\left(\cos \dfrac{5\pi}{6} + \text{i}\sin \dfrac{5\pi}{6}\right) = 2\e^{\frac{5\pi}{6}}$.

\item En se servant du résultat précédent : 

$Z = \dfrac{z_1}{z_2^3} = \dfrac{6\e^{\text{i}\frac{\pi}{4}}}{\left(2\e^{\text{i}\frac{5\pi}{6}}\right)^3} =  \dfrac68 \dfrac{\e^{\text{i}\frac{\pi}{4}}}{\e^{\text{i}\frac{5\pi}{2}}}$.

Or $\e^{\text{i}\frac{5\pi}{2}} = \e^{\text{i}\frac{\pi}{2}}$, donc 

$Z = \dfrac34\e^{\text{i}\left(\frac{\pi}{4} - \frac{\pi}{2} \right)} = \dfrac34\e^{-\text{i}\frac{\pi}{4}}$.
\end{enumerate}

