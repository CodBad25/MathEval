
\medskip

\textbf{Dans cet exercice, seulement 4 questions au choix parmi les 6 questions proposées sont à traiter. Toutes ces questions sont indépendantes les unes des autres.}

\medskip

\textbf{Question 1}

\medskip

$g$ est une fonction définie et dérivable sur $[0\;;\;+\infty[$.

On admet que la dérivée de $g$ est la fonction $g'$ définie sur $[0\;;\;+\infty[$ par: 

\[g'(t)=6\e^{-t}\left (1-t\right ).\]

\begin{enumerate}
\item Étudier le signe de $g'(t)$ sur $[0\;;\;+\infty[$.
\item En déduire les variations de $g$ sur $[0\;;\;+\infty[$.
\end{enumerate}

\bigskip

\textbf{Question 2}

\medskip

Le plan est muni d'un repère orthonormé \Ouv. 

Soit A et B les points d'affixes respectives:

\[z_{\text{A}} = \e^{\text{i} \frac{5\pi}{6}} \text{ et } z_{\text{B}} = \e^{-\text{i} \frac{2\pi}{3}}.\]

\begin{enumerate}
\item Les points A et B sont correctement représentés sur l'une des figures ci-dessous.

Laquelle ? Aucune justification n'est attendue.

\begin{tabularx}{\linewidth}{X|X|X|X}
\psset{unit=1cm,arrowsize=2pt 2}
\def\xmin {-1.5}   \def\xmax {1.5}
\def\ymin {-2}   \def\ymax {1.5}
\begin{pspicture*}(\xmin,\ymin)(\xmax,\ymax)
%\psgrid[subgriddiv=1,  gridlabels=0, gridcolor=lightgray] 
%\psaxes[arrowsize=3pt 3, ticksize=-2pt 2pt, labels=none]{->}(0,0)(\xmin,\ymin)(\xmax,\ymax) 
\uput[dl](0,0){O} 
\uput[d](0.5,0){$\vec{u}$}  \uput[l](0,0.5){$\vec{v}$} 
\pscircle(0,0){1}
\psline{->}(-1,0)(1,0) \psline{->}(0,-1)(0,1) 
\psdots[dotstyle=x,dotscale=1.4,linecolor=blue](1;-30)(1;-120)
\uput[-30](1;-30){\blue A} \uput[-120](1;-120){\blue B} 
\uput[d](0,-1.5){\bf Figure 1}
\end{pspicture*}
&
\psset{unit=1cm,arrowsize=2pt 2}
\def\xmin {-1.5}   \def\xmax {1.5}
\def\ymin {-2}   \def\ymax {1.5}
\begin{pspicture*}(\xmin,\ymin)(\xmax,\ymax)
%\psgrid[subgriddiv=1,  gridlabels=0, gridcolor=lightgray] 
%\psaxes[arrowsize=3pt 3, ticksize=-2pt 2pt, labels=none]{->}(0,0)(\xmin,\ymin)(\xmax,\ymax) 
\uput[dl](0,0){O} 
\uput[d](0.5,0){$\vec{u}$}  \uput[l](0,0.5){$\vec{v}$} 
\pscircle(0,0){1}
\psline{->}(-1,0)(1,0) \psline{->}(0,-1)(0,1) 
\psdots[dotstyle=x,dotscale=1.4,linecolor=blue](1;150)(1;-120)
\uput[150](1;150){\blue A} \uput[-120](1;-120){\blue B} 
\uput[d](0,-1.5){\bf Figure 2}
\end{pspicture*}
&
\psset{unit=1cm,arrowsize=2pt 2}
\def\xmin {-1.5}   \def\xmax {1.5}
\def\ymin {-2}   \def\ymax {1.5}
\begin{pspicture*}(\xmin,\ymin)(\xmax,\ymax)
%\psgrid[subgriddiv=1,  gridlabels=0, gridcolor=lightgray] 
%\psaxes[arrowsize=3pt 3, ticksize=-2pt 2pt, labels=none]{->}(0,0)(\xmin,\ymin)(\xmax,\ymax) 
\uput[dl](0,0){O} 
\uput[d](0.5,0){$\vec{u}$}  \uput[l](0,0.5){$\vec{v}$} 
\pscircle(0,0){1}
\psline{->}(-1,0)(1,0) \psline{->}(0,-1)(0,1) 
\psdots[dotstyle=x,dotscale=1.4,linecolor=blue](1;-30)(1;120)
\uput[-30](1;-30){\blue A} \uput[120](1;120){\blue B} 
\uput[d](0,-1.5){\bf Figure 3}
\end{pspicture*}
&
\psset{unit=1cm,arrowsize=2pt 2}
\def\xmin {-1.5}   \def\xmax {1.5}
\def\ymin {-2}   \def\ymax {1.5}
\begin{pspicture*}(\xmin,\ymin)(\xmax,\ymax)
%\psgrid[subgriddiv=1,  gridlabels=0, gridcolor=lightgray] 
%\psaxes[arrowsize=3pt 3, ticksize=-2pt 2pt, labels=none]{->}(0,0)(\xmin,\ymin)(\xmax,\ymax) 
\uput[dl](0,0){O} 
\uput[d](0.5,0){$\vec{u}$}  \uput[l](0,0.5){$\vec{v}$} 
\pscircle(0,0){1}
\psline{->}(-1,0)(1,0) \psline{->}(0,-1)(0,1) 
\psdots[dotstyle=x,dotscale=1.4,linecolor=blue](1;150)(1;120)
\uput[150](1;150){\blue A} \uput[120](1;120){\blue B} 
\uput[d](0,-1.5){\bf Figure 4}
\end{pspicture*}
\end{tabularx}

\item Montrer qu'un argument de $\dfrac{z_{\text A}}{z_{\text B}}$ est $\dfrac{-\pi}{2}$.
\end{enumerate}

\bigskip

\textbf{Question 3}

\medskip

Résoudre dans $]1\;;\; +\infty[$ l'équation:

\[\ln\left (x-1\right ) + \ln \left (x+1\right ) + \ln\left (x\right ) = \ln \left (x^2-1\right) - \ln\left (0,5\right).\]

\bigskip

\textbf{Question 4}

\medskip

On considère l'équation différentielle (E): $y'=-y+2$.

\begin{enumerate}
\item Déterminer l'ensemble des solutions de l'équation différentielle (E).
\item En déduire la solution $f$ de l'équation différentielle (E) qui s'annule en 0.
\end{enumerate}

\bigskip

\textbf{Question 5}

\medskip

Soit la fonction $f$ définie sur $\R$ par $f(x)=x^2-2\e^{x}$.

\begin{enumerate}
\item  Montrer que pour tout réel $x$ de $\R$, $f(x)=\e^{x}\left (x^2\e^{-x}-2\right )$.
\item En déduire $\ds\lim_{x\to +\infty} f(x)$.
\end{enumerate}

\bigskip

\textbf{Question 6}

\medskip

\begin{center}
\fbox{
\begin{minipage}{0.9\textwidth}
\textbf{Rappel :} pour $a$ et $b$ deux réels, on a les formules suivantes :
\[\bullet~~\cos(a + b) = \cos (a) \cos(b) - \sin(a) \sin(b)\]
\[\bullet~~\cos(a - b) = \cos (a) \cos(b) + \sin(a) \sin(b)\]
\[\bullet~~\sin(a + b) = \sin(a) \cos(b) + \cos (a) \sin(b)\]
\[\bullet~~\sin(a - b) = \sin(a) \cos(b) - \cos (a) \sin(b)\]
\end{minipage}}
\end{center}

On considère un signal électrique dont l'expression en fonction du temps $t$ est donnée par:\\
\[u(t)= \ds\sqrt{3}\;\cos \left (t\right ) - \sin \left (t\right ).\]

\begin{enumerate}
\item  Montrer que le signal $u$ peut s'écrire pour tout $t$ réel sous la forme :\\
\[u(t) = 2\;\cos \left ( t+ \dfrac{\pi}{6} \right ).\]

\item Résoudre dans $[0\;;\;\pi[$, l'équation $u(t)=1$.
\end{enumerate}


On pourra s'aider du demi-cercle trigonométrique ci-dessous :

\begin{center}
\psset{unit=1.5cm}
\begin{pspicture}(-4,-1)(4,4.5)
%\psgrid[subgriddiv=0,gridlabels=0,gridcolor=zzzzzz](0,0)(-5,-5)(5,5)
\psset{dotstyle=*,dotsize=3pt 0,linewidth=0.8pt,arrowsize=3pt 2,arrowinset=0.25}
%\psaxes[xAxis=true,yAxis=true,Dx=1,Dy=1,labels=none,ticksize=0,subticks=2, , arrowsize=4pt 4]{->}(0,0)(-5,-5)(5,5)
%%%
\psarc(0,0){4}{0}{180}
\psdots(4;0)(4;30)(4;45)(4;60)(4;90)(4;120)(4;135)(4;150)(4;180)
\psline(4;0)(4;180) \psline(0,0)(4;90)
%%%%%
\psset{linestyle=dotted,linewidth=1.3pt,dotstyle=+,dotscale=1.5}
{\boldmath
%%%
\psline(2,0)(4;60)(4;120)(-2,0)
\psdots(2,0)(0,3.464)(-2,0)
\uput[d](2,0){$\frac{1}{2}$} 
\uput*[l](0,3.464){$\frac{\sqrt{3}}{2}$} 
\uput[d](-2,0){$-\frac{1}{2}$}
\uput[60](4;60){$\frac{\pi}{3}$} \uput[120](4;120){$\frac{2\pi}{3}$} 
%%%
\psline(2.828,0)(4;45)(4;135)(-2.828,0)
\psdots(2.828,0)(0,2.828)(-2.828,0)
\uput[d](2.828,0){$\frac{ \sqrt{2}}{2}$} 
\uput*[l](0,2.828){$\frac{\sqrt{2}}{2}$} 
\uput[d](-2.828,0){$-\frac{ \sqrt{2}}{2}$}
\uput[45](4;45){$\frac{\pi}{4}$} \uput[135](4;135){$\frac{3\pi}{4}$} 
%%%
\psline(3.464,0)(4;30)(4;150)(-3.464,0)
\psdots(3.464,0)(0,2)(-3.464,0)
\uput[d](3.464,0){$\frac{ \sqrt{3}}{2}$} 
\uput*[l](0,2){$\frac{1}{2}$} 
\uput[d](-3.464,0){$-\frac{ \sqrt{3}}{2}$}
\uput[30](4;30){$\frac{\pi}{6}$} \uput[150](4;150){$\frac{5\pi}{6}$} 
%%%
\uput[r](4;0){\small $0$} \uput[u](4;90){$\frac{\pi}{2}$} 
\uput[l](4;180){\small $\pi$}  \uput[d](0,0){\small $0$}
}%% fin du \boldmath
\end{pspicture}
\end{center}

\bigskip


