
\medskip

\begin{enumerate}
\item Pour $z_{\text{A}}$ : puisque un argument est $\frac{5\pi}{6} = \pi - \frac{\pi}{6}$ : A appartient au deuxième quadrant, donc 2 et 4 sont possibles.

Pour $z_{\text{B}}$ : puisque un argument est $-\frac{2\pi}{3} = - \pi + \frac{\pi}{3}$ B appartient au troisième cadran : il ne reste que la figure 2. 

\item $\dfrac{z_{\text A}}{z_{\text B}} = \dfrac{\e^{\text{i} \frac{5\pi}{6}}}{\e^{-\text{i} \frac{2\pi}{3}}} = \e^{\text{i} \left(\frac{5\pi}{6} + \frac{2\pi}{3}\right)} = \e^{\text{i} \left(\frac{5\pi}{6} + \frac{4\pi}{6}\right)} = \e^{\text{i} \left(\frac{9\pi}{6}\right)} = \e^{\text{i} \frac{3\pi}{2}} = \e^{\text{i} \left(2\pi - \frac{\pi}{2}\right)}  = \e^{\text{i} \frac{-\pi}{2}}$.
\end{enumerate}

\bigskip

