
\begin{center}
\textbf{Loi de refroidissement de Newton}
\end{center}

La loi de refroidissement de Newton indique que la vitesse de refroidissement d'un matériau est proportionnelle à la différence entre la température $\theta$ (en degré Celsius) de ce matériau à l'instant $t$ (en minute) et la température A constante de l'air ambiant.

Cela se traduit par la relation :
\[\theta'(t) = \alpha(\theta(t) - A),\]

où $\theta$ est la fonction définie et dérivable sur l'intervalle $[0~;~ +\infty[$ modélisant la température du matériau en fonction du temps $t$, en prenant comme origine du temps l'instant où la pièce en acier est mise à refroidir.

La valeur du coefficient $\alpha$, qui est négatif, dépend du matériau.

Une pièce en acier, initialement à la température de $600\degres$ C, est mise à refroidir à l'air libre dans une pièce à $20\degres$ C. Pour cet acier, $\alpha$ vaut $-0,1$.

\medskip

\emph{Indiquer la lettre de la réponse exacte. Aucune justification n'est demandée.}

\medskip

La fonction $\theta$ est solution de l'équation différentielle :

\begin{tabularx}{\linewidth}{*{2}{X}}
\textbf{a.~~}$y= -0,1y' +2$&\textbf{b.~~}$y= -0,1y' +20$\\
\textbf{c.~~}$y' = -0,1y + 2$&\textbf{d.~~}$y' = 0,1y + 20$
\end{tabularx}

\bigskip

