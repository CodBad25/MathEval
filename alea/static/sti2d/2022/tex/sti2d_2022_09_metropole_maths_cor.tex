
\medskip

\textbf{Question 1}

\medskip

\begin{enumerate}
\item $\ln(35) = \ln (5 \times 7) = \ln 5 + \ln 7 \rightarrow$ réponse \textbf{b.}

\item $\e^{20} = \e^{5 + 15} = \e^{5} \times \e^{15} \rightarrow$ réponse \textbf{d.}
\end{enumerate}

\bigskip

\textbf{Question 2}

\medskip

\begin{enumerate}
\item On a $f(0) = 28 \ln 1 + 70 = 70$, soit 70 battements par minutes.

\item Il faut résoudre dans l'intervalle [0~;~100], l'équation $f(t) = 185$, soit :
\begin{align*}
&28\,\ln (x + 1) + 70 = 185 \\
\iff &28\,\ln (x + 1) = 115 \\
\iff &\ln (x + 1) = \dfrac{115}{28} \\
\iff &x + 1 = \text{e}^{\frac{115}{28}} \\
\iff &x = \text{e}^{\frac{115}{28}} - 1 \\
\iff &x \approx 59,77,
\end{align*}
soit 60~(m), au mètre près.
\end{enumerate}

\bigskip

\textbf{Question 3}

\medskip

\begin{enumerate}
\item On sait que les solutions de l'équation $y' = - 0,2y$ sur $[0; +\infty[$ sont les fonctions du type $t \longmapsto  \alpha \text{e}^{- 0,2t}$, avec $\alpha \in \R$ ;

Une fonction constante $y = k$ est solution de l'équation différentielle si :
\[y' = 0 = - 0,2k + 44 \iff  44 = 0,2k \iff k = 220.\]

Les solutions de l'équation différentielle sont donc les fonctions :

\[ t \longmapsto \alpha \text{e}^{-0,2t}  + 220,\: \alpha \in \R.\]

\item Puisque la température est modélisée par la fonction $f$, définie sur $[0; +\infty[$  par $f(t) = \alpha \text{e}^{-0,2t}  + 220$, avec $\alpha \in \R$ et que :
\[f(0) = 25 \iff \alpha \text{e}^0  + 220  = 25 \iff \alpha + 220 = 25 \iff \alpha = - 195,\]

on a donc :
\[f(t) = 220 - 195\text{e}^{-0,2t}.\]
\end{enumerate}

\bigskip

\textbf{Question 4}

\medskip

\begin{enumerate}
\item On a : $|z| = \sqrt{\left(\sqrt{3}\right)^2 + (- 1)^2} = \sqrt{3 + 1} = \sqrt{4} = 2$ ;

Donc : $z = 2\left(\dfrac{\sqrt{3}}{2} + \text{i} \dfrac{-1}{2}\right) = 2 \left(\cos \frac{-\pi}{6} + \text{i}\sin \frac{-\pi}{6}\right) = 2 \text{e}^{-\text{i}\frac{\pi}{6}}$.

\item Tout d'abord : $z' = -\sqrt{2}\e^{\text{i}\frac{\pi}{4}} = \sqrt{2}\text{e}^{\text{i}\pi}\times \e^{\text{i}\frac{\pi}{4}} = \sqrt{2}\text{e}^{\text{i}\frac{5\pi}{4}}$.

Donc : $\dfrac{z}{z'} = \dfrac{2 \text{e}^{-\text{i}\frac{\pi}{6}}}{\sqrt{2}\text{e}^{\text{i}\frac{5\pi}{4}}} =  \sqrt{2}\text{e}^{\text{i}\left(-\frac{\pi}{6} - \frac{5\pi}{4}\right)} = \sqrt{2}\text{e}^{-\text{i}\left(\frac{17\pi}{12}\right)} = \sqrt{2}\text{e}^{\text{i}\frac{7\pi}{12}}$.
\end{enumerate}

\bigskip

\textbf{Question 5}

\medskip

Il faut trouver le temps $t$ au bout duquel :

\begin{align*}
&N(t) = \dfrac{N(0)}{2} = N(0) \e^{-0,086t} \\
\iff &\dfrac12 = \e^{-0,086t} \\
\iff &\ln \dfrac12 = - 0,086t \\
\iff &- \ln 2 = - 0,086t \\
\iff &t = \dfrac{\ln 2}{0,086} \\
\iff &t \approx 8,059,
\end{align*}
soit 8 jours au jour près.

\bigskip

\textbf{Question 6}

\medskip

\begin{enumerate}
\item Sur $\R$, on a :

$f(x) = \sin(x) + \cos(x)$ ;

$f'(x) = \cos(x) - \sin(x)$ ;

$f''(x) = - \sin (x) - \cos (x)$.

Donc $f(x) + f''(x) = 0 \iff f$ est solution d' équation différentielle $y'' + y = 0$ sur $\R$.

\item Soit la fonction $g$ définie par : $g(x) = \sqrt{2}\cos\left(x-\frac{\pi}{4}\right)$.

D'après le formulaire :
\begin{align*}
g(x) &= \sqrt{2}\left(\cos (x) \cos \dfrac{\pi}{4} + \sin (x) \sin \dfrac{\pi}{4} \right) \\
&= \sqrt{2}\left( \cos (x) \times \dfrac{\sqrt{2}}{2} + \sin (x) \times \dfrac{\sqrt{2}}{2} \right) \\
&= \cos (x) \times \dfrac{2}{2} + \sin (x) \times \dfrac{2}{2} \\
&= \cos (x) + \sin (x) \\
&= f(x).
\end{align*}
On a donc pour tout $x \in \R, \: f(x) = \sqrt{2}\cos\left(x-\dfrac{\pi}{4}\right)$.
\end{enumerate}

\bigskip


