
\medskip

\begin{enumerate}
\item On a $g(0) = -1$. Puisque $A$ est commun aux deux courbes, on a aussi $f(0) = -1$; mais par le calcul, $f(0) = a + b \mathrm e^0 = a+b$. Il s'ensuit l'égalité $a + b = -1$.
\item
	\begin{enumerate}
		\item On a, pour tout réel $x$, $g'(x) = 2 x - 4$, puis $g'(0) = 2\times 0 - 4 = -4$.
		\item Puisque $\mathcal C_f$ et $\mathcal C_g$ ont la même tangente $T$ en A, les dérivées évaluées en $0$ doivent être identiques; ainsi on a $f'(0) = -4$. Mais par le calcul, $f'(x) = b \mathrm e^x$ pour tout réel $x$, donc $f'(0) = b\mathrm e^0 = b$. Il vient alors:
\begin{itemize}
\item par identification de $f'(0)$: $b = -4$;
\item puis, comme $a+ b = -1$, on a $a - 4 = -1$, soit $a = -1+4 = 3$.
\end{itemize}
	\end{enumerate}
\end{enumerate}

\bigskip

