
\medskip

\begin{enumerate}
\item La fonction $g$ est une somme de fonctions dérivables sur $]0~;~+\infty[$, elle est donc dérivable sur $]0~;~+\infty[$, et pour tout $x$ de $]0; +\infty[$, on a $g'(x) = \dfrac 1 2 (2x) - \dfrac 1 x = x - \dfrac 1 x$.

Or, on a, pour tout $x$ de  $]0~;~+\infty[$:
\[
\dfrac{(x-1)(x+1)}{x} = \dfrac{x^2-1}{x}
= \dfrac {x^2} x - \dfrac 1 x
= x - \dfrac 1 x = g'(x).
\]

\item On étudie le signe de $g'(x)$ grâce à sa forme factorisée donné à la question précédente. Puisque l'on définit la fonction sur  $]0~;~+\infty[$, le dénominateur de $g'$ est positif, et le facteur $x + 1$ est aussi positif. Il s'ensuit que $g'(x)$ a le même signe que $x - 1$, c'est-à-dire:
\begin{itemize}
\item négatif sur $]0~;~1[$;
\item positif sur $]1~;~+\infty[$;
\item nul pour $x = 1$.
\end{itemize}
Par conséquent, $g$ admet pour sens de variation les choses suivantes:
\begin{itemize}
\item $g$ est décroissante sur $]0~;~1[$;
\item $g$ est croissante sur $]1~;~+\infty[$;
\item et $g$ admet un minimum en $1$.
\end{itemize}
Ce minimum vaut $g(1) = \dfrac 1 2 \times 1^2 - \ln (1) = \dfrac 1 2 - 0 = \dfrac 1 2$.
\end{enumerate}

\bigskip

