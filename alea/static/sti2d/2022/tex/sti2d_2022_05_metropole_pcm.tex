
\medskip

\begin{center}
\textbf{Modèle de la vitesse de refroidissement d'un lait écrémé}
\end{center}

Dans le domaine de l'agroalimentaire, la question du refroidissement des produits préparés peut être cruciale. On peut citer par exemple la problématique de la durée de refroidissement du lait produit dans une ferme : afin d'éviter la prolifération microbienne, il convient de minimiser cette durée de refroidissement.

Afin d'étudier l'évolution de la température d'une masse de liquide en contact avec l'atmosphère d'une pièce en fonction du temps, l'expérience suivante est réalisée. Une masse de lait écrémé $m = 150$ g est chauffée à une température de 63,4~\degres C. On laisse ensuite le lait se refroidir à l'air libre.

Pendant toute la durée de l'expérience, la température de l'air de la pièce reste constante et inférieure à celle du lait.

La température du lait, exprimée en degré Celsius, en fonction du temps $t$, exprimé en minute, est modélisée par la fonction $T$ définie sur $[0~;~ +\infty[$ par : 
\[T(t) = 37 \times \text{e}^{- \frac{20t}{459}} + 26,4.\]

\begin{enumerate}[start=4]
\item Calculer $T(0)$ et interpréter ce résultat dans le contexte de l'exercice.

\item Déterminer $\displaystyle\lim_{t \to + \infty} T(t)$.

Selon ce modèle, quelle est la température de l'air de la pièce ? Justifier.

\item Selon ce modèle, au bout de combien de temps la température du lait vaut-elle $40\degres C$ ? Donner le résultat en minute et seconde.
\end{enumerate}

\bigskip


