
\medskip

\begin{enumerate}
\item $\left |z\right | = \left |-1+\text{i}\right | = \ds\sqrt{(-1)^2+1^2}=\ds\sqrt{2}$.

Donc :

$z=\ds\sqrt{2}\left (-\dfrac{1}{\sqrt{2}} +\text{i} \dfrac{1}{\sqrt{2}}\right )=\ds\sqrt{2}\left (-\dfrac{ \sqrt{2}}{2} + \text{i}\dfrac{ \sqrt{2}}{2}\right )$.

Le nombre $z$ a pour argument le réel $\theta$ tel que $\cos \left (\theta\right ) = -\dfrac{ \sqrt{2}}{2}$ et $\sin \left (\theta\right ) = \dfrac{ \sqrt{2}}{2}$ ; donc on peut prendre $\theta = \dfrac{3\pi}{4}$.

L'écriture exponentielle de $z$ est donc $\sqrt{2}\e^{\text{i}\frac{3\pi}{4}}$.

\item 
$z^4 = \left (\sqrt{2} \e^{\text{i}\frac{3\pi}{4}}\right )^4
= \left (\sqrt{2} \right )^4 \times \e^{\text{i}\frac{3\pi}{4}\times 4}
= 4\times \e^{3\pi}
= 4\times(-1)
=-4$.

Donc la partie imaginaire de $z^4$ est 0.
\end{enumerate}

\bigskip

