
\medskip

\textbf{Dans cet exercice, seulement 4 questions au choix parmi les 6 questions proposées sont à traiter. Toutes ces questions sont indépendantes les unes des autres.}

\bigskip

\textbf{Question 1}

\medskip

Une entreprise réalise des bouchons par injection plastique. On modélise la température (en degré Celsius) d'un bouchon plastique à l'issue de sa fabrication, en fonction du temps $t$ (en seconde) par l'équation différentielle :

\[y' = - 0,1y + 7.\]

Montrer que la fonction $\theta$ définie par $\theta(t) = 80\text{e}^{-0,1t} + 70$ sur l'intervalle $[0~;~ +\infty[$ est solution de cette équation différentielle et qu'elle vérifie la condition initiale $\theta(0) = 150$.

\bigskip

\textbf{Question 2}

\medskip

Soit le nombre complexe $z = -1 + \text{i}$.

\medskip

\begin{enumerate}
\item Montrer que $z = \sqrt{2}\text{e}^{\text{i}\frac{3\pi}{4}}$.
\item Quelle est la partie imaginaire de $z^4$ ? Justifier.
\end{enumerate}

\bigskip

\textbf{Question 3}

\medskip

Une voiture électrique, dont l'accumulateur est totalement déchargé, est branchée à une borne de rechargement. L'énergie emmagasinée par l'accumulateur (en kilowattheure), notée $E$, peut être modélisée en fonction du temps $t$ écoulé (en heure) par la fonction $E$ définie pour $t \in [0~;~ +\infty[$ par :

\[E(t) = 18\left(1 - \text{e}^{-0,45t}\right).\]

On admet que cette voiture a une énergie de stockage limitée à 18 kWh.

Déterminer l'instant $t_0$, arrondi à la minute, à partir duquel la moitié de cette énergie de stockage limite a été emmagasinée.

\bigskip

\textbf{Question 4}

\medskip

On considère une fonction $f$ dérivable sur $]0~;~ +\infty[$ dont la fonction dérivée $f'$ est, donnée, pour tout $x \in ]0~;~ +\infty[$ , par :

\[f'(x) = \dfrac{-3x+2}{x}.\] 

Étudier le sens de variation de la fonction $f$ sur $]0~;~ +\infty[$.

\newpage

\textbf{Question 5}

\medskip

On considère l'équation : 

\[3 \ln(x) - \ln (x + 30) = 2\ln (5),\]

où $x$ appartient à l'intervalle $]0~;~+\infty[$.

Donner, parmi les quatre propositions suivantes, la solution de cette équation.

\begin{center}
\begin{tabularx}{\linewidth}{*{4}{X}}
\textbf{a.~~} 0&\textbf{b.~~}$\text{e}^{-5}$&\textbf{c.~~} 10&\textbf{d.~~} 20
\end{tabularx}
\end{center}

\bigskip

\textbf{Question 6}

\medskip

Une société de peinture utilise, dans le cadre de son activité, une nacelle élévatrice (dite \og nacelle à ciseaux\fg).

On note $h(t)$ la hauteur (en mètre) de la nacelle à l'instant $t$ (en seconde) suivant la mise en route.

On suppose que $h$ est la fonction de la variable réelle $t$ définie et dérivable sur $[0~;~+\infty[$ d'expression :

\[h(t) = - 15\text{e}^{-0,2t} +18.\]

\hfill D'après: \emph{https://www.haulotte.fr/produitlh18-sx}

\begin{enumerate}
\item Déterminer la hauteur initiale de la nacelle.
\item Déterminer la limite de la fonction $h$ en $+ \infty$. Interpréter cette limite dans le contexte de l'exercice.
\end{enumerate}

\bigskip


