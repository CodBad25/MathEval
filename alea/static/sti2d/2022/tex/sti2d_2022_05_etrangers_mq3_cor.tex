
\medskip

On a $x^2 - 1 = (x + 1)(x - 1)$, donc $\ln \left(x^2 - 1\right) = \ln [(x + 1)(x - 1)] = \ln (x + 1) + \ln (x - 1)$.

Les logarithmes de cette équation sont définis si :

$x + 1 > 0, \quad x - 1 > 0, \quad x > 0$, soit si $x > 1$.

Conclusion : les solutions sont à chercher dans l'intervalle$]1\;;\; +\infty[$.

Tout d'abord $\ln \left (x^2-1\right ) - \ln\left (0,5\right ) = \ln \dfrac{x^2 - 1}{0,5} = \ln 2\left(x^2 - 1 \right)$.

On peut donc écrire :

$\ln\left (x - 1\right ) + \ln \left (x + 1\right ) - \ln\left (x\right ) = \ln \left (x^2-1\right ) - \ln\left (0,5\right ) \iff \ln x(x - 1)(x + 1) = \ln 2\left(x^2 - 1 \right)$ et par croissance de la fonction logarithme népérien :

$x\left(x^2 - 1\right)= 2\left(x^2 - 1 \right) \iff  x\left(x^2 - 1\right) - 2\left(x^2 - 1 \right) = 0 \iff \left(x^2 - 1\right)(x - 2) = 0 \iff$

$ (x + 1)(x - 1)(x - 2) = 0$.

Il y a donc trois possibilités : $x = - 1$, ou $x = 1$ ou $x = 2$, mais seul $2 \in ]1~;~+ \infty[$. Donc $S = \{2\}$.

\bigskip

