
\medskip

\textbf{Question 1}

\medskip

\begin{enumerate}
\item On a :
\begin{align*}
4\ln(3)+2\ln(5) &= \ln\bigl(3^4\bigr) + \ln\bigl(5^2\bigr)\\
&= \ln\bigl(3^4\times 5^2\bigr)\\
&= \ln(\np{2025}).
\end{align*}

\item On a :
\begin{align*}
A &= 2\ln\left(\mathrm e^4\right) - 3\ln\left(\frac 1 {\mathrm e}\right)\\&= 2\times 4 \ln(\mathrm e ) - 3\times (- \ln(\mathrm e)\\
&= 2\times 4  - 3\times (-1)\\
A &= 11.
\end{align*}
\end{enumerate}

\bigskip

\textbf{Question 2}

\medskip

\begin{enumerate}
\item On écrit, en multipliant numérateur et dénominateur par $- \text{i}$ :
\[ z = \dfrac{(-1+ \text{i})(-3\text{i})}{(3\text{i})(-3\text{i})} = \dfrac{-(-3\text{i}) - 3\text{i}^2}{-9 \text{i}^2} = \dfrac {3\text{i} + 3}{9} = \dfrac 3 9 + \dfrac 3 9 \text{i} = \dfrac 1 3 + \dfrac 1 3 \text{i}.\]

\item Le module de $z$ est :
\[ |z| = \sqrt{\left(\dfrac 1 3\right)^2 + \left(\dfrac 1 3\right)^2}
= \sqrt{\dfrac 1 9 + \frac 1 9}
= \sqrt{\dfrac 2 9} 
= \dfrac{\sqrt 2} 3.\]

Soit $u = \dfrac z {|z|}$. Ce nombre complexe, de module $1$, a le même argument que $z$.

On a :
\begin{align*}
u &= \dfrac {\dfrac 1 3 + \dfrac 1 3 \text{i}}{\dfrac {\sqrt 2} 3}\\
&= \left(\dfrac 1 3 + \dfrac 1 3  \text{i} \right) \dfrac 3 {\sqrt 2}\\
&= \dfrac 1 3 \times \dfrac 3 {\sqrt 2} + \dfrac 1 3  \text{i} \times \dfrac 3 {\sqrt 2}\\
&= \dfrac 1 {\sqrt 2} + \dfrac 1 {\sqrt 2}  \text{i}\\
&= \dfrac {\sqrt 2} 2 + \dfrac {\sqrt 2} 2  \text{i}.
\end{align*}
Un argument de $u$ est donc $\theta$ tel que :
\begin{list}{\textbullet}{}
\item $\cos \theta = \cos \dfrac a {|z|} = \dfrac{\dfrac 1 3}{\dfrac{\sqrt 2} 3} = \dfrac{1}{\sqrt 2} = \dfrac {\sqrt 2} 2$,
\item $\sin \theta = \sin \dfrac b {|z|} = \dfrac{\dfrac 1 3}{\dfrac{\sqrt 2} 3} = \dfrac{1}{\sqrt 2} = \dfrac{\sqrt 2} 2$.
\end{list}
On prend $\theta = \dfrac \pi 4$.

Ainsi, sous forme exponentielle, on a $z = \dfrac{\sqrt 2} 3 \mathrm e^{\frac \pi 4  \text{i}}$.
\end{enumerate}

\bigskip

\textbf{Question 3}

\medskip

\begin{enumerate}
\item On a $2y' + y = 0$ si et seulement si $y' + \dfrac 1 2 y = 0$, ou bien si et seulement si $y' = - \dfrac 1 2 y$. Les solutions $y$ de cette équation différentielle sont donc de la forme $y = C \mathrm e^{-\frac 1 2 x}$, avec $C \in \R$.

\item Comme $\mathcal C_f$ passe par le point $A: \bigl(\ln(9)~;~1\bigr)$, on a $f\bigl(\ln(9)\bigr) = 1$. En remplaçant $x$ par $\ln(9)$ et $y$ par $1$ dans l'expression générale des solutions, il vient
\begin{align*}
1 &= C \mathrm e^{-\frac 1 2 \ln(9)}\\
1 &= C \mathrm e^{- \ln\bigl(\sqrt 9\bigr)}\\
1 &= C \mathrm e^{\ln\left(\frac 1 {\sqrt 9}\right)}\\
1 &= C \dfrac 1 {\sqrt 9}\\
\sqrt 9 &= C
\end{align*}
d'où l'on tire $C = 3$, et par suite $f(x) = 3 \mathrm{e}^{-\frac 1 2 x}$.
\end{enumerate}

\bigskip

\textbf{Question 4}

\medskip

\begin{enumerate}
\item On a $g(0) = -1$. Puisque $A$ est commun aux deux courbes, on a aussi $f(0) = -1$; mais par le calcul, $f(0) = a + b \mathrm e^0 = a+b$. Il s'ensuit l'égalité $a + b = -1$.
\item
	\begin{enumerate}
		\item On a, pour tout réel $x$, $g'(x) = 2 x - 4$, puis $g'(0) = 2\times 0 - 4 = -4$.
		\item Puisque $\mathcal C_f$ et $\mathcal C_g$ ont la même tangente $T$ en A, les dérivées évaluées en $0$ doivent être identiques; ainsi on a $f'(0) = -4$. Mais par le calcul, $f'(x) = b \mathrm e^x$ pour tout réel $x$, donc $f'(0) = b\mathrm e^0 = b$. Il vient alors:
\begin{itemize}
\item par identification de $f'(0)$: $b = -4$;
\item puis, comme $a+ b = -1$, on a $a - 4 = -1$, soit $a = -1+4 = 3$.
\end{itemize}
	\end{enumerate}
\end{enumerate}

\bigskip

\textbf{Question 5}

\medskip

\begin{enumerate}
\item La fonction $g$ est une somme de fonctions dérivables sur $]0~;~+\infty[$, elle est donc dérivable sur $]0~;~+\infty[$, et pour tout $x$ de $]0; +\infty[$, on a $g'(x) = \dfrac 1 2 (2x) - \dfrac 1 x = x - \dfrac 1 x$.

Or, on a, pour tout $x$ de  $]0~;~+\infty[$:
\[
\dfrac{(x-1)(x+1)}{x} = \dfrac{x^2-1}{x}
= \dfrac {x^2} x - \dfrac 1 x
= x - \dfrac 1 x = g'(x).
\]

\item On étudie le signe de $g'(x)$ grâce à sa forme factorisée donné à la question précédente. Puisque l'on définit la fonction sur  $]0~;~+\infty[$, le dénominateur de $g'$ est positif, et le facteur $x + 1$ est aussi positif. Il s'ensuit que $g'(x)$ a le même signe que $x - 1$, c'est-à-dire:
\begin{itemize}
\item négatif sur $]0~;~1[$;
\item positif sur $]1~;~+\infty[$;
\item nul pour $x = 1$.
\end{itemize}
Par conséquent, $g$ admet pour sens de variation les choses suivantes:
\begin{itemize}
\item $g$ est décroissante sur $]0~;~1[$;
\item $g$ est croissante sur $]1~;~+\infty[$;
\item et $g$ admet un minimum en $1$.
\end{itemize}
Ce minimum vaut $g(1) = \dfrac 1 2 \times 1^2 - \ln (1) = \dfrac 1 2 - 0 = \dfrac 1 2$.
\end{enumerate}

\bigskip

\textbf{Question 6}

\medskip

\begin{enumerate}
\item Soit $z$ le nombre complexe défini par $z = 1 + \text{i} \sqrt 3$.

Ce nombre complexe a pour module $|z| = \sqrt{1^2 + \bigl(\sqrt 3\bigr)^2} = \sqrt {1 + 3} = \sqrt 4 = 2$.

Par ailleurs, le nombre complexe :
\[ \dfrac z {|z|} = \frac{1 + \text{i} \sqrt 3} 2 = \dfrac 1 2 + \text{i} \dfrac{\sqrt 3} 2, \]
est de module $1$ et admet pour argument un nombre $\psi$ tel que $\cos \psi = \dfrac 1 2$ et $\sin \psi = \dfrac{\sqrt 3} 2$, soit $\psi = \dfrac \pi 3$ comme argument principal. Ce nombre $\psi$ est aussi un argument de $z$. 

Alors, il suffit de prendre $U_{\text{max}} = |z| = 2$, $\omega = 50$ et $\varphi = - \psi = - \dfrac \pi 3$ pour obtenir la bonne expression. En effet :
\begin{align*}
2\cos\left(50 t - \dfrac \pi 3\right) &= 2 \left(\cos(50t) \cos\left(\dfrac \pi 3\right) + \sin(50t) \sin\left(\dfrac \pi 3\right)\right)\\
&= 2 \left(\cos(50t) \times \dfrac 1 2 + \sin(50t) \times \dfrac {\sqrt 3} 2 \right)\\
&=\cos(50t) \times 1 + \sin(50t) \times \sqrt 3\\
&= U(t).
\end{align*}

\item Il suffit de calculer $f = \dfrac {50}{2\pi} = \dfrac {25} {\pi} \simeq 8$ Hz, une fois arrondi à l'unité.
\end{enumerate}

\bigskip


