
\medskip

\textbf{Dans cet exercice, seulement 4 questions au choix parmi les 6 questions proposées sont à traiter. Toutes ces questions sont indépendantes les unes des autres.}

\bigskip

\textbf{Question 1}

\medskip 

\emph{Pour les deux questions suivantes, indiquer la lettre de la réponse exacte. Aucune justification n'est demandée.}

\medskip

\begin{enumerate}
\item  Le nombre $\ln(35)$ est égal à :

\begin{tabularx}{\linewidth}{*{4}{>{\centering \arraybackslash}X}}
\textbf{a.}~~ $\ln(5) \times \ln(7)$& \textbf{b.}~~$\ln(5) + \ln(7)$ &\textbf{c.}~~$\ln(30) + \ln(5)$ &\textbf{d.}~~$\ln(30) \times \ln(5)$\\
\end{tabularx}

\item Le nombre $\e^{20}$ est égal à :

\begin{tabularx}{\linewidth}{*{4}{>{\centering \arraybackslash}X}}
\textbf{a.}~~ $\e^4\times\e^5 $& \textbf{b.}~~$\e^4 + \e^5$ &\textbf{c.}~~$ \e^5+\e^{15}$ &\textbf{d.}~~$ \e^5\times\e^{15}$\\
\end{tabularx}
\end{enumerate}

\bigskip

\textbf{Question 2}

\medskip

Lors d'une course, on a mesuré la fréquence cardiaque d'un coureur de \np[m]{100}.

Cette fréquence cardiaque, en battements par minute, est modélisée par la fonction $f$ définie sur [0; 100] par $f(x) = 28\,\ln (x + 1) + 70$ où $x$ est la distance parcourue, en mètre, depuis le départ de la course.
\begin{enumerate}
\item Selon ce modèle, quelle est la fréquence cardiaque de ce coureur au départ de la course ?
\item Selon ce modèle, au bout de combien de mètres la fréquence cardiaque de ce sportif est-elle égale à $185$ battements par minute ? Arrondir à l'unité.
\end{enumerate}

\bigskip

\textbf{Question 3}

\medskip

La température d'un four, exprimée en degré Celsius, en fonction du temps $t$, exprimé en minute, est modélisée par une fonction $f$ définie et dérivable sur $[0~;~+\infty[$, solution de l'équation différentielle (E) : $y' = -0,2 y + 44$.

\begin{enumerate}
\item Déterminer les solutions de cette équation différentielle sur [0; $+\infty$[.
\end{enumerate}
On suppose que la température initiale du four est 25\textcelsius.
\begin{enumerate}[start=2]
\item En prenant $f(0) = 25$, donner une expression de $f(t)$, pour tout $t$ de [0; $+\infty$[.
\end{enumerate}

\bigskip

\textbf{Question 4}

\medskip

On note i le nombre complexe de module $1$ et d'argument $\frac{\pi}{2}$ : 

On pose $z = \sqrt{3}-\text{i}$ et $z' = -\sqrt{2}\e^{\text{i}\frac{\pi}{4}}$.

\begin{enumerate}
\item  Déterminer la forme exponentielle de $z$. Détailler les calculs.

\item En déduire la forme exponentielle de $\dfrac{z}{z'}$.
\end{enumerate}

\bigskip

\textbf{Question 5}

\medskip

L'iode 131 est un élément radioactif qui se désintègre selon la loi $N(t) = N(0) \e^{-0,086t}$, où $N(0)$ est le nombre de noyaux au début de l'observation et $N(t)$ le nombre de noyaux à l'instant $t$, exprimé en jour.

Déterminer le temps au bout duquel la moitié des noyaux d'iode 131 se sont désintégrés (demi-vie). On donnera le résultat en nombre de jours arrondi à l'unité.

\bigskip

\textbf{Question 6}

\medskip

On considère la fonction $f$ définie sur $\R$ par $f(x) = \sin(x) + \cos(x)$,
\begin{enumerate}
\item  Montrer que $f$ est solution de l'équation différentielle $y'' + y = 0$.
\item  Montrer que, pour tout nombre réel $x$, $f(x) =\sqrt{2}\cos\left(x-\dfrac{\pi}{4}\right)$.
\end{enumerate}

\begin{center}
\fbox{
\begin{minipage}{0.9\textwidth}
\textbf{Rappel :} pour $a$ et $b$ deux réels, on a les formules suivantes :
\[\bullet~~\cos(a + b) = \cos (a) \cos(b) - \sin(a) \sin(b)\]
\[\bullet~~\cos(a - b) = \cos (a) \cos(b) + \sin(a) \sin(b)\]
\[\bullet~~\sin(a + b) = \sin(a) \cos(b) + \cos (a) \sin(b)\]
\[\bullet~~\sin(a - b) = \sin(a) \cos(b) - \cos (a) \sin(b)\]
\end{minipage}}
\end{center}

\bigskip


