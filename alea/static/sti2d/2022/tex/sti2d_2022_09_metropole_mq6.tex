
\medskip

On considère la fonction $f$ définie sur $\R$ par $f(x) = \sin(x) + \cos(x)$,
\begin{enumerate}
\item  Montrer que $f$ est solution de l'équation différentielle $y'' + y = 0$.
\item  Montrer que, pour tout nombre réel $x$, $f(x) =\sqrt{2}\cos\left(x-\dfrac{\pi}{4}\right)$.
\end{enumerate}

\begin{center}
\fbox{
\begin{minipage}{0.9\textwidth}
\textbf{Rappel :} pour $a$ et $b$ deux réels, on a les formules suivantes :
\[\bullet~~\cos(a + b) = \cos (a) \cos(b) - \sin(a) \sin(b)\]
\[\bullet~~\cos(a - b) = \cos (a) \cos(b) + \sin(a) \sin(b)\]
\[\bullet~~\sin(a + b) = \sin(a) \cos(b) + \cos (a) \sin(b)\]
\[\bullet~~\sin(a - b) = \sin(a) \cos(b) - \cos (a) \sin(b)\]
\end{minipage}}
\end{center}



