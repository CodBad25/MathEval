
\medskip

Une société de peinture utilise, dans le cadre de son activité, une nacelle élévatrice (dite \og nacelle à ciseaux\fg).

On note $h(t)$ la hauteur (en mètre) de la nacelle à l'instant $t$ (en seconde) suivant la mise en route.

On suppose que $h$ est la fonction de la variable réelle $t$ définie et dérivable sur $[0~;~+\infty[$ d'expression :

\[h(t) = - 15\text{e}^{-0,2t} +18.\]

\hfill D'après: \emph{https://www.haulotte.fr/produitlh18-sx}

\begin{enumerate}
\item Déterminer la hauteur initiale de la nacelle.
\item Déterminer la limite de la fonction $h$ en $+ \infty$. Interpréter cette limite dans le contexte de l'exercice.
\end{enumerate}

