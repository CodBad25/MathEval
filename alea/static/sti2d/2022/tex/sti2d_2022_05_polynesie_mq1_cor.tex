
\medskip

$\theta(t) = 80\e^{-0,1t} + 70$ donc $\theta'(t)=80\times(-0,1)\e^{-0,1t} = -8\e^{-0,1t}$.

Or $-0,1 \theta(t)+7 = -0,1\left (80\e^{-0,1t} + 70\right ) + 7 = -8\e^{-0,1t}-7+7 = -8\e^{-0,1t}= \theta'(t)$.

Donc la fonction $\theta$ est solution de l'équation différentielle $y' = - 0,1y + 7$.

$\theta(0)=80\e^{0}+70 = 80+70=150.$

On peut donc dire que la fonction $\theta$ est solution de cette équation différentielle et qu'elle vérifie la condition initiale $\theta(0) = 150$.

\bigskip

