
\medskip

\begin{enumerate}
\item $f(0) = 6,571- 2,671 \e^{-\frac{0}{261}} = 6,571 - 2,671\e^0 = 6,571 - 2,671 = 3,9$. 

C'est le pH initial.

\item 
	\begin{enumerate}
		\item On voit sur le graphique que 5 a pour antécédent 140. Or $140 = 2 \times 60 + 20$.

Le pH est à 5 au bout de 2 heures 20 minutes.

\begin{center}
\psset{xunit=0.0275cm,yunit=1.8cm,labelFontSize=\scriptstyle,comma=true,arrowsize=2pt 3}
\begin{pspicture}(-2,-0.5)(500,3.3)
\multido{\n=0+20}{25}{\psline[linewidth=0.75pt,linecolor=lightgray](\n,0)(\n,3.5)}
\multido{\n=0+0.1}{36}{\psline[linewidth=0.75pt,linecolor=lightgray](0,\n)(490,\n)} 
\psaxes[linewidth=0.95pt,Dx=100,Dy=0.5,Oy=3]{->}(0,0)(0,-0.1)(490,3.5)
\psplot[linewidth=1.5pt,linecolor=blue,plotpoints=5000]{0}{490}{2.71828 x 261 div  neg  exp 2.671 mul neg 3.571 add}
\uput[d](250,2.4){\blue $\mathcal{C}_f$}
\uput [d](220,-0.3){temps (min)} \uput[180]{90}(-35,1.3){pH}
\psline[linewidth=1.5pt,linestyle=dashed,ArrowInside=->](0,2)(140,2)(140,0)
\uput[d](140,0){140}
\end{pspicture}
\end{center}

		\item Dans $[0~;~+ \infty[$ :
		\begin{align*}
		&f(t ) = 5 \\
		\iff &6,571- 2,671 \e^{-\frac{t}{ 261}} = 5 \\
		\iff &6,571 - 5 = 2,671 \e^{-\frac{t}{ 261}} \\
		\iff &1,571 = 2,671 \e^{-\frac{t}{ 261}} \\
		\iff &\dfrac{1,571}{2,671} = \e^{-\frac{t}{261}},
		\end{align*}
		puis par croissance de la fonction logarithme népérien :
		
		\[\ln \left(\dfrac{1,571}{2,671}\right) = -\dfrac{t}{261} \text{ et enfin } t = - 261\left(\dfrac{1,571}{2,671}\right) \approx 138,52.\]

Donc le pH est à 5 au bout de 2 heures 18 minutes et $0,52 \times 60 = 31,2$~s donc environ 2 h 19~min., soit une minute de moins que le résultat trouvé graphiquement.
	\end{enumerate}

\item On sait que $\displaystyle \lim_{t\to +\infty} -\frac{t}{261} = - \infty$, donc $\displaystyle  \e^{-\frac{t}{261}}\lim_{t\to +\infty} = 0$.

On a donc $\displaystyle \lim_{t\to +\infty}6,571- 2,671 \times 0 = 6,571$.

Ceci est accord avec les valeurs relevées lors de l'expérience.
\end{enumerate}

\bigskip


