
\begin{center}
\textbf{Validité des informations d'une brochure}
\end{center}

Dans cet exercice, on étudie la chute d'un parachutiste, avant l'ouverture de son parachute, sous l'effet de son poids.

On modélise la vitesse du parachutiste $\left(\text{en m} \cdot \text{s}^{-1}\right)$, en fonction du temps $t$ écoulé (en seconde) depuis le largage, par la fonction $v$, solution de l'équation différentielle :
\[\dfrac{\text{d}v}{\text{d}t}(t) = - 0,16 v(t) + 9,81.\]

On suppose que $v(0) = 0$.

\medskip

\begin{enumerate}[start=3]
\item Démontrer que $v(t) = \dfrac{981}{16} \left(1 - \text{e}^{-0,16t}\right)$, pour $t$ réel positif.
\end{enumerate}

\medskip

La brochure commerciale présentant le saut en parachute indique que le parachutiste atteint la vitesse de $200$ km$\cdot$h$^{-1}$ en moins de quarante secondes.

\begin{enumerate}[start=4]
\item Convertir 200 km$\cdot$h$^{-1}$ en mètre par seconde $\left(\text{en m} \cdot \text{s}^{-1}\right)$.
\item Valider ou infirmer l'indication de la brochure.
\end{enumerate}

\bigskip


