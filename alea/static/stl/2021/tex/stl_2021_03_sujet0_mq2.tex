
\medskip

\begin{center}
\textbf{Figure 1}

\psset{xunit=1cm,yunit=1cm,labelFontSize=\scriptstyle,comma=true}
\begin{pspicture}(-1,-0.75)(11,3.75)
\multido{\n=0+1}{11}{\psline[linewidth=0.75pt,linecolor=lightgray](\n,0)(\n,3.5)}
\multido{\n=0+1}{4}{\psline[linewidth=0.75pt,linecolor=lightgray](0,\n)(10.25,\n)}
\psframe[linewidth=0.8pt, linecolor=gray](9,0)(9.3,0.3)
\psaxes[linewidth=0.8pt,Dx=1,Dy=1]{->}(0,0)(10.25,3.5)
\pstVerb{ /xmax 9 ln 1 add def }
\pscustom[fillstyle=hlines, hatchcolor=cyan, hatchwidth=0.8pt, hatchsep=5.5pt, hatchangle=-45]{
	\psplot[plotpoints=1000]{1}{9}{x ln}
	\psline(9,0)
	\psplot[plotpoints=1000]{9}{1}{0}
	\closepath
}
\psplot[plotpoints=1000,linewidth=1.25pt,linecolor=blue]{1}{9}{x ln}
\psplot[plotpoints=1000,linewidth=1.25pt,linecolor=blue]{1}{9}{0}
\psline[linecolor=blue,linewidth=1.25pt](9,0)(!9 9 ln)
\uput[ul](1,0){A}\uput[dr](9,0){B}\uput[ur](!9 9 ln){C}\uput[ur](5.4,2.4){$C_f$}
\psline[arrows=->,linewidth=1pt](5.5,2.5)(!5 5 ln)
\psdot[dotsize=4pt,linecolor=red](1,0)
\psdot[dotsize=4pt,linecolor=red](9,0)
\psdot[dotsize=4pt,linecolor=red](!9 9 ln)
\end{pspicture}
\end{center}

\begin{itemize}
\item Sur la figure 1 ci-dessus, l'unité de longueur est le centimètre ;
\item la courbe $C_f$ tracée est celle de la fonction $f$ définie sur $[1~;~9]$ par $f(x) = \ln(x)$ ;
\item le point A a pour coordonnées $(1~;~0)$, le point B a pour coordonnées $(9~;~0)$ ;
\item C est le point de $C_f$ d'abscisse 9.
\end{itemize}

On désigne par $\Delta$ le domaine hachuré sur la figure 1, délimité par la courbe $C_f$, l'axe des abscisses et le segment [BC]. On note $A_2$ l'aire de $\Delta$, exprimée en cm$^2$.

\smallskip

Dans le but d'utiliser la méthode des rectangles pour estimer $A_2$, on a écrit la fonction Python ci-dessous :

\begin{lstlisting}[frame=single, language=Python, basicstyle=\ttfamily, breaklines=true, xleftmargin=80pt, xrightmargin=80pt]
from math import log as ln
def meth_rect(pas):
    s = 0
    x = 1
    while x<9:
        s = s+ln(x)*pas
        x = x+pas
    return s
\end{lstlisting}

\begin{enumerate}
\item Laquelle des figures ci-dessous correspond à l'exécution de l'instruction \texttt{meth\_rect(2)} ?

$\textit{Aucune justification n'est attendue.}$

\noindent
\begin{minipage}[t]{0.48\textwidth}

\begin{center}
\psset{xunit=0.55cm,yunit=0.5cm,labelFontSize=\scriptstyle,comma=true}
\begin{pspicture}(-1,-0.75)(11,3.75)
\multido{\n=0+1}{11}{\psline[linewidth=0.75pt,linecolor=lightgray](\n,0)(\n,3.5)}
\multido{\n=0+1}{4}{\psline[linewidth=0.75pt,linecolor=lightgray](0,\n)(10.25,\n)}
\psaxes[linewidth=0.8pt,Dx=1,Dy=1]{->}(0,0)(10.25,3.5)
\psframe[fillstyle=hlines,hatchangle=45,hatchsep=4pt,hatchcolor=red,linewidth=0.5pt,linecolor=red](1,0)(!3 3 ln)
\psframe[fillstyle=hlines,hatchangle=45,hatchsep=4pt,hatchcolor=red,linewidth=0.5pt,linecolor=red](!1 2 add 0)(!5 5 ln)
\psframe[fillstyle=hlines,hatchangle=45,hatchsep=4pt,hatchcolor=red,linewidth=0.5pt,linecolor=red](!1 4 add 0)(!7 7 ln)
\psframe[fillstyle=hlines,hatchangle=45,hatchsep=4pt,hatchcolor=red,linewidth=0.5pt,linecolor=red](!1 6 add 0)(!9 9 ln)
\psplot[plotpoints=1000,linewidth=1.25pt,linecolor=blue]{1}{9}{x ln}
\psplot[plotpoints=1000,linewidth=1.25pt,linecolor=blue]{1}{9}{0}
\psline[linecolor=blue,linewidth=1.25pt](9,0)(!9 9 ln)
\uput[ul](1,0){A}\uput[dr](9,0){B}\uput[ur](!9 9 ln){C}\uput[ul](!3 3 ln){$C_f$}
\psdot[dotsize=4pt,linecolor=red](1,0)
\psdot[dotsize=4pt,linecolor=red](9,0)
\psdot[dotsize=4pt,linecolor=red](!9 9 ln)
\end{pspicture}

\smallskip

Figure 2

\end{center}

\end{minipage}\hfill
\begin{minipage}[t]{0.48\textwidth}

\begin{center}
\psset{xunit=0.55cm,yunit=0.5cm,labelFontSize=\scriptstyle,comma=true}
\begin{pspicture}(-1,-0.75)(11,3.75)
\multido{\n=0+1}{11}{\psline[linewidth=0.75pt,linecolor=lightgray](\n,0)(\n,3.5)}
\multido{\n=0+1}{4}{\psline[linewidth=0.75pt,linecolor=lightgray](0,\n)(10.25,\n)}
\psaxes[linewidth=0.8pt,Dx=1,Dy=1]{->}(0,0)(10.25,3.5)
\psframe[fillstyle=hlines,hatchangle=45,hatchsep=4pt,hatchcolor=red,linewidth=0.5pt,linecolor=red](1,0)(!2 2 ln)
\psframe[fillstyle=hlines,hatchangle=45,hatchsep=4pt,hatchcolor=red,linewidth=0.5pt,linecolor=red](!1 1 add 0)(!3 3 ln)
\psframe[fillstyle=hlines,hatchangle=45,hatchsep=4pt,hatchcolor=red,linewidth=0.5pt,linecolor=red](!1 2 add 0)(!4 4 ln)
\psframe[fillstyle=hlines,hatchangle=45,hatchsep=4pt,hatchcolor=red,linewidth=0.5pt,linecolor=red](!1 3 add 0)(!5 5 ln)
\psframe[fillstyle=hlines,hatchangle=45,hatchsep=4pt,hatchcolor=red,linewidth=0.5pt,linecolor=red](!1 4 add 0)(!6 6 ln)
\psframe[fillstyle=hlines,hatchangle=45,hatchsep=4pt,hatchcolor=red,linewidth=0.5pt,linecolor=red](!1 5 add 0)(!7 7 ln)
\psframe[fillstyle=hlines,hatchangle=45,hatchsep=4pt,hatchcolor=red,linewidth=0.5pt,linecolor=red](!1 6 add 0)(!8 8 ln)
\psframe[fillstyle=hlines,hatchangle=45,hatchsep=4pt,hatchcolor=red,linewidth=0.5pt,linecolor=red](!1 7 add 0)(!9 9 ln)
\psplot[plotpoints=1000,linewidth=1.25pt,linecolor=blue]{1}{9}{x ln}
\psplot[plotpoints=1000,linewidth=1.25pt,linecolor=blue]{1}{9}{0}
\psline[linecolor=blue,linewidth=1.25pt](9,0)(!9 9 ln)
\uput[ul](1,0){A}\uput[dr](9,0){B}\uput[ur](!9 9 ln){C}\uput[ul](!3 3 ln){$C_f$}
\psdot[dotsize=4pt,linecolor=red](1,0)
\psdot[dotsize=4pt,linecolor=red](9,0)
\psdot[dotsize=4pt,linecolor=red](!9 9 ln)
\end{pspicture}

\smallskip

Figure 3

\end{center}

\end{minipage}

%\vspace{0.5cm}
%\hrule
%\vspace{0.5cm}

\noindent
\begin{minipage}[t]{0.48\textwidth}

\begin{center}
\psset{xunit=0.55cm,yunit=0.5cm,labelFontSize=\scriptstyle,comma=true}
\begin{pspicture}(-1,-0.75)(11,3.75)
\multido{\n=0+1}{11}{\psline[linewidth=0.75pt,linecolor=lightgray](\n,0)(\n,3.5)}
\multido{\n=0+1}{4}{\psline[linewidth=0.75pt,linecolor=lightgray](0,\n)(10.25,\n)}
\psaxes[linewidth=0.8pt,Dx=1,Dy=1]{->}(0,0)(10.25,3.5)
\psframe[fillstyle=hlines,hatchangle=45,hatchsep=4pt,hatchcolor=red,linewidth=0.5pt,linecolor=red](3,0)(!5 3 ln)
\psframe[fillstyle=hlines,hatchangle=45,hatchsep=4pt,hatchcolor=red,linewidth=0.5pt,linecolor=red](!3 2 add 0)(!7 5 ln)
\psframe[fillstyle=hlines,hatchangle=45,hatchsep=4pt,hatchcolor=red,linewidth=0.5pt,linecolor=red](!3 4 add 0)(!9 7 ln)
\psplot[plotpoints=1000,linewidth=1.25pt,linecolor=blue]{1}{9}{x ln}
\psplot[plotpoints=1000,linewidth=1.25pt,linecolor=blue]{1}{9}{0}
\psline[linecolor=blue,linewidth=1.25pt](9,0)(!9 9 ln)
\uput[ul](1,0){A}\uput[dr](9,0){B}\uput[ur](!9 9 ln){C}\uput[ul](!3 3 ln){$C_f$}
\psdot[dotsize=4pt,linecolor=red](1,0)
\psdot[dotsize=4pt,linecolor=red](9,0)
\psdot[dotsize=4pt,linecolor=red](!9 9 ln)
\end{pspicture}

\smallskip

Figure 4

\end{center}

\end{minipage}\hfill
\begin{minipage}[t]{0.48\textwidth}

\begin{center}
\psset{xunit=0.55cm,yunit=0.5cm,labelFontSize=\scriptstyle,comma=true}
\begin{pspicture}(-1,-0.75)(11,3.75)
\multido{\n=0+1}{11}{\psline[linewidth=0.75pt,linecolor=lightgray](\n,0)(\n,3.5)}
\multido{\n=0+1}{4}{\psline[linewidth=0.75pt,linecolor=lightgray](0,\n)(10.25,\n)}
\psaxes[linewidth=0.8pt,Dx=1,Dy=1]{->}(0,0)(10.25,3.5)
\psframe[fillstyle=hlines,hatchangle=45,hatchsep=4pt,hatchcolor=red,linewidth=0.5pt,linecolor=red](2,0)(!3 2 ln)
\psframe[fillstyle=hlines,hatchangle=45,hatchsep=4pt,hatchcolor=red,linewidth=0.5pt,linecolor=red](!2 1 add 0)(!4 3 ln)
\psframe[fillstyle=hlines,hatchangle=45,hatchsep=4pt,hatchcolor=red,linewidth=0.5pt,linecolor=red](!2 2 add 0)(!5 4 ln)
\psframe[fillstyle=hlines,hatchangle=45,hatchsep=4pt,hatchcolor=red,linewidth=0.5pt,linecolor=red](!2 3 add 0)(!6 5 ln)
\psframe[fillstyle=hlines,hatchangle=45,hatchsep=4pt,hatchcolor=red,linewidth=0.5pt,linecolor=red](!2 4 add 0)(!7 6 ln)
\psframe[fillstyle=hlines,hatchangle=45,hatchsep=4pt,hatchcolor=red,linewidth=0.5pt,linecolor=red](!2 5 add 0)(!8 7 ln)
\psframe[fillstyle=hlines,hatchangle=45,hatchsep=4pt,hatchcolor=red,linewidth=0.5pt,linecolor=red](!2 6 add 0)(!9 8 ln)
\psplot[plotpoints=1000,linewidth=1.25pt,linecolor=blue]{1}{9}{x ln}
\psplot[plotpoints=1000,linewidth=1.25pt,linecolor=blue]{1}{9}{0}
\psline[linecolor=blue,linewidth=1.25pt](9,0)(!9 9 ln)
\uput[ul](1,0){A}\uput[dr](9,0){B}\uput[ur](!9 9 ln){C}\uput[ul](!3 3 ln){$C_f$}
\psdot[dotsize=4pt,linecolor=red](1,0)
\psdot[dotsize=4pt,linecolor=red](9,0)
\psdot[dotsize=4pt,linecolor=red](!9 9 ln)
\end{pspicture}

\smallskip

Figure 5

\end{center}

\end{minipage}

\medskip

\item Comparer $A_2$ à la valeur \texttt{9.307920700315046} renvoyée par l'exécution de \texttt{meth\_rect(2)}.
\end{enumerate}

\bigskip

