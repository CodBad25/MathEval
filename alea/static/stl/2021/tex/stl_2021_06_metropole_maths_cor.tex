
\medskip

\textbf{Question 1}

\medskip

$g(0) = (2 \times 0 - 1)\e^{0} = -1$

\bigskip

\textbf{Question 2}

\medskip

\begin{enumerate}
\item $g'(x) = 2 \times \e^{-x} + (2x-1) \times \left(-1\e^{-x}\right) = \left(2 - 2x + 1\right)\e^{-x} = (-2x + 3)\e^{-x}$.

\item $g'(x)<0$ si et seulement si $x>\dfrac{3}{2}$.

Par conséquent $g$ est strictement décroissante sur $\left]\dfrac{3}{2}~;~+\infty\right[$.

Or : $g\left(\frac{3}{2}\right) = 2\e^{-\frac{3}{2}}$.

Nous avons donc sur $\left]\dfrac{3}{2}~;~+\infty\right[$, $g(x) < 2\e^{-\frac{3}{2}}$.
\end{enumerate}

\bigskip

\textbf{Question 3}

\medskip

\[\cos \left(\dfrac{9\pi}{5}\right) = \dfrac{\sqrt 5 + 1}{4} \qquad \text{et} \qquad \dfrac{9\pi}{5} = \dfrac{10\pi -\pi}{5} = 2\pi-\dfrac{\pi}{5}\]

Or, pour tout réel $x$, on a :
\[\cos \left(2\pi - x\right) = \cos \left(-x\right) \qquad \text{et} \qquad \cos \left(-x\right) = \cos \left(x\right).\]

Donc :
\begin{align*}
\cos\left(\dfrac{9\pi}{5}\right) &= \cos\left(2\pi -\dfrac{\pi}{5}\right) \\
&= \cos\left(-\dfrac{\pi}{5}\right) \\
&= \cos\left(\dfrac{\pi}{5}\right) \\
&= \dfrac{\sqrt{5}+1}{4}
\end{align*}

\bigskip

\textbf{Question 4}

\medskip

$I = \displaystyle\int_0^2 (2x - 1)\d x= \left[x^2-x\right]_0^2=4-2=2$.

\bigskip

\textbf{Question 5}

\begin{align*}
A &= 5\ln \left(\e^3\right) - 4 \ln \left[\dfrac{1}{\text{e}^2}\right) \\
&= 5 \times 3\ln(\e) - 4 \times (-2\ln(\e)) \\
&= 5 \times 3 - 4 \times (-2) \\
&= 23.
\end{align*}

\bigskip

\textbf{Question 6}

\medskip

Calculons d'abord ED :
\[\text{ED}^2=\text{EC}^2+\text{CD}^2= 2\text{EC}^2\]

Par conséquent :
\[\text{ED} = \text{EC}\sqrt{2}\ \text{ et }  \text{ED} \cos \left(\frac{\pi}{4}\right)= \text{CE}\times \frac{\sqrt{2}}{2}\]

\[\vect{\text{EB}}\cdot\vect{\text{ED}}= \left\|\vect{\text{EB}}\right\|\, \left\|\vect{\text{ED}}\right\| \cos \left(\frac{\pi}{4}\right) = 3\times 3\sqrt{2}\times \frac{\sqrt{2}}{2}= 9.\]

Nous pouvions aussi dire que C est le projeté orthogonal de D sur (EB).

Nous aurions pu écrire directement :
\[\vect{\text{EB}} \cdot \vect{\text{ED}}=  \text{EB}\times \text{EC}=3\times 3=9.\]

\bigskip


