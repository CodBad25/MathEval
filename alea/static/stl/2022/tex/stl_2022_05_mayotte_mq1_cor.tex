
\medskip

\begin{enumerate}
\item L'équation réduite d'une droite est de la forme $y=mx + p$ où :
\[m = \dfrac{y_{\text{B}}-y_{\text{A}}}{x_{\text{B}}-x_{\text{A}}} = \dfrac{-4-(-2)}{-2-(-3)} = -2\]
 
Écrivons qu'elle passe par A$(-3~;~-2)$ : $-2 = -2 \times (-3) + p$ d'où $p = -8$. 

L'équation réduite de (AB) est : $y = - 2x - 8$.
\item Il en résulte que la valeur exacte de $h'(-2)$ est $-2$, puisque le nombre dérivé de la fonction en $a$ est le coefficient directeur de la tangente à la courbe en ce point.
\item Déterminons les coordonnées des points d'intersection de la droite $T$ avec chacun des axes du repère.
\begin{itemize}
\item Avec l'axe des abscisses, $y = 0$ soit $-2x-8 = 0$ d'où $x = - 4$. Le point a pour coordonnées $(-4~;~0)$.
\item Avec l'axe des ordonnées, $x = 0$ soit $y = -8$. Le point a pour coordonnées $(0~;~-8)$.
\end{itemize}
\end{enumerate}

\bigskip

