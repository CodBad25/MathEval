
\medskip

\begin{enumerate}[start=4]
\item Soit les points A$(0~;~-1,2)$ et B$(2~;~-2,4)$ du graphique (semblant appartenir à la droite, éloignés, lecture facile des coordonnées). 

\[a = \dfrac{y_{\text B}-y_{\text A}}{x_{\text B}-x_{\text A}}
=\dfrac{-2,4 - (-1,2)}{2 \times 10^6 - 0} = - \dfrac{1,2}{2 \times 10^6} = -0,6  \times 10^{-6} = -6  \times 10^{-7}\]

Donc $-6  \times 10^{-7}$ est une valeur estimée du coefficient directeur $a$ de la droite (AB).

On en déduit :
\[k = - a = -(-6  \times 10^{-7}) = 6  \times 10^{-7} \text{ s}^{-1}.\]

\item L'équation différentielle $y'=ay$ a pour solutions les fonctions $f$ définies par :
\[f(t)=k \e^{at},\]

avec $k\in\R$.

Donc les solutions de l'équation différentielle $(E)$ sont définies par :
\[f(t)=  k\e^{-6\cdot 10^{-7}t},\]

avec $k\in\R$.

\[f(0) = 0,3 \iff k\e^{-6\cdot 10^{-7}\times 0} = 0,3 \iff k = 0,3\]

La solution cherchée est la fonction $f$ définie  sur $[0~;~+\infty[$ par :
\[f(t) = 0,3 \e^{-6\cdot 10^{-7}t}.\]

\item $60\times 24\times 60 \times 60 = \np{5184000}$ donc 60 jours correspondent à $\np{5184000}$ secondes.

$f(\np{5184000})\approx 0,013$ donc il reste environ $1,3$\,\% de quantité de matière de saccharose dans la canette au bout de 60 jours.
\end{enumerate}

\bigskip


