
\medskip

\begin{enumerate}
\item $C'(t)=-k_1C_0\e^{-k_1t}$. Par conséquent, nous avons bien $y' = -k_1y$ et $C$ est bien une solution de cette équation différentielle.

Montrons que $C(0) = C_0$ :
\[C(0)=C_0\times \e^{-k_1\times0}=C_0\times 1=C_0.\]
\item Calculons l'instant $t$ pour lequel $C(t)=\frac{1}{2}C_0$.

Résolvons :
\begin{align*}
C_0\e^{-\np{0.017}t}&=\frac{1}{2}C_0\\
\e^{-\np{0.017}t}&=\frac{1}{2}\\
- \np{0.017} t&=-\ln 2\\
t&=\dfrac{\ln (2)}{\np{0.017}}\\
t&\approx 41
\end{align*}
Ce résultat montre qu'au bout d'environ \np[s]{41} la concentration en ions hydroxyde est la moitié de la concentration initiale.
\end{enumerate}

\bigskip


