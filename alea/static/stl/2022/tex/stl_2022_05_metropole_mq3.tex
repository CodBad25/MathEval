
\medskip

On considère une fonction $g$ définie et dérivable sur l'intervalle [0~;~13].

On note $g'$ sa fonction dérivée.

On donne ci-dessous la courbe représentative de la \textbf{fonction dérivée}\boldmath{} $g'$ \unboldmath sur l'intervalle [0~;~13].

\begin{center}
\psset{unit=0.5cm}
\begin{pspicture}(-1,-2)(13,8)
\psaxes[linewidth=1.25pt,labels=none]{->}(0,0)(-1,-2)(13,8)%Dx=20,Dy=20
\uput[d](1,0){\footnotesize 1} \uput[d](4,0){\footnotesize 4} \uput[d](13,0){\footnotesize 
13} \uput[l](0,1){\footnotesize 1}\uput[dl](0,0){\footnotesize 0}
\uput[u](13,0){$x$}\uput[l](0,8){$y$}
\psplot[plotpoints=2000,linewidth=1.25pt,linecolor=red]{0}{13}{2.71828 2 x 0.5 mul sub exp 1 sub}
\end{pspicture}
\end{center}
Julien affirme que la fonction $g$ est décroissante sur l'intervalle [0~;~13].

Julien a-t-il raison ? Justifier.

\bigskip

