
\medskip

On s'intéresse à l'énergie stockée dans la batterie d'un téléphone portable. Cette grandeur s'exprime en kW\cdot h. Lorsque la batterie est totalement chargée, l'énergie stockée vaut 0,715 kW\cdot h.

Lors du branchement de la batterie vide sur une borne de recharge, l'énergie stockée dans la batterie (en kW\cdot h) en fonction du temps $t$ (en heure) est modélisée par une fonction $f$ telle que, pour tout nombre réel $t \geqslant 0$ :
\[f(t) = -0,715 \e^{-t} + 0,715.\]

On considère la fonction $F$ définie sur $[0~;~+\infty[$ par :

\[F(t) = 0,715 t + 0,715 \e^{-t}.\]

\begin{enumerate}
\item Vérifier que $F$ est une primitive de $f$ sur $[0~;~+\infty[$.
\end{enumerate}

\smallskip

On admet que l'énergie stockée moyenne de la batterie sur [0~;~3,5] est égale à :

\[m = \dfrac{1}{3,5} [F(3,5) - F(0)].\]

\begin{enumerate}[resume]
\item Cette énergie stockée moyenne est-elle égale à la moitié de l'énergie stockée maximale ? Justifier la réponse.
\end{enumerate}

