
\medskip

\textbf{Question 1}

\medskip

$f(0) = (-4 \times 0 + 8)\e^{0} = 8\ times 1 = 8$ qui est bien un nombre entier.

\bigskip

\textbf{Question 2}

\medskip

On veut déterminer $\displaystyle\lim_{x \to + \infty} (-4x + 8)\e^{-x} =\ds\lim _{x \to +\infty} \left (-4x\e^{-x} +8\e^{-x}\right )$.

On sait que :
\[\ds\lim_{x \to +\infty} \e^{-x} = 0 \text{ donc } \ds\lim_{x \to +\infty} 8\e^{-x} = 0,\]

et comme $x\e^{-x} = \dfrac{x}{\e^x}$, d'après le théorème des croissances comparées :
\[\ds\lim_{x \to +\infty} \dfrac{x}{\e^x} = 0 \text{ et } \ds\lim_{x \to +\infty} -4\dfrac{x}{\e^x} = 0.\]

Donc $\ds\lim_{x \to +\infty} \left(-4x\e^{-x} +8\e^{-x}\right) = 0$,
et donc $\displaystyle\lim_{x \to + \infty} (-4x + 8)\e^{-x} =0 $.

\bigskip

\textbf{Question 3}

\medskip

Sur $[0~;~+\infty[$ :
\begin{align*}
f'(x) &= (-4) \times \e^{-x} + (-4x+8) \times (-1)\e^{-x} \\
&= (-4+4x-8)\e^{-x} \\
&= (4x-12)\e^{-x}
\end{align*}

\bigskip

\textbf{Question 4}

\medskip

$\text{A} = \dfrac{\text{e}^8 \times \text{e}^{-3}}{\left(\text{e}^{0,5}\right)^4}
= \dfrac{\e^{8-3}}{\e^{0,5\times 4}}
= \dfrac{\e^{5}}{\e^{2}}= \e^{5 - 2} = \e^{3}$

\bigskip

\textbf{Question 5}

\medskip

\[\cos \left(\dfrac{9\pi}{5}\right) = \dfrac{\sqrt 5 + 1}{4} \qquad \text{et} \qquad \dfrac{\pi}{5} = \dfrac{10\pi - 9\pi}{5} = 2\pi - \dfrac{9\pi}{5}\]

Or, pour tout réel $x$, on a :
\[\cos \left(2\pi - x\right) = \cos \left(-x\right) \qquad \text{et} \qquad \cos \left(-x\right) = \cos \left(x\right).\]

Donc :
\begin{align*}
\cos \left(\dfrac{\pi}{5}\right) &= \cos \left(2\pi - \dfrac{9\pi}{5}\right) \\
&= \cos \left(-\dfrac{9\pi}{5}\right) \\
&= \cos \left ( \dfrac{9\pi}{5}\right) \\
&= \dfrac{\sqrt{5}+1}{4}
\end{align*}

\bigskip

\textbf{Question 6}

\medskip

$\vect{\text{AB}} \cdot \vect{\text{AC}} = 2 \times 3 + (-1) \times 2 = 6 - 2 = 4$.

\bigskip


