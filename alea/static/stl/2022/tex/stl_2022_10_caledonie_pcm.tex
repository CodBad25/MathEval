
\begin{center}
\textbf{Saponification de l'éthanoate d'éthyle}
\end{center}

On souhaite déterminer le temps de demi-réaction de la saponification de l'éthanoate d'éthyle par l'exploitation mathématique de la loi de vitesse. L'équation de la réaction modélisant cette transformation chimique est :
\[\mathrm{CH_{3}COOCH_{2}CH_{3} + HO^{-} \rightarrow CH_{3}COO^{-} + CH_{3}CH_{2}OH}\]
\begin{center}
éthanoate d'éthyle + ion hydroxyde $\rightarrow$ ion éthanoate + éthanol
\end{center}

\medskip

On note $C(t)$ la concentration en ions hydroxyde, exprimée en mol/L, à l'instant $t$, exprimé en seconde et $C_0$ la concentration en ions hydroxyde à l'instant $t = 0$.

Dans les conditions d'un protocole donné, $C_0 = 0,016$ mol/L et $k_1 = 0,017~\text{S}^{-1}$. 

La fonction $C$ est donc solution de l'équation différentielle $(E)$ suivante :
\[y'= - k_1y\qquad  (E)\]

\smallskip

\begin{enumerate}
\item Vérifier que la fonction $C$ définie sur $[0~;~ +\infty[$ par $C(t) = C_0\text{e}^{-k_1 t}$ est une solution de $(E)$.

Montrer que $C(0) = C_0$. On admet que $C$ est la seule solution de $(E)$ qui vérifie $C(0) = C_0$.
\item Déterminer par le calcul le temps de demi-réaction $t_{1/2}$. On donnera la valeur exacte, puis l'arrondi à la seconde. Interpréter ce résultat dans le contexte de l'exercice.
\end{enumerate}

\bigskip


