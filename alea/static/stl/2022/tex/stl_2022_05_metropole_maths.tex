
\medskip

\textbf{Vous traiterez 4 questions au choix parmi les 6 questions proposées.}

\medskip

\textbf{Question 1}

\medskip

Soit la fonction $f$ définie et dérivable sur $\R$ par :

\[f(x) = (8x - 2)\e^{-x}\]

On note $f'$ sa fonction dérivée. Déterminer $f'(x)$ pour tout $x \in  \R$.

\bigskip

\textbf{Question 2}

\medskip

Soit la fonction $f$ définie sur $\R$ par :

\[f(x) = (8x - 2)\e^{-x}\]

Résoudre $f(x) = 0$.

\bigskip

\textbf{Question 3}

\medskip

On considère une fonction $g$ définie et dérivable sur l'intervalle [0~;~13].

On note $g'$ sa fonction dérivée.

On donne ci-dessous la courbe représentative de la \textbf{fonction dérivée}\boldmath{} $g'$ \unboldmath sur l'intervalle [0~;~13].

\begin{center}
\psset{unit=0.5cm}
\begin{pspicture}(-1,-2)(13,8)
\psaxes[linewidth=1.25pt,labels=none]{->}(0,0)(-1,-2)(13,8)%Dx=20,Dy=20
\uput[d](1,0){\footnotesize 1} \uput[d](4,0){\footnotesize 4} \uput[d](13,0){\footnotesize 
13} \uput[l](0,1){\footnotesize 1}\uput[dl](0,0){\footnotesize 0}
\uput[u](13,0){$x$}\uput[l](0,8){$y$}
\psplot[plotpoints=2000,linewidth=1.25pt,linecolor=red]{0}{13}{2.71828 2 x 0.5 mul sub exp 1 sub}
\end{pspicture}
\end{center}
Julien affirme que la fonction $g$ est décroissante sur l'intervalle [0~;~13].

Julien a-t-il raison ? Justifier.

\bigskip

\textbf{Question 4}

\medskip

Montrer que : $\dfrac{\ln \left(\sqrt{8}\right)}{\ln \left(\sqrt{2}\right)}= 3$.

\bigskip

\textbf{Question 5}

\medskip

Soit $f$ la fonction définie et dérivable sur $\R$ par $f(x) = \e^{6x} - 1$. 

Déterminer la limite de la fonction $f$ lorsque $x$ tend vers $-\infty$.

\bigskip

\textbf{Question 6}

\medskip

ABCD est un carré de côté 4 et ABF est un triangle rectangle en B avec BF $= 3$ comme indiqué sur la figure ci-dessous.

\begin{center}
\psset{unit=1cm}
\begin{pspicture}(6.2,4.1)
%\psgrid
\psframe(0.2,0.2)(3.4,3.4)%CBAD
\psline(3.4,0.2)(5.8,0.2)(3.4,3.4)%BFA
\psline(0.2,3.4)(3.4,0.2)
\psline(0.1,1.8)(0.3,1.8)\psline(3.3,1.8)(3.5,1.8)
\psline(1.8,3.5)(1.8,3.3)\psline(1.8,0.3)(1.8,0.1)
\uput[d](4.6,0.2){3}
\uput[l](0.2,1.8){4}
\uput[u](3.4,3.4){A} \uput[d](3.4,0.2){B} \uput[d](0.2,0.2){C} \uput[u](0.2,3.4){D} \uput[d](5.8,0.2){F}
\psframe(3.4,0.2)(3.65,0.45)
\psarc(3.4,0.2){0.5}{90}{135}\rput(3.1,0.9){$45\degres$}
\psdots(3.4,3.4)
\end{pspicture}
\end{center}

Donner la valeur du produit scalaire $\vect{\text{BF}} \cdot \vect{\text{BD}}$.

\bigskip


