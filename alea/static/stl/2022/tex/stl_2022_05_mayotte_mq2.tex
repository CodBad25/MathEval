
\medskip

On donne, ci-dessous $\mathcal{C}_h$, la courbe représentative d'une fonction $h$, définie et dérivable sur l'intervalle $[-5~;~5]$.

\begin{center}
\psset{unit=1cm,arrowsize=2pt 3}
\begin{pspicture}(-5,-5)(5,2)
\psgrid[gridlabels=0pt,subgriddiv=1,gridwidth=1pt,griddots=10,gridcolor=gray]
\psaxes[linewidth=1.25pt]{->}(0,0)(-5,-5)(5,2)
\uput[u](4.8,0){$x$}
\uput[r](0,1.75){$y$}
\psecurve[linewidth=1.25pt,linecolor=blue](-6,-2.1)(-5,-2)(-4,-2.1)(-3,-2.5)(-2,-4)(-1.5,-4.5)(-1,-4.3)(0,-2.65)(1,-1)(2,0)(3,1)(4,1.8)(5,2)(6,1.9)
\uput[d](-4.5,-2){\blue $\mathcal{C}_h$}
\end{pspicture}
\end{center}

Soit $H$ une primitive de $h$ sur l'intervalle $[-5~;~5]$.

À l'aide du graphique, donner le sens de variation de la fonction $H$ sur l'intervalle $[-5~;~5]$.

\bigskip

