
\medskip

\textbf{Question 1}

\medskip

$f(0)=(0+5)\e^{0}=5$ donc $f(0)$ est bien un entier.

\bigskip

\textbf{Question 2}

\medskip

Sur $[0~;~+\infty[$ :
\[f'(x)=1\times\e^{3x} + (x-5)\times 3\e^{3x}=(1+3x-15)\e^{3x}=(3x-14)\e^{3x}.\]
Ce qui corresponds bien à l'expression de $f'(x)$ donnée.

\bigskip

\textbf{Question 3}

\medskip

\begin{list}{\textbullet}{}
\item Pour $a>0$ et $b>0$, on a: $\ln\left (\dfrac{a}{b}\right ) = \ln(a)-\ln(b)$.\\
Donc $\ln \left(\dfrac{25}{8}\right)=\ln(25)-\ln(8)$.
\item Pour $a>0$ et $n$ entier naturel non nul, on a: $\ln\left (a^n\right )=n\ln(a)$.\\
Donc $\ln(25)=\ln(5^2)=2\ln(5)$ et $\ln(8)=\ln(2^3)=3\ln(2)$.
\end{list}

$\mathcal{A}=2\ln(5)-3\ln(2)=-3\ln(2)+2\ln(5)$

\bigskip

\textbf{Question 4}

\medskip

\begin{list}{\textbullet}{}
\item L'équation différentielle (E$'$): $y'=3y$ a pour solutions les fonctions $g_0$ définies par $g_0(x)=k\e^{3x}$ où $k$ est un réel quelconque.
\item L'équation différentielle (E) a pour solution particulière la fonction $g$ définie par $g(x)=4$.
\item L'équation différentielle (E) a donc pour solutions les fonctions $f$ telles que $f=g+g_0$, autrement dit définies par $f(x)=k\e^{3x}+4$.
\item $f(0)=8 \iff k\e^{0}+4=8 \iff k=4$
\end{list}

La fonction $f$, solution de (E), qui vérifie $f(0) = 8$ est définie sur $\R$ par $f(x)=4\e^{3x}+4$.

\bigskip


