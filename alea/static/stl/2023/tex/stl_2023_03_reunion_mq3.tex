
\medskip

On s'intéresse à l'équipement des habitants d'une grande ville en ordinateurs depuis 2000.

La part (exprimée en \,\%) des habitants de cette ville ayant au moins un ordinateur est modélisée par la fonction $f$ définie sur $[0~;~+ \infty]$ par :
\[f(t) = \dfrac{94,6}{1 + \e^{0,6 - 0,2t}}\]

où $t$ est la durée écoulée (en année) depuis l'année 2000.

Montrer que le taux d'équipement ne peut jamais être supérieur à 94,6\,\%.

\bigskip

