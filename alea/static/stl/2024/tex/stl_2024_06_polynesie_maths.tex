
\medskip

\textbf{Les questions 1, 2, 3 et 4 sont indépendantes les unes des autres.}

\medskip

Dans cet exercice, on considère la fonction $f$ définie sur $\R$ par : 
\[f(x) = 5\e^{2x+1}.\]

\begin{enumerate}
\item Parmi les programmes suivants, écrits en langage Python, un seul affiche les images par $f$ des réels 0~;~0, 1~;~0,2~;~$\dots$~;~0,9.

Indiquer sans justifier sur la copie la lettre correspondant à ce programme.

\vspace{0.3cm}

\textbf{a.~}~~\fbox{\begin{minipage}{4cm}
from math import exp\\ 
for k in range(10) :\\
\phantom{xxxxxx} x=k/10\\
\phantom{xxxxxx}y=5*exp(2*x+1)\\        
\phantom{xxxxxx}print(y)   
\end{minipage}}
\hspace{3cm}
\textbf{b.~}~~\fbox{
\begin{minipage}{4cm}
from math import exp\\
for k in range(10) :\\
\phantom{xxxxxx}y=5*exp(2*k+1)\\
print(y)\\
\end{minipage}}

\textbf{c.~}~~\fbox{\begin{minipage}{4cm}
from math import exp\\
for k in range(0,9) :\\
\phantom{xxxxxx}y=5*exp(2*x+1)\\
\phantom{xxxxxx}print(y)\\
\end{minipage}}
\hspace{3cm}
\textbf{d.~}~~\fbox{
\begin{minipage}{4cm}
from math import exp\\
for k in range(0,9) :\\
\phantom{xxxxxx}y=5*exp(2*x+1)\\
print(y)\\
\end{minipage}}

\medskip

\item Résoudre dans $\R $ l'équation $f(x) = 5$.
\item L'affirmation suivante est-elle vraie ou fausse ? Justifier.

\og Tout nombre réel $x$ négatif ou nul a une image par $f$ inférieure ou égale à 5. \fg
\item On considère la fonction $F$ définie sur $\R$ par : 
\[F(x) = ~\dfrac{5}{2}\e^{2 x+ 1}.\]
	\begin{enumerate}
		\item Montrer que la fonction $F$ est une \textbf{primitive } sur $\R$ de la fonction $f$.
		\item En déduire la valeur exacte, puis la valeur approchée à l'entier près, de :
\[\int_0^1 f(x)\mathrm{d}x.\]
	\end{enumerate}
\end{enumerate}

\bigskip


