
\medskip

\begin{enumerate}
\item 
	\begin{enumerate}
		\item $\bullet~$La courbe $\mathcal{C}$ contient le point de coordonnées (0~;~2), donc $f(2) = 0$.
		
$\bullet~$La tangente $T$ contient les points (0~;~12) et (2~;~0) ; son coefficient directeur égal au nombre dérivé $f'(2)$ est donc $\dfrac{0 - 12}{2 - 0} = \dfrac{12}{2} = - 6$.
		\item L'ordonnée à l'origine est égale à 12 et le coefficient directeur est égal à $- 6$, donc :

$M(x~;~y) \in T \quad \text{si}\quad y = -x + 12$.
		\item~
		
\begin{center}
\psset{unit=1cm,arrowsize=2pt 3}
\begin{pspicture}(8.5,2)
\psframe(8.5,2)\psline(0,1.5)(8.5,1.5)\psline(2.5,0)(2.5,2)
\uput[u](1.25,1.4,){$x$}\uput[u](2.7,1.4,){$-2$}\uput[u](4.5,1.4,){$0$}
\uput[u](6.5,1.4,){$4$}\uput[u](8.4,1.4,){$6$}
\rput(1.25,0.75){Variations de $f$}
\psline{->}(2.7,0.4)(4.3,1.3,) \psline{->}(4.7,1.3)(6.3,0.4) \psline{->}(6.7,0.4,)(8.3,1.3,)
\uput[d](4.5,1.5){8}\uput[u](6.5,0){$-8$}\uput[d](4.5,1.5){8}
\end{pspicture}
\end{center}
	\end{enumerate}
\begin{enumerate}
	\item %Montrer que, pour tout réel $x$ de l'intervalle $[-2~;~6]$, on a $f'(x) = 1,5 x(x - 4)$.
$f$ est une fonction polynôme dérivable pour tout réel $s$ et :

$f'(x) = 3 \times 0,5x^2 - 2 \times 3x = 1,5x^2 - 6x = 1,5x(x - 4)$.
\item %Étudier le signe de $f'(x)$ et retrouver le tableau de variation de la fonction $f$ sur l'intervalle $[-2~;~6]$.
On établit le tableau de signes de cette fonction dérivée :

\begin{center}
\psset{unit=1cm,arrowsize=2pt 3}
\begin{pspicture}(8.5,2)
\psframe(8.5,2)\psline(0,1.5)(8.5,1.5)\psline(2.5,0)(2.5,2)
\psline(0,1.)(8.5,1.)\psline(0,0.5)(8.5,0.5)
\uput[u](1.25,1.4,){$x$}\uput[u](2.7,1.4,){$-2$}\uput[u](4.5,1.4,){$0$}
\uput[u](6.5,1.4,){$4$}\uput[u](8.4,1.4,){$6$}
\rput(1.25,1.25){$1,5x$}\rput(3.5,1.25){$-$}\rput(5.5,1.25){$+$}\rput(7.5,1.25){$+$}
\rput(1.25,0.75){$x - 4$}\rput(3.5,0.75){$-$}\rput(5.5,0.75){$-$}\rput(7.5,0.75){$+$}
\rput(1.25,0.25){$f'(x)$}\rput(3.5,0.25){$+$}\rput(5.5,0.25){$-$}\rput(7.5,0.25){$+$}
\end{pspicture}
\end{center}
	\end{enumerate}
\item $f(x) \leqslant -6 x + 12$ sur l'intervalle [0~;~2] par : géométriquement la courbe $\mathcal{C}$ est \textbf{au-dessous} de la tangente $T$ sur l'intervalle [0~;~2].
\end{enumerate}

\bigskip

