
\medskip

\begin{enumerate}
\item
	\begin{enumerate}
		\item L'affirmation est fausse puisque la probabilité est supérieure à 1.
		\item Il y a 15 positifs non dopés sur 20 positifs, soit une probabilité de $\dfrac{15}{20} = \dfrac{3 \times 5}{4 \times 5} = \dfrac{3 \times 5\times 5}{4 \times 5\times 5} = \dfrac{75}{100}$ : affirmation exacte.
		\item IL y a 15 positifs non dopés et 2 négatifs et dopés : il y a donc en tout $15 + 2$ erreurs, soit une proportion de $\dfrac{17}{200} = \dfrac{8,5}{100} = 8,5\,\%$ : affirmation exacte.
	\end{enumerate}
\item~

$\bullet~$la probabilité de réussir les deux services est égale à $0,9 \times 0,9 = 0,81$ ;

$\bullet~$la probabilité de rater les deux services est égale à $0,1 \times 0,1 = 0,01$

Conclusion : la probabilité de réussir un seul service est donc égale à 

$1 - (0,81 + 0,01) = 1 - 0,82 = 0,18$. L'affirmation est fausse.
\end{enumerate}
