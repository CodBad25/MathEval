Une biologiste désire étudier l'évolution de la population de singes sur une île. \\
En 2025, elle estime qu'il y a 1\,000 singes sur l'île.

\subsubsection*{A. Premier modèle}
Chaque année, la population de singes baisse de 10\%.

\begin{enumerate}
\item Montrer qu'en 2026, il y aura 900 singes sur l'île.

\item Pour tout entier naturel $n$, on note $u_n$ le nombre de singes sur l'île pour l'année $2025 + n$.

On a donc $u_0 = 1\,000$.

\begin{enumerate}
\item Indiquer ce que représente $u_2$ et calculer sa valeur.
\item Déterminer la nature de la suite $(u_n)$ et préciser sa raison.
\item Donner les variations de cette suite.
\end{enumerate}

\item Selon ce modèle, la population de singes est-elle menacée d'extinction ? Justifier.
\end{enumerate}

\bigskip

\begin{minipage}[t]{0.75\textwidth}
\subsubsection*{B. Second modèle}
On admet que l'évolution du nombre de singes est modélisée par la suite $(v_n)$ ainsi définie :

\[
\begin{cases}
v_{n+1} = 0,9v_n + 150~; & n \in \N \\
v_0 = 1\,000~,
\end{cases}
\]

où $v_n$ désigne le nombre de singes sur l'île pour l'année $2\,025 + n$.

\begin{enumerate}
\item Avec ce modèle, quelle sera la population de singes en 2\,026 ? \\
Détailler le calcul.

\item La feuille de calcul ci-contre donne les valeurs arrondies à l'unité des premiers termes de la suite $(v_n)$. \\
Quelle formule, destinée à être étirée vers le bas, faut-il saisir dans la cellule B3 pour obtenir les termes de la suite $(v_n)$ ?

\item Indiquer en quelle année la population de singes dépassera pour la première fois 1\,400 individus.
\end{enumerate}
\end{minipage}
\hfill
\begin{minipage}[t]{0.2\textwidth}
\vspace{0pt}
\centering
\renewcommand{\arraystretch}{1.2}
\begin{tabular}{|>{\columncolor{lightgray}}c|c|c|}
\hline
\rowcolor{lightgray}
\textbf{} & \textbf{A} & \textbf{B} \\ \hline
\textbf{1} & $n$ & $V_n$ \\ \hline
\textbf{2} & 0 & 1\,000 \\ \hline
\textbf{3} & 1 & 1\,050 \\ \hline
\textbf{4} & 2 & 1\,095 \\ \hline
\textbf{5} & 3 & 1\,136 \\ \hline
\textbf{6} & 4 & 1\,172 \\ \hline
\textbf{7} & 5 & 1\,205 \\ \hline
\textbf{8} & 6 & 1\,234 \\ \hline
\textbf{9} & 7 & 1\,261 \\ \hline
\textbf{10} & 8 & 1\,285 \\ \hline
\textbf{11} & 9 & 1\,306 \\ \hline
\textbf{12} & 10 & 1\,326 \\ \hline
\textbf{13} & 11 & 1\,343 \\ \hline
\textbf{14} & 12 & 1\,359 \\ \hline
\textbf{15} & 13 & 1\,373 \\ \hline
\textbf{16} & 14 & 1\,386 \\ \hline
\textbf{17} & 15 & 1\,397 \\ \hline
\textbf{18} & 16 & 1\,407 \\ \hline
\textbf{19} & 17 & 1\,417 \\ \hline
\textbf{20} & 18 & 1\,425 \\ \hline
\textbf{21} & 19 & 1\,432 \\ \hline
\end{tabular}
\end{minipage}




