Victor sort un plat du four. La température du plat est alors égale à 180\degres C. Il place ce plat dans une pièce dont la température est égale à 25\degres C. Le plat refroidit. \\
Le plat ne pourra être servi que lorsque sa température sera devenue inférieure ou égale à 40\degres C.

On étudie le refroidissement du plat selon deux modèles mathématiques.

\subsubsection*{Partie A : Premier modèle.}
On suppose que la baisse de la température du plat est \textit{proportionnelle} à la durée du refroidissement, c'est-à-dire au nombre de minutes écoulées depuis la sortie du four.

On constate que 3 minutes après la sortie du four, la température du plat est égale à 105\degres C.

\begin{enumerate}
\item De combien de degrés le plat a-t-il baissé en 3 minutes ? En 1 minute ?

\item Vérifier que la température du plat, 5 minutes après la sortie du four, est égale à 55\degres C.

\item Selon ce modèle, quelle serait la température du plat, 8 minutes après la sortie du four ? Ce premier modèle semble-t-il pertinent ?
\end{enumerate}

\subsubsection*{Partie B : Second modèle.}
On dispose toujours des données suivantes :
\begin{itemize}
\item la température de la pièce est égale à 25\degres C
\item la température du plat à la sortie du four est égale à 180\degres C
\item la température du plat, 3 minutes après la sortie du four, est égale à 105\degres C
\end{itemize}

Pour tout entier naturel $n$ on note $U_n$, la différence entre la température du plat et la température de la pièce, $n$ minutes après la sortie du four.

\underline{Exemple :} 3 minutes après la sortie du four, l'écart avec la température de la pièce est égal à $105 - 25 = 80$. On a donc $U_3 = 80$.

\begin{enumerate}
\item Justifier que $U_0 = 155$.

\item On suppose que chaque minute la différence $U_n$ diminue de 20\%.
\begin{enumerate}
\item Justifier que, pour tout entier naturel $n$, on a $U_{n+1} = 0,8U_n$.
\item En déduire la nature de la suite $(U_n)$ et donner sa raison.
\item Exprimer $U_n$ en fonction de $n$, pour tout entier naturel $n$.
\item On dispose des données suivantes :
\end{enumerate}
\end{enumerate}

\vspace{-0.9\baselineskip}
\begin{center}
\begin{tabular}{|c|c|c|c|c|c|c|c|c|c|c|c|c|c|}
\hline
$n$ & 3 & 4 & 5 & 6 & 7 & 8 & 9 & 10 & 11 & 12 & 13 & 14 & 15 \\
\hline
$U_n$ & 80 & 64 & 51,2 & 41 & 32,8 & 26,2 & 21 & 16,8 & 13,4 & 10,7 & 8,6 & 6,9 & 5,5 \\
\hline
\end{tabular}
\end{center}

\vspace{-0.9\baselineskip}
\[U_n \text{ arrondi à } 10^{-1}\]

\vspace{-0.5\baselineskip}
\begin{itemize}
    \item[]
    \begin{itemize}
    		\item[] Au bout de combien de minutes, Victor pourra-t-il servir le plat ?
	\end{itemize}
\end{itemize}




