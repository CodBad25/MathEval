
%\medskip
%
%\textbf{Pour cette première partie, aucune justification n'est demandée et une seule réponse est possible par question. Pour chaque question, reportez son numéro sur votre copie et indiquez votre réponse.}

\medskip

\textbf{Question 1}

\begin{minipage}{0.58\linewidth}
On considère l'arbre de probabilité ci-contre. \\
\\
On cherche la probabilité de l'évènement $B$.\\
\\
On a
\end{minipage}\hfill
\begin{minipage}{0.37\linewidth}
\begin{center}
\pstree[treemode=R,levelsep=2cm,nodesepB=2pt]{\TR{}}
{\pstree[nodesepA=2pt]{\TR{$A$}\taput{$0,4$}}
	{\TR{$B$} \taput{$0,3$}
	\TR{$\overline{B}$} \tbput{\blue $0,7$}
	}
\pstree[nodesepA=2pt]{\TR{$\overline{A}$}\tbput{\blue $0,6$}}
	{\TR{$B$} \taput{\blue $0,1$}
	\TR{$\overline{B}$} \tbput{$0,9$}
	}
}
\medskip
\end{center}
\end{minipage}

\medskip

\begin{tabularx}{\linewidth}{|*{4}{>{\centering \arraybackslash}X|}}
\hline
\textbf{A.}&\textbf{B.}& \textbf{C.}& \textbf{D.}\\ 
$p(B) = 0,18$&$p(B) = 0,12$&$p(B) = 0,66$&$p(B) = 0,3$\\ 
\hline
\end{tabularx}

\medskip

\begin{tabular}{@{\hspace{0.03\linewidth}} | p{0.95\linewidth}}
On a complété l'arbre de probabilités en bleu.

D'après la formule des probabilités totales:

$\aligned[b]
p\left (B\right )
& = p\left (A\cap B\right ) + p\left (\overline{A}\cap B\right )
= p\left (A\right ) \times p_{A}\left (B\right ) + p\left (\overline{A}\right ) \times p_{\overline{A}}\left (B\right ) \\
& =  0,4\times 0,3 + 0,6\times 0,1 = 0,12 + 0,06 = 0,18
\endaligned$
\hfill\textbf{Réponse A.}
\end{tabular}

\bigskip

\textbf{Question 2}

Une tablette coûte $200$ euros. Son prix diminue de $30\,\%$. \\
Le prix après cette diminution est :

\medskip

\begin{tabularx}{\linewidth}{|*{4}{>{\centering \arraybackslash}X|}}
\hline
\textbf{A.}&\textbf{B.}& \textbf{C.}& \textbf{D.}\\
140 euros&170 euros&194~euros&197~euros\\ 
\hline
\end{tabularx}

\medskip

\begin{tabular}{@{\hspace{0.03\linewidth}} | p{0.95\linewidth}}
Diminuer de 30\,\%, c'est multiplier par $1-\dfrac{30}{100}$ soit $0,7$.

$200\times 0,7=140$ \hfill\textbf{Réponse A.}
\end{tabular}

\bigskip

\textbf{Question 3}

\medskip

Une réduction de 50\,\% suivi d'une augmentation de 50\,\% équivaut à :

\medskip

\begin{tabularx}{\linewidth}{|*{4}{>{\centering \arraybackslash}X|}}
\hline
\textbf{A.}&\textbf{B.}& \textbf{C.}& \textbf{D.}\\ 
une réduction de 50\,\%&une réduction de 25\,\%& une augmentation de 25\,\%&une augmentation de 75\,\%\\ 
\hline
\end{tabularx}

\medskip

\begin{tabular}{@{\hspace{0.03\linewidth}} | p{0.95\linewidth}}
Pratiquer une réduction de 50\,\%, c'est multiplier par $1-\dfrac{50}{100}=0,5$.

Augmenter de 50\,\%, c'est multiplier par $1+\dfrac{50}{100}=1,5$.

Réduire de 50\,\% puis augmenter de 50\,\%, c'est multiplier par $0,5\times 1,5= 0,75$.

$0,75=1-\dfrac{25}{100}$ donc multiplier par $0,75$ c'est effectuer une réduction de 25\,\%.

\hfill\textbf{Réponse B.}
\end{tabular}


\textbf{Question 4}

\medskip

Dans un lycée, le quart des élèves sont internes, parmi eux, la moitié sont des filles. 

La proportion des filles internes par rapport à l'ensemble des élèves du lycée est égale à :

\medskip

\begin{tabularx}{\linewidth}{|*{4}{>{\centering \arraybackslash}X|}}
\hline
\textbf{A.}&\textbf{B.}& \textbf{C.}& \textbf{D.}\\
 4\,\%&12,5\,\%&25\,\%& 50\,\%\\ 
 \hline
\end{tabularx}

\medskip

\begin{tabular}{@{\hspace{0.03\linewidth}} | p{0.95\linewidth}}
La moitié du quart est $\dfrac{1}{2}\times \dfrac{1}{4}=\dfrac{1}{8}$ 
et
$\dfrac{1}{8} = \dfrac{12,5}{100}$ 
\hfill\textbf{Réponse B.}
\end{tabular}

\bigskip

\textbf{Question 5}

\medskip

On considère le nombre $N = \dfrac{10^7}{5^2}$. On a :

\medskip

\begin{tabularx}{\linewidth}{|*{4}{>{\centering \arraybackslash}X|}}
\hline
\textbf{A.}&\textbf{B.}& \textbf{C.}& \textbf{D.}\\ 
 $N = 2^5$&$N = \np{20000}$&$N = \dfrac{1}{10^5}$&$N = 4 \times 10^5$\rule[-10pt]{0pt}{25pt}\\ 
 \hline
\end{tabularx}

\medskip

\begin{tabular}{@{\hspace{0.03\linewidth}} | p{0.95\linewidth}}
$N = \dfrac{10^7}{5^2} = \dfrac{10^2\times 10^5}{5^2}
= \dfrac{10^2}{5^2}\times 10^5
= \left (\dfrac{10}{5}\right )^2 \times 10^5
= 2^2 \times 10^5
= 4\times 10^5$

\hfill\textbf{Réponse D.}
\end{tabular}

\bigskip

\textbf{Question 6}

\medskip

Un appareil a besoin d'une énergie de $7,5 \times 10^6$ joules pour se mettre en route.

À combien de kiloWatts-heure (kWh) cela correspond-il ?

\hfill \emph{Données} : 1 kWh $= 3,6 \times 10^6$~J
\medskip

\begin{tabularx}{\linewidth}{|*{4}{>{\centering \arraybackslash}X|}}
\hline
\textbf{A.}&\textbf{B.}& \textbf{C.}& \textbf{D.}\\ 
0,5 kWh&2,08 kWh&5,3 kWh&20,35 kWh\\ 
\hline
\end{tabularx}

\medskip

\begin{tabular}{@{\hspace{0.03\linewidth}} | p{0.95\linewidth}}
1 kWh $= 3,6 \times 10^6$~J donc 2 kWh $= 7,2 \times 10^6$~J

Donc $7,5 \times 10^6$~J correspond à un peu plus de 2 kWh.
\hfill\textbf{Réponse B.}
\end{tabular}

\bigskip

\textbf{Question 7}

\medskip

Le plan est muni d'un repère orthogonal. On note $d$ la droite passant par les points 
A$(0~;~-1)$ et B(2~;~5).

Le coefficient directeur de la droite $d$ est égal à :

\medskip

\begin{tabularx}{\linewidth}{|*{4}{>{\centering \arraybackslash}X|}}
\hline
\textbf{A.}&\textbf{B.}& \textbf{C.}& \textbf{D.}\\ 
$- \dfrac 12$&$2$&3&$\dfrac 13$\rule[-10pt]{0pt}{25pt}\\ 
\hline
\end{tabularx}

\medskip

\begin{tabular}{@{\hspace{0.03\linewidth}} | p{0.95\linewidth}}
Le coefficient directeur de la droite $d$ est égal à 
$\dfrac{y_{\text{B}}- y_{\text{A}}}{x_{\text{B}}- x_{\text{A}}}
= \dfrac{5-(-1)}{2-0}= \dfrac{6}{2} = 3$.

\hfill\textbf{Réponse C.}
\end{tabular}


\textbf{Question 8}

\begin{minipage}{0.7\linewidth}
On a représenté ci-contre une droite $D$.\\

Parmi les quatre équations ci-dessous, la seule susceptible de représenter la droite $D$
est :
\end{minipage}\hfill
\begin{minipage}{0.23\linewidth}
\psset{unit=1cm,arrowsize=2pt 3}
\begin{pspicture*}(-1.4,-1.4)(1.6,1.4)
\psaxes[Dx=5,Dy=5]{->}(0,0)(-1.4,-1.4)(1.6,1.4)
\psplot[plotpoints= 600,linecolor=blue]{-1.4}{1.4}{2 x mul neg}
\uput[u](1.3,0){\small $x$}\uput[r](0,1.3){\small $y$}
\uput[l](-0.5,1){\blue $D$}
\uput[dl](0,0){\footnotesize 0}
\end{pspicture*}
\end{minipage}

\medskip

\begin{tabularx}{\linewidth}{|*{4}{>{\centering \arraybackslash}X|}}
\hline
\textbf{A.}			&\textbf{B.}		& \textbf{C.}			& \textbf{D.}\\
$2x - y = 0$& $2x  + y + 1 = 0$ & $y = x^2 -(x + 1)^2 + 1$ & $y = 2x - 1$\\ 
\hline
\end{tabularx}

\medskip

\begin{tabular}{@{\hspace{0.03\linewidth}} | p{0.95\linewidth}}
\begin{list}{\textbullet}{La droite a un coefficient directeur négatif et passe par l'origine.}
\item \textbf{A.} $2x-y=0$ équivaut à $y=2x$; on peut éliminer car le coefficient directeur est positif.
\item \textbf{B.} $2x  + y + 1 = 0$ équivaut à $y=-2x-1$; on peut éliminer car la droite ne passe pas par l'origine.
\item \textbf{D.} $y = 2x - 1$; on peut éliminer car la droite ne passe pas par l'origine.
\item \textbf{C.} $y = x^2 -(x + 1)^2 + 1$ équivaut à $y = x^2 -\left (x^2+2x+1\right ) +1$

\qquad  équivaut à $y=x^2 - x^2 - 2x -1+1$ soit $y=-2x$.
\end{list}

\hfill\textbf{Réponse C.}
\end{tabular}

\bigskip

\textbf{Question 9}

\medskip

On note $\mathcal{S}$ l'ensemble des solutions de l'équation $x^2 = 10$
sur $\R$. On a :

\medskip

\begin{tabularx}{\linewidth}{|*{4}{>{\centering \arraybackslash}X|}}
\hline
\textbf{A.}	&\textbf{B.}		&\textbf{C.}			&\textbf{D.}\\ 
$\mathcal{S} = \{-5~;~5\} $ &$\mathcal{S} = \left\{-\sqrt 5~;~\sqrt 5\strut\right\}$ & $\mathcal{S} = \left\{-\sqrt{10}~;~\sqrt{10}\strut\right\}$ & $\mathcal{S} = \emptyset$\\
\hline
\end{tabularx}

\medskip

\begin{tabular}{@{\hspace{0.03\linewidth}} | p{0.95\linewidth}}
L'équation admet deux solutions $-\sqrt{10}$ et $\sqrt{10}$.
\hfill\textbf{Réponse C.}
\end{tabular}

\bigskip

\textbf{Question 10}

\medskip

La fonction $f$ définie sur $\R$ par 
$f(x) = (3x - 15)(x + 2)$
admet pour tableau de signes :

\medskip

\renewcommand{\arraystretch}{1.2}
\psset{unit=1.8em}

\begin{tabularx}{\linewidth}{|*{2}{>{\centering\arraybackslash}X|}}
\hline
\textbf{A.} & \textbf{B.} \\
\hline
$\begin{tablvar}{7}
\hline
x & -\infty && -2 && 5 && +\infty \\
\hline
f(x) & & + & 0 & - & 0 & + & \\
\hline
\end{tablvar}$
&
$\begin{tablvar}{7}
\hline
x & -\infty && -2 && 5 && +\infty \\
\hline
f(x) & & - & 0 & + & 0 & - & \\
\hline
\end{tablvar}$ \\
\hline
\textbf{C.} & \textbf{D.} \\
\hline
$\begin{tablvar}{7}
\hline
x & -\infty && -5 && 2 && +\infty \\
\hline
f(x) & & + & 0 & - & 0 & + & \\
\hline
\end{tablvar}$
&
$\begin{tablvar}{7}
\hline
x & -\infty && -5 && 2 && +\infty \\
\hline
f(x) & & - & 0 & + & 0 & - & \\
\hline
\end{tablvar}$ \\
\hline
\end{tabularx}

\renewcommand{\arraystretch}{1}
\medskip

\begin{tabular}{@{\hspace{0.03\linewidth}} | p{0.95\linewidth}}
$3x-15$ s'annule et change de signe pour $x=5$, et $x+2$ s'annule et change de signe pour $x=-2$.

On établit le tableau de signes complet.

\[\begin{tablvar}{7}
\hline
x & -\infty && -2 && 5 && +\infty \\
\hline
3x-15 & & - & \barre & - & \barre[0] & + & \\
\hline
x+2 & & - & \barre[0] & + & \barre & + & \\
\hline
f(x) & & + & \barre[0] & - & \barre[0] & + & \\
\hline
\end{tablvar}\]

\hfill\textbf{Réponse A.}
\end{tabular}

\bigskip

\textbf{Question 11}

\medskip

L'expression développée de $(2x + 0,5)^2$ est :

\medskip

\begin{tabularx}{\linewidth}{|*{4}{>{\centering \arraybackslash}X|}}
 \hline
 \textbf{A.}			&\textbf{B.}		& \textbf{C.}			& \textbf{D.}\\
$4x^2 + x + 0,25$	&	$4x^2+ 4x + 2$	&	$4x^2 + 2x + 0,25$	&$4x^2 + 2x + 1$\\ 
\hline
\end{tabularx}

\medskip

\begin{tabular}{@{\hspace{0.03\linewidth}} | p{0.95\linewidth}}
$(2x + 0,5)^2 = (2x)^2 + 2\times 2x \times 0,5 + (0,5)^2= 4x^2 + 2x +0,25$
\hfill\textbf{Réponse C.}
\end{tabular}

\bigskip

\textbf{Question 12}

\medskip

Lorsqu'un point mobile suit une trajectoire circulaire de rayon $R$, en mètre (m), son accélération centripète $a$ (en m/s$^2$) s'exprime en fonction de la vitesse  (en m/s) de la manière suivante:
$a = \dfrac{v^2}{R}$.

L'expression permettant, à partir de cette formule, d'exprimer la vitesse $v$ est :

\medskip

\begin{tabularx}{\linewidth}{|*{4}{>{\centering \arraybackslash}X|}}
\hline
\textbf{A.}&\textbf{B.}& \textbf{C.}& \textbf{D.}\\ 
$v = aR^2$&$v = \sqrt{aR}$&$v = \sqrt{\dfrac aR}$&$v = \dfrac{a^2}{R}$\rule[-10pt]{0pt}{25pt} \\
\hline
\end{tabularx}

\medskip

\begin{tabular}{@{\hspace{0.03\linewidth}} | p{0.95\linewidth}}
$a = \dfrac{v^2}{R}$ équivaut à $aR=v^2$ et donc $v=\ds\sqrt{aR}$.
\hfill\textbf{Réponse B.}
\end{tabular}

%%%%%%%%%%%%%%%%%%%%%%%%%%%%%%%%%%%%%%%%%%%%%%%
\vspace{1cm}

\begin{center}
\textbf{DEUXIÈME PARTIE (14 pts)}
\end{center}

\medskip

