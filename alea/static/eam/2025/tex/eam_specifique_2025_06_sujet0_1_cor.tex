
\medskip

%\textbf{Pour cette première partie, aucune justification n'est demandée et une seule réponse est possible par question. Pour chaque question, reportez son numéro sur votre copie et indiquez votre réponse.}
%
%\medskip

\begin{enumerate}
\item Donner un ordre de grandeur de $101 \times 99$ :

\begin{tabularx}{\linewidth}{*{4}{X}}
\textbf{a.~} \np{100}&\textbf{b.~}\np{1000}&\textbf{c.~}\np{10000}&\textbf{d.~}\np{100000}
\end{tabularx}

\medskip

\begin{tabular}{@{\hspace{0.03\linewidth}} | p{0.95\linewidth}}
$101\approx 100$ et $99\approx 100$ donc $101\times 99\approx 100\times 100$ et donc $101\times 99\approx \np{10000}$

\hfill\textbf{Réponse c.}
\end{tabular}

\bigskip

\item Un prix augmente de $20\,\%$ puis diminue de $20\,\%$.

Après ces deux évolutions, on peut affirmer que :

\textbf{a.~} Le prix est égal à sa valeur de départ.

\textbf{b.~}Le prix est strictement supérieur à sa valeur de départ.

\textbf{c.~}Le prix est strictement inférieur à sa valeur de départ.

\textbf{d.~}On ne peut pas savoir : cela dépend de la valeur de départ.

\medskip

\begin{tabular}{@{\hspace{0.03\linewidth}} | p{0.95\linewidth}}
Augmenter de 20\,\%, c'est multiplier par $1+\dfrac{20}{100}=1,2$.

Diminuer de 20\,\%, c'est multiplier par $1-\dfrac{20}{100}=0,8$.

Augmenter  puis diminuer de 20\,\%, c'est multiplier par $1,2\times 0,8=0,96<1$.

\hfill\textbf{Réponse c.}
\end{tabular}

\bigskip

\item Par combien faut-il multiplier une quantité positive pour que celle-ci diminue de 2,3\,\%?

\begin{tabularx}{\linewidth}{*{4}{X}}
\textbf{a.~} 1,23&\textbf{b.~}0,977&\textbf{c .~}0,77&\textbf{d.~} 1,023
\end{tabularx}

\medskip

\begin{tabular}{@{\hspace{0.03\linewidth}} | p{0.95\linewidth}}
Diminuer de $2,3$\,\%, c'est multiplier par $1-\dfrac{2,3}{100}=\dfrac{100}{100}-\dfrac{2,3}{100}=\dfrac{97,7}{100}=0,977$.

\hfill\textbf{Réponse b.}
\end{tabular}

\bigskip

\item Dans un lycée, 50 élèves étudient le Grec, ce qui représente $4\,\%$ de du nombre d'élèves inscrits dans ce lycée.

Le nombre d'élèves inscrits dans ce lycée est égal à :

\begin{tabularx}{\linewidth}{*{4}{X}}
\textbf{a.~} 2&\textbf{b.~}200&\textbf{c.~}125&\textbf{d.~}\np{1250}
\end{tabularx}

\medskip

\begin{tabular}{@{\hspace{0.03\linewidth}} | p{0.95\linewidth}}
Si $n$ est le nombre d'élèves du lycée, on a: $50=\dfrac{4}{100}\times n$ donc $\dfrac{50\times 100}{4}=n$ donc $n=\np{1250}$.
\hfill\textbf{Réponse d.}
\end{tabular}

\bigskip

\item Le volume d'un glacier diminue de $3\,\%$ chaque année.

Si $V(n)$ désigne le volume du glacier pour l'année $n$ on a :

\begin{tabularx}{\linewidth}{*{2}{X}}
\textbf{a.~} $V(n+1) = V(n) - 0,03$&\textbf{b.~}$V(n+1) = 0,03 \times V(n)$\\
\textbf{c.~}$V(n+1) = 0,97 \times V(n)$&\textbf{d.~}$V(n+1) = V(n)-0,97$
\end{tabularx}

\medskip

\begin{tabular}{@{\hspace{0.03\linewidth}} | p{0.95\linewidth}}
Diminuer de 3\,\%, c'est multiplier par $1-\dfrac{3}{100}=0,97$.
\hfill\textbf{Réponse c.}
\end{tabular}

\bigskip

\item ~

\begin{minipage}{0.65\linewidth}
Dans un repère du plan on a représenté une droite $D$.\\

Le coefficient directeur de cette droite est égal à :

\begin{tabularx}{\linewidth}{*{4}{X}}
\textbf{a.~}$-3$&\textbf{b.~}$-1$&\textbf{c.~} 2&\textbf{d.~} 3
\end{tabularx}
\end{minipage}
\hfill
\begin{minipage}{0.3\linewidth}
\scalebox{0.5}{
\psset{unit=0.8cm,arrowsize=2pt 2}
\begin{pspicture*}(-2.5,-4)(3,4)
\psgrid[subgriddiv=5,  gridlabels=0, gridcolor=lightgray, subgridcolor=lightgray!50]
\psaxes[linewidth=1.25pt]{->}(0,0)(-2.5,-3.99)(3,4)
\psplot[plotpoints=600,linewidth=1.25pt,linecolor=blue]{-2}{3}{2 3 x mul sub}
\uput[ur](1.3,-2){\blue $D$}
\end{pspicture*}
}
\end{minipage}
%\end{multicols}

\medskip

\begin{tabular}{@{\hspace{0.03\linewidth}} | p{0.95\linewidth}}
La droite passe par les points de coordonnées (0 ; 2) et $(1\;; -1)$ donc son coefficient directeur est égal à $\dfrac{-1-2}{1-0}=-3$.
\hfill\textbf{Réponse~a.}
\end{tabular}

\bigskip

\item Dix stylos coûtent en tout 13 euros.

Le prix de trois stylos est égal à :

\begin{tabularx}{\linewidth}{*{4}{X}}
\textbf{a.~} $3,60$ euros&\textbf{b.~} $6,90$ euros&\textbf{c.~} $3,90$ euros&\textbf{d.~} $6,50$ euros
\end{tabularx}

\medskip

\begin{tabular}{@{\hspace{0.03\linewidth}} | p{0.95\linewidth}}
On fait un tableau de proportionnalité:
\begin{tabular}{|c|c|c|}
\hline
nombre & 10 & 3\\
\hline
prix & 13 & ?\\
\hline
\end{tabular}

$\dfrac{3\times 13}{10}=3,90$
\hfill\textbf{Réponse c.}
\end{tabular}

\bigskip

\item Une athlète parcourt 1 km en 5 minutes. Quelle est sa vitesse moyenne ?

\begin{tabularx}{\linewidth}{*{4}{X}}
\textbf{a.~} 8~km/h &\textbf{b.~}10~km/h&\textbf{c.~} 12~km/h&\textbf{d.~} 14~km/h
\end{tabularx}

\medskip

\begin{tabular}{@{\hspace{0.03\linewidth}} | p{0.95\linewidth}}
Dans une heure, il y a 60 minutes donc 12 fois 5 minutes.
L'athlète parcourt 1 km en 5 minutes donc 12 km en 60 minutes.
\hfill\textbf{Réponse c.}
\end{tabular}

\bigskip

\item Sur 60 personnes présentes à une exposition, on distingue trois groupes:\\
 groupe A : 30 personnes,  groupe B : 12 personnes, et  groupe C : les autres.

Quelle représentation décrit la situation ?

\begin{tabularx}{\linewidth}{*{4}{>{\centering\arraybackslash}X}}
\textbf{a.~}&\textbf{b.~}&\textbf{c.~}&\textbf{d.~}\\
\psset{unit=1cm}
\begin{pspicture}(-1,-1)(1,1)
\pswedge[fillstyle=solid,fillcolor=black](0.05;30){0.95}{-30}{90}
\pswedge[fillstyle=solid,fillcolor=gray](0.05;150){0.95}{90}{210}
\pswedge[fillstyle=solid,fillcolor=blue](0.05;270){0.95}{210}{330}
\end{pspicture}&
\psset{unit=1cm}
\begin{pspicture}(-1,-1)(1,1)
\pswedge[fillstyle=solid,fillcolor=black](0.05;0){0.95}{-90}{90}
\pswedge[fillstyle=solid,fillcolor=gray](0.05;150){0.95}{90}{210}
\pswedge[fillstyle=solid,fillcolor=blue](0.05;240){0.95}{210}{270}
\end{pspicture}&
\psset{unit=1cm}
\begin{pspicture}(-1,-1)(1,1)
\pswedge[fillstyle=solid,fillcolor=black](0.05;30){0.95}{-30}{90}
\pswedge[fillstyle=solid,fillcolor=gray](0.05;195){0.95}{90}{300}
\pswedge[fillstyle=solid,fillcolor=blue](0.05;315){0.95}{300}{330}
\end{pspicture}&
\psset{unit=1cm}
\begin{pspicture}(-1,-1)(1,1)
\pswedge[fillstyle=solid,fillcolor=black](0.05;0){0.95}{-90}{90}
\pswedge[fillstyle=solid,fillcolor=gray](0.05;135){0.95}{90}{180}
\pswedge[fillstyle=solid,fillcolor=blue](0.05;225){0.95}{180}{270}
\end{pspicture}
\end{tabularx}

\medskip

\begin{tabular}{@{\hspace{0.03\linewidth}} | p{0.95\linewidth}}
Le groupe A contient la moitié de l'effectif total donc on peut éliminer les représentations A et c.

Le groupe B contient 12 personnes, et le groupe C le reste soit 18 personnes. Les effectifs sont différents donc on peut éliminer la représentation d.

\hfill\textbf{Réponse b.}
\end{tabular}

\bigskip

\item On considère les deux séries ci-dessous.

Série A : 9 ; 10 ; 10 ; 11 et 
Série B : 7 ; 10 ; 10 ; 13

Une seule des quatre propositions suivantes est vraie.

\textbf{a.~} La moyenne de la série A est strictement supérieure à la moyenne de la série B.

\textbf{b.~} La moyenne de la série B est strictement supérieure à la moyenne de la série A.

\textbf{c.~} L'écart--type de la série A est strictement supérieur à l'écart-type de la série B.

\textbf{d.~} L'écart--type de la série B est strictement supérieur à l'écart-type de la série A.

\medskip

\begin{tabular}{@{\hspace{0.03\linewidth}} | p{0.95\linewidth}}
Les deux séries ont la même moyenne, 10, mais les nombres sont plus éloignés de 10 dans la série B.
\hfill\textbf{Réponse d.}
\end{tabular}

\bigskip

\item Le volume $V$ d'un cylindre de hauteur $h$ et de rayon $r$ est égal à $V=\pi r^{2} h$.

On cherche à isoler $h$. On a :

\begin{tabularx}{\linewidth}{*{4}{X}}
\textbf{a.~}$h=\sqrt{\frac{V}{\pi r^{2}}}$&\textbf{b.~} $h=\frac{\pi r^{2}}{V}$&\textbf{c.~}$h=\frac{V}{\pi r^{2}}$&\textbf{d.~} $h=\frac{r^{2}}{\pi V}$
\end{tabularx}

\medskip

\begin{tabular}{@{\hspace{0.03\linewidth}} | p{0.95\linewidth}}
$V=\pi r^{2} h$ donc $\dfrac{V}{\pi r^2}=h$
\hfill\textbf{Réponse c.}
\end{tabular}

\bigskip

\item ~

\begin{minipage}{8cm}
Soit $f$ une fonction définie sur l'intervalle $[-4~;~4]$ dont la représentation graphique est donnée ci-contre.

L'ensemble $\mathcal{S}$ des solutions de l'équation $f(x) = 0$ est :

\begin{tabularx}{\linewidth}{*{2}{X}}
\textbf{a.~} $\mathcal{S}=\{0\}$&\textbf{b.~} $\mathcal{S}=[-3~;~2]$\\
\textbf{c.~} $\mathcal{S}=\{-3~;~-1~;~1~;~2\}$&\textbf{d.~} $\mathcal{S}=\{1,5\}$
\end{tabularx}
\end{minipage}
\hfill
\begin{minipage}{4.5cm}
\scalebox{0.7}{
\psset{unit=0.7cm,arrowsize=2pt 3}
\begin{pspicture*}(-4.1,-3.5)(4.1,4.2)
\psgrid[subgriddiv=2,  gridlabels=0, gridcolor=lightgray, subgridcolor=lightgray!50]
\psaxes[linewidth=1.25pt,labelFontSize=\scriptstyle](0,0)(-4.1,-3.5)(4.1,4.2)
\uput[u](3.9,0){$x$}\uput[r](0,3.9){$y$}
\psplot[plotpoints=2000,linewidth=1.25pt,linecolor=red]{-4}{4}{x 1 sub x 2 sub mul x 1 add mul x 3 add mul 4 div}
\end{pspicture*}
}
\end{minipage}

\medskip

\begin{tabular}{@{\hspace{0.03\linewidth}} | p{0.95\linewidth}}
Les solutions de l'équation $f(x) = 0$ sont les abscisses des points d'intersection de la courbe représentant $f$ et de l'axe des abscisses.
\hfill\textbf{Réponse c.}
\end{tabular}

\end{enumerate}

\bigskip

\begin{center}
\textbf{\textsc{DEUXIÈME  PARTIE} (14 points)}
\end{center}

\medskip

