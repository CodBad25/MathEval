Indiquer si les affirmations sont vraies ou fausses. La justification est obligatoire.

\centering \textit{Les deux questions sont indépendantes.}

\begin{enumerate}
\item Un employé reçoit des appels téléphoniques. \\
On estime que la probabilité qu'un appel dure plus de cinq minutes est égale à 0,3. \\
On suppose que les durées des différents appels sont indépendantes.

Ce matin, l'employé reçoit deux appels.

\textbf{Affirmation 1 :} \\
La probabilité que les deux appels durent tous les deux plus de cinq minutes est égale à 0,09.

\textbf{Affirmation 2 :} \\
La probabilité qu'un appel exactement sur les deux dure plus de cinq minutes est égale à 0,21.

\bigskip

\item Le gérant d'une piscine s'intéresse à la présence de bactéries dans l'eau. \\
Il effectue un prélèvement. Ce prélèvement montre que la concentration de bactéries est égale à 1\,000 bactéries par millilitre. Le seuil maximal autorisé est égal à 1\,500 bactéries par millilitre.

On admet que la concentration de bactéries est modélisée par la fonction $f$ définie sur l'intervalle $[0~;~+\infty[$ par
\[ f(t) = 1,1^t, \]
où $f(t)$ désigne la concentration, en milliers de bactéries par millilitre, et $t$ désigne la durée, en heure, écoulée depuis que le prélèvement a été effectué.

\textbf{Affirmation 3 :} \\
La fonction $f$ est croissante sur l'intervalle $[0~;~+\infty[$.

\textbf{Affirmation 4 :} \\
La concentration de bactéries deux heures après le prélèvement est inférieure au seuil maximal autorisé.
\end{enumerate}

