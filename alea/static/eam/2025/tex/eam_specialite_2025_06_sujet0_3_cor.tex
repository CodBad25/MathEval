
\medskip

Le plan est muni d'un repère orthogonal.

\medskip

\textbf{Partie A}

\medskip

On considère la fonction $P$ définie sur l'intervalle $[-5\;; 3]$ par :
$P(x) = 2 x^{2}+ x - 10$.

\begin{enumerate}
\item 
	\begin{enumerate}
		\item% Déterminer les racines de $P$.
$P(x) = 2 x^{2}+ x - 10$ est un polynôme du second degré.

$\Delta=b^2-4ac=1^2-4\times 2 \times (-10)=1+80=81=9^2$

Les racines de $P$ sont:

$x'=\dfrac{-b-\sqrt{\Delta}}{2a}=\dfrac{-1-9}{4}=-\dfrac{10}{4}=-\dfrac{5}{2}$ et		
$x''=\dfrac{-b+\sqrt{\Delta}}{2a}=\dfrac{-1+9}{4}=\dfrac{8}{4}=2$
		
		\item% En déduire l'axe de symétrie de la parabole d'équation $y= P(x)$.
$\dfrac{x'+x''}{2}= \dfrac{-\frac{5}{2}+2}{2} = \dfrac{\frac{-5}{2} + \frac{4}{2}}{2} = \dfrac{\frac{-1}{2}}{2} = -\dfrac{1}{4}$

L'axe de symétrie de la parabole d'équation $y= P(x)$ est donc la droite verticale d'équation $x=-\dfrac{1}{4}$.
			\end{enumerate}
			
\item %Établir le tableau de signes de la fonction $P$ sur l'intervalle $[-5~:~3]$.
Le polynôme $P$ a pour racines $x'=-\dfrac{5}{2}$ et $x''=2$, donc 

$P(x)=a(x-x')(x-x'') = 2\left (x-\left( -\dfrac{5}{2}\right ) \right ) \left (x-2\right )= 2\left (x+\dfrac{5}{2}\right )\left (x-2\right )$

On établit  le tableau de signes de la fonction $P$ sur l'intervalle $[-5~:~3]$.

\[\begin{tablvar}{7}
\hline
x & -5 && -\frac{5}{2} && 2 && 3 \\
\hline
x+\frac{5}{2} & & - & \barre[0] & + & \barre & + & \\
\hline
x-2 & & - & \barre & - & \barre[0] & + & \\
\hline
2 & & + & \barre & + & \barre & + & \\
\hline
P(x) & & + & \barre[0] & - & \barre[0] & + & \\
\hline
\end{tablvar}\]
\end{enumerate}

\medskip

\textbf{Partie B}

\medskip

On considère la fonction $f$ définie et dérivable sur l'intervalle  $[-5~:~3]$  dont on donne ci-dessous la courbe représentative $\mathcal{C}_{f}$.

\begin{center}
\fbox{
\psset{xunit=1.2cm,yunit=0.2cm,arrowsize=2pt 3}
\begin{pspicture*}(-5.5,-14)(3.5,22)
\psgrid[xunit=0.6cm,yunit=1cm,subgriddiv=5,  gridlabels=0, gridcolor=lightgray,subgridcolor=lightgray!50](-11,-3)(7,6)
\psaxes[linewidth=0pt,Dx=0.5,Dy=50,labels=none](0,0)(-5.49,-14)(3.5,22)
\psaxes[linewidth=1.25pt,Dx=1,Dy=5,labelFontSize=\scriptstyle]{->}(0,0)(-5.49,-14)(3.5,22)
\multido{\n=-4.5+1.0}{8}{\uput[u](\n,0){\scriptsize\np{\n}}}
\uput{12pt}[dl](0,0){\scriptsize 0}
\psplot[plotpoints=2000,linewidth=1.25pt,linecolor=red]{-5}{3}{x dup mul 4 mul 14 x mul sub 8 add 2.71828 0.5 x mul exp mul}
\uput[u](-4.2,17){\red $\mathcal{C}_{f}$}
\psdots(2,-10.873)\uput[d](2,-10.873){A}
\psline[linecolor=blue](-5.5,-10.873)(3.5,-10.873)
\uput [u](-4.3,-10.873){\blue $T$}
%%%%%%%%%%%%%%%%%%%%%%
\psset{linecolor=blue}
\psline[linestyle=dashed](-2.5,0)(-2.5,19.48)
\psline[linestyle=dashed](2,0)(2,-10.87)
\psline[linewidth=1.5pt]{]-[}(-2.5,0)(2,0)
\end{pspicture*}
}
\end{center}

La tangente $T$ à la courbe $\mathcal{C}_{f}$ au point A d'abscisse 2 est horizontale.

\begin{enumerate}
\item %Donner la valeur du nombre dérivé $f'(2)$.
$f'(2)$ est le coefficient directeur de la tangente à la courbe $\mathcal{C}_f$ au point de la courbe d'abscisse 2; c'est donc le coefficient directeur de la droite $T$. Or la droite $T$ est horizontale, donc son coefficient directeur est égal à 0.

On en déduit que $f'(2)=0$.

\item% Résoudre, avec la précision permise par le graphique, l'inéquation $f'(x) < 0$.
Les solutions de l'inéquation $f'(x) < 0$ sont les valeurs de $x$ pour lesquelles la fonction $f$ est strictement décroissante, soit l'intervalle $]-2,5\;; 2[$ d'après le graphique.

\item On sait que sur l'intervalle $[-5~;~3]$ : $f(x)=\left(4 x^{2}-14 x + 8\right) \e^{0,5x}$.

%Démontrer que, pour tout $x$ appartenant à l'intervalle $[-5~;~3]$, on a : $f'(x)= P(x) \e^{0,5 x}$

D'après la formule de dérivation d'un produit:

$\aligned
f'(x)
&= \left (4\times 2x - 14\right )\times \e^{0,5x} + \left(4 x^{2}-14 x + 8\right)\times 0,5 \e^{0,5x} \\
&= \left ( 8x-14 + \left (4x^2-14x+8\right )\times 0,5\right )\e^{0,5x}\\
&= \left ( 8x-14 +2x^2-7x+4\right )\e^{0,5x}\\
&= \left ( 2x^2 +x-10\right )\e^{0,5x}\\
&= P(x)\e^{0,5x}
\endaligned$

\item %En utilisant les résultats de la \textbf{partie A}, dresser le tableau de variation de la fonction $f$ sur l'intervalle $[-5~;~3]$. (Il n'est pas demandé de calculer les images).
D'après la partie A, on connaît le signe de $P(x)$. De plus, on sait que $\e^{0,5x}>0$ pour tout réel $x$.
On peut donc établir le tableau de signes de $f'(x)$, puis le tableau de variations de la fonction $f$.

\[\renewcommand{\fleche}{\ncline[linewidth=1.2pt,arrowsize=2pt 3,nodesep=2.5pt]{->}}
\begin{tablvar}{7}
\hline
x & -5 && -\frac{5}{2} && 2 && 3 \\
\hline
P(x) & & + & \barre[0] & - & \barre[0] & + & \\
\hline
\e^{0,5x} & & + & \barre & + & \barre & + & \\
\hline
f'(x) = P(x)\e^{0,5x} & & + & \barre[0] & - & \barre[0] & + & \\
\hline
\variations{\mil{\text{variations de } f} & \bas{~} && \haut{~} && \bas{~} && \haut{~}}
\hline
\end{tablvar}\]

\end{enumerate}

