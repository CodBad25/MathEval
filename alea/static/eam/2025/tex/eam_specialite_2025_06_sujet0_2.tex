En 2020, une ville comptait 10\,000 habitants.

On modélise l'évolution du nombre d'habitants de cette ville par la suite $(u_n)$ définie ainsi :

\[
\begin{cases}
u_{n+1} = 1,08u_n - 300, & n \in \N \\
u_0 = 10\,000~;
\end{cases}
\]

où $u_n$ représente le nombre d'habitants pour l'année $2020 + n$.

\begin{enumerate}
\item Indiquer ce que représente $u_1$ et calculer sa valeur.

\item On considère la suite $(v_n)$ définie pour tout entier naturel $n$ par $v_n = u_n - 3\,750$.
\begin{enumerate}
\item Déterminer $v_0$.
\item Démontrer que pour tout entier naturel $n$, on a $v_{n+1} = 1,08v_n$.
\item En déduire la nature de la suite $(v_n)$.
\item Pour tout entier naturel $n$, exprimer $v_n$ en fonction de $n$.
\item En déduire que pour tout entier naturel $n$, on a $u_n = 6\,250 \times 1,08^n + 3\,750$.
\end{enumerate}

\medskip

\item 

\begin{minipage}[t]{0.6\textwidth}
Le tableau ci-dessous, extrait d'une feuille automatisée de calcul, a été obtenu par recopie vers le bas après avoir saisi la formule suivante dans la cellule B2 :

\begin{center}
\fbox{\text{= 6250*1,08\^{}A2 + 3750}}
\end{center}

La municipalité envisage d'ouvrir une nouvelle école maternelle dès que la population atteindra 19\,000 habitants.

\medskip

La construction d'un tel établissement nécessitant deux ans, déterminer l'année à partir de laquelle la construction de l'école doit commencer.

\medskip

\underline{\textit{\textbf{Aide au calcul :}}}

$10\,000 - 3\,750 = 6\,250~;$

\smallskip

$1,08 \times 4\,050 = 4\,374~;$

\smallskip

$\dfrac{4\,050}{1,08} = 3\,750~;$

\smallskip

$3\,750 \times 1,08 = 4\,050~;$
\end{minipage}
\hfill
\begin{minipage}[t]{0.35\textwidth}
\vspace{0pt}
\begin{center}
\begin{tabular}{|c|c|c|}
\hline
\rowcolor{gray!20}
 & A & B \\
\hline
\cellcolor{gray!20}1 & $n$ & $u_n$ \\
\hline
\cellcolor{gray!20}2 & 0 & 10\,000 \\
\hline
\cellcolor{gray!20}3 & 1 & 10\,500 \\
\hline
\cellcolor{gray!20}4 & 2 & 11\,040 \\
\hline
\cellcolor{gray!20}5 & 3 & 11\,623,2 \\
\hline
\cellcolor{gray!20}6 & 4 & 12\,253,056 \\
\hline
\cellcolor{gray!20}7 & 5 & 12\,933,30048 \\
\hline
\cellcolor{gray!20}8 & 6 & 13\,667,96452 \\
\hline
\cellcolor{gray!20}9 & 7 & 14\,461,40168 \\
\hline
\cellcolor{gray!20}10 & 8 & 15\,318,31381 \\
\hline
\cellcolor{gray!20}11 & 9 & 16\,243,77892 \\
\hline
\cellcolor{gray!20}12 & 10 & 17\,243,28123 \\
\hline
\cellcolor{gray!20}13 & 11 & 18\,322,74373 \\
\hline
\cellcolor{gray!20}14 & 12 & 19\,488,56323 \\
\hline
\cellcolor{gray!20}15 & 13 & 20\,747,64829 \\
\hline
\cellcolor{gray!20}16 & 14 & 22\,107,46015 \\
\hline
\cellcolor{gray!20}17 & 15 & 23\,576,05696 \\
\hline
\cellcolor{gray!20}18 & 16 & 25\,162,14152 \\
\hline
\cellcolor{gray!20}19 & 17 & 26\,875,11284 \\
\hline
\cellcolor{gray!20}20 & 18 & 28\,725,12187 \\
\hline
\cellcolor{gray!20}21 & 19 & 30\,723,13162 \\
\hline
\end{tabular}
\end{center}
\end{minipage}
\end{enumerate}




