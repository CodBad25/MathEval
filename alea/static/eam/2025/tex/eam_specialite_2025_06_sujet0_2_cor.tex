
\medskip

En 2020, une ville comptait 10 000 habitants

On modélise l'évolution du nombre d'habitants de cette ville par la suite $(u_n)$ définie ainsi : 
\[\left\{\begin{array}{l c l}
u_0&=&\np{10000}\\
u_{n+1}&=&1,08u_n - 300,\quad n \in \N 
\end{array}\right.\]

où $u_n$ représente le nombre d'habitants pour l'année $2020 + n$.

%\medskip

\begin{enumerate}
\item% Indiquer ce que représente $u_1$ et calculer sa valeur.
$u_1$  représente le nombre d'habitants pour l'année $2020 + 1$ soit le nombre d'habitants en 2021.

$u_{1}=1,08 u_0 - 300 = 1,08\times \np{10000}-300 = \np{10800}-300=\np{10500}$

\item On considère la suite $(v_n)$ définie pour tout entier naturel $n$ par $v_n = u_n  - \np{3750}$. 
	\begin{enumerate}
		\item %Déterminer $v_0$.
$v_0=u_0 - 	\np{3750} = \np{10000} - \np{3750} = \np{6250}$
		
		\item % Démontrer que pour tout entier naturel $n$, on a $v_{n+1} = 1,08v_n$.
Pour tout entier naturel $n$, on a $v_n=u_n-\np{3750}$ donc $u_n=v_n+\np{3750}$.

$\aligned
v_{n+1}
& =u_{n+1}-\np{3750}
 = 1,08u_n -300 -\np{3750}
= 1,08\left ( v_n +\np{3750} \right ) -300 - \np{3750}\\
& =1,08 v_n + \np{4050} - \np{4050}
= 1,08 v_n
\endaligned$

		\item% En déduire la nature de la suite $(v_n)$.
Donc la suite $(v_n)$ est géométrique de raison $q=1,08$ et de premier terme $v_0 = \np{6250}$.

		\item On en déduit que, pour tout entier naturel $n$, on a 
$v_n=v_0\times q^n = \np{6250}\times 1,08^n$.		
		
		\item %En déduire que pour tout entier naturel, on a $u_n = \np{6250} \times 1,08^n + \np{3750}$
Pour tout $n$ on a $v_n= \np{6250}\times 1,08^n$ et $u_n=v_n+\np{3750}$ donc on peut en déduire que $u_n = \np{6250}\times 1,08^n +\np{3750}$.
	\end{enumerate}
	
\item ~

\begin{minipage}{0.66\linewidth}
Le tableau ci-contre, extrait d'une feuille automatisée de calcul, a été obtenu par recopie vers le bas après avoir saisi la formule suivante dans la cellule B2 :

\begin{center}
\fbox{\texttt{= 6250*1,08\^{}A2 + 3750}}
\end{center}

La municipalité envisage d'ouvrir une nouvelle école
 maternelle dès que la population atteindra \np{19000} habitants.\\

\end{minipage}
\hfill
\begin{minipage}{0.3\linewidth}
$\begin{array}{|>{\cellcolor{lightgray}}c|c|c|}\hline
\rowcolor{lightgray}&\text{A}&\text{B}\\ \hline
1&\text{n}& \text{Un}\\ \hline
2&0&\np{10000}\\ \hline
3&1&\np{10500}\\ \hline
\cdots & \cdots & \cdots \\ \hline
13&11&\np{18322,74373}\\ \hline
14&12&\np{19488,56323}\\ \hline
15&13&\np{20747,64829}\\ \hline
\cdots & \cdots & \cdots \\ \hline
\end{array}$
\end{minipage}

\medskip

La population dépasse $\np{19000}$ habitants pour $n=12$ soit en 2032.

La construction d'un tel établissement nécessite deux ans; il faut donc commencer les travaux en 2030.

\end{enumerate}

\bigskip

%%%%%%%%%%%%%%%%%%%%%%%%%%%%%%%%%%%%%%%%%%%%%%%%%%
