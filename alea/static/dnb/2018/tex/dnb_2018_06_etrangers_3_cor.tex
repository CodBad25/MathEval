
\medskip

\begin{enumerate}
\item Il y a 8 assemblages possibles. 4 assemblages ont un cadran rouge avec un bracelet de couleur et 4 assemblages ont un cadran jaune avec un bracelet de couleur.
\item Parmi les 8 assemblages possibles, un seul dispose d'un cadran rouge et d'un bracelet rouge. Cela correspond à 1 assemblage parmi 8. Donc la probabilité d'obtenir une montre toute rouge est de  $\dfrac{1}{8}$.
\item Parmi les 8 assemblages possibles, un dispose d'un cadran rouge et d'un bracelet rouge et un autre dispose d'un cadran jaune et d'un bracelet jaune, soit deux assemblages avec une montre d'une seule couleur. Cela correspond à 2 assemblages parmi 8. Donc la probabilité d'obtenir une montre d'une seule couleur est de  $\dfrac{2}{8} = \dfrac{1}{4}$.
\item Parmi les 8 assemblages possibles, puisque 2 assemblages concernent une montre d'une seule couleur, alors les 6 autres sont un assemblage de deux couleurs différentes. Cela correspond à 6 assemblages parmi 8. Donc la probabilité d'obtenir une montre de deux couleurs différentes est de  $\dfrac{6}{8} = \dfrac{3}{4}$.
\end{enumerate}

\vspace{0,5cm}

