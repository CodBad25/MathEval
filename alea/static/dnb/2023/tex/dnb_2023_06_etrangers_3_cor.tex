
\medskip

%Amir et Sonia ont chacun inventé un programme de calcul.
%
%\begin{center}
%\begin{tabularx}{\linewidth}{X X}
%Programme d'Amir&Programme de Sonia\\
%$\bullet~~$Choisir un nombre &$\bullet~~$Choisir un nombre \\
%$\bullet~~$Soustraire 5&$\bullet~~$Ajouter 3\\
%$\bullet~~$Prendre le double du résultat&$\bullet~~$ Multiplier le résultat par le nombre choisi\\
% &$\bullet~~$ Soustraire 16
%\end{tabularx}
%\end{center}

\begin{enumerate}
\item %Montrer que si le nombre choisi au départ est 6 alors on obtient 2 avec le programme d'Amir et on obtient 38 avec celui de Sonia.
Si le nombre choisi au départ est 6 alors avec le programme d'Amir on obtient: $(6 - 5) \times 2 = 2$.

Avec le programme de Sonia, on obtient : $(6 + 3) \times 6 - 16 =54 - 16 = 38$.
\item %Amir et Sonia souhaitent savoir s'il existe des nombres choisis au départ pour lesquels les deux programmes renvoient le même résultat.

%Pour cela, ils complètent la feuille de calcul ci-dessous :
%
%\begin{center}
%\begin{tabularx}{\linewidth}{|c|l|*{7}{>{\centering \arraybackslash}X|}}\hline
%&A&B &C&D &E &F &G &H\\ \hline
%1&Nombre choisi&$-2$&$-1$&0&1&2&3&4\\ \hline
%2&Programme d'Amir&$-14$& $-12$& $-10$& $-8$& $-6$& $-4$& $-2$\\ \hline
%3&Programme de Sonia& $-18$ &$-18$ &$-16$ &$-12$ &$-6$ &$2$ &$12$\\ \hline
%\end{tabularx}
%\end{center}

\emph{Aucune justification n'est attendue pour les deux questions ci-dessous.}

\medskip

	\begin{enumerate}
		\item %Parmi les trois propositions suivantes, recopier sur votre copie la formule qui a été saisie dans la cellule B2 avant d'être étirée vers la droite.
%\begin{center}
%\begin{tabularx}{\linewidth}{*{2}{>{\centering \arraybackslash}X|}>{\centering \arraybackslash}X}
%=(B1 $- 5)* 2$ &=($- 2- 5)*2$ &=B1 $- 5*2$
%\end{tabularx}
%\end{center}
La formule qui a été saisie dans la cellule B2 avant d'être étirée vers la droite est : 

=(B1 $- 5) *$ 2
		\item %En vous aidant de la feuille de calcul, quel nombre doivent-ils choisir pour obtenir des résultats égaux avec les deux programmes?
D'après la feuille de calcul, le nombre qu'ils doivent choisir pour obtenir des résultats égaux avec les deux programmes est $2$ puisque l'on obtient $-6$ avec les deux programmes.
	\end{enumerate}
\item Sonia et Amir souhaitent vérifier s'il existe d'autres nombres permettant d'obtenir des résultats égaux avec les deux programmes.

Pour cela, ils décident d'appeler $x$ le nombre choisi au départ de chacun des programmes.
	\begin{enumerate}
		\item %Montrer que le résultat obtenu avec le programme de Sonia est donné par 
Le résultat obtenu avec le programme de Sonia est donné par $(x+3)\times x - 16 = x^2 + 3x - 16$.
		\item %On admet que les programmes donnent le même résultat si on choisit comme nombre de départ les solutions de l'équation $(x - 2)(x + 3) = 0$.
		
%Résoudre cette équation et en déduire les valeurs pour lesquelles les deux programmes de calcul
%renvoient le même résultat.
Les programmes donnent le même résultat si 

$(x - 5) \times 2 = x^2 + 3x - 16$, c'est-à-dire $2x - 10 = x^2 + 3x - 16$, d'où $x^2 +x - 6 = 0$ et en factorisant on obtient bien $(x-2)(x+3)= 0$.

Les solutions de cette équation-produit nul sont $x - 2 =0$ ou $x + 3 = 0$ c'est-à-dire $x = 2$ (on retrouve la solution donnée par le tableur) ou $x = -3$.

Donc les deux programmes de calcul renvoient le même résultat si on choisit au départ $-3$ ou $2$.
	\end{enumerate}
\end{enumerate}

\bigskip

