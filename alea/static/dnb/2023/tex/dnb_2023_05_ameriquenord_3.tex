
\medskip

\textbf{Les deux parties sont indépendantes}

\medskip

\textbf{Partie A : évolution du nombre de visiteurs sur un site touristique}

\medskip

\begin{enumerate}
\item Le diagramme ci-dessous représente le nombre de visiteurs par an de 2010 à 2021 sur ce site.

\begin{center}
\psset{xunit=0.95cm,yunit=0.000018cm}
\begin{pspicture}(-1.5,-70000)(12.5,450000)
\psaxes[linewidth=0pt,Dx=50,Dy=500000,labelFontSize=\scriptstyle](0,0)(0,0)(12.5,450000)
\psframe[fillstyle=solid,fillcolor=purple](0.7,0)(1.3,300000)\psframe[fillstyle=solid,fillcolor=purple](1.7,0)(2.3,310000)
\psframe[fillstyle=solid,fillcolor=purple](2.7,0)(3.3,320000)\psframe[fillstyle=solid,fillcolor=purple](3.7,0)(4.3,320000)
\psframe[fillstyle=solid,fillcolor=purple](4.7,0)(5.3,300000)\psframe[fillstyle=solid,fillcolor=purple](5.7,0)(6.3,320000)
\psframe[fillstyle=solid,fillcolor=purple](6.7,0)(7.3,330000)\psframe[fillstyle=solid,fillcolor=purple](7.7,0)(8.3,350000)
\psframe[fillstyle=solid,fillcolor=purple](8.7,0)(9.3,360000)\psframe[fillstyle=solid,fillcolor=purple](9.7,0)(10.3,400000)
\psframe[fillstyle=solid,fillcolor=purple](10.7,0)(11.3,187216)\psframe[fillstyle=solid,fillcolor=purple](11.7,0)(12.3,219042)
\uput[d](6.25,-50000){Année}\rput{90}(-1.5,225000){Nombre de visiteurs}
\rput(6.25,475000){Nombre de visiteurs sur le site touristique par année
}
\multido{\n=1+1,\na=2010+1}{12}{\uput[d](\n,0){\small \na}}
\multido{\n=0+50000}{10}{\psline[linewidth=0.6pt](0,\n)(12.5,\n)\uput[l](0,\n){\footnotesize \np{\n}}}
\end{pspicture}
\end{center}

	\begin{enumerate}
		\item Quel a été le nombre de visiteurs en 2010 ? Aucune justification n'est attendue.
		\item En quelle année le nombre de visiteurs a-t-il été le plus élevé ? Aucune justification n'est attendue.
	\end{enumerate}
\item Le tableau ci-dessous indique le nombre de visiteurs sur le site touristique de cette ville en 2020 et en 2021 :
\begin{center}
\begin{tabularx}{0.75\linewidth}{|l|*{2}{>{\centering \arraybackslash}X|}}\hline
Année 				&2020 		&2021\\ \hline
Nombre de visiteurs &\np{187216}&\np{219042}\\ \hline
\end{tabularx}
\end{center}

Le maire de cette ville avait pour objectif que le nombre de visiteurs progresse d'au moins 15\,\% entre 2020 et 2021.

L'objectif a-t-il été atteint ?
\end{enumerate}

\medskip

\textbf{Partie B :  étude des prix des hôtels de cette ville}

\medskip

Sur une période donnée, on relève les prix facturés pour une nuit par les hôtels de cette ville.

\begin{center}
\begin{tabularx}{\linewidth}{|m{4.5cm}|*{8}{>{\centering \arraybackslash}X|}}\hline
Prix facturés pour une nuit (en euro)&60 &80 &85 &90 &110 &120 &350 &500\\ \hline
Effectif &\np{1200} &\np{1350} &\np{1000} &\np{1100} &\np{1200} &\np{1300} &900 &300\\ \hline
\end{tabularx}
\end{center}

\begin{enumerate}[resume]
\item Déterminer l'étendue des prix facturés.
\item Quelle est la moyenne des prix facturés pour une nuit ? Arrondir à l'euro près.
\item L'association des hôteliers de cette ville cherche à attirer des touristes et annonce : \og Dans les hôtels de notre ville, au moins la moitié des nuits est facturée à moins de $100$~\euro{} \fg. Est-ce vrai ?
\end{enumerate}

\bigskip

