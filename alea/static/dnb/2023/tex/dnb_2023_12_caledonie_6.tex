
\medskip

\begin{minipage}{0.6\linewidth}
Un hexagone régulier est un polygone à 6 côtés de même longueur et dont tous les angles mesurent $120\degres{}$.

 Les hexagones réguliers se retrouvent fréquemment dans la nature, notamment dans les ruches
d'abeilles.
\end{minipage}\hfill
\begin{minipage}{0.38\linewidth}
\psset{unit=1cm}
\begin{pspicture}(-2.3,-2.3)(2.3,2.3)
\pspolygon(2;0)(2;60)(2;120)(2;180)(2;240)(2;300)%CDEFAB
\uput[r](2;0){C} \uput[ur](2;60){D} \uput[ul](2;120){E} \uput[l](2;180){F} \uput[dl](2;240){A} \uput[dr](2;300){B} 
\psarc(2;0){0.35}{120}{240}\psarc(2;60){0.35}{180}{300}\psarc(2;120){0.35}{240}{0}
\psarc(2;180){0.35}{-60}{60}\psarc(2;240){0.35}{0}{120}\psarc(2;300){0.35}{60}{180}
\uput[ul](1.75;300){\small $120\degres$}
\multido{\n=30+60}{6}{\rput(1.73;\n){$\bullet$}}
\end{pspicture}
\end{minipage}

\begin{enumerate}
\item 
	\begin{enumerate}
		\item Calculer la mesure de l'angle $\widehat{\text{XBC}}$ dans la figure ci-dessous. :
		
\begin{minipage}{0.43\linewidth}
\psset{unit=1cm}
\begin{pspicture}(-1.38,-2.3)(3,1.3)
%\psgrid
\psline(1.732;210)(2;240)(2;300)(2;0)(1.732;30)%?ABC?
\uput[dl](2;240){A}\uput[d](2;300){B}\uput[r](2;0){C}\uput[d](3.464;-30){X}
\psarc(2;240){0.35}{0}{120}\psarc(2;300){0.35}{60}{180}\psarc(2;0){0.35}{120}{240}
\rput(1.35;300){$120\degres$}
\psline[linestyle=dashed](2;300)(3.464;-30)
\psarc[linecolor=red](2;300){0.45}{0}{60}\rput(1.6,-1.4){\red ?}
\end{pspicture}
\end{minipage}\hfill
\begin{minipage}{0.55\linewidth}
Les points A, B et X sont alignés.
\end{minipage}

		\item Sur le script ci-dessous, compléter les deux informations manquantes du bloc Hexagone pour qu'il trace un hexagone régulier.
		
\begin{center}
\textbf{Bloc Hexagone}

\bigskip

\begin{scratch}
%\initmoreblocks{définir \namemoreblocks{trace_carré \ovalmoreblocks{coté_carré}}}
\initmoreblocks{définir \namemoreblocks{Hexagone}}
%\blocklook{définir{Hexagone}}
\blockpen{stylo en position d'écriture}
\blockrepeat{répéter \ovalnum{....} fois}
{\blockmove{avancer de \ovalvariable{longueur}}
\blockmove{tourner \turnleft{} de \ovalnum{.....} degrés}
}
\blockpen{relever le stylo}
\end{scratch}
\end{center}

	\end{enumerate}
	
Rappel : \emph{s'orienter à $90\degres{}$ permet au lutin de se déplacer vers la droite.}

\begin{minipage}{0.47\linewidth}
\item On considère le script ci-contre qui utilise le
bloc Hexagone précédent : 
	\begin{enumerate}
		\item Combien d'hexagones réguliers ce
script trace-t-il ?
		\item Quelle est la longueur des côtés du 1\up{er} hexagone régulier tracé?
		\item Quelle est la longueur des côtés du 2\up{e} hexagone régulier tracé?
		\item Parmi les dessins ci-dessous, lequel correspond à ce script ?
	\end{enumerate}
\end{minipage}\hfill
\begin{minipage}{0.52\linewidth}
%\setscratch{scale=.75}
\begin{scratch}
\blockinit{Quand \greenflag est cliqué}
\blockmove{s'orienter à \ovaloperator{ \ovalnum{90}}}
\blockvariable{mettre \ovalvariable{longueur} à \ovalnum{32}}
\blockrepeat{répéter \ovalnum{5} fois}
{\blocklook{Hexagone}
\blockvariable{mettre \ovalvariable{longueur} à \ovaloperator{\ovalvariable{longueur} * 1.5 }}
}
\end{scratch}
\end{minipage}
\end{enumerate}

\begin{center}
\begin{tabularx}{\linewidth}{|*{3}{>{\centering \arraybackslash}X|}}\hline
\textbf{Dessin 1}&\textbf{Dessin 2}&\textbf{Dessin 3}\\ \hline
\psset{unit=1cm}
\begin{pspicture}(-2.8,-2.8)(2.,1.8)
%\psgrid
\def\hexa{\pspolygon(0.4;0)(0.4;60)(0.4;120)(0.4;180)(0.4;240)(0.4;300)}
\multido{\n=-2+0.8}{5}{\rput(\n,0){\hexa}}
\end{pspicture}&
\psset{unit=0.85cm}
\begin{pspicture}(-2.8,-2.8)(2.8,2.8)
%\psgrid
\rput(-0.3,0.5){\pspolygon(0.45;0)(0.45;60)(0.45;120)(0.45;180)(0.45;240)(0.45;300)}
\rput(-0.3,0.5){\pspolygon(0.6;0)(0.6;60)(0.6;120)(0.6;180)(0.6;240)(0.6;300)}
\rput(-0.3,0.5){\pspolygon(0.9;0)(0.9;60)(0.9;120)(0.9;180)(0.9;240)(0.9;300)}
\rput(-0.3,0.5){\pspolygon(1.6;0)(1.6;60)(1.6;120)(1.6;180)(1.6;240)(1.6;300)}
\rput(-0.3,0.5){\pspolygon(2.4;0)(2.4;60)(2.4;120)(2.4;180)(2.4;240)(2.4;300)}
\end{pspicture}&
\psset{unit=1cm}
\begin{pspicture}(-1.,-1)(3.4,4)
%\psgrid
\rput(0.4;60){\pspolygon(0.4;0)(0.4;60)(0.4;120)(0.4;180)(0.4;240)(0.4;300)}
\rput(0.6;60){\pspolygon(0.6;0)(0.6;60)(0.6;120)(0.6;180)(0.6;240)(0.6;300)}
\rput(0.9;60){\pspolygon(0.9;0)(0.9;60)(0.9;120)(0.9;180)(0.9;240)(0.9;300)}
\rput(1.35;60){\pspolygon(1.35;0)(1.35;60)(1.35;120)(1.35;180)(1.35;240)(1.35;300)}
\rput(2.025;60){\pspolygon(2.025;0)(2.025;60)(2.025;120)(2.025;180)(2.025;240)(2.025;300)}
\end{pspicture}\\ \hline
\end{tabularx}
\end{center}




