
\medskip

Un funiculaire est un type de transport en commun circulant sur des rails et dont la traction est
assurée par câble. Il est généralement utilisé pour des lignes comportant des fortes pentes.

Les documents suivants permettent de répondre aux questions.

\medskip
{\footnotesize
	\textbf{Document 1 : tarifs du funiculaire}

\begin{center}
\begin{tabularx}{\linewidth}{|m{4.5cm}|X|m{4.5cm}|X|}\hline
\multicolumn{2}{|c}{Tarif individuel}& \multicolumn{2}{|c|}{Tarif de groupe à partir de 20 personnes}\\
\multicolumn{2}{|c}{(tarif enfant accordé pour les enfants de 5 à 11 ans)}& \multicolumn{2}{|c|}{(adultes et enfants)}\\ \hline
Aller simple par adulte&8 euros		&Aller simple par adulte&7 euros\\ \hline
Aller-retour par adulte&10 euros	&Aller-retour par adulte&8,50 euros\\ \hline
Aller simple par enfant&6,50 euros	&Aller simple par enfant &5,50 euros\\ \hline
Aller-retour par enfant&8 euros		&Aller- retour par enfant&7 euros\\ \hline
\end{tabularx}
\end{center}

\textbf{Document 2 : trajet du funiculaire vu de profil}

\begin{center}
\psset{unit=1cm,arrowsize=2pt 3}
\begin{pspicture}(15,5.5)
\psline(0.5,0)(10.7,0)(10.7,4)%DEB
\psline[linewidth=1.25pt](10.7,4)(6.8,2.9)(0.5,0)%BAD
\psline[linestyle=dashed](6.8,2.9)(10.7,2.9)%AC
\psline[linewidth=0.6pt]{<->}(0.3,0.6)(6.6,3.5)\rput{24.7}(3.45,2.25){448,5 m}
\psline[linewidth=0.6pt]{<->}(11.5,2.9)(11.5,4)\uput[r](11.5,3.45){Dénivelée : 50 m}
\uput[d](8.75,2.9){\small Longueur horizontale}
\psframe(10.7,2.9)(10.4,3.2)
\uput[ul](6.8,2.9){A} \uput[u](10.7,4){B} \uput[dr](10.7,2.9){C} \uput[dl](0.5,0){D} \uput[dr](10.7,0){E} 
\end{pspicture}
\end{center}
}

\begin{enumerate}
\item Un groupe constitué de 12 adultes et de 8 enfants (âgés de 6 à 10 ans) fait un aller-retour en funiculaire.
	\begin{enumerate}
		\item Déterminons le prix à payer par le groupe en utilisant le tarif individuel 
		
		$12\times 10+8\times 8=120+64=184$ 
		le prix à payer est donc de 184 \euro.
		\item Déterminons le prix à payer par le groupe en utilisant le tarif de groupe.
		
		$12\times\np{8.50}+8\times \np{7}=102+56=158$
		
		Le prix à payer est alors de 158 \euro.
		\item Déterminons le pourcentage de la réduction obtenue en appliquant le tarif groupe par rapport au tarif individuel.
		
		$\dfrac{158-184}{184}\approx \np{0.1413}$
		
		Le pourcentage de réduction est donc de \np{14.13}\,\%.
	\end{enumerate}
\item Sur la première partie du trajet [DA], le funiculaire parcourt $448,5$~m en 8~min 45~s.

Déterminons sa vitesse moyenne en mètres par seconde ? 

Sachant que la vitesse est une distance divisée par un temps, nous avons alors 

 $v=\dfrac{\np{448.5}}{8\times 60+45}\approx \np{0.854}$

Sa vitesse, au centième près, en moyenne est d'environ \np[m/s]{0.85}.

 .
\item Sur la dernière partie du trajet [AB], la pente est de 25\,\% et la dénivelée BC est de $50$~m, calculons la longueur horizontale AC.
\end{enumerate}

{ \footnotesize \emph{Définition} : Pente $= \dfrac{\text{Dénivelée}}{\text{Longueur horizontale}}$}

Nous avons donc : $\dfrac{50}{\text{AC}}=\dfrac{25}{100}$  d'où $AC = 200$

\bigskip

