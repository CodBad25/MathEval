
\medskip

Cet exercice est un questionnaire à choix multiples QCM

%\textbf{Aucune justification n'est demandée}
%
%Pour chaque question, trois réponses sont proposées, une seule est exacte.
%
%Écrire sur votre copie, le numéro de la question et la réponse correspondante.

\medskip

\textbf{Question 1 :} soit $f$, la fonction définie par $f(x) = -2x + 3$.
Réponse B : la fonction est décroissante ($a = - 2$) et l'ordonnée à l'origine est égale à 3.
%Quelle est la représentation de la fonction $f$ ?
%
%\begin{tabularx}{\linewidth}{*{3}{>{\centering \arraybackslash}X}}
%Réponse A  &Réponse B&  Réponse C\\ 
%\psset{unit=0.5cm,arrowsize=2pt 3}
%\begin{pspicture*}(-4,-2.5)(4,5)
%\psaxes[linewidth=1.25pt,labelFontSize=\scriptstyle]{->}(0,0)(-4,-2.5)(4,5)
%\psplot[linewidth=1.25pt]{-4}{4}{1.5 x mul 3 add}
%\end{pspicture*}&
%\psset{unit=0.5cm,arrowsize=2pt 3}
%\begin{pspicture*}(-4,-2.5)(4,5)
%\psaxes[linewidth=1.25pt,labelFontSize=\scriptstyle]{->}(0,0)(-4,-2.5)(4,5)
%\psplot[linewidth=1.25pt]{-4}{4}{4 2 x mul sub}
%\end{pspicture*}&
%\psset{unit=0.5cm,arrowsize=2pt 3}
%\begin{pspicture*}(-4,-2.5)(4,5)
%\psaxes[linewidth=1.25pt,labelFontSize=\scriptstyle]{->}(0,0)(-4,-2.5)(4,5)
%\psplot[linewidth=1.25pt]{-4}{4}{2 x mul 2 sub}
%\end{pspicture*}
%\end{tabularx}

\medskip

%\begin{minipage}{0.65\linewidth}\textbf{Question 2 :}
%On considère la fonction dont la représentation graphique est donnée ci-contre.
%
%D'après le graphique, quelle est l'image de 1  par cette fonction ?
%
%\begin{tabularx}{\linewidth}{*{3}{>{\centering \arraybackslash}X|}}
%Réponse A  &Réponse B&  Réponse C\\ 
%L'image de 1 est 2&L'image de 1 est  -2&L'image de 1 est 0\\
%\end{tabularx}
%\end{minipage}\hfill
%\begin{minipage}{0.28\linewidth}
%\psset{unit=0.5cm,arrowsize=2pt 3}
%\begin{pspicture*}(-4,-2.5)(4,3)
%\psaxes[linewidth=1.25pt,labelFontSize=\scriptstyle]{->}(0,0)(-4,-2.5)(4,3)
%\psplot[linewidth=1.25pt]{-4}{2}{3 x mul x 3 exp sub}
%\end{pspicture*}
%\end{minipage}
\medskip

\textbf{Question 2 :} On lit que l'image de 1 est 2. Réponse A.

\medskip

\textbf{Question 3 :}

On donne ci-dessous un tableau de valeurs de la fonction $h$ définie par $h(x) = - x + 1$ réalisé à l'aide d'un tableur :

La réponse est C : c'est la seule qui utilise la cellule B1.
\medskip

\textbf{Question 4 :}

%Quelle est la forme développée de l'expression $(3x - 7)^2$ ?
%
%\begin{tabularx}{\linewidth}{*{3}{>{\centering \arraybackslash}X|}}
%Réponse A  &Réponse B&  Réponse C\\ 
%$3x^2 - 49$& $9x^2 - 42x + 49$&$9x^2 - 49$.
%\end{tabularx}
$(3x - 7)^2 = (3x)^2 - 2 \times 3x \times 7 + 7^2 = 9x^2 - 42x + 49$ : réponse B.

\bigskip

