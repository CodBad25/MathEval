
\medskip

Les cinq situations suivantes sont indépendantes.

\medskip

\textbf{Situation 1}

\medskip

Décomposer en produit de facteurs premiers le nombre $780$. 

Aucune justification n'est attendue.

\medskip

\textbf{Situation 2}

\medskip

On rappelle qu'un jeu de 32 cartes est composé de quatre familles (trèfle, carreau, cœur, pique).

Chaque famille est composée de huit cartes: 7, 8, 9, 10, valet, dame, roi et as.

L'expérience aléatoire consiste à tirer une carte au hasard dans ce jeu de 32 cartes.

\medskip

\textbf{a.~} Quelle est la probabilité d'obtenir le 8 de pique ? 

Aucune justification n'est attendue.

\textbf{b.~} Quelle est la probabilité d'obtenir un roi ou un cœur ? 

Aucune justification n'est attendue.

\medskip

\textbf{Situation 3}

\medskip

Développer et réduire l'expression  $A = (2x + 5)(3x - 4)$.

\medskip

\textbf{Situation 4}

\medskip

\begin{minipage}{0.48\linewidth}
\textbf{a.~} Quel est le volume, en cm$^3$, de ce prisme droit ?

\textbf{b.~} Convertir ce résultat en litre.

Rappel: 1 L = 1 dm$^3$.

\end{minipage}\hfill
\begin{minipage}{0.48\linewidth}
\psset{unit=0.9cm,arrowsize=2pt 3}
\begin{pspicture}(8.3,5.5)
\pspolygon(0.7,0.7)(5.7,0.7)(0.7,2.9)
\psline(5.7,0.7)(7.9,2.8)(2.9,5)(0.7,2.9)
\psline[linestyle=dashed](0.7,0.7)(2.9,2.8)(2.9,5)
\psline[linestyle=dashed](2.9,2.8)(7.9,2.8)
\psframe(2.9,2.8)(3.1,3)\psframe(0.7,0.7)(0.9,0.9)
\psline[linewidth=0.6pt]{<->}(0.7,0.5)(5.7,0.5)\uput[d](3.1,0.5){80 cm}
\psline[linewidth=0.6pt]{<->}(0.5,0.7)(0.5,2.9)\rput{90}(0.3,1.8){60 cm}
\psline[linewidth=0.6pt]{<->}(5.9,0.7)(8.1,2.8)\rput{46}(7,1.5){120 cm}
\psline[linewidth=0.6pt]{<->}(7.9,3)(2.9,5.2)\rput{-25}(5.4,4.4){100 cm}
\end{pspicture}
\end{minipage}

\medskip

\textbf{Situation 5}

\medskip

\begin{minipage}{0.48\linewidth}
Le polygone 2 est un agrandissement du polygone 1. 

Le coefficient de cet agrandissement est 3.

L'aire du polygone 1 est égale à 11 cm$^2$.

Quelle est l'aire du polygone 2 ?
\end{minipage}\hfill
\begin{minipage}{0.48\linewidth}
\begin{tabular}{|m{3.4cm} m{3.4cm}|}\hline
\multicolumn{2}{|m{6.8cm}|}{Représentation de la situation qui n'est pas à l'échelle:
}\\
\psset{unit=1cm}
\begin{pspicture}(1.3,1)
\pspolygon(0.4,0)(1.1,0.2)(0.9,0.7)(0.4,0.9)(0,0.4)
\end{pspicture}&\psset{unit=3cm}
\begin{pspicture}(1.3,1)
\pspolygon(0.4,0)(1.1,0.2)(0.9,0.7)(0.4,0.9)(0,0.4)
\end{pspicture}\\ 
Polygone 1&Polygone 2\\ \hline
\end{tabular}
\end{minipage}

\bigskip

