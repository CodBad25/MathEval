
\medskip

%On considère le programme de calcul suivant :
%
%\medskip
%
%\begin{itemize}
%\item[$\bullet~~$]Choisir un nombre
%\item[$\bullet~~$]Prendre le carré de ce nombre
%\item[$\bullet~~$]Multiplier le résultat par 2
%\item[$\bullet~~$]Ajouter le nombre de départ
%\item[$\bullet~~$]Soustraire 66
%\end{itemize}
%
%\medskip

\begin{enumerate}
\item 
	\begin{enumerate}
		\item %Montrer que si le nombre choisi au départ est 4, le résultat obtenu est $-30$.
On a $4^2 = 16$, puis $16 \times 2 = 32$, puis
		
$32 + 4 = 36$, puis $36 - 66 = - 30$.
		\item %Quel résultat obtient-on si le nombre choisi au départ est $-3$ ?
$(-3)^2 = 9$, puis $2 \times 9 = 18$, puis 
		
$18 + (- 3) = 15$ et $15 - 66 = - 51$.
	\end{enumerate}
\item 
	\begin{enumerate}
		\item %On s'intéresse au bloc d'instruction ci-dessous intitulé \og Programme de calcul \fg.
%On souhaite le compléter pour calculer le résultat obtenu avec le programme de calcul en fonction du nombre choisi au départ.

%On précise que deux variables ont été créées: \og nombre choisi\fg{} qui correspond au nombre choisi au départ, et \og Résultat \fg.

%\begin{center}
%\begin{scratch}
%\blockcontrol{Programme de calcul}
%\end{scratch}
%
%\begin{scratch}
%\blockmove{mettre{\selectmenu{Résultat} à \ovaloperator{\ovalmove{A} * \ovalmove{nombre choisi}}}}
%\blockmove{mettre{\selectmenu{Résultat} à \ovaloperator{\ovalmove{B} * \ovalmove{nombre choisi}}}}
%\blockmove{mettre{\selectmenu{Résultat} à \ovaloperator{Résultat + \ovalmove{nombre choisi}}}}
%\blockmove{mettre{\selectmenu{Résultat} à \ovaloperator{Résultat - \ovalmove{66}}}}
%\end{scratch}
%\end{center}
%Écrire sur votre copie le contenu qui doit être inséré dans les emplacements A et B. 

Pour A : on met nombre choisi (pour obtenir le carré).

Pour B : on met 2 (pour calculer le double).
%\textbf{Aucune justification n'est attendue pour cette question.}
		\item  %Lucie insère le bloc précédent dans le script ci-dessous et observe la réponse donnée par le lutin:

%\begin{tabular}{l l}
%\multicolumn{1}{c}{Script}&\multicolumn{1}{c}{Réponse du lutin}\\
%\begin{scratch}
%\blockinit{Quand \greenflag est cliqué}
%\blockmove{mettre{\selectmenu{nombre choisi} à \ovalmove{0}}}
%\blockrepeat{répéter \ovalnum{20} fois}
%{
%\blockcontrol{Programme de calcul}
%\blockif{si {\selectmenu{Résultat = 0}  alors}}
%{\blocklook{dire {regrouper{On peut choisir comme nombre de départ et {\ovalmove{nombre choisi}}}}}
%\blockmove{mettre{\selectmenu{nombre choisi} à \ovalmove{nombre choisi} + \ovalnum{0,5}}}
%}
%}
%\end{scratch}&On peut choisir comme nombre du départ 5,5.
%%
%\end{tabular}
%
%À quoi correspond la valeur 5,5 donnée comme réponse par le lutin avec le programme de Lucie ?
La valeur 5,5 est une valeur possible comme nombre de départ pour que le résultat final soit $0$.
	\end{enumerate}
\item On nomme $x$ le nombre choisi au départ.
	\begin{enumerate}
		\item %Déterminer l'expression obtenue par ce programme de calcul en fonction de $x$.
		On a successivement :

$x\:;\quad x^2\:;\quad  2x^2\:;\quad  2x^2 + x\:;\quad  2x^2 + x - 66$.
		\item %On admet que $(2x - 11) (x + 6)$ est la forme factorisée de l'expression trouvée à la question précédente.
		
%Pour quelle(s) valeur(s) de $x$, le résultat obtenu avec le programme est-il égal à $0$ ?
$2x^2 + x - 66 = (2x - 11) (x + 6) = 0$ : un produit est nul si l'un des facteurs est nul, soit 

$\left\{\begin{array}{l c l}
2x - 11&=&0\\
x + 6&=&0
\end{array}\right.$, ou encore $\left\{\begin{array}{l c l}
2x &=&11\\
x &=&- 6
\end{array}\right.$ et enfin $\left\{\begin{array}{l c l}
x &=&\frac{11}{2}\\
x &=&- 6
\end{array}\right.$

Les nombres qui donnent donc comme résultat final 0, sont 5,5 (déjà vu) et $- 6$.
	\end{enumerate}
\end{enumerate}

\bigskip

