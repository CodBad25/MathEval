
	\medskip

	\textbf{PARTIE A}
	
	\medskip

\begin{enumerate}
\item 
	\begin{enumerate}
			\item	Le nombre de visiteurs en 2010 est $ 300~000 $.
			\item	Le nombre de visiteurs a été le plus élevé en $ 2019 $.
		\end{enumerate}
		
	\item
		Une augmentation de $ 15\% $ revient à un nombre de visiteurs égal à :

\(\displaystyle	V_{\text{finale}}=\left(1+\dfrac{15}{100}\right)\times V_{\text{initiale}}= 1,15 \times \np{187216} = \np{215298,4} \).

Or cette valeur est inférieure à la valeur \np{219042} atteinte en $2021$.

Donc l'objectif du maire a bien été atteint.
	\end{enumerate}

\bigskip

\textbf{PARTIE B}

\medskip

\begin{enumerate}[resume]	
\item L'étendue de cette série vaut : $ Max. - Min. = 500 - 60 = 440 $. Soit $ 440 $ euros.
\item On calcule la moyenne de la série :\\
		\(\displaystyle \dfrac{60\times \np{1200} + 80 \times \np{1350} + 85\times \np{1000} + 90\times \np{1100} + 110\times \np{1200} + 120\times \np{1300} + 350 \times 900 + 500 \times 300}{\np{8350}}\)\\
$\approx 134$.

La moyenne des prix facturés pour une nuit est donc de $134$~euros.
\item Il y a $\np{8350}$ nuits au total.

Or, \(\displaystyle \dfrac{\np{8350}}{2}= \np{4175}\). On calcule les effectifs cumulés, jusqu'à la valeur $ 90 $ euros incluse (dernière valeur inférieure à $ 100 $) : $\np{1200} + \np{1350} + \np{1000} + \np{1100} = \np{4650}$.

$ 4~650 > 4~175 $ donc l'affirmation de l'association est vraie.

\textbf{OU}

On cherche la médiane de la série. Il y a $ 8~350 $	 nuits au total.

Or, \(\displaystyle \dfrac{8~350}{2}=4~175\). La médiane de la série est donc entre la \np{4175}-ième et la \np{4176}-ième valeur. Par lecture dans le tableau, on trouve que la médiane vaut $ 90 $ euros. Donc la moitié des valeurs de la série sont inférieures ou égales à $ 90 $, donc l'affirmation de l'association est vraie.
	\end{enumerate}

\bigskip

