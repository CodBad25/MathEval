
\medskip

La transat Jacques Vabre est une course de bateaux qui relie la ville du Havre, en France métropolitaine, à la ville de Fort-de-France, en Martinique.

\begin{enumerate}
\item Avec la précision permise par la carte,  la latitude de la ville de Fort-de-France repérée par une croix sur la carte ci-dessous est de $14,5\degres{}$~Nord, et sa longitude est de $61\degres{}$ Ouest.

\begin{center}
\includegraphics[width=0.9\linewidth]{mappemonde_Poly_sept_2023}
\end{center}

\item Lors de l'édition 2021, $75$ bateaux ont participé à cette course, répartis dans quatre catégories en fonction du parcours à réaliser : Class $40$, Ocean Fifty, Imoca, Ultim.

Le tableau ci-dessous présente les catégories, les effectifs engagés, les distances parcourues et le palmarès de la Transat:

\begin{center}
\begin{tabular}{|l| >{\centering\arraybackslash}m{2cm}|m{2.25cm}|m{2.5cm}|m{3cm}|}\cline{2-5}
\multicolumn{1}{c|}{~}&\textbf{Nombres de bateaux de la catégorie}&\textbf{Distance du parcours}&\textbf{Nom du bateau vainqueur de la catégorie}&\textbf{Durée de course du vainqueur}\\ \hline
\textbf{Class 40}	&43	&\np{4600} milles&Redman	&21 jours 22 heures 33 minutes\\ \hline
\textbf{Ocean Fifty}	&7	&\np{5800} milles&Primonial	&15 jours 13 heures 27 minutes\\ \hline
\textbf{Imoca}		&20	&\np{5800} milles&LinkedOut	&18 jours 1 heure 21 minutes\\ \hline
\textbf{Ultim}		&5	&\np{7500}~milles&Maxi Edmond de Rothschild&16 jours 1 heure 48 minutes\\ \hline
\end{tabular}
\end{center}

%\textbf{Information :}
%
%Un mille nautique est une unité de mesure marine qui équivaut à 1,852 km environ.

	\begin{enumerate}
		\item % Montrer que le bateau LinkedOut met 2 jours 11 heures et 54 minutes de plus que le bateau Primonial pour effectuer son parcours.
Le bateau Primonial met 15 jours 13 heures et 27 minutes pour effectuer son parcours,  et le bateau LinkedOut met 18 jours 1 heure et 21 minutes pour effectuer son parcours, soit 18 jours 0 heure et 81 minutes, ou encore 17 jours 24 heures et 81 minutes.

\begin{center}
\begin{tabular}{rcccl}
& jours & heures & minutes\\
& 17 & 24 & 81 & LinkedOut\\
-- & 15 & 13 & 27 & Primonial\\
\hline
& 2 & 11 & 54
\end{tabular}
\end{center}

Donc le bateau LinkedOut met 2 jours 11 heures et 54 minutes de plus que le bateau Primonial pour effectuer son parcours.

		\item La moyenne des distances parcourues par l'ensemble des $75$ bateaux est, en mille:
		
$\dfrac{43 \times \np{4600} \times 7 \times \np{5800} \times 20 \times \np{5800} \times 5 \times \np{7500}}{75}
= \dfrac{\np{391900}}{75} \approx \np{5225}$		
		
		\item La vitesse moyenne du bateau Redman a été d'environ 8,7~milles/h.
		
%Montrer que la vitesse moyenne du bateau Maxi Edmond de Rothschild a été environ $2,2$ fois plus grande que celle du bateau Redman.

Le bateau Maxi Edmond de Rothschild a parcouru \np{7500}~milles en 16 jours, 1 heure et 48 minutes, soit
$16\times 24 + 1 + \dfrac{48}{60}$ heures,
c'est-à-dire
$385,8$ heures.

\np{7500}~milles en $385,8$ heures, fait une moyenne de $\dfrac{\np{7500}}{385,8}$ soit environ $19,45$~milles/h.

$\dfrac{19,45}{8,7}\approx 2,2$ donc la vitesse moyenne du bateau Maxi Edmond de Rothschild a été environ $2,2$ fois plus grande que celle du bateau Redman.

		\item Un journaliste affirme que la distance parcourue par un bateau de la catégorie Ocean Fifty est environ égale à un quart de périmètre de l'équateur de la Terre.
		
La distance parcourue par un bateau de la catégorie Ocean Fifty est de \np{5800}~milles, soit en kilomètres: $\np{5800}\times 1,852\approx \np{10742}$.
		
En sachant que le rayon de l'équateur est de \np{6370}~km, le périmètre de l'équateur de la terre est, en km: $2\times \pi \times \np{6370}$ soit environ $\np{40024}$.

$\dfrac{\np{40024}}{4}=\np{10006}$ ce qui est un peu loin des \np{10742}~km.
	\end{enumerate}
\end{enumerate}




