
\medskip

\begin{enumerate}
\item 
	\begin{enumerate}
		\item La fonction $f$, dont la représentation graphique est ci-dessous  est-elle une fonction affine ? Justifier votre réponse.
		
\begin{center}
\psset{unit=0.75cm,arrowsize=2pt 3}
\begin{pspicture*}(-5,-5.1)(6,13)
\psgrid[gridlabels=0pt,subgriddiv=1,gridwidth=0.4pt]
\psaxes[linewidth=1.25pt,labelFontSize=\scriptstyle]{->}(0,0)(-5,-5)(6,13)
\psplot[plotpoints=2000,linewidth=1.25pt,linecolor=red]{-5}{5}{x 1 sub x 3 add mul}
\uput[d](5.75,0){$x$}\uput[l](0,12.75){$y$}\uput[r](2.95,11.15){\red $\mathcal{C}_f$}
\end{pspicture*}
\end{center}

		\item À l'aide de ce graphique ci-dessus, compléter, ci-dessous, le tableau de valeurs de la fonction $f$.
		\begin{center}
\begin{tabularx}{\linewidth}{|*{8}{>{\centering \arraybackslash}X|}}\hline
	&A		&B		&C		&D		&E		&F		&G\\ \hline
1	&$x$	&$-3$	&$-2$	&$-1$	&0		&1		&2\\ \hline
2	& $f(x)$& 0 	& $-3$	&\ldots	&\ldots	&\ldots	&\ldots\\ \hline
\end{tabularx}
\end{center}
	\end{enumerate}

Parmi les trois formules suivantes, l'une correspond à l'expression de la fonction $f$.

Elle a été saisie dans la cellule B2 puis étendue dans la cellule C2 du tableau ci-dessus.

\begin{center}
\begin{tabularx}{\linewidth}{|*{3}{>{\centering \arraybackslash}X|}}\hline
=B1 + 3 &=(B1 + 3)$*$(B1 $-$ 1)& =SOMME(B1 : G1) \\ \hline
\end{tabularx}
\end{center}
\begin{enumerate}[resume] 
		\item Noter la bonne formule sur votre copie.
	\end{enumerate}
\item On considère la fonction affine $g$ définie par $g(x) = 2x + 1$. 
	\begin{enumerate}
		\item Calculer l'image de $-2$ par la fonction $g$.
		\item Calculer $g(3)$.
		\item Déterminer l'antécédent de $2$ par la fonction $g$.
		\item Tracer, sur le graphique précédent, la représentation graphique de la fonction~$g$.
	\end{enumerate}
\item L'expression de la fonction $f$ ci-dessus est $f(x) = (x + 3)(x - 1)$.
	\begin{enumerate}
		\item Développer et réduire l'expression $(x + 3)(x - 1)$.
		\item Pour quelle(s) valeur(s) de $x$, a-t-on $f(x) = g(x)$ ?
	\end{enumerate}
\end{enumerate}

\bigskip

