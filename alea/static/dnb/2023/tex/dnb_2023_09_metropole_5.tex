
\medskip

\emph{La figure ci-dessous n'est pas à l'échelle}

\begin{center}
\psset{unit=1cm,arrowsize=2pt 3}
\begin{pspicture}(-1.5,-1)(12,7.4)
%\psgrid
\pspolygon[linewidth=1.25pt](0,0)(9.3,0)(2.2,5)%BCA
\psline[linewidth=1.25pt](2.2,5)(2.2,0)%AH
\psline[linewidth=1.25pt](1.03,2.4)(5.9,2.4)%MN
\psframe(2.2,0)(2.5,0.3)
\psline[linewidth=0.6pt]{<->}(-1,0.45)(0,2.85)\rput{67.4}(-0.65,1.85){2,5 cm}
\psline[linewidth=0.6pt]{<->}(0,2.85)(1.2,5.45)\rput{67.4}(0.35,4.35){2,7 cm}
\psline[linewidth=0.6pt]{<->}(3,6)(10.1,1)\rput{-35.15}(6.55,3.75){8,5 cm}
\psline[linewidth=0.6pt]{<->}(0,-0.8)(2.2,-0.8)\uput[d](1.1,-0.8){2 cm}
\uput[u](2.2,5){A} \uput[l](1.03,2.4){M} \uput[ur](5.9,2.4){N}
\uput[dl](0,0){B} \uput[d](2.2,0){H} \uput[dr](9.3,0){C}
\end{pspicture}
\end{center}

Dans le triangle ABC ci-dessus, M est un point du côté [AB], N est un point du côté [AC], et H est un point du côté [BC] ; les droites (MN) et (BC) sont parallèles.

On donne :
\begin{itemize}
\item[$\bullet$~~] AC $= 8,5$~cm ;
\item[$\bullet$~~] AM $= 2,7$~cm;
\item[$\bullet$~~] MB $= 2,5$~cm;
\item[$\bullet$~~] BH $= 2$~cm.
\end{itemize}

On rappelle que toutes les réponses doivent être justifiées.

\medskip

\begin{enumerate}
\item Calculer AB.
\item Montrer que la longueur AH est égale à $4,8$~cm.
\item Calculer la mesure de l'angle $\widehat{\text{ACH}}$. Arrondir au degré près.
\item Calculer la longueur HC. Arrondir au cm près.
\item Un élève affirme que: \og AN est inférieure à $4$~cm.\fg. A-t-il raison ? 
\item Calculer l'aire du triangle AHC.
\end{enumerate}
