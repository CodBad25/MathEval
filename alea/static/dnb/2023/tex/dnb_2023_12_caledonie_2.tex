
\medskip

José, un agriculteur vivant dans la commune du Mont-Dore, veut préparer des paniers de légumes bio pour ses clients.

Il a déjà récolté 39 salades, 78 carottes et 51 aubergines.

Il veut que tous les paniers aient la même composition et utiliser tous les légumes.

La décomposition de 39 en produit de facteurs premiers est : $3 \times 13$.

\medskip

\begin{enumerate}
\item 
	\begin{enumerate}
		\item Décomposer en facteurs premiers les nombres 78 et 51.
		\item En déduire le nombre de paniers maximum que José peut préparer.
		\item Combien de salades, de carottes et d'aubergines y aurait-il dans chaque panier?
	\end{enumerate}
\end{enumerate}

Finalement, José décide de préparer 13 paniers.

\begin{enumerate}[resume]
\item 
	\begin{enumerate}
		\item Combien d'aubergines ne seront pas utilisées? Justifier votre réponse.
		\item Combien doit-il cueillir au minimum d'aubergines supplémentaires pour pouvoir toutes les utiliser ?
			\end{enumerate}
\end{enumerate}

José souhaite que ses 13 paniers contiennent également des tomates.

Il estime qu'il en a entre $110$ et $125$~prêtes à être récoltées.

\begin{enumerate}[resume]
\item Combien doit-il en cueillir au maximum pour éviter les pertes et pour que chaque panier ait toujours la même composition ?

\textbf{Toute trace de recherche, même non aboutie, sera prise en compte.}
\end{enumerate}

\bigskip

