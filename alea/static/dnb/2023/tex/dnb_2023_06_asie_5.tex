
\medskip

Un marchand de glaces souhaite préparer ses ventes pour l'été prochain.

Voici quelques informations concernant son activité en juillet et août 2022.


\begin{center}
\renewcommand{\arraystretch}{1.2}
\begin{tabular}{|p{6cm}|p{1cm}|p{6cm}|}
\cline{1-1}\cline{3-3} 
\centering\textbf{Prix de vente des pots de glace} & & \quad\textbf{Dimension de la cuillère à glace}\rule[-10pt]{0pt}{30pt}\\
\hspace*{1cm}1 boule: $2,80$ \euro & 
& \hfill\includegraphics[scale=0.6]{glace}\\
\hspace*{1cm}2 boules: $3,50$ \euro & &\quad Diamètre: $4,2$ cm \rule[-10pt]{0pt}{30pt}\\
\cline{1-1}\cline{3-3} 
\end{tabular}
\end{center}

\begin{center}
{\renewcommand{\arraystretch}{1.2}
\begin{tabularx}{0.5\linewidth}{|*{3}{>{\centering \arraybackslash}X|}}
\multicolumn{3}{c}{\textbf{Nombre de pots de glace vendus\rule[-10pt]{0pt}{0pt}}}\\
\cline{2-3}
\multicolumn{1}{c}{}&\cellcolor{lightgray}Juillet 2022 &\cellcolor{lightgray}Août 2022\\ 
\hline
\cellcolor{lightgray}Semaine 1 & 453 & 860 \\ \hline
\cellcolor{lightgray}Semaine 2 & 649 &\np{1003}\\ \hline
\cellcolor{lightgray}Semaine 3 & 786 & 957 \\ \hline
\cellcolor{lightgray}Semaine 4 & 854 & 838 \\ \hline
\end{tabularx}}
\hfill
\begin{tabular}{|l|}
\hline
\hfill{~}\textbf{Rappels}\hfill{~}\rule[-5pt]{0pt}{20pt}\\
\textbullet~~Le volume d'une boule de rayon $r$\\
\hphantom\textbullet~~est donné par la formule:\\
\hfill{~}$V=\dfrac{4}{3}\times \pi \times r^3$\hfill{~}\rule[-15pt]{0pt}{35pt}\\
\textbullet~~1 dm$^3$ = 1 L\\
\hline
\end{tabular}
\end{center}


\begin{enumerate}
\item Calculer le nombre moyen de pots de glace vendus par semaine au cours de la période de juillet à août 2022.
\item Parmi tous les pots de glace vendus au cours de cette période, 67\,\% sont des pots à une boule. Calculer la somme que rapporte la vente des pots de glace au cours de cette période.
\item On modélise les boules de glace réalisées avec la cuillère à glace par des boules de $4,2$~cm de diamètre.
	\begin{enumerate}
		\item Montrer que le volume d'une boule de glace est d'environ 39~cm$^3$.
		\item Le vendeur utilise des bacs de glace contenant 10~L chacun.

Combien peut-il faire de boules de glace au maximum, avec la glace contenue dans un bac ?
	\end{enumerate}
\end{enumerate}

