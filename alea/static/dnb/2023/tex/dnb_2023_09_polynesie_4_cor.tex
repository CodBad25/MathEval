
\medskip

Les propriétaires d'une maison souhaitent créer une rampe d'accès à leur terrasse.

Cette rampe devra avoir la forme d'un prisme droit à base triangulaire comme représenté sur le schéma en perspective cavalière ci-dessous:

\begin{center}
\psset{unit=1cm}
\begin{pspicture}(0,-1)(15,5.3)
\pscurve(0,0)(0.33,0.33)(0,0.5)(0.5,0.66)(0,1)(1,1.6)(1.9,2)(2.5,3.2)(3.2,3.7)(3.4,4.2)(4.2,4.1)(4.7,4.3)
\psline(4.7,4.3)(8.3,4.3)(12.1,3.3)(14.8,3.3)
\psline(12.1,3.3)(7.5,0)(3.6,1)(0,1)
\psline(0,0)(10,0)
\psline(8.3,4.3)(3.6,1)(3.6,0)%DAC
\psline[linestyle=dashed](3.6,0)(8.3,3.3)(8.3,4.3)%CFD
\psline[linestyle=dashed](8.3,3.3)(12.1,3.3)%FE
\pscurve(10,0)(10.3,0.8)(10.5,1)(11.3,0.8)(12,1.1)(12.2,2.2)(13,2.5)(14,2.8)(14.3,2.7)(14.8,3.3)
\rput(4.2,2.9){TERRASSE}\rput(8.4,1.9){RAMPE D'ACCÈS}\rput(11.2,1.6){SOL}
\uput[ul](3.6,1){A}\uput[d](7.5,0){B}\uput[d](3.6,0){C}
\uput[u](8.3,4.3){D}\uput[ur](12.1,3.3){E}\uput[dr](8.3,3.3){F}
\end{pspicture}
\end{center}

%Les figures ci-dessus ne sont pas à l'échelle.

\begin{multicols}{2}
Vue de face de la rampe :

\psset{unit=0.8cm}
\begin{pspicture}(-1,-0.8)(7,3)
\pspolygon(0,0)(6,0)(0,2)%CBA
\psframe(0,0)(0.2,0.2)
\uput[dl](0,0){C}\uput[d](6,0){B}\uput[u](0,2){A}
\end{pspicture}

\columnbreak

\begin{list}{\textbullet}{On donne les informations suivantes :}
\item la hauteur [AC] de la rampe mesure $30$~cm ;
\item AB $= 124$~cm ;
\item la longueur BE de la rampe mesure $9$~m ;
\item l'angle $\widehat{\text{ACB}}$ est un angle droit.
\end{list}
\end{multicols}

%\medskip

\begin{enumerate}
\item %Déterminer la mesure de l'angle $\widehat{\text{ABC}}$ que doit faire la rampe avec le sol du jardin.
Dans le triangle ACB rectangle en C, on a:
$\sin \left (\widehat{\text{ABC}}\right )= \dfrac{\text{AC}}{\text{AB}} = \dfrac{30}{124}$.

On en déduit que l'angle $\widehat{\text{ABC}}$ mesure, au degré près, $14\degres{}$.

%On arrondira au degré près.

\item %Montrer que la longueur BC doit être environ égale à $120$~cm.
Le triangle ACB est rectangle en C donc

$\text{AB}^2 = \text{AC}^2 + \text{BC}^2$
donc
$\text{AB}^2 - \text{AC}^2 = \text{BC}^2$
ou encore $124^2-30^2 = \text{BC}^2$,
et donc $\text{BC}^2 = \np{14476}$.

On en déduit que BC vaut, en centimètre, environ 120.

\item Pour réaliser cette rampe, les propriétaires envisagent de se faire livrer $2$~m$^3$ de béton.

%Ce volume est-il suffisant ?

La longueur BE de la rampe mesure $9$~m soit 900~cm.

La rampe est un prisme de base le triangle ACB et de hauteur BE donc son volume vaut, en cm$^3$:
$\left (\text{aire de ABC}\right ) \times \text{BE}$ soit
$\dfrac{\text{AC} \times \text{BC}}{2}\times \text{BE}$
soit environ
$\dfrac{30\times 120}{2}\times 900$
c'est-à-dire $\np{1620000}$.

Le volume de la rampe est donc, en m$^3$, d'environ $1,62$.

Donc le volume de 2~m$^3$ de béton est suffisant.

\item %En utilisant le volume de $2$~m$^3$ de béton, sans modifier les longueurs AC et BE de la rampe, quelle serait la valeur de BC ?
On cherche BC pour utiliser les 2~m$^3$ de béton soit $\np{2000000}$~cm$^3$.

Donc BC est tel que:
$\dfrac{\text{AC} \times \text{BC}}{2}\times \text{BE} = \np{2000000}$
donc
$\text{BC} = \dfrac{\np{2000000}\times 2}{\text{AC}\times \text{BE}}
= \dfrac{\np{4000000}}{30\times 900}$
soit 148~cm en arrondissant au centimètre.

%On arrondira au centimètre près.
\end{enumerate}


