
\medskip

Des élèves organisent, pour leur classe, un jeu au cours duquel il est possible de gagner des lots. Pour cela, ils placent dans une urne trois boules noires numérotées de 1 à 3, et quatre boules rouges numérotées de 1 à 4, toutes indiscernables au toucher.

\bigskip

\textbf{Partie A : étude du jeu}

\medskip

\begin{enumerate}
\item On pioche au hasard une boule dans l'urne.
	\begin{enumerate}
		\item Quelle est la probabilité de tirer une boule rouge ?
		\item Quelle est la probabilité de tirer une boule dont le numéro est un nombre pair ?
	\end{enumerate}	
\item Le jeu consiste à piocher, dans l'urne, une première boule, la remettre dans l'urne puis en piocher une seconde.

Pour chacune des boules tirées, on note la couleur ainsi que le numéro.

Pour gagner un lot, il faut tirer la boule rouge numérotée 1 et une boule noire.

Quelle est la probabilité de gagner ?
\end{enumerate}

\bigskip

\textbf{Partie B : constitution des lots}

\medskip

Pour constituer les lots, on dispose de 195 figurines et 234 autocollants.

Chaque lot sera composé de figurines ainsi que d'autocollants.

Tous les lots sont identiques.

Toutes les figurines et tous les autocollants doivent être utilisés.

\medskip

\begin{enumerate}
\item Peut-on faire 3 lots ?
\item Décomposer 195 en produit de facteurs premiers.
\item Sachant que la décomposition en produit de facteurs premiers de $234$ est $2 \times 3^2 \times 13$ :
	\begin{enumerate}
		\item Combien de lots peut-on constituer au maximum ?
		\item De combien de figurines et d'autocollants sera alors composé chaque lot ?
	\end{enumerate}
\end{enumerate}

\bigskip

