
\medskip

%Dans cette exercice, on étudie la probabilité de gain des deux jeux ci-dessous.
%
%\bigskip
%
\textbf{Partie A }
%
%\medskip
%
%\begin{tabularx}{\linewidth}{X|X}
%Jeu 1&Jeu 2\\
%Un sac contient cinq boules indiscernables au toucher, dont une portant la lettre N, deux, portant la lettre G et deux portant la lettre P.& Une roue à six  secteurs angulaires identiques numérotées de un à six.
%\end{tabularx}
%
\medskip

\begin{enumerate}
\item %On considère le jeu 1.

%On pioche une boule au hasard dans ce sac et on note la lettre inscrite sur la boule choisie.

%On considère qu'on a gagné si on pioche la lettre G.

%Montrer que la probabilité de gagner avec ce jeu et de $\dfrac25$.
Il y a deux boules avec la lettre G sur 5 boules, d'où $P(G) = \dfrac25 = \dfrac{4}{10} = 0,4$.
\item %On considère le jeu 2.

%On fait tourner la roue et on note le nombre d'inscrits sur le secteur pointé par la flèche.

%On considère qu'on a gagné si on s'arrête sur un nombre premier.

%Quelle est la probabilité de gagner à ce jeu ?
Les nombres premiers sont : 2, \: 3 et 5 : il y a 3 cas favorables sur 6, donc la probabilité de gagner est égale à $\dfrac36 = \dfrac12 = 0,5$.
\item 
	\begin{enumerate}
		\item %Quel est le jeu qui présente la plus faible probabilité de gagner ?
On a $0,4 < 0,5$ :c'est le jeu 1 qui a la plus faible probabilité de gagner.
		\item %Proposer une liste de boules à rajouter pour que la probabilité de gagner avec le jeu 1 soit de $\dfrac14$.
On a $\dfrac14 = \dfrac28$ : le numérateur représente le nombre de boules G (on les a déja) et le dénominateur le nombre total de boules (8). Comme on a déjà 5 boules il faut donc en rajouter 3 qui ne soient pas marquées G, par exemple 3 P ou 2P et 1 N.
	\end{enumerate}
\end{enumerate}

\bigskip

\textbf{Partie B}

\medskip

%\textbf{Dans cette partie, toute trace de recherche sera valorisé}
%
%\medskip
%
%On choisit finalement de combiner ces deux jeux.
%
%Dans un premier temps, le joueur doit tirer une boule dans le sac du jeu 1.
%
%On doit ensuite faire tourner la roue du jeu 2.
%
%Le joueur gagne un lot s'il a tiré une boule portant la lettre G et si la roue s'arrête sur un secteur angulaire dont le numéro est un nombre premier.
%
%Quelle est la probabilité de gagner à cette combinaison des deux jeux ?
\textbf{Méthode 1} : principe multiplicatif :

$P(\text{gagner}) = P(\text{G}) \times P(\text{nombre premier}) = \dfrac25 \times \dfrac12 = \dfrac15 = \dfrac{2}{10} = 0,2$.

\textbf{Méthode 2} On peut faire un tableau à double entrée de 5 colonnes (tirage de l'une des boules) et 6 lignes (arrêt sur l'un des six secteurs).

Les cas favorables sont G1-2, G1-3, G1-5 et G2-2, G2-3 et G2-5 soit 6 cas favorables sur 30, d'où une probabilité de $\dfrac{6}{30} = \dfrac{1\times 6}{6 \times 5} = \dfrac15 = 0,2$.

\bigskip

