
\medskip

\begin{enumerate}
\item $ AN^{2} = 13^{2} = 169$ .

$LN^{2} + AL^{2} = 5^{2} + 12^{2} = 25 + 144 = 169$

donc $AN^{2} = LN^{2} + AL^{2}$.

D'après la réciproque du théorème de Pythagore, le triangle $LNA$ est bien rectangle en $L$.
\item D'après la question précédente, $(AL) \perp (LN)$.

D'après le codage de l'énoncé, $(HO) \perp (LN)$.

Donc les droites $(AL)$ et $(HO)$ perpendiculaires à une même droite,
sont parallèles. D'autre part

Les points $N,H,A $ et $N, O, L $ sont alignés.

Les droites $(AL)$ et $(HO)$ sont parallèles.

D'après le théorème de Thalès

\(\displaystyle	\dfrac{NO}{NL}=\dfrac{NH}{NA}=\dfrac{OH}{AL}\) ~~ soit ~~ \(\displaystyle	\dfrac{3}{5}= \dfrac{NH}{13}=\dfrac{OH}{12} = \dfrac{3}{5} = \dfrac{6}{10}\),  d'où ~~\(\displaystyle OH = \dfrac{12 \times 6}{10} = \dfrac{72}{10} = 7,2~(\text{cm}) \).
\item Dans le triangle $ LNA $ rectangle en $ L $, 
\(\displaystyle \cos({\widehat{LNA}})=\dfrac{\text{côté adjacent}}{\text{hypoténuse}}=\dfrac{LN}{AN}= \dfrac{5}{13}\).

La calculatrice donne avec la fonction inverse de la fonction cosinus : $\widehat{LNA} \approx 67 \degres{}$.

\item L'angle $\widehat{LNA}$ est un angle commun aux deux triangles.
	
$\widehat{HON}=\widehat{ALN}=90 ~ \text{degrés}$.

Donc les triangles $ LNA $ et $ OHN $ ont deux paires d'angles de même mesures, donc ils sont semblables.
\item
	\begin{enumerate}
		
		\item On calcule les différentes aires :

\(\displaystyle A_{LNA}=\dfrac{5\times12}{2}=30~(\text{cm}^{2})\).

\(\displaystyle A_{OHN}=\dfrac{3\times7,2}{2}=10,8~~(\text{cm}^{2})\).

\(\displaystyle A_{LOHA}=A_{LNA} - A_{OHN}=19,2~~(\text{cm}^{2})\).

\item \(\displaystyle \dfrac{A_{LOHA}}{A_{LAN}}=\dfrac{19,2}{30}=0,64=\dfrac{64}{100}\).

La proportion est donc \(\displaystyle \dfrac{64}{100}\).
\end{enumerate}
	\end{enumerate}


