
\medskip

Sur un blog de couture, Archibald a trouvé une fiche technique pour tracer un pentagramme (étoile à cinq branches).

\begin{center}
\textbf{FICHE TECHNIQUE POUR TRACER UNE ÉTOILE A CINQ BRANCHES}
\end{center}
\textbf{1.~~}Tracer un cercle de centre O, puis tracer deux diamètres
perpendiculaires [AB] et [CD].

\textbf{2.~~} Placer le milieu du segment [OC]. Le nommer J.

\textbf{3.~~} Placer la pointe du compas sur J, placer le crayon sur C et tourner.

\textbf{4.~~} Représenter la demi-droite [JA]. Elle coupe ce cercle en M.

\textbf{5.~~} Placer la pointe du compas sur A, placer le crayon sur M et tourner.

\textbf{6.~~} Le cercle obtenu coupe le cercle de centre O et de rayon [OC] en E et F{}.

\textbf{7.~~} À partir du point F{}, reporter trois fois la longueur EF sur le cercle pour obtenir dans cet ordre les points G, H et I.

\textbf{8.~~} Tracer les segments [EG], [GI], [IF], [FH] et [HE].
%&
\begin{center}
\psset{unit=0.5cm}
\begin{pspicture}(-3,-3)(3,5)
\pscircle(0,0){3}
\pspolygon[fillstyle=solid,fillcolor=lightgray](3;45)(3;189)(3;333)(3;477)(3;621)
\uput[ur](3;45){F} \uput[ul](3;189){I} \uput[dr](3;333){G} 
\uput[ul](3;477){E} \uput[dl](3;621){H} 
\end{pspicture}
\end{center}

\medskip

\begin{enumerate}
\item Compléter et terminer la construction de l'étoile à cinq branches débutée par Archibald ci-dessous. On fera apparaître les points B,  D, J, M,  E, F,  G, H  et I.

\medskip


\begin{pspicture}(-3,-3)(3,5)
%\psgrid
\pscircle(0,0){3}
\psdots(0,0)(-3,0)(3,0)(0,-3)(0,3)(-1.5,0)(-0.8,1.36)(1.74,2.44)
\pscircle[linestyle=dashed](-1.5,0){1.5}
\psline(-3,0)(3,0)\psline(0,-3)(0,3)\uput[ul](0,3){A}
\psline[linestyle=dashed](-1.5,0)(1,5)
\pscircle[linestyle=dashed](0,3){1.82}
\uput[l](-3,0){C}\uput[ur](0,0){O}
%\pspolygon[fillstyle=solid,fillcolor=lightgray](3;45)(3;189)(3;333)(3;477)(3;621)
\end{pspicture}

\medskip


\item Réécrire la troisième consigne sur la copie en utilisant le vocabulaire mathématique adapté.
\item En utilisant cette fiche technique,  Anaïs a obtenu la construction ci-dessous.
\begin{center}
\psset{unit=1cm}
\begin{pspicture}(-3,-3)(3,3)
\pscircle(0,0){3}
\pspolygon[fillstyle=solid,fillcolor=lightgray](3;45)(3;189)(3;333)(3;477)(3;621)
\end{pspicture}
\end{center}

Elle mesure les angles $\widehat{\text{EGI}}$ et $\widehat{\text{EHI}}$ et constate qu'ils sont égaux. Est-ce le cas pour tous les pentagrammes construits avec cette méthode ?
\end{enumerate}

