
\medskip

\begin{enumerate}
\item %Anne et Jean ont acheté $630$ dragées roses et $810$~dragées blanches qu'ils ont mises dans un sachet. On suppose que les dragées sont indiscernables au toucher.
	\begin{enumerate}
		\item %Combien Anne et Jean ont-ils acheté de dragées au total ?
Anne et Jean ont  acheté à eux deux $630 + 810 = \np{1440}$~dragées.
		\item %Anne prend au hasard une dragée dans le sachet. Quelle est la probabilité qu'elle obtienne une dragée blanche ?
		Il y a 810 dragées blanches parmi les \np{1440} dragées ; la probabilité est donc égale à : $\dfrac{810}{\np{1440}} = \dfrac{81}{144} = \dfrac{9 \times 9}{9 \times 16} = \dfrac{9}{16} = \np{0,5625}$.
	\end{enumerate}
\item %Avec ces dragées, ils réalisent des ballotins pour leur mariage de sorte que: 
%\begin{itemize}
%\item le nombre de dragées roses est le même dans chaque ballotin ;
%\item le nombre de dragées blanches est le même dans chaque ballotin ;
%\item toutes les dragées soient utilisées.
%\end{itemize}

	\begin{enumerate}
		\item %Peuvent-ils réaliser $21$ ballotins?
		On a $\dfrac{630}{21} = \dfrac{9 \times 7 \times 10}{3 \times 7} 
 = 3 \times 10 = 30$ et $\dfrac{810}{21} = \dfrac{3 \times 270}{3 \times 7} = \dfrac{270}{7}$ qui n'est pas un entier  : ils ne peuvent réaliser 21 ballotins identiques
 		\item %Décomposer $630$ et $810$ en produits de facteurs premiers.
	$630 = 9 \times 7 \times 10 = 9 \times 7 \times 2 \times 5 = 2 \times 3^2 \times 5 \times7$ et 
	
	$810 = 81 \times 10 = 9 \times 9 \times 2 \times 5 = 2\times 3^4 \times 5$
		\item %En déduire le nombre maximum de ballotins qu'Anne et Jean pourront réaliser.
%Donner alors la composition de chaque ballotin.

Les facteurs communs à 630 et 810 les plus nombreux sont : un facteur 2, deux facteurs 3 et un facteur 5  : autrement dit le plus grand diviseur de 630 et de 810 est le produit $2 \times 3^2 \times 5 = 9 \times 10 = 90$.

On a $630 = 90 \times 7$ et $810 = 90 \times 9$.

Conclusion : Anne et Jean pourront faire 90 ballotins identiques de 7 dragées roses et 9 dragées blanches.
	\end{enumerate}
\end{enumerate}

\bigskip

