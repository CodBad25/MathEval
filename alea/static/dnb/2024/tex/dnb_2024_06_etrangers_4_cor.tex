
\medskip


\begin{enumerate}
	\item Dans le triangle ABE, le côté le plus long est [AB].

On a, d'une part : \quad $\mathrm{AB}^2 = 5,5^2 = 30,25$.

D'autre part : \quad $\mathrm{AE}^2 + \mathrm{EB}^2 = 4,4^2 + 3,3^2 = 19,36 + 10,89 = 30,25$.

On constate que :\quad $\mathrm{AB}^2 = \mathrm{AE}^2 + \mathrm{EB}^2$.

D'après la réciproque du théorème de Pythagore, on en déduit que le triangle AEB est rectangle en E, [AB] étant l'hypoténuse.

	\item Dans le triangle AEB, rectangle en E, on a :\quad $\cos\left(\widehat{\mathrm{ABE}}\right) = \dfrac{\mathrm{EB}}{\mathrm{AB}} = \dfrac{3,3}{5,5} = \dfrac{33}{55}  = \dfrac{11 \times 3}{11 \times 5} = \dfrac35 = \dfrac{6}{10} = 0,6$.

On en déduit :\quad $\widehat{\mathrm{ABE}} \approx53,1$\degres{}\quad (obtenu à la calculatrice).

La mesure de l'angle $\widehat{\mathrm{ABE}}$, arrondie au degré est donc de 53\,\degre.

	\item Puisque les points E, A et F sont alignés, dans cet ordre, que les points E, B et D sont alignés dans le même ordre et que les droites (AB) et (FD) sont parallèles, d'après le théorème de Thalès, on sait que : \quad $\dfrac{\mathrm{EB}}{\mathrm{ED}} = \dfrac{\mathrm{EA}}{\mathrm{EF}} = \dfrac{\mathrm{AB}}{\mathrm{FD}}$.


	 En particulier : \quad $\dfrac{\mathrm{EB}}{\mathrm{ED}} = \dfrac{\mathrm{AB}}{\mathrm{FD}}$.

	 En remplaçant les longueurs connues : \quad $\dfrac{3,3}{3,3 + 6,6} = \dfrac{5,5}{\mathrm{FD}}$

	 Avec un produit en croix :\quad $\mathrm{FD} = \dfrac{5,5 \times (3,3+6,6)}{3,3} = \dfrac{5,5\times 9,9}{3,3} = 5,5 \times 3 = 16,5$.

	 La longueur FD est donc de \np[cm]{16,5}.

	\item Une homothétie de centre E transformant le triangle EAB en le triangle EFD transforme notamment le segment [EB] en [ED], comme les points B et D sont sur la même demi-droite d'extrémité E, le rapport de l'homothétie est positif, et il vaut :

	$\dfrac{\mathrm{ED}}{\mathrm{EB}}=\dfrac{9,9}{3,3} = 3$.

	Le rapport de cette homothétie est donc 3.
	\end{enumerate}


