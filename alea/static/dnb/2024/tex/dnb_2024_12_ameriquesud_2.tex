
\bigskip

On considère deux fonctions $f$ et $g$ définies par :

\[f(x) = x^2 - x - 6 \qquad \qquad g(x) = -2x.\]

\begin{enumerate}
\item 
	\begin{enumerate}
		\item Montrer que l'image de 5 par la fonction $f$ est 14.
		\item Déterminer l'antécédent de 4 par la fonction $g$.
		\end{enumerate}		
Pour calculer des images de nombres par les fonctions $f$et $g$, on utilise un tableur et on obtient la copie d'écran suivante:

\begin{center}
\begin{tabularx}{0.9\linewidth}{|>{\cellcolor{lightgray}}c|c|*{7}{>{\centering \arraybackslash}X|}}\hline
\rowcolor{lightgray}	&\text{A}&B&C&D&E&F&G&H\\ \hline
1	&$x$&$-4$&$-3$&$-2$&$-1$&0&1&2\\ \hline
2	&$f(x) = x^2 - x - 6$&14&6&0&$-4$&$-6$&$-6$&$-4$\\ \hline
3	&$g(x) = -2x$&8&6&4&2&0&$-2$&$-4$\\ \hline
\end{tabularx}
\end{center}

	\begin{enumerate}[resume]
		\item À l'aide des informations précédentes, citer deux antécédents de 14 par la fonction $f$.
		
		\item  Quelle formule a-t-on pu saisir dans la cellule B2 avant de l'étirer vers la droite jusqu'à la cellule H2 ?
		\item Existe-t-il un nombre qui a la même image par la fonction $f$ et par la fonction $g$ ?
		\end{enumerate}
\item 
	\begin{enumerate}
		\item Montrer que, pour tout nombre $x,\, f(x)$ est égal à $(x + 2)(x - 3)$. 
		\item Résoudre l'équation $f(x) = 0$.
	\end{enumerate}
\end{enumerate}

\bigskip

