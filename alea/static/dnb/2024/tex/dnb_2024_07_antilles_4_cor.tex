
\medskip

On considère le programme de calcul ci-dessous :

\begin{center}
\begin{tabular}{|l|}\hline
$\bullet~$Choisir un nombre\\
$\bullet~$Mettre ce nombre au carré\\
$\bullet~$Soustraire le triple du nombre de départ\\
$\bullet~$Soustraire 4\\ \hline
\end{tabular}
\end{center}

\smallskip

\begin{enumerate}
\item %Montrer que si on choisit 5 comme nombre de départ, le résultat du programme est 6.
On a successivement : $5 \to 5^2 = 25 \to 25 - 3 \times 5 = 10  \to, 10 - 4 = 6$.
\item %On choisit $x$ comme nombre de départ.
De même avec $x$ au départ : 

$x \to x^2 \to x^2 - 3x \to x^2 - 3x - 4$.
%Exprimer le résultat du programme en fonction de $x$.
\item %Vérifier que l'on peut écrire ce résultat sous la forme $(x+ 1)(x - 4)$
On développe $(x+ 1)(x - 4) = x^2 - 4x + x  - 4 = x^2 - 3x - 4 $. On retrouve l'expression de la question 2.

On a donc $x^2 - 3x - 4 = (x + 1)(x - 4)$.
\item %Déterminer les nombres à choisir au départ pour que le résultat du programme soit 0.
Il faut trouver un ou des nombres $x$ tels que $x^2 - 3x - 4 = 0$ ou d'après la question précédente tels que :

$(x + 1)(x - 4) = 0$.

Un produit de facteurs est nul si l'un des facteurs est nul , soit 

$\left\{\begin{array}{l c l}
x +1&=&0\\
&\text{ou}&\\
x - 4&=&0
\end{array}\right.$ d'où $\left\{\begin{array}{l c l}
x &=&- 1\\
&\text{ou}&\\
x &=& 4
\end{array}\right.$.

Il y a donc deux nombres qui donnent finalement 0 : ce sont $- 1$ et 4.
\item %Juliette a écrit le programme ci-dessous :

%\begin{scratch}[num blocks]
%\blockinit{quand \greenflag est cliqué}
%\blockmove{demander \ovalnum{Choisir un nombre} et attendre}
%\blockvariable{mettre \selectmenu{x} à \ovalmove{réponse}}
%\blockvariable{mettre \selectmenu{y} à \ovaloperator{\ovalnum{\ldots}*\ovalnum{\ldots}}}
%\blockvariable{mettre \selectmenu{z} à \ovaloperator{\ovalnum{3}*\ovalnum{x}}}
%\blockvariable{mettre \selectmenu{Résultat} à \ovaloperator{\ovalnum{\ldots}-\ovalnum{\ldots} - 4}}
%\blocklook{dire \ovalnum{Résultat } pendant \ovalnum{5}secondes}
%\end{scratch}
Juliette doit compléter en ligne 4 et 6 :

\begin{scratch}
\blockvariable{mettre \selectmenu{y} à \ovaloperator{\ovalvariable{x}*\ovalvariable{x}}}
\blockvariable{mettre \selectmenu{Résultat} à \ovaloperator{\ovalvariable{y}-\ovalvariable{z} - 4}}
\end{scratch}
\end{enumerate}

\medskip

%Recopier et compléter sur la copie les lignes 4 et 6 du programme afin que celui-ci corresponde au programme de calcul encadré.

\bigskip

