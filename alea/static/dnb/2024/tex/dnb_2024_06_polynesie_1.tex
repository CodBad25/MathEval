
\medskip

Cet exercice est un questionnaire à choix multiples (QCM).

Pour chaque question, quatre affirmations sont proposées. \textbf{Une seule affirmation est exacte.}

\textbf{Sur la copie}, écrire le numéro de la question et l'affirmation choisie. Aucune justification n'est attendue.

\medskip

\begin{enumerate}
\item 

ABC est un triangle tel que AB = 20 cm, BC = 21 cm et AC = 29 cm. On peut affirmer que:

\begin{center}
\begin{tabularx}{\linewidth}{|*{4}{>{\small}X|}}\hline
ABC est un triangle ABC en A&ABC est un triangle rectangle en B&
ABC est un triangle rectangle en C &ABC n'est pas un triangle rectangle\\ \hline
\end{tabularx}
\end{center}

\begin{minipage}{0.49\linewidth}
\item Voici la représentation graphique d'une fonction $f$.


La fonction $f$ est définie par:
\end{minipage}\hfill
\begin{minipage}{0.48\linewidth}
\psset{unit=0.8cm,arrowsize=2pt 3}
\begin{pspicture}(-4,-1)(3.5,3.5)
\psgrid[gridlabels=0pt,subgriddiv=1,gridwidth=0.15pt]
\psaxes[linewidth=1.25pt, labelFontSize=\scriptstyle]{->}(0,0)(-4,-1)(3.5,3.5)
\psplot[plotpoints=600,linewidth=1.25pt,linecolor=red]{-4}{3.5}{x 0.5 mul 1 add}
\end{pspicture}
\end{minipage}

\begin{center}
\renewcommand\arraystretch{1.9}
\begin{tabularx}{\linewidth}{|*{4}{X|}}\hline
$f(x) = 2x - 2$&$f(x) = 2x + 1$&$f(x) = \dfrac x2 - 2$&$f(x) = \dfrac x2 + 1$\\ \hline
\end{tabularx}
\end{center}

\begin{minipage}{0.55\linewidth}
\item Sur la figure ci-contre, le carré \no 2 est l'image du carré \no 1 par :
\end{minipage}\hfill
\begin{minipage}{0.42\linewidth}
\psset{unit=1cm,arrowsize=2pt 3}
\begin{pspicture}(-2,-2)(4,1)
\psframe[linewidth=1.5pt](-2,0)(-1,1)
\psframe[linewidth=1.5pt](2,0)(4,-2)
\psline(-1,1)(2,-2)\psline(-0.55,-0.1)(-0.55,0.1)\psline(-0.5,-0.1)(-0.5,0.1)
\psline(0.55,-0.1)(0.55,0.1)\psline(0.5,-0.1)(0.5,0.1)\psline(1.55,-0.1)(1.55,0.1)\psline(1.5,-0.1)(1.5,0.1)
\psline(-1,0)(2,0)
\rput(-1.5,0.5){1}\rput(3,-1){2}
\uput[d](-1,0){C}\uput[ur](0,0){O}\uput[u](2,0){E}
\end{pspicture}
\end{minipage}

\begin{center}
\begin{tabularx}{\linewidth}{|*{4}{>{\small}X|}}\hline
la symétrie centrale de centre O&la translation qui transforme C en E&l'homothétie de centre O et de rapport 2&l'homothétie de centre O et de rapport $- 2$\\ \hline
\end{tabularx}
\end{center}

\item Le cocktail Bora-Bora est composé de jus d'ananas, de jus de fruit de la passion et de jus de citron dans le ratio de 10~:~6~:~2. Pour réaliser $90$ cL de ce cocktail, il faut prévoir exactement :

\begin{center}
\begin{tabularx}{\linewidth}{|*{4}{X|}}\hline
6 cL de jus de fruit de la passion&30 cL de jus de fruit de
la passion&54 cl de jus de fruit de la passion&45 cL de jus de fruit de 
la passion\\ \hline
\end{tabularx}
\end{center}

\item  Un maraîcher a cueilli $408$ pommes et $168$ poires. Il décide de remplir des sacs pour ses clients comportant chacun le même nombre de pommes et le même nombre de poires, en utilisant tous les fruits cueillis.

Le plus grand nombre de sacs qu'il peut ainsi remplir est :

\begin{center}
\begin{tabularx}{\linewidth}{|*{4}{X|}}\hline
48 sacs&24 sacs&8 sacs&6 sacs\\ \hline
\end{tabularx}
\end{center}
\end{enumerate}

\bigskip

