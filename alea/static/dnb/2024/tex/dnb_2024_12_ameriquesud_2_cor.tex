
\bigskip

%On considère deux fonctions $f$ et $g$ définies par :

\[f(x) = x^2 - x - 6 \qquad \qquad g(x) = -2x.\]

\begin{enumerate}
\item 
	\begin{enumerate}
		\item %Montrer que l'image de 5 par la fonction $f$ est 14.
L'image de 5 par la fonction $f$ est $f(5) = 5^2 - 5 - 6 = 25 - 11 = 14$.
		\item %Déterminer l'antécédent de 4 par la fonction $g$.
L'antécédent de 4 par la fonction $g$ est le nombre $x$ tel que $g(x) = 4$,\\
 soit $x = \dfrac{4}{-2} = - 2$.

%Pour calculer des images de nombres par les fonctions $f$et $g$, on utilise un tableur et on obtient la copie d'écran suivante:
\begin{center}
\begin{tabularx}{0.9\linewidth}{|c|c|*{7}{>{\centering \arraybackslash}X|}}\hline
	&\text{A}&B&C&D&E&F&G&H\\ \hline
1	&$x$&$-4$&$-3$&$-2$&$-1$&0&1&2\\ \hline
2	&$f(x) = x^2 - x - 6$&14&6&0&$-4$&$-6$&$-6$&$-4$\\ \hline
3	&$g(x) = -2x$&8&6&4&2&0&$-2$&$-4$\\ \hline
\end{tabularx}
\end{center}

		\item %À l'aide des informations précédentes, citer deux antécédents de 14 par la fonction $f$.
On a vu que 5 a pour image 14 à la question 1 et le tableur montre que $- 4$ a aussi pour image 14 : donc $-4$ et 5 ont pour image 14 par la fonction $f$
		\item  %Quelle formule a-t-on pu saisir dans la cellule B2 avant de l'étirer vers la droite jusqu'à la cellule H2 ?
		On a écrit dans la cellule \texttt{B2} : \texttt{= B1*B1 -- B1 -- 6}.
		\item %Existe-t-il un nombre qui a la même image par la fonction $f$ et par la fonction $g$ ?
On lit sur le tableur :

$f(-3 ) = 6$ et $g(-3) = 6$ d'une part et $f(2) = - 4, \, g(2) = - 4$ d'autre part : il existe donc au moins deux nombres $- 3$ et 2 qui ont les mêmes images par $f$ et $g$.
		\end{enumerate}
\item 
	\begin{enumerate}
		\item %Montrer que, pour tout nombre $x,\, f(x)$ est égal à $(x + 2)(x - 3)$.
On développe $(x + 2)(x - 3) = x^2 - 3x + 2x - 6 = x^2 - x - 6 = f(x)$, quel que soit le nombre $x$, donc $f(x) = (x + 2)(x - 3)$.

		\item %Résoudre l'équation $f(x) = 0$.
D'après la question précédente résoudre $f(x) = 0$ revient à résoudre l'équation-produit $(x + 2)(x - 3) = 0$ : ce produit est nul si l'un des facteurs est nul, donc si $\left\{\begin{array}{l c l}
x + 2&=&0\\x - 3&=&0
\end{array}\right.$ ou 
$\left\{\begin{array}{l c l}
x &=&- 2\\x &=&3
\end{array}\right.$.
L'ensemble des solutions est donc $S = \{- 2~;~3\}$.
	\end{enumerate}
\end{enumerate}

\bigskip

