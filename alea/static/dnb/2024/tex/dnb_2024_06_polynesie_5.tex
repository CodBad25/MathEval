
\medskip

Dans cet exercice, les deux parties sont indépendantes.

On considère les fonctions $f$ et $g$ définies par 

\begin{center} $f(x) = (x + 2)^2 - x$\quad et \quad  $g(x) = 7x + 4$.\end{center}

\smallskip

\textbf{Partie A}

\medskip

\begin{enumerate}
\item Calculer $f(- 4)$.
\item Déterminer un antécédent de $3$ par la fonction $g$.
\end{enumerate}

\medskip

\textbf{Partie B}

\medskip

Trois élèves, Paul, Jane et Morgane, cherchent à résoudre l'équation $f(x) = g(x)$ par trois méthodes différentes.

\medskip

\begin{enumerate}
\item Paul utilise un tableur.

Il calcule ainsi les images des entiers compris entre $-3$ et $3$ par les fonctions $f$ et $g$.

\begin{center}
\begin{tabularx}{\linewidth}{|l|*{8}{>{\centering \arraybackslash}X|}}\hline
&A &B&C&D&E &F &G &H \\ \hline
1&{\Large $x$ }\vspace{-0.25cm}&$-3$&$ -2$& $-1$& 0 &1 &2 &3\\ \hline
2&$f(x)$&4&2 &2&4&8&14&22\\ \hline
3& $g(x)$& $-17$& $-10$& $-3$& 4& 11& 18& 25\\ \hline
\end{tabularx}
\end{center}

	\begin{enumerate}
		\item Quelle formule a-t-il saisie en cellule B3 puis étirée vers la droite pour compléter la ligne 3 du tableau ?
		\item Avec cette méthode, quelle(s) solution(s) trouve-t-il à l'équation $f(x) = g(x)$ ?
	\end{enumerate}
\item Jane utilise un logiciel de programmation.

Le programme suivant qu'elle a créé permet de tester l'égalité $f(x) = g(x)$ pour une valeur de $x$ choisie par l'utilisateur.

\begin{scratch}[num blocks]
\renewcommand*\numblock[1]{ligne \itshape#1}
\blockinit{quand \greenflag est cliqué}
\blockmove{demander \ovalnum{Choisir un nombre} et attendre}
\blockvariable{mettre \selectmenu{image par f} à \ovaloperator{\ovaloperator{\ovalmove{réponse}+\ovalnum{2}}*\ovaloperator{\ovalmove{réponse}+\ovalnum{2}}-\ovalmove{réponse}}}
\blockvariable{mettre \selectmenu{image par g} à \ovaloperator{\ovalnum{}*\ovalmove{réponse}}+\ovalnum{}}
 \blockifelse{si\ovaloperator{\ovalvariable{image par f} =\ovalvariable{image par g}}  alors}
 {\blocklook{dire \ovalnum{le nombre choisi est une solution de f(x)=g(x)} pendant \ovalnum{2} secondes}}
 {\blocklook{dire \ovalnum{le nombre choisi n'est pas une solution de f(x)=g(x)} pendant \ovalnum{2} secondes}}
\end{scratch}


Elle décide de tester toutes les valeurs entières entre $-5$ et 3.
	\begin{enumerate}
		\item Compléter sur le programme précédent, la ligne 4 du programme de Jane afin d'obtenir l'image par la fonction $g$ du nombre choisi.
		\item Quelle réponse donne le programme si le nombre choisi est $0$ ?
		\item En déduire une solution de l'équation $f(x) = g(x)$.
	\end{enumerate}
\item Morgane décide de résoudre cette équation par le calcul.
	\begin{enumerate}
		\item Démontrer que l'équation $f(x) = g(x)$ peut se ramener à l'équation $x^2 - 4x = 0$. 
		\item Factoriser l'expression $x^2 - 4x$.
		\item En déduire les solutions de l'équation $f(x) = g(x)$.
	\end{enumerate}
\item Dire pour chaque élève s'il a résolu l'équation $f(x) = g(x)$.

Expliquer pourquoi.
\end{enumerate}


