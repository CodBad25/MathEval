
\medskip


\begin{enumerate}
\item \textbf{Affirmation \no 1} : Fausse.

En effet, sur représentation en perspective cavalière, on voit que l'assemblage est constitué de 4 cubes, trois étant posés sur un plan, et le quatrième posé au dessus des trois autres. Sur les vues \og horizontales\fg{} (vue de face, vue de dos, vue de gauche et la vue de droite), ce quatrième cube sera visible au dessus des autres.

Sur le dessin présenté, on ne voit pas le quatrième cube.

\begin{minipage}{10cm}
	\emph{Remarque :} On peut aussi dessiner sur sa copie la vue que l'on devrait avoir de droite, qui donne le dessin ci-contre :
\end{minipage}
\hfill~
\begin{tikzpicture}[scale=0.7, baseline={(0,0.7)}]
	\draw(0,0) rectangle (2,1) (0,0) rectangle (1,2) ;
\end{tikzpicture}\hfill~

\item \textbf{Affirmation \no 2} : Fausse

Dans la configuration, on a :
\begin{itemize}[label=\textbullet]
	\item S, N et O sont alignés, dans cet ordre;
	\item S, U et D sont alignés, dans le même ordre.
\end{itemize}

Puisque l'on a :
\begin{itemize}[label=\textbullet]
	\item d'une part :\quad $\dfrac{\mathrm{SN}}{\mathrm{SO}} = \dfrac{1}{2}$, car N est le milieu de [SO];
	\item d'autre part : \quad $ \dfrac{\mathrm{SU}}{\mathrm{SD}} = \dfrac{5}{5 + 6} = \dfrac{5}{11}$;
	\item $\dfrac{1}{2} \neq \dfrac{5}{11}$
\end{itemize}

d'après la contraposée du théorème de Thalès, on en déduit que les droites (NU) et (OD) ne peuvent pas être parallèles.

\item \textbf{Affirmation \no 3} : Vraie.

En effet, nous avons deux expériences aléatoires dans lesquelles on est en situation d'équiprobabilité : les boules sont indiscernables au toucher dans l'une et le dé est non truqué dans l'autre.

La probabilité d'un événement est donc toujours égale au nombre d'issues favorables sur le nombre d'issues total.

Dans la première expérience, 6 issues sont favorables à l'événement (les 6 boules bleues) avec 10 issues possibles (6 + 4 = 10 boules dans l'urne), donc la probabilité qu'il se réalise est de $\dfrac{6}{10}= \dfrac{3}{5} = 0,6$.

Dans la seconde expérience, 3 issues sont favorables à l'événement (les issues 2; 4 et 6) avec 6 issues possibles (les six numéros portés par les faces du dé), donc la probabilité qu'il se réalise est de $\dfrac{3}{6}=\dfrac{1}{2}=0,5$.

On a bien : \quad $0,6 \geqslant 0,5$, \quad donc la probabilité d'obtenir une boule bleue est effectivement supérieure à la probabilité que le numéro annoncé soir pair.

\end{enumerate}

\bigskip

