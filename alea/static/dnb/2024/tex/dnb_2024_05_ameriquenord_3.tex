
Un cinéma propose trois tarifs :

\textbf{Tarif \og Classique\fg{}} : La personne paye chaque entrée $11$~\euro.

\textbf{Tarif \og Essentiel \fg{}} : La personne paye un abonnement annuel de $50$~\euro{} puis chaque entrée coûte $5$~\euro.

\textbf{Tarif \og Liberté \fg} : La personne paye un abonnement annuel de $240$~\euro{} avec un nombre d'entrées illimité.

\begin{enumerate}
\item Avec le tarif \og Classique \fg, une personne souhaite acheter trois entrées au cinéma.

Combien va-t-elle payer ?

\item Avec le tarif \og Essentiel \fg, une personne souhaite aller huit fois au cinéma.

Montrer qu'elle va payer $90$~\euro.

\item Dans la suite, $x$ désigne le nombre d'entrées au cinéma.

On considère les trois fonctions $f, g$ et $h$ suivantes :


\[f: x \longmapsto 50 + 5x \quad g: x \longmapsto 240 \quad h: x \longmapsto 11 x
\]

Associer, sans justifier, chacune de ces fonctions au tarif correspondant.

Le graphique ci-dessous représente le prix à payer en fonction du nombre d'entrées pour chacun de ces trois tarifs.

\begin{center}
\psset{xunit=0.25cm,yunit=0.03cm,arrowsize=2pt 3}
\begin{pspicture*}(-5,-30)(60,310)
\psaxes[linewidth=1.25pt,Dx=5,Dy=50]{->}(0,0)(0,0)(45,290)
\multips(1,0)(1,0){44}{\psline[linecolor=lightgray](0,0)(0,270)}
\multips(0,10)(0,10){27}{\psline[linecolor=lightgray](0,0)(44,0)}
\psline[linecolor=blue](0,240)(45,240)\uput[ul](25,270){\red $(d_1)$}
\psline[linecolor=cyan](0,50)(45,270)\uput[ul](48,270){\cyan $(d_2)$}
\psline[linecolor=red](25,275)\uput[r](45,240){\blue $(d_3)$}
\uput[u](52,-5){Nombre d'entrées}
\uput[r](-5,305){Prix à payer en \euro}
\end{pspicture*}
\end{center}

La droite ($d_{1}$) représente la fonction correspondant au tarif \og Classique\fg.

La droite ($d_{2}$) représente la fonction correspondant au tarif \og Essentiel\fg.

La droite $(d_{3})$ représente la fonction correspondant au tarif \og Liberté \fg.

\item Quel tarif propose un prix proportionnel au nombre d'entrées ?

\item Pour les questions suivantes, aucune justification n'est attendue.
	\begin{enumerate}
		\item Avec $150$~\euro, combien peut-on acheter d'entrées au maximum avec le tarif \og Essentiel \fg ?

		\item À partir de combien d'entrées, le tarif \og Liberté \fg{} devient-il le tarif le plus intéressant?

		\item Si on décide de ne pas dépasser un budget de $200$~\euro, quel est le tarif qui permet d'acheter le plus grand nombre d'entrées ?
		\end{enumerate}
\end{enumerate}
	
