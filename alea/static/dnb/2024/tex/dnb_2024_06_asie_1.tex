
\emph{Cet exercice est un questionnaire à choix multiples (QCM). Aucune justification n'est demandée. Pour chaque question, quatre réponses (A, B, C et D) sont proposées. Une seule réponse est exacte. Recopier sur la copie le numéro de la question et la réponse.}

\medskip

\textbf{Question 1}

Lequel de ces quatre nombres est premier ?
\begin{center}
\begin{tabularx}{\linewidth}{|*{4}{>{\centering \arraybackslash}X|}}\hline
Réponse A 	&Réponse B 	&Réponse C 	&Réponse D\\ \hline
1 			&21			& 37		& 54\\ \hline
\end{tabularx}
\end{center}

\textbf{Question 2}

L'aire totale du patron d'un cube d'arête $5$~cm est égale à...

\begin{center}
\begin{tabularx}{\linewidth}{|*{4}{>{\centering \arraybackslash}X|}}\hline
Réponse A &Réponse B &Réponse C &Réponse D\\ \hline
125 cm$^2$ &150 cm$^2$& 120 cm$^2$& 100 cm$^2$\\ \hline
\end{tabularx}
\end{center}

\textbf{Question 3}

Une forme factorisée de l'expression littérale $4x^2 - 9$ est\ldots 

\begin{center}
\begin{tabularx}{\linewidth}{|*{4}{>{\centering \arraybackslash}X|}}\hline
Réponse A &Réponse B &Réponse C &Réponse D\\ \hline
$(4x- 3)(4x +3)$ &$(2x- 3)(2x +3)$& $(2x- 3)^2$& $(4x- 9)(4x +9)$\\ \hline
\end{tabularx}
\end{center}

\textbf{Question 4}

Un écran de télévision est au format 16 : 9 ce qui signifie que la longueur et la largeur de l'écran sont dans le ratio $16 : 9$.

Dans ce cas, si la longueur de l'écran est de $110$~cm, sa largeur est d'environ ...

\begin{center}
\begin{tabularx}{\linewidth}{|*{4}{>{\centering \arraybackslash}X|}}\hline
Réponse A &Réponse B &Réponse C &Réponse D\\ \hline
62 cm &103 cm& 196 cm& 94 cm\\ \hline
\end{tabularx}
\end{center}

\textbf{Question 5}

On considère la série de valeurs : \quad 4,1\quad 3,67\quad 4,23\quad 4,5\quad  3,4

Quelle est la médiane de cette série ?

\begin{center}
\begin{tabularx}{\linewidth}{|*{4}{>{\centering \arraybackslash}X|}}\hline
Réponse A &Réponse B &Réponse C &Réponse D\\ \hline
0,83 &4,1& 4,23& 3,98\\ \hline
\end{tabularx}
\end{center}

