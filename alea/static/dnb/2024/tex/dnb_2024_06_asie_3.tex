
\begin{minipage}{0.45\linewidth}Trois élèves construisent chacun en vraie grandeur une même figure puis la découpent.

Ils obtiennent ainsi, à eux trois, trois pièces identiques, comme ci-contre.
\end{minipage}\hfill
\begin{minipage}{0.51\linewidth}
\psset{unit=1cm}
\def\logo{\begin{pspicture}(-1.732,0)(1.732,2)
\pspolygon(-1.732,1)(0,0)(1.732,1)\psarc(0,0){2}{30}{150}\psline(0,0)(0,1)
\end{pspicture}}
\begin{pspicture}(6.6,5)
\rput(1.8,2.6){\logo}\rput{-80}(5,2.4){\logo}\rput{170}(3,1){\logo}
\end{pspicture}

\end{minipage}

Le schéma ci-dessous représente la pièce construite par chaque élève avec les indications suivantes:

\begin{itemize}[label=$\bullet~$]
\item Les droites (AB) et (CG) sont perpendiculaires ;
\item Les points A, C et B sont alignés;
\item L'arc de cercle qui relie le point A au point B a pour centre le point G;
\item AC = CB;
\item CG = 10 cm et BG = 20 cm.
\end{itemize}
\begin{center}
\psset{unit=1.8cm}
\begin{pspicture}(-1.732,0)(1.732,2)
\pspolygon(-1.732,1)(0,0)(1.732,1)\psarc(0,0){2}{30}{150}
\psframe(0,1)(-0.2,0.8)
\uput[l](-1.732,1){A} \uput[r](1.732,1){B} \uput[u](0,1){C} \uput[d](0,0){G} 
\psline(-0.876,1.1)(-0.876,0.9)\psline(-0.846,1.1)(-0.846,0.9)
\psline(0.846,1.1)(0.846,0.9)\psline(0.876,1.1)(0.876,0.9)
\psline(0,0)(0,1)
\end{pspicture}
\end{center}
%dessin au dessous

\medskip

\begin{enumerate}
\item Démontrer que la longueur BC mesure environ 17,3 cm.
\item Quelle est l'aire du triangle BAG ? \emph{On donnera une valeur arrondie à l'unité}.
\item 
	\begin{enumerate}
		\item Montrer que l'angle $\widehat{\text{CGB}}$ mesure exactement $60\degres{}$.
		\item En déduire la mesure de l'angle $\widehat{\text{AGB}}$.
	\end{enumerate}
\item Les trois élèves pensent qu'ils peuvent former un disque complet avec leurs 3 pièces.

Expliquer pourquoi ils ont raison.
\item En déduire l'aire de la pièce obtenue par chacun des élèves. On donnera une valeur arrondie à l'unité.
\end{enumerate}

