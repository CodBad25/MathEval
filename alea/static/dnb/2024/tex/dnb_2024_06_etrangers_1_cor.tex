
\medskip

\begin{enumerate}
\item \textbf{Bonne réponse :} $1,93 \times 10^{-101}$

On a :\quad $0,193 \times 10^{-100} = 1,93 \times 10^{-1} \times 10^{-100} = 1,93 \times 10^{-101}$.

L'écriture scientifique de $0,193 \times 10^{-100}$ est : \quad $1,93 \times 10^{-101}$.

\item \textbf{Bonne réponse :} $\np[km/h]{84,2}$.

$\np[h]{5}\, \np[min]{42}$ correspondent à \quad $5 + \dfrac{42}{60} = 5 + \dfrac{6 \times 7}{6 \times 10} = 5 + \dfrac{7}{10} = \np[h]{5,7}$.

La vitesse moyenne de Lili, en km/h est donc :\quad $\dfrac{480}{5,7} = \dfrac{1600}{19} \approx 84,21$

Sa vitesse moyenne en km/h, arrondie au dixième est donc de $\np[km/h]{84,2}$.

\item \textbf{Bonne réponse : } Oui, en écrivant le nombre 2

En effet, puisqu'on suppose que chaque secteur a autant de chance d'être désigné, avec 15 secteurs, la probabilité de désigner un secteur est de $\dfrac{1}{15}$.

Comme 8 des secteurs portent le numéro 2, en inscrivant 2 dans le secteur où le nom-bre a été effacé, on aura 9 secteurs favorables à l'évènement sur 15, donc une probabilité de $\dfrac{9}{15}$, soit, en simplifiant par 3 : $\dfrac{3}{5}$.

\item \textbf{Bonne réponse :} rien de particulier.

Si on range les nombres dans l'ordre croissant : \quad $1~;~3~;~5~;~10~;~10~;~11~;~17$.

Pour cette série, l'étendue est de :\quad $17 - 1 = 16 \neq 5$.

Il y a 7 valeurs, et 7 est un nombre impair, donc la médiane est la $\dfrac{7 + 1}{2} = 4$\up{e} valeur de la série, c'est-à-dire $10 \neq 5$.

La moyenne de la série est :\quad $\dfrac{1+3+5+10+10+11+17}{7} = \dfrac{57}{7}\approx 8,1\neq 5$

Aucun des éléments évoqués n'est égal à 5.

\item \textbf{Bonne réponse :} $\dfrac{4}{15}$

Puisque Léa paye $\dfrac{1}{5}$ du prix au moment de la commande, il lui reste $1 - \dfrac{1}{5} = \dfrac{4}{5}$ du prix à payer.

Si ce qu'il lui reste à payer est réparti équitablement en trois paiements, alors chaque paiement représente : $\dfrac{1}{3}\times \dfrac{4}{5} = \dfrac{4}{15}$ du prix total du vélo.

\end{enumerate}


