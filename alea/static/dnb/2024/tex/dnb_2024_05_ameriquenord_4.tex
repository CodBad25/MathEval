
M. et M\up{me} Martin veulent construire une terrasse en béton dans leur jardin. Ils souhaitent que leur terrasse ait une hauteur de $15 \mathrm{~cm}$. Les représentations ci-dessous ne sont pas à l'échelle.


\begin{center}
\psset{xunit=0.5cm,yunit=0.6cm,arrowsize=2pt 3}
\begin{pspicture}(26,9.8)
\pspolygon(2,0.5)(22.5,2.8)(22.6,4.7)(2.1,2.4)%AIJE
\psline(14.3,1.9)(14.4,3.8)(13.1,9.2)%DFG
\psline(22.6,4.7)(13.1,9.2)(0.7,8.5)(2.1,2.4)%JGHE
\psline(0.7,8.5)(0.6,6.7)(2,0.5)%HBA
\psline[linestyle=dashed](0.6,6.7)(13.1,7.3)(22.5,2.8)%BCI
\psline[linestyle=dashed](13.1,9.2)(13.1,7.3)(14.3,1.9)%GCD
\uput[d](2,0.5){A} \uput[l](0.6,6.7){B} \uput[dl](13.1,7.3){C} \uput[d](14.3,1.9){D}
\uput[ur](2.1,2.4){E} \uput[ur](14.4,3.8){F} \uput[u](13.1,9.2){G} \uput[ul](0.7,8.5){H}
\uput[d](22.5,2.8){I} \uput[u](22.6,4.7){J}
\psline{<->}(22.7,2.8)(22.8,4.7)\uput[r](22.75,3.75){15 cm}
\rput(7.6,5.8){Terrasse en béton}
\rput(7,10){\textbf{Vue en perspective de la terrasse}}
\end{pspicture}
\end{center}


\begin{center}
\psset{unit=1cm,arrowsize=2pt 3}
\begin{pspicture}(14,5)
\pspolygon(1,1)(14,1)(9.5,4)(1,4)%EJGH
\psline[linestyle=dashed](9.5,4)(9.5,1)%GF
\uput[dl](1,1){E} \uput[d](9.5,1){F} \uput[u](9.5,4){G} \uput[ul](1,4){H} \uput[dr](14,1){J}
\psframe(9.5,1)(9.7,1.2)
\uput[u](5.25,4){6 m}\uput[l](1,2.5){3 m}
\psline[linewidth=0.6pt]{<->}(1,0)(14,0)\uput[u](7.5,0){10 m}
\rput(5.25,2.5){EFGH est un rectangle}
\rput(5.25,5){\textbf{Vue de dessus de la terrasse}}

\end{pspicture}
\end{center}

\textbf{Rappel :}

Le volume d'un prisme est donné par la formule : $V=$ Aire de la base $\times$ Hauteur


\begin{enumerate}
\item Montrer que FJ $= 4$~m.

\item Afin de pouvoir couler le béton, M. et M\up{me} Martin doivent délimiter la terrasse en installant des planches tout autour. Quelle longueur de planches doivent-ils acheter au minimum ?

\item M. et M\up{me} Martin souhaitent réaliser $4 \mathrm{~m}^{3}$ de béton.
	\begin{enumerate}
		\item Montrer que le volume de la terrasse est bien inférieur à $4 \mathrm{~m}^{3}$.

		\item Sachant que pour faire 1 m$^{3}$ de béton, il faut $250$~kg de ciment, quelle masse de ciment (en kg) doivent-ils acheter pour réaliser 4 m$^{3}$ de béton ?
		
		\item Pour faire du béton, on ajoute de l'eau à un mélange de ciment, de gravier et de sable.
Dans ce mélange, les masses de ciment - gravier - sable sont dans le ratio $2~:~ 7~:~5$.

Déterminer (en kg), la masse de gravier et la masse de sable nécessaires pour réaliser les 4 m$^{3}$ de béton.

	\end{enumerate}
	
\item M. et M\up{me} Martin souhaitent peindre la surface supérieure de leur terrasse.

À l'aide des documents 1, 2 et 3 , déterminer le type et le nombre de pots nécessaires pour effectuer ces travaux avec un coût minimum.

\medskip

\textbf{Document 1 :} Pots de peinture proposés

\begin{center}
\begin{tabular}{|c|c|c|}\cline { 2 - 3 }
\multicolumn{1}{c|}{} 		&Pot A 	& Pot B \\ \hline
Contenance (en litres) 	&5 		& 10 \\ \hline
Prix (en euros) 			&79,90 	& 129,90 \\ \hline
\end{tabular}
\end{center}

\textbf{Document 2 :} L'offre du mois : Moins 50\,\% sur le deuxième article identique.

\medskip

\textbf{Document 3 :}

Deux couches de peinture sont nécessaires. 1 litre de peinture permet de réaliser une couche de 5~m$^{2}$.
\end{enumerate}

