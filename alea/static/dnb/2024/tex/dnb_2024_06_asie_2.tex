
Voici trois affirmations. Pour chacune d'entre elles, justifier si elle est vraie ou fausse.

\medskip

\begin{minipage}{0.48\linewidth}
\begin{enumerate}
\item Voici un assemblage de quatre cubes identiques représenté en perspective cavalière.
\end{enumerate}
\end{minipage}
\hfill
\begin{minipage}{0.48\linewidth}
\psset{unit=1cm,arrowsize=2pt 3}
\begin{pspicture}(5.6,3.4)
%\psgrid
\pspolygon(2,0.4)(3,0.2)(3,1.3)(2,1.5)
\pspolygon(2,1.5)(3,1.3)(3,2.4)(2,2.6)
\psline(3,0.2)(4,0)(4,1.1)(3,1.3)
\psline(4,0)(4.6,0.3)(4.6,1.4)(4,1.1)
\psline(4.6,0.3)(5.2,0.6)(5.2,1.7)(4.6,1.4)
\psline(4.6,1.4)(3.6,1.6)(3,1.3)
\psline(5.2,1.7)(4.2,1.9)(3.6,1.6)
\psline(3.6,1.6)(3.6,2.7)(3,2.4)
\psline(3.6,2.7)(2.6,2.9)(2,2.6)
\psline[linestyle=dashed](2,0.4)(2.6,0.7)(2.6,2.9)
\psline[linestyle=dashed](2.6,0.7)(4.6,0.3)
\psline[linestyle=dashed](3,0.2)(4.2,0.8)(4.2,1.9)
\psline[linestyle=dashed](4.2,0.8)(5.2,0.6)
\psline[linestyle=dashed](3.6,1.6)(3.6,0.5)
\psline[linestyle=dashed](2,1.5)(2.6,1.8)(3.6,1.6)
\rput(1,0.5){Face avant}
\psline{->}(1,0.65)(2.45,1)
\end{pspicture}
\end{minipage}


\textbf{Affirmation \no 1} : \og La vue de droite est représentée par le dessin ci-dessous. \fg

\emph{Le dessin n'est pas à l'échelle.}

\begin{center}
\psset{unit=1cm,arrowsize=2pt 3}
\begin{pspicture}(3,1.5)
\psframe(3,1.5)\psline(1.5,0)(1.5,1.5)
\end{pspicture}
\end{center}

\begin{enumerate}[start=2]
\item On considère le schéma ci-dessous (qui n'est pas à l'échelle) :

\begin{center}
\psset{unit=1cm,arrowsize=2pt 3}
\begin{pspicture}(12,5.8)
%\psgrid
\pspolygon(6.75,3)(0.2,1.5)(5.4,5)(8.8,0)(2.8,3.2)%UOSDN
\uput[ur](6.75,3){U} \uput[l](0.2,1.5){O} \uput[u](5.4,5){S}
\uput[r](8.8,0){D} \uput[ul](2.8,3.2){N} \uput[d](4.2,2.3){E} 
\rput(10.6,4.5){ON = 6 cm} \rput(10.6,4){SU = 5 cm} \rput(10.6,3.5){UD = 6 cm}
\psline[linecolor=red](1.6,2.6)(1.8,2.4)\psline[linecolor=red](4,4.2)(4.2,4)
\end{pspicture}
\end{center}

\textbf{Affirmation \no 2} : \og Les droites (NU) et (OD) sont parallèles. \fg

\item On considère deux expériences aléatoires.

Dans la première expérience aléatoire, on tire une boule dans une urne opaque et on annonce sa couleur. Dans l'urne, il y a 4 boules rouges et 6 boules bleues indiscernables au toucher.

Dans la seconde expérience aléatoire, on lance un dé non truqué avec des faces numérotées de 1 à 6 et on annonce le nombre qui apparaît sur la face du dessus.

\textbf{Affirmation \no 3} : \og La probabilité d'obtenir une boule bleue dans l'urne est supérieure à la probabilité d'obtenir un nombre pair avec le dé \fg.
\end{enumerate}

