
\medskip

%\emph{Cet exercice est un questionnaire à choix multiple (QCM).\\
%Pour chaque question, trois réponses ({\rm A, B} ou {\rm C}) sont proposées.\\
%Une seule réponse est exacte.\\
%Recopier sur la copie le numéro de la question et la lettre correspondant à la réponse exacte.\\
%Aucune justification n'est demandée.}
%
%\begin{center}
%\begin{tabularx}{\linewidth}{|m{8.5cm}|*{3}{>{\centering \arraybackslash}X|}}\hline
%\textbf{Question}&\textbf{Réponse A} &\textbf{Réponse B} &\textbf{Réponse C}\\ \hline
%\textbf{1.~}
%On considère la fonction $f$ définie par $f(x) = 3x - 2$.
%
%Quelle est l'image de $-4$ par cette fonction ?&$-14$&$-10$&$-3$\\ \hline
%\textbf{2.~}
%Combien vaut $(-5)^3$ ?&$-125$ &$-15$ &125\\ \hline
%\textbf{3.~}
%Quelle est l'image du point J par la translation qui transforme C en A ?
%
%\begin{center}
%\psset{unit=1cm}
%\begin{pspicture}(3.1,2.2)
%\psdots(0,0)(1,0)(2,0)(3,0)(0,1)(1,1)(2,1)(3,1)(0,2)(1,2)(2,2)(3,2)
%\uput[ur](0,1){A} \uput[ur](0,2){B} \uput[ur](1,2){C} \uput[ur](1,1){D}
%\uput[ur](1,0){E} \uput[ur](0,0){F} \uput[ur](2,2){G} \uput[ur](3,2){H}
%\uput[ur](3,1){I} \uput[ur](2,1){J} \uput[ur](3,0){K} \uput[ur](2,0){L}
%\end{pspicture}
%\end{center}&H&E&D\\ \hline
%\textbf{4.~}
%
%\begin{minipage}{2.5cm}Quel est l'antécédent de 3 par la fonction $f$ ?
%\end{minipage} \hfill
%\begin{minipage}{4cm}
%\psset{unit=0.5cm}
%\begin{pspicture*}(-1,-3.5)(4,4)
%\psgrid[gridlabels=times =0.15pt,subgriddiv=1]
%\psaxes[linewidth=1.25pt,labelFontSize=\scriptstyle]{->}(0,0)(-1,-3)(4,4)
%\psplot[plotpoints=500,linewidth=1.25pt,linecolor=blue]{-1}{3}{3 2 x mul sub}
%\uput[ur](1,1){\blue $\mathcal{C}_f$}
%\end{pspicture*}
%\end{minipage}&3&$- 3$&0\\ \hline
%\textbf{5.~}
%On a mesuré les tailles, en m, de sept élèves :
%
%\[1,46~;~1,65~;~1,6~;~1,72~;~1,7~;~1,67~;~1,75\]
%
%Quelle est la médiane, en m, de ces tailles ?&1,72&1,67&1,65\\ \hline
%\\ \hline
%\begin{minipage}{4cm}
%\textbf{6.~} {\small Dans le triangle ABC rectangle en A ci-contre, qui n'est pas en vraie grandeur, quelle est la valeur de $\cos \alpha$ ?}
%\end{minipage}\hfill
%\begin{minipage}{4cm}
%\psset{unit=0.7cm}
%\begin{pspicture}(5.8,2.4)
%\pspolygon(0.6,0.3)(5.1,0.3)(0.6,2.1)%ABC
%\uput[dl](0.6,0.3){A} \uput[dr](5.1,0.3){B} \uput[ul](0.6,2.1){C}
%\uput[l](0.6,1.2){3} \uput[d](2.85,0.3){4} \uput[ur](2.85,1.2){5} \uput[ul](4.6,0.18){\small $\alpha$}
%\psframe(0.6,0.3)(0.8,0.5)\psarc(5.1,0.3){0.5}{160}{180}
%\end{pspicture}
%\end{minipage}&0,8&0,75&0,6\\ \hline
%\end{tabularx}
%%\end{enumerate}
%\end{center}
\begin{enumerate}
\item $f(x) = 3x - 2$, donc $f(-4) = 3 \times (- 4) - 2 = -12 - 2 = - 14$. (réponse A)
\item $(- 5)^3 = (- 5) \times (- 5) \times (- 5) = - 125$. (réponse A)
\item Si C a pour image A par $t_{\vect{\text{CA}}}$, alors J a pour image E. (réponse B)
\item L'antécédent de 3 est 0 :\: $f(0) = 3$. (réponse C)
\item On a dans l'ordre croissant : 1,46~;~1,6~;~1,65~;~ {\red 1,67}~;~1,7~;~ 1,72~;~1,75

Il y a autant de tailles inférieures à 1,67 que de tailles supérieures à 1,67 : 1,67 est la médiane. (réponse B)
\item On a par définition : $\cos \alpha = \dfrac{\text{long. côté adjacent à }\alpha}{\text{long. hypoténuse}} = \dfrac45 = \dfrac{8}{10} = 0,8$. (réponse A)
\end{enumerate}

\bigskip

