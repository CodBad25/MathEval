
\textbf{PARTIE A}

\medskip

\begin{enumerate}
\item D'après le tableau indicatif des distances, la distance entre Marseille et Strasbourg est de 803 km. Pour l'aller et le retour, on aura donc une distance totale de $803 \times 2 = \np[km]{1606}$.

\item En choisissant la formule B, ils vont donc payer :

$300 + \np{1606}\times 0,25 = 300 + 401,25  = 701,50$~(\euro).

Ce calcul confirme le coût total annoncé.

\item On peut déjà dire que le prix payé avec la formule B est inférieur à celui de la formule C, car :\quad $701,50 < 900$.

Avec la formule A, le prix payé serait : \quad$ 1606 \times 0,5 = 803 > 701,25 $. Le prix A est aussi plus élevé que le B.

La formule la plus avantageuse est donc la formule B.

\item Si la voiture doit faire 1606 km (pour la distance aller et retour), alors elle va consommer :\quad $1606\times \dfrac{5,6}{100} = \np[L]{89,936}$ de carburant.

Si le prix moyen du carburant est de 1,87 \euro{} par litre, le prix du carburant sera de :\quad $1,87 \times 89,936 = 168,18032$, soit, au centime près : 168,18\,\euro{}.

Le coût total du voyage inclus la location de la voiture, l'achat du carburant nécessaire et les péages, et donc est de : \quad $701,50 + 168,18 + 115,80 = 985,48$\,\euro{}.

Comme $985,48 \leqslant 1000$, le budget est suffisant pour ce voyage.
\end{enumerate}

\medskip

\textbf{PARTIE B }: Étude des formules

\begin{enumerate}[resume]
\item Pour la formule A : $0,50 \times x = 0,5x$;

	Pour la formule B : $300 + 0,25 \times x = 0,25x + 300$;

	Pour la formule C : $900$ (ici, le prix est fixe et ne dépend donc pas de $x$).

\item Pour la formule A, on a une formule qui est l'expression d'une fonction linéaire, représentée par une droite passant par l'origine du repère : cela ne peut-être que la courbe 3;

La formule B est l'expression d'une fonction affine non linéaire, dont l'ordonnée à l'origine est 300, elle est donc représentée par une droite qui coupe l'axe des ordonnées à la graduation 300 : c'est la courbe 2;

La formule C est l'expression d'une fonction constante égale à 900, représentée par une droite horizontale coupant l'axe des ordonnées à la graduation 900 : c'est donc la courbe 1 (on pouvait aussi procéder par élimination, mais de toutes façons, aucune justification n'était attendue, ici).

\item Résolvons l'équation : \quad$\aligned[t] 0,25x + 300 = 0,5x &\iff 300 = 0,25x\\
&\iff \dfrac{300}{0,25} = x\\
&\iff x = \np{1200}\endaligned$

L'équation a une unique solution, \np{1200}.

Cela signifie que c'est pour \np{1200}~km parcourus que le prix payé avec la formule B est le même que celui avec la formule A.

Graphiquement, cela correspond donc à l'abscisse du point d'intersection entre les courbes 2 (qui représente le prix payé avec la formule B) et 3 (prix payé avec la formule A).

\item
	\begin{enumerate}
		\item Puisqu'il n'y a pas de justification attendue, on peut procéder par lecture graphique : après l'abscisse 2400, et donc notamment pour \np{2500}~km parcourus, c'est la courbe 1 qui est sous les deux autres, c'est donc la formule C qui est la plus avantageuse.

		\item Pour que la formule A soit la plus intéressante, il faut que la courbe 3 soit la plus basse. Ceci est vrai pour les abscisses entre 0 et \np{1200}, donc n'importe quelle distance choisie dans l'intervalle $[0 ~;~\np{1200}]$ est une bonne réponse. (même si la réponse 1200 km n'est pas la meilleure, car, les formules A et B sont toutes les deux les plus intéressantes pour cette distance là).

		\item Ici, cela va dépendre de la distance parcourue :
		\begin{itemize}[label=\textbullet]
			\item de 0 km à \np{1200}~km, le moins cher est la formule A ;
			\item de \np{1200}~km à \np{2400}~km, le moins cher est la formule B ;
			\item de \np{2400}~km à \np{2600}~km parcourus, le moins cher est la formule C.
		\end{itemize}
		Pour exactement \np{1200}~km parcourus, les formules A et B sont au même prix, pour exactement 2400 km parcourus, les formules B et C sont au même prix.

	\end{enumerate}
\end{enumerate}

