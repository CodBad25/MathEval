
\medskip

\begin{enumerate}
\item Anne et Jean ont acheté $630$ dragées roses et $810$~dragées blanches qu'ils ont mises dans
un sachet. On suppose que les dragées sont indiscernables au toucher.
	\begin{enumerate}
		\item Combien Anne et Jean ont-ils acheté de dragées au total ?
		\item Anne prend au hasard une dragée dans le sachet. Quelle est la probabilité qu'elle
obtienne une dragée blanche ?
	\end{enumerate}
\item Avec ces dragées, ils réalisent des ballotins pour leur mariage de sorte que: 
\begin{itemize}
\item le nombre de dragées roses est le même dans chaque ballotin ;
\item le nombre de dragées blanches est le même dans chaque ballotin ;
\item toutes les dragées soient utilisées.
\end{itemize}

	\begin{enumerate}
		\item Peuvent-ils réaliser $21$ ballotins?
		\item Décomposer $630$ et $810$ en produits de facteurs premiers.
		\item En déduire le nombre maximum de ballotins qu'Anne et Jean pourront réaliser.
Donner alors la composition de chaque ballotin.
	\end{enumerate}
\end{enumerate}

\bigskip

