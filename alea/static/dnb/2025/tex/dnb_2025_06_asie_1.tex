
\medskip

\emph{Cet exercice est un questionnaire à choix multiple ({\rm QCM}). Aucune justification n'est demandée. Pour chaque question, quatre propositions ({\rm A, B, C} et {\rm D}) sont données. \\
\textbf{Une seule est exacte}. Recopier sur la copie le numéro de la question, ainsi que la lettre de la réponse.}

\medskip

\textbf{Question 1:}

\medskip

Dans une urne, on dispose de 4 boules bleues, 6 boules violettes, 7 boules rouges, 3 boules jaunes, toutes indiscernables au toucher. On tire une boule au hasard.

Quelle est la probabilité d'obtenir une boule violette ?

\smallskip

\renewcommand \arraystretch {2.2}
\begin{tabularx}{\linewidth}{|*{4}{>{\centering \arraybackslash}X|}}
\hline
Proposition A &Proposition B &Proposition C& Proposition D\\\hline
$\dfrac{6}{14}$&$\dfrac{1}{4}$&$\dfrac{3}{10}$&$\dfrac{14}{20}$\\
\hline 
\end{tabularx}

\bigskip

\textbf{Question 2:}

\medskip

Calculer 70\,\% d'une quantité revient à multiplier cette quantité par :

\smallskip

\begin{tabularx}{\linewidth}{|*{4}{>{\centering \arraybackslash}X|}}
\hline
Proposition A &Proposition B &Proposition C& Proposition D\\\hline
 0,30& 0,70& 1,70&1,30\\
 \hline
 \end{tabularx}
 
\bigskip

\textbf{Question 3 :}

\medskip


On considère la série suivante composée des 5 valeurs : 7 ; 18 ; 12 ; 13 ; 15.

\smallskip

\begin{tabularx}{\linewidth}{|*{4}{>{\centering \arraybackslash}X|}}
\hline
Proposition A &Proposition B &Proposition C& Proposition D\\\hline
L'étendue de cette série est 8 &La médiane de cette série est 12& La moyenne de cette série est 53&La moyenne de cette série est 13\\\hline
\end{tabularx}

\medskip



\medskip

\begin{minipage}[]{0.56\linewidth}
\textbf{Question 4:}

Une fonction affine $f$ a pour représentation graphique la courbe $\mathcal{C}_f$ ci-contre.

L'expression de la fonction $f$ est:
\medskip
\begin{tabularx}{\linewidth}{|*{2}{>{\centering \arraybackslash}X|}}
\hline
Proposition A & $f(x) = 2x+4$\\ \hline
Proposition B&$f(x) = 4x- 2$\\\hline
Proposition C &$f(x) = -2x +4$\\\hline
Proposition D& $f(x) = -4x+ 2$\\\hline
\end{tabularx}
\end{minipage}
\hfill
\begin{minipage}[]{0.42\linewidth}

\psset{xunit=1.3cm,yunit=1.3cm,labelFontSize=\scriptstyle,showorigin=false}
\begin{pspicture*}(-1,-1)(3.7,4.2)
\psgrid[gridcolor=lightgray,gridlabels=0pt]
\psaxes[linewidth=1.25pt]{->}(0,0)(-0.95,-0.95)(3.7,4.2)
\psplot[linewidth=1.25pt,linecolor=blue,plotpoints=500]{-0.8}{2.75}{x 2 neg mul 4 add}
\uput[u](0.8,2.5){\blue $\mathcal{C}_f$}
\end{pspicture*}
\end{minipage}

\vspace {0.5cm}

