
\textbf{PARTIE A :}

\medskip

Un magasin a reçu $650$ poissons dont $350$ poissons de type A et $300$ poissons de type B.

La responsable du magasin souhaite vendre ces poissons par lots de sorte que :

\begin{itemize}[label=$\bullet~$]
\item le nombre de poissons de type A soit le même dans chaque lot ;
\item le nombre de poissons de type B soit le même dans chaque lot ;
\item tous les poissons soient répartis dans les lots.
\end{itemize}

\medskip

\begin{enumerate}
\item Parmi les trois propositions suivantes, laquelle correspond à la décomposition en produits de facteurs premiers du nombre $300$ ? \textbf{Aucune justification n'est demandée.}

\begin{center}\begin{tabularx}{0.6\linewidth}{|*{3}{>{\centering \arraybackslash}X|}}\hline
Proposition 1 &Proposition 2&Proposition 3\\ \hline
$2^2 \times 5 \times 15$& $2\times 2\times 3\times 5\times 5$&
$22 \times3 \times 5^2$\\ \hline
\end{tabularx}
\end{center}

\item Donner la décomposition en produit de facteurs premiers du nombre $350$.
\item Quel nombre maximal de lots la responsable du magasin pourra-t-elle constituer ?
\item Dans ce cas, combien y aura-t-il de poissons de chaque type dans chaque lot ?
\end{enumerate}

\bigskip


\textbf{PARTIE B :}

\medskip

Le magasin a d'autres poissons, appelés \og poissons combattants \fg.

\medskip

\begin{enumerate}
\item En captivité, il faut prévoir au moins $15$ litres d'eau par poisson combattant.

Sachant qu'un aquarium est rempli aux $\dfrac45$ de sa hauteur, lequel doit-on choisir pour un poisson combattant ?

\begin{center}\begin{tabular}{|m{3cm}|m{3cm}|m{6cm}|}\hline
\textbf{Aquarium 1}&\textbf{Aquarium 2} &\textbf{Rappels}\\
\psset{unit=1cm}
\begin{pspicture}(-1.2,-0.4)(1.2,1.5)
\psellipse(0,1.2)(0.8,0.2)
\psellipticarc(0,0)(0.8,0.2){180}{0}
\psellipticarc[linestyle=dashed](0,0)(0.8,0.2){0}{180}
\psline(-0.8,0)(-0.8,1.2)\psline(0.8,0)(0.8,1.2)
\end{pspicture}

{\centering \textbf{Cylindre}}

Diamètre de la base = 30 cm

Hauteur : 25 cm&
\psset{unit=1cm}
\begin{pspicture}(2,1.7)
\psframe(1.2,1.2)\psline(1.2,0)(1.7,0.5)(1.7,1.7)(1.2,1.2)%côté droit
\psline(1.7,1.7)(0.5,1.7)(0,1.2)
\psline[linestyle=dashed](0,0)(0.5,0.5)(1.7,0.5)
\psline[linestyle=dashed](0.5,0.5)(0.5,1.7)
\end{pspicture}

Pavé droit

Longueur : 28 cm

Largeur : 28 cm

Hauteur : 30 cm&
Le volume d'un pavé droit est donné par la formule

{\centering $V = \text{Longueur} \times \text{Largeur} \times \text{Hauteur}$}

Le volume d'un cylindre de rayon de la base $r$ est donné par la formule

{\centering $V = \pi \times r^2 \times \text{Hauteur}$}

{\centering 1 dm$^3 = 1$ L}\\ \hline
\end{tabular}
\end{center}


\item Le prix d'un poisson combattant est de $15$~\euro. Une famille achète un poisson combattant et un aquarium. L'aquarium coûte $40$~\euro.

Le vendeur propose une remise de $15\,\%$ sur le prix total.

Combien va payer la famille ?
\end{enumerate}
