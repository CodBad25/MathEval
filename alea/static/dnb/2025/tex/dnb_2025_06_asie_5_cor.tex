
\subsection*{Partie A}
\begin{enumerate}
	\item C'est la proposition 2, car on a bien :\quad $ 300=2 ^ 2 \times 3 \times 5^2$, \quad de plus dans les propositions 1 et 3, il y a des facteurs qui ne sont pas premiers (15 et 22 respectivement).

	\item On a : \hfill~
	  $\begin{aligned}[t]
			350&=35 \times 10\\
			&=5 \times 7 \times 2\times 5\\
			&=2 \times 5^2 \times 7
	\end{aligned}$\hfill~

	\item Pour respecter les consignes : tous les lots sont identiques et tous les poissons sont répartis dans les lots, il faut que le nombre de lots soit à la fois un diviseur de 300 et de 350.

	$\operatorname{PGCD}(300;350) =2^1 \times 5^2=2 \times 25=50$.

	Le magasin pourra constituer 50 lots maximum.

	\item $\dfrac{350}{50}=7$ et $\dfrac{300}{50} = 6$.

	Dans chaque lot il y aura $7$ poissons de type A et $6$ poissons de type B.
\end{enumerate}

\subsection*{Partie B}

\begin{enumerate}
	\item Pour l'aquarium 1, les $\dfrac{4}{5}$ de la hauteur représentent $\dfrac{4}{5}\times 25 = \np[cm]{20}$.

	Le volume d'eau sera donc celui d'un cylindre de rayon \np[cm]{15}  et de hauteur \np[cm]{20} :

	$\mathcal{V}_1 = \pi  \times 15^2 \times 20 = 4500\pi \approx \np[cm^3]{14137}$\quad soit $ \mathcal{V}_1 \approx \np[dm^3]{14,2} \approx \np[L]{14,2}$.

	L'aquarium 1 ne suffit pas.


	Pour l'aquarium 2, les $\dfrac{4}{5}$ de la hauteur représentent $\dfrac{4}{5}\times 30 = \np[cm]{24}$.

	Le volume d'eau sera donc celui d'un pavé droit, de dimensions \np[cm]{28}, \np[cm]{28} et \np[cm]{24}.

	$\mathcal{V}_2 = 28 \times 28 \times 24 = \np[cm^3]{18816} = \np[dm^3]{18,816} = \np[L]{18,816} > \np[L]{15}$

	\textbf{Réponse :} C'est l'aquarium 2 qu'il faut choisir.

\item Prix : $(15 + 40) \times \left(1-\dfrac{15}{100}\right)=55 \times 0,85 = 46,75$ \euro{}

\textbf{Réponse :} Le prix à payer sera donc de :\quad 46,75 \euro{}.

On peut aussi calculer $15\%$ de $55$: $\dfrac{15}{100} \times 55=8,25$ puis le soustraire à $55$. On peut aussi utiliser un produit en croix.
\end{enumerate}

