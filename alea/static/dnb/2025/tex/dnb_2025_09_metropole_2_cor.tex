
\medskip

%Cet exercice est un questionnaire à choix multiple(QCM). Pour chaque question, quatre
%réponses (A, B, C ou D) sont proposées. Une seule réponse est exacte. Recopier sur la copie le numéro de la question et la lettre correspondant à la réponse exacte. Aucune justification n'est demandée.
%
%\medskip

\begin{enumerate}
\item %On considère la série suivante :
%
%\[4~;~8~;~11~;~7~;~2~;~3~;~14\]
%
L'étendue est égale à $14 - 2 = 12$ : réponse C.
%Quelle est l'étendue de cette série ?
%\begin{center}
%\begin{tabularx}{0.75\linewidth}{|*{4}{>{\centering \arraybackslash}X|}}\hline
%A&B&C&D\\ \hline
%10 &7& 12& 14\\ \hline
%\end{tabularx}
%\end{center}
\item %Quel est le volume correspondant à 1 L ?

%\begin{center}
%\begin{tabularx}{0.75\linewidth}{|*{4}{>{\centering \arraybackslash}X|}}\hline
%A&B&C&D\\ \hline 
%1 m$^3$& 1cm$^3$&1dm$^3$&1mm$^3$\rule[-2mm]{0mm}{6mm}\\ \hline
%\end{tabularx}
%\end{center}
1 L = 1 dm$^3$ : réponse C.
\item %Quel est le nombre dont l'écriture scientifique est $8,6 \times 10^{-4}$ ?

%\begin{center}
%\begin{tabularx}{0.75\linewidth}{|*{4}{>{\centering \arraybackslash}X|}}\hline
%A&B&C&D\\ \hline 
%\np{86000}& \np{0,00086}& \np{- 0,00086}& \np{0,000086}\\ \hline
%\end{tabularx}
%\end{center}
$8,6 \times 10^{-4} = \np{0,00086}$ : réponse B.
\item %La longueur et la largeur du drapeau de la France sont dans le ratio $3~:~2$.

%Quelle est la largeur du drapeau de la France dont la longueur est égale à 90~cm ? 
On a $\dfrac{\text{Longueur}}{\text{largeur}} = \dfrac32 = \dfrac{90}{\text{largeur}}$, d'où par produits en croix : $3 \times \text{largeur} = 2 \times 90$ et en simplifiant par 3 : $\text{largeur} = 2 \times 30 = 60$ : réponse D.

%\begin{center}
%\begin{tabularx}{0.75\linewidth}{|*{4}{>{\centering \arraybackslash}X|}}\hline
%A&B&C&D\\ \hline
%54 cm &135 cm& 45 cm&  60 cm\\ \hline
%\end{tabularx}
%\end{center}

\item %Le prix d'un parfum est passé de $75$~\euro{} à $60$~\euro.

%Quel pourcentage de réduction a été appliqué ?
La baisse est de $75 - 60 = 15$~(\euro) pour un prix initial de 75~\euro, soit une baisse en pourcent de :

$\dfrac{15}{75} \times 100 = \dfrac{15 \times 1}{15 \times 5} \times 100 = \dfrac{1}{5} \times 100 = 20$~(\%) : réponse D.
%\begin{center}
%\begin{tabularx}{0.75\linewidth}{|*{4}{>{\centering \arraybackslash}X|}}\hline
%A&B&C&D\\ \hline
%80\,\%& 25\,\%& 15\,\%& 20\,\%\\ \hline
%\end{tabularx}
%\end{center}

\item %Quelle est la forme factorisée de $4x^2 - 25$ ?

%\begin{center}
%\begin{tabularx}{0.75\linewidth}{|*{4}{>{\centering \arraybackslash}X|}}\hline
%A			&B					&C					&D\\ \hline
%$(2x - 5)^2$&$(2x - 5)(2x + 5)$& $(4x - 5)(4x + 5)$& $(4x - 5)^2$\rule[-2mm]{0mm}{6mm}\\ \hline
%\end{tabularx}
%\end{center}
$4x^2 - 25 = (2x)^2 - 5^2$: cette différence de deux carrés se factorise en 

$(2x + 5)(2x - 5) = (2x - 5)(2x + 5)$ : réponse B.
\end{enumerate}

\bigskip

