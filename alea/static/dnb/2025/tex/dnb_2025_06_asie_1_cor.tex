
\begin{enumerate}
	\item \textbf{Bonne réponse : } $\dfrac{3}{10}$.

	Il y a :\quad $4 + 6 + 7 + 3 = 20$\quad boules en tout dans l'urne.

	Comme elles sont indiscernables au toucher, on est en situation d'équiprobabilité, donc la probabilité est égale à : \quad $\dfrac{\text{nombre d'issues favorables}}{\text{nombre d'issues possibles}}$.

	Il y a 6 boules violettes, donc 6 issues favorables, et 20 boules en tout, donc 20 issues possibles. La probabilité est donc :\quad $\dfrac{6}{20} = \dfrac{3}{10}$.

	\item \textbf{Bonne réponse : } 0,70.

	En effet, prendre 70\,\% d'une quantité revient à multiplier cette quantité par le coefficient multiplicateur :\quad $\dfrac{70}{100} = 0,70$.

	\emph{Remarque :} les autres propositions sont fausses :
	\begin{itemize}
		\item $0,30 = 1 - \dfrac{70}{100}$ : c'est le coefficient multiplicateur d'une \textbf{baisse} de 70\,\%;
		\item $1,70 = 1 + \dfrac{70}{100}$ : c'est le coefficient multiplicateur d'une \textbf{hausse} de 70\,\%;
		\item $1,30 = 1 + \dfrac{30}{100}$ : c'est le coefficient multiplicateur d'une \textbf{hausse} de 30\,\%.
	\end{itemize}

	\item \textbf{Bonne réponse : } La moyenne de cette série est 13.

	En effet : \quad$\dfrac{7 + 18 + 12 + 13 + 15}{5} = 13$.

	\emph{Remarque :} les autres propositions sont fausses et peuvent correspondre à des erreurs classiques :
	\begin{itemize}
		\item L'étendue de la série n'est pas 8, c'est $18 - 7 = 11$. L'erreur faite ici, c'est de prendre la dernière valeur moins la première, quand la suite n'est pas rangée dans l'ordre croissant;
		\item la médiane n'est pas 12, c'est 13. 12 est bien la valeur centrale, mais, là encore, la série n'est pas rangée dans l'ordre croissant;
		\item La moyenne est 13, comme on l'a calculé plus haut. L'erreur ici, serait de faire le calcul en oubliant les parenthèses : \quad $(7 + 18 + 12 + 13 + 15)\div 5 = 13$

		mais \quad $7 + 18 + 12 + 13 + 15\div 5 = 53$.
	\end{itemize}

	\item \textbf{Bonne réponse :} $f(x) = -2x + 4$.

	$f$ étant une fonction affine, son expression est de la forme :\quad $f(x) = ax + b$.

	La courbe $\mathcal{C}_f$ coupe l'axe des ordonnées au point de coordonnées $(0 ; 4)$, donc l'ordonnée à l'origine est $b = 4$.

	De plus, quand on part d'un point de $\mathcal{C}_f$ (le point de coordonnées $(0 ; 4)$, par exemple) si on avance d'une unité en abscisse, alors, pour retomber sur $\mathcal{C}_f$, il faut évoluer de $-2$ unités verticalement (pour arriver sur le point de coordonnées $(1 ; 2)$). Le coefficient directeur $a = -2$.

	Ainsi, l'expression de $f$ est bien :\quad$f(x) = -2x + 4$.
\end{enumerate}

