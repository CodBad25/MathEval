
\medskip

Cet exercice est un questionnaire à choix multiple (QCM).

Pour chaque question, quatre réponses sont proposées. \textbf{Une seule réponse est exacte}.

Recopier sur la copie le numéro de la question et la réponse choisie. Aucune justification n'est demandée.

\begin{center}
\begin{tabularx}{\linewidth}{|m{5cm}|*{4}{>{\centering \arraybackslash}X|}}\hline
Questions&Réponse A& Réponse B& Réponse C& Réponse D\\ \hline
\textbf{1.} $(- 3)^2$ est égal à&$-9$& $-6$ &6& 9\\ \hline
\textbf{2.} La décomposition en produit de facteurs premiers du nombre 360 est&$2^3 \times 9 \times 5$&$8 \times 3^2 \times 5$&$2^3 \times 3^2 \times 7$&$2^3 \times 3^2 \times 5$\\ \hline
\textbf{3.} Un rectangle d'aire $135$ cm$^2$ a pour largeur 3 cm. Combien mesure sa longueur ?&15 cm&45 cm&132 cm&405 cm\\ \hline
\textbf{4.} Quelle expression littérale correspond à la longueur du segment [BG] ?

\psset{unit=1cm,arrowsize=2pt 3}
\begin{pspicture}(4.8,1.4)
%\psgrid
\psline(0,0.7)(4.7,0.7)\psline(0.1,0.6)(0.1,0.8)\psline(1.3,0.6)(1.3,0.8)
\psline(2.4,0.6)(2.4,0.8)\psline(4.6,0.6)(4.6,0.8)
\psline(0.5,0.6)(0.7,0.8)\psline(1.75,0.6)(1.95,0.8)
\psline{<->}(0.1,0.5)(1.3,0.5)\psline{<->}(2.4,0.5)(4.6,0.5)
\uput[u](0.1,0.7){B}\uput[u](1.3,0.7){D}\uput[u](2.4,0.7){E}\uput[u](4.6,0.7){G}
\uput[d](0.7,0.5){$x$}\uput[d](3.5,0.5){3}
\end{pspicture}&$3x^2$&$2x^2 + 3$&$5x$&$2x + 3$\\ \hline
\textbf{5.} Le rectangle ADCB est partagé en neuf rectangles identiques.

\psset{unit=1cm,arrowsize=2pt 3}
\begin{pspicture}(4.7,2.9)
%\psgrid
\psframe(0.3,0.3)(4.5,2.4)\psline(0.3,1)(4.5,1)\psline(0.3,1.7)(4.5,1.7)
\psline(1.7,0.3)(1.7,2.4)\psline(3.1,0.3)(3.1,2.4)
\uput[ul](0.3,2.4){\footnotesize A}\uput[u](1.7,2.4){\footnotesize G}\uput[u](3.1,2.4){\footnotesize I}\uput[ur](4.5,2.4){\footnotesize D}
\uput[l](0.3,1.7){\footnotesize E}\uput[ur](1.7,1.7){\footnotesize F}\uput[ur](3.1,1.7){\footnotesize H}\uput[r](4.5,1.7){\footnotesize J}
\uput[l](0.3,1){\footnotesize K}\uput[ur](1.7,1){\footnotesize L}\uput[ur](3.1,1){\footnotesize M}\uput[r](4.5,1){\footnotesize N}
\uput[dl](0.3,0.3){\footnotesize B}\uput[d](1.7,0.3){\footnotesize O}\uput[d](3.1,0.3){\footnotesize P}\uput[dr](4.5,0.3){\footnotesize C}
\end{pspicture}

L'image du rectangle GFHI par la translation qui transforme D en M est le rectangle&EKLF&
HMNJ&KBOL&MPCN\\ \hline
\end{tabularx}
\end{center}


