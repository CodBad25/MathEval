
\emph{Dans cet exercice, aucune justification n'est attendue}

\medskip

\textbf{Partie 1 : les motifs}

\begin{center}
\begin{tabularx}{\linewidth}{|*{3}{>{\centering \arraybackslash}X|}}\hline
Script 1&Script 2&Script 3\\ \hline
&&\\
\begin{scratch}[scale=0.8]
\initmoreblocks{définir \namemoreblocks{Motif 1}}
\blockpen{stylo en position d'écriture}
\blockrepeat{répéter \ovalnum{3} fois}
{
\blockmove{avancer de \ovalvariable{30} pas}
\blockmove{tourner \turnleft{} de \ovalnum{120} degrés}
}
\blockpen{relever le stylo}
\end{scratch}&
\begin{scratch}[scale=0.8]
\initmoreblocks{définir \namemoreblocks{Motif 2}}
\blockpen{stylo en position d'écriture}
\blockrepeat{répéter \ovalnum{6} fois}
{
\blockmove{avancer de \ovalvariable{30} pas}
\blockmove{tourner \turnleft{} de \ovalnum{60} degrés}
}
\blockpen{relever le stylo}
\end{scratch}&
\begin{scratch}[scale=0.8]
\initmoreblocks{définir \namemoreblocks{Motif 3}}
\blockpen{stylo en position d'écriture}
\blockrepeat{répéter \ovalnum{2} fois}
{\blockmove{avancer de \ovalvariable{30} pas}
\blocklook{Partie du script effacée}
\blocklook{(voir question 2)}
}
\blockpen{relever le stylo}
\end{scratch}\\ 
&&\\ \hline
\end{tabularx}
\rput(4.9,2.5){
\psellipse[fillstyle=solid,fillcolor=lightgray,linecolor=lightgray](-0.5,0)(1.6,0.8)
\psellipse[fillstyle=solid,fillcolor=lightgray,linecolor=lightgray](0.5,0)(1.6,0.8)}
\rput(4.9,2.5){
\parbox{3cm}{\centering{}Partie du\\script effacée\\(voir question 2)}}
\end{center}

\medskip

\begin{enumerate}
\item Les scripts 1 et 2 permettent chacun d'obtenir un des dessins ci-dessous. Associer chacun des scripts à son dessin.

\begin{center}
\begin{tabularx}{\linewidth}{|*{2}{>{\centering \arraybackslash}X|}}\hline
Dessin 1&Dessin 2\\ \hline
\psset{unit=0.9cm}
\begin{pspicture}(-1.5,-1.5)(1.5,1.3)
\pspolygon(1.5;0)(1.5;60)(1.5;120)(1.5;180)(1.5;240)(1.5;300)
\end{pspicture}&
\psset{unit=1cm}
\begin{pspicture}(-1.8,-1)(1.8,1.8)
\pspolygon(1.4;-30)(1.4;90)(1.4;210)
\end{pspicture}\\ \hline
\end{tabularx}
\end{center}


\begin{minipage}{0.55\linewidth}
\item Le script 3 permet d'obtenir le losange ci-contre.

La partie du script effacée contient les 3 instructions A, B et C ci-dessous.

Sur votre copie, recopier dans le bon ordre les instructions cachées.
%\end{enumerate}
\textbf{Chaque instruction ne doit être utilisée qu'une seule fois.}
\end{minipage}\hfill
\begin{minipage}{0.38\linewidth}
\psset{unit=1cm,arrowsize=2pt 3}
\begin{pspicture}(0,-1)(5.5,3.5)
\pspolygon(2,0)(5.2,0)(3.6,2.77)(0.4,2.77)
\psarc(5.2,0){0.4}{120}{180}\psarc(2,0){0.4}{0}{120}
\rput(0,0){Départ}\uput[d](3.6,0){30 pas} \uput[ur](4.4,1.38){30 pas}
\rput(4.5,0.4){$60\degres$}\rput(2.8,0.4){$120\degres$}
\psline{->}(0.7,0)(2,0)
\end{pspicture}
\end{minipage}

\begin{center}
%\begin{tabularx}{\linewidth}{|*{3}{>{\centering \arraybackslash}X|}}\hline
\begin{tabular}{|c|c|c|}
\hline
Instruction A &Instruction B& Instruction C\\
 \hline
 &&\\
\begin{scratch}
\blockmove{tourner \turnleft{} de \ovalnum{60} degrés}
\end{scratch}
&
\begin{scratch}\blockmove{tourner \turnleft{} de \ovalnum{120} degrés}
\end{scratch}
&
\begin{scratch}\blockmove{avancer de \ovalnum{30} pas}
\end{scratch}\\
&&\\
 \hline
\end{tabular}
%\end{tabularx}
\end{center}
\end{enumerate}


\textbf{Partie 2 : le script principal}
%\setdefaultscratch{scale=0.8}

\medskip

\begin{tikzpicture}
  \node (algo) {%
    \begin{scratch}[scale=0.9]
      \blockinit{Quand \greenflag est cliqué}
      \blockmove{aller à x: \ovalnum{- 200} y: \ovalnum{0}}
      \blockpen{effacer tout}
      \blockmove{s'orienter à \ovalnum{90}}
      \blockvariable{mettre \ovalvariable{Motif} à nombre aléatoire entre \ovalnum{1} et \ovalnum{3}}
      \blockifelse{si \booloperator{\ovalmove{Motif} = \ovalnum{3}} alors}
      {\blockrepeat{répéter \ovalnum{6} fois}
        {
          \blocklook{Motif 3}
          \blockmove{avancer de \ovalnum{60} pas}
        }
        \blocklook{dire \ovalnum{Voici le dessin !}}
      }
      {%\blocklook{sinon}
        \blocklook{dire \ovalnum{Perdu !}}}
    \end{scratch}
  };
  \node[right=of algo] {
    \begin{tblr}{width=0.3\linewidth,colspec={X[c,1]},%
      hline{1,2,4,6},vlines,stretch=2,
      row{1}={font={\bfseries}},
      row{3,5}={font={\small}}}
      Rappels                                                                                \\
      \begin{scratch}[scale=0.8]
        \blockvariable{nombre aléatoire entre \ovalnum{1} et \ovalnum{3}}
      \end{scratch} \\
      donne un nombre entier au hasard parmi 1 ; 2 et 3.                                     \\
      \begin{scratch}[scale=0.8]
        \blockmove{s'orienter à \ovalnum{90}}
      \end{scratch}                                         \\
      oriente le lutin horizontalement vers la droite.
    \end{tblr}
  };
\end{tikzpicture}
\medskip

\begin{enumerate}[start=3]
\item Quelles sont les coordonnées du point de départ du lutin ?
\item Parmi les 5 captures d'écran proposées ci-dessous, seules deux sont possibles. Lesquelles?

\begin{center}
\small
%\begin{tabularx}{\linewidth}{|l|X|}
\begin{tabular}{|l|m{9.7cm}|}
\hline
Capture d'écran \no 1&\psset{unit=0.8cm}
\def\losan{\pspolygon(0,0)(1,0)(0.5,0.866)(-0.5,0.866)}
\begin{pspicture}(-1,0)(5,1.5)
\multido{\n=0+1}{6}{\rput(\n,0){\losan}}
\rput(2.5,1.25){Voici le dessin !}
\end{pspicture}
\\ \hline
Capture d'écran \no 2&\psset{unit=0.8cm}
\def\losan{\pspolygon(0,0)(1,0)(0.5,0.866)(-0.5,0.866)}
\begin{pspicture}(-1,0)(5,1.5)
\multido{\n=0+2}{6}{\rput(\n,0){\losan}}
\rput(2.5,1.25){Voici le dessin !}
\end{pspicture}
\\ \hline
Capture d'écran \no 3&\psset{unit=0.8cm}
\def\losan{\pspolygon(0,0)(1,0)(0.5,0.866)(-0.5,0.866)}
\begin{pspicture}(-1,0)(5,1.5)
%\multido{\n=0+1}{5}{\rput(\n,0){\losan}}
\rput(2.5,1.25){Perdu !}
\end{pspicture}
\\ \hline
Capture d'écran \no 4&\psset{unit=0.8cm}
\def\losan{\pspolygon(0,0)(1,0)(0.5,0.866)(-0.5,0.866)}
\begin{pspicture}(-1,0)(5,1.5)
\multido{\n=0+2}{3}{\rput(\n,0){\losan}}
\rput(2.5,1.25){Voici le dessin !}
\end{pspicture}
\\ \hline
Capture d'écran \no 5&\psset{unit=0.8cm}
\def\losan{\pspolygon(1,0)(0.5,0.866)(-0.5,0.866)(0,0)}
\begin{pspicture}(-2.5,-2)(4,2)
\rput(4.5,0){Voici le dessin !}
\multido{\n=-60+60,\na=0+60}{6}{\rput{\n}(1.;\na){\losan}}
\end{pspicture}
\\ \hline
\end{tabular}
%\end{tabularx}
\end{center}


\item On clique sur le drapeau vert, et on observe le message affiché.

Quelle est la probabilité que le message affiché soit \og Voici le dessin! \fg{} ?
\item On lance de nouveau le programme 100 fois et on regroupe les résultats obtenus dans le tableau suivant:

\begin{center}
\begin{tabularx}{\linewidth}{|*{3}{>{\centering \arraybackslash}X|}}\hline
Message du lutin&\og Voici le dessin! \fg{}& \og Perdu! \fg{}\\ \hline
Effectif&40&60\\ \hline
\end{tabularx}
\end{center}

	\begin{enumerate}
		\item Calculer la fréquence de l'affichage \og Voici le dessin! \fg{}.
		\item Pourquoi ce résultat est-il différent de celui obtenu à la question 5 ?
	\end{enumerate}
\end{enumerate}
