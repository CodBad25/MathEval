
\begin{enumerate}
\item Si on choisit 5, on a :
	\begin{itemize}[label=\textbullet]
		\item à gauche : \quad $5 + 4 = 9$ \quad et à droite :\quad $5 - 2 = 3$;
		\item en multipliant : \quad $9 \times 3 = 27$;
		\item en soustrayant le carré de 5 :\quad $27 - 5^2 = 27 - 25 = 2$.
	\end{itemize}
	
Avec 5 comme nombre de départ, on a bien 2 comme résultat final.

\item 
	\begin{enumerate}
		\item \textbf{Bonne réponse :} $(x + 4)(x - 2) - x^2$, expression C.

En effet, si on note $x$ le nombre choisi, on a :
\begin{itemize}[label=\textbullet]
\item à gauche : \quad $x + 4$ \quad et à droite :\quad $x - 2$;
\item en multipliant : \quad $(x + 4)(x - 2)$;
\item en soustrayant le carré de $x$ :\quad $(x + 4)(x - 2) - x^2$.
\end{itemize}

Dans l'expression A, on oublie les parenthèses, dans l'expression D, on confond le carré de $x$ avec le double de $x$, et dans l'expression B, on cumule les deux erreurs des expressions A et D.

\item Développons notre expression :

$\begin{aligned}
(x + 4)(x - 2) - x^2 &= x^2 - 2x + 4x - 8 - x^2\\
	& = x^2 - x^2 + (4-2)x - 8\\
	& = 2x - 8\\
 \end{aligned}$

On a bien le résultat final égal à $2x - 8$, sous sa forme développée et réduite.
\end{enumerate}

\item 
	\begin{enumerate}
		\item La représentation \no 1 ne convient pas, car la fonction $f$ a une expression de la forme $f(x) = ax + b$, c'est donc une fonction affine, et donc, sa représentation graphique est une droite : la représentation \no 1 n'est pas une droite, elle ne convient pas.

La représentation \no 2 ne convient pas non plus, car le coefficient directeur de $f$ est $2$, qui est positif. Cela signifie que, si on part d'un point qui est sur la représentation de $f$, et que l'on avance d'une unité en abscisse, alors il faut évoluer de $+2$, soit augmenter de 2 unités en ordonnées : c'est ce qui se passe pour la représentation \no 3, mais la représentation \no 2, il faudrait \textbf{diminuer} de 2 unités en ordonnées, c'est pour cela que la représentation \no 2 ne convient pas.

		\item La représentation \no 3 passe par le point de coordonnées $(4 ; 0)$, donc l'image de 4 par la fonction $f$ est 0.
	\end{enumerate}
\item Si on veut que le résultat final soit égal à 100, et que l'on cherche le nombre à choisir, cela revient à résoudre l'équation $f(x) = 100$.

$\begin{aligned}
	f(x) = 100 &\iff 2x - 8 = 100\\
	&\iff 2x = 108\\
	&\iff x = \dfrac{108}{2}\\
	&\iff x = 54
\end{aligned}$

Pour obtenir 100 comme résultat final, il faut avoir choisi 54 comme nombre de départ.
\end{enumerate}

\bigskip

