
On considère les deux programmes de calcul suivants :

\begin{center}
\begin{tabularx}{\linewidth}{m{6.cm}|X}
\textbf{Programme A}&\textbf{Programme B}\\
\begin{itemize}[label= $\bullet~~$]
\item Choisir un nombre
\item Multiplier par 3
\item Ajouter 15
\item Diviser par 3
\item Soustraire le nombre de départ
\end{itemize}&\psset{unit=0.825cm,arrowsize=2pt 3}
\vspace*{-2cm}
\begin{pspicture}(-4,0)(4,6)
\footnotesize
\rput(0,5){Choisir un nombre}\psframe(-1.9,5.4)(1.9,4.6)
\psline{->}(-1.9,5)(-2.6,5)(-2.6,4.1)
\rput(-2.6,3.7){Soustraire 1}\psframe(-3.8,4.1)(-1.4,3.3)\psline{->}(1.9,5)(2.6,5)(2.6,4.1)
\rput(2.6,3.7){Soustraire 6}\psframe(1.4,4.1)(3.8,3.3)
\psline{->}(-1.4,3.7)(-0.6,3.7)(-0.6,2.3)\psline{->}(1.4,3.7)(0.6,3.7)(0.6,2.3)
\rput(0,1.9){Multiplier les deux résultats obtenus}\psframe(-3.3,2.3)(3.3,1.5)
\psline{->}(0,1.5)(0,0.8)
\rput(0,0.5){Ajouter 5}\psframe(-1,0.8)(1,0.2)
%\psframe
\end{pspicture}
\end{tabularx}
\end{center}

\begin{enumerate}
\item Montrer que, lorsque le nombre choisi est 4, le résultat obtenu avec le programme A est 5.
\item Montrer que, lorsque le nombre choisi est $- 2$, le résultat obtenu avec le programme A est 5.
\item Justifier que l'affirmation suivante est vraie :

\begin{center}\og Le programme A donne toujours le même résultat. \fg\end{center}

\item Lorsque le nombre choisi est 10, quel résultat obtient-on avec le programme B ?
\item Il existe exactement deux nombres pour lesquels les programmes A et B fournissent à chaque fois des résultats identiques.

Quels sont ces deux nombres?
\end{enumerate}


