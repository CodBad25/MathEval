
\medskip

On considère le programme de calcul suivant :

\begin{center}
\begin{tabular}{|l|}\hline
$\bullet~~$Choisir un nombre\\
$\bullet~~$Multiplier le nombre choisi par $-2$\\
$\bullet~~$Ajouter 4 au résultat\\
$\bullet~~$Multiplier le résultat obtenu par 4\\ \hline
\end{tabular}
\end{center}

\smallskip

\begin{enumerate}
\item Montrer que si l'on choisit 1 comme nombre de départ dans le programme, le résultat obtenu est 8.
\item Quel est le résultat si le nombre de départ est $-2$ ?
\item Si l'on note $x$ le nombre de départ, montrer que le résultat peut s'écrire $-8x + 16$.
\item 
	\begin{enumerate}
		\item Résoudre l'équation $-8x + 16 = 4$.
		\item En déduire le nombre de départ qu'il faut choisir pour obtenir 4 comme résultat.
	\end{enumerate}
\item Parmi les trois représentations graphiques ci-dessous, quelle est celle qui représente la fonction $f$ définie par $f(x) =-8x + 16$ ? Expliquer la démarche.

\begin{center}
\begin{tabularx}{\linewidth}{|*{3}{>{\centering \arraybackslash}X|}}\hline
Représentation graphique 1& Représentation graphique 2& Représentation graphique 3\\ \hline
\psset{xunit=1cm,yunit=0.25cm,arrowsize=2pt 3,comma=true}
\begin{pspicture*}(-2.8,-2.25)(0.5,19)
\multido{\n=-2.5+0.5}{8}{\psline[linewidth=0.2pt](\n,-2)(\n,19)}
\multido{\n=-2+2}{11}{\psline[linewidth=0.2pt](-2.5,\n)(0.5,\n,)}
\psaxes[linewidth=1.25pt,Dx=0.5,Dy=2,labelFontSize=\scriptstyle](0,0)(-2.5,-2)(0.5,19)
\psplot[plotpoints=2000,linewidth=1.25pt]{-2.5}{0.5}{8 x mul 16  add}
\end{pspicture*}&
\psset{unit=1cm,yunit=0.25cm,arrowsize=2pt 3,comma}
\begin{pspicture*}(-1.8,-2.25)(1.5,19)
\multido{\n=-1.5+0.5}{7}{\psline[linewidth=0.2pt](\n,-2)(\n,19)}
\multido{\n=-2+2}{11}{\psline[linewidth=0.2pt](-1.5,\n)(1.5,\n,)}
\psaxes[linewidth=1.25pt,Dx=0.5,Dy=2,labelFontSize=\scriptstyle](0,0)(-1.5,-2)(1.5,19)
\psplot[plotpoints=2000,linewidth=1.25pt]{-1.5}{1.5}{8  8 x mul sub}
\end{pspicture*}&
\psset{unit=1cm,yunit=0.25cm,arrowsize=2pt 3,comma}
\begin{pspicture*}(-0.8,-2.25)(2.5,19)
\multido{\n=-1+0.5}{9}{\psline[linewidth=0.2pt](\n,-2)(\n,19)}
\multido{\n=-2+2}{11}{\psline[linewidth=0.2pt](-1,\n)(2.5,\n,)}
\psaxes[linewidth=1.25pt,Dx=0.5,Dy=2,labelFontSize=\scriptstyle](0,0)(-0.5,-2)(2.5,19)
\psplot[plotpoints=2000,linewidth=1.25pt]{-0.5}{2.5}{16 8 x mul sub}
\end{pspicture*}\\ \hline 
\end{tabularx}
\end{center}
\end{enumerate}

\bigskip

