
\emph{Dans cet exercice, les cinq situations sont indépendantes. Il est rappelé que chaque réponse doit être justifiée sauf indication contraire.}

\medskip
\begin{itemize}[label=$\bullet~~$]
\item \textbf{Situation 1}

%Dans une urne de $40$ boules indiscernables au toucher, $5$ sont rouges, $20$ sont vertes et $15$ sont blanches. L'expérience consiste à tirer au hasard une boule de l'urne et à noter sa couleur.
La probabilité est égale à $\dfrac{20}{40} = \dfrac12 = 0,5$.

%\smallskip
%
%Calculer la probabilité d'obtenir une boule verte.
\item \textbf{Situation 2}

%Décomposer en produit de facteurs premiers le nombre \np{1050}.
$\np{1050} = 105 \times 10 = 5 \times 21 \times 2 \times 5 = 5\times 3\times 7\times 2 \times 5 = 2\times 3\times 5^2 \times 7$.
%\emph{Aucune justification n'est attendue}.
\item \textbf{Situation 3}

%Un article coûte $25$~\euro. Calculer son prix après une augmentation de $14\,\%$.
Augmenter de 14\,\%, c'est multiplier par $1 + \dfrac{14}{100} = 1 + 0,14 = 1,14$ et 

$25 \times 1,14 = \dfrac{1,14 \times 100}{4} = \dfrac{114}{4} = \dfrac{57}{2} = 28,5$.

Le nouveau prix est 28,50~\euro.
\item \textbf{Situation 4}

%\begin{tabularx}{\linewidth}{X|X}
%Le polygone 2 est un agrandissement du polygone 1.
%
%Le coefficient de cet agrandissement est $2,5$.
%
%L'aire du polygone 1 est égale à $7,5$~cm$^2$.
%
%Calculer l'aire du polygone 2.&\emph{La figure ci-dessous n'est pas à l'échelle}.
%
%\begin{center}
%\psset{unit=0.7cm}
%\begin{pspicture}(0,-1)(6,4)
%\pspolygon(0,0)(3,0)(3,2)(1,3)(0,2)
%\pspolygon(4.7,0.7)(5.7,0.7)(5.7,1.4)(5.1,1.7)(4.7,1.4)
%\rput(1.5,-1){Polygone 2}\rput(5.2,-1){Polygone 1}
%\end{pspicture}
%\end{center}
%\end{tabularx}
Si les longueurs sont multipliées par $k$, les aires le sont par $k^2$, soit ici $2,5^2 = 6,25$.

L'aire du polygone 2 est donc $7,5 \times 6,25 = 46,875~\left(\text{cm}^2\right)$.
\item \textbf{Situation 5}

\medskip

%Dans une classe de 3\up{e} on note la répartition des tailles des élèves dans le tableau suivant:
%
%\begin{tabularx}{\linewidth}{|m{1.25cm}|*{8}{>{\centering \arraybackslash}X|}}\hline
%\textbf{Taille (en cm)}&152&157& 160& 162& 165&170& 174& 180\\ \hline
%\textbf{ Effectif}& 2& 4& 2& 5& 2&4&6&5\\ \hline
%\end{tabularx}
%
%\medskip

\begin{enumerate}
\item %Quelle est la moyenne des tailles des élèves de cette classe ?
Si $\overline{t}$ est la taille moyenne, alors :

$\overline{t} = \dfrac{2\times 152\times 4 \times 157 + 2 \times 160 + 5 \times 162 + 2 \times 165 + 4 \times 170 + 6 \times 174 + 5 \times 180}{2 + 4 + 2 + 5 + 2 + 4 + 6 + 5} = \dfrac{\np{5016}}{30} = 167,2$~(cm).
\item %Quelle est la médiane des tailles des élèves de cette classe ?
Dans l'ordre croissant la 15\up{e} taille est 165~cm et la 16\up{e}, \, 170~(cm).
Toute valeur entre 165 cm et 170 cm peut être prise comme médiane de cette série statistique.
\end{enumerate}
\end{itemize}



