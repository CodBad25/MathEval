
\medskip

\textbf{Partie A :} Le programme de Zoé

\begin{center}
\begin{tabular}{|l|}\hline
$\bullet~$Choisir un nombre\\
$\bullet~$Soustraire 4\\
$\bullet~$Multiplier par 2\\
$\bullet~$Ajouter 8.\\ \hline
\end{tabular}
\end{center}

\medskip

\begin{enumerate}
\item $10 \longmapsto 6 \longmapsto 12 \longmapsto 20$.
\item De même en partant de $- 7$ : $- 7  \longmapsto -11  \longmapsto -22  \longmapsto - 14$.

\item En partant du nombre $a$ : $a  \longmapsto a - 4  \longmapsto 2(a	 - 4) = 2a - 8  \longmapsto 2a$ : on obtient effectivement le double du nombre de départ.
\end{enumerate}

\medskip

\textbf{Partie B : Le programme de Fred}

\medskip

\begin{enumerate}[resume]
\item 
On obtient $x  \longmapsto 4x  \longmapsto 4x + 10  \longmapsto 5(4x + 10) = 5 \times 4x + 5 \times 10 = 20x + 50$.

\item Il faut trouver $x$ tel que $20x + 50 = 75$, soit en ajoutant $- 50$ à chaque membre : $20x = 25$ et en multipliant chaque membre par $\dfrac{1}{20}$, d'où $x = 25 \times \dfrac{1}{20} = \dfrac{25}{20} = \dfrac{5}{4} = 1,25$.

\item IL faut écrire 
\begin{scratch}\blockvariable{mettre \selectmenu{résultat} à
	\ovaloperator{\ovalvariable{résultat} - \ovalnum{50}}}
	\end{scratch}
\end{enumerate}

\bigskip

