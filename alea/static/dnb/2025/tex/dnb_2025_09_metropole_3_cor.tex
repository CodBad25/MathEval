
\medskip

\begin{minipage}{0.45\linewidth}
\begin{itemize}[label=$\bullet~$]
\item ABC un triangle rectangle en B ;
\item les points B, E et C sont alignés ainsi que
les points A, D, F et C ;
\item les droites (BD) et (EF) sont parallèles :
\item AB = 10 cm, BC = 7,5 cm, BE = 3 cm,

BD = 6 cm et CF = 2,7 cm.
\end{itemize}
\end{minipage}
\begin{minipage}{0.5\linewidth}
\psset{unit=0.9cm}
\begin{pspicture}(-0.5,-0.5)(8.5,6.5)
%\psgrid
\pspolygon(0,0)(7.8,0)(0,5.8)
\psline(2.78,3.75)
\psline(0,2.4)(1.63,4.6)
\uput[d](7.8,0){A}\uput[dl](0,0){B}\uput[ul](0,5.8){C}\uput[ur](2.78,3.75){D}
\uput[l](0,2.4){E}\uput[ur](1.63,4.6){F}
\psframe(0.3,0.3)
%\psarc(3.9,0){3.9}{0}{120}
%\psarc(0,4.1){1.7}{-90}{90}
\end{pspicture}
\end{minipage}

\medskip

\begin{enumerate}
\item
	\begin{enumerate}
		\item %Montrer que CE $= 4 ,5$ cm.
De BE + EC = BC, soit $3 + \text{EC} = 7,5$, on déduit que EC = CE $= 7,5 - 3 = 4,5$~(cm).
		\item %Démontrer que la longueur EF est égale à $3,6$ cm.
C, E et B d'une part, C, F et D sont alignés et les droites (EF) et (BD) sont parallèles : d'après le théorème de Thalès :

$\dfrac{\text{CE}}{\text{CB}} = \dfrac{\text{CF}}{\text{CD}}$, soit $\dfrac{4,5}{7,5} = \dfrac{\text{EF}}{6}$.

On en déduit que EF $ = 6 \times \dfrac{4,5}{7,5} = 3,6$~(cm).
	\end{enumerate}
\item %Démontrer que le triangle CEF est rectangle en F.
On a CF$^2 = 2,7^2 = 7,29$ ;

EF$^2 = 3,6^2 = 12,96$ ;

CE$^2 = 4,5^2 = 20,25$.

Or $7,29 + 12,96 = 20,25$ ou encore $\text{EF}^2 + \text{CE}^2 = \text{CE}^2$ : d'après la réciproque du théorème de Pythagore, cette égalité montre que EFC est un triangle rectangle en F.
\item
	\begin{enumerate}
		\item %Calculer la mesure de l'angle $\widehat{\text{BCA}}$. Arrondir au degré.
Dans le triangle ABC rectangle en B, on a $\tan \widehat{\text{BCA}} = \dfrac{\text{AB}}{\text{BC}} = \dfrac{10}{7,5} \approx 1,333$.

La calculatrice donne $\widehat{\text{BCA}} \approx 53,12$~(degrés), soit environ $53\degres$ au degré près.
		\item %Les triangles ABC et CEF sont-ils semblables ?
Les triangles ABC et EFC ont deux angles de même mesure : les angles droits en B et respectivement et l'angle $\widehat{\text{C}}$ : leurs troisièmes angles ont donc même mesure et ces deux triangles sont semblables.
	\end{enumerate}
\end{enumerate}

\bigskip

