
Dans cet exercice, aucune justification n'est attendue.

\textbf{Rappel:}

L'instruction \begin{scratch}\blockmove{s'orienter à 90}\end{scratch} signifie que le lutin se dirige vers la droite.

\bigskip

\textbf{PARTIE A :}

\medskip

\begin{minipage}{0.67\linewidth}Un élève souhaite tracer un hexagone à partir de 6 triangles équilatéraux comme sur la figure ci-contre.
\end{minipage}\hfill
\begin{minipage}{0.3\linewidth}
\begin{pspicture}(-2,-1.5)(2,1.5)
\psline[linewidth=1pt](-1.6,0)(1.6,0)\psline[linewidth=1pt](-0.8,-1.384)(0.8,1.384)\psline[linewidth=1pt](-0.8,1.384)(0.8,-1.384)
\psline[linewidth=1pt](-1.6,0)(-0.8,1.384)\psline[linewidth=1pt](-1.6,0)(-0.8,-1.384)
\psline[linewidth=1pt](1.6,0)(0.8,1.384)\psline[linewidth=1pt](1.6,0)(0.8,-1.384)
\psline[linewidth=1pt](-0.8,1.384)(0.8,1.384)\psline[linewidth=1pt](-0.8,-1.384)(0.8,-1.384)
\end{pspicture}
\end{minipage}

Pour cela, il commence par écrire le script ci-dessous du motif \og triangle équilatéral\fg :


\begin{minipage}[]{0.6\linewidth}
\begin{scratch}[num blocks]
\blockinit{définir triangle équilatéral}
\blockrepeat{répéter \ovalnum{} fois}
{
\blockmove{avancer de \ovalnum{} pas}
\blockmove{tourner de \turnleft{} de \ovalnum{} degrés}
}
\end{scratch}
\end{minipage}
\hspace{3em}
\begin{minipage}[]{0.45\linewidth}
\psset{unit=1.3cm}
\begin{pspicture}(-1,-1)(2,2)
\psline[linewidth=0.5pt](0,0)(0.8,1.384)\psline[linewidth=0.5pt](1.6,0)(0.8,1.384)\psline[linewidth=0.5pt](0,0)(1.6,0)
\end{pspicture}
\end{minipage}

\begin{enumerate}
\item  Compléter et recopier sur la copie les lignes 2, 3 et 4 du script pour que le lutin dessine un triangle équilatéral de côté 50 pas.
\item Cet élève teste les deux programmes A et B. Il obtient les deux dessins ci-dessous.\\
Quel programme permet de tracer l'hexagone souhaité ?
\end{enumerate}
\begin{minipage}{10cm}

\begin{tabular}{|c|c|}\hline
\multicolumn{2}{|c|}{Programmes testés}\\\hline
Programme A & Programme B\\\
\begin{minipage}[]{4cm}
\begin{scratch}[scale=0.8]
\blockinit{quand la touche A est pressée}
\blockmove{aller à x: \ovalmove{ 0} y:  \ovalmove{0}}
\blockmove{s'orienter à \ovalnum{90}}
\blockpen{effacer tout}
\blockpen{stylo en position d’écriture}
\blockrepeat{répéter \ovalnum{6} fois}
{
\blockmoreblocks{triangle équilatéral}
\blockmove{tourner \turnleft{} de \ovalnum{60} degrés}
}
\end{scratch}
\end{minipage}
&
\begin{minipage}[]{4cm}
\begin{scratch}[scale=0.8]
\blockinit{quand la touche B est pressée}
\blockmove{aller à x: \ovalmove{ 0} y:  \ovalmove{0}}
\blockmove{s'orienter à \ovalnum{90}}
\blockpen{effacer tout}
\blockpen{stylo en position d’écriture}
\blockrepeat{répéter \ovalnum{6} fois}
{
\blockmoreblocks{triangle équilatéral}
\blockmove{tourner \turnleft{} de \ovalnum{120} degrés}
}
\end{scratch}
\end{minipage}\\
\hline
\end{tabular}
\end{minipage}
\hfill
\begin{minipage}[]{4cm}
Dessins obtenus

\begin{pspicture}(-2,-1.5)(2,1.5)
\psline[linewidth=0.5pt](-1.6,0)(1.6,0)\psline[linewidth=0.5pt](-0.8,-1.384)(0.8,1.384)\psline[linewidth=0.5pt](-0.8,1.384)(0.8,-1.384)
\psline[linewidth=0.5pt](-1.6,0)(-0.8,1.384)\psline[linewidth=0.5pt](-1.6,0)(-0.8,-1.384)
\psline[linewidth=0.5pt](1.6,0)(0.8,1.384)\psline[linewidth=0.5pt](1.6,0)(0.8,-1.384)
\psline[linewidth=0.5pt](-0.8,1.384)(0.8,1.384)\psline[linewidth=0.5pt](-0.8,-1.384)(0.8,-1.384)
\end{pspicture}

\begin{pspicture}(-2,-1.5)(2,1.5)
\psline[linewidth=0.5pt](-1.6,0)(1.6,0)\psline[linewidth=0.5pt](-0.8,-1.384)(0.8,1.384)\psline[linewidth=0.5pt](-0.8,1.384)(0.8,-1.384)
\psline[linewidth=0.5pt](-1.6,0)(-0.8,1.384)
%\psline[linewidth=0.5pt](-1.6,0)(-0.8,-1.384)
\psline[linewidth=0.5pt](1.6,0)(0.8,1.384)%\psline[linewidth=0.5pt](1.6,0)(0.8,-1.384)
\psline[linewidth=0.5pt](-0.8,-1.384)(0.8,-1.384)
\end{pspicture}
\end{minipage}

\vspace{1.5cm}

\textbf{PARTIE B:}

Un autre élève souhaite tracer un hexagone régulier de 50 pas de côté comme sur la figure ci-dessous:

\medskip

\begin{minipage}[]{5cm}
figure obtenue

\begin{pspicture}(-3,-1.5)(1,1.5)
%\psgrid
\psline[linewidth=0.5pt](-1.6,0)(-0.8,1.384)
\psline[linewidth=0.5pt](-1.6,0)(-0.8,-1.384)
\psline[linewidth=0.5pt](1.6,0)(0.8,1.384)
\psline[linewidth=0.5pt](1.6,0)(0.8,-1.384)
\psline[linewidth=0.5pt](-0.8,1.384)(0.8,1.384)
\psline[linewidth=0.5pt](-0.8,-1.384)(0.8,-1.384)
\psline[linewidth=1pt]{->}(-2,-1.384)(-0.8,-1.4)
\uput[l](-1,-1){\footnotesize Point de} 
\uput[l](-0.5,-1.7){\footnotesize coordonnées (0,0)}
\end{pspicture}

\medskip 

Informations sur les hexagones 
\begin{pspicture}(-2,-1.5)(1,1.5)
%\psline[linewidth=0.5pt](-1.6,0)(1.6,0)\psline[linewidth=0.5pt](-0.8,-1.384)(0.8,1.384)\psline[linewidth=0.5pt](-0.8,1.384)(0.8,-1.384)
\psline[linewidth=0.5pt](-1.6,0)(-0.8,1.384)\psline[linewidth=0.5pt](-1.6,0)(-0.8,-1.384)
\psline[linewidth=0.5pt](1.6,0)(0.8,1.384)\psline[linewidth=0.5pt](1.6,0)(0.8,-1.384)
\psline[linewidth=0.5pt](-0.8,1.384)(0.8,1.384)\psline[linewidth=0.5pt](-0.8,-1.384)(0.8,-1.384)
\uput[ul](0.9,-1.15){$120^o$}
\end{pspicture}

\end{minipage}
\hfill
\begin{minipage}[]{5cm}
Il a écrit le programme suivant :

\begin{scratch}[scale=0.8]
\blockinit{quand \greenflag est cliqué}
\blockmove{aller à x: \ovalmove{ 0} y:  \ovalmove{0}}
\blockmove{s'orienter à \ovalnum{90}}
\blockpen{stylo en position d’écriture}
\blockpen{effacer tout}
\blockrepeat{répéter \ovalnum{A} fois}
{
\blockspace
}
\end{scratch}

\end{minipage}

\begin{enumerate}
\item Sur la copie, recopier le bloc \og répéter\fg en remplaçant A par sa valeur et en le complétant avec 2 instructions choisies parmi les 6 instructions proposées ci-dessous:
\end{enumerate}

\begin{scratch}
\blockmove{avancer de \ovalnum{50} pas} \end{scratch}\hfill \begin{scratch}\blockmove{tourner \turnright{} de \ovalnum{120} degrés}\end{scratch}\hfill \begin{scratch} \blockmove{tourner \turnleft{} de \ovalnum{60} degrés}\end{scratch}

\medskip

\begin{scratch}\blockmove{avancer de \ovalnum{5} pas}\end{scratch}\hfill \begin{scratch}\blockmove{tourner \turnleft{} de \ovalnum{120} degrés} \end{scratch} \hfill \begin{scratch}\blockmove{tourner \turnright{} de \ovalnum{60} degrés}\end{scratch}

