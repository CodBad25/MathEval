
%\emph{La figure ci-dessous n'est pas en vraie grandeur}.
%
%\begin{center}
%\psset{unit=0.8cm}
%\begin{pspicture}(0,-3.4)(10,4.5)
%\pspolygon(0,0)(9,0)(9,-2.7)(0,4)%EBCD
%\psline(0,4)(2.3,0)%DM
%\psframe(0.4,0.4)\psframe(9,0)(8.6,-0.4)
%\psarc(2.3,0){0.6}{120}{180}
%\uput[ur](5.3,0){A}\uput[u](9,0){B}\uput[d](9,-2.7){C}
%\uput[u](0,4){D}\uput[d](0,0){E}\uput[d](2.3,0){M}
%\uput[r](9,-1.35){30 m}\uput[dl](7.2,-1.6){50 m}\uput[ur](2.6,2.1){70 m}
%\rput(1.3,0.5){$60\,\degres$}
%\end{pspicture}
%\end{center}
%
%On a les données suivantes:
%
%\medskip
%
%\begin{itemize}[label= \small $\bullet~~$]
%\item Les points A, B, E et M sont alignés
%\item Les points A, C et D sont alignés
%\item ADE est un triangle rectangle en E
%\item ABC est un triangle rectangle en B
%\item AD $= 70$ m
%\item BC $= 30$ m
%\item AC $= 50$ m 
%\item $\widehat{\text{DME}} = 60\degres$
%\end{itemize}

\medskip

\begin{enumerate}
\item %Calculer la longueur AB.
Dans le triangle ABC rectangle en B, le théorème de Pythagore permet d'écrire :

AC$^2 = \text{AB}^2 + \text{BC}^2$, soit $50^2 = \text{AB}^2 + 40^2$, d'où 

$\text{AB}^2 = 50^2 - 30^2 = (50 + 30)(50 - 30) = 80 \times 20 = \np{1600} = 40^2$.

Conclusion AB $= 40$~(m).
\item %Montrer que les droites (DE) et (BC) sont parallèles.
Les droites (DE) et (BC) sont parallèles car elles sont perpendiculaires à la droite (AB)
\item %Montrer que la longueur DE est égale à $42$~m.
$\bullet~$Les points B, A et E sont alignés ;

$\bullet~$Les points C, A et D sont alignés ;

$\bullet~$Les droites (DE) et (BC) sont parallèles ;

On a donc une configuration de Thalès qui permet d'écrire :

$\dfrac{\text{AB}}{\text{AE}} = \dfrac{\text{BC}}{\text{DE}} = \dfrac{\text{AC}}{\text{AD}}$.

En particulier $\dfrac{\text{AB}}{\text{AE}} = \dfrac{\text{BC}}{\text{DE}}$ soit $\dfrac{50}{70} = \dfrac{30}{\text{DE}}$, d'où $50\text{DE} = 30 \times 70$, soit DE $ = \dfrac{30 \times 70}{50} = 42$~(m).

\item %Montrer que la longueur EM est environ égale à $24,2$~m.
Le triangle DME rectangle en E a un angle en M de $60\degres$, donc en D de $30\degres$ : c'est un demi-triangle équilatéral et donc ME = $\dfrac 12$ DM.

On sait qu'alors DE $ = \text{DM} \dfrac{\sqrt{3}}{2}$ soit $42 = \text{DM} \dfrac{\sqrt{3}}{2}$ d'où DM $= \dfrac{84}{\sqrt{3}}$ et donc ME $= \dfrac{42}{\sqrt{3}} \approx 24,2$~(m).

(On peut aussi utiliser le théorème de Pythagore dans le triangle DME).
\item %En déduire l'aire du triangle AMD.
L'aire du triangle DME est donc égale à :

$\mathcal{A}(\text{DME}) = \dfrac{\text{DE} \times \text{EM}}{2} 
= \dfrac{42 \times \frac{42}{\sqrt{3}}}{2} \approx 509,3~\left(\text{m}^2\right)$.

En reprenant les égalités de Thalès on a $\dfrac{\text{AB}}{\text{AE}} = 
\dfrac{\text{BC}}{\text{DE}}$, soit $\dfrac{40}{\text{AE}} = \dfrac{\text{30}}{42}$,

d'où $30 \text{AE} = 40 \times 42$ et AE $ = \dfrac{40 \times 42}{30} = 56$~(m).

L'aire du triangle ADE est donc égale à :

$\mathcal{A}(\text{ADE}) = \dfrac{\text{AE} \times \text{DE}}{2} = \dfrac{42 \times 56}{2} = 21 \times 56 = \np{1176}~\left(\text{m}^2\right)$.

Finalement $\mathcal{A}(\text{DME}) = \mathcal{A}(\text{ADE}) - \mathcal{A}(\text{DME}) \approx \np{1176} - 509,2$, soit $\mathcal{A}(\text{DME}) \approx 666,8~\left(\text{m}^2\right)$.

\end{enumerate}

