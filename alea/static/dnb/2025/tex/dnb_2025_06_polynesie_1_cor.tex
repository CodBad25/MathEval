
\begin{enumerate}
\item On peut lire dans la cellule \textsf{D2} que l'effectif correspondant aux élèves de 12 ans est 8. Il y a donc 8 élèves de 12 ans inscrits à l'activité d'escalade.

\item Le nombre total, c'est l'effectif total, c'est donc la somme des différents effectifs.

On a \quad $N = 1 + 3 + 8 + 12 + 4 + 2 = 30$.

Il y a en tout 30 élèves inscrits à l'escalade.

\item Dans la cellule \textsf{H2}, on peut inscrire la formule : \quad \textsf{= SOMME(B2 : G2)},

ou bien, si on ne connaît pas la fonction somme :\quad \textsf{= B2 + C2 + D2 + E2 + F2 + G2}.

On rappelle que \textsf{B2 : G2} représente le \og bloc \fg de cellules qui va de \textsf{B2} (en haut à gauche) à \textsf{G2} (en bas à droite).

\item Les élèves qui ont 14 ans ou plus sont au nombre de $4 + 2 = 6$ (les 4 qui ont 14 ans et les 2 qui ont 15 ans).

Cela représente :\quad $\dfrac{6}{30} = \dfrac{1 \times 6}{5\times 6} = \dfrac{1}{5}$.

Le professeur a donc raison.

\item Calculons l'âge moyen des élèves inscrits :

$\overline{a} = \dfrac{10 \times 1 + 110 \times 3 + 12 \times 8 + 13 \times 12 + 14 \times 4 + 15 \times 2}{30} = \dfrac{381}{30} = 12,7$.

Comme $12,7 < 13$, on peut dire que la moyenne d'âge n'a pas augmenté, au contraire, elle a baissé légèrement. (de 0,3 ans, soit entre 3 et 4 mois).

\item S'il y a une hausse de 10~\% du nombre  d'inscrits, alors, l'année prochaine, il y aura : \quad $30 \times \left(1 + \dfrac{10}{100}\right) = 30 \times 1,1 = 33$ inscrits.
\end{enumerate}

\bigskip

