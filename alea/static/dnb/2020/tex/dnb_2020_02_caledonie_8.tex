
\medskip

On souhaite représenter 6 bassins rectangulaires à l'aide d'un logiciel de programmation comme sur la figure \no 1 ci-dessous :

\begin{center}
\psset{unit=1cm}
\begin{pspicture}(10,6)
\uput[l](3,1){Point de départ}
\rput(3,1){\large $\times$}
\rput(5,6){La figure n'est pas à l'échelle}
\multido{\n=3.00+1.25}{6}{\rput(\n,1){\psframe(1,4)}}
\psline{<->}(3,0.5)(10.25,0.5)
\uput[d](6.625,0.5){220 pixels}
\uput[d](5,0){Figure \no 1}
\end{pspicture}
\end{center}

\medskip

\begin{enumerate}
\item Compléter le script, ci-dessous, du bloc \og bassin \fg{} pour qu'il permette de tracer
un bassin rectangulaire de largeur $30$ pixels et de longueur $150$ pixels.

\begin{scratch}
\initmoreblocks{définir \namemoreblocks{bassin}}
	\blockpen{stylo en position d'écriture}
	\blockrepeat{répéter \ovalnum{\ldots} fois}
		{\blockmove{avancer de \ovalnum{30}}
		\blockmove{tourner \turnleft{} de \ovalnum{\ldots} degrés}
		\blockmove{avancer de \ovalnum{\ldots}}
		\blockmove{tourner \turnleft{} de \ovalnum{\ldots} degrés}
		}			
\end{scratch}

\item  Le script ci-dessous doit permettre d'obtenir la figure \no 1. Il utilise le bloc \og bassin \fg{} du script ci-dessus.

\medskip

\begin{multicols}{2}
\begin{center}
%\begin{tabularx}{0.8\linewidth}{*{2}{>{\centering \arraybackslash}X}}
\begin{scratch}
\blockinit{Quand \greenflag est cliqué}
\blockmove{s'orienter à \ovalnum{90} degrés}
\blockpen{effacer tout}
	\blockrepeat{répéter \ovalnum{6} fois}
		{\blocksound{bassin}
		\blockpen{relever le stylo}
		\blockmove{avancer de \ovalnum{?}}
		}
\end{scratch}
\end{center}

\columnbreak

~\vfill

\begin{center}
\textbf{Rappel :}

\begin{scratch}
\blockmove{s'orienter à \ovalnum{90} degrés}
\blockspace[0.5]
\blockmove{(0) à droite}
\blockmove{($-90$) à gauche}
\blockmove{(0) vers le haut}
\blockmove{(180) vers le bas}
\end{scratch}
%\end{tabularx}
\end{center} 

\vfill~

\end{multicols}

Sachant que la longueur totale de la figure \no 1 est de $220$ pixels, quelle valeur doit être placée à la dernière ligne dans la consigne \og avancer de \fg{} ? Justifier la réponse.

\emph{Dans cette question, toute trace de recherche, même incomplète ou non fructueuse, sera prise en compte dans l'évaluation.}
\end{enumerate}

