
\medskip

\textbf{Partie A :}

\medskip

Dans un bassin, l'aquaculteur relève la masse de 100 crevettes.

Il a regroupé les résultats obtenus dans le tableau suivant:

\begin{center}
\begin{tabularx}{\linewidth}{|c|l|*{8}{>{\centering \arraybackslash}X|}}\hline
			&\centering\arraybackslash A 					&B 	&C 	&D 	&E 	&F 	&G 	&H	&I\\ \hline
1			&Masse (en grammes) &18 &19 &21 &23 &25 &26 &28	&\\ \hline
2			&Effectif			&7 	&12 &19 &25 &14 &13 &10	&\\ \hline
\end{tabularx}
\end{center}

\smallskip


\begin{enumerate}
\item Dans la cellule I2 on saisit la formule $\fbox{=\text{SOMME(B2:H2)}}$. Quel nombre s'affiche dans cette cellule ?
\item  On choisit au hasard une crevette. Toutes les crevettes ont la même probabilité d'être choisies.
	\begin{enumerate}
		\item Quelle est la probabilité que la masse de la crevette soit de $21$ grammes ?
		\item Quelle est la probabilité que la masse de la crevette soit supérieure ou égale à $25$ grammes ?
	\end{enumerate}
\end{enumerate}

\bigskip

\textbf{Partie B :}

\medskip

Lors de la pêche, on relève la masse (en grammes) de quelques crevettes.

Voici la série de valeurs obtenues:

\[\text{20 -- 18 -- 17 -- 28 -- 28 -- 22 -- 24 -- 24 -- 22 -- 24}\]

\smallskip

\begin{enumerate}
\item Calculer la moyenne de cette série.
\item Calculer la médiane de cette série. Interpréter ce résultat.
\end{enumerate}

\vspace{0,5cm}

