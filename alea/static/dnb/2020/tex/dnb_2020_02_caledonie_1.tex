
\medskip

Pour chacune des affirmations suivantes, indiquer si elle est VRAIE ou FAUSSE et justifier la réponse.

\begin{center}
\newlength\savedwidth
\newcommand\whline{\noalign{\global\savedwidth\arrayrulewidth\global\arrayrulewidth 3pt}
\hline \noalign{\global\arrayrulewidth\savedwidth}}
\begin{tabularx}{\linewidth}{!{\vrule width 2pt}X!{\vrule width 2pt}}
\whline
\\
\textbf{Données:} $f$ est la fonction définie par $f(x) = 2(x - 3)$.\\
\textbf{Affirmation 1 }: L'image de 5 par la fonction $f$ est $4$.\\ 
\\
\whline
\\
\textbf{Données :} Le parc éolien de Prony est composé de 84 éoliennes. Chaque éolienne produit en moyenne \np{256000} Watts.\\
\textbf{Affirmation 2 }: Le parc éolien produit au total environ 21,5 mégawatts en moyenne.\\ \\
\whline
\\
\textbf{Données:} Sur la figure ci-dessous, les droites (AD) et (CB) sont sécantes en E.\\
\begin{center}
\psset{unit=1cm}
\begin{pspicture}(0,0.5)(6,5.6)
%\psgrid
\rput(0.6,5.5){On a :}
\rput(1,5){CE = 1,6 cm}
\rput(1,4.5){DE = 1,2 cm}
\rput(1,4){EA = 2,8 cm}
\rput(1,3.5){ES = 3,4 cm}
\pspolygon(5,5.5)(5.5,0.5)(3,4)(3.4,1.5)%ABCD
\uput[ur](5,5.5){A} \uput[dr](5.5,0.5){B} \uput[ul](3,4){C} \uput[d](3.4,1.5){D} 
\end{pspicture}
\end{center}\\
\textbf{Affirmation 3 }: Les droites (AB) et (CD) sont parallèles.\\ 
\\
\whline
\\
\textbf{Données:} Le pentagone ci-dessous est composé de 5 triangles.\\
\begin{center}
\psset{unit=1cm}
\begin{pspicture}(0,1)(8,5)
%\psgrid
\rput(1,5){On sait que :}
\rput(1,4){$\widehat{\text{CAB}} = \widehat{\text{BAF}} = \widehat{\text{FAE}} = \widehat{\text{EAD}} = \widehat{\text{DAC}}$}
\def\tri{\pspolygon(0,0)(1.4;0)(1.4;72)\rput(0.7;0){$\times$}\rput(0.7;72){$\times$}}
\multido{\n=-10+72}{5}{\rput{\n}(6,3){\tri}}
\rput(6,3){\uput[dl](0,0){A}}\rput(6,3){\uput[r](1.5;350){F}} \rput(6,3){\uput[ur](1.5;62){B}} \rput(6,3){\uput[ul](1.5;134){C}} \rput(6,3){\uput[dl](1.5;206){D}} \rput(6,3){\uput[dl](1.5;278){E}} 
\end{pspicture}
\end{center}\\
\textbf{Affirmation 4: } L'angle de la rotation de centre A qui transforme C en D dans le sens des aiguilles d'une montre est $60$\degres{}.\\ 
\\
\whline
\end{tabularx}
\end{center}

\vspace{0,5cm}

