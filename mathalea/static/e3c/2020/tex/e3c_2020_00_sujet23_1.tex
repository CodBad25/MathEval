
\medskip

Ce QCM comprend 5 questions.

Pour chacune des questions, une seule des quatre réponses proposées est correcte.

Les questions sont indépendantes.

Pour chaque question, indiquer le numéro de la question et recopier sur la copie la lettre
correspondante à la réponse choisie.

Aucune justification n’est demandée mais il peut être nécessaire d’effectuer des recherches au
brouillon pour aider à déterminer votre réponse.

Chaque réponse correcte rapporte 1 point. Une réponse incorrecte ou une question sans réponse
n’apporte ni ne retire de point.

\medskip

\textbf{Question 1}

\medskip 

On considère la fonction définie sur $\R$ par $f(x) = -x^2 - x + 6$. On admet que l’une
des quatre courbes ci-dessous représente la fonction $f$. Laquelle ?

\begin{tabularx}{\linewidth}{*{4}{X}}

\textbf{a.~~}  .

\psset{xunit=0.5cm,yunit=0.5cm,labelsep=0.1pt,labelFontSize=\scriptstyle,showorigin=false}
\begin{pspicture}(-3.5,-7)(3,2.6)
\multido{\n=-3+1}{6}{\psline[linewidth=1.45pt](\n,-7.2)(\n,2.14)}
\multido{\n=-7+1}{10}{\psline[linewidth=0.45pt](-3.4,\n)(2.1,\n)}
\psaxes[linewidth=0.95pt]{->}(0,0)(-3.54,-7)(2.3,2.1)
\uput[ur](0,0){O}
\def\Func{x 3 add x 2 sub mul }
\psplot[plotpoints=1000,linewidth=1.25pt,linecolor=blue]{-3.2}{2.1}{\Func}
\end{pspicture}
&
\textbf{b.~~} 

\psset{xunit=0.5cm,yunit=0.5cm,labelsep=0.1pt,labelFontSize=\scriptstyle,showorigin=false}
\begin{pspicture}(-2.5,-7.4)(3,2.5)
\multido{\n=-2+1}{6}{\psline[linewidth=0.45pt](\n,-7)(\n,2.14)}
\multido{\n=-7+1}{10}{\psline[linewidth=0.45pt](-2.2,\n)(3.,\n)}
\psaxes[linewidth=1pt]{->}(0,0)(-2.54,-7)(3.3,2.)
\uput[ur](0,0){O}
\def\Func{x 3 sub x 2 add mul }
\psplot[plotpoints=1000,linewidth=1.25pt,linecolor=green]{-2.1}{3.1}{\Func}
\end{pspicture}
&
\textbf{c.~~}

\psset{xunit=0.5cm,yunit=0.5cm,labelFontSize=\scriptstyle,showorigin=false,labelsep=0.1pt}
\begin{pspicture}(-3.4,-2)(2.5,7.4)
\multido{\n=-3+1}{6}{\psline[linewidth=0.45pt](\n,-2.2)(\n,7)}
\multido{\n=-2+1}{10}{\psline[linewidth=0.45pt](-3.4,\n)(2.1,\n)}
\psaxes[linewidth=0.95pt]{->}(0,0)(-3,-2.)(2.1,6.9)
\uput[ur](0,0){O}
\def\Func{x 3 add x 2 sub mul 1 neg mul }
\psplot[plotpoints=1000,linewidth=1.25pt,linecolor=red]{-3.2}{2.1}{\Func}
\end{pspicture}
&
\textbf{d.~~} 

\psset{xunit=0.5cm,yunit=0.5cm,labelFontSize=\scriptstyle, labelsep=0.1pt,showorigin=false}
\begin{pspicture}(-2.5,-2.6)(3,7.4)
\multido{\n=-2+1}{6}{\psline[linewidth=0.45pt](\n,-2.2)(\n,7)}
\multido{\n=-2+1}{10}{\psline[linewidth=0.45pt](-2.4,\n)(3.1,\n)}
\psaxes[linewidth=0.95pt]{->}(0,0)(-2.54,-2.2)(3.3,6.6)\uput[ur](0,0){O}
\def\Func{x 3 sub x 2 add  mul 1 neg mul}
\psplot[plotpoints=1000,linewidth=1.25pt,linecolor=gray]{-2.2}{3.1}{\Func}
\end{pspicture}
\end{tabularx}
\medskip

\textbf{Question 2}

\medskip 

On pose pour tout réel $x$ : $A(x) = \e^{2x}$. On a alors, pour tout $x \in \R$ :

\medskip

\begin{tabularx}{\linewidth}{*{4}{X}}
\textbf{a.~~}$A(x) = 2\e^{x}$ &\textbf{b.~~}$A(x) = \e^{x^2} $  &\textbf{c.~~}$A(x) = \e^x + \e^2$& \textbf{d.~~}$A(x) = \left(\e^x\right)^2$.
\end{tabularx}

\medskip

\textbf{Question 3}

\medskip 


Le plan est muni d’un repère orthonormé.

Les droites d’équations $2x + y + 1 = 0$ et $3x -2y + 5 = 0$

\medskip

\begin{tabularx}{\linewidth}{*{2}{X}}
\textbf{a.~~}sont sécantes en $A(1~;~1)$. & 
\textbf{b.~~}sont sécantes en $B(1~;~-1)$.\\
\textbf{c.~~}sont sécantes en $C(-1~;~1)$.& 
\textbf{d.~~}ne sont pas sécantes. 
\end{tabularx}

\medskip

\textbf{Question 4}

\medskip 

Le plan est muni d’un repère orthonormé.

Les droites d’équations $x + 3y-5 = 0$ et $3x -y+ 6 = 0 $ sont :

\medskip

\begin{tabularx}{\linewidth}{*{2}{X}}
\textbf{a.~~} perpendiculaires. &\textbf{b.~~}  sécantes non perpendiculaires.\\\textbf{c.~~}parallèles.& \textbf{d.~~}confondues.
\end{tabularx}

\medskip

\textbf{Question 5}

\medskip 

On considère la fonction Python ci-dessous :
\begin{python}
def suite(n) :
	u=2
	k=0
	while k<n :
		u=u+k
		k=k+1
	return u
\end{python}

Quelle valeur renvoie l’appel suite(5) ?

\medskip

\begin{tabularx}{\linewidth}{*{4}{X}}
\textbf{a.~~} $5$ &\textbf{b.~~} $8$&\textbf{c.~~}$12$& \textbf{d.~~} $17$.
\end{tabularx}

\vspace{0,5cm}

