
\medskip

Ce QCM comprend 5 questions indépendantes.

Pour chacune d'elles, une seule des réponses proposées est exacte.

Indiquer pour chaque question sur la copie la lettre correspondant à la réponse choisie.

Aucune justification n'est demandée.

Chaque réponse correcte rapporte 1 point. Une réponse incorrecte ou une absence de réponse n'apporte ni ne retire de point.

\medskip

\begin{enumerate}
\item  Pour tout réel $x$, $\cos(25\pi+x)$ est égal à :

\medskip
\begin{tabularx}{\linewidth}{*{4}{X}}
\textbf{a.~~} $ \cos(x)$ &\textbf{b.~~} $-\cos(x) $&\textbf{c.~~}$ \cos(-x)$& \textbf{d.~~} $-1$.
\end{tabularx}


\item On considère une fonction $f$ définie et dérivable sur l'intervalle $[-10 ;10]$.
On donne ci-dessous le tableau de variation de la fonction $f$:

\begin{center}
\begin{pspicture}(0,0)(10,2.5)
%\psgrid
\psframe(0,0)(10,2.5)
\psline(0,2)(10,2)
\psline(0,1.5)(10,1.5)
\psline(1.6,0)(1.6,2.5)
\rput(0.8,2.25){$x$}\rput(2,2.25){$-10$}\rput(4.5,2.25){$-2$}\rput(7,2.25){$3$}\rput(9.6,2.25){$10$}
\rput(0.8,1.75){$f'(x)$}\rput(3.25,1.75){$-$}\rput(4.5,1.75){$0$}\rput(5.75,1.75){$+$} \rput(7,1.75){$0$}\rput(8.3,1.75){$-$}
\rput(0.75,0.9){$f(x)$}
\psline{->}(2.4,1.2)(4,0.25)\psline{->}(4.9,0.25)(6.8,1.2)\psline{->}(7.6,1.2)(9.4,0.25)
\rput(1.97,1.28){0}\rput(7,1.28){4}
\rput(4.5,0.18){$-5$}\rput(9.6,0.18){3}
\end{pspicture}
\end{center}

On note $\mathcal{C}$ la courbe représentative de $f$ dans le plan muni d'un repère \Oij.

La tangente à la courbe $\mathcal{C}$ au point d'abscisse 3 a pour coefficient directeur :

\medskip
\begin{tabularx}{\linewidth}{*{4}{X}}
\textbf{a.~~} $  0 $ &\textbf{b.~~} $3 $&\textbf{c.~~}$4 $& \textbf{d.~~} $  10 $.
\end{tabularx}
\medskip
\item
$E$ et $F$ sont deux évènements indépendants d'un même univers.

On sait que $p(E)=0,4$ et $p(F)=0,3$ alors :

\medskip
\begin{tabularx}{\linewidth}{*{4}{X}}
\textbf{a.~~} $p(E\cup F)=0,7 $ &\textbf{b.~~} $p(E\cap F)=1,2$ &\textbf{c.~~}$p(E\cap F)=0 $& \textbf{d.~~} $p(E\cap F)=0,12  $.
\end{tabularx}
\medskip
\item. L'ensemble des solutions de l'inéquation $-3x^2+11x+1 \leqslant -3$ est :

\medskip
\begin{tabularx}{\linewidth}{*{2}{X}}
\textbf{a.~~} $ \left\{-\dfrac{1}{3}~;~4\right\} $ &\textbf{b.~~} $\left[-\dfrac{1}{3}~;~4\right] $\\\textbf{c.~~}$\left]-\infty~;~-\dfrac{1}{3}\right]\cup\left[4~;~+\infty\right[ $& \textbf{d.~~} $ \left]-\infty ~;~-\dfrac{1}{3}\right[\cup\left]4~;~+\infty\right[  $.
\end{tabularx}
\medskip
\item La loi de probabilité d'une variable aléatoire $X$ est donnée par ce tableau :

\begin{center}
\begin{tabularx}{0.7\linewidth}{|m{1.5cm}|*{4}{>{\centering \arraybackslash}X|}}\hline
$x_i$ 		&$-3$ 	&2		&5 	&10\\\hline
$p(X=x_i)$	&0,3	&0,21	&0,13	&0,36\\\hline
\end{tabularx} 
\end{center}

On peut en déduire que :

\medskip
\begin{tabularx}{\linewidth}{*{4}{X}}
\textbf{a.~~} $P(X>2)=0,49 $ &\textbf{b.~~} $P(X>2)=0,51 $&\textbf{c.~~}$P(X\geqslant 2)=0,49 $& \textbf{d.~~} $P(X\geqslant 2)=0,51$.
\end{tabularx} 
\medskip
\end{enumerate}

\vspace{0,5cm}

