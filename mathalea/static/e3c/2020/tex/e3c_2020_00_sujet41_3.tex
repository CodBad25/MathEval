
\medskip

\textbf{Partie A}

\medskip

Soit la suite géométrique $\left(u_n\right)$
de raison $0,999$ et de premier terme $u_0 = \np{82695}$. 


\medskip

\begin{enumerate}
\item Calculer $u_{19}$.
\item Calculer $S = u_0 + u_1 + \ldots + u_{19}$.
\end{enumerate}

\bigskip

\textbf{Partie B}

\medskip

La population d'un pays s'élevait à \np{82695000}~habitants au premier janvier 2016.

\medskip

Sans tenir compte des flux migratoires, on estime que la population baisse de 0,1\,\% chaque année.

Déterminer une estimation de l'effectif de la population de ce pays au premier janvier 2035.

\bigskip

\textbf{Partie C}

\medskip

Dans cette partie, on tient compte des flux migratoires: on estime qu'en 2016, le solde migratoire (différence entre les entrées et les sorties du territoire) est positif et s'élève à \np{58700} personnes.

De plus, on admet que la baisse de 0,1\,\% de la population ainsi que le solde migratoire restent constants chaque année suivant 2016.

On propose la fonction suivante écrite sous Python:

\begin{center}
\begin{tabular}{l}
\texttt{def population(N):}\\
\qquad \texttt{p=\np{82695000}}\\
\qquad \texttt{for I in range(1, N+1):}\\
\qquad \qquad  \texttt{p=0,999*p + \np{58700}}\\
\qquad \texttt{return p}\\
\end{tabular}
\end{center}

\smallskip

\begin{enumerate}
\item Si on saisit: \og population (2) \fg, quelle valeur nous retourne cette fonction ?
\item Si on saisit: \og population (19) \fg, la valeur arrondie à l'entier retournée par
cette fonction est \np{82243175}.

Que représente ce nombre dans le contexte de la partie C ?
\end{enumerate}

\bigskip

