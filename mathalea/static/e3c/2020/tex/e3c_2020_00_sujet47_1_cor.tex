
\subsection*{Question 1}

On a \(\vec{u} - \vec{v}\begin{pmatrix} 2 \\ -4 \end{pmatrix}\), donc :
\[
\|\vec{u} - \vec{v}\|^2 = (\vec{u} - \vec{v}) \cdot (\vec{u} - \vec{v}) = 4 + 16 = 20.
\]
Il en résulte que :
\[
\|\vec{u} - \vec{v}\| = \sqrt{20} = \sqrt{4 \times 5} = \sqrt{4} \times \sqrt{5} = 2\sqrt{5}.
\]

\subsection*{Question 2}

\[
f(x) = x^2 + 2x + 5 = (x + 1)^2 + 4,
\]
c'est donc une somme de carrés supérieure ou égale à 4, donc supérieure à zéro.

\subsection*{Question 3}

On a : \(\sin\left(\dfrac{\pi}{6}\right) = \sin\left(\dfrac{5\pi}{6}\right) = \dfrac{1}{2}\).

\subsection*{Question 4}

L'appel de cette fonction renvoie le premier terme de la suite tel que \(u_n \leqslant 6\).

\subsection*{Question 5}

Pour tout réel \(x\), \(\e^{3x-5} \times \e^{4-3x} = \e^{3x-5+4-3x} = \e^{-1} = \dfrac{1}{\e}\).

