
\medskip

Un propriétaire propose à un commerçant deux types de contrat pour la location d'un local pendant 3 ans.

\begin{description}
\item[1\up{er} contrat :]

un loyer de 200 \euro{} pour le premier mois puis une augmentation de 5 \euro{} par mois jusqu'à la fin du bail.

\item [2\up{e} contrat :]

 un loyer de 200 \euro{} pour le premier mois puis une augmentation de 2\,\% par mois jusqu'à la fin du bail.
\end{description}
On modélise ces deux contrats par des suites $\left(u_n\right)$ et $\left(v_n\right)$, de sorte que pour tout entier $n\geqslant 1$, le prix du loyer le $n$-ième mois avec le 1\up{er} contrat est représenté par $u_n$ et le prix du loyer le $n$-ième mois avec le 2\ieme contrat est représenté par $v_n$.
On a ainsi $u_1=v_1=200$.

\medskip

\begin{enumerate}
\item  Calculer, pour chacun des deux contrats, le loyer du deuxième mois puis le loyer du troisième mois.
\item Le commerçant a écrit un programme en langage Python qui lui permet de déterminer $u_n$ et $v_n$ pour une valeur donnée de $n$.
\begin{center}
\begin{tabular}[]{|ccl|}
\hline
1& & u=200 \\
2& &v=200\\
3& &n=int(input("Saisir une valeur de n :"))\\
4& &for i in range(1,n):\\
5& &\hspace{1.5em}u= \dotfill\\
6& &\hspace{1.5em}v= \dotfill \\
7& &print("Pour n =",n,"on a","u =",u," et v =",v)\\
\hline
\end{tabular}
\end{center}

\begin{enumerate}
\item Recopier et compléter les lignes 5 et 6 de ce programme.
\item Quels nombres obtiendra-t-on avec n = 4 ?
\end{enumerate}
\item Déterminer, pour tout entier $n\geqslant 1$, l'expression de $u_n$ et de $v_n$ en fonction de $n$.
\item Quel contrat coûtera le moins cher au total pour l'ensemble d'un bail de 3 ans ?
\end{enumerate}

\vspace{0,5cm}

