  
\medskip

On considère la fonction $f$ définie sur $\R$ par $f(x) = x^3 + 3x^2 + 3x -63$.
On appelle $\mathcal{C}$ sa courbe représentative dans un repère orthonormé.

\medskip

\begin{enumerate}
\item Déterminer $f'(x)$.
\item Étudier le signe de $f'(x)$ sur $\R$.
\item Établir le tableau de variations de la fonction $f$ sur $\R$.
\item Justifier que la tangente à la courbe $\mathcal{C}$ au point d'abscisse $-1$ est la droite $\mathcal{D}$ d'équation
$y= -64$.
\item Déterminer en quels points de la courbe $\mathcal{C}$ la tangente à la courbe est parallèle à la droite
d'équation $y = 3x - 100$.
\end{enumerate}

\vspace{0,5cm}

