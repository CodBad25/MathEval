
\medskip

Une chaîne de salons de coiffure propose à ses clients qui viennent pour une coupe deux prestations supplémentaires cumulables:
\begin{itemize}
\item une coloration naturelle à base de plantes appelée \og couleur-soin \fg,
\item des mèches blondes pour donner du relief à la chevelure, appelées \og effet coup de soleil \fg.
\end{itemize}

\smallskip

Il apparaît que 40\,\% des clients demandent une \og couleur-soin \fg. Parmi ceux qui ne veulent pas de \og couleur soin \fg, 30\,\% des clients demandent un \og effet coup de soleil \fg. Par ailleurs, 24\,\% des clients demandent une \og couleur soin\fg et un \og effet coup de soleil \fg.

On interroge un client au hasard.

On notera $C$ l'évènement \og Le client souhaite une "couleur-soin."\fg.

On notera $E$ l'évènement \og Le client souhaite un "effet coup de soleil."\fg.

\medskip

\begin{enumerate}
\item Donner les valeurs de $P(C)$, $P( C \cap E)$ et $P_{\overline{C}}(E)$.
\item Calculer la probabilité que le client ne souhaite ni une \og couleur-soin \fg, ni un \og
effet coup de soleil \fg.
\item Montrer que la probabilité de l'évènement $E$ est égale à $0,42$.
\item Les évènements $C$ et $E$ sont-ils indépendants ?
\end{enumerate}

\bigskip

