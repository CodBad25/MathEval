
\subsection*{1.}

On a :
\[
AB^2 = (1 - (-2))^2 + (2 - 1)^2 = 9 + 1 = 10.
\]
Le carré du rayon est égal à 10.

\[
M(x\,;\,y) \in \mathcal{C} \iff AM^2 = 10 \iff (x + 2)^2 + (y - 1)^2 = 10.
\]

\subsection*{2.}

Avec \(\overrightarrow{AB} \begin{pmatrix} 3 \\ 1 \end{pmatrix}\) et \(\overrightarrow{AE} \begin{pmatrix} 2 \\ -6 \end{pmatrix}\), on a :
\(\overrightarrow{AB} \cdot \overrightarrow{AE} = 6 - 6 = 0\).

\subsection*{3.}

La question précédente montre que les vecteurs \(\overrightarrow{AB}\) et \(\overrightarrow{AE}\) sont orthogonaux, donc les droites \((AB)\) et \((AE)\) sont perpendiculaires.

\subsection*{4.}

On a donc :
\begin{align*}
&M(x\,;\,y) \in (AE) \\
\iff &\overrightarrow{AM} \cdot \overrightarrow{AB} = 0 \\
\iff &3(x + 2) + 1(y - 1) = 0 \\
\iff &3x + y + 5 = 0.
\end{align*}

\subsection*{5.}

Un point \(M(x\,;\,y)\) est commun à la droite \((AE)\) et au cercle \(\mathcal{C}\) si ses coordonnées vérifient leurs équations et donc le système :
\begin{align*}
&\begin{cases}
(x + 2)^2 + (y - 1)^2 = 10 \\
3x + y + 5 = 0
\end{cases} \\
\iff &\begin{cases}
(x + 2)^2 + (-3x - 5 - 1)^2 = 10 \\
y = -3x - 5
\end{cases} \\
\iff &\begin{cases}
(x + 2)^2 + (-3x - 6)^2 = 10 \quad (1)\\
y = -3x - 5
\end{cases}
\end{align*}
\((1)\) donne :
\begin{align*}
x^2 + 4x + 4 + 9x^2 + 36x + 36 &= 10 \\
10x^2 + 40x + 30 &= 0 \\
x^2 + 4x + 3 &= 0.
\end{align*}

L'équation \(x^2 + 4x + 3 = 0\) a une racine évidente : \(-1\) et comme le produit des racines est égal à 3, l'autre racine est \(-3\).

En remplaçant successivement \(x\) par \(-1\) puis par \(-3\) dans l'équation \(y = -3x - 5\), on obtient deux points communs au cercle et à la droite : les points \((-1 \,;\, -2)\) et \((-3 \,;\, 4)\).

