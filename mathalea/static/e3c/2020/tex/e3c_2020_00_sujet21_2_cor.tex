	
	\subsection*{1.}
	
	À la sortie du four \(t = 0\), donc
	\[
	f(0) = 1375e^{-0,075 \times 0} + 25 = 1375 + 25 = 1400.
	\]
	
	\subsection*{2.}
	
	La fonction \(f\) est dérivable sur \(\left[0 ; +\infty\right[\) et sur cet intervalle :
	\[
	f'(t) = 1375 \times (-0,075)e^{-0,075t} = -103,125e^{-0,075t}.
	\]
	
	Comme quel que soit \(t\), \(e^{-0,075t} > 0\), on a donc \(f'(t) < 0\) : la fonction \(f\) est donc strictement décroissante de 1400 à 25.
	
	Ce résultat était prévisible car la pièce ne peut que se refroidir et atteindre la température ambiante.
	
	\subsection*{3.}
	
	On a
$
	f(10) = 1375e^{-0,075 \times 10} + 25 \approx 674,4$ : la pièce ne peut être travaillée car trop chaude.\\
$
	f(14) = 1375e^{-0,075 \times 14} + 25 \approx 506,2$ : la pièce peut être travaillée mais pour peu de temps.

	
	\subsection*{4.}
	
	a. 	\begin{verbatim}
		from math import exp
		
		def f(t):
		     return 1375 * exp(-0,075 * t) + 25
		
		def seuil():
		     t = 0
		     temperature = f(t)
		     while temperature >= 600:
		          t = t + 0,1
		          temperature = f(t)
	    	return t
	\end{verbatim}
	
	
	
	
	
	b. On trouve \(t = 17,3 \, h\), soit environ 17 h 18 min.
	
