
\medskip

Une fleuriste met en vente quatre sortes de bouquets dont les tarifs et la composition sont
indiqués dans le tableau ci-dessous :

\medskip
\begin{center}
\begin{tabularx}{0.85\linewidth}{|*{2}{X|}}\hline
Bouquet de tulipes orange : 10,50 \euro  & Bouquet de roses orange : 23,50 \euro\\\hline
Bouquet de tulipes blanches : 11,60 \euro& Bouquet de roses blanches : 25,50 \euro\\\hline
\end{tabularx}
\end{center}
\medskip

\begin{itemize}
\item 72\,\% des bouquets mis en vente ne contiennent que des roses.
\item Les autres bouquets mis en vente ne contiennent que des tulipes.
\item 20\,\% des bouquets de tulipe mis en vente ne contiennent que des tulipes orange.
\item 36\,\% des bouquets mis en vente ne contiennent que des roses blanches.
\end{itemize}
Un client achète au hasard un bouquet parmi ceux mis en vente par la fleuriste. On note :

\begin{itemize}
\item $R$ l'évènement : \og Le bouquet acheté par ce client est composé de roses.\fg
\item $B$ l'évènement : \og Le bouquet acheté par ce client est composé de fleurs blanches. \fg
\end{itemize}
Les évènements contraires des évènements $R$ et $B$ sont notés respectivement $\overline{R}$ et $\overline{B}$.

\medskip

\begin{enumerate}
\item 
	\begin{enumerate}
		\item Donner, sans justifier, la probabilité $p(R \cap B)$.
		\item Recopier et compléter le plus possible l'arbre de probabilité ci-dessous en
traduisant uniquement les données de l'énoncé.
\begin{center}

\psset{nodesepA=0pt,nodesepB=3pt,treesep=0.75,labelsep=0.1pt,levelsep=2.5cm}
\pstree[treemode=R]{\TR{}}
{\pstree{\TR{}\taput{}}
	{
	\TR{}\taput{}
	\TR{}\tbput{}
	}
\pstree{\TR{~}\tbput{}}
	{\TR{}\taput{}
	\TR{}\tbput{}
	}
}
\end{center}

		\item Montrer que $p(B) = 0,584$.
	\end{enumerate}
\item On note $X$ la variable aléatoire qui donne le prix d'un bouquet acheté par un client.
	\begin{enumerate}
		\item  Recopier et compléter le tableau ci-dessous donnant, pour chaque valeur $x_i$ de $X$,
la probabilité de l'évènement $\{X = x_i \}$. Justifier.

\begin{center}
\begin{tabularx}{0.6\linewidth}{|m{1.82cm}|*{4}{>{\centering \arraybackslash}X|}}\hline
\centering $x_i$		&	&	&	& \\\hline
\centering $p(X = x_i)$ &	&	&	&\\\hline
\end{tabularx}
\end{center}

		\item Calculer l'espérance de la variable aléatoire $X$. \emph{On arrondira le résultat au centième.}
	\end{enumerate}
\end{enumerate}

\vspace{0,5cm}

