
\medskip

Lorsqu'il s'entraîne au tennis, Roger utilise un lance-balle.

Cette machine lance les balles soit sur le coup droit soit sur le revers du joueur.
On la remplit de balles et on la programme de la façon suivante : deux tiers des
balles seront lancées sur le coup droit du joueur, le reste sur son revers.
On s'intéresse à la réussite des frappes de Roger pendant une séance
d'entraînement.

\begin{description}
\item[]On note $D$ l'évènement : \og le joueur reçoit la balle sur son coup droit \fg.

\item[]On note $\overline{D}$ l'évènement contraire de l'évènement D.
\end{description}

Roger réussit $\dfrac{9}{10}$ de ses coups droits et 75\,\% de ses revers.

On note $S$ l'évènement : \og La frappe de Roger est un succès \fg.

\medskip

\begin{enumerate}
\item Donner $ p\left(\overline{D}\right)$.
\item Compléter l'arbre pondéré situé ci-dessous représentant la situation.

\begin{center}
\psset{nodesepA=0pt,nodesepB=3pt,treesep=0.75,levelsep =3cm,labelsep=0.51pt}
\pstree[treemode=R]{\TR{}}
{\pstree{\TR{$D$~}\taput{$\frac{2}{3}$}}
	{
	\TR{$S$}\taput{$\dots$}
	\TR{$\overline{S}$}\tbput{$\dots$}
	}
\pstree{\TR{$\overline{D}$~}\tbput{$\dots$}}
	{\TR{$S$}\taput{$\dots$}
	\TR{$\overline{S}$}\tbput{$\dots$}
	}
}
\end{center}


\item Calculer $p\left(\overline{D}\cap S\right)$. Interpréter ce résultat dans le contexte de l'exercice.
\item Montrer que la probabilité que la frappe de Roger soit un succès est égale à 0,85.
\item Sachant que la frappe que vient de réaliser Roger est un succès, calculer la probabilité que ce soit sur un revers. Arrondir le résultat au centième.
\end{enumerate}

\vspace{0,5cm}

