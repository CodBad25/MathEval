  
\medskip

Une entreprise produit entre 1 millier et 5 milliers de pièces par jour. Le coût moyen de production d'une pièce, en milliers d'euros, pour $x$ milliers de pièces produites, est donné par la fonction $f$ définie pour tout réel $x\in[1~;~5]$ par :

\[f(x)=\dfrac{0,5x^3-3x^2+x+16}{x}\]

\begin{enumerate}
\item Calculer le coût moyen de production d'une pièce lorsque l'entreprise produit 2 milliers de pièces.
\item On admet que [de] $f$ est dérivable sur [1 ; 5] et on note $f'$ sa fonction dérivée.

 Montrer que pour tout réel $x\in[1~;~5]$, \[f'(x)=\dfrac{x^3-3x^2-16}{x^2}\]
\item Vérifier que, pour tout réel $x$, \[x^3 - 3x^2 - 16 =(x - 4)(x^2 +x + 4)\]
\item En déduire le tableau de variation de $f$ sur [1~;~5].
\item Déterminer le nombre de pièces à fabriquer pour que le coût moyen de production d'une pièce soit minimal, ainsi que la valeur de ce coût minimal.
\end{enumerate}

\vspace{0,5cm}

