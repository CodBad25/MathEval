	\subsection*{1.}
	
	\(\overrightarrow{AB} \left( \begin{array}{c} -1 \\ 5 \end{array} \right)\) et \(\overrightarrow{AC} \left( \begin{array}{c} -5 \\ 2 \end{array} \right)\), d’où \(\overrightarrow{AB} \cdot \overrightarrow{AC} = 5 + 10 = 15\).
	
	\subsection*{2.}
	
	a. Soit \(\overrightarrow{AB} \cdot \overrightarrow{AD} = \overrightarrow{AB} \cdot (\overrightarrow{AC} + \overrightarrow{CD}) = \overrightarrow{AB} \cdot \overrightarrow{AC}\), car \(\overrightarrow{AB}\) et \(\overrightarrow{CD}\) sont orthogonaux.
	
	b. \(\overrightarrow{AB} \cdot \overrightarrow{AD} = \overrightarrow{AB} \cdot \overrightarrow{AC}\) soit \(AB \times AD = 15\).
	
	Or
	\[
	AB^2 = 1 + 25 = 26, \quad \text{donc} \quad AB = \sqrt{26}, \quad \text{donc} \quad AD = \dfrac{15}{\sqrt{26}}.
	\]
	
	\subsection*{3.}
	
	Dans le triangle \(ACD\) rectangle en \(D\), le théorème de Pythagore s’écrit :\\
$
	CD^2 + AD^2 = AC^2$  soit $ CD^2 + \left(\dfrac{15}{\sqrt{26}}\right)^2 = 5^2 + 2^2 = 29$, \\
	 d’où $ AD^2 = 29 - \dfrac{225}{26} = \dfrac{29 \times 26 - 225}{26} = \dfrac{754 - 225}{26} = \dfrac{529}{26}.$
	
	Donc la hauteur du triangle \(ABC\) issue de \(C\) a pour longueur
	\[
	CD = \sqrt{\dfrac{529}{26}}.
	\]
	
	\subsection*{4.}
	
	En prenant comme base \([AB]\), avec \(AB = \sqrt{26}\) et comme hauteur \([CD]\), avec \(CD = \sqrt{\dfrac{529}{26}}\).
	
	\[
	A(ABC) = \dfrac{AB \times CD}{2} = \dfrac{\sqrt{26} \times \sqrt{\dfrac{529}{26}}}{2} = \dfrac{\sqrt{529}}{2} = \dfrac{23}{2} = 11,5.
	\]
	
