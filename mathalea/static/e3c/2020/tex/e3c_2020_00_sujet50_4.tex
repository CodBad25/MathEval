
\medskip

Fanny est inscrite dans un club d'athlétisme. Elle pratique le penta bond (le penta bond est un enchaînement de cinq bonds après une course d'élan).

La première semaine d'entraînement, Fanny réalise un saut de $8$ m.

Chaque semaine, la longueur de son saut augmente de $0,1$ m.

Pour $n$ entier naturel non nul, on note $s_n$ la longueur, en mètres, de son saut la $n$-ième semaine d'entraînement.

Puisque lors de la première semaine d'entraînement, Fanny réalise un saut de $8$ m, on a $s_1 = 8$.

\medskip

\begin{enumerate}
\item Pour $n \geqslant 2$, on considère la fonction Python suivante.

\begin{center}
\begin{tabularx}{0.4\linewidth}{|X|}\hline
\texttt{def saut(n)}\\
\quad  \texttt{s=8}\\
\quad  \texttt{for k in range(2~n+1):}\\
\qquad\quad \texttt{s=s+0,1}\\
\quad \texttt{return s}\\ \hline
\end{tabularx}
\end{center}

	\begin{enumerate}
		\item Quelle valeur $s$ est-elle renvoyée par la commande \texttt{saut(4)} ?
		\item Interpréter cette valeur dans le contexte de l'exercice.
	\end{enumerate}
\item Exprimer avec justification $s_n$ en fonction de $n$ pour $n$ entier naturel non nul.
\item Pour être qualifiée à une compétition, Fanny doit faire un saut d'au moins $12$~mètres.
	\begin{enumerate}
		\item À partir de quelle semaine, Fanny réalisera-t-elle un tel saut ? 
		\item Justifier votre réponse.
	\end{enumerate}
\end{enumerate}
