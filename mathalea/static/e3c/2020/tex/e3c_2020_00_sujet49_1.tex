  
\medskip

Ce QCM comprend 5 questions indépendantes.

Pour chacune d'elles, une seule des affirmations proposées est exacte.
 
Indiquer pour chaque question sur la copie la lettre correspondant à la réponse choisie.

 Aucune justification n'est demandée.
 
Chaque réponse correcte rapporte 1 point. Une réponse incorrecte ou une absence de réponse n'apporte ni ne retire de point.

\begin{enumerate}
\item On considère la droite $d$ dont une équation cartésienne dans un repère orthonormé est $2x-3y+4=0$.

\begin{tabularx}{\linewidth}{*{2}{X}}
\textbf{a.~~} Un vecteur directeur de $d$ est $\vv{u} \binom{-6}{4}$. &\textbf{b.~~} Un vecteur normal de $d$ est $\vv{n}\binom{-12}{18}$. \\
\textbf{c.~~} Le point $ C(-5~;~2)$ appartient à la droite $d$.& \textbf{d.~~} La droite $d$ coupe la droite d'équation 

$-x+3y-2=0$ au point F (1~;~2).
\end{tabularx}

\item Dans un repère orthonormé le cercle $\mathcal{C}$ a pour équation $x^2-2x+y^2+y=3$ 

et la droite $\mathcal{D}$ pour équation $y=1$.


\begin{tabularx}{\linewidth}{*{1}{X}}
\textbf{a.~~} $\mathcal{C}$ et $\mathcal{D}$ n'ont aucun point d'intersection.\\
\textbf{b.~~} $\mathcal{C}$ et $\mathcal{D}$ ont un seul point d'intersection.\\
\textbf{c.~~} $\mathcal{C}$ et $\mathcal{D}$ ont deux points d'intersection.\\
\textbf{d.~~}On ne peut pas savoir combien $\mathcal{C}$ et $\mathcal{D}$ ont de points d'intersection.\\
\end{tabularx}

\item La fonction $f$ est la fonction définie sur l'ensemble des réels par $f(x)= \cos(2x)$.

\begin{tabularx}{\linewidth}{*{2}{X}}
\textbf{a.~~}$f $est paire.  &\textbf{b.~~}$f$ est impaire. \\\textbf{c.~~}$f$ n'est ni paire ni impaire.& \textbf{d.~~}$f$ a pour période $\dfrac{\pi}{2}$ .
\end{tabularx}

\item Soit la suite $\left(u_n\right)$ définie par $u_0=1$ et pour tout entier naturel $n$ par $u_{n+1}=\dfrac{1}{2}\left(u_n + \dfrac{2}{u_n}\right)$.

On définit en langage Python une fonction \og suite \fg{} pour calculer $u_n$ connaissant $n$.

\begin{tabular}{*{2}{lr}}
\textbf{a.~~} 
\begin{minipage}[]{0.4\linewidth}
\begin{python}
 def suite(n):
	u=0 
	for i in range (1,n+1):
		u=1/2*(u+2/u) 
	return u
 \end{python}
\end{minipage} 
 &
\textbf{b.~~}
\begin{minipage}[]{0.4\linewidth}
\begin{python}
 def suite(n):
	u=1 
	for i in range (1,n+1):
		u=1/2*(u+2/u)
	return n
\end{python}
\end{minipage}\\

\textbf{c.~~}
\begin{minipage}[]{0.4\linewidth}
\begin{python}
 def suite(n):
	u=1
	for i in range (1,n+1):
		u=1/2*u+2/u 
	return u
\end{python}
\end{minipage}&
\textbf{d.~~}
\begin{minipage}[]{0.4\linewidth}
\begin{python}
 def suite(n):
	u=1 
	for i in range (1,n+1):
		u=1/2*(u+2/u)
	return u 
\end{python}
\end{minipage}
\end{tabular}

\item L'équation $\e^x=1$ :

\begin{tabularx}{\linewidth}{*{2}{X}}
\textbf{a.~~}n'a pas de solution.  &\textbf{b.~~}a pour solution le nombre 1. \\\textbf{c.~~}a pour solution le nombre 0.& \textbf{d.~~} a pour solution le nombre $\e$.
\end{tabularx}
\end{enumerate}

\vspace{0,5cm}

