
\subsection*{1.}

Prendre 150 \% de quelque chose, c'est le multiplier par \(\dfrac{150}{100} = 1{,}5\).

Donc :
\begin{align*}
&u_1 = 2, \\
&u_2 = 2 \times 1{,}5 = 3, \\
&u_3 = 3 \times 1{,}5 = 4{,}5, \\
&u_4 = 4{,}5 \times 1{,}5 = 6{,}75.
\end{align*}

\subsection*{2.}

On a vu que, quel que soit \(n \in \mathbb{N}, n \geqslant 1\), \(u_{n+1} = 1{,}5 u_n\).
  
Cette relation montre que la suite \((u_n)\) est géométrique de raison \(q = 1{,}5\), de premier terme \(u_1 = 2\).

\subsection*{3.}

On sait que, pour \(n \in \mathbb{N}, n \geqslant 1\), \(u_n = u_1 \times q^{n-1}\), soit \(u_n = 2 \times 1{,}5^{n-1}\).

\subsection*{4.}

\begin{python}
	i = 1
	u = 2
	longueur = 2
	while longueur < 1000 :
		i = i + 1
		u = 1.5 * u
		longueur = longueur + u
	print(i)
\end{python}

Il faut calculer : \(L = u_1 + u_2 + \dots + u_{14} (1)\).  

Or : \(1{,}5 L = u_2 + u_3 + \dots + u_{15} (2)\),

et par différence : \(0{,}5 L = u_{15} - u_1\),

d'où : \(L = 2(u_{15} - u_1) = 2(2 \times 1{,}5^{14} - 2) \approx 1163{,}7\) (mm),

soit effectivement environ 1{,}164 m au mm près.

