  
\medskip

\emph{Ce QCM comprend 5 questions indépendantes. Pour chacune d'elles, une seule des réponses proposées est exacte.}

\emph{Indiquer pour chaque question sur la copie la lettre correspondant à la réponse choisie.}

 \emph{Aucune justification n'est demandée.}

\emph{Chaque réponse correcte rapporte 1 point. Une réponse incorrecte ou une absence de réponse n'apporte ni ne retire de point.}

\begin{enumerate}
\item Si $\sin x=\dfrac{1}{3}$ alors


\begin{tabularx}{\linewidth}{*{4}{X}}
\textbf{a.~~} $ \sin(x+\pi)=-\frac{1}{3}$ &\textbf{b.~~} $\sin(x-\pi)=\dfrac{1}{3}$&\textbf{c.~~}$\cos(x)=\dfrac{2}{3}$& \textbf{d.~~} $\sin(x+15\pi)=\frac{1}{3}$.
\end{tabularx}
 
\item Parmi les paraboles ci-dessous laquelle représente une fonction qui n'admet aucune racine ?

\begin{tabularx}{\linewidth}{*{2}{X}}
\textbf{a.~~}

\psset{xunit=0.8cm,yunit=0.8cm,labelFontSize=\scriptstyle,showorigin=false}
\begin{pspicture}(-4.4,-2)(2.2,2.6)
\multido{\n=-4.2+0.2}{31}{\psline[linewidth=0.35pt,linecolor=lightgray](\n,-2)(\n,2.4)}
\multido{\n=-2+0.2}{22}{\psline[linewidth=0.35pt,linecolor=lightgray](-4.4,\n)(2.2,\n)}
\multido{\n=-4+1}{7}{\psline[linewidth=0.45pt](\n,-2)(\n,2.4)}
\multido{\n=-2+1}{5}{\psline[linewidth=0.45pt](-4.4,\n)(2.2,\n)}
\psaxes[linewidth=0.95pt]{->}(0,0)(-4.4,-2)(2.1,2.4)
\def\Func{x 3 add x 1 sub mul 0.4 mul }
\psplot[plotpoints=2000,linewidth=0.85pt,linecolor=red]{-4.2}{2.1}{\Func}
\uput[dl](0,0){O}
\end{pspicture}
&\textbf{b.~~} 
 
\psset{xunit=0.8cm,yunit=0.8cm,labelFontSize=\scriptstyle,showorigin=false}
\begin{pspicture}(-4.4,-3.8)(0.4,1)
\multido{\n=-4.2+0.2}{23}{\psline[linewidth=0.35pt,linecolor=lightgray](\n,-3.6)(\n,0.8)}
\multido{\n=-3.6+0.2}{23}{\psline[linewidth=0.35pt,linecolor=lightgray](-4.4,\n)(0.2,\n)}
\multido{\n=-4+1}{5}{\psline[linewidth=0.45pt](\n,-3.6)(\n,0.8)}
\multido{\n=-3+1}{4}{\psline[linewidth=0.45pt](-4.4,\n)(0.2,\n)}
\psaxes[linewidth=0.95pt]{->}(0,0)(-4.4,-3.62)(0.2,1)
\def\Func{x 3 add x 2 add mul 0.8 neg mul }
\psplot[plotpoints=2000,linewidth=0.85pt,linecolor=green]{-4.4}{-0.4}{\Func}
\uput[dl](0,0){O}
\end{pspicture}
\\
\textbf{c.~~}

\psset{xunit=1cm,yunit=0.8cm,labelFontSize=\scriptstyle,showorigin=false}
\begin{pspicture}(-1,-0.6)(3.2,3.4)
\multido{\n=-0.8+0.2}{21}{\psline[linewidth=0.35pt,linecolor=lightgray](\n,-0.4)(\n,3.2)}
\multido{\n=-0.4+0.2}{19}{\psline[linewidth=0.35pt,linecolor=lightgray](-0.8,\n)(3.2,\n)}
\multido{\n=0+1}{4}{\psline[linewidth=0.45pt](\n,-0.4)(\n,3.2)}
\multido{\n=0+1}{4}{\psline[linewidth=0.45pt](-0.8,\n)(3.2,\n)}
\psaxes[linewidth=0.95pt]{->}(0,0)(-0.8,-0.4)(3.4,3.2)
\def\Func{x 1 sub x 1 sub mul 2 mul }
\psplot[plotpoints=2000,linewidth=0.85pt,linecolor=red]{-0.24}{2.29}{\Func}
\uput[dl](0,0){O}
\end{pspicture}

& \textbf{d.~~} 

\psset{unit=1.2cm,labelFontSize=\scriptstyle,showorigin=false,comma=true}
\begin{pspicture}(-0.4,-1)(3.2,2.6)
\multido{\n=0+0.1}{30}{\psline[linewidth=0.35pt,linecolor=lightgray](\n,-0.4)(\n,2.4)}
\multido{\n=-0.4+0.1}{28}{\psline[linewidth=0.35pt,linecolor=lightgray](0,\n)(3,\n)}
\multido{\n=0+0.5}{6}{\psline[linewidth=0.45pt](\n,-0.4)(\n,2.4)}
\multido{\n=0+0.5}{5}{\psline[linewidth=0.45pt](0,\n)(3,\n)}
\psaxes[linewidth=0.95pt,Dx=0.5,labels=x]{->}(0,0)(0,-0.5)(3,2.4)
\def\Func{x 1.5 sub x 1.5 sub mul  0.75 add }
\psplot[plotpoints=2000,linewidth=0.85pt,linecolor=blue]{0.2}{2.7}{\Func}
\uput[dl](0,0){O}
\end{pspicture}

\end{tabularx}
\item Soit la fonction $f$ définie sur l'intervalle $]0~;~+\infty[$ par $f(x)= 2x- \dfrac{1}{x}$

Le coefficient directeur de la tangente à la courbe représentative de $f$ au point d'abscisse 1 est :

\medskip

\begin{tabularx}{\linewidth}{*{4}{X}}
\textbf{a.~~} $1$ &\textbf{b.~~} $3$&\textbf{c.~~}$-1$& \textbf{d.~~} $0$.
\end{tabularx}

\item Dans le plan muni d'un repère orthonormé, l'ensemble des points $M(x~;~y)$ tels que

$x^2-2x+y^2+6y+2=0$ est :

\medskip

\begin{tabularx}{\linewidth}{*{4}{X}}
\textbf{a.~~}une parabole  &
\textbf{b.~~}\begin{minipage}[t]{2.89cm}
{\footnotesize le cercle de centre $\Omega$ de coordonnées} \mbox{$(-1~;~3)$} et de rayon 8.\end{minipage} &
\textbf{c.~~}\begin{minipage}[t]{2.89cm}
{\footnotesize  le cercle de centre $\Omega$ de coordonnées} \mbox{$(1~;~-3)$} et de rayon $2\sqrt{2}$.\end{minipage}&
 \textbf{d.~~} une droite.
\end{tabularx}
 
\item La loi de probabilité d'une variable aléatoire $X$ donnant le gain en euros, d'un joueur, à un jeu, est donnée par le tableau suivant :

\begin{center}
\begin{tabularx}{0.5\linewidth}{|m{1.4cm}|*{3}{>{\centering \arraybackslash}X|}}
\hline
\centering $x_i$& $-10$ &6& 10\\
\hline
$P(X=x_i)$&$\dfrac{1}{4}$&$\dfrac{3}{8}$&$\dfrac{3}{8}$\rule[-3mm]{0mm}{9mm}\\
\hline
\end{tabularx}
\end{center}

Sur un grand nombre de parties, le gain moyen que peut espérer le joueur est :

\medskip

\begin{tabularx}{\linewidth}{*{4}{X}}
\textbf{a.~~} 3,5 euros &\textbf{b.~~}4 euros &\textbf{c.~~}2 euros& \textbf{d.~~6 euros} .
\end{tabularx}
\end{enumerate}

\vspace{0,5cm}

