
\medskip

Cet exercice est un questionnaire à choix multiples (QCM). Pour chacune des questions, une seule des quatre propositions est correcte.

Les questions sont indépendantes. Pour chaque question, indiquer le numéro de la question et recopier sur la copie la lettre correspondant à la réponse choisie. Aucune justification n'est demandée.

Chaque réponse exacte rapporte un point. Une réponse fausse ou une absence de réponse ne rapporte aucun point.

\medskip

\textbf{Question 1}

\medskip

On lance deux fois une pièce équilibrée, de manières identiques et indépendantes.

Si le joueur obtient 2 Faces, il perd 5~\euro, s'il obtient exactement une Face, il gagne 2~\euro, s'il obtient 2 Piles il gagne 4~\euro.

 On note $G$ la variable aléatoire correspondant au gain algébrique du joueur, en euros.

\begin{center}
\begin{tabularx}{\linewidth}{|*{4}{X|}}\hline
\textbf{a.~~}$E(G) = 0,75$& \textbf{b.~~}$E(G) = 3$&\textbf{c.~~}$E(G) = 1$&\textbf{d.~~}$E(G) =- 4$ \\ \hline
\end{tabularx}
\end{center}

\medskip

\textbf{Question 2}

\medskip

$A$ et $B$ sont deux évènements, et on donne $P(A) = \dfrac{3}{7}$,\, $P(B) = \dfrac{3}{20}$,\, $P(A \cup B) = \dfrac{4}{7}$.

\begin{center}
\begin{tabularx}{\linewidth}{|*{4}{X|}}\hline
\textbf{a.~~}A et B sont indépendants.& \textbf{b.~~}$P_A(B) = \dfrac{3}{980}$&\textbf{c.~~}$P(A \cap B) = \dfrac{1}{140}$&\textbf{d.~~}$P_A(B) = \dfrac{1}{60}$ \\ \hline
\end{tabularx}
\end{center}

\medskip

\textbf{Question 3}

\medskip

On donne l'arbre de probabilités ci-dessous, ainsi que la probabilité $P(C) = 0,48$.

\begin{center}
\pstree[treemode=R,nodesepA=0pt,nodesepB=3pt]{\TR{}}
{\pstree{\TR{$A~~$} \naput{0,2}}
	{\TR{$C~~$}\naput{0,6}
	\TR{$\overline{C}~~$}\nbput{0,4}
	}
\pstree{\TR{$\overline{A}~~$} \nbput{0,8}}
	{\TR{$C~~$}\naput{$x$}
	\TR{$\overline{C}~~$}\nbput{$1 - x$}
	}
}
\end{center}

\smallskip

\begin{center}
\begin{tabularx}{\linewidth}{|*{4}{X|}}\hline
\textbf{a.~~}$x = 0,6$& \textbf{b.~~}$x = 0,36$&\textbf{c.~~}$x = 0,45$&\textbf{d.~~}$x = \dfrac{0,48}{0,12}$\rule[-3mm]{0mm}{9mm} \\ \hline
\end{tabularx}
\end{center}
\medskip

\textbf{Question 4}

\medskip

On a tracé la courbe représentative $\mathcal{C}_f$ d'une fonction $f$ dans un repère orthonormé, ainsi que deux de ses tangentes, au point E d'abscisse 2 et au point G d'abscisse 4.

\medskip

\parbox{0.48\linewidth}{Les coordonnées des points E, F, G, H placés dans le repère ci-contre peuvent être lues graphiquement, ce sont des entiers.

La tangente à $\mathcal{C}_f$ au point E est la droite (EF). 

La tangente à $\mathcal{C}_f$ au point G est la droite (GH). 

On note $f'$ la fonction dérivée de $f$.}\hfill
\parbox{0.51\linewidth}{\psset{unit=0.9cm}
\begin{pspicture*}(-0.75,-1.5)(6.5,6.5)
\psgrid[gridlabels=0pt,subgriddiv=1,gridwidth=0.4pt](0,-1)(7,7)
\psaxes[linewidth=1.25pt]{->}(0,0)(-0.5,-1.5)(6.5,6.5)
\psplot[plotpoints=2000,linewidth=1.25pt,linecolor=blue]{0.5}{6.5}{x 5 sub x 3 exp 3 mul 14 div x dup mul 1.75 mul sub 3.5 x mul add 19 7 div sub mul}
\psplot[plotpoints=2000,linewidth=1.pt]{0}{4}{2 x mul 1 sub}
\psplot[plotpoints=2000,linewidth=1.pt]{2.5}{6}{15 3 x mul sub}
\uput[dr](0,-1){F} \uput[ul](2,3){E}\uput[ur](4,3){G}\uput[dl](5,0){H}
\uput[r](0.6,6){\blue $\mathcal{C}_f$}\uput[dl](0,0){O}
\psdots(0,-1)(2,3)(4,3)(5,0)
\end{pspicture*}}

\begin{center}
\begin{tabularx}{\linewidth}{|*{4}{X|}}\hline
\textbf{a.~~}$f'(2) = 4$& \textbf{b.~~}$f'(2) = 3$&\textbf{c.~~}$f'(4)=3$&\textbf{d.~~}$f'(4)= - 3$ \\ \hline
\end{tabularx}
\end{center}

\medskip

\textbf{Question 5}

\medskip

On considère la fonction Python suivante :

\begin{center}
$\begin{array}{l}
\text{def } evolu(k) :\\
\qquad  i = 200\\
\qquad n = 0\\
\qquad \text{while}\: i < k :\\
\qquad\qquad i= 1.2*i + 10\\
\qquad\qquad n = n+1\\
\qquad \text{return}\: n\\
\end{array}$
\end{center}

\begin{center}
\begin{tabularx}{\linewidth}{|*{4}{X|}}\hline
\textbf{a.~~}$evolu(500) = 4$& \textbf{b.~~}$evolu(600) = 5$&\textbf{c.~~}$evolu(300) = 3$&\textbf{d.~~}$evolu(400) = 4$\\ \hline
\end{tabularx}
\end{center}

\bigskip

