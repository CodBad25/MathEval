
\medskip

Cet exercice est un QCM en 5 questions. Pour chacune des questions, une seule des quatre réponses proposées est correcte. Les questions sont \textbf{indépendantes}.

Pour chaque question, indiquer le numéro de la question et recopier sur la copie la lettre correspondante à la réponse choisie. Aucune justification n'est demandée, cependant des traces de recherche au brouillon peuvent aider à trouver la bonne réponse.
Chaque réponse correcte rapporte 1 point. Une réponse incorrecte ou une question sans réponse n'apporte, ni ne retire de point.

\medskip

\textbf{Question 1}

Dans le repère orthogonal suivant on a tracé quatre courbes, chacune associée à une
fonction de variable réelle $x$ et d'expression $\text{e}^{\lambda x}$ où $\lambda$ est un paramètre réel.


\begin{center}
\psset{unit=1cm}
\begin{pspicture*}(-3.1,-0.25)(3,5)
\psaxes[linewidth=1.25pt,Dx=10,Dy=10]{->}(0,0)(-3,-0)(3,5)
\psplot[plotpoints=2000,linewidth=1.25pt]{-3}{3}{2.71828 x 1.8 div neg exp}\uput[l](-2.5,4){\small$\mathcal{C}_f$}
\psplot[plotpoints=2000,linewidth=1.25pt]{-3}{3}{2.71828  x 1.1 div neg exp}\uput[l](-1.6,4){\small$\mathcal{C}_g$}
\psplot[plotpoints=2000,linewidth=1.25pt]{-3}{3}{2.71828 2 x mul exp}\uput[l](0.7,4){\small$\mathcal{C}_h$}
\psplot[plotpoints=2000,linewidth=1.25pt]{-3}{3}{2.71828  x   exp}\uput[r](1.4,4){\small$\mathcal{C}_k$}
\end{pspicture*}
\end{center}

Quelle courbe possède le plus petit paramètre $\lambda$ ? 

\begin{center}
\begin{tabularx}{\linewidth}{|*{4}{>{\centering \arraybackslash}X|}}\hline
\textbf{a.~~}$\mathcal{C}_f$&\textbf{b.~~}$\mathcal{C}_g$&\textbf{c.~~}$\mathcal{C}_h$&\textbf{d.~~}$\mathcal{C}_k$\\ \hline
\end{tabularx}
\end{center}

\medskip

\textbf{Question 2}

On choisit au hasard un couple ayant deux enfants et on note $X$ la variable aléatoire égale au nombre de filles du couple. On admet que la probabilité qu'un enfant soit une fille est égale à 0,5 et qu'il y a indépendance du sexe de l'enfant entre deux naissances.

Déterminer $P(X \geqslant 1)$.

\begin{center}
\begin{tabularx}{\linewidth}{|*{4}{>{\centering \arraybackslash}X|}}\hline
\textbf{a.~~} 0,25&\textbf{b.~~} 0,5&\textbf{c.~~} $\dfrac{1}{3}$&\textbf{d.~~} 0,75\rule[-3mm]{0mm}{9mm}\\ \hline
\end{tabularx}
\end{center}

\medskip

\textbf{Question 3}

\medskip

On a représenté ci-dessous la courbe $\mathcal{C}$ de la fonction sinus dans un repère orthogonal.

\begin{center}
\psset{trigLabels,labelFontSize=\scriptstyle,xunit=0.75cm}
\begin{pspicture*}(-4.8,-1.25)(14,1.25)
\psgrid[gridlabels=0pt,subgriddiv=1,xunit=\psPiH,gridwidth=0.15pt]
\psaxes[dx=\psPiH,linewidth=1.25pt,trigLabelBase=2](0,0)(-4.8,-1.25)(14,1.25)
\psplot[xunit=0.75cm,plotpoints=2000,linewidth=1.25pt]{-4.8}{14}{x RadtoDeg sin}
\psline[linewidth=0.2pt](-4.8,0.4)(14,0.4)
\psdots(-3.55,0.4)(0.42,0.4)(2.7,0.4)(6.675,0.4)(13.,0.4)
\uput[ur](-3.55,0.4){$A_1$}\uput[ul](0.42,0.4){$A_0$}
\uput[ur](2.7,0.4){$A_2$}\uput[ul](6.675,0.4){$A_3$}
\uput[ul](13,0.4){$A_4$}
\uput[dr](11.5,-.8){$\mathcal{C}$}
\end{pspicture*}
\end{center}

$A_0,\, A_1,\, A_2,\, A_3$ et $A_4$ sont des points de $\mathcal{C}$ et ils ont tous la même ordonnée. 

Parmi les segments suivants, lequel a pour longueur la période de la fonction sinus ?

\begin{center}
\begin{tabularx}{\linewidth}{|*{4}{>{\centering \arraybackslash}X|}}\hline
\textbf{a.~~}$\left[A_0~;~A_1 \right]$&\textbf{b.~~}$\left[A_0~;~A_2 \right]$&\textbf{c.~~}$\left[A_0~;~A_3 \right]$&\textbf{d.~~}$\left[A_0~;~A_4 \right]$\\ \hline
\end{tabularx}
\end{center}

\medskip

\textbf{Question 4}

\medskip

Soit la fonction $f$ définie sur $\R$ par $f(x) = 0,5x^2 - 2x + 1$.

On considère l'équation $f(x) = 0$, d'inconnue $x \in \R$.

 L'ensemble des solutions de cette équation est:

\begin{center}
\begin{tabularx}{\linewidth}{|*{4}{>{\centering \arraybackslash}X|}}\hline
\textbf{a.~~}$\emptyset$&\textbf{b.~~}$\left\{2 - \sqrt{2}~;~2 + \sqrt{2}\right\}$&\textbf{c.~~}$\left\{2 - \sqrt{6}~;~2 + \sqrt{6}\right\}$&\textbf{d.~~}$\left\{4 - 2\sqrt{2}~;~4 + 2\sqrt{2}\right\}$\rule[-3mm]{0mm}{9mm}\\ \hline
\end{tabularx}
\end{center}

\medskip

\textbf{Question 5}

\medskip

ABC est un triangle tel que: AB = 5, BC = 2, $\widehat{\text{ABC}} = 60\degres$. 

\begin{center}
\psset{unit=1cm}
\begin{pspicture}(-6,-0.5)(0.5,2)
\pspolygon(0,0)(2;120)(5;180)
\psarc(0,0){0.4}{120}{180}
\uput[r](0,0){B}\uput[u](2;120){C}\uput[l](5;180){A}
\end{pspicture}
\end{center}

La longueur AC est égale à :

\begin{center}
\begin{tabularx}{\linewidth}{|*{4}{>{\centering \arraybackslash}X|}}\hline
\textbf{a.~~}$\sqrt{19}$ &\textbf{b.~~}$\sqrt{21}$ & \textbf{c.~~}$\sqrt{28}$ &\textbf{d.~~}$\sqrt{29}$\rule[-3mm]{0mm}{9mm}\\ \hline
\end{tabularx}
\end{center}

\medskip

