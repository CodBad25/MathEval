
\subsection*{1.}

\begin{center}
\begin{tabular}{|c|c|c|c|}
\hline
 & Nombre de pois jaunes & Nombre de pois verts & Total \\ \hline
Nombre de pois ridés  & 100  & 500  & 600   \\ \hline
Nombre de pois lisses & 200  & 9 200 & 9 400 \\ \hline
Total                 & 300  & 9 700 & 10 000 \\ \hline
\end{tabular}
\end{center}

\subsection*{2.}

Il faut trouver \(P(\overline{J} \cap \overline{R}) = \dfrac{9200}{10000} = 0,92\).

\subsection*{3.}

On a \(P(\overline{J}) = 0,97\).

\subsection*{4.}

\begin{itemize}
    \item Il faut trouver \(P_R(J)\), c'est-à-dire la probabilité de trouver un pois jaune parmi les 600 ridés. Cette probabilité est égale à \(\dfrac{1}{6}\).
    \item Il en résulte que \(P_R(\overline{J}) = 1 - \dfrac{1}{6} = \dfrac{5}{6}\).
\end{itemize}

\subsection*{5.}

On a \(P_J(R) = \dfrac{100}{300} = \dfrac{1}{3}\).

Parmi les pois ridés, il y a une chance sur 3 de choisir un pois jaune.

