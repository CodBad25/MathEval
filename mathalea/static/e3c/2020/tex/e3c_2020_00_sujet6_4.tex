
\medskip

On considère un cône de révolution ayant une génératrice de longueur $20$~cm et d'une hauteur $h$ en cm.

On rappelle que le volume $V$ en cm$^3$ d'un cône de révolution de base un disque d'aire $\mathcal{A}$ en cm$^2$ et de hauteur $h$ en cm est : $V = \dfrac{1}{3}\mathcal{A}h$.

Dans cet exercice, on cherche la valeur de la hauteur $h$ qui rend le volume du cône maximum.

\begin{center}
\psset{unit=1cm,linewidth=1.3pt}
\begin{pspicture}(-2,-0.5)(2,4)
%\psgrid
\psellipse(0,0)(2,0.5)
\psline[linestyle=dashed](2,0)(0,0)(0,4)
\psline(-2,0)(0,4)(2,0)
\uput[r](0,2){$h$}\uput[ur](1,2){20~cm}
\psline(0.3,0)(0.3,0.3)(0,0.3)
\end{pspicture}
\end{center}

\begin{enumerate}
\item Exprimer le rayon de la base en fonction de $h$.
\item Démontrer que le volume du cône, en fonction de sa hauteur $h$, est :

\[V(h) = \dfrac{\pi}{3}\left(400h - h^3\right).\]

\item  Quelle hauteur $h$ choisir pour que le volume du cône soit maximum ?
\end{enumerate}
