
\subsection*{Question 1}

On a :
\begin{align*}
	f_1 &= f_0 + 108 \\
	&= 2500 + 108 = 2608.
\end{align*}

Augmenter de 3,8\%, c'est multiplier par :
\begin{align*}
	1 + \frac{3,8}{100} &= 1 + 0,038 = 1,038.
\end{align*}
Ainsi, 
\begin{align*}
	c_1 &= c_0 \times 1,038 = 2500 \times 1,038 = 2595.
\end{align*}

\subsection*{Question 2}

La suite $(f_n)$ est une suite arithmétique de raison $108$ et de premier terme $f_0 = 2500$, donc :
\begin{align*}
	f_{n+1} &= f_n + 108.
\end{align*}

La suite $(c_n)$ est une suite géométrique de raison $1,038$ et de premier terme $c_0 = 2500$, donc :
\begin{align*}
	c_{n+1} &= c_n \times 1,038.
\end{align*}

\subsection*{Question 3}

On a les expressions suivantes pour $n \in \mathbb{N}$ :
\begin{align*}
	f_n &= 2500 + 108n, \\
	c_n &= 2500 \times 1,038^n.
\end{align*}

\subsection*{Question 4}

On complète l'algorithme pour déterminer le nombre de mois à attendre pour que le nombre potentiel de flacons commandés dépasse celui des flacons produits :
\begin{python}
	n = 0
	f = 2500
	c = 2500
	
	while f > c:
		n = n + 1
		f = f + 108
		c = c * 1.038
\end{python}

\subsection*{Question 5}

Il faut comparer $F_{12}$ et $C_{12}$, les sommes cumulées des flacons produits et commandés après 12 mois.

On calcule $F_{12}$ :
\begin{align*}
	F_{12} &= f_0 + f_1 + \dots + f_{11} \\
	&= 12 \times \left( \frac{f_0 + f_{11}}{2} \right) \\
	&= 12 \times \left( \frac{2500 + (2500 + 11 \times 108)}{2} \right) \\
	&= 12 \times (2500 + 594) = 12 \times 3094 = 37128.
\end{align*}

Pour $C_{12}$, on a une suite géométrique de raison $1,038$ :
\begin{align*}
	C_{12} &= c_0 + c_1 + \dots + c_{11}.
\end{align*}
En utilisant la somme des termes d'une suite géométrique, on a :
\begin{align*}
	C_{12} &= 2500 \times \frac{1,038^{12} - 1}{1,038 - 1} \\
	&\approx 37136.
\end{align*}

Ainsi, à la fin de l'année, le nombre total de flacons commandés ($37136$) sera légèrement supérieur à celui des flacons produits ($37128$), ce qui confirme que le modèle prédit une insuffisance de production.

