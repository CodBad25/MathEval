
\medskip

En 2002, Camille a acheté une voiture, son prix était alors de \np{10500}~\euro. La valeur de cette voiture a baissé de 14\,\% par an.

\medskip

\begin{enumerate}
\item La valeur de cette voiture est modélisée par une suite. On note $P_n$ la valeur de la voiture en l'année $2002+ n$. On a donc $P_0 = \np{10500}$.
	\begin{enumerate}
		\item Déterminer la nature de la suite $\left(P_n\right)$.
		\item Quelle était la valeur de cette voiture en 2010 ?
	\end{enumerate}
\item Camille aimerait savoir à partir de quelle année la valeur de sa voiture est inférieure à \np{1500}~\euro. 

Pour l'aider, on réalise le programme Python incomplet ci- dessous.

	\begin{enumerate}
		\item Recopier et compléter sur votre copie les deux parties en pointillé du programme ci-dessous :

\begin{center}
\begin{tabular}{|l|}\hline
\texttt{def algo( ) :}\\
\quad \texttt{P= \np{10500}}\\
\quad \texttt{n = 2002}\\
\quad \texttt{while P \ldots \ldots:}\\
\qquad \texttt{P= \ldots \ldots.}\\
\qquad \texttt{n=n+1} \\
\quad \texttt{return(n)}\\ \hline
\end{tabular}
\end{center}

		\item Donner la valeur renvoyée par ce programme.
	\end{enumerate}
\end{enumerate}

\bigskip

