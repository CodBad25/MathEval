
\medskip

On considère les deux suites suivantes :

\setlength\parindent{9mm}
\begin{itemize}
\item[$\bullet~~$]la suite $\left(u_n\right)$ définie pour tout entier $n$ par:

\[u_n =  \dfrac{8n-4}{n+1}\]

\item[$\bullet~~$]la suite $\left(u_n\right)$ définie par $v_0 = 0$ et $v_{n+1} = 0,5v_n +3,5$ pour tout entier $n$.
\end{itemize}
\setlength\parindent{0mm}

\medskip

\begin{enumerate}
\item Calculer les termes d'indice 3 des suites $\left(u_n\right)$ et $\left(v_n\right)$.
\item On s'intéresse aux variations de la suite $\left(u_n\right)$.
 Pour cela, on considère la fonction $f$ définie sur $[0~;~+ \infty[$ par :
 
 \[f(x) = \dfrac{8x- 4}{x + 1}\]
 
	\begin{enumerate}
		\item Démontrer que la fonction $f$ est croissante sur $[0~;~+ \infty[$.
		\item En déduire la monotonie de la suite $\left(u_n\right)$.
	\end{enumerate} 
\item On considère l'affirmation suivante :

\[\og \text{pour tout entier }\, n,\: u_n < v_n \fg.\]

Camille pense que cette affirmation est vraie alors que Dominique pense le contraire.

Pour les départager, on réalise le programme suivant écrit en langage Python:

\begin{center}
\begin{tabularx}{0.5\linewidth}{|X|}\hline
\quad \texttt{def algo( )  :}\\
\quad \texttt{n = 0}\\
\quad \texttt{u = -4}\\
\quad \texttt{v = 0}\\
\quad \texttt{while u < v}\\
\qquad \texttt{n = n+1}\\
\qquad \texttt{u = (8*n - 4)/(n + 1)}\\ 
\qquad \texttt{v = 0,5*v + 3,5}\\
\quad \texttt{return(n)}\\ \hline
\end{tabularx}
\end{center}

Le programme renvoie la valeur 11. Qui de Camille ou Dominique a raison ? 

Expliquer.
\end{enumerate}

\bigskip

