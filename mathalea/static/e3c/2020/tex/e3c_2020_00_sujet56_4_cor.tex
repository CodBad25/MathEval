
\subsection*{1.}

L'augmentation en pourcentage de 2016 à 2017 est égale à :
\[
\dfrac{13{,}7 - 12}{12} \times 100 = \dfrac{1{,}7}{12} \times 100 \approx 14{,}17 \, \%.
\]

\subsection*{2.}

De 2016 à 2019, le taux d'évolution a été de : \(\dfrac{18{,}2}{12} \approx 1{,}51667\).

Si \( t \) est le taux moyen sur ces trois ans, on a \( t^3 = 1{,}51667 \).

La calculatrice donne \( t \approx 1{,}14894 \), soit une augmentation moyenne annuelle de 14,89 \% au centième près.

\subsection*{3.}

Augmenter de 15 \% revient à multiplier par \( 1 + \dfrac{15}{100} = 1 + 0{,}15 = 1{,}15 \).

Si \( u_n \) désigne le nombre d'abonnés l'année \( 2016 + n \), on a donc avec \( u_0 = 12 \), quel que soit \( n \in \mathbb{N} \), \( u_{n+1} = 1{,}15 u_n \) : ceci montre que la suite \( (u_n) \) est une suite géométrique de raison \(q = 1,15\) et de premier terme \( u_0 = 12 \).

\subsection*{4.}

On sait que, quel que soit \( n \in \mathbb{N} \), \( u_n = 12 \times 1{,}15^n \).

2020 correspond à \( n = 4 \), d'où :
\[
u_4 = 12 \times 1{,}15^4 \approx 20{,}988,
\]
soit \(20{,}99\) au centième près.

\subsection*{5.}

\begin{center}
\begin{python}
def Seuil() :
    n = 2016
    A = 12
    while A < 40 :
        A = A * 1.15
        n = n + 1
    return n
\end{python}
\end{center}

\(\textit{Remarque : }\) En faisant tourner cet algorithme on constate qu'il s'arrête pour \( n = 9 \), soit en 2025.

