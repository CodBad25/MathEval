
\medskip

Un parfumeur propose l'un de ses parfums, appelé \og Fleur Rose \fg, et cela uniquement avec
deux contenances de flacons : un de \np[ml]{30}  ou un de \np[ml]{50}. Pour l'achat d'un flacon \og Fleur Rose \fg, il propose une offre promotionnelle sur un autre parfum appelé \og Bois d'ébène \fg.

On dispose des données suivantes :

\begin{itemize}
\item 58\,\% des clients achètent un flacon de parfum \og Fleur Rose \fg{} de \np[ml]{30} et, parmi ceux là, 24\,\% achètent également un flacon du parfum \og Bois d'ébène \fg ;
\item 42\,\% des clients achètent un flacon de parfum \og Fleur Rose \fg{} de \np[ml]{50}  et, parmi ceux là,
13\,\% achètent également un flacon du parfum \og Bois d'ébène \fg.
\end{itemize}

On admet qu'un client donné n'achète qu'un seul flacon de parfum \og Fleur de Rose \fg{} (soit en
\np[ml]{30} soit en \np[ml]{50}), et que s'il achète un flacon du parfum \og Bois d'ébène \fg, il n'en achète
aussi qu'un seul flacon.

On choisit au hasard un client achetant un flacon du parfum \og Fleur Rose \fg. On considère les
évènements suivants :
\begin{itemize}
\item $F$ : \og le client a acheté un flacon \og Fleur Rose \fg{} de \np[ml]{30} \fg ;
\item $B$ : \og le client a acheté un flacon \og Bois d'ébène \fg.
\end{itemize}

\medskip

\begin{enumerate}
\item Construire un arbre pondéré traduisant les données de l'exercice.
\item Calculer la probabilité $p(F \cap B)$.
\item Calculer la probabilité que le client ait acheté un flacon \og Bois d'ébène \fg.
\item Un flacon \og Fleur Rose \fg{} de \np[ml]{30} est vendu 40 \euro, un flacon \og Fleur Rose \fg{} de \np[ml]{50} est vendu 60 \euro{} et un flacon \og Bois d'ébène \fg{} 25 \euro.

On note $X$ la variable aléatoire correspondant
au montant total des achats par un client du parfum \og Fleur Rose \fg.
	\begin{enumerate}
		\item  Déterminer la loi de probabilité de $X$.
		\item Calculer l'espérance de $X$ et interpréter le résultat dans le contexte de l'exercice.
	\end{enumerate}
\end{enumerate}

\vspace{0,5cm}

