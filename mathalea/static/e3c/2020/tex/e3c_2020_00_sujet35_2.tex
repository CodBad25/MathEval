  
\medskip

Un organisme de vacances propose des séjours en France et à l'étranger pour des jeunes.
 Ces derniers sont répartis en deux catégories suivant leur âge : adolescents ou jeunes enfants. 
 \begin{itemize}
\item 40\,\% des participants sont des adolescents et parmi eux, 70\,\% choisissent un séjour à l'étranger.
\item Parmi les jeunes enfants, 90\,\% choisissent un séjour en France.
\end{itemize}

On interroge au hasard un participant à un séjour de cet organisme.

On note :
\begin{itemize}
\item $A$ l'évènement \og le participant est un adolescent\fg,
\item $F$ l'évènement \og le participant choisit un séjour en France \fg.
\end{itemize}

\medskip

\begin{enumerate}
\item  Recopier et compléter sur la copie les branches de l'arbre de probabilité ci-dessous pour qu'il représente la situation.

\begin{center}
\psset{nodesepA=0pt,nodesepB=3pt,treesep=0.75,labelsep=0.1pt,levelsep=2.5cm}
\pstree[treemode=R]{\TR{}}
{\pstree{\TR{$A$~~}\taput{}}
	{
	\TR{$F$}\taput{}
	\TR{$\overline{F}$}\tbput{}
	}
\pstree{\TR{$\overline{A}$~~}\tbput{}}
	{\TR{$F$}\taput{}
	\TR{$\overline{F}$}\tbput{}
	}
}
\end{center}

\item Calculer la probabilité que le participant soit un adolescent et qu'il choisisse un séjour à l'étranger.
\item Montrer que la probabilité qu'un participant choisisse un séjour à l'étranger est 0,34.
\item Calculer la probabilité que le participant ne soit pas un adolescent, sachant qu'il part à l'étranger. Donner la valeur arrondie au centième de cette probabilité.
\item On interroge au hasard, et de manière indépendante, deux participants à des séjours de cet organisme pour connaître s'ils ont choisi un séjour en France ou non. L'organisation de ce sondage est telle qu'une même personne peut être interrogée deux fois.

 Calculer la probabilité qu'au moins un des deux participants ait choisi un séjour en France. Donner cette probabilité arrondie au centième.
\end{enumerate}

\vspace{0,5cm}

