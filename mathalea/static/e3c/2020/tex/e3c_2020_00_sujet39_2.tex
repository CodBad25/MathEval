
\medskip

\begin{enumerate}
\item  Soit la fonction $f$ définie sur l'intervalle $[0~;~+\infty[$ par 

\[f(x) = x^2 - 3x + 4.\]

Étudier les variations de $f$ sur $[0~;~+\infty[$.
\item  Dans un repère orthonormé, on considère la courbe $\mathcal{C}$ représentant la fonction racine carrée et le point $A(2~;~0)$.

\begin{center}

\psset{xunit=1cm,yunit=1cm,labelFontSize=\scriptstyle,showorigin=false}
\begin{pspicture}(-2.4,-1.)(5.5,3.6)
\multido{\n=-1.8+0.2}{34}{\psline[linewidth=0.25pt,linecolor=lightgray](\n,-0.99)(\n,3.2)}
\multido{\n=-0.8+0.2}{20}{\psline[linewidth=0.25pt,linecolor=lightgray](-1.8,\n)(4.8,\n)}
\multido{\n=-1+1}{6}{\psline[linewidth=0.45pt](\n,-0.9)(\n,3.2)}
\multido{\n=0+1}{4}{\psline[linewidth=0.45pt](-1.8,\n)(4.8,\n)}
\psaxes[linewidth=1.25pt]{-}(0,0)(-1.8,-0.99)(4.9,3.4)
\def\Func{x 0.5  exp}
\psplot[plotpoints=2000,linewidth=1.25pt,linecolor=blue]{0.01}{4.8}{\Func}
\uput[dl](0,0){O}
\psdots[dotstyle=Mul,dotscale=1.5,](2,0)(3.42,1.849324)
\uput[dr](2,0){\footnotesize A}\uput[ur](3.42,1.849324){\footnotesize M}
\psline[linewidth=0.5pt,linecolor=blue](2,0)(3.42,1.849324)
\psset{linecolor=blue}
\uput[ur](4.4,2.2){ $\mathcal{C}$}
\end{pspicture}
\end{center}

\begin{enumerate}
\item Soit $M(x~;~y)$ un point de $\mathcal{C}$. Exprimer $y$ en fonction de $x$.
\item En déduire que $AM^2 = x^2 -3x + 4$.
\item Déterminer les coordonnées du point de $\mathcal{C}$ le plus proche de A.

\emph{Ce point est noté {\rm B} pour la suite.}
\item Un élève affirme que la tangente en B à $\mathcal{C}$ est perpendiculaire au segment [AB].
A-t-il raison ? Justifier.
\end{enumerate}
\end{enumerate}

\vspace{0,5cm}

