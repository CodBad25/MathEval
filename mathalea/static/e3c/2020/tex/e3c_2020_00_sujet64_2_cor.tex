
\subsection*{1.}

Retrancher 2 \% revient à multiplier par \(1 - \dfrac{2}{100} = 1 - 0{,}02 = 0{,}98\).

On a donc \(d_2 = 50 \times 0{,}98 = 49\) (km).

\subsection*{2.}

Si \(d_n\) est la distance parcourue le \(n\)-ième jour, alors \(d_{n+1} = d_n \times 0{,}98\).

Ceci montre que la suite \((d_n)\) est géométrique de premier terme \(d_1 = 50\) et de raison \(q = 0{,}98\).

\subsection*{3.}

On sait qu'alors, pour tout entier naturel \(n \geqslant 1\), \(d_n = d_1 \times q^{n-1} = 50 \times 0{,}98^{n-1}\).

\subsection*{4.}

\begin{center}
\begin{python}
def nb jours() :
    j = 1
    u = 50
    S = 50
    while S < 2000 :
        u = 0.98 * u
        S = S + u
        j = j + 1
    return j
\end{python}
\end{center}

\subsection*{5.}

L'algorithme s'arrête le \(80^{\text{ème}}\) jour.

