
\subsection*{1.}

\paragraph{a.} On a \( u_1 = 1{,}015 \times 50 = 50{,}75 \).

\paragraph{b.} En février 2018, on compte donc 50 750 créations d'entreprise.

\subsection*{2.}

\paragraph{a.} Pour tout entier naturel \( n \), on a \( u_{n+1} = 1{,}015 u_n \).

La suite \( (u_n) \) est donc une suite géométrique de raison \( q = 1{,}015 \) et de premier terme \( u_0 = 50 \).

\paragraph{b.} Pour tout entier naturel \( n \), on a donc \( u_n = 50 \times 1{,}015^n \).

\paragraph{c.} On calcule :
\begin{align*}
S &= u_0 + u_1 + \dots + u_{11} \\
&= 50 \times \dfrac{1 - 1{,}015^{12}}{1 - 1{,}015} \\
&\approx 652
\end{align*}

Il y a donc bien eu environ 652 000 créations d'entreprise en 2018.

