
\medskip

Ce QCM comprend 5 questions.

Pour chacune des questions, une seule des quatre réponses proposées est correcte.

Les questions sont indépendantes.

Pour chaque question, indiquer le numéro de la question et recopier sur la copie la lettre correspondante à la réponse choisie.

Aucune justification n'est demandée mais il peut être nécessaire d'effectuer des recherches au brouillon pour aider à déterminer votre réponse.

Chaque réponse correcte rapporte $1$ point. Une réponse incorrecte ou une question sans réponse n'apporte ni ne retire de point.

\medskip

\textbf{Question 1}

\medskip

Soit $f$ la fonction définie sur $\R$ par $f(x) = \sin(x) - x$. 

Parmi les propositions suivantes, laquelle est vraie  ?

\begin{center}
\begin{tabularx}{\linewidth}{|*{4}{X|}}\hline
\textbf{a.~~}$f$ est paire&\textbf{b.~~}$f$ est impaire&\textbf{c.~~}Pour tout réel $x$,\: $f(x +2\pi) = f(x)$&\textbf{d.~~}Pour tout réel $x$,\: $f(x +\pi) = - f(x)$\\ \hline
\end{tabularx}
\end{center}

\medskip

\textbf{Question 2}

\medskip

Dans l'intervalle $]-\pi~;~\pi]$, l'équation $2 \cos(x) - \sqrt{3} = 0$ a pour solutions :

\begin{center}
\begin{tabularx}{\linewidth}{|*{4}{X|}}\hline
\textbf{a.~~}$- \frac{\pi}{6}$ et $\frac{\pi}{6}$&\textbf{b.~~}$- \frac{\pi}{4}$ et $\frac{\pi}{4}$&\textbf{c.~~}$- \frac{\pi}{3}$ et $\frac{\pi}{3}$&\textbf{d.~~}$- \frac{2\pi}{3}$ et $\frac{2\pi}{3}$\\ \hline
\end{tabularx}
\end{center}
\medskip

\textbf{Question 3}

\medskip

\parbox{0.65\linewidth}{Soit ABCD un parallélogramme tel que :

AB $= 3$, AD $= 4$ et $\widehat{\text{BAD}} = \dfrac{\pi}{3}$.

Alors $\vect{\text{DA}} \cdot  \vect{\text{DC}}$ est égal à :} \hfill
\parbox{0.33\linewidth}{\psset{unit=0.8cm}
\begin{pspicture}(0,-0.5)(4.5,3)
\pspolygon(0,0)(4,0)(5.5,2.6)(1.5,2.6)%ADCB
\uput[dl](0,0){A}\uput[dr](4,0){D}\uput[ur](5.5,2.6){C}\uput[ul](1.5,2.6){B}
\psarc(0,0){0.6}{0}{60}
\rput(0.8,0.5){$\frac{\pi}{3}$}
\end{pspicture}}

\begin{center}
\begin{tabularx}{\linewidth}{|*{4}{X|}}\hline
\textbf{a.~~}$12$&\textbf{b.~~}$- 12$&\textbf{c.~~}$6$&\textbf{d.~~}$- 6$\\ \hline
\end{tabularx}
\end{center}

\medskip

\textbf{Question 4}

\medskip

Le plan est muni d'un repère orthonormé \Oij.

On considère la droite $\left(d_1\right)$ d'équation $3x - 4y + 1 = 0$. La droite $\left(d_2\right)$ perpendiculaire à $\left(d_1\right)$ et passant par le point A(1~;~1) a pour équation :

\begin{center}
\begin{tabularx}{\linewidth}{|*{4}{X|}}\hline
\textbf{a.~~}$4x + 3y = 0$&\textbf{b.~~}$4x + 3y - 7 = 0$&\textbf{c.~~}$x + y - 2 = 0$&\textbf{d.~~}$-4x + 3y + 1 = 0$\\ \hline
\end{tabularx}
\end{center}

\medskip

\textbf{Question 5}

\medskip
Le plan est muni d'un repère orthonormé \Oij. Les droites $(d)$ et $\left(d'\right)$ d'équations respectives $2x - y + 5 = 0$ et $-4x + 2y + 7 = 0$ sont:

\begin{center}
\begin{tabularx}{\linewidth}{|*{4}{X|}}\hline
\textbf{a.~~}confondues&\textbf{b.~~}sécantes&\textbf{c.~~}parallèles&\textbf{d.~~}perpendiculaires\\ \hline
\end{tabularx}
\end{center}

\bigskip

