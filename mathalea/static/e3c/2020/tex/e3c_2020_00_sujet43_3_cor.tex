
\subsection*{1.}

On sait que :  et , donc :
\[
\begin{cases}
A(-1\,;\,3) \in (AB) \\
B(5\,;\,0) \in (AB)
\end{cases}
\iff
\begin{cases}
3 = -a + b \\
0 = 5a + b
\end{cases}
\Rightarrow -3 = 6a \iff a = -\dfrac{1}{2},
\]
puis \( b = a + 3 = -\dfrac{1}{2} + 3 = \dfrac{5}{2} \).

On a donc :
\[
M(x\,;\,y) \in (AB) \iff y = -\dfrac{1}{2}x + \dfrac{5}{2}.
\]

\subsection*{2.}

\begin{align*}
&M(x\,;\,y) \in \mathcal{D} \\
\iff &\overrightarrow{CM} \cdot \vec{n} = 0 \\
\iff &- (x - 9) + 3(y - 3) = 0 \\
\iff& -x + 3y = 0.
\end{align*}

\subsection*{3.}

\( \mathcal{D} \) a pour vecteur directeur \( \vec{d} \begin{pmatrix} -3 \\ -1 \end{pmatrix} \) et \( (AB) \) a pour vecteur directeur \( \overrightarrow{AB} \begin{pmatrix} 6 \\ -3 \end{pmatrix} \).

Or, \( \det\left(\vec{d}, \overrightarrow{AB}\right) = 9 + 6 = 15 \neq 0 \) : les droites \( \mathcal{D} \) et \( (AB) \) ne sont pas parallèles.

\subsection*{4.}

\[
\vec{d} \cdot \overrightarrow{AB} = -3 \times 6 + (-1) \times (-3) = -18 + 3 = -15 \neq 0,
\]
ces vecteurs ne sont pas orthogonaux, donc les droites \( \mathcal{D} \) et \( (AB) \) ne sont pas perpendiculaires.

\subsection*{5.}

On a :
\[AC^2 = (9 - (-1))^2 + (3 - 3)^2 = 10^2,
\]
donc \( AC = 10 \).

D'après le théorème d'Al-Kashi :
\begin{align*}
AC^2 &= AE^2 + EC^2 - 2 \times AE \times EC \times \cos \widehat{AEC} \\
100 &= \left(2\sqrt{5}\right)^2 + \left(2\sqrt{10}\right)^2 - 2 \times 2\sqrt{5} \times 2\sqrt{10} \times \cos \widehat{AEC} \\
100 &= 20 + 40 - 8\sqrt{50} \cos \widehat{AEC} \\
8\sqrt{50} \cos \widehat{AEC} &= -40,
\end{align*}

d'où l'on exprime :
\begin{align*}
\cos \widehat{AEC} &= -\dfrac{5}{\sqrt{50}} \\
&= -\dfrac{5}{\sqrt{5^2 \times 2}} \\
&= -\dfrac{5}{5\sqrt{2}} \\
&= -\dfrac{5}{5\sqrt{2}} \\
&= -\dfrac{1}{\sqrt{2}} \\
&= -\dfrac{\sqrt{2}}{2}.
\end{align*}

On en déduit que \( \widehat{AEC} = \dfrac{3\pi}{4} \).

