	
	\subsection*{1.}
	
	\subsubsection*{a.}
	
	On a entré dans la cellule B3 : "=1,05*B2-12"
	
	\subsubsection*{b.}
	
	20 \% de 1 000 représentent \(0,20 \times 1000 = 200\). Il faut donc attendre la cinquième année.
	

	\subsection*{2.}
	
	On a pour tout naturel \(n\),
\begin{align*}
			v_{n+1}& = u_{n+1} - 240\\
			& = 1,05u_n - 12 - 240 \\
			&= 1,05u_n - 252 \\
			&= 1,05(u_n - \dfrac{252}{1,05}) \\
			&= 1,05(u_n - 240) \\
			&= 1,05v_n
\end{align*}

	
	
	L'égalité, vraie pour tout naturel \(n\), \(v_{n+1} = 1,05v_n\) montre que la suite \((v_n)\) est géométrique de raison 1,05 et de premier terme \(v_0 = u_0 - 240 = 1000 - 240 = 760\).
	
	\subsection*{3.}
	
	On sait qu'alors pour tout naturel \(n\), \(v_n = 760 \times 1,05^n\).
	
	Or \(v_n = u_n - 240\) entraîne que \(u_n = v_n + 240 = 760 \times 1,05^n + 240\).
	
	\subsection*{4.}
	
	Application : pour \(n = 20\),
	\[
	u_{20} = 760 \times 1,05^{20} + 240 \approx 2256,51
	\]
	
