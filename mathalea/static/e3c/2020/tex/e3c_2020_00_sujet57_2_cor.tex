
\( f(x) = x^3 + 3x^2 + 3x - 63 \).

\subsection*{1.}

\(f'(x) = 3x^2 + 6x + 3 = 3(x^2 + 2x + 1) = 3(x + 1)^2\).

\subsection*{2.}

Produit de deux nombres positifs, \(f'(x)\) est positive quel que soit le réel \(x\).

\subsection*{3.}

Puisque \(f(x) \geqslant 0\), la fonction \(f\) est croissante sur \(\mathbb{R}\).

\subsection*{4.}

Une équation réduite de la tangente à la courbe \( \mathcal{C} \) au point d'abscisse \(-1\) est :
\[
y - f(-1) = f'(-1)(x + 1).
\]
Avec :
\[
f(-1) = -1 + 3 - 3 - 63 = -64 \text{ et } f'(-1) = 3(-1 + 1)^2 = 3 \times 0 = 0,
\]
l'équation devient \(y - (-64) = 0\) ou \(y = -64\).

La droite a un coefficient directeur égal à 3. La tangente à la courbe \( \mathcal{C} \) est parallèle à la droite si :

\begin{align*}
&f'(x) = 3 \\
\iff &3(x + 1)^2 = 3 \\
\iff &(x + 1)^2 = 1 \\
\iff &(x + 1)^2 - 1 = 0 \\
\iff &(x + 1 + 1)(x + 1 - 1) = 0 \\
\iff &x(x + 2) = 0.
\end{align*}

Il y a donc deux solutions : \( x = 0 \) et \( x = -2 \).

Pour \( x = 0 \), \( f(0) = -63 \) et pour \( x = -2 \), \( f(-2) = -8 + 12 - 6 - 63 = -65 \). 

Les tangentes aux points \( (0 \,;\, -63) \) et \( (-2 \,;\, -65) \) sont parallèles à la droite d'équation \( y = 3x - 100 \).


