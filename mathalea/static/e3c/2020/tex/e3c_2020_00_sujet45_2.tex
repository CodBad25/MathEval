
\medskip

Une urne contient deux boules rouges et trois boules noires toutes indiscernables au toucher.

On tire au hasard une première boule en notant sa couleur puis on la remet dans l'urne.

On tire ensuite toujours au hasard une deuxième boule en notant sa couleur.

On note $R$ l'évènement \og tirer une boule rouge \fg{} et $N$ l'évènement \og tirer une boule noire \fg.

\medskip

\begin{enumerate}
\item Recopier et compléter sur la copie l'arbre pondéré ci-dessous associé à cette expérience. 

\begin{center}
\pstree[treemode=R,nodesepA=0pt,nodesepB=3pt,levelsep=2.5cm]{\TR{}}
{
\pstree{\TR{$R$~~}\taput{\ldots}}
	{\TR{$R$} \taput{\ldots}
	\TR{$N$} \tbput{\ldots}
	}
\pstree{\TR{$N$~~}\tbput{\ldots}}
	{\TR{$R$}\taput{\ldots}
	\TR{$N$}\tbput{\ldots}
	}
}
\end{center}
	
\item Quelle est la probabilité de tirer deux boules rouges ?
\item Si un joueur tire une boule rouge, il gagne $20$ euros. S'il tire une boule noire, il perd $10$ euros.

On note $X$ la variable aléatoire égale au gain algébrique du joueur, en euros, à l'issue des deux tirages successifs.

Déterminer la loi de probabilité de la variable aléatoire $X$.
\item Calculer la probabilité que le joueur gagne de l'argent.
\item Calculer l'espérance de la variable aléatoire $X$ et en donner une interprétation.
\end{enumerate}

\bigskip

