
\subsection*{Question 1}

Une équation de la tangente à la courbe représentative de \(g\) au point d'abscisse 2 est :
\[
y - g(2) = g'(2)(x - 2).
\]
Avec \(g'(x) = 4x + 5\), \(g'(2) = 8 + 5 = 13\) et \(g(2) = 8 + 10 - 4 = 14\), l'équation devient :
\begin{align*}
y - 14 &= 13(x - 2) \\
y &= 13x - 26 + 14 \\
y &= 13x - 12.
\end{align*}

\subsection*{Question 2}

Avec \(\overrightarrow{AD} \begin{pmatrix} -2 \\ 3 \end{pmatrix}\) et \(\overrightarrow{BD} \begin{pmatrix} -7 \\ 5 \end{pmatrix}\), on a :
\[
\overrightarrow{AD} \cdot \overrightarrow{BD} = 14 + 15 = 29.
\]

\subsection*{Question 3}

Une équation de la parallèle \( \mathcal{D}' \) à \( \mathcal{D} \) contenant \( A \) est de la forme :
\[
M(x\,;\,y) \in \mathcal{D}' \iff 3x - 4y + c = 0.
\]
Or :
\[
A(4\,;\,8) \in \mathcal{D}' \iff 3 \times 4 - 4 \times 8 + c = 0,
\]
d'où  \(c = 20\).

Une équation de \( \mathcal{D}' \) est donc \( 3x - 4y + 20 = 0 \).

\subsection*{Question 4}

On sait que pour tout entier naturel \( n \), \( u_n = u_0 \times q^n = 10 \times (-1{,}2)^n \).

En particulier :
\[
u_{3000} = 10 \times (-1{,}2)^{3000} \approx 3 \times 10^{238} > 1000.
\]

\subsection*{Question 5}

\begin{center}
\begin{python}
def algo() :
    V = 1
    n = 0
    while V < 100000 :
        n = n + 1
        V = 4 * V + 2
    return(n)
\end{python}
\end{center}
Le programme donnera \( n = 8 \).

