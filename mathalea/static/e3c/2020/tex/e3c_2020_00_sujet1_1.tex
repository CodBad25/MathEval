
\medskip

Ce QCM comprend 5 questions.

Pour chacune des questions, une seule des quatre réponses proposées est correcte.

Les questions sont indépendantes.

Pour chaque question, indiquer le numéro de la question et recopier sur la copie la lettre correspondante à la réponse choisie.

Aucune justification n'est demandée mais il peut être nécessaire d'effectuer des recherches au brouillon pour aider à déterminer votre réponse.

Chaque réponse correcte rapporte $1$ point. Une réponse incorrecte ou une question sans réponse n'apporte ni ne retire de point.

\medskip

\textbf{Question 1}

ABC est un triangle tel que AB = 5, AC = 6 et $\widehat{\text{BAC}}= \dfrac{\pi}{4}$. Alors $\vect{\text{AB}} \cdot \vect{\text{AC}}$ est égal à :

\begin{center}
\begin{tabularx}{\linewidth}{|*{4}{X|}}\hline
\textbf{a.~~}$15\sqrt{2}$&\textbf{b.~~}$15\sqrt{3}$&\textbf{c.~~}$\dfrac{15}{2}$\rule[-3mm]{0mm}{9mm}&\textbf{d.~~}$15$\\ \hline
\end{tabularx}
\end{center}

\medskip

\textbf{Question 2}

ABCD est un carré de centre O tel que AB $=1$. Alors $\vect{\text{AB}} \cdot \vect{\text{OB}}$ est égal à :

\begin{center}
\begin{tabularx}{\linewidth}{|*{4}{X|}}\hline
\textbf{a.~~}$1$&\textbf{b.~~}$0$&\textbf{c.~~}$0,5$&\textbf{d.~~}$- 1$\\ \hline
\end{tabularx}
\end{center}

\textbf{Question 3} 

$\vect{u}$ et $\vect{u}$ sont deux vecteurs orthogonaux tels que $\|\vect{u}\| = 2$ et $\|\vect{v}\| = 1$. 

$\left(\vect{u}+\vect{v}\right).\left(2\vect{u} - \vect{v}\right)$ est égal à :

\begin{center}
\begin{tabularx}{\linewidth}{|*{4}{X|}}\hline
\textbf{a.~~}$ 6$&\textbf{b.~~}$9$&\textbf{c.~~}$13$&\textbf{d.~~}$7$\\ \hline
\end{tabularx}
\end{center}

On se place dans un repère orthonormé du plan.

Sur la figure ci-dessous, on a tracé la courbe représentative notée $C$ d'une fonction $f$ définie sur $\R$.

La droite $D$ est tangente à la courbe $C$ au point A(5~;~0).

\begin{center}
\psset{unit=0.8cm}
\begin{pspicture}(-6,-2)(8,5.25)
\psgrid[gridlabels=0pt,gridwidth=0.4pt,subgriddiv=1]
\psaxes[linewidth=1.25pt]{->}(0,0)(-6,-2)(8,5)
%\psplot[plotpoints=2000,linewidth=1.25pt,linecolor=red]{-5}{8}{x 4 add x 6 sub mul x 4 exp 1 add div 0.25 sub}
%\psplot[plotpoints=2000,linewidth=1.25pt,linecolor=orange]{-4.9}{2}{4 x 3 exp 3 div x dup mul 2.5 mul add 4 x mul add 3 mul 26 div sub}
%\psplot[plotpoints=2000,linewidth=1.25pt,linecolor=red]{-5}{8}{x 4 add x 5 sub mul x dup mul  9 x mul add 0.25 sub mul 2.7182 0.0001 x mul exp div}%   5 div neg 
\psplot[plotpoints=2000,linewidth=1.25pt]{-5}{8}{5 3 div x 3  div sub}
%\pscurve[linewidth=1.25pt,linecolor=blue](-4.85,5)(-4.82,4)(-4.7,3)(-4.5,1)(-4,0)(-3,1.3)(-2,3.75)(-1,4.4)(0,4)(1,3.1)(2,2)(3,1.1)(4,0.45)(5,0)(6,-0.25)(7,-0.35)(8,-0.4)
\psbezier[linewidth=1.25pt,linecolor=red](-4.85,5)(-4.5,1)(-4.6,0.4)(-4,0)
\psbezier[linewidth=1.25pt,linecolor=red](-4,0)(-2.5,0.1)(-3,4)(-1,4.35)
\psbezier[linewidth=1.25pt,linecolor=red](-1,4.35)(2,3.8)(2.25,0.6)(5,0)
\psbezier[linewidth=1.25pt,linecolor=red](5,0)(7,-0.6)(8,-1)(8,-1)
\end{pspicture}
\end{center}

\textbf{Question 4}

\medskip

On note $f'$ la dérivée de la fonction $f$, Alors $f'(5)$ est égal à :

\begin{center}
\begin{tabularx}{\linewidth}{|*{4}{X|}}\hline
\textbf{a.~~}$3$&\textbf{b.~~}$- 3$&\textbf{c.~~}$\dfrac{1}{3}$\rule[-3mm]{0mm}{9mm}&\textbf{d.~~}$- \dfrac{1}{3}$\\ \hline
\end{tabularx}
\end{center}

\textbf{Question 5}

\medskip

Pour tout réel $x$ de l'intervalle $] - \infty~;~ 0]$, on a :

\begin{center}
\begin{tabularx}{\linewidth}{|*{4}{X|}}\hline
\textbf{a.~~}$f'(x) \leqslant 0$&\textbf{b.~~}$f'(x) \geqslant 0$&\textbf{c.~~}$f(x) \geqslant 0$&\textbf{d.~~}$f'(x) \leqslant 0$\\ \hline
\end{tabularx}
\end{center}

\bigskip

