
\subsection*{Question 1}

\(3 \times \dfrac{10^n}{2n+1} = 3 \times \dfrac{10^n}{2n \times 2} = 3 \times \left(\dfrac{10}{2}\right)^n \times \dfrac{1}{2} = \dfrac{3}{2} \times 5^n\).

\((u_n)\) est donc une suite géométrique de premier terme \(\dfrac{3}{2}\) (ou 1,5) et de raison 5.

\subsection*{Question 2}

Avec \(\overrightarrow{AB} \begin{pmatrix} 4 \\ 3 \end{pmatrix}\) et \(\overrightarrow{CM} \begin{pmatrix} x + 1 \\ y - 1 \end{pmatrix}\), on a :
\begin{align*}
&M(x\,;\,y) \in \Delta \\
\iff &\overrightarrow{CM} \cdot \overrightarrow{AB} = 0 \\
\iff &4(x + 1) + 3(y - 1) = 0 \\
\iff &4x + 4 + 3y - 3 = 0 \\
\iff &4x + 3y + 1 = 0.
\end{align*}

\subsection*{Question 3}

\begin{align*}
&2\cos(x + \pi) + 1 = 0 \\
\iff &2\cos(x + \pi) = -1 \\
\iff &\cos(x + \pi) = -\dfrac{1}{2}.
\end{align*}

Or on sait que \(\cos\left(\dfrac{2\pi}{3}\right) = -\dfrac{1}{2}\), on a donc :
\[
\cos(x + \pi) = \cos\left(\dfrac{2\pi}{3}\right),
\]

puis :
\[
(x + \pi) = \dfrac{2\pi}{3} \quad \text{ou} \quad (x + \pi) = -\dfrac{2\pi}{3},
\]

soit :
\[
x = -\dfrac{\pi}{3} \quad \text{ou} \quad x = -\dfrac{5\pi}{3}.
\]

La première n'appartient pas à l'intervalle \(\left[0 \,;\, \dfrac{\pi}{2} \right]\) ; c'est donc la seconde :
\[
-\dfrac{5\pi}{3} + 2\pi = -\dfrac{5\pi}{3} + \dfrac{6\pi}{3} = \dfrac{\pi}{3}.
\]

\subsection*{Question 4}

Le dénominateur étant suprérieur ou égal à 1, la fonction est dérivable et en la dérivant comme un quotient :
\[
f'(x) = \dfrac{\e^x (1 + \e^x) - \e^x \times \e^x}{(1 + \e^x)^2} = \dfrac{\e^x}{(1 + \e^x)^2}.
\]

\subsection*{Question 5}

\( f(x) \) est un trinôme du second degré de coefficient principal \(-4{,}5 < 0\) ; sa représentation graphique est une parabole dont la concavité est tournée vers le bas.

On voit que le maximum de \( f \) est obtenu pour \( x = -2 \) et qu'alors ce maximum est égal à \( 4{,}5 \). Les propositions A, B et D sont donc fausses.

On voit que \( f(-5) = f(1) = 0 \) : donc \(-5\) et \(1\) sont les racines du trinôme. On sait que ce trinôme est négatif sauf entre les racines ; c'est bien ce qui est indiqué dans le tableau C.

