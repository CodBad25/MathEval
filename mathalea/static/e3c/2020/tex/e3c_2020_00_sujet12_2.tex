
\medskip

Aujourd'hui les chardons (une plante vivace) ont envahi 300 m$^2$ des champs d'une région.
 Chaque semaine, la surface envahie augmente de 5\,\% par le développement des racines, auquel s'ajoutent 15 m$^2$ suite à la dissémination des graines.
 
Pour tout entier naturel $n$, on note $u_n$ la surface envahie par les chardons, en m$^2$, après $n$ semaines; on a donc $u_0 = 300$~m$^2$.

\medskip

\begin{enumerate}
\item 
\begin{enumerate}
\item Calculer $u_1$ et $u_2$.
\item Montrer que la suite $\left(u_n\right)$ ainsi définie, n'est ni arithmétique ni géométrique.
\end{enumerate}
\end{enumerate}	

On admet dans la suite de l'exercice que, pour tout entier naturel $n$,\: $u_{n+1} = 1,05u_n + 15$.
\begin{enumerate}[resume]
\item On considère la suite $\left(v_n\right)$, définie pour tout entier naturel $n$, par: $v_n = u_n + 300$.
\begin{enumerate}
\item Calculer $v_0$, puis montrer que la suite $\left(v_n\right)$ est géométrique de raison $q = 1,05$.
\item Pour tout entier naturel $n$, exprimer $v_n$ en fonction de $n$, puis montrer que $u_n = 600 \times 1,05^n - 300$.
\end{enumerate}
\item Est-il correct d'affirmer que la surface envahie par les chardons aura doublé au bout de $8$ semaines ? Justifier la réponse.
\end{enumerate}

\bigskip

