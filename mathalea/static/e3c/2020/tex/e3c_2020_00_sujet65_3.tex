
\medskip

Un complexe cinématographique a ouvert ses portes en 2018 en périphérie d'une ville.
En 2018, le complexe a accueilli 180 mille spectateurs. La gestionnaire du complexe prévoit une augmentation de 4\,\% par an de la fréquentation du complexe.

Soit $n$ un entier naturel. On note $u_n$ le nombre de spectateurs, en milliers, du complexe cinématographique pour l'année $(2018+n)$. On a donc $u_0=180$.

\medskip

\begin{enumerate}
\item  Étude de la suite $\left(u_n\right)$.
	\begin{enumerate}
		\item  Calculer le nombre de spectateurs en 2019.
		\item Justifier que la suite $\left(u_n\right)$ est géométrique. Préciser sa raison.
		\item Exprimer $u_n$ en fonction de $n$, pour tout entier naturel $n$.
	\end{enumerate}
\item Un cinéma était déjà installé au centre-ville. En 2018, il a accueilli \np{260 000} spectateurs. Avec l'ouverture du complexe, le cinéma du centre-ville prévoit de perdre \np{10 000} spectateurs par an.
Pour $n$, entier naturel, on note $v_n$ le nombre de spectateurs, en milliers, accueillis dans le cinéma du centre-ville l'année $(2018 + n)$. On a donc $v_0 = 260$.
	\begin{enumerate}
		\item Quelle est la nature de la suite $\left(v_n\right)$ ?
		\item On donne le programme ci-dessous, écrit en Python.

\begin{python}
def cinema() :
	n = 0
	u = 180
	v = 260
	while u < v :
		n = n + 1
		u = 1.04*u
		v = v - 10
	return n
\end{python}
	\end{enumerate}
Quelle est la valeur renvoyée lors de l'exécution de la fonction cinema() ? 

L'interpréter dans le contexte de l'exercice.
\end{enumerate}

\vspace{0,5cm}

