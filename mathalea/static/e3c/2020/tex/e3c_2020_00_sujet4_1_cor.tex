	\section*{Exercice 1 (5 points)}
	
	\subsection*{Question 1}
	Le milieu $I$ de $[AB]$ a pour coordonnées $(3, 4)$. Si $(\Delta)$ est la médiatrice de $[AB]$, alors
	\[
	M(x, y) \in (\Delta) \iff \overrightarrow{IM} \cdot \overrightarrow{AB} = 0 \iff -2(x - 3) + 4(y - 4) = 0 \iff -2x + 4y - 10 = 0 \iff x - 2y + 5 = 0
	\]
	La réponse correcte est \textbf{b.}
	
	\subsection*{Question 2}
	Les vecteurs $\overrightarrow{MP}$ et $\overrightarrow{MN}$ sont orthogonaux, donc les droites $(MP)$ et $(MN)$ sont perpendiculaires : le triangle $MNP$ est donc rectangle en $M$ et l’ensemble des points est le cercle de diamètre $[NP]$.\\ La réponse correcte est \textbf{b.}
	
	\subsection*{Question 3}
	On a $g'(x) = 3x^2 - 4$ et en particulier $g'(-1) = 3 - 4 = -1$.\\ Si $t$ est la tangente, $M(x, y) \in T \iff y - g(-1) = g'(-1)(x - (-1))$.\\ Avec $g(-1) = -1 + 4 + 5 = 8$, on a donc :
	\[
	M(x, y) \in T \iff y - 8 = -1(x + 1) \iff y = -x + 7
	\]
	La réponse correcte est \textbf{c.}
	
	\subsection*{Question 4}
	En écrivant $y = x^2 + x + 3 = \left( x + \dfrac{1}{2} \right)^2 - \dfrac{1}{4} + 3 = \left( x + \dfrac{1}{2} \right)^2 + \dfrac{11}{4}$, on voit que la droite d’équation $x = -0,5$ est axe de symétrie de la parabole.\\
	 La réponse correcte est \textbf{d.}
	
	\subsection*{Question 5}
	L’inéquation peut s’écrire $3e^{x+2} < 3e^4$ ou en simplifiant par 3 :\\ $e^{x+2} < e^4$ et en multipliant par $e^{-4}$, $e^{0} < e^{x-2}$, soit finalement par croissance de la fonction exponentielle, $0 < x - 2$ ou $x < 2$. L’ensemble des solutions est l’intervalle $] -\infty, 2[$.\\
	 La réponse correcte est \textbf{c.}
	
