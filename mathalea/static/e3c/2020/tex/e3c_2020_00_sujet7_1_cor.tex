	\section*{Exercice 1 (5 points)}
	
	
	\subsection*{Question 1}
	

On a $\overrightarrow{EF} \left( \begin{array}{c} 4 \\ 6 \end{array} \right)$ et $\overrightarrow{EC} \left( \begin{array}{c} 10 \\ 15 \end{array} \right)$.\\
$Det\left(\overrightarrow{EF};\overrightarrow{EC}\right) =\begin{vmatrix} 4&10\\6&15\end{vmatrix}=60-60=0$\\
Ces deux vecteurs sont donc colinéaires. Les droites $(EF)$ et $(EC)$ sont parallèles et donc confondues : le point $C$ appartient à la droite $(EF)$. La réponse correcte est \textbf{c.}
	
	\subsection*{Question 2}
	
L’équation s’écrit $y + 2x - 4 = 0$ et on sait qu’alors le vecteur $\overrightarrow{n_4} \left( \begin{array}{c} -2 \\ 1 \end{array} \right)$ est un vecteur normal à la droite. La réponse correcte est \textbf{d.}
	
	\subsection*{Question 3}
	
\[
	\overrightarrow{AD} \cdot \overrightarrow{AI} = \overrightarrow{AD} \cdot (\overrightarrow{AB} + \overrightarrow{BI}) = \overrightarrow{AD} \cdot \overrightarrow{AB} + \overrightarrow{AD} \cdot \overrightarrow{BI} = 0 + AD \times \dfrac{1}{2} BC = 6 \times 3 = 18
	\]
	
La réponse correcte est \textbf{b.}
	
	\subsection*{Question 4}
	
On a $\dfrac{14\pi}{3} - 4\pi = \dfrac{14\pi}{3} - \dfrac{12\pi}{3} = \dfrac{2\pi}{3}$ : point E.\\
La réponse correcte est \textbf{a.}
	
	\subsection*{Question 5}
	
Le réel $x$ est représenté par un point du deuxième cadran ; $\cos x$ est donc négatif avec $\sin^2 x + \cos^2 x = 1 \Leftrightarrow \cos^2 x = 1 - \sin^2 x = 1 - 0,8^2 = 1 - 0,64 = 0,36 = 0,6^2$. Conclusion : $\cos x = -0,6$.\\
La réponse correcte est \textbf{b.}
	
