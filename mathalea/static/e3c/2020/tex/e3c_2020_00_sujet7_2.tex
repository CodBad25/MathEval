
\medskip
Un magasin de téléphonie mobile lance une offre sur ses smartphones de la marque Pomme vendus à $800$~\euro{} : il propose une assurance complémentaire pour $50$~\euro{} ainsi qu'une coque à $20$~\euro.

Ce magasin a fait les constatations suivantes concernant les acheteurs de ce smartphone :

\setlength\parindent{9mm}
\begin{itemize}
\item[$\bullet~~$] 40\,\% des acheteurs ont souscrit à l'assurance complémentaire.
\item[$\bullet~~$] Parmi les acheteurs qui ont souscrit à l'assurance complémentaire, 20\,\% ont acheté en plus la coque.
\item[$\bullet~~$] Parmi les acheteurs qui n'ont pas souscrit à l'assurance complémentaire, deux sur trois n'ont pas acheté la coque.
\end{itemize}
\setlength\parindent{0mm}

On interroge au hasard un client de ce magasin ayant acheté un smartphone de la marque Pomme.

On considère les évènements suivants:

\setlength\parindent{9mm}
\begin{itemize}
\item[$\bullet~~$]$A$ : \og le client a souscrit à l'assurance complémentaire \fg{} ;
\item[$\bullet~~$]$C$ : \og le client a acheté la coque \fg.
\end{itemize}
\setlength\parindent{0mm}

\medskip

\begin{enumerate}
\item Calculer la probabilité que le client ait souscrit à l'assurance complémentaire et ait acheté la coque.
\item Montrer que $P(C) = 0,28$.
\item Le client interrogé a acheté la coque.

Quelle est la probabilité qu'il n'ait pas souscrit à l'assurance complémentaire ? 
\item Déterminer la dépense moyenne d'un client de ce magasin ayant acheté un smartphone de la marque Pomme.

On pourra noter $X$ la variable aléatoire qui représente la dépense en euros d'un client de ce magasin ayant acheté un smartphone de la marque Pomme.
\end{enumerate}

\bigskip

