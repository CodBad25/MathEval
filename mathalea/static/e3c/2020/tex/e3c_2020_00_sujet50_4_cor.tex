
\subsection*{1.}

Pour \(n \geqslant 2\), on considère la fonction Python suivante :

\begin{center}
\begin{python}
def saut(n) :
    s = 8
    for k in range(2, n + 1) :
        s = s + 0.1
    return s
\end{python}
\end{center}

\paragraph{a.} La commande \texttt{saut(4)} renvoie comme valeur de \(s\), \(8 + 4 \times 0{,}1 = 8{,}4\).

\paragraph{b.} Ceci signifie que la 5e semaine, Fanny fera des penta bonds de \(8{,}40\) m.

Chaque semaine, le penta bond est incrémenté de \(0{,}1\) m, donc \(s_n = 8 + n \times 0{,}1\).

\subsection*{2.}

\paragraph{a.} En gagnant \(0{,}1\) m chaque semaine, il faudra à Fanny pour atteindre 12 m :
\[
\dfrac{12 - 8}{0{,}1} = 40 \text{ semaines}.
\]

\paragraph{b.} Il faut résoudre l'équation \(s_n = 12\), soit :
\begin{align*}
&8 + 0{,}1 \times n = 12 \\
\iff &0{,}1 \times n = 4 \\
\iff &n = 40.
\end{align*}

Fanny fera un penta bond de 12 m la 41e semaine.

