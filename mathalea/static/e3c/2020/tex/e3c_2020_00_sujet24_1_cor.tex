	
	\subsection*{Question 1}
	
	Un vecteur directeur de la droite est 
	$\vec u\begin{pmatrix} -5 \\ 4 \end{pmatrix}$, donc un vecteur normal est par exemple $\vec{n}  \begin{pmatrix} 4 \\ 5 \end{pmatrix}$.
	
	\subsection*{Question 2}
	
	Le point $H(3 ; 4)$ appartient à la droite $(d)$ si et seulement si :
	\[
	4 \times 3 + 5 \times 4 - 32 = 0
	\]
	ce qui est vrai.
	
	De plus, le vecteur $\overrightarrow{AH} \begin{pmatrix} -4 \\ -5 \end{pmatrix}$ est bien orthogonal au vecteur directeur de la droite d'équation $4x + 5y - 32 = 0$, soit $\vec d \begin{pmatrix} -5 \\ 4 \end{pmatrix}$.
	
	Ainsi, le vecteur $\overrightarrow{AH} \begin{pmatrix} -4 \\ -5 \end{pmatrix}$ est un vecteur normal à la droite.
	
	\subsection*{Question 3}
	
	Le point $M(x ; y)$ appartient au cercle $\mathcal{C}(A, R = 2)$ si et seulement si :
	\[
	AM^2 = 2^2 \quad \text{soit} \quad (x - (-1))^2 + (y - 3)^2 = 4,
	\]
	ce qui équivaut à :
	\[
	(x + 1)^2 + (y - 3)^2 = 4.
	\]
	
	\subsection*{Question 4}
	
	L'équation de la parabole est donnée par :
	\[
	y = 3x^2 - 9x + 5.
	\]
	On peut la réécrire sous forme canonique :
	\[
	y = 3 \left( x^2 - 3x \right) + 5 = 3 \left[ \left( x - \dfrac{3}{2} \right)^2 - \dfrac{9}{4} \right] + 5 = 3 \left( x - \dfrac{3}{2} \right)^2 - \dfrac{27}{4} + 5 = 3 \left( x - \dfrac{3}{2} \right)^2 - \dfrac{7}{4}.
	\]
	Cette forme montre que le minimum de la courbe est atteint pour $x = \dfrac{3}{2}$ et que le minimum vaut $f\left( \dfrac{3}{2} \right) = -\dfrac{7}{4}$.
	
	Ainsi, le sommet de la parabole est $S\left( \dfrac{3}{2} ; -\dfrac{7}{4} \right)$ et l'axe de symétrie a pour équation $x = \dfrac{3}{2}$.
	
	\subsection*{Question 5}
	
	Le trinôme $-3x^2 + 9x - 5$ a un discriminant :
	\[
	\Delta = 81 - 4 \times (-3) \times (-5) = 81 - 60 = 21.
	\]
	Comme $\Delta > 0$, le trinôme a deux racines :
	\[
	x_1 = \dfrac{-9 + \sqrt{21}}{-6} \approx 0,74 \quad \text{et} \quad x_2 = \dfrac{-9 - \sqrt{21}}{-6} \approx 2,26.
	\]
	Puisque $a = -3 < 0$, le trinôme est négatif sur les intervalles en dehors des racines, donc il est négatif sur l'intervalle $[x_1 ; x_2]$.
