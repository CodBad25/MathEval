
\medskip
À partir d'un premier segment de \qty{2}{\mm}, on ajoute successivement un nouveau segment mesurant \qty{150}{\percent}
de la longueur du précédent.

Pour tout entier naturel $n\geqslant1$, on désigne par $u_n$ la longueur, en mm, du $n$-ième segment.

Ainsi $u_1=2$ et $u_2=3$.
\begin{enumerate}
\item Déterminer $u_3$ et $u_4$.
\item Pour tout entier naturel $n$ supérieur à 1, exprimer $u_{n+1}$ en fonction de $u_n$.\\
En déduire la nature de la suite $(u_n)$.
\item Pour tout entier naturel $n\geqslant1$, exprimer $u_n$ en fonction de $n$.
\item On cherche à déterminer à partir de combien de segments la longueur totale dépasse 1~mètre.\hspace*{-8.5pt}\\
On réalise pour cela un programme écrit en langage \textsf{Python}.\\
Recopier et compléter sur la copie ce programme pour qu'il affiche le nombre attendu de segments.

\begin{center}
\begin{tabular}[]{|l|}
\hline
i = 1\\
u = 2\\
longueur = 2\\
while longueur < 1000:\\
\hspace{1.3em}	i = $\dots$\\
\hspace{1.3em}	u = $\dots$\\
\hspace{1.3em}	longueur = $\dots$\\
print(i)\\
\hline
\end{tabular}
\end{center}

\item Ce programme affiche 14.\\
Déterminer, par le calcul, la longueur de la spirale formée des 14 premiers segments. Arrondir le
résultat au mm.
\end{enumerate}

\medskip
