  
\medskip

Ce QCM comprend 5 questions indépendantes. Pour chacune d'elles, une seule des réponses
proposées est exacte.

Indiquer pour chaque question sur la copie la lettre correspondant à la réponse choisie.
Aucune justification n'est demandée.

Chaque réponse correcte rapporte $1$ point. Une réponse incorrecte ou une absence de
réponse n'apporte ni ne retire de point.

\medskip
\textbf{Question 1}
\medskip 


L'équation $2x^2 - 8x + 6 = 0 $ admet deux solutions. Leur somme $S$ et leur produit $P$ sont :

\medskip

\begin{tabularx}{\linewidth}{*{4}{X}}
\textbf{A.~~} \begin{minipage}[t]{2cm}
$S = -8$\\
$P = 6$\end{minipage} &\textbf{B.~~} \begin{minipage}[t]{2cm}$S= -4$\\$P = 3$\end{minipage}&\textbf{C.~~}\begin{minipage}[t]{2cm}$S = 4$\\
$P = 3$\end{minipage}& \textbf{D.~~} \begin{minipage}[t]{2cm}$S = 3$\\$P =-4$\end{minipage}.
\end{tabularx}

\medskip

\textbf{Question 2}

\medskip 

$\alpha$ est un nombre réel tel que $\sin(\alpha) = 0,5$. On a alors :

\medskip

\begin{tabularx}{\linewidth}{*{4}{X}}
\textbf{A.~~} $\sin(\pi-\alpha) = 0,5$ &\textbf{B.~~} $\sin(\pi-\alpha) = -0,5$&\textbf{C.~~}$\sin(\pi-\alpha) = -\dfrac{\sqrt{3}}{2}$& \textbf{D.~~} $\sin(\pi-\alpha) =\dfrac{\pi}{6}$.
\end{tabularx}

\medskip

\textbf{Question 3}

\medskip 

Dans un repère orthonormé du plan, on considère le cercle d'équation :
$(x-3)^2 + (y+ 0,5)^2 = \dfrac{25}{4}$

On peut affirmer que :

\medskip

\begin{tabularx}{\linewidth}{*{4}{X}}
\textbf{A.~~} ce cercle a un rayon
de 6,25. &\textbf{B.~~} ce cercle passe par
le point $R(5~;~-2$).&\textbf{C.~~}le centre de ce cercle a pour
coordonnées $(-3~;~0,5)$& \textbf{D.~~} aucune des réponses A., B. ou C. n'est
correcte.
\end{tabularx}

\medskip

\textbf{Question 4}

\medskip 

Dans un repère orthonormé du plan, une équation cartésienne de la droite passant par le
point $A(2~;~-4)$ et de vecteur normal $\vv{n} \ (5~;~6)$ est :

\medskip

\begin{tabularx}{\linewidth}{*{4}{X}}
\textbf{A.~~} $6x -5y - 32 = 0$ &\textbf{B.~~} $6x + 5y + 8 = 0
$&\textbf{C.~~}$5x+ 6y + 14 = 0$& \textbf{D.~~} $5x + 6y - 14 = 0$.
\end{tabularx}

\medskip

\textbf{Question 5}

\medskip 

On considère la fonction $f$ définie sur $R$ par $f(x) = (2x + 3)\e^x$.

La fonction dérivée de la fonction $f$ est notée $f'$. On a alors :

\medskip

\begin{tabularx}{\linewidth}{*{4}{X}}
\textbf{A.~~} $f'(x) = 2\e^x$ &\textbf{B.~~}$f'(x) = (2x + 3)\e^x$
 &\textbf{C.~~}$f'(x) = (2x + 1)\e^x$& \textbf{D.~~}$f'(x) = (2x + 5)\e^x$ .
\end{tabularx}

\vspace{0,5cm}

