
\medskip

La concentration d'un médicament dans le sang en mg.L$^{-1}$ au cours du temps $t$,
exprimé en heure, est modélisée par la fonction $f$ définie sur $[0~;~+ \infty[$ par :

\[f(t) = t\e^{-0,5t}\]

dont la représentation graphique est donnée ci-dessous.

\medskip

\begin{center}
\psset{yunit=3cm,labelFontSize=\scriptstyle}
\begin{pspicture}(-0.75,-0.2)(11,1.1)
\multido{\n=0.0+0.2}{56}{\psline[linewidth=0.25pt,linecolor=lightgray](\n,0)(\n,1)}
\multido{\n=0.0+0.2}{6}{\psline[linewidth=0.25pt,linecolor=lightgray](0,\n)(11,\n)}
\multido{\n=0+1}{11}{\psline[linewidth=0.45pt](\n,0)(\n,1)}
\multido{\n=0+1}{2}{\psline[linewidth=0.45pt](0,\n)(10.8,\n)}
\psaxes[linewidth=1.25pt]{->}(0,0)(-0.1,0)(11,1)
\psaxes[linewidth=1.25pt](0,0)(0,0)(11,1)
\def\Func{2.71828 x 0.5 neg  mul exp  x mul  }
\psplot[plotpoints=3000,linewidth=1.25pt,linecolor=red]{0}{11}{\Func}
\uput[dr](6,0.5){\red $\mathcal{C}$}
\end{pspicture}
\end{center}

\begin{enumerate}
\item Calculer la valeur exacte de $f(4)$ et interpréter le résultat dans le contexte de
l'exercice.
\item On note $f'$ la fonction dérivée de $f$.
Montrer que pour tout $t\in  [0~;~+\infty[$,\[ f'(t) = (1 - 0,5t)\e^{-0,5t}\].
\item Étudier le signe de $f'(t)$ sur $[0~;~+ \infty[$.
\item Déduire de la question précédente le tableau de variations de la fonction $f$
sur $[0~;~+ \infty[$.
\item Quelle est la concentration maximale du médicament dans le sang ? On
donnera la valeur exacte, puis une valeur approchée à $10^{-2}$ près.
\end{enumerate}

\vspace{0,5cm}

