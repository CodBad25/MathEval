	
	\subsection*{1.}
	
	Avec \(\overrightarrow{AB} \left( \begin{array}{c} 12 \\ 6 \end{array} \right)\) et \(\overrightarrow{HC} \left( \begin{array}{c} 6 \\ -12 \end{array} \right)\),\\
	 on a \(\overrightarrow{AB} \cdot \overrightarrow{HC} = 12 \times 6 + 6 \times (-12) = 72 - 72 = 0\).
	
	De même avec \(\overrightarrow{AC} \left( \begin{array}{c} 12 \\ -12 \end{array} \right)\) et \(\overrightarrow{HB} \left( \begin{array}{c} 6 \\ 6 \end{array} \right)\), 
	\\on a \(\overrightarrow{AC} \cdot \overrightarrow{HB} = 12 \times 6 + 6 \times (-12) = 72 - 72 = 0\).
	
	\subsection*{2.}
	
	On a donc (CH) est perpendiculaire à (AB) et (BH) est perpendiculaire à (AC). Les droites (CH) et (BH) sont donc deux hauteurs du triangle ABC : elles sont donc sécantes en H orthocentre du triangle et la troisième hauteur est la droite (CH).
	
	\subsection*{3.}
	
	On a \(KA^2 = 9^2 + (-3)^2 = 81 + 9 = 90\),\\
	\(KB^2 = 3^2 + 9^2 = 9 + 81 = 90\),\\
	\(KC^2 = (-3)^2 + (-9)^2 = 9 + 81 = 90\).
	
	Or \(KA^2 = KB^2 = KC^2 = 90\) entraîne \(KA = KB = KC = R\). \\
	Le point K est équidistant de A, B et C : c’est donc le centre du cercle circonscrit au triangle ABC.
	
	\subsection*{4.}
	
	Avec \(M(8, 7)\) et avec \(G(g, g')\),
	\[
	\overrightarrow{AM} \left( \begin{array}{c} 12 \\ -3 \end{array} \right), \quad \overrightarrow{AM} \left( \begin{array}{c} 8 \\ -2 \end{array} \right).
	\]
	
	On a donc
	\[
	\overrightarrow{AG} \left( \begin{array}{c} 8 \\ -2 \end{array} \right) = \overrightarrow{AG} \left( \begin{array}{c} g + 4 \\ g' - 10 \end{array} \right).
	\]
	
	On en déduit que \(g = 8 - 4 = 4\) et \(g' = 10 - 2 = 8\). Donc \(G(4, 8)\).
	
	\subsection*{5.}
	
	On a
	\[
	\overrightarrow{GH} \left( \begin{array}{c} -2 \\ 2 \end{array} \right) \quad et \quad \overrightarrow{GK} \left( \begin{array}{c} 1 \\ 1 \end{array} \right).
	\]
	
	On a donc de façon évidente \(\overrightarrow{GH} = 2 \overrightarrow{GK}\) : les vecteurs sont colinéaires, les droites (GH) et (GK) sont parallèles mais ont le point G commun, donc les points G, H et K sont alignés.
	
