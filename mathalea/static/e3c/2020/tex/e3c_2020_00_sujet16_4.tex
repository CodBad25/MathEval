
\medskip

Un apiculteur souhaite étendre son activité de production de miel à une nouvelle région.

Au printemps 2019, il achète $300$ colonies d’abeilles qu’il installe dans cette région.

Il consulte les services spécialisés de la région et s’attend à perdre 8\,\% des colonies
chaque hiver. Pour maintenir son activité et la développer, il prévoit d’installer 50
nouvelles colonies chaque printemps, à partir de l’année suivante.

\begin{enumerate}
\item  On donne le programme suivant écrit en langage Python :
\begin{python}
def algo( ) :
	C = 300
	N = 0
	while C < 400 :
		C = C*0.92+50
		N = N+1
	return(N)
\end{python}

	\begin{enumerate}
		\item  Recopier et compléter en ajoutant des colonnes, le tableau ci-dessous qui
reproduit l’avancement du programme pas à pas :

Les valeurs seront arrondies à l’entier le plus proche.

\smallskip

\begin{tabular}[]{|*{4}{>{\centering \arraybackslash}m{1.92cm}|}m{1.92cm}}
\hline
C 					&300	& 326	& \dotfill 	& \\\hline
\og C < 400 \fg{} ?	& oui	& oui 	& \dotfill	&\\\hline
\end{tabular}
\smallskip

		\item Quelle est la valeur de N renvoyée par le programme ?

Interpréter cette valeur dans le contexte de l’exercice.
	\end{enumerate}
\end{enumerate}

Le nombre de colonies est modélisée par une suite. On note $C_n$ une estimation du
nombre de colonies au printemps de l’année 2019 + $n$.

Ainsi $C_0 = 300$ est le nombre de colonies au printemps 2019.

On admet que pour tout entier naturel $n$, on a :

\[C_{n+1} = 0,92C_n + 50\]

\begin{enumerate}[resume]
\item La suite $\left(C_n\right)$, est-elle arithmétique ? La suite $\left(C_n\right)$ est-elle géométrique ?
\item On admet que $C_n = 625 - 325 \times 0,92^n $ pour tout entier naturel $n$.

L’apiculteur pourra-t-il atteindre les $700$ colonies ?
\end{enumerate}
