	
	\textbf{Question 1}
	
	Le coefficient $a = -1 < 0$, la fonction est donc croissante puis décroissante. Les réponses a. et b. sont éliminées. De plus, $f(2) = -4 - 2 + 6 = 0$, c’est donc c.
	
	\textbf{Question 2}
	
	\[
	A(x) = (e^x)^2 = e^{2x}.
	\]
	
	\textbf{Question 3}
	
	Les droites ont respectivement pour vecteurs directeurs $\overrightarrow{u}\left(\begin{array}{c}-1 \\ 2\end{array}\right)$ et $\overrightarrow{v}\left(\begin{array}{c}2 \\ 3\end{array}\right)$. Ces vecteurs ne sont pas colinéaires, les droites ne sont donc pas parallèles mais sécantes.
	
	La première équation peut s’écrire $y = -1 - 2x$. En remplaçant dans la seconde équation :
	
	\[
	3x - 2(-1 - 2x) + 5 = 0,
	\]
	\[
	3x + 2 + 4x + 5 = 0,
	\]
	\[
	7x + 7 = 0, \quad \text{d’où} \quad x = -1 \quad \text{et} \quad y = -1 + 2 = 1.
	\]
	Les droites sont donc sécantes en $C(-1; 1)$.
	
	\textbf{Question 4}
	
	Les droites ont respectivement pour vecteurs directeurs $\overrightarrow{u}\left(\begin{array}{c}-3 \\ 1\end{array}\right)$ et $\overrightarrow{v}\left(\begin{array}{c}1 \\ 3\end{array}\right)$. Ces vecteurs ne sont pas colinéaires, mais leur produit scalaire est nul :
	\[
	\overrightarrow{u} \cdot \overrightarrow{v} = -3 \times 1 + 1 \times 3 = 0,
	\]
	donc les vecteurs sont orthogonaux et les droites sont perpendiculaires.
	
	\textbf{Question 5}
	
	La fonction $suite(5)$ renvoie $12$.
	
