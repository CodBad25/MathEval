
\medskip

Cet exercice est un questionnaire à choix multiples.

Pour chacune des questions suivantes, une seule des quatre réponses proposées est exacte. 

Aucune justification n'est demandée.

Une bonne réponse rapporte un point. Une mauvaise réponse, une réponse multiple ou l'absence de réponse ne rapporte ni n'enlève aucun point.

Indiquer sur la copie le numéro de la question et la réponse correspondante.

\medskip

\textbf{Question 1}

\medskip 

Soit la fonction $P$ définie sur $\R$ par $P(x)=(x^2+x+1)(x-1)$.
L'équation $P(x)=0$ :

\medskip

\begin{tabularx}{\linewidth}{*{2}{X}}
\textbf{a.~~}n'a pas de solution sur $\R$ &\textbf{b.~~}a une unique solution sur $\R$ \\\textbf{c.~~}a exactement deux solutions sur $\R$& \textbf{d.~~} a exactement trois solutions sur $\R$.
\end{tabularx}

\medskip

\textbf{Question 2}

\medskip 

Soit la fonction $f$ définie sur $\R$ par $f(x) = (7x - 23)\left(\e^x+1\right)$.

L'équation $f(x)=0$ :

\medskip

\begin{tabularx}{\linewidth}{*{2}{X}}
\textbf{a.~~}admet $x=1$ comme solution &\textbf{b.~~}admet deux solutions sur $\R $\\\textbf{c.~~} admet $x=\frac{23}{7}$ comme solution & \textbf{d.~~}admet $x=0$ comme solution.
\end{tabularx}


\medskip

\textbf{Question 3}

\medskip 

Dans le plan rapporté à un repère orthonormé, le cercle de centre A$(-4~;2)$ et de rayon $r=\sqrt{2}$ a pour équation :

\medskip

\begin{tabularx}{\linewidth}{*{2}{X}}
\textbf{a.~~} $(x+4)^2+(y-2)^2=\sqrt{2} $ &\textbf{b.~~} $(x-4)^2+(y-2)^2=4 $\\\textbf{c.~~}$(x+4)^2+(y-2)^2=2 $& \textbf{d.~~} $(x-4)^2+(y+2)^2=2  $.
\end{tabularx}

\medskip

\textbf{Question 4}

\medskip 

Dans le plan rapporté à un repère orthonormé, on considère les vecteurs $\vv{u}(m+1~;~- 1)$ et $\vv{v}(m~;~2)$ où $m$ est un réel.

Une valeur de $m$ pour laquelle les vecteurs $\vv{u}$ et $\vv{v}$ sont orthogonaux est :

\medskip
\begin{tabularx}{\linewidth}{*{4}{X}}
\textbf{a.~~} $m= -\frac{2}{3}$ &\textbf{b.~~} $m=-2 $&\textbf{c.~~}$m=2 $& \textbf{d.~~} $m=-1  $.
\end{tabularx}

\medskip

\textbf{Question 5}

\medskip 

Dans le plan rapporté à un repère orthonormé, une équation cartésienne de la droite $D$ passant par le point $A(-2~;~5)$ et admettant pour vecteur normal
 $\vv{n}\, (-1~;~3)$ est :

\medskip

\begin{tabularx}{\linewidth}{*{2}{X}}
\textbf{a.~~} $-x + 3y + 7 = 0 $ &\textbf{b.~~} $x-3y + 17 = 0 $\\\textbf{c.~~}$-3x - y - 1 = 0$& \textbf{d.~~} $-x - 3y + 13 = 0  $.
\end{tabularx}

\vspace{0,5cm}

