
\medskip

On considère la fonction $f$ définie sur $\R$ par $f(x) = \left (2x - 1\right )\text{e}^x$.

On note $f'$ la fonction dérivée de la fonction $f$,

\medskip

\begin{enumerate}
\item Montrer que pour tout réel $x$,\, $f'(x) = \left (2x + 1\right )\text{e}^x$.
\item Étudier le signe de $f'(x)$ sur $\R$.
\item En déduire le tableau de variation de la fonction $f$ sur $\R$. Dans les questions suivantes, on note $\mathcal C$ la courbe représentative de la fonction $f$ dans un repère.
\item Déterminer les coordonnées du point d'intersection de $\mathcal C$ avec l'axe des ordonnées. 
\item Déterminer une équation de la tangente $T$ à $\mathcal C$ au point d'abscisse $0$.
\end{enumerate}

\bigskip

