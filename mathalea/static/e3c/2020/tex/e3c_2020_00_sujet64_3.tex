  
\medskip


Soit \Oij{} un repère orthonormé.

On considère le cercle $\mathcal{C}$ de centre A(2~;~5) et de rayon 5

\medskip

\begin{enumerate}
\item Montrer qu'une équation du cercle $\mathcal{C}$ est : $x^2 + y^2 - 4x - 10y= -4$.
\item Vérifier que le point $B(5~;~9$) appartient à ce cercle.
\item Que peut-on dire de la tangente au cercle au point B et de la droite (AB) ?
\item Déterminer une équation de la tangente au cercle au point B.
\item Calculer les coordonnées des points d'intersection du cercle $\mathcal{C}$ avec l'axe des ordonnées.
\end{enumerate}

\vspace{0,5cm}

