
\subsection*{1.}

Retrancher 10 \%, c'est multiplier par \(1 - \frac{10}{100} = 0{,}9\).

Donc :
\begin{align*}
&u_1 = 0{,}9 u_0 = 200 \times 0{,}9 = 180, \\
&u_2 = 0{,}9 u_1 = 180 \times 0{,}9 = 162.
\end{align*}

\subsection*{2.}

Puisque, pour tout entier naturel \(n\), \(u_{n+1} = 0{,}9 u_n\), cette égalité montre que la suite \((u_n)\) est géométrique de raison \(q = 0{,}9\) et de premier terme \(u_0 = 200\).

On a donc, pour tout entier naturel \(n\) : \(u_n = 200 \times 0{,}9^n\).

\subsection*{3.}

On a : \(u_{12} = 200 \times 0{,}9^{12} \approx 56{,}49\) (€).

\subsection*{4.}

\begin{center}
\begin{python}
def seuil(x):
    u = 200
    n = 0
    while u > x :
        u = u * 0.9
        n = n + 1
    return n
\end{python}
\end{center}

\subsection*{5.}

Au bout de sept semaines le prix est environ de \(95{,}66\) €.

