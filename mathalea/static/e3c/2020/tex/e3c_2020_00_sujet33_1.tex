
\medskip

Cet exercice est un questionnaire à choix multiples (QCM).\\
Pour chacune des questions, une seule des réponses proposées est exacte.\\
Indiquer sur la copie le numéro de la question ainsi que la réponse choisie.\\
Aucune justification n'est attendue.\\
\emph{Une réponse juste rapporte un point, une
réponse fausse ou l'absence de réponse n'enlèvent pas de point.}

\begin{description}
\item[Question 1:]\ %

Dans un repère du plan, la droite $(d)$ a pour équation: $2x-3y+1=0$.

Un vecteur directeur de la droite $(d)$ est:

\begin{enumerate}[label=\textbf{\alph{*}.},itemjoin=\hfil,afterlabel={~~}]
\item $\vect{u}(2\,;\,-3)$ \item $\vect{v}(3\,;\,2)$ \item $\vect{w}(-3\,;\,1)$ \item $\vect{r}\left(1\,;\,\dfrac{3}{2}\right)\cdotp$
\end{enumerate}

\item[Question 2:]\ %

Dans un repère du plan, la droite $(d)$ a pour équation: $2x-3y+1=0$. Un vecteur normal à la droite $(d)$
est:

\begin{enumerate}[label=\textbf{\alph{*}.},itemjoin=\hfil,afterlabel={~~}]
\item $\vect{u}(2\,;\,3)$ \item $\vect{v}(3\,;\,2)$ \item $\vect{w}(-3\,;\,1)$ \item $\vect{r}\left(1\,;\,\dfrac{3}{2}\right)\cdotp$
\end{enumerate}

\item[Question 3:]\ %

On donne trois points distincts: $\mathsf{A}$, $\mathsf{B}$ et $\mathsf{C}$.\\
Les points $\mathsf{D}$ et $\mathsf{E}$ sont tels que $\vect{\mathsf{EB}}=\vect{\mathsf{BA}}$ et $\vect{\mathsf{ED}}=2\vect{\mathsf{BC}}$.

On a:

\begin{enumerate}[label=\textbf{\alph{*}.},itemjoin=\hfil,afterlabel={~~}]
\item $\mathsf{A}$ est le milieu de $[\mathsf{EB}]$ \item $\mathsf{B}$ est le milieu de $[\mathsf{ED}]$
\end{enumerate}

\begin{enumerate}[resume,label=\textbf{\alph{*}.},itemjoin=\hfil,afterlabel={~~}]
\item $\mathsf{C}$ est le milieu de $[\mathsf{AD}]$ \item $\mathsf{D}$ est le milieu de $[\mathsf{AC}]$.
\end{enumerate}

\item[Question 4:]\ %

Soit $x$ un nombre réel. Dans un repère orthonormé, les vecteurs $\vect{u}(-x+4\,;\,7)$ et $\vect{v}(9\,;\,2x-5)$ sont
orthogonaux lorsque $x$ est égal à:

\begin{enumerate}[label=\textbf{\alph{*}.},itemjoin=\hfil,afterlabel={~~}]
\item $\dfrac{1}{5}$
\item 10
\item $-\dfrac{1}{5}$
\item 6.
\end{enumerate}

\item[Question 5:]\ %

Dans un repère orthonormé, on considère les points $\mathsf{A}(-1\,;\,-2)$, $\mathsf{B}(2\,;\,0)$, $\mathsf{C}(3\,;\,-1)$ et\\
$\mathsf{D}(-3\,;\,4)$.

Alors $\vect{\mathsf{AC}}\cdot\vect{\mathsf{BD}}$ est égal à:

\begin{enumerate}[label=\textbf{\alph{*}.},itemjoin=\hfil,afterlabel={~~}]
\item $-16$
\item 11
\item 21
\item $-24$.
\end{enumerate}
\end{description}
