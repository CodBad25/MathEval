	
	\subsection*{1.}
	Diminuer de 1,5\%, c'est multiplier par \(1 - \dfrac{1,5}{100} = 0,985\).
	
	Donc \(d_1 = 400 \times 0,985 = 394\).
	
	Après une baisse de 1,5\%, la moyenne de déchets sera en 2019 de 394 kg.
	
	\subsection*{2.a}
	
	D'une année sur l'autre, on multiplie la quantité de déchets d'une année par \( 0,985 \).
	
	On a donc, pour tout entier naturel \( n \), \( d_{n+1} = 0,985 \, d_n \), égalité qui montre que la suite \( (d_n) \) est une suite géométrique de raison \( 0,985 \) et de premier terme \( d_0 = 400 \).
	
	\subsection*{2.b}
	
	On sait que, pour tout entier naturel \( n \), \( d_n = d_0 \times 0,985^n = 400 \times 0,985^n \).
	
	\subsection*{3.a}
	
	On obtient une masse inférieure à 365 kg pour \( n = 7 \), soit en 2025.
	
	\subsection*{3.b}
	\begin{python}
		def dechet(m) :
		d = 400
		n = 0
		while d > m :
		d = d * 0.985
		n = n + 1
		return (2018 + n)
	\end{python}
	
