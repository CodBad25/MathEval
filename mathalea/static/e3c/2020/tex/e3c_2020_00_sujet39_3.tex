
\medskip

Une balle est lâchée d'une hauteur de 3 mètres au-dessus du sol. Elle touche le sol et rebondit. À chaque rebond, la balle perd 25\,\% de sa hauteur précédente.

On modélise la hauteur de la balle par une suite $\left(h_n\right)$ où $h_n$ désigne la hauteur maximale de la balle, en mètres, après le $n$-ième rebond. 

On a donc $h_0=3$.

\medskip

\begin{enumerate}
\item  Calculer $h_1$ et $h_2$.
\item La suite $\left(h_n\right)$ est-elle arithmétique ? Justifier.
\item Donner la nature de la suite $\left(h_n\right)$ en précisant ses éléments caractéristiques.
\item Déterminer la hauteur, arrondie au cm, de la balle après 6 rebonds.
\item La fonction \og seuil \fg{} est définie ci-dessous en langage Python.

\begin{center}
\renewcommand\arraystretch{0.95}
\begin{tabular}[]{|c m{3.5cm}|}
\hline
1&def seuil():\\
2&\hspace{2em} h=3\\
3& \hspace{2em}n=0\\
4& \hspace{2em}while \dotfill :\\
5& \hspace{3.75em}h=\dotfill\\
6& \hspace{3.75em}n=n+1\\
7&\hspace{2em} return n\\\hline
\end{tabular}
\end{center}


Recopier et compléter les lignes 4 et 5 pour que cette fonction renvoie le nombre de rebonds à partir duquel la hauteur maximale de la balle sera inférieure ou égale à 10 centimètres.
\end{enumerate}

\vspace{0,5cm}

