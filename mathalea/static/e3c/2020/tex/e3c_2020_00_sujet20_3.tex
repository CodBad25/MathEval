
\medskip

La famille A décide de diminuer de 2\,\% par mois sa quantité de déchets produits par mois à partir du 1\up{er} janvier 2020.

Au mois de décembre 2019, elle a produit $120$~kg de déchets.

\medskip

\begin{enumerate}
\item Justifier qu'au bout de 2 mois, la famille A aura produit environ $115$~kg de déchets.
\end{enumerate}

On admet que la quantité de déchets produits chaque mois conserve la même évolution toute l'année.

On modélise l'évolution de la production de déchets de la famille A par la suite de terme général $a_n$, où $a_n$ représente la quantité, en kg, de déchets produits par la famille A $n$ mois après décembre 2019.

Ainsi, $a_0$ représente la quantité de déchets produits durant le mois de décembre 2019, $a_1$ représente la quantité de déchets produits durant le mois de janvier 2020, etc.

\begin{enumerate}[resume]
\item  
	\begin{enumerate}
		\item Déterminer la nature de la suite $\left(a_n\right)$.
		\item Pour tout entier naturel $n$, exprimer $a_n$ en fonction de $n$.
		\item Déterminer la quantité totale de déchets que produira la famille A durant l'année
2020.

On arrondira le résultat à l'unité.

On rappelle que :

Soit $\left(a_n\right)_{n\in \N}$ une suite géométrique de raison $q$,\,$ q \ne 1$. La somme $S$ de termes consécutifs est égale à $S = u_1 + u_2 +\ldots + u_n = u_1 \times \dfrac{1 - q^n}{1 - q}$.
		\item On donne le programme ci-dessous.
		
%\begin{center}
%\begin{tabularx}{0.5\linewidth}{m{0.5cm}| X}
%\multicolumn{1}{l}{1}&\texttt{def S(n):}\\
%2&\texttt{U = 128}\\
%3&\texttt{S = 0}\\
%4&\texttt{for k in range (n):}\\
%5&\multicolumn{1}{l}{\quad|~\texttt{U = 0.98*U}}\\
%6&\multicolumn{1}{l}{\quad|~\texttt{S = S + U}}\\
%7&\texttt{return (S)}\\
%\multicolumn{1}{l}{8}&\\
%\end{tabularx}
%\end{center}
 
\begin{center}
\begin{tabular}{c l}
\texttt 1 & \texttt{{\blue def} S(n) :}\\
\texttt 2 &  \vline \quad \texttt{U = 120}\\
\texttt 3 &  \vline \quad \texttt{S = 0}\\
\texttt 4 &  \vline \quad \texttt{{\blue for} k {\blue in} range (n) :}\\
\texttt 5 &  \vline \quad \vline \quad \texttt{U = 0.98 * U}\\
\texttt 6 &  \vline \quad \vline \quad \texttt{S = S + U}\\
\texttt 7 &  \vline \quad \texttt{{\blue return} (S)}\\
\texttt 8 & 
\end{tabular}
\end{center} 
 
Que représente le résultat renvoyé par la fonction si on entre l'instruction S(6) ?
	\end{enumerate}
\end{enumerate}

\bigskip

