	
	\subsection*{1.}
	
	\subsubsection*{a.}
	
	\begin{align*}
&M(x ; y) \in C \cap d \\
\iff& f(x)=2x+3\\
\iff&  2x + 3 = 2x^3 + 2x^2 - 2x + 3 \\
\iff& 2x^3 + 2x^2 - 4x = 0 \\
\iff& 2x (x^2 + x - 2) = 0.
	\end{align*}
	
	\subsubsection*{b.}
	
	On résout l’équation précédente :
	
	\[
	2x (x^2 + x - 2) =0 \iff
	\begin{cases}
		x = 0 \\
		x^2 + x - 2 = 0
	\end{cases}.
	\]
	
	La première solution est \(x = 0\), d’où \(y = 2 \times 0 + 3 = 3\).
	
	Pour l’équation du second degré :
	\[
	x^2 + x - 2 = 0
	\]
	
	\[
	\Delta = 1^2 - 4 \times (-2) = 1 + 8 = 9 = 3^2 > 0.
	\]
	
	Il y a donc deux solutions :
	
	\[
	x = \dfrac{-1 + 3}{2} = 1 \quad et \quad x = \dfrac{-1 - 3}{2} = -2,
	\]
	
	d’où les ordonnées respectives
	
	\[
	f(1) = 2 \times 1 + 3 = 5 \quad et \quad f(-2) = -4 + 3 = -1.
	\]
	
	Les points communs à \(d\) et à \(C\) ont pour coordonnées : 	\((0 ; 3)\), \((1 ; 5)\), \((-2 ; -1)\).
	
	\subsection*{2.}
\begin{flushright}
	\begin{center}
			\definecolor{ccqqqq}{rgb}{0.8,0,0}
	\definecolor{qqttff}{rgb}{0,0.2,1}
	\definecolor{cqcqcq}{rgb}{0.75,0.75,0.75}
	\begin{tikzpicture}[line cap=round,line join=round,>=triangle 45,x=1.0cm,y=1.0cm]
		\draw [color=cqcqcq,dash pattern=on 1pt off 1pt, xstep=1.0cm,ystep=1.0cm] (-3,-0.76) grid (2,8.14);
		\draw[->,color=black] (-3,0) -- (2,0);
		\foreach \x in {-3,-2,-1,1}
		\draw[shift={(\x,0)},color=black] (0pt,2pt) -- (0pt,-2pt) node[below] {\footnotesize $\x$};
		\draw[->,color=black] (0,-0.76) -- (0,8.14);
		\foreach \y in {,1,2,3,4,5,6,7,8}
		\draw[shift={(0,\y)},color=black] (2pt,0pt) -- (-2pt,0pt) node[left] {\footnotesize $\y$};
		\draw[color=black] (0pt,-10pt) node[right] {\footnotesize $0$};
		\clip(-3,-0.76) rectangle (2,8.14);
		\draw[line width=2pt,color=qqttff, smooth,samples=100,domain=-3.0:2.0] plot(\x,{2*(\x)^3+2*(\x)^2-2*(\x)+3});
		\draw[line width=1.6pt,color=ccqqqq, smooth,samples=100,domain=-3.0:2.0] plot(\x,{2*(\x)+1.74});
		\draw[line width=1.6pt,color=ccqqqq, smooth,samples=100,domain=-3.0:2.0] plot(\x,{2*(\x)+7.23});
	\end{tikzpicture}
	\end{center}
\end{flushright}
	
	On voit qu’il existe deux valeurs de \(a\) (ordonnée à l’origine des droites tangentes à la courbe), \(a \approx 1,9\) et \(a \approx 7,1\).
	
	Remarque : il faut trouver des points d’abscisse \(x\) tels que le nombre dérivé est égal à 2, soit résoudre l’équation :
	\[
	6x^2 + 4x - 2 = 2 \iff 6x^2 + 4x - 4 = 0 \iff 3x^2 + 2x - 2 = 0.
	\]
	
	On a :
	\[
	\Delta = 4 - 4 \times 3 \times (-2) = 4 + 24 = 28 > 0.
	\]
	
	Il y a deux solutions :
	
	\[
	x = \dfrac{-2 + \sqrt{28}}{6} \approx 0,549 \quad et \quad x = \dfrac{-2 - \sqrt{28}}{6} \approx -1,215.
	\]
	
	Les droites sont tracées en rouge.
	
	\subsection*{3.}
	
	\subsubsection*{a.}
	
	Le polynôme \(f(x)\) est dérivable sur \(\mathbb{R}\) et sur cet intervalle :
	\[
	f'(x) = 6x^2 + 4x - 2 = 2(3x^2 + 2x - 1).
	\]
	
	Or pour le trinôme \(3x^2 + 2x - 1\), \(\Delta = 4 + 12 = 16 = 4^2 > 0\). Le trinôme a deux racines :
	\[
	x_1 = \dfrac{-2 + 4}{6} = \dfrac{2}{6} = \dfrac{1}{3} \quad et \quad x_2 = \dfrac{-2 - 4}{6} = \dfrac{-6}{6} = -1.
	\]
	
	On sait qu’alors \(3x^2 + 2x - 1 = 3(x + 1)(x - \dfrac{1}{3})\). Donc
	\[
	f'(x) = 6(x + 1)(x - \dfrac{1}{3}).
	\]
	
	\subsubsection*{b.}
	
	On sait que \(f'(x)\) est positive sauf sur l’intervalle \(\left] -1 ; \dfrac{1}{3} \right[\), donc la fonction est croissante sauf sur l’intervalle \(\left] -1 ; \dfrac{1}{3} \right[\) où elle est décroissante.
	
