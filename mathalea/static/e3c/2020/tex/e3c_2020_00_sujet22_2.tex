 
\medskip

Les résultats seront arrondis à l’unité.

La quantité (en kg) de déchets ménagers produite par habitant d’une ville de taille moyenne a été de \np[kg]{537} en 2019 et la municipalité espère réduire ensuite cette production de 1,5\,\% par an.

Pour tout entier naturel $n$, on note $d_n$ la quantité (en kg) de déchets ménagers produite par habitant de cette ville durant l’année 2019 +$n$, on a donc $d_0=537$.

\medskip

\begin{enumerate}
\item Montrer par un calcul que $d_1=0,985\times d_0$.
\item Pour tout entier naturel $n$ , exprimer $d_{n+1}$ en fonction de $d_n$.
\item En déduire la nature de la suite $(d_n)$ puis une expression de $d_n$ en fonction de $n$.
\item On souhaite savoir à partir de quelle année la production moyenne de déchets produite par chaque habitant
sera inférieure à celle enregistrée en 2019 au niveau national, à savoir \np[kg]{513} .

Pour cela, on considère l’algorithme suivant rédigé en langage Python.

\begin{center}
\texttt{
\begin{tabular}[]{cl}
1& def année() :\\
2& \qquad n = 0\\
3& \qquad d = 537 \\
4& \qquad While d > $\cdots$ :\\
5& \qquad \qquad n = n + 1\\
6& \qquad \qquad d = $\cdots$\\
7& \qquad return (n)
\end{tabular}
}
\end{center}

\begin{enumerate}
\item Recopier et compléter l’algorithme afin de répondre au problème posé
\item À partir de quelle année la production moyenne de déchets produite par chaque habitant sera-t-elle inférieure à celle enregistrée en 2019 au niveau national ?
\end{enumerate}
\end{enumerate}

%\vspace{1cm}

