
\medskip

Une angine peut être provoquée soit par une bactérie (angine bactérienne) soit par un virus (angine virale). On admet qu'un malade ne peut pas être à la fois porteur du virus et de la bactérie. L'angine est bactérienne dans 20\,\% des cas.

Pour déterminer si une angine est bactérienne, on dispose d'un test. Le résultat du test peut être positif ou négatif. Le test est conçu pour être positif lorsque l'angine est bactérienne mais il présente des risques d'erreur :

\medskip

$\bullet~~$ si l'angine est bactérienne, le test est négatif dans 30\,\% des cas

$\bullet~~$ si l'angine est virale, le test est positif dans 10\,\% des cas

\medskip

On choisit au hasard un malade atteint d'angine. On note:

$\bullet~~$ $B$ l'évènement: \og l'angine est bactérienne \fg{} ;

$\bullet~~$ $T$ l'évènement: \og le test effectué sur le malade est positif \fg. 

\medskip

Si besoin, les résultats seront arrondis à $10^{-3}$ près.

\medskip

\begin{enumerate}
\item Représenter la situation par un arbre pondéré.
\item Quelle est la probabilité que l'angine soit bactérienne et que le test soit positif ?
\item Montrer que la probabilité que le test soit positif est $0,22$.
\item Un malade est choisi au hasard parmi ceux dont le test est positif. Quelle est la
probabilité pour que son angine soit bactérienne?
\end{enumerate}


