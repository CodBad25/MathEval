
\medskip

Un constructeur de véhicules fabrique deux types d'automobiles: \og Citadine\fg{} ou \og Routière \fg.

Pour ces véhicules, ce constructeur propose deux finitions :

\begin{itemize}
\item \og Sport\fg{} au tarif de \np{2500} euros par véhicule, 
\item \og Luxe\fg{} au tarif de \np{4000} euros par véhicule.
\end{itemize}

En consultant le carnet de commandes de ce constructeur, on recueille les indications suivantes :

\setlength\parindent{9mm}
\begin{itemize}
\item[$\bullet~~$] 80\,\% des clients ont commandé une automobile \og Citadine \fg. Les autres clients ont commandé une automobile \og Routière \fg.
\item[$\bullet~~$] Parmi les clients possédant une automobile \og Citadine \fg, 70\,\% ont pris la finition \og Sport \fg.
\item[$\bullet~~$] Parmi les clients possédant une automobile \og Routière \fg, 60\,\% ont pris la finition \og Luxe \fg.
\end{itemize}
\setlength\parindent{0mm}

On choisit un client au hasard et on considère les évènements suivants : 

\setlength\parindent{9mm}
\begin{itemize}
\item[$\bullet~~$] $C$ : \og Le client a commandé une automobile \og Citadine\fg{} \fg,
\item[$\bullet~~$] $L$ : \og Le client a choisi la finition \og Luxe\fg{} \fg.
\end{itemize}
\setlength\parindent{0mm}

D'une manière générale, on note $\overline{A}$ l'évènement contraire d'un évènement $A$.

On note $X$ la variable aléatoire qui donne le montant en euros de la finition choisie par un client.

\medskip

\begin{enumerate}
\item Construire l'arbre pondéré de probabilité traduisant les données de l'exercice.
\item Calculer la probabilité que le client ait commandé une automobile \og Citadine\fg{} et ait choisi la
finition \og Luxe \fg, c'est-à-dire calculer $P(C \cap L)$.
\item Justifier que $P(L) = 0,36$.
\item La variable aléatoire $X$ ne prend que deux valeurs $a$ et $b$.
	\begin{enumerate}
		\item Déterminer les probabilités $P(X = a)$ et $P(X = b)$. 
		\item Déterminer l'espérance de $X$.
	\end{enumerate}
\end{enumerate}

\bigskip

