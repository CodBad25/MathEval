
\bigskip

\textbf{Partie A :}

\medskip

$\left(u_n\right)$ est une suite géométrique de premier terme $u_0 = \np{25000}$ et de raison $0,94$.

$\left(v_n\right)$ est une suite définie par : $v_n = 50(104 + 25n)$ pour tout entier naturel $n$.

\medskip

\begin{enumerate}
\item Déterminer une forme explicite de la suite $\left(u_n\right)$.
\item Calculer la somme des sept premiers termes de la suite $\left(u_n\right)$. 
\item Comparer les termes $u_0$ et $v_0$ puis $u_{20}$ et $v_{20}$.
\item Déterminer le plus petit entier naturel $n$ tel que $u_n < v_n$.
\end{enumerate}

\bigskip

\textbf{Partie B :}

\medskip

Un concessionnaire de voitures propose des voitures équipées d'un moteur diesel ou d'un moteur essence.

Durant sa première année d'existence en 1995, il a vendu \np{25000}~véhicules avec un moteur diesel et \np{5200} véhicules avec un moteur essence.

Ses ventes de voitures avec un moteur diesel ont diminué de 6\,\% chaque année, alors que ses ventes de voitures avec un moteur essence ont augmenté de \np{1250}~unités tous les ans.

En quelle année les ventes de voitures avec un moteur essence ont elles dépassé les ventes de voitures avec un moteur diesel?


\bigskip

