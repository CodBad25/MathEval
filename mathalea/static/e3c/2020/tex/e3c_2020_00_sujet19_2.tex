  
\medskip

On administre à un patient un médicament par injection intraveineuse.

La première injection est de \np[ml]{10}, puis toutes les heures on lui en injecte \np[ml]{1} .

On étudie l’évolution de la quantité de médicament présente dans le sang en prenant le modèle suivant :

\begin{itemize}
\item  on estime que 20\,\% de la quantité de médicament présente dans le sang est éliminée chaque heure ;
 
 \item pour tout entier naturel $n$, on note $U_n$ la quantité de médicament en ml présente dans le sang au bout de $n$ heures.
\end{itemize}
Ainsi, $U_0=10$.
\begin{enumerate}
\item  Justifier que $U_1=9$.
\item Montrer que, pour tout entier naturel $n$, $U_{n+1}=0,8 U_{n}+1$.

On donne ci-dessous la représentation graphique de la suite $\left(U_n\right)$ :

\psset{xunit=0.10cm,yunit=0.7cm,labelFontSize=\scriptstyle,showorigin=false,dotstyle=Mul}
\begin{pspicture}(-5.26,-1)(112,12.6)
%  \multido{\n=-4.2+0.2}{33}{\psline[linewidth=0.35pt,linecolor=lightgray](\n,-2)(\n,2.4)}
%  \multido{\n=-2.2+0.2}{24}{\psline[linewidth=0.35pt,linecolor=lightgray](-4.4,\n)(2.2,\n)}
%   \multido{\n=-4+1}{7}{\psline[linewidth=0.45pt](\n,-2)(\n,2.4)}
%  \multido{\n=-2+1}{5}{\psline[linewidth=0.45pt](-4.4,\n)(2.2,\n)}
\psaxes[linewidth=0.95pt,Dx=10]{->}(0,0)(-5,-0.5)(110,12)
\def\Func{0.8 x exp 5 mul 5 add }
\psplot[plotpoints=100, plotstyle=dots,dotsize=3pt,linecolor=red]{0}{100}{\Func}
\end{pspicture}

\item Conjecturer la limite de la suite $(U_n)$.

On considère l’algorithme suivant :
\begin{center}
\begin{tabular}[]{|l|}
\hline
$U\leftarrow 10$\\
$N\leftarrow 0$\\
Tant que $U > 5,1$ faire\\
\hspace{1.5em}$U\leftarrow 0,8*U+1$\\
\hspace{1.5em}$N\leftarrow N+1$\\
Fin du tant que\\
Afficher $N$ \\
\hline
\end{tabular}
\end{center}
\item À quoi sert cet algorithme ?

\item À l’aide de l’extrait du tableau de valeurs de la suite $\left(U_n\right)$ donné ci-dessous, donner la valeur de $N$ à l’issue de l’exécution de cet algorithme.

\begin{center}
\begin{tabularx}{\linewidth}{|*{8}{>{\centering \arraybackslash} X|}} \hline
$n$&8&9&10&11&12&13&14\\\hline
$U_n$&\np{5,838861}&\np{5,671089}&\np{5,536871}&\np{5,429497}&\np{5,343597}&\np{5,274878}&\np{5,219902}\\\hline
\end{tabularx}

\bigskip

\begin{tabularx}{\linewidth}{|*{8}{>{\centering \arraybackslash} X|}} \hline
$n$&15&16&17&18&19&20&21 \\\hline
$U_n$&\np{5,175922}&\np{5,140737}&\np{5,11259}&\np{5,090072}&\np{5,072058}&\np{5,057646}&\np{5,046117} \\\hline
 \end{tabularx}
 
\bigskip

\begin{tabularx}{\linewidth}{|*{8}{>{\centering \arraybackslash} X|}} \hline
$n$&22&23&24&25&26&27&28\\
 \hline
$U_n$&\np{5,036893}&\np{5,029515}&\np{5,023612}&\np{5,018889}&\np{5,015112}&\np{5,012089}&\np{5,009671}\\\hline
\end{tabularx}
\end{center}
\end{enumerate}

\vspace{0,5cm}

