
\subsection*{1.}

On a donc \( u_2 = 200 + 5 = 205 \) et \( u_3 = u_2 + 5 = 205 + 5 = 210 \).

De même, on a :
\[
v_2 = 200 \times \left(1 + \dfrac{2}{100}\right) = 200 \times 1{,}02 = 204 \quad \text{et} \quad v_3 = v_2 \times 1{,}02 = 208{,}80 \text{ (en euros)}.
\]

\subsection*{2.}

\paragraph{a.}
\begin{center}
\begin{python}
u = 200
v = 200
n = int(input("Saisir une valeur de n :"))
for i in range(1,n) :
    u = u + 5 
    v = v * 1.02
print("Pour n =",n,"on a","u =",u,"et v =",v)
\end{python}
\end{center}

\paragraph{b.} On a :
\[
u_4 = u_3 + 5 = 215 \quad \text{et} \quad v_4 = v_3 \times 1{,}02 \approx 110{,}98.
\]

\subsection*{3.}

Quel que soit le naturel \( n \), \( n \geq 1 \), on a :
\begin{itemize}
    \item \( u_{n+1} = u_n + 5 \) : la suite \( (u_n) \) est une suite arithmétique de premier terme 200 et de raison 5, donc \( u_n = 200 + 5(n - 1) \).
    \item \( v_{n+1} = v_n \times 1{,}02 \) : la suite \( (v_n) \) est une suite géométrique de premier terme 200 et de raison 1{,}02, donc \( v_n = 200 \times 1{,}02^{n-1} \).
\end{itemize}

\subsection*{4.}

3 ans correspondent à \( 3 \times 12 = 36 \) mois de loyers.

\begin{itemize}
    \item Avec le premier contrat :
    \[
    S_{36} = u_1 + u_2 + \dots + u_{36},
    \]
    que l'on peut écrire :
    \[
    S_{36} = 200 + 205 + \dots + 200 + 5 \times 35,
    \]
    ou encore :
    \[
    S_{36} = 200 + 5 \times 35 + \dots + 205 + 200,
    \]
    et en sommant membres à membres :
    \[
    2S_{36} = 36 \times (200 + 200 + 5 \times 35) = 36 \times 575 = 20700,
    \]
   	d'où \(S_{36} = 10350\) (euros).

    \item Avec le second contrat :
    \[
    T_{36} = v_1 + v_2 + \dots + v_{36},
    \]
    que l'on peut écrire :
    \[
    T_{36} = 200 + 200 \times 1{,}02 + \dots + 200 \times 1{,}02^{35} \quad (1),
    \]
    d'où, par produit par 1{,}02 :
    \[
    1{,}02 T_{36} = 200 \times 1{,}02 + 200 \times 1{,}02^2 + \dots + 200 \times 1{,}02^{36} \quad (2),
    \]
    et par différence \((2) - (1)\) :
    \[
    0{,}02 T_{36} = 200 \times 1{,}02^{36} - 200,
    \]
    d'où :
    \[
    T_{36} = \dfrac{200 \times 1{,}02^{36} - 200}{0{,}02} \approx 10398{,}87 \text{ (euros)}.
    \]
\end{itemize}

C'est donc le premier contrat qui reviendra le moins cher (de peu!).

