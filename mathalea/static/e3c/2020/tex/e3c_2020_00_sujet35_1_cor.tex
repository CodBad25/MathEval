
\subsection*{Question 1}

Pour l'équation \(3x^2 - 4x + 1 = 0\), \(0\) est une racine évidente, et comme le produit des racines est égal à \(\dfrac{1}{3}\), l'autre racine est \(\dfrac{1}{3}\).

On sait que ce trinôme est positif sauf entre les racines, donc \(S = \left]-\infty\,;\,\dfrac{1}{3}\right[ \cup [1\,;\,+\infty[\).

\subsection*{Question 2}

\(\vec{u}\) et \(\vec{v}\) sont orthogonaux si et seulement si :
\[
3(a + 2) - a = 0 \iff 2a + 6 = 0 \iff a = -3.
\]
Réponse \(\textbf{C.}\)

\subsection*{Question 3}

\begin{align*}
&M(x\,;\,y) \in \textit{d} \\
\iff &\overrightarrow{AM} \cdot \vec{u} = 0 \\
\iff &1(x - (-2)) + 2(y - 3) = 0 \\
\iff &x + 2y - 4 = 0.
\end{align*}

\subsection*{Question 4}

On sait que pour tout \(n \in \mathbb{N}\), \(u_n = 3 \times 2^n\).

Donc : \(S_{10} = 3 + 6 + 12 + \dots + 3 \times 2^{10} \quad (1)\),

et : \(2S_{10} = 6 + 12 + \dots + 3 \times 2^{10} + 3 \times 2^{11} \quad (2)\).

En faisant \((2) - (1)\), on obtient : \(S_{10} = 3 \times 2^{11} - 3 = 3(2^{11} - 1)\).

\subsection*{Question 5}

\(x > 1\), donc le dénominateur n'est pas nul, donc la fonction \(f\) est dérivable sur \(]1\,;\,+\infty[\), et sur cet intervalle :
\[
f'(x) = \dfrac{2(x - 1) - 1(2x + 1)}{(x - 1)^2} = \dfrac{2x - 2 - 2x - 1}{(x - 1)^2} = \dfrac{-3}{(x - 1)^2} = -\dfrac{3}{(x - 1)^2}.
\]

