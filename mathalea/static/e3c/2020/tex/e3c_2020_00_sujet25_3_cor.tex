	
	\subsection*{Question 1}
	
	\begin{itemize}
		\item[a)] On ajoute chaque année $1,4$, donc $T_{n+1} = T_n + 1,4$, quel que soit $n \in \mathbb{N}$. La suite $(T_n)$ est donc arithmétique de premier terme $T_0 = 14$ et de raison $1,4$.
		\item[b)] Cherchons la solution de l'inéquation $14 + 1,4n > 35$ :
		\begin{align*}
			1,4n &> 21 \\
			n &> \frac{21}{1,4} = 15.
		\end{align*}
		Au bout de $15$ ans, soit en $2034$, la température sera de $35^\circ$.
	\end{itemize}
	
	\subsection*{Question 2}
	
	\begin{itemize}
		\item[a)] Baisser de $10\%$ c'est multiplier par :
		\begin{align*}
			1 - \frac{10}{100} = 0,9.
		\end{align*}
		On a donc $P_{n+1} = 0,9 P_n$, quel que soit $n \in \mathbb{N}$. La suite $(P_n)$ est donc géométrique de raison $0,9$ et de premier terme $673$.
		\item[b)] Pour tout $n \in \mathbb{N}$, on a :
		\begin{align*}
			P_n = 673 \times 0,9^n.
		\end{align*}
	\end{itemize}
	
	\subsection*{Question 3}
	
	L'algorithme indique qu'en $2026$, la hauteur des précipitations sera inférieure ou égale à $300$ mm.
	
