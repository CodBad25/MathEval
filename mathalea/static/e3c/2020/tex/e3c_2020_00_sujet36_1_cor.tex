
\subsection*{Question 1}

Un vecteur directeur de \(\mathcal{D}\) est \(\vec{u} \begin{pmatrix} -5 \\ 4 \end{pmatrix}\), donc un vecteur normal est par exemple \(\vec{n} \begin{pmatrix} 4 \\ 5 \end{pmatrix}\).

\subsection*{Question 2}

\begin{align*}
&x^2 - 2x + y^2 = 3 \\
\iff &(x - 1)^2 - 1 + y^2 = 3 \\
\iff &(x - 1)^2 + y^2 = 4 \\
\iff &(x - 1)^2 + (y - 0)^2 = 2^2 \\
\iff &AM^2 = 2^2,
\end{align*}
signifie que les points \(M\) appartiennent au cercle de centre \(A(1\,;\,0)\) et de rayon 2.

\subsection*{Question 3}

Avec \(S = 15 + 16 + 17 + \dots + 243\), on peut aussi écrire :
\[
S = 243 + 242 + \dots + 17 + 16 + 15.
\]
En sommant membre à membre :
\[
2S = 229 \times 258 \quad \text{d'où} \quad S = \dfrac{229 \times 258}{2} = 229 \times 129 = 29541.
\]

\subsection*{Question 4}

\(f'(x) = 1\e^x + (x + 1) \e^x = \e^x(1 + x + 1) = (x + 2) \e^x\).

\subsection*{Question 5}

D'après la loi des probabilités totales :
\begin{align*}
p(B) &= p(A \cap B) + p(\overline{A} \cap B) \\
&= p(A) \times p_A(B) + p(\overline{A}) \times p_{\overline{A}}(B) \\
&= \dfrac{1}{3} \times \dfrac{2}{5} + \dfrac{2}{3} \times \dfrac{1}{4} \\
&= \dfrac{2}{15} + \dfrac{1}{6} = \dfrac{4}{30} + \dfrac{5}{30} \\
&= \dfrac{9}{30} = \dfrac{3}{10} = 0{,}3.
\end{align*}

