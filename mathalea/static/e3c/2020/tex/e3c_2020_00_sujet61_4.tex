
\medskip

Une entreprise vend des smartphones d'un seul modèle \og haut de gamme \fg.

Le service marketing modélise le nombre de smartphones modèle \og haut de gamme\fg{}
vendus par trimestre en fonction du prix de vente $x$ par la fonction $N$ définie par 

\[N(x) = 100\e^{-2x}\,  \text{où :}\]


\setlength\parindent{1cm}
\begin{itemize}
\item $x$ est le prix de vente en \textbf{milliers d'euros} d'un smartphone modèle \og haut de gamme \fg.

 Le prix du smartphone modèle \og haut de gamme\fg{} est compris entre 400~\euro{} et \np{2000}~\euro{} ; on a donc $x \in [0,4~;~2]$.
\item $N(x)$ est le nombre de smartphones modèle \og haut de gamme\fg{} vendus trimestriellement en \textbf{millions d'unités}.
\end{itemize}
\setlength\parindent{0cm}

\smallskip

\begin{enumerate}
\item Si le service commercial fixe le prix de vente de ce smartphone modèle \og haut de gamme\fg{} à \np{1000}~\euro, quel sera le nombre de smartphones vendus trimestriellement ? 

On arrondira le résultat à mille unités.
\end{enumerate}

La recette trimestrielle $R(x)$ est obtenue en multipliant le nombre de smartphones modèle \og haut de gamme\fg{} vendus par le prix de vente. 

On obtient $R(x) = x \times  N(x)$ en \textbf{milliards d'euros}.

Le coût de production en milliards d'euros en fonction du nombre de smartphones modèle \og haut de gamme\fg{} fabriqués est modélisé par la fonction $C$ définie par $C(x) = 0,4 \times N(x)$ où $x$ est le prix de vente en \textbf{milliers d'euros}.

Le bénéfice est obtenu en calculant la différence entre la recette et le coût de production.

\smallskip

\begin{enumerate}[resume]
\item Vérifier que le bénéfice trimestriel peut être estimé à $8,120$ milliards d'euros pour un prix de vente \np{1000}~\euro.
\item Montrer que le bénéfice trimestriel s'exprime en milliards d'euros en fonction du prix de vente $x$ en milliers d'euros par : $B(x) = (100x - 40) \e^{-2x}.$
\item On admet que pour tout réel $x \in  [0,4~;~2],\: B'(x) = (180 - 200x )\e^{-2x}$.

 Étudier les variations de la fonction $B$ sur l'intervalle [0,4~;~2].
\item À quel prix faut-il vendre ces smartphones pour assurer un bénéfice maximal ?
\end{enumerate}
