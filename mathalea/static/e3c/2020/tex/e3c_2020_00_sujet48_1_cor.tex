
\subsection*{Question 1}

On constate avec les deux points de la tangente de coordonnées \( (2\,;\,2) \) et \( (5\,;\,4) \) que celle-ci a un coefficient directeur égal à :
\[
f'(2) = \dfrac{4 - 2}{5 - 2} = \dfrac{2}{3}.
\]

Une équation de cette tangente est :
\[
y - f(2) = f'(2)(x - 2) \quad \text{soit} \quad y - 2 = \dfrac{2}{3}(x - 2) \quad \text{ou} \quad y = \dfrac{2}{3}(x - 2) + 2.
\]

\subsection*{Question 2}

On voit que si \( A \) correspond à l'arc de mesure \( \alpha \), alors \( \cos(\alpha) = -\dfrac{1}{2} \), donc \( \alpha = -\dfrac{2\pi}{3} \).

\subsection*{Question 3}

\[
f(x) = ax^2 + bx = x(ax + b).
\]
Ce trinôme a donc deux racines : 0 et \( -\dfrac{b}{a} < 0 \) car \( a \) et \( b \) sont tous les deux supérieurs à zéro. Donc réponse \textbf{d.}

\subsection*{Question 4}

La droite \( \mathcal{D} \) a pour vecteur directeur \( \vec{d} \begin{pmatrix} 2 \\ 1 \end{pmatrix} \), d'où : \(\vec{d} \cdot \vec{u} = 2 - 2 = 0\).

Les vecteurs \( \vec{d} \) et \( \vec{u} \) sont orthogonaux : réponse \textbf{b.}

\subsection*{Question 5}

Les distances parcourues chaque jour sont les termes d'une suite arithmétique de premier terme 12 (km) et de raison \(-0,5\) (km).

Si \( d_n \) est la distance parcourue le \( n \)-ième jour, alors :
\[
d_n = 12 - 0,5(n - 1).
\]

On a :
\[
S_{10} = 12 + 11{,}5 + 11 + \dots + 12 - 4{,}5,
\]
que l'on peut écrire :
\[
S_{10} = 12 - 4{,}5 + 12 - 4 + \dots + 12.
\]
En sommant membre à membre :
\begin{align*}
2S_{10} &= 10 \times (24 - 4{,}5) \\
&= 240 - 45 \\
&= 195,
\end{align*}
d'où :
\[
S_{10} = \dfrac{195}{2} = 97{,}5.
\]

