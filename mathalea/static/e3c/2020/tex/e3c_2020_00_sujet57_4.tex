
\medskip

Dans un repère orthonormé du plan, on considère les points A$(-2~;~1)$, B$(1~;~2)$ et
E$(0~;~-5)$. 

On appelle $\mathscr{C}$ le cercle de centre A passant par B.

\medskip

\begin{enumerate}
\item Justifier qu'une équation du cercle $\mathscr{C}$ est $(x + 2)^2 + (y - 1)^2 = 10.$
\item Calculer $\vv{AB}\cdot\vv{AE}$.
\item Que peut-on en déduire pour les droites (AB) et (AE) ?
\item Déterminer une équation cartésienne de la droite (AE).
\item Calculer les coordonnées des points d'intersection de (AE) et du cercle $\mathscr{C}$.
\end{enumerate}
