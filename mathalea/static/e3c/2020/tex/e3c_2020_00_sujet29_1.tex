
\medskip

Cet exercice est un questionnaire à choix multiple (QCM).

Pour chacune des questions, une seule des  quatre réponses proposées est correcte. 

Pour chaque question, indiquer le numéro de la question et recopier sur la copie la lettre correspondante à la réponse choisie.

Aucune justification n'est demandée mais il peut être nécessaire d'effectuer des recherches au
brouillon pour aider à déterminer la réponse.

Chaque réponse correcte rapporte 1 point. Une réponse incorrecte ou une question sans réponse
n'apporte, ni ne retire de point.

\medskip

\begin{enumerate}
\item L'arbre pondéré ci-dessous représente une situation où A, B, C et D sont des
évènements d'une expérience aléatoire :

\begin{center}

\psset{nodesepA=0pt,nodesepB=3pt,treesep=0.75,labelsep=0.1pt,levelsep=2.75cm}
\pstree[treemode=R]{\TR{}}
{\pstree{\TR{$A$~~}\taput{$\np{0.12}$}}
	{
	\TR{$D$}\taput{$\np{0.5}$}
	\TR{$\overline{D}$}\tbput{$\dots$}
	}
\pstree{\TR{$B$~~}\taput{$\np{0.24}$}}
	{\TR{$D$}\taput{$\dots$}
	\TR{$\overline{D}$}\tbput{$\np{0.8}$}
	}
\pstree{\TR{$C$~~}\tbput{$\dots$}}
	{\TR{$D$}\taput{$\dots$}
	\TR{$\overline{D}$}\tbput{$\np{0.9}$}
	}	
}

\end{center}

La probabilité de l'évènement $D$ est égale à :

\medskip

\begin{tabularx}{\linewidth}{*{4}{X}}
\textbf{a.~~} $0,06 $ &\textbf{b.~~} $0,8 $&\textbf{c.~~}$ 0,5 $& \textbf{d.~~} $0,172  $.
\end{tabularx}

\medskip

\item L'ensemble des solutions réelles de l'inéquation $- 2x^2 - 5x + 3 < 0 $ est :

\medskip

\begin{tabularx}{\linewidth}{*{2}{X}}
\textbf{a.~~} $\left] - 3~;~\dfrac{1}{2}\right[$ &\textbf{b.~~} $\left] -\infty~;~-3  \right[\cup \left] \dfrac{1}{2}~;~+\infty\right[ $ \\\textbf{c.~~}$ \left] -\infty~;~-\dfrac{1}{2}\right[\cup\left]3~;~+\infty\right[ $& \textbf{d.~~} $\left] -\dfrac{1}{2}~;~3\right[  $.
\end{tabularx}

\medskip

\item On considère la droite $\mathcal{D}$ d'équation $2x - 8y + 1 = 0$.

Les coordonnées d'un vecteur normal à $\mathcal{D}$ sont :

\medskip

\begin{tabularx}{\linewidth}{*{4}{X}}
\textbf{a.~~}$ \dbinom{1}{-4}$ &\textbf{b.~~} $\dbinom{8}{-2} $&\textbf{c.~~}$\dbinom{-8}{2} $& \textbf{d.~~} $\dbinom{-4}{1}$.
\end{tabularx}

\medskip

\item Dans un repère orthonormé, l'équation du cercle de centre $A (-2~;~-4)$ et de rayon 2
est :

\medskip

\begin{tabularx}{\linewidth}{*{2}{X}}
\textbf{a.~~}$ x^2 - 4x + y^2 - 8y + 16 = 0$ &\textbf{b.~~} $x^2 + 4x + y^2 + 8y + 16 = 0 $\\\textbf{c.~~}$ x^2 - 4x + y^2 - 8y + 18 = 0$& \textbf{d.~~} $x^2 + 4x + y^2 + 8y+ 18 = 0  $.
\end{tabularx}

\medskip

\item On considère la suite $\left(u_n\right)$ définie par : $u_0 = 1$ et pour tout entier naturel non nul $n$, 

\[u_{n+1} = u_n + 2n - 3\]

\medskip

\begin{tabularx}{\linewidth}{*{2}{X}}
\textbf{a.~~} $u_1 = 0 $ &\textbf{b.~~} $\left(u_n\right)$ est arithmétique \\\textbf{c.~~}$u_3 = -2 $& \textbf{d.~~} $\left(u_n\right)$ est décroissante.
\end{tabularx}
\end{enumerate}

\vspace{0,5cm}

