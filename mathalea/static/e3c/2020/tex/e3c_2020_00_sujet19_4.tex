  
\medskip

Une résidence de vacances propose uniquement deux formules :

\begin{itemize}
\item la formule \og pension complète \fg{} dans laquelle 3 repas par jour sont fournis ;
\item la formule \og demi-pension \fg{} dans laquelle sont fournis uniquement le petit déjeuner et le dîner.
\end{itemize}
Pour l’année 2018, 65\,\% des clients ont choisi la pension complète ; les autres ont choisi la formule \og demi-pension \fg.

Parmi les clients qui ont choisi la demi-pension, 30\,\% ont réservé l’option \og ménage \fg{} en fin de semaine.

De plus, 70\,\% des clients qui ont choisi la pension complète ont réservé l’option \og ménage\fg.

On choisit un client au hasard parmi ceux de l’année 2018 et l’on considère les évènements suivants :

\begin{itemize}
\item C : le client a choisi la formule \og pension complète \fg{} ;
\item M : le client a choisi l’option \og ménage \fg.
\end{itemize}

\medskip

\begin{enumerate}
\item  Recopier sur la copie et compléter l’arbre pondéré ci-dessous :
\begin{center}
\psset{nodesepA=0pt,nodesepB=3pt,treesep=0.75,labelsep=0.1pt,levelsep=2.5cm}
\pstree[treemode=R]{\TR{}}
{\pstree{\TR{$C$~~}\taput{$\np{0.65}$}}
	{
	\TR{$M$}\taput{$\dots$}
	\TR{$\overline{M}$}\tbput{$\dots$}
	}
\pstree{\TR{$\overline{C}$~~}\tbput{$\dots$}}
	{\TR{$M$}\taput{$\dots$}
	\TR{$\overline{M}$}\tbput{$\dots$}
	}
}
\end{center}
\item Calculer $p(C\cap M)$.
\item Montrer que la probabilité que le client ait réservé l’option \og ménage\fg{} est égale à $0,56$.
\item Calculer la probabilité que le client ait choisi la formule \og pension complète \fg{} sachant qu’il a réservé l’option \og ménage\fg.
\item Voici la grille de tarifs de la résidence de vacances pour l’année 2018:

\begin{center}\begin{tabular}[]{|lr|}
\hline
Une semaine de pension complète&800~\euro\\
Une semaine de demi-pension&650~\euro\\
Option  \og ménage \fg&50~\euro\\
\hline
\end{tabular}
\end{center}

On note $X$ la variable aléatoire égale au montant payé par un client de 2018.

Calculer $p(X = 850)$.
\end{enumerate}
