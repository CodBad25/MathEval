
\medskip

\emph{Cet exercice est un QCM et comprend cinq questions. Les questions sont indépendantes.}

\emph{Pour chacune d'elles, une seule des réponses proposées est exacte.}

\emph{Pour chaque question, indiquer le numéro de la question et recopier sur la copie la lettre
correspondant à la réponse choisie.}
\emph{Aucune justification n'est demandée, mais il peut être nécessaire d'effectuer des recherches au brouillon pour aider à déterminer votre réponse.}
\emph{Chaque réponse correcte rapporte $1$ point. Une réponse incorrecte ou une absence de
réponse n'apporte, ni ne retire de point.}

\medskip

\textbf{Question 1}


\begin{minipage}[]{3.3cm}
On donne ci-contre la courbe représentative
$\mathcal{C}_f$ d'une fonction $f$.
Cette courbe a une tangente $\mathcal{T}$ au point
$A(-3~;~3)$.
\end{minipage}\hfill
\begin{minipage}[]{11.75cm}
\psset{xunit=1.3cm,yunit=0.25cm,labelFontSize=\scriptstyle}
\begin{pspicture}(-4.5,-6.5)(4.5,21)
 \multido{\n=-4+1}{9}{\psline[linewidth=0.75pt,linecolor=lightgray,linestyle=dashed](\n,-6)(\n,20.2)}
\multido{\n=-6+1}{27}{\psline[linewidth=0.75pt,linecolor=lightgray,linestyle=dashed](-4.1,\n)(4.2,\n) }
\psaxes[linewidth=0.95pt,Dy=5]{->}(0,0)(-4.2,-6)(4.3,20.6)
\psplot[linewidth=1.25pt,linecolor=blue,plotpoints=5000]{-4}{4}{x  x mul  0.33333 mul  4 sub x mul}
\psplot[linewidth=1.25pt,linecolor=red,plotpoints=5000]{-4}{0.2}{x  5 mul  18 add}
\uput[ul](-3,3){A}\psdot[dotstyle=+,dotscale =1.74,dotangle=45](-3,3)
\uput[dl](3.8,5){\blue $\mathcal{C}_f$}\uput[ur](0.2,18){$\mathcal{T}$}
\end{pspicture}
\end{minipage}

L'équation réduite de cette tangente est :

\begin{tabularx}{\linewidth}{*{4}{X}}
\textbf{a.~~} $y =\dfrac{1}{5}x - 3,7$ &\textbf{b.~~} $ y =\dfrac{1}{5} x + 18$&\textbf{c.~~}$y = 5x + 18$& \textbf{d.~~} $y = 5x -3,7$.
\end{tabularx}

\medskip

\textbf{Question 2}
\medskip 

On reprend la fonction $f$ de la question précédente. La représentation graphique de sa
fonction dérivée est :

\begin{tabularx}{\linewidth}{*{2}{X}}
\textbf{a.~~}
\psset{xunit=0.5cm,yunit=0.25cm,labelFontSize=\scriptstyle}
\begin{pspicture}(-4.5,-5)(4.5,5.5)
\psaxes[linewidth=0.95pt,labels=x,Dy=10]{->}(0,0)(-4.2,-6)(4.3,6)
\psplot[linewidth=0.8pt,linecolor=blue,plotpoints=5000]{-4}{4}{x  x mul  0.33333 neg mul  4 add x mul }
\end{pspicture}
&\textbf{b.~~}
\psset{xunit=0.5cm,yunit=0.25cm,labelFontSize=\scriptstyle}
\begin{pspicture}(-4.5,-4.5)(4.5,5.5)
\psaxes[linewidth=0.95pt,labels=x,Dy=10]{->}(0,0)(-4.2,-4.6)(4.3,4.6)
\psplot[linewidth=0.8pt,linecolor=blue,plotpoints=5000]{-3}{3}{x  x mul  4 sub  }
\end{pspicture}
\\\hline
\textbf{c.~~}
\psset{xunit=0.5cm,yunit=0.25cm,labelFontSize=\scriptstyle}
\begin{pspicture}(-4.5,-2.5)(4.5,2.5)
\psaxes[linewidth=0.95pt,labels=x,Dy=10]{->}(0,0)(-4.2,-2)(4.3,5)
\psplot[linewidth=0.8pt,linecolor=blue,plotpoints=5000]{-2}{2}{x  x mul    1 sub }
\end{pspicture}
& \textbf{d.~~}
\psset{xunit=0.5cm,yunit=0.25cm,labelFontSize=\scriptstyle}
\begin{pspicture}(-4.5,-4.5)(4.5,5.5)
\psaxes[linewidth=0.95pt,labels=x,Dy=10]{->}(0,0)(-4.2,-4.6)(4.3,4.6)
\psplot[linewidth=0.8pt,linecolor=blue,plotpoints=5000]{-3}{3}{x  x mul   neg   4 add  }
\end{pspicture}
\\\hline
\end{tabularx}

\medskip
\textbf{Question 3}
\medskip 

L'expression $\cos(x + \pi) + \sin (x +\frac{\pi}{2})$ est égale à :

\begin{tabularx}{\linewidth}{*{4}{X}}
\textbf{a.~~} $-2 \cos(x)$ &\textbf{b.~~} $0 $&\textbf{c.~~}$\cos(x) + \sin (x)$& \textbf{d.~~} $2\cos(x)$.
\end{tabularx}
 
\medskip
\textbf{Question 4}
\medskip 

On considère la fonction polynôme du second degré $f$ définie sur $\R$ par
$f (x) = -2x^2 + 4x + 6$.

Cette fonction est strictement positive sur l'intervalle :

\begin{tabularx}{\linewidth}{*{4}{X}}
\textbf{a.~~} $] -\infty ; -1[\cup ]3; +\infty[$ &\textbf{b.~~}$] -1 ; 3[ $&\textbf{c.~~}$] - \infty; -3[ \cup  ]1; +\infty[$& \textbf{d.~~}$] - 3 ; 1[$ .
\end{tabularx}


\medskip
\textbf{Question 5}
\medskip 

On considère la fonction $h$ définie sur $\R$ par $h(x) = (2x -1)\e^x$ .

La fonction dérivée de la fonction $h$ est définie sur $\R$ par :

\begin{tabularx}{\linewidth}{*{4}{X}}
\textbf{a.~~} $h'(x) = 2\e^x$  &\textbf{b.~~}$h'(x) = (2x + 1)\e^x$ &\textbf{c.~~}$ h'(x) = (2x -1)e^x$& \textbf{d.~~} $h'(x) = -e^x$.
\end{tabularx}

\vspace{1cm}

