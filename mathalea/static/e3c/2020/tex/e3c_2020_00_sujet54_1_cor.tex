
\subsection*{Question 1}

Si \( \sin x = \dfrac{1}{3} \), alors la réponse est \textbf{a.}

\subsection*{Question 2}

Eéponse \textbf{d.} : la parabole n'a pas de point commun avec l'axe des abscisses.

\subsection*{Question 3}

On a : \(f'(x) = 2 + \dfrac{1}{x^2}\),

donc en particulier : \(f'(1) = 2 + 1 = 3\).

Réponse \textbf{b.}

\subsection*{Question 4}

\begin{align*}
&x^2 - 2x + y^2 + 6y + 2 = 0  \\
\iff &(x - 1)^2 - 1 + (y + 3)^2 - 9 + 2 = 0 \\
\iff &(x - 1)^2 + (y + 3)^2 = 8
\end{align*}
Ceci est l'équation du cercle de centre \( \Omega(1\,;\,-3) \) et de rayon \( \sqrt{8} = 2\sqrt{2} \).

\subsection*{Question 5}

L'espérance mathématique de la variable aléatoire \( X \) est :
\[
E(X) = -10 \times \dfrac{1}{4} + 6 \times \dfrac{3}{8} + 10 \times \dfrac{3}{8} = -2{,}5 + 2{,}25 + 3{,}75 = 3{,}50 \text{ (€)}.
\]
\(\textit{Remarque : }  3{,}50\) et non 3{,}5 car le décime d'euro n'existe pas en Europe.

