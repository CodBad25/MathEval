
\medskip

On applique une tension sinusoïdale $u$ aux bornes d'un circuit électrique comportant en série une résistance et une diode idéale.

Le temps $t$ est exprimé en seconde.

La tension est donnée par la fonction $u$ définie pour tout réel $t \geqslant 0$ par:

\[u(t) = \sqrt{3}\sin \left(100\pi t + \dfrac{\pi}{3}\right).\]

La diode est non passante si $u(t) \leqslant \dfrac{\sqrt{3}}{2}$ et elle est passante si $u(t) > \dfrac{\sqrt{3}}{2}$.

\medskip

\begin{enumerate}
\item La diode est-elle passante à l'instant $t = 0$ ?
\item Calculer $u\left(\dfrac{1}{100}\right)$. Interpréter le résultat.
\item On admet que $u\left(t + \dfrac{2}{100}\right) = u(t)$ pour tout $t \geqslant 0$. 

En déduire une propriété de la fonction $u$.
\item On donne ci-dessous la courbe représentative de la fonction $u$ sur l'intervalle [0~;~ 0,02] :

\begin{center}
\psset{xunit=500cm,yunit=1cm,algebraic=true,comma=true}
\begin{pspicture}(-0.0001,-2)(0.02,2)
\multido{\n=0.000+0.001}{21}{\psline[linewidth=0.2pt](\n,-2)(\n,2)}
\multido{\n=-2.0+0.5}{9}{\psline[linewidth=0.2pt](0,\n)(0.02,\n)}
\psaxes[linewidth=1.25pt,Dx=0.005,labelFontSize=\scriptstyle]{->}(0,0)(0,-2)(0.02,2)
\psplot[plotpoints=2000,linewidth=1.25pt,linecolor=blue]{0}{0.02}{1.73205*sin(314.159*x+1.0472)}
\end{pspicture})
\end{center}
On cherche à savoir au bout de combien de temps la diode devient non
passante pour la première fois.
	\begin{enumerate}
		\item Conjecturer la solution du problème à l'aide du graphique.
		\item Calculer $u(0,005)$ et conclure.
	\end{enumerate}
\end{enumerate}

\medskip


