
\subsection*{1.}

Retrancher 4 \% de quelque chose revient à le multiplier par \(1 - \dfrac{4}{100} = 1 - 0{,}04 = 0{,}96\).

Quel que soit le naturel \(n\), on a donc \(u_{n+1} = 0{,}96u_n\), ce qui montre que la suite \((u_n)\) est une suite géométrique de raison \(q = 0{,}96\) et de premier terme \(u_0 = 900\).

\subsection*{2.}

On obtiendra pour \texttt{Suite(5)} : \( \approx 733{,}835 \), soit environ 734.

\subsection*{3.}

On sait que, pour tout naturel \(n\), \(u_n = 900 \times 0{,}96^n\).

\subsection*{4.}

2030 correspond à \(n = 11\) et \(u_{11} = 900 \times 0{,}96^{11} \approx 574\) : la maternité sera donc fermée en 2030.

\subsection*{5.}

On a \(u_9 \approx 623\) et \(u_{10} \approx 598\).

La maternité devrait fermer en 2029.

