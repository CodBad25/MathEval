
\medskip

Afin d'établir les liens entre le surpoids et l'alimentation, on interroge les enfants des écoles primaires d'une ville.

L'enquête révèle que 60\,\% des enfants boivent 1 boisson sucrée ou plus par jour.

Parmi les enfants buvant 1 boisson sucrée ou plus par jour, un enfant sur 8 est en surpoids, contre seulement 8\,\% pour les enfants buvant moins d'une boisson sucrée par jour.
 
On choisit un enfant au hasard parmi les enfants des écoles primaires de la ville et on considère les évènements suivants :

\begin{itemize}
\item $B$ : \og l'enfant boit 1 boisson sucrée ou plus par jour \fg,
\item $S$ : \og l'enfant est en surpoids \fg.
\end{itemize}

Les évènements contraires de $B$ et de $S$ sont notés respectivement $\overline{B}$ et $\overline{S}$.

Pour tout évènement $A$ et $B$, avec $B$ un évènement de probabilité non nulle, la probabilité de $A$ sachant $B$ est notée $p_B(A)$.

\medskip

\begin{enumerate}
\item Justifier que $p_B(S) = 0,125$.
\item Représenter la situation par un arbre pondéré.
\item Calculer $p(B \cap S)$.
\item Déterminer la probabilité que l'enfant soit en surpoids.
\item On a choisi un enfant en surpoids. Quelle est la probabilité qu'il boive une boisson sucrée ou plus par jour ? On arrondira le résultat au millième.
\end{enumerate}

\vspace{0,5cm}

