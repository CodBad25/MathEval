
\medskip

Une balle en caoutchouc est lâchée sans vitesse initiale d'une hauteur de 2 mètres au-dessus du sol.

Le choc n'étant pas parfaitement élastique, la balle rebondit jusqu'à une hauteur de
1,60~mètre et continue à rebondir, en atteignant après chaque rebond une hauteur égale au
$\dfrac{4}{5}$ de la hauteur du rebond précédent.

On modélise les hauteurs atteintes par la balle par une suite $\left(h_n\right)$ où pour tout entier naturel $n$, $h_n$ est la hauteur, exprimée en mètres, atteinte par la balle au $n$-ième  rebond.On a alors $h_0 = 2$.

\medskip

\begin{enumerate}
\item 
	\begin{enumerate}
		\item Donner $h_1$ et $h_2$.
		\item Pour tout entier naturel $n$, exprimer $h_{n+1}$ en fonction de $h_n$.
		\item En déduire la nature de la suite $\left(h_n\right)$. On précisera sa raison et son premier terme. 
		\item Déterminer le sens de variation de la suite $\left(h_n\right)$.
	\end{enumerate}
\item  Déterminer le nombre minimal $N$ de rebonds à partir duquel la hauteur atteinte par la balle est inférieure à $20$cm. Expliquer la démarche employée.
\end{enumerate}

\bigskip

