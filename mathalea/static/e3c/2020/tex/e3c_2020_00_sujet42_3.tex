
\medskip

\textbf{Partie A}

\medskip

Étudier sur $\R$ le signe de $P(x)= - 10x^2 - 40x + 120$.

\medskip

\textbf{Partie B}

\medskip

On se place dans un plan muni d'un  repère orthonormé. La courbe $H$ représentée sur le graphique ci -dessous est l'ensemble des points de l'hyperbole d'équation:

\[y = \dfrac{10x + 4}{x + 2}\]

avec $x$ appartenant à l'intervalle [0~;~8].

Pour toute abscisse $x$ dans l'intervalle [0~;~8], on construit le rectangle ABDE comme indiqué sur la figure. On donne les informations suivantes :

\medskip

\parbox{0.35\linewidth}{
\setlength\parindent{1cm}
\begin{itemize}[label=\textbullet]
\item A et B sont sur l'axe des abscisses;
\item A est d'abscisse $x$;
\item B et D ont pour abscisse 8 ;
\item E appartient à la courbe H ;
\item D et E ont la même ordonnée.
\end{itemize}
\setlength\parindent{0cm}
}\hfill
\parbox{0.63\linewidth}{\psset{unit=0.7cm}
\begin{pspicture}(-1,-1)(9,9)
\psframe[fillstyle=solid,fillcolor=lightgray](1,0)(8,4.66667)
\psgrid[gridlabels=0pt,subgriddiv=1](0,0)(9,9)
\psaxes[linewidth=1.25pt,labelFontSize=\scriptstyle]{->}(0,0)(0,0)(9,9)
\psplot[plotpoints=2000,linewidth=1.25pt,linecolor=red]{0}{8}{10 x mul 4 add x 2 add div}
\uput[dl](1,0){\footnotesize A}\uput[dr](8,0){\footnotesize B}
\uput[ur](8,4.66667){\footnotesize D}\uput[ul](1,4.66667){\footnotesize E}
\uput[u](3.2,7){\red $H$}
\end{pspicture}}

\medskip

L'objectif de ce problème est de déterminer la ou les valeurs éventuelles $x$ de l'intervalle [0~;~8] correspondant à un rectangle ABDE d'aire maximale.

\medskip

\begin{enumerate}
\item Déterminer l'aire du rectangle ABDE lorsque $x = 0$.
\item Déterminer l'aire du rectangle ABDE lorsque $x =  4$.
\end{enumerate}

On définit la fonction $f$ qui à tout réel $x$ de [0~;~8], associe l'aire du rectangle ABDE. On admet que:

\[f(x) = \dfrac{-10x^2 + 76x + 32}{x + 2}.\]

\begin{enumerate}[resume]
\item Répondre au problème posé.
\end{enumerate}

\bigskip

