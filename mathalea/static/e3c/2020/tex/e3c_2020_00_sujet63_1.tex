
\medskip

Ce QCM comprend 5 questions. Pour chacune des questions, une seule des quatre réponses proposées est correcte. Les questions sont indépendantes.

Pour chaque question, indiquer le numéro de la question et recopier sur la copie la lettre correspondante à la réponse choisie.

Aucune justification n'est demandée mais il peut être nécessaire d'effectuer des recherches au brouillon pour aider à déterminer votre réponse.

Chaque réponse correcte rapporte $1$ point. Une réponse incorrecte ou une question sans réponse n'apporte, ni ne retire de point.

\medskip

\textbf{Question 1}

Une fonction du second degré $f$ a pour forme canonique valable pour tout réel $x$ :

\[f(x) = 3(x + 2)^2 + 5.\]

Concernant son discriminant :

\begin{center}
\begin{tabularx}{\linewidth}{|*{4}{X|}}\hline
\textbf{a.~~} on peut dire qu'il est nul&\textbf{b.~~} on peut dire qu'il est nul est strictement
positif&\textbf{c.~~} on peut dire qu'il est strictement négatif&\textbf{d.~~}on ne peut rien dire sur son signe\\ \hline
\end{tabularx}
\end{center}

\textbf{Question 2}

Un vecteur directeur de la droite d'équation $2x + 3y + 5 = 0$ est : 

\begin{center}
\begin{tabularx}{\linewidth}{|*{4}{X|}}\hline
\textbf{a.~~}$\vect{u}(2~;~3)$&\textbf{b.~~}$\vect{u}(-3~;~2)$ &\textbf{c.~~}$\vect{u}(3~;~2)$&\textbf{d.~~}$\vect{u}(- 2~;~3)$\rule[-3mm]{0mm}{8mm}\\ \hline
\end{tabularx}
\end{center}

\textbf{Question 3}

Dans un repère orthonormé du plan, on considère les points A$(3~;~-1)$, B$(4~;~2)$ et C$(1~;~1)$.

Le produit scalaire $\vect{\text{AB}} \cdot \vect{\text{AC}}$ est égal à :

\begin{center}
\begin{tabularx}{\linewidth}{|*{4}{X|}}\hline
\textbf{a.~~}$- 4$&\textbf{b.~~}$2$&\textbf{c.~~} $4$&\textbf{d.~~}$8$\\ \hline
\end{tabularx}
\end{center}

\textbf{Question 4}

Soit $g$ la fonction définie sur l'ensemble des nombres réels par $g(x) = (2x + 1)\text{e}^x$.

Pour tout réel $x$,\, $g'(x)$ est égal à :

\begin{center}
\begin{tabularx}{\linewidth}{|*{4}{X|}}\hline
\textbf{a.~~}$2\text{e}^x$&\textbf{b.~~}$2x\text{e}^x$&\textbf{c.~~} $ (2x +2)\text{e}^x$&
\textbf{d.~~}$(2x +3)\text{e}^x$\\ \hline
\end{tabularx}
\end{center}

\textbf{Question 5}

Pour tout réel $x$,\, $\sin(x + \pi)$ est égal à :

\begin{center}
\begin{tabularx}{\linewidth}{|*{4}{X|}}\hline
\textbf{a.~~}$\cos x$&\textbf{b.~~}$\sin x$&\textbf{c.~~} $- \cos x$&\textbf{d.~~}$- \sin x$\\ \hline
\end{tabularx}
\end{center}

\bigskip

