	\section*{Exercice 4 (5 points)}
	
\subsection*{1. Calculer le nombre $u_1$ de visionnages une semaine après le début de la diffusion.}
	
Augmenter de 2\%, revient à multiplier par $1 + 0,02 = 1,02$.
	\[
	u_1 = u_0 \times 1,02 = 120000 \times 1,02 = 122400
	\]
	
	\subsection*{2. Justifier que pour tout entier naturel $n$, $u_n = 120000 \times 1,02^n$.}
	
La suite $(u_n)$ est donc une suite géométrique de premier terme $120 000$ et de raison $1,02$. On sait qu’alors pour tout naturel $n$, $u_n = 120000 \times 1,02^n$.
	
	\subsection*{3. À partir de combien de semaines le nombre de visionnages hebdomadaire sera-t-il supérieur à 150 000 ?}
	
	Il faut résoudre dans $\mathbb{N}$, l’inéquation :
	\[
	120000 \times 1,02^n > 150000 \quad \text{donc} \quad 1,02^n > \dfrac{15}{12} = \dfrac{5}{4} = 1,25
	\]
	
	La calculatrice donne $u_{11} \approx 1,24$ et $u_{12} \approx 1,26824$. Il y aura plus de 150 000 visionnages la 12ème semaine.
	
\subsection*{4. Voici un algorithme écrit en langage Python :}
L’algorithme affichera 60. Cela signifie que la 60ème semaine il y aura plus de 400 000 visionnages.
	
	\subsection*{5. On pose pour tout entier naturel $n$ : $S_n = u_0 + \ldots + u_n$. Montrer que l’on a : $S_n = 6000000 \times (1,02^{n+1} - 1)$. En déduire le nombre total de visionnages au bout de 52 semaines (arrondir à l’unité).}
	
\begin{align*}
	S_n& = u_0 + \ldots + u_n = 120000 + 120000 \times 1,02 + 120000 \times 1,02^2 + \ldots + 120000 \times 1,02^n\\
	1,02 S_n& = 120000 \times 1,02 + 120000 \times 1,02^2 + \ldots + 120000 \times 1,02^{n+1}
\end{align*}
D’où par différence :
	$$	0,02 S_n = 120000 \times (1,02^{n+1} - 1) $$
	et par conséquent en multipliant par $50$ (inverse de $0,02$) :\\
	$$ S_n = 50 \times 120000 (1,02^{n+1} - 1) = 6000000 (1,02^{n+1} - 1)$$
En particulier :
	\[
	S_{52} = 50 \times 120000 (1,02^{53} - 1) = 6000000 (1,02^{53} - 1) \approx 11138008,4 	\]
	Soit environ $11 138 008$ visionnages à l’unité près

	
