
\medskip

\emph{Ce QCM comprend cinq questions indépendantes.}

\emph{Pour chacune des questions, une seule des quatre réponses proposées est correcte.}

\emph{Pour chaque question, indiquer le numéro de la question et recopier sur la copie la lettre correspondant à la réponse choisie.}

\emph{ \textbf{Aucune justification n’est demandée mais il peut être nécessaire d’effectuer des recherches} au brouillon pour aider à déterminer votre réponse.}

\emph{Chaque réponse correcte rapporte $1$ point. \textbf{Une réponse incorrecte ou une question sans réponse n’apporte ni ne retire de point}.}

\medskip

\textbf{Question 1}

\medskip

Sur la figure ci-dessous, nous avons tracé dans un repère orthonormé la courbe représentative $\mathcal{C}$ d’une fonction $f$ dérivable sur $\R$ et la tangente à $\mathcal{C}$ au point d’abscisse 4. Cette tangente est représentée par la droite $\mathcal{D}$. On note $f'$ la fonction dérivée de la fonction $f$ .

\begin{center}
\psset{labelFontSize=\scriptstyle,showorigin=false}
\begin{pspicture}(-3.5,-3.8)(6,4.5)
\multido{\n=-2+1}{8}{\psline[linewidth=0.75pt,linecolor=lightgray](\n,-3.5)(\n,4.2)}
\multido{\n=-3+1}{8}{\psline[linewidth=0.75pt,linecolor=lightgray](-2.5,\n)(5.3,\n)}
\psaxes[linewidth=0.95pt,]{->}(0,0)(-2.9,-3)(5.5,4.4)
\psdot[dotstyle=+,dotscale =1.4,dotangle=45](4,-1)
\uput[ur](4,-1){A} 
\psplot[linewidth=1.15pt,linecolor=blue,plotpoints=2000]{-1}{5}{x 2 sub 2 exp 0.5 neg  mul 1 add}
\psplot[linewidth=1.15pt,linecolor=red,plotpoints=2000]{1.32}{5}{x 2 neg   mul 7 add}
\uput[dl](0,0){O}\uput[d](0.5,0.7){\blue $\mathcal{C}$}\uput[d](2.25,3.5){\red $\mathcal{D}$}
\end{pspicture}
\end{center}

Le réel $f'(4)$ est égal à :

\medskip

\begin{tabularx}{\linewidth}{*{4}{X}}
\textbf{a.~~} $-1$ &\textbf{b.~~} $-2$&\textbf{c.~~} $7$& \textbf{d.~~} $1$.
\end{tabularx}

\medskip

\textbf{Question 2}

\medskip

Soit $f$ la fonction définie sur $\R$ par $ f(x) = x^3 - 2x^2 + 1$. On admet que $f$ est une fonction dérivable sur $\R$. Dans un repère, une équation de la tangente à la courbe représentative de la fonction $f$ au point d’abscisse 1 est :

\medskip

\begin{tabularx}{\linewidth}{*{4}{X}}
\textbf{a.~~} $y=-1$ &\textbf{b.~~} $y=-x$&\textbf{c.~~} $y=-x+1$& \textbf{d.~~} $y=x$.
\end{tabularx}

\medskip
\textbf{Question 3}
\medskip

Pour tout réel $x$ , $\dfrac{\e^x\times\e^{-3x}}{\e^{-x}}$ est égal à :

\medskip

\begin{tabularx}{\linewidth}{*{4}{X}}
\textbf{a.~~} $ \e^{-x}$ &\textbf{b.~~} $\e^{3x}$&\textbf{c.~~}$\e^{-3x}$& \textbf{d.~~} $\e^{x}$.
\end{tabularx}

\medskip

\textbf{Question 4}

\medskip

Soit $f$ une fonction polynôme du second degré dont la courbe représentative dans un repère orthonormé est donnée ci-dessous.

\begin{center}
\psset{labelFontSize=\scriptstyle,showorigin=false}
\begin{pspicture}(-4,-5)(3,4)
\multido{\n=-3+1}{6}{\psline[linewidth=0.75pt,linecolor=lightgray](\n,-4.5)(\n,3.2)}
\multido{\n=-4+1}{8}{\psline[linewidth=0.75pt,linecolor=lightgray](-3.5,\n)(2.5,\n)}
\psaxes[linewidth=0.95pt,]{->}(0,0)(-3.5,-4.8)(2.5,3.4)
%   \psdot[dotstyle=+,dotscale =1.4,dotangle=45](4,-1)
%  \uput[ur](4,-1){A} 
\psplot[linewidth=1.15pt,linecolor=blue,plotpoints=5000]{-2.48}{1.5}{x  2 exp 2 mul x 2  mul add 4 sub}
%\psplot[linewidth=1.15pt,linecolor=orange,plotpoints=3000]{1.32}{5}{x 2 neg   mul 7 add}
\uput[dl](0,0){O}%\uput[d](0.5,0.7){$\mathcal{C}$}\uput[d](2.25,3.5){$\mathcal{D}$}
\end{pspicture}
\end{center}

Pour tout réel $x$, une expression de $f(x)$ est :

\medskip

\begin{tabularx}{\linewidth}{*{4}{X}}
\textbf{a.~~}$ f(x)=x^2+x-2$&\textbf{b.~~}$f(x)=-x^2-4$&\textbf{c.~~}$f(x)=2x^2+2x-4$&\textbf{d.~~}$f(x)=-3x^2-3x+6$
\end{tabularx}

\medskip

\textbf{Question 5}

\medskip

L’ensemble  $\mathcal{S}$ des solutions de l’inéquation d’inconnue $x\in\R$ : $-x^2-2x+8>0$ est :

\medskip

\begin{tabularx}{\linewidth}{*{4}{X}}
\textbf{a.~~} $\mathcal{S} = [-4~;~2]$&\textbf{b.~~} $\mathcal{S}=]-4~;~2[$&\textbf{c.~~}\footnotesize $\mathcal{S} = ]-\infty~;~-4]\cup]2~;~+\infty[$&\textbf{d.~~}  $\mathcal{S} = \{-4~;~2\}$\
\end{tabularx}

\vspace{0,5cm}

