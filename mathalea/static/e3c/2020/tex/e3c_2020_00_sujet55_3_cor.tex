
\subsection*{1.}

\begin{align*}
&M(x\,;\,y) \in d_1 \\
\iff &\overrightarrow{CM} \cdot \overrightarrow{AB} = 0 \\
\iff &0 \times (x - 1) + 6(y - 1) = 0 \\
\iff &y - 1 = 0.
\end{align*}

\subsection*{2.}

La droite contenant un sommet et perpendiculaire au côté opposé est une hauteur.

\subsection*{3.}

\begin{align*}
&M(x\,;\,y) \in d_2 \\
\iff &\overrightarrow{BM} \cdot \overrightarrow{AC} = 0 \\
\iff &-6(x - 7) + 3(y - 4) = 0 \\
\iff &-6x + 3y + 42 - 12 = 0 \\
\iff &-2x + y + 10 = 0.
\end{align*}

\subsection*{4.}

Si \( H \) est le point d'intersection de deux hauteurs, c'est donc l'orthocentre, point commun aux trois hauteurs. La troisième hauteur est donc \( (AH) \), qui est perpendiculaire à \( (BC) \). Les vecteurs \( \overrightarrow{AH} \) et \( \overrightarrow{BC} \) sont donc orthogonaux, et leur produit scalaire est donc nul.

\begin{center}
\psset{xunit=0.5cm,yunit=0.5cm,labelsep=0.1pt,labelFontSize=\scriptstyle,showorigin=false}
\begin{pspicture}(-2,-4.5)(11,8)
 \multido{\n=-1+1}{13}{\psline[linewidth=0.75pt,linecolor=lightgray](\n,-4)(\n,7.5)}
 \multido{\n=-4+1}{12}{\psline[linewidth=0.75pt,linecolor=lightgray](-1.5,\n)(11.2,\n)}
 \psaxes[linewidth=0.95pt]{->}(0,0)(-1.5,-4)(11,7.5)
\uput[dl](0,-0.2){\footnotesize O}\uput[l](6.8,-2.5){A}\uput[l](6.8,4.5){B}\uput[l](0.8,0.5){C}\uput[l](5.2,1.4){H}
\psdots[dotstyle=+,dotscale =1.4,dotangle=45](1,1)(7,4)(7,-2)
\psline[linewidth=1pt](7,-2)(7,4)
\psline[linewidth=1pt](7,4)(1,1)
\psline[linewidth=1pt](1,1)(7,-2)
\psline[linewidth=1pt,linecolor=blue](1,1)(8,1)
\psline[linewidth=1pt,linecolor=red](8,-4)(4,4)
\psline[linewidth=1pt,linecolor=blue](4,-2)(8,6)
\end{pspicture}
\end{center}

