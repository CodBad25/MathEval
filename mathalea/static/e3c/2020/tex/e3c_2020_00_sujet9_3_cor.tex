	\section*{Exercice 3 (5 points)}	
	\subsection*{1. Déterminer par lecture graphique les coordonnées des points $A$, $B$, $C$, $K$ et $L$.}
	
	\[
	A(0; 35), B(24; 35), C(24; 0), K(12; 35), L(24; 7)
	\]
	
	\subsection*{2. Un visiteur affirme : « Moins de 70 \% de la surface de la place est éclairée ». Cette affirmation est-elle exacte ?}
	
	\[
	\begin{array}{l}
		\text{Aire éclairée} = \text{Aire}(OCBA) - \text{Aire}(OCL) - \text{Aire}(OKA) \\
		= 24 \times 35 - \dfrac{1}{2} \times 24 \times 7 - \dfrac{1}{2} \times 12 \times 35 \\
		= 24 \times 35 - 12 \times 7 - 6 \times 35 \\
		= 840 - 84 - 210 = 840 - 294 = 456 \text{ u.a.} \\
		\\
		\dfrac{456}{840} = \dfrac{19}{35} \approx 0,54 \text{ soit environ 54\% de la place est éclairée.}
	\end{array}
	\]
L’affirmation est exacte.
	
	\subsection*{3. a. Donner les coordonnées des vecteurs $\overrightarrow{OK}$ et $\overrightarrow{OL}$.}
	
	\[
	\overrightarrow{OK} \left( \begin{array}{c} 12 \\ 35 \end{array} \right) \quad \text{et} \quad \overrightarrow{OL} \left( \begin{array}{c} 24 \\ 7 \end{array} \right)
	\]
	
	\subsection*{b. Montrer que le produit scalaire $\overrightarrow{OK} \cdot \overrightarrow{OL}$ est égal à 533.}
	
	\[
	\overrightarrow{OK} \cdot \overrightarrow{OL} = 12 \times 24 + 35 \times 7 = 288 + 245 = 533
	\]
	
	\subsection*{c. En déduire la mesure, arrondie au degré, de l’angle $\widehat{KOL}$.}
On sait que l’on a aussi :
	\[
	\overrightarrow{OK} \cdot \overrightarrow{OL} = OK \times OL \times \cos(\overrightarrow{OK}; \overrightarrow{OL})
	\]
Avec :
	\[
	OK = \sqrt{12^2 + 35^2} = \sqrt{144 + 1225} = \sqrt{1369} = 37
	\]
	\[
	OL = \sqrt{24^2 + 7^2} = \sqrt{576 + 49} = \sqrt{625} = 25
	\]
L’égalité devient :
	\[
	533 = 37 \times 25 \times \cos(\overrightarrow{OK}; \overrightarrow{OL}) \quad \text{d’où} \quad \cos(\overrightarrow{OK}; \overrightarrow{OL}) = \dfrac{533}{37 \times 25} = \dfrac{533}{925}
	\]
La calculatrice donne $\widehat{KOL} \approx 54,81$ soit $55^\circ$ au degré près.
	
