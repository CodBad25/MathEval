  
\medskip

Ce QCM comprend cinq questions.
 Pour chacune des questions, une seule des quatre
réponses proposées est correcte. 

Les questions sont indépendantes.

Pour chaque question, indiquer le numéro de la question et recopier sur la copie la lettre
correspondante à la réponse choisie.

Aucune justification n’est demandée mais il peut être nécessaire d’effectuer des recherches
au brouillon pour aider à déterminer votre réponse.

Chaque réponse correcte rapporte 1 point. Une réponse incorrecte ou une question sans
réponse n’apporte, ni ne retire aucun point.

\medskip
Question 1
\medskip 


On considère la fonction $f$ définie sur $\R$ par $f(x)=2x^2 +6x-8$.

 Parmi les propositions suivantes, laquelle est juste ?
 
 \begin{tabularx}{\linewidth}{*{2}{X}}
\textbf{a.~~} $f(x)= 2( x - 4)(x +1)$ &\textbf{b.~~} $f(x)= (2x+8)(2x-2) $\\
\textbf{c.~~}$f(x)= 2(x+4)(x-1) $& \textbf{d.~~} $ f(x)= 2(x+3)(x-2)$.
\end{tabularx}

\medskip
Question 2
\medskip 

Pour tout réel $x$, $\dfrac{(\e^x)^2}{\e^{-x}}$ est égal à :

\begin{tabularx}{\linewidth}{*{4}{X}}
\textbf{a.~~} $\e^{x^2+x}$ &\textbf{b.~~} $\e^{3x} $&\textbf{c.~~}$\e^2 $& \textbf{d.~~} $\e^{-2} $.
\end{tabularx}

 \medskip
Question 3
 \medskip 

 Dans le plan muni d’un repère, soit  $\mathcal{C}$ la courbe représentative de la fonction $g$ définie sur $\R$
 par $g(x)=\e^x $. L’équation de la tangente à la courbe $\mathcal{C}$ au point d’abscisse 0 est :
 
 \begin{tabularx}{\linewidth}{*{4}{X}}
\textbf{a.~~} $y=-x-1 $ &\textbf{b.~~} $y=-x+1 $&\textbf{c.~~}$ y=x+1$& \textbf{d.~~} $ y=x$.
\end{tabularx}

 \medskip
 Question 4
 \medskip 

On considère la fonction $f$ définie sur $\R$ par $f(x) = (-x + 1)\e^x$.
On note $f'$ la fonction dérivée de la fonction $f$. Parmi les propositions suivantes, laquelle est
juste ?

\begin{tabularx}{\linewidth}{*{2}{X}}
\textbf{a.~~} $f'(x)=-x\e^x $ &\textbf{b.~~} $f'(x)=(x-2)\e^x $\\\textbf{c.~~}$f'(x)=(-x+2)\e^x $& \textbf{d.~~} $f'(x)= x\e^{-x}$.
\end{tabularx}

\medskip
Question 5
\medskip 

Dans le plan muni d’un repère orthonormal, on considère la courbe représentative d’une fonction~$f$ définie et dérivable sur $\R$.

\begin{center}
\psset{labelFontSize=\scriptstyle,showorigin=false, labelsep=0.1pt,arrowsize=2pt 2}
\begin{pspicture}(-4.5,-1)(5.5,8.5)
\multido{\n=-4.2+0.2}{48}{\psline[linewidth=0.25pt,linecolor=lightgray](\n,-0.8)(\n,7.8)}
\multido{\n=-0.8+0.2}{44}{\psline[linewidth=0.25pt,linecolor=lightgray](-4.4,\n)(5.4,\n)}
\multido{\n=-4+1}{10}{\psline[linewidth=0.45pt](\n,-0.8)(\n,7.8)}
\multido{\n=0+1}{8}{\psline[linewidth=0.45pt](-4.4,\n)(5.4,\n)}
\psaxes[linewidth=0.95pt]{->}(0,0)(-4.4,-0.8)(5.4,8)\uput[dl](-0.1,-0.1){O}
%\pscurve[linecolor=blue,linewidth=0.7pt](-3,0)(-2.8,3)(-2.6,5)(-2.5,6)(-2,7.4)(-1,5.4)(-0.4,4)(0,3)(0.6,2)
\psplot[plotpoints=2000,linewidth=1.25pt,linecolor=red]{-3}{5}{x 3 add 2.71828 x exp div}
\psline[linecolor=violet, linewidth=0.8pt,]{<->}(-2.4,7.4)(-1.6,7.4)
\psline[linecolor=violet, linewidth=0.8pt,]{<->}(-1,5)(1,1)
\uput[ur](-4,6){\red $y=f(x)$}\uput[dl](-1,5){T}
\end{pspicture}
\end{center}
Parmi les propositions suivantes, laquelle n’est pas juste ?

\begin{tabularx}{\linewidth}{*{4}{X}}
\textbf{a.~~} $f'(-2)=0 $ &\textbf{b.~~} $f'(3)=-2 $&\textbf{c.~~}$f(0)=3 $& \textbf{d.~~} $f'(0)=-2 $.
\end{tabularx}


\vspace{1cm}

