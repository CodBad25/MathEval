
\subsection*{1.}

\paragraph{a.} Augmenter de 4 \% revient à multiplier par \(1 + \dfrac{4}{100} = 1 + 0{,}04 = 1{,}04\).

On a donc :
\[
u_1 = u_0 \times 1{,}04 = 180 \times 1{,}04 = 187{,}2.
\]
Le nombre de spectateurs en 2019 devrait être égal à 187 200.

\paragraph{b.} Le nombre de spectateurs est celui de l'année d'avant multiplié par \(1{,}04\). On a donc, pour tout naturel \(n\), \(u_{n+1} = u_n \times 1{,}04\), ce qui montre que la suite \((u_n)\) est géométrique de premier terme \(u_0 = 180\) et de raison \(q = 1{,}04\).

\paragraph{c.} On sait qu'alors \(u_n = u_0 \times 1{,}04^n = 180 \times 1{,}04^n\), quel que soit \(n \in \mathbb{N}\).

\subsection*{2.}

\paragraph{a.} On a donc pour tout naturel \( n \), \( v_{n+1} = v_n - 10 \), ce qui montre que la suite \( (v_n) \) est une suite arithmétique de premier terme \( v_0 = 260 \) et de raison \(r = -10\).


\paragraph{b.} La fonction \texttt{cinema()} renvoie \( n = 5 \).

\[
\begin{array}{|c|c|c|}
\hline
n & u_n & v_n \\
\hline
0 & 180 & 260 \\
1 & 187{,}2 & 250 \\
2 & 194{,}7 & 240 \\
3 & 202{,}5 & 230 \\
4 & 210{,}6 & 220 \\
5 & 219 & 210 \\
\hline
\end{array}
\]

Ceci montre que la 5\textsuperscript{e} année, soit en 2023, le nombre de spectateurs du nouveau complexe sera supérieur à celui de l'ancien cinéma.


