
\subsection*{Question 1}

Pour l'équation \(2x^2 - 9x + 4 = 0\), on a :
\[
\Delta = 81 - 32 = 49 = 7^2 > 0.
\]
L'équation a donc deux solutions :
\[
x_1 = \dfrac{9 + 7}{4} = 4 \quad \text{et} \quad x_2 = \dfrac{9 - 7}{4} = \dfrac{1}{2}.
\]
On sait que le trinôme est du signe de \(a = 2\), donc positif sauf entre les racines.

L'inéquation \(2x^2 - 9x + 4 \geqslant 0\) a donc pour ensemble de solutions :
\[
S = \left] -\infty\,;\,\dfrac{1}{2} \right] \cup [4\,;\,+\infty[.
\]

\subsection*{Question 2}

Écriture canonique :
\[
-x^2 + 4x = -(x^2 - 4x) = -[(x - 2)^2 - 4] = -(x - 2)^2 + 4.
\]
On voit que pour \(x = 2\), la fonction a pour maximum \(4\).

\subsection*{Question 3}

Soit \(\Delta\) la droite ; alors :
\[
M(x\,;\,y) \in \Delta \Longleftrightarrow \overrightarrow{AM} \cdot \vec{n} = 0 \Longleftrightarrow 2(x - 0) - 5(y - (-7)) = 0 \Longleftrightarrow 2x - 5y - 35 = 0.
\]

\subsection*{Question 4}

On peut écrire :
\[
x^2 - 4x + y^2 + 6y = 12 \Longleftrightarrow (x - 2)^2 - 4 + (y + 3)^2 - 9 = 12 \Longleftrightarrow (x - 2)^2 + (y + 3)^2 = 25.
\]
On reconnaît l'équation du cercle de centre \((2\,;\,-3)\) et de rayon \(5\).

\subsection*{Question 5}

Si \(x = -1\), alors \(y = 1\) : \(C(-1\,;\,1) \in d\).

Si \(x = -4\), alors \(y = 3\), \(D(-4\,;\,3) \in d\).

Donc \(\overrightarrow{CD} \begin{pmatrix} -3 \\ 2 \end{pmatrix}\) est un vecteur directeur de \(d\).

Or \(\overrightarrow{AB} \begin{pmatrix} 4 \\ 6 \end{pmatrix}\) et \(\overrightarrow{AB} \cdot \overrightarrow{CD} = -12 + 12 = 0\) : les vecteurs sont orthogonaux, donc la droite \(d\) est perpendiculaire à la droite \((AB)\).

