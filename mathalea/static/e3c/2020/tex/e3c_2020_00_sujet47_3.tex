
\medskip

Un artisan fabrique de la confiture qu'il vend à un grossiste. Le coût, en euros, de fabrication
de $x$ kilos de confiture est :

\[C(x) = 0,1x^2 + 0,7x + 100, \hspace{5em}\text{pour } x \in [0~;~160].\]

\smallskip

\begin{enumerate}
\item Chaque kilo est vendu $14$~\euro. Exprimer la recette $R$ en fonction de $x$.
\item Soit $B$ la fonction représentant le bénéfice de l'artisan, définie sur [0~;~160].

$B$ a pour expression

\[B(x) =- 0,1x^2 + 13,3x - 100.\]

Étudier le signe de $B(x)$. En déduire l'intervalle dans lequel doit se trouver le nombre de
kilos de confiture à vendre pour que l'artisan réalise un bénéfice positif.
\item On note $B'$ la fonction dérivée de la fonction $B$.
	\begin{enumerate}
		\item Déterminer $B'(x)$.
		\item Dresser le tableau de variation de $B$ sur l'intervalle [0~;~160].
		\item Donner le nombre de kilos à vendre pour que le bénéfice soit maximal ainsi que son
montant.
	\end{enumerate}
\end{enumerate}

\vspace{0,5cm}

