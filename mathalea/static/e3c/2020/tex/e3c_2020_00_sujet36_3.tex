
\medskip

En 1995, le taux de scolarisation des jeunes de 18 ans atteignait 84,8\,\%, du fait d'une forte
progression de la poursuite d'études dans le second cycle général et technologique jusqu'au
baccalauréat.

Une étude de l'INSEE montre que ce taux de scolarisation a régulièrement diminué au cours
des dix années suivantes.

On considère que la diminution du taux de scolarisation à 18 ans est chaque année de 1\,\% à
partir de 1995.

Pour tout entier naturel $n$, on modélise le taux de scolarisation des jeunes de 18 ans en
1995 +$ n$, par une suite $\left(u_n\right)$ ; ainsi $u_0 = 84,8$.

\medskip

\begin{enumerate}
\item Quel est le taux de scolarisation des jeunes âgés de 18 ans en 1996 ?
\item Déterminer, en justifiant, la nature de la suite $\left(u_n\right)$.
\item On donne le programme suivant en langage Python :
\begin{python}
U=84.8
n=0
while U > 80:
	U=0.99*U
	n=n+1
\end{python}

Déterminer la valeur numérique que contient la variable $n$ à l'issue de l'exécution du
programme. Interpréter cette valeur dans le contexte de l'énoncé.
\item Exprimer, pour tout entier naturel $n$, $u_n$ en fonction de $n$.
\item Quel est le taux de scolarisation des jeunes de 18 ans en 2005 ?
\end{enumerate}

\vspace{0,5cm}

