
\bigskip

\begin{minipage}[]{8.6cm}
Soit $f$ la fonction définie sur $\R$ par : 

\[f(x) = (5 - 2x)\e^x.\]

On note $\mathcal{C}$ la courbe représentative de $f$. Sur la figure ci-contre, on a tracé
la courbe $\mathcal{C}$ dans un repère orthogonal où les unités ont été effacées.
A est le point d'intersection de $\mathcal{C}$ avec l'axe des ordonnées et B le point
d'intersection de $\mathcal{C}$ avec l'axe des abscisses.

D est le point de $\mathcal{C}$ dont l'ordonnée est le maximum de la fonction $f$ sur $\R$.
\end{minipage}
\hspace{1.5em}
\begin{minipage}[]{6cm}
\psset{xunit=0.7cm,yunit=0.3cm,labelFontSize=\scriptstyle,showorigin=false}
\begin{pspicture*}(-3,-3)(3.4,12)
%\psgrid
\psaxes[linewidth=0.95pt,labels=none,ticks=none]{->}(0,0)(-3,-3)(3.3,11.5)
\def\Func{2.71828 x exp 5 2 x mul sub mul }
\psplot[plotpoints=2000,linewidth=0.85pt,linecolor=red]{-3}{2.6}{\Func}
\psdots[dotstyle=Mul,dotscale=1.9,linecolor=blue](0,5)(2.5,0)(1.5,8.9634)
\uput[ul](0,5){A}\uput[u](1.5,8.9634){D}\uput[ur](2.5,0){B}
\psplot[plotpoints=2000,linewidth=1pt,linecolor=blue]{-3}{2.7}{x 3 mul 5 add}
\end{pspicture*}
\end{minipage}

\medskip

\begin{enumerate}
\item Calculer les coordonnées des points A et B.
\item Soit $f'$ la fonction dérivée de $f$ sur $\R$. Montrer que, pour tout réel $x$,

\[f'(x) = (3 - 2x)\e^x .\]

\item Étudier le sens de variation de la fonction $f$.
\item En déduire que le point D admet comme coordonnées $\left(1,5~;~2\e^{1,5}\right)$.
\item Déterminer une équation de la tangente à la courbe $\mathcal{C}$ au point A, puis vérifier, à l'aide de l'équation obtenue, que le point D n'appartient pas à cette tangente.
\end{enumerate}

\vspace{0,5cm}

