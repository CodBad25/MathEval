  
\medskip

Le centre commercial « L'autre faubourg » de Cholet a été conçu en forme circulaire de 110 m de rayon permettant une visibilité à 360\textdegree{} et une accessibilité optimale, notamment aux personnes à mobilité réduite.

Le parking, situé à l'intérieur du disque, dessert l'ensemble des 32 magasins.

On munit le plan d'un repère orthonormé de centre O.

L'unité est le mètre.

Les entrées des magasins du centre commercial sont situées sur le cercle $\mathcal{C}$ de
centre O et de rayon 110.

\begin{center}
\psset{xunit=0.04cm,yunit=0.04cm,labelFontSize=\scriptstyle}
\begin{pspicture}(-130,-130)(130,130)
\psaxes[linewidth=1.25pt,Dx=10,Dy=10,labels=none]{->}(0,0)(-120,-120)(120,120)
\psset{unit=0.04cm}
\pscircle[linewidth=1.25pt,linecolor=blue](0,0){110}
\uput[dl](0,0){0}\uput[d](10,0){10}
\uput[l](0,10){10}\uput[d](115,0){$x$}
\uput[l](0,115){$y$}\uput[u](-100,-90){\blue \large\boldmath{$\mathcal{C}$}}
\end{pspicture}
\end{center}

\begin{enumerate}
\item  Une allée centrale couverte a été construite afin de permettre aux automobilistes de rejoindre les magasins en cas d'intempéries. Elle est modélisée par la droite (AD) avec A$(-30~;~15)$ et D$(80~;~- 40)$.
	\begin{enumerate}
		\item Déterminer une équation du cercle $\mathcal{C}$.
		\item Démontrer que le point O appartient à la droite (AD).
	\end{enumerate}
\item Camille qui vient de garer sa voiture en $G(-10~;~-10)$ sous une pluie battante, souhaite se mettre à l'abri sous cette allée centrale, le plus rapidement possible.
	\begin{enumerate}
		\item  Calculer le produit scalaire $\vv{AG}\cdot\vv{AO}$ .
		\item Le point de la droite (AD) le plus proche de G est-il O ?
	\end{enumerate}
\end{enumerate}







