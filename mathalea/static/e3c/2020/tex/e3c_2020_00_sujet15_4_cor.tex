	

	\subsection*{1.}
	
	Pour \(x = 3\), l’entreprise reçoit \(R(3) = 3 \times 680 = 2040\) euros.
	
	Le coût de production de ces 3 kilomètres de tissu est :
	\[ C(3) = 15 \times 3^3 - 120 \times 3^2 + 500 \times 3 + 750 = 405 - 1080 + 1500 + 750 = 1575. \]
	
	Il y a donc un bénéfice de \(R(3) - C(3) = 2040 - 1575 = 465\) euros.
	
	\subsection*{2.}
	
	Pour \(0 \leq x \leq 10\), on a :
	\[ B(x) = R(x) - C(x) = 680x - (15x^3 - 120x^2 + 500x + 750) = 680x - 15x^3 + 120x^2 - 500x - 750 = -15x^3 + 120x^2 + 180x - 750. \]
	
	\subsection*{3.}
	
	La fonction polynôme \(B\) est dérivable sur \(\mathbb{R}\), donc en particulier sur \([0 ; 10]\) et sur cet intervalle, on a :
	\[ B'(x) = -45x^2 + 240x + 180 = 15(-3x^2 + 16x + 12). \]
	
	\subsection*{4.}
	
	D’après le résultat précédent, le signe de \(B'(x)\) est celui du facteur \(-3x^2 + 16x + 12\).
	
	Or pour ce trinôme : \(\Delta = 16^2 - 4 \times (-3) \times 12 = 256 + 144 = 400 = 20^2 > 0\), donc ce trinôme a deux racines :
	\[ x_1 = \dfrac{-16 + 20}{-6} = -\dfrac{2}{3} \quad \text{et} \quad x_2 = \dfrac{-16 - 20}{-6} = 6. \]
	
	On sait que ce trinôme est négatif (signe de \(-3\)), sauf entre les racines.
	
	\subsection*{5.}
	
	On a \(x_1 \approx -0,67\) et \(x_2 = 6\). Comme \(B\) est croissante sur \(\left[ -\dfrac{2}{3} ; 6 \right]\), la plus grande valeur de \(B\) est obtenue pour :
	\[ B(6) = -15 \times 6^3 + 120 \times 6^2 + 180 \times 6 - 750 = 1410 \] 
	Soit 1410 \euro\\
	Elle doit produire 6 km de tissu.
	
