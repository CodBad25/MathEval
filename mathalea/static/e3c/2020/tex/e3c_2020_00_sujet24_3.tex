
\medskip

Laura reçoit chaque jour beaucoup de courriels. Pour se protéger des courriels indésirables, elle achète un logiciel anti-spam.

Chaque jour, 

\begin{itemize}
\item 35\,\% des courriels reçus par Laura sont indésirables ;
\item 95 \,\% des courriels indésirables sont automatiquement bloqués par le logiciel anti-spam.
\item Parmi les courriels qui ne sont pas indésirables, le logiciel anti-spam en bloque 2\,\%.
\end{itemize}
On choisit au hasard un courriel reçu par Laura.
Chaque courriel a la même probabilité d’être choisi.

On considère les évènements suivants :
\begin{itemize}
\item $I$ : \og le courriel choisi est indésirable \fg,
\item $S$ : \og le logiciel anti-spam bloque le courriel choisi \fg.
\end{itemize}

Pour tout évènement $A$, on note $\overline{A}$ l’évènement contraire de l’évènement A.

Pour tout évènement $A$ et $B$ avec $B$ un évènement de probabilité non nulle, la probabilité de $A$ sachant $B$ est notée $ p_{B}(A)$.

\medskip

\begin{enumerate}
\item Recopier et compléter sur la copie l’arbre de probabilité traduisant la situation.

\begin{center}
\psset{nodesepA=0pt,nodesepB=3pt,treesep=0.75,labelsep=0.1pt,levelsep=2.75cm}
\pstree[treemode=R]{\TR{}}
{\pstree{\TR{$I$~}\taput{$\np{0.35}$}}
	{
	\TR{$S$}\taput{$\np{0.95}$}
	\TR{$\overline{S}$}\tbput{$\dots$}
	}
\pstree{\TR{$\overline{I}$~}\tbput{$\dots$}}
	{\TR{$S$}\taput{$\dots$}
	\TR{$\overline{S}$}\tbput{$\dots$}
	}
}
\end{center}
\item Calculer la probabilité que le courriel reçu par Laura ne soit pas indésirable et soit bloqué par le logiciel anti-spam.
\item Montrer que $p(S)= \np{0,3455}$.
\item Le logiciel anti-spam a bloqué un courriel reçu par Laura. Calculer la probabilité que ce courriel soit indésirable. On donnera le résultat arrondi à $10^{-3}$.
\item Le fournisseur du logiciel anti-spam affirme que son logiciel se trompe dans moins de 2\,\% des cas. Est-ce vrai ? Justifier votre réponse.
\end{enumerate}

\vspace{0.5cm}

