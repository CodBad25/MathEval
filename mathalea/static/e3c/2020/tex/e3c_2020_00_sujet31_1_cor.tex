
\subsection*{Question 1}

Le trinôme a deux racines, donc \(\Delta > 0\) ce qui élimine \textbf{b.} et \textbf{c.}

La fonction est décroissante puis croissante, donc \(a > 0\) : réponse \textbf{a.}

\subsection*{Question 2}

2, 3, 5, 7, 11, 13, 17, 19, 23, 29 sont les 10 naturels premiers entre 1 et 30.
  
On a donc \(p(X = 2) = \dfrac{10}{30} = \dfrac{1}{3}\) et donc \(p(X = -1) = \dfrac{2}{3}\).

D'où : \(E(X) = 2 \times \dfrac{1}{3} - 1 \times \dfrac{2}{3} = \dfrac{2}{3} - \dfrac{2}{3} = 0.\)

\subsection*{Question 3}

\(\dfrac{e^6 \times e^3}{e^2} = e^{6+3-2} = e^7.\)

\subsection*{Question 4}

On sait que pour tout naturel \(n \geqslant 1\), \(u_n = 2 - 5(n - 1) = 7 - 5n\).

\subsection*{Question 5}

La droite d'équation \(-4x + 8y = 0\) a pour vecteur directeur \(\vec{d} \begin{pmatrix} -8 \\ -4 \end{pmatrix}\) et ce vecteur est orthogonal à \(\vec{u}\).

