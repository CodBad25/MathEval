
\medskip

Cet exercice est un questionnaire à choix multiple (QCM) comportant cinq questions.

Pour chacune des questions, une seule des quatre réponses proposées est correcte.
Les questions sont indépendantes.

Pour chaque question, indiquer le numéro de la question et recopier sur la copie la lettre correspondante à la réponse choisie.

Aucune justification n'est demandée mais il peut être nécessaire d'effectuer des recherches au brouillon pour aider à déterminer la réponse.

Chaque réponse correcte rapporte $1$ point. Une réponse incorrecte ou une question sans réponse n'apporte ni ne retire de point.

\bigskip

\textbf{Question 1}

%\medskip
%
%\begin{center}
%\begin{tabularx}{\linewidth}{|*{4}{X|}}\hline
%\textbf{a.~~}$0,2$&\textbf{b.~~}$0,45$&\textbf{c.~~}$0,15$&\textbf{d.~~}$0,95$\\ \hline
%\end{tabularx}
%\end{center}

\medskip

\parbox{0.55\linewidth}{Soit $a$,\, $b$ et $c$ trois réels tels que $a \ne 0$ et soit $g$ la fonction définie sur $\R$ par:

\[g(x)= ax^2 + bx+ c.\]

Soit $\Delta$ son discriminant.

La représentation graphique de la fonction $g$ dans un repère orthonormé est donnée ci-contre.

Alors on peut affirmer que: 

\textbf{a.}  $a>0$ et $\Delta > 0$

\textbf{b.} $a > 0$ et $\Delta < 0$ 

\textbf{c.} $a<0$ et $\Delta > 0$ 

\textbf{d.} $a<0$ et $\Delta < 0$}\hfill
\parbox{0.45\linewidth}{\psset{unit=0.8cm}
\begin{pspicture}(-1,-4)(6,5)
\psgrid[gridlabels=0pt,subgriddiv=1,griddots=10]
\psaxes[linewidth=1.25pt]{->}(0,0)(-1,-4)(6,5)
\psplot[plotpoints=2000,linewidth=1.25pt,linecolor=red]{0.25}{5.75}{4 x 3 sub dup mul sub}
\uput[ul](2,3){\red $\mathcal{C}_g$}
\end{pspicture}}

\bigskip

\textbf{Question 2}

\medskip

On considère la fonction $f$ dont la fonction dérivée est la fonction $g$ considérée dans la question~1.

Le tableau des variations de $f$ est :

\parbox{0.48\linewidth}{\textbf{a.~~}

\psset{unit=1cm}
\begin{pspicture}(6,2)
\psframe(6,2)\psline(2,0)(2,2) \psline(0,1.5)(6,1.5)
\uput[u](1,1.4){$x$}\uput[u](2.3,1.4){$- \infty$}\uput[u](4,1.4){$3$}\uput[u](5.6,1.4){$+\infty$}
\rput(1,1){Variations}\rput(1,0.5){de $f$}
\psline{->}(2.5,0.4)(3.5,1)\psline{->}(4.5,1)(5.5,0.4)
\end{pspicture}}\hfill
\parbox{0.48\linewidth}{\textbf{b.~~}

\psset{unit=1cm}
\begin{pspicture}(6,2)
\psframe(6,2)\psline(2,0)(2,2) \psline(0,1.5)(6,1.5)
\uput[u](1,1.4){$x$}\uput[u](2.3,1.4){$- \infty$}\uput[u](4,1.4){$3$}\uput[u](5.6,1.4){$+\infty$}
\rput(1,1){Variations}\rput(1,0.5){de $f$}
\psline{->}(2.5,1)(3.5,0.4)\psline{->}(4.5,0.4)(5.5,1)
\end{pspicture}}

\textbf{c.~~}

\psset{unit=1cm}
\begin{pspicture}(8,2)
\psframe(8,2)\psline(2,0)(2,2) \psline(0,1.5)(8,1.5)
\uput[u](1,1.4){$x$}\uput[u](2.3,1.4){$- \infty$}\uput[u](4,1.4){$1$}\uput[u](6,1.4){$5$}\uput[u](7.6,1.4){$+\infty$}
\rput(1,1){Variations}\rput(1,0.5){de $f$}
\psline{->}(2.5,1)(3.5,0.4)\psline{->}(4.5,0.4)(5.5,1)\psline{->}(6.5,1)(7.5,0.4)
\end{pspicture}

\textbf{d.~~}

\psset{unit=1cm}
\begin{pspicture}(8,2)
\psframe(8,2)\psline(2,0)(2,2) \psline(0,1.5)(8,1.5)
\uput[u](1,1.4){$x$}\uput[u](2.3,1.4){$- \infty$}\uput[u](4,1.4){$1$}\uput[u](6,1.4){$5$}\uput[u](7.6,1.4){$+\infty$}
\rput(1,1){Variations}\rput(1,0.5){de $f$}
\psline{->}(2.5,0.4)(3.5,1)\psline{->}(4.5,1)(5.5,0.4)\psline{->}(6.5,0.4)(7.5,1)
\end{pspicture}

\bigskip

\textbf{Question 3}

\medskip

On considère à nouveau la fonction $f$ dont la fonction dérivée est la fonction $g$ considérée dans la question 1. On sait de plus que $f(3) = 7$.

La tangente à la courbe représentative de $f$ au point d'abscisse 3 a pour équation réduite:

\begin{center}
\begin{tabularx}{\linewidth}{|*{4}{X|}}\hline
\textbf{a.~~}$-2x + 3y + 11 = 0$&\textbf{b.~~}$3x - 2y - 9 = 0$&\textbf{c.~~}$x - 3y - 10 = 0$&\textbf{d.~~}$3x +2y +3 = 0$\\ \hline
\end{tabularx}
\end{center} 

\bigskip

\textbf{Question 4}

\medskip

Dans un repère orthonormé, on considère les points A$(5~;~-1)$, B(3~;~2) et C$(1~;~-3)$.

Une équation cartésienne de la droite perpendiculaire à (AB) et passant par C est :
 
\begin{center}
\begin{tabularx}{\linewidth}{|*{4}{X|}}\hline
\textbf{a.~~}$-2x + 3y + 11 = 0$&\textbf{b.~~}$3x - 2y - 9 = 0$&\textbf{c.~~}$x - 3y - 10 = 0$&\textbf{d.~~}$3x +2y +3 = 0$\\ \hline
\end{tabularx}
\end{center} 
Le vecteur $\vect{\text{AB}}\binom{-2}{3}$ est normal à tout vecteur directeur de la droite perpendiculaire. Une équation de cette droite est donc :

$3x + 2y + c = 0$ et comme le couple (1~;~-3) vérifie cette équation on a $3 - 6 = c = 0 \iff c  = 3$.

On a finalement $3x + 2y + 3 = 0$. 
\bigskip

\textbf{Question 5}

\medskip

Dans un repère orthonormé, on considère les points A$(5~;~-1)$, B(3~;~2) et C$(1~;~-3)$.

Une mesure, arrondie au degré, de l'angle $\widehat{\text{ABC}}$, est:

\begin{center}
\begin{tabularx}{\linewidth}{|*{4}{X|}}\hline
\textbf{a.~~}11&\textbf{b.~~}25&\textbf{c.~~}55&\textbf{d.~~}88\\ \hline
\end{tabularx}
\end{center} 

\bigskip

