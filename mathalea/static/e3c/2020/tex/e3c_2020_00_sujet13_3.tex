
\medskip
Un cafetier propose à ses clients des cookies au chocolat ou aux noisettes en s'approvisionnant dans trois boulangeries. Un client prend un cookie au hasard.

On note:

$C$ l'évènement \og le cookie est au chocolat \fg,

$N$ l'évènement \og le cookie est aux noisettes \fg,

$B_1$ l'évènement \og le cookie provient de la boulangerie 1 \fg, 

$B_2$ l'évènement \og le cookie provient de la boulangerie 2 \fg,

$B_3$ l'évènement \og le cookie provient de la boulangerie 3 \fg.

On suppose que :

\setlength\parindent{1cm}
\begin{itemize}
\item la probabilité que le cookie provienne de la boulangerie 1 est de 0,49 ;
\item la probabilité que le cookie provienne de la boulangerie 2 est de 0,36 ;
\item  $P_{B_2}(C) = 0,4$ où $P_{B_2}(C)$ est la probabilité conditionnelle de $C$ sachant $B_2$ ;
\item la probabilité que le cookie soit aux noisettes sachant qu'il provient de la troisième boulangerie est de $0,3$.
\end{itemize}
\setlength\parindent{0cm}

L'arbre pondéré ci-dessous correspond à la situation et donne une information
supplémentaire : le nombre $0,6$ sur la branche de $B_1$ à $C$.

\begin{center}
\pstree[treemode=R,nodesepA=0pt,nodesepB=3pt]{\TR{}}
{\pstree{\TR{$B_1$~~}\naput{\ldots}}
	{\TR{$C$} \naput{0,6}
	\TR{$N$}\tbput{\ldots}
	}
\pstree{\TR{$B_2$~~}\naput{\ldots}}
	{\TR{$C$} \taput{\ldots}
	\TR{$N$}\tbput{\ldots}
	}
\pstree{\TR{$B_3$~~}\nbput{\ldots}}
	{\TR{$C$}\taput{\ldots} 
	\TR{$N$}\tbput{\ldots}
	}	
}
\end{center}

\smallskip

\begin{enumerate}
\item Exprimer par une phrase l'information donnée par le nombre 0,6 sur la branche de $B_1$ à $C$.
\item Recopier et compléter sur la copie l'arbre pondéré ci-dessus.
\item Définir par une phrase l'évènement $B_1 \cap C$ et calculer sa probabilité.
\item Montrer la probabilité $P(C)$ d'avoir un cookie au chocolat est égale à $0,543$.
\item Calculer la probabilité d'avoir un cookie provenant de la boulangerie 2 sachant qu'il est au chocolat. On donnera le résultat arrondi au millième.
\end{enumerate}

\bigskip

