	\section*{Exercice 4 (5 points)}
	
	\subsection*{1. Exprimer le rayon de la base en fonction de $h$.}
	
	Le théorème de Pythagore appliqué au triangle rectangle d’hypoténuse la génératrice s’écrit :
	\[
	20^2 = h^2 + r^2 \iff r^2 = 400 - h^2 \iff r = \sqrt{400 - h^2}
	\]
	
	\subsection*{2. Démontrer que le volume du cône, en fonction de sa hauteur $h$, est : $V(h) = \dfrac{\pi}{3}(400h - h^3)$.}
	
	On a donc :
	\[
	V = \dfrac{1}{3} \pi (400 - h^2)h = \dfrac{\pi}{3}(400h - h^3).
	\]
	
	\subsection*{3. Quelle hauteur $h$ choisir pour que le volume du cône soit maximum ?}
	
	La fonction polynôme $V$ est dérivable et
	\[
	V'(h) = \dfrac{\pi}{3}(400 - 3h^2).
	\]
	
	Le signe de $V'(h)$ est celui de $400 - 3h^2$.
	\begin{itemize}
		\item $400 - 3h^2 > 0 \iff 400 > 3h^2 \iff h^2 < \dfrac{400}{3} \iff h < \sqrt{\dfrac{400}{3}}$.
		\item $400 - 3h^2 < 0 \iff 400 < 3h^2 \iff h^2 > \dfrac{400}{3} \iff h > \sqrt{\dfrac{400}{3}}$.
		\item $400 - 3h^2 = 0 \iff 400 = 3h^2 \iff h^2 = \dfrac{400}{3} \iff h = \sqrt{\dfrac{400}{3}}$.
	\end{itemize}
	
	Du signe de la dérivée résultent les variations de $V$ :
	\begin{itemize}
		\item Sur $[0; \sqrt{\dfrac{400}{3}}]$, la fonction est croissante.
		\item Sur $[\sqrt{\dfrac{400}{3}}; +\infty]$, la fonction est décroissante.
	\end{itemize}
	
	\[
	V\left(\sqrt{\dfrac{400}{3}}\right) = \dfrac{\pi}{3}\left(400 \times \sqrt{\dfrac{400}{3}} - \left(\sqrt{\dfrac{400}{3}}\right)^3\right) \approx \numprint{3224.53} \text{ cm}^3
	\]
	
	et ce pour une hauteur optimale de $\sqrt{\dfrac{400}{3}} \approx \numprint{11.55}$ cm.
	
