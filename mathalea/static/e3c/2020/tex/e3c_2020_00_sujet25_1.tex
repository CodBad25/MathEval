
\medskip

Ce QCM comprend 5 questions. 

Pour chacune des questions, une seule des quatre réponses proposées est correcte. 

Les questions sont indépendantes.

Pour chaque question, indiquer le numéro de la question et recopier sur la copie la lettre correspondante à la réponse choisie.

Aucune justification n’est demandée mais il peut être nécessaire d’effectuer des recherches au brouillon pour aider à déterminer votre réponse.

Chaque réponse correcte rapporte 1 point. Une réponse incorrecte ou une question sans réponse n’apporte ni ne retire de point.

\medskip
\textbf{Question 1} :
\medskip 

Dans le plan muni d’un repère orthonormé, l’ensemble des points M de coordonnées $(x~;~y)$ vérifiant : $(x + 1)^2+(y - 1)^2 = 9$ est :

\begin{center}
\begin{tabularx}{\linewidth}{*{4}{X}}
\textbf{a.~~} un cercle  &\textbf{b.~~} une droite &\textbf{c.~~}une parabole & \textbf{d.~~}l’ensemble vide .
\end{tabularx}
\end{center}

\medskip
\textbf{Question 2} :
\medskip 

Combien y-a-t-il de fonctions polynômes du second degré qui s’annulent en 1 et en 3 ?

\begin{center}
\begin{tabularx}{\linewidth}{*{4}{X}}
\textbf{a.~~} $0$ &\textbf{b.~~} $ 1 $ seule&\textbf{c.~~}$2$& \textbf{d.~~} une infinité.
\end{tabularx}
\end{center}

\medskip
\textbf{Question 3} :
\medskip 

Une fonction polynôme du second degré :

\begin{center}
\begin{tabularx}{\linewidth}{*{4}{X}}
\textbf{a.~~} est nécessairement de signe constant sur $\R$ &\textbf{b.~~}n’est jamais de signe constant sur $\R$ $ $&\textbf{c.~~}est nécessairement positive sur $\R$& \textbf{d.~~} peut être ou non de signe constant sur $\R$.
\end{tabularx}
\end{center}

\medskip
\textbf{Question 4} :
\medskip 

Pour tout réel $x$, $\e^{2x+1}=$

\begin{center}
\begin{tabularx}{\linewidth}{*{4}{X}}
\textbf{a.~~} $\e^{2x} + \e $ &\textbf{b.~~} $\e^{2x} \times \e $&\textbf{c.~~}$(\e^{x+1})^2 $& \textbf{d.~~} $(2x+1)\times \e $.
\end{tabularx}
\end{center}

\medskip
\textbf{Question 5} :
\medskip 

Dans un repère orthonormé, la droite $d$ d’équation cartésienne $2x-5y-4=0$

\begin{center}
\begin{tabularx}{\linewidth}{*{4}{X}}
\textbf{a.~~}coupe l’axe des ordonnées au point de coordonnées $(0~;~-4)$ &\textbf{b.~~}passe par le point de coordonnées $(2~;~0,2)$&\textbf{c.~~}admet  $\vv{u}\ \binom{2}{-5}$ pour vecteur normal& \textbf{d.~~}admet $\vv{u}\ \binom{2}{-5}$   pour vecteur directeur.
\end{tabularx}
\end{center}


\vspace{1cm}

