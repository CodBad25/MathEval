
\medskip

Dans une usine, un four cuit des céramiques à la température de \np{1000}~\degres C.
À la fin de la cuisson, on éteint le four et commence alors la phase de refroidissement. Pour un nombre entier naturel $n$,on note $T_n$ la température en degré Celsius du four au bout de $n$ heures écoulées à partir de l'instant où il a été éteint. 

On a donc $T_0 = \np{1000}$.

La température $T_n$ est calculée grâce à l'algorithme suivant :

\begin{center}
\begin{tabular}{|l|}\hline
$T \gets \np{1000}$\\
Pour $i$ allant de 1 à $n$\\
\quad $T \gets  0,82\times T + 3,6$\\
Fin pour\\ \hline
\end{tabular}
\end{center}

\medskip

\begin{enumerate}
\item Quelle est la température du four après une heure de refroidissement ?
\item Exprimer $T_{n+1}$ en fonction de $T_n$ ;
\item Déterminer la température du four arrondie à l'unité après 4 heures de refroidissement.
\item La porte du four peut être ouverte sans risque pour les céramiques dès que sa
température est inférieure à 70~\degres C. Afin de déterminer le nombre d'heures au bout duquel le four peut être ouvert sans risque, on définit une fonction \og froid \fg{} en langage Python.

\begin{center}
\begin{tabular}{|c l|}\hline
1&\texttt{def froid() :~\quad~}\\
2&\quad\texttt{T= \np{1000}}\\
3&\quad\texttt{n=0 }\\
4&\quad\texttt{while \ldots}\\
5&\qquad \texttt{T= \ldots}\\
6 &\qquad \texttt{n=n+1} \\
7&\quad\texttt{return n}\\ \hline
\end{tabular}
\end{center}

Recopier et compléter les instructions 4 et 5.
\item Déterminer le nombre d'heures au bout duquel le four peut être ouvert sans risque pour les céramiques.
\end{enumerate}

\bigskip

