
\medskip
Soit $g$ la fonction définie sur l'intervalle $\interval{-5}{5}$ par:
\[
g(x)=\e^x-x+1.
\]
\begin{enumerate}
\item On admet que $g$ est dérivable sur l'intervalle $\interval{-5}{5}$ et on note $g'$ sa fonction dérivée.

Calculer $g'(x)$.

\item Étudier les variations de la fonction $g$ sur l'intervalle $\interval{-5}{5}$.
\item Démontrer que $g$ est strictement positive sur $\interval{-5}{5}$, c'est-à-dire que:
\[
\text{pour tout }x\in\interval{-5}{5},\ g(x)>0.
\]
\end{enumerate}
Soit $f$ la fonction définie sur $\interval{-5}{5}$ par:
\[
f(x)=x +1+\frac{x}{\e^x}\cdotp
\]
On appelle $\mathcal{C}_f$ sa courbe représentative dans un repère du plan.

On admet que $f$ est dérivable sur l'intervalle $\interval{-5}{5}$ et on note $f'$ sa fonction dérivée.
\begin{enumerate}[resume]
\item Démontrer que pour tout réel $x$ de $\interval{-5}{5}$,
\[
f'(x)=\frac{1}{\e^x}\times g(x).
\]
En déduire les variations de $f$ sur l'intervalle $\interval{-5}{5}$.
\item Déterminer une équation de la tangente à $\mathcal{C}_f$ au point d'abscisse 0.
\end{enumerate}
