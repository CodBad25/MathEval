
\medskip

Une entreprise fabrique $q$ milliers d'objets, $q \in [1~;~20]$. Le coût total de fabrication, exprimé en euros en fonction de $q$, est donné par l'expression :

\[C(q) = q^3 - 18q^2 +750q + 200.\]

\begin{enumerate}
\item 
	\begin{enumerate}
		\item Calculer le coût total de fabrication de \np{5000} objets.
		\item Déterminer le coût moyen de fabrication d'un millier d'objets lorsqu'on fabrique \np{5000} objets.
	\end{enumerate}
\item Le coût moyen $C_M(q)$ de fabrication de $q$ milliers d'objets, exprimé en euros, est donné par l'expression:

\[C_M(q) = \dfrac{C(q)}{q}   = q^2 - 18q + 750 + \dfrac{ 200}{q}.\]
 
	\begin{enumerate}
		\item On note $C'_M$ la fonction dérivée, sur l'intervalle [1~;~20], de la fonction $C_M$.
		
 Montrer que, pour tout $q \in [1~;~20]$,

\[C'_M(q) = \dfrac{2(q - 10)\left(q^2 + q + 10\right)}{q^2}\]

		\item Étudier le signe de $C'_M$ et dresser le tableau de variation de la fonction $C_M$ sur l'intervalle [1~;~20].
		\item Quel est le coût moyen minimal et pour quelle quantité d'objets est-il obtenu ?
	\end{enumerate}
\end{enumerate}

\bigskip

