
\subsection*{1.}

On a \(\dfrac{664}{800} = \dfrac{8 \times 83}{8 \times 100} = \dfrac{83}{100}\) ce qui représente une réduction de \(1 - \dfrac{83}{100} = \dfrac{17}{100}\).

\subsection*{2.}

On a vu que réduire de 17\%, c'est multiplier la taille de l'image par 0,83.  
On a donc :
\[
t_{n+1} = 0,83 t_n.
\]
La suite \((t_n)\) est donc une suite géométrique de premier terme \(t_0 = 800\) et de raison \(0,83\).

\subsection*{3.}

On sait que pour tout entier naturel \(n\), \(t_n = t_0 \times 0,83^n\) ou \(t_n = 800 \times 0,83^n\).

\subsection*{4.}

Il faut que \(A = 50\) : tant que \(t > A\), l'algorithme tourne.

\subsection*{5.}

Il faut réduire 15 fois la taille pour obtenir une image d'à peu près 48,9 ko.

