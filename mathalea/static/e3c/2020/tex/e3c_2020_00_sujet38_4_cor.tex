
\subsection*{1.}

\(f(x) = 8x - 2x^3\).

\paragraph{a.} \( f \) est une fonction polynôme dérivable sur \( \mathbb{R} \), donc sur \([0\,;\,2]\) et sur cet intervalle :
\[
f'(x) = 8 - 6x^2 = 2(4 - 3x^2).
\]
Comme \( 2 > 0 \), le signe de \( f'(x) \) est celui de \( 4 - 3x^2 \).

\paragraph{b.} \( 4 - 3x^2 \) est un trinôme dont les racines sont \( -\dfrac{2}{\sqrt{3}} \) et \( \dfrac{2}{\sqrt{3}} \).

Comme le coefficient \( a = -3 > 0 \), on sait que la fonction est croissante sauf sur l'intervalle \( \left] -\dfrac{2}{\sqrt{3}} ; \dfrac{2}{\sqrt{3}} \right[ \) où elle est décroissante.

\( f \) a donc un maximum local en \( x = -\dfrac{2}{\sqrt{3}} \), tel que :
\begin{align*}
f\left( -\dfrac{2}{\sqrt{3}} \right) &= 8 \times \left( -\dfrac{2}{\sqrt{3}} \right) - 2\left( -\dfrac{2}{\sqrt{3}} \right)^3 \\
&= -\dfrac{16}{\sqrt{3}} + \dfrac{16}{3\sqrt{3}} \\
&= -\dfrac{48}{3\sqrt{3}} + \dfrac{16}{3\sqrt{3}} \\
&= -\dfrac{32}{3\sqrt{3}} \approx -6{,}16.
\end{align*}

\( f \) a un minimum local en \( x = \dfrac{2}{\sqrt{3}} \), tel que :
\begin{align*}
f\left( \dfrac{2}{\sqrt{3}} \right) &= 8 \times \dfrac{2}{\sqrt{3}} - 2\left( \dfrac{2}{\sqrt{3}} \right)^3 \\
&= \dfrac{16}{\sqrt{3}} - \dfrac{16}{3\sqrt{3}} \\
&= \dfrac{48}{3\sqrt{3}} - \dfrac{16}{3\sqrt{3}} \\
&= \dfrac{32}{3\sqrt{3}} \approx 6{,}16.
\end{align*}

Comme on étudie la fonction \( f \) sur l'intervalle \([0\,;\,2]\), on a donc :

\( f \) est croissante sur \(\left[0\,;\,\dfrac{2}{\sqrt{3}}\right]\), puis décroissante sur \(\left[\dfrac{2}{\sqrt{3}}\,;\,2\right]\).

\subsection*{2.}

\begin{center}
\psset{xunit=1cm,yunit=1cm,labelFontSize=\scriptstyle,showorigin=false}
\begin{pspicture}(-4.2,-1.2)(4.2,7.2)
\psframe[fillstyle=solid, fillcolor= blue](-1,1)(1,4)
\multido{\n=-4+0.2}{41}{\psline[linewidth=0.25pt,linecolor=lightgray](\n,-1)(\n,7)} % Lignes verticales
\multido{\n=-0.8+0.2}{40}{\psline[linewidth=0.25pt,linecolor=lightgray](-4,\n)(4,\n)} % Lignes horizontales
\multido{\n=-4+1}{9}{\psline[linewidth=0.45pt](\n,-1)(\n,7)} % Lignes verticales plus épaisses
\multido{\n=0+1}{8}{\psline[linewidth=0.45pt](-4,\n)(4,\n)} % Lignes horizontales plus épaisses
\psaxes[linewidth=0.95pt]{-}(0,0)(-4,-0.99)(4,7)
\def\Func{x x  mul  }
\psplot[plotpoints=2000,linewidth=0.85pt,linecolor=red]{-2.646}{2.646}{\Func}
\def\FuncTwo{8 x mul 2 x 3 exp mul sub}
\psplot[plotpoints=2000,linewidth=0.85pt,linecolor=blue]{0}{2}{\FuncTwo}
\uput[dl](0,0){O}
\psdots[dotstyle=Bullet,dotscale=1.5,](1,1)(-1,1)(-1,4)(1,4)
\uput[dr](1,1){\footnotesize $M$}\uput[ur](1,4){\footnotesize $E$}\uput[ul](-1,4){\footnotesize $F$}\uput[dl](-1,1){\footnotesize $S$}
\psline[linewidth=1.25pt](-4,4)(4,4)\uput[u](3.8,4){$\mathscr{D}$}
\end{pspicture}
\end{center}

\paragraph{a.}

Pour \( x = 1 \) l'aire est égale à \( 2 \times (4 - 1) = 6 \) et pour \( x = 1{,}5 \) l'aire est égale à \( 3 \times (4 - 2{,}25) = 5{,}25 \). L'aire n'est donc pas constante.

\paragraph{b.}
On a \( M(x\,;\,x^2) \), \( S(-x\,;\,x^2) \), \( E(x\,;\,4) \) et \( F(-x\,;\,4) \).

Si \( x \in [0\,;\,2] \), alors \( SM = 2x \) et \( ME = 4 - x^2 \). 

Donc l'aire du rectangle \( MSFE \), \( \mathcal{A}(MSFE) \), est égale à :
\[
\mathcal{A}(MSFE) = 2x \times (4 - x^2) = 8x - 2x^3 = f(x).
\]

\paragraph{c.} On a vu que sur l'intervalle \( ]0\,;\,2[ \), \( f \) a un maximum égal à :
\[
f\left( \dfrac{2}{\sqrt{3}} \right) = \dfrac{32}{3\sqrt{3}} \approx 6{,}16.
\]

