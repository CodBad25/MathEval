	
	\subsection*{Question 1}
	Pour tout réel \(x\),
	\begin{align*}
		\dfrac{e^{2x}}{e^x+1} &= e^{2x - x - 1} \\
		&= e^{x - 1}.
	\end{align*}
	
	\subsection*{Question 2}
	Les points \(M(x\,;\,y)\) communs aux deux courbes ont des abscisses qui vérifient :
	\begin{align*}
		&15x^2 + 10x - 1 = 19x^2 - 22x + 10 \\
		\iff & 4x^2 - 32x + 11=0.
	\end{align*}
	Or pour cette équation du second degré :
	\[
	\Delta = 32^2 - 4 \times 4 \times 11 = 1024 - 176 = 848 > 0.
	\]
	Le discriminant est supérieur à zéro, donc cette équation a deux solutions distinctes ; les deux courbes ont deux points communs.
	
	\subsection*{Question 3}
	\begin{align*}
		&M(x\,;\,y) \in \mathcal{C}(A, R = 5)\\
		\iff	&AM^2 = 5^2 \\
		\iff&   (x - 3)^2 + (y - (-1))^2 = 25 \\
		\iff&   (x - 3)^2 + (y + 1)^2 = 25.
	\end{align*}
	
	\subsection*{Question 4}
	Un vecteur directeur de \((d)\) est par exemple \(\vec{d}\begin{pmatrix}-2 \\ 3\end{pmatrix}\), donc un vecteur normal est par exemple \(\vec{n}\begin{pmatrix}3 \\ 2\end{pmatrix}.\)
	
	\subsection*{Question 5}
	On sait que \(1 + 2 + 3 + \cdots + n = \dfrac{n(n + 1)}{2}.\)
	
	Il faut donc trouver le plus petit naturel tel que :
	\begin{align*}
		&\dfrac{n(n + 1)}{2} > 5000 \\
		\iff \quad & n(n + 1) > 10000 \\
		\iff \quad & n^2 + n - 10000 > 0.
	\end{align*}
	
	Pour le trinôme \(n^2 + n - 10000\), on a 
	\[
	\Delta = 1 + 40000 = 40001 > 0.
	\]
	Le trinôme a donc deux racines :
	\[
	n_1 = \dfrac{-1 + \sqrt{40001}}{2} \approx 99,5 \quad \text{et} \quad n_2 = \dfrac{-1 - \sqrt{40001}}{2} \approx -100,5.
	\]
	On sait que le trinôme est négatif sur l'intervalle \([n_2\,;\,n_1]\), donc le plus petit entier pour lequel le trinôme est positif est \(100\).
	
