
\medskip

La médiathèque d'une petite ville a ouvert ses portes début janvier 2013 et a enregistré \np{2 500}~inscriptions pour l'année 2013.

Elle estime que, chaque année, 80\,\% des anciens inscrits renouvellent leur inscription l'année suivante et qu'il y aura également $400$ nouveaux adhérents.

Pour tout entier naturel $n$, on peut donc modéliser le nombre d'inscrits à la médiathèque $n$ années après 2013 par une suite numérique $(a_n)$ définie par :
$ a_0=\np{2500}$ et $a_{n+1}=0,8a_n + 400$.

\medskip

\begin{enumerate}
\item  Calculer $a_1$ et $a_2$.
\item On pose, pour tout entier naturel $n$, $v_n=a_n- \np{2000}$.
	\begin{enumerate}
		\item Démontrer que $\left(v_n\right)$ est une suite géométrique de raison 0,8. Préciser son premier terme.
		\item Exprimer, pour tout entier naturel $n$, $v_n$ en fonction de $n$.
		\item En déduire que pour tout entier naturel $n$, $a_n=500\times 0,8^n + \np{2000}$.
		\item Déterminer le plus petit entier naturel $n$ tel que $a_n\leqslant \np{2010}$. Interpréter ce résultat dans le contexte de l'exercice. 
	\end{enumerate}
\end{enumerate}
