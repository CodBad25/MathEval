
\medskip

D'après l'ADEME (Agence De l'Environnement et de la Maîtrise de l'Énergie), chaque Français
a produit une masse moyenne de \np[kg]{365}  de déchets ménagers en 2018.

Un maire, étant informé que la masse moyenne de déchets ménagers dans sa commune en
2018 était de \np[kg]{400}  par habitant, décide d'une campagne annuelle de sensibilisation au
recyclage qui conduit à une réduction de cette production de 1,5\,\% par an, et cela dès
l'année 2019.

On modélise alors la masse moyenne de déchets ménagers par habitant calculée en fin
d'année dans cette commune par une suite $(d_n)$ où pour tout entier naturel $n$, $d_n$
correspond à la masse moyenne de déchets ménagers par habitant, en kg, pour l'année
2018+ $n$.

Ainsi, $d_0 = 400$.

\medskip

\begin{enumerate}
\item Prouver que $d_1 = 394$. Interpréter ce résultat.
\item 
	\begin{enumerate}
		\item  Déterminer la nature de la suite $(d_n)$. Préciser sa raison et son premier terme.
		\item Pour tout entier naturel $n$, exprimer $d_n$ en fonction de $n$.
	\end{enumerate}
\item 
	\begin{enumerate}
		\item  D'après le tableau de valeurs suivant, en quelle année la masse moyenne de déchets
ménagers par habitant deviendra-t-elle inférieure à \np[kg]{365} ?

\smallskip

\begin{tabular}[]{|l|*{9}{c|}}
\hline
$n$& 0& 1& 2& 3& 4& 5& 6 &7& 8\\
\hline
$d_n$& 400& 394& 388,09& 382,27& 376,53& 370,89& 365,32& 359,84& 354,45\\
\hline
\end{tabular}

		\item Écrire une fonction Python qui retourne l'année à laquelle la masse moyenne de déchets
ménagers par habitant de la commune devient inférieure à \np[kg]{365}.
	\end{enumerate}
\end{enumerate}
