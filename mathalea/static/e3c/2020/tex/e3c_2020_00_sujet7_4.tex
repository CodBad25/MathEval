
\medskip
 
Le plan est muni d'un repère orthonormé. La courbe représentative $\mathcal{C}$ d'une fonction $f$ définie sur $\R$ est donnée ci-dessous:

\begin{center}
\psset{unit=1cm}
\begin{pspicture*}(-3.2,-2.2)(4.2,6.2)
\psgrid[gridlabels=0pt,subgriddiv=1,gridwidth=0.4pt](-4,-3)(5,7)
\psaxes[linewidth=1.25pt]{->}(0,0)(-3.2,-2.2)(4.2,6.2)
\psplot[plotpoints=2000, linewidth=1.25pt, linecolor=blue]{-2.8}{3.8}{6 x add x dup mul sub 0.5 mul}
\psdots[dotstyle=x,dotscale=2](3,0)(-2,0)(-1,2)(0,3)(1,5)
\uput[ur](1,5){E}\uput[ur](1,3){\blue $\mathcal{C}$}
\end{pspicture*}
\end{center}

\smallskip

\begin{enumerate}
\item Par lecture graphique, résoudre l'équation $f(x) = 0$ d'inconnue $x$.
\item  On donne $f'(x) = -x + 0,5$ pour tout réel $x$.

Déterminer qu'une équation de la tangente $T$ à la courbe $\mathcal{C}$ au point d'abscisse $-1$ est $y = 1,5x + 3,5$.
\item  On considère le point E de coordonnées (1~;~5).

Dans cette question, on cherche à déterminer les points de la courbe $\mathcal{C}$ en
lesquels la tangente passe par le point E.
	\begin{enumerate}
		\item Montrer que le point E appartient à la tangente $T$.
		\item Déterminer l'autre point de la courbe en lequel la tangente passe par le
point E.
	\end{enumerate}
\end{enumerate}
