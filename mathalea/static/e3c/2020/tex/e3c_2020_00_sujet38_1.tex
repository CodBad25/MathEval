
\medskip

Ce QCM comprend 5 questions indépendantes. 

Pour chacune d'elles, une seule des réponses proposées est exacte.

Indiquer pour chaque question sur la copie la lettre correspondant à la réponse choisie. Aucune justification n'est demandée.

Chaque réponse correcte rapporte 1 point. Une réponse incorrecte ou une absence de réponse n'apporte ni ne retire de point.

\medskip

\begin{enumerate}
\item  L'inéquation $-3(x-2)(x+1)>0$ admet pour ensemble des solutions :

\medskip

\begin{tabularx}{1.05\linewidth}{*{4}{X}}
\textbf{a.~~} $[-1~;~2] $ &\textbf{b.~~}\small $]-\infty~;~-1[\cup[2~;~\infty[ $&\textbf{c.~~}$]-1~;~2[ $& \textbf{d.~~}\small $ ]-\infty~;~-1[\cup]2~;~+\infty[ $.
\end{tabularx}

\item Soit $x$ un nombre réel. Le réel $\cos(x+3\pi)$ est égal à :

\medskip

\begin{tabularx}{1.05\linewidth}{*{4}{X}}
\textbf{a.~~} $\cos(x) $ &\textbf{b.~~} $-\cos(x) $&\textbf{c.~~}$\sin(x) $& \textbf{d.~~} $-\sin (x)  $.
\end{tabularx}

\item  Dans un repère orthonormé, on considère la droite $d$ passant par le point $A(1~;~2)$ et dont un vecteur normal est le vecteur $\vv{v}\ \binom{2}{-3}$. 

Une équation de la droite $d$ est :

\medskip

\begin{tabularx}{1.05\linewidth}{*{4}{X}}
\textbf{a.~~} $2x+3y-8=0 $ &\textbf{b.~~} $ x+2y+4=0 $&\textbf{c.~~}$ 2x-3y-4=0$& \textbf{d.~~} $y=\frac{2}{3}x+\frac{4}{3}  $.
\end{tabularx}

\item On considère la fonction $f$ définie sur $[0~;~+\infty[$ par $f(x)=\dfrac{x^2}{x+1}$.

On note $\mathscr{C}$ sa courbe représentative sur $[0~;~+\infty[$.

Le coefficient directeur de la tangente à $\mathscr{C}$ au point d'abscisse 1 est :

\medskip

\begin{tabularx}{1.05\linewidth}{*{4}{X}}
\textbf{a.~~} $ \dfrac{1}{2}$ &\textbf{b.~~} $\dfrac{3}{4} $&\textbf{c.~~}$\dfrac{3}{2} $& \textbf{d.~~} $ 2 $.
\end{tabularx}

\item L'ensemble des points $M(x~;~y)$ dont les coordonnées vérifient l'équation $x^2-2x+y^2+4y=4$ est :

\medskip

\begin{tabularx}{1.05\linewidth}{*{2}{X}}
\textbf{a.~~}une droite $ $ &\textbf{b.~~} \small le cercle de centre $ A(1~;~-2)$ et de rayon 3\\\textbf{c.~~}\small le cercle de centre $ B(-1~;~2)$ et de rayon 9& \textbf{d.~~}l'ensemble vide.
\end{tabularx}
\end{enumerate}

\vspace{0,5cm}

