
\medskip

Ce QCM comprend 5 questions.

Pour chacune des questions, une seule des quatre réponses proposées est correcte.

Les questions sont indépendantes.

Pour chaque question, indiquer le numéro de la question et recopier sur la copie la lettre correspondante à la réponse choisie.

Aucune justification n'est demandée mais il peut être nécessaire d'effectuer des recherches au brouillon pour aider à déterminer votre réponse.

Chaque réponse correcte rapporte $1$ point. Une réponse incorrecte ou une question sans réponse n'apporte ni ne retire de point.

\medskip

\textbf{Question 1}

\medskip

$\cos(x)  = - \frac{\sqrt{3}}{2}$ pour:

\begin{center}
\begin{tabularx}{\linewidth}{|*{4}{X|}}\hline
\textbf{a.~~}$x = \dfrac{5\pi}{6}$ &\textbf{b.~~}$x = \dfrac{4\pi}{3}$&\textbf{c.~~}$x = -\dfrac{\pi}{3}$&\textbf{d.~~}$x = - \dfrac{\pi}{6}$\rule[-3mm]{0mm}{9mm}\\ \hline
\end{tabularx}
\end{center}

\medskip

\textbf{Question 2}

\medskip

Dans le plan muni d'un repère, on considère la droite (AB) passant par les points
A(-2~;~7) et B$(4~;~-5)$. Un vecteur directeur de la droite (AB) est :

\begin{center}
\begin{tabularx}{\linewidth}{|*{4}{X|}}\hline
\textbf{a.~~}$\vect{u}\binom{2}{2}$ &\textbf{b.~~}$\vect{u}\binom{-12}{6}$&\textbf{c.~~}$\vect{u}\binom{6}{-12}$&\textbf{d.~~}$\vect{u}\binom{2}{-12}$\rule[-3mm]{0mm}{9mm}\\ \hline
\end{tabularx}
\end{center}

\medskip

\textbf{Question 3}

\medskip

Dans le plan muni d'un repère, la droite d'équation $y=- 2x + 5$ pour vecteur directeur :

\begin{center}
\begin{tabularx}{\linewidth}{|*{4}{X|}}\hline
\textbf{a.~~}$\vect{u}\binom{2}{1}$ &\textbf{b.~~}$\vect{u}\binom{-1}{2}$&\textbf{c.~~}$\vect{u}\binom{1}{2}$&\textbf{d.~~}$\vect{u}\binom{-2}{1}$\rule[-3mm]{0mm}{9mm}\\ \hline
\end{tabularx}
\end{center}

\medskip

\textbf{Question 4}

\medskip

\parbox{0.58\linewidth}{Dans le plan muni d'un repère, la représentation graphique d'une parabole $P$ est donnée ci-contre. La forme canonique de son équation est :

\begin{center}
\begin{tabularx}{\linewidth}{|*{2}{X|}}\hline
\textbf{a.~~}$y = (x + 2)^2 + 5y$ &\textbf{b.~~}$y = (x - 5)^2 + 1$\\ \hline
\textbf{c.~~}$(x - 1) + 2y$&\textbf{d.~~}$y = (x - 2)^2 + 1$\\ \hline
\end{tabularx}
\end{center}} \hfill\parbox{0.38\linewidth}{\psset{unit=1cm}
\begin{pspicture*}(-1,-0.6)(4,5)
\psgrid[gridlabels=0pt,subgriddiv=1](0,0)(4,5)
\psaxes[linewidth=1.25pt]{->}(0,0)(0,0)(4,5)
\psplot[plotpoints=2000,linewidth=1.25pt,linecolor=red]{-1}{4}{x 2 sub dup mul 1 add}
\end{pspicture*}}

\medskip

\textbf{Question 5}

\medskip

Soit le cercle d'équation cartésienne $(x + 2)^2 + (y - 3)^2 = 9$ dans le
plan muni d'un repère orthonormé:

\begin{center}
\begin{tabularx}{\linewidth}{|*{4}{X|}}\hline
\textbf{a.~~}Le cercle a pour centre C$(-2~;~3)$ &\textbf{b.~~}Le cercle a pour centre C$(3~;~-2)$&\textbf{c.~~}Le cercle a pour rayon $R = 9^2$&\textbf{d.~~}Le cercle a pour centre C$(2~;~-3)$\\ \hline
\end{tabularx}
\end{center} 

\bigskip

