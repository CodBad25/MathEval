	\section*{Exercice 3 (5 points)}
	

	
	\subsection*{1. Calculer les termes d’indice 3 des suites $(u_n)$ et $(v_n)$.}
	
	\[
	\begin{array}{|c|c|c|}
		\hline
			n & u_n & v_n \\
		\hline
		0 & -4 & 0 \\
			\hline
		1 & 2 & 3,5 \\
			\hline
		2 & 4 & 5,25 \\
			\hline
		3 & 5 & 6,125 \\
			\hline
	\end{array}
	\]
	
	\subsection*{2. On s’intéresse aux variations de la suite $(u_n)$. Pour cela, on considère la fonction $f$ définie sur $[0; +\infty[$ par : $f(x) = \dfrac{8x - 4}{x + 1}$}
	
	\subsubsection*{a. Démontrer que la fonction $f$ est croissante sur $[0; +\infty[$.}
	
	Sur $[0; +\infty[$, la fonction est dérivable et sur cet intervalle :
	\[
	f'(x) = \dfrac{8(x + 1) - (8x - 4)}{(x + 1)^2} = \dfrac{8x + 8 - 8x + 4}{(x + 1)^2} = \dfrac{12}{(x + 1)^2}
	\]
	
	$f'(x) > 0$ car quotient de deux nombres supérieurs à zéro ; la fonction $f$ est donc strictement croissante sur l’intervalle $[0; +\infty[$.
	
	\subsubsection*{b. En déduire la monotonie de la suite $(u_n)$.}
	
	De la croissance de la fonction résulte la stricte croissance de la suite $(u_n)$.
	
	\subsection*{3. On considère l’affirmation suivante : « pour tout entier $n$, $u_n < v_n$ ». Camille pense que cette affirmation est vraie alors que Dominique pense le contraire. Pour les départager, on réalise le programme suivant écrit en langage Python :}
	
	\begin{python}
	def algo():
		n = 0
		u = -4
		v = 0
		while u < v:
			n = n + 1
			u = (8 * n - 4) / (n + 1)
			v = 0,5 * v + 3,5
		return n
	\end{python}
	
	Le programme dit que $u_{11} > v_{11}$, donc Dominique a raison.\\ Effectivement $u_{11} = \dfrac{84}{12} = 7$ et $v_{11} = 6,99658203125$.
	
