	
	\subsection*{Question 1}
	La fonction \( g \) est dérivable sur \( \mathbb{R} \) et sur cet intervalle :
	\[
	g'(x) = 100 e^{100x}.
	\]
	On sait que, quel que soit le réel \( x \), \( e^{100x} > 0 \) : la fonction \( g \) est donc strictement croissante sur \( \mathbb{R} \).
	
	\subsection*{Question 2}
\begin{align*}
	f(x) &= 100x^2 + 10x + 1\\
	& = 100 \left( x^2 + 0,1x + 0,01 \right)\\
	&= 100 \left[ (x + 0,05)^2 - 0,05^2 + 0,01 \right]\\
	& = 100 \left[ (x + 0,05)^2 + 0,0075 \right].
\end{align*}
	Le minimum de la fonction est obtenu lorsque \( x = -0,05 \) et ce minimum est égal à \( f(-0,05) = 0,75 \).
	
	L'axe de symétrie de la parabole représentative est \( x = -0,05 \).
	
	\subsection*{Question 3}
	On a 
\begin{align*}
	&a(x) = b(x)\\
	 \iff &3x^2 + 15x + 1 = 25x^2 + 5x - 100\\
	  \iff&22x^2 - 10x - 101 = 0.
\end{align*}
	On a :
	\[
	\Delta = (-10)^2 - 4 \times 22 \times (-101) = 100 + 8888 = 8988 > 0.
	\]
	L'équation a donc deux solutions. Les deux courbes ont deux points d'intersection.
	
	\subsection*{Question 4}
	\begin{align*}
		S &= 1 + 5 + 5^2 + \cdots + 5^{10} \quad (1), \\
		5S &= 5 + 5^2 + \cdots + 5^{10} + 5^{11} \quad (2), \\
		(2) - (1) \iff 4S &= 5^{11} - 1, \quad \text{d'où} \quad S = \dfrac{5^{11} - 1}{4} = 12207031.
	\end{align*}
	
	\subsection*{Question 5}
	Comme \( f'(-1) = 0 \) et \( f'(-1) \times f'(3) = 0 \).
	
	\medskip
	\textbf{a.} strictement positif \\
	\textbf{b.} strictement négatif \\
	\textbf{c.} égal à 0 \\
	\textbf{d.} égal à \( f'(-3) \).
	
