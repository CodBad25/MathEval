
\medskip

Ce QCM comprend 5 questions indépendantes. Pour chacune d'elles, une seule des réponses proposées est exacte.

Indiquer pour chaque question sur la copie la lettre correspondant à la réponse choisie.

Aucune justification n'est demandée.

Chaque réponse correcte rapporte $1$ point. Une réponse incorrecte ou une absence de réponse n'apporte ni ne retire de point.

\medskip

\begin{enumerate}
\item Pour tout réel $x$, \,$\sin(7\pi - x)$ est égal à :

\begin{center}
\begin{tabularx}{\linewidth}{|*{4}{X|}}\hline
\textbf{a.~~}$\sin x$&\textbf{b.~~}$- \sin x$&\textbf{c.~~}$\cos x$&\textbf{d.~~}$- \cos x$\\ \hline
\end{tabularx}
\end{center}

\item  Dans laquelle des quatre situations proposées ci-dessous le produit scalaire $\vect{\text{AB}} \cdot \vect{\text{AC}}$ est- il égal à 6 ?

\begin{center}
\begin{tabularx}{\linewidth}{|*{4}{>{\small}X|}}\hline
\textbf{a.~~}ABC est un triangle tel que: 

AB $= 6$, AC $= 4$ et 

BC $= 8$.
&\textbf{b.~~}Dans un repère orthonormé du plan: A$(-3~;~ 5)$, B$(2~;~-2)$
et C(1~;~7).
&\textbf{c.~~}ABC est un triangle rectangle en B tel que :
 
AB $= 3$ et $\text{BC} = 2$
&\textbf{d.~~}ABC est un triangle tel que :

AB $= 6$, AC $= 4$ et 

$\widehat{\text{BAC}} = 30\degres$.\\ \hline
\end{tabularx}
\end{center}
 
\item On considère la fonction $f$ définie sur $\R$ par: $f(x) = \dfrac{3x + 4}{x^2 + 1}$.

$f$ est dérivable sur $\R$ et, pour tout réel $x$,\; $f'(x)$ est égal à :

\begin{center}
\begin{tabularx}{\linewidth}{|*{4}{X|}}\hline
\textbf{a.~~}$\dfrac{3}{2x}$&\textbf{b.~~}$\dfrac{9x^2+8x+3\rule{0pt}{10pt}}{\left(x^2+1\right)^2}$&\textbf{c.~~}$\dfrac{-3x^2- 8x+3\rule{0pt}{10pt}}{\left(x^2+1\right)^2}$&\textbf{d.~~}$9x^2 + 8x+ 3$\\ \hline
\end{tabularx}
\end{center}

\item  Le plan est rapporté à un repère orthonormé.

L'ensemble des points $M(x~;~y)$ tels que $x^2 + y^2 - 10x +6y +30 = 0$ est :

\begin{center}
\begin{tabularx}{\linewidth}{|*{4}{X|}}\hline
\textbf{a.~~}une droite&\textbf{b.~~}une parabole&\textbf{c.~~}un cercle&\textbf{d.~~}ni une droite, ni une parabole, ni un cercle.\\ \hline
\end{tabularx}
\end{center}

\item La somme $1 + 5 + 5^2 + 5^3 + \ldots + 5^{30}$ est égale à :

\begin{center}
\begin{tabularx}{\linewidth}{|*{4}{X|}}\hline
\textbf{a.~~}$\dfrac{1 - 5^{30}\rule{0pt}{10pt}}{4\rule[-3pt]{0pt}{0pt}}$
&\textbf{b.~~}$\dfrac{5^{30} - 1}{4}$
&\textbf{c.~~}$\dfrac{1 - 5^{31}}{4}$
&\textbf{d.~~}$\dfrac{5^{31} - 1}{4}$\\ \hline
\end{tabularx}
\end{center}
\end{enumerate}

\bigskip

