
\medskip

Une agence de voyage propose deux formules week-end pour se rendre à Londres au départ de Nantes. Les clients choisissent leur moyen de transport : train ou avion.

De plus, s'ils le souhaitent, ils peuvent compléter leur formule par l'option \og visites guidées \fg.

Une étude a produit les données suivantes:

\setlength\parindent{9mm}
\begin{itemize}
\item[$\bullet~~$] 40\,\% des clients optent pour l'avion;
\item[$\bullet~~$]parmi les clients ayant choisi le train, 50\,\% choisissent aussi l'option \og visites guidées \fg{} ;
\item[$\bullet~~$]12\,\% des clients ont choisi à la fois l'avion et l'option \og visites guidées \fg.
\end{itemize}
\setlength\parindent{0mm}

\smallskip

On interroge au hasard un client de l'agence ayant souscrit à une formule week-end à Londres.

On considère les évènements suivants:

\setlength\parindent{9mm}
\begin{itemize}
\item $A$ : \og le client a choisi l'avion\fg{} ;
\item $V$ : \og le client a choisi l'option \og visites guidées \fg{}\fg.
\end{itemize}
\setlength\parindent{0mm}

\medskip

\begin{enumerate}
\item Déterminer $P_A(V)$.
\item Démontrer que la probabilité pour que le client interrogé ait choisi l'option
\og visites guidées\fg{} est égale à $0,42$.
\item Calculer la probabilité pour que le client interrogé ait pris l'avion sachant qu'il
n'a pas choisi l'option \og visites guidées \fg. Arrondir le résultat au centième.
\item On interroge au hasard deux clients de manière aléatoire et indépendante.

Quelle est la probabilité qu'aucun des deux ne prennent l'option \og visites guidées\fg{} ?
\end{enumerate}
