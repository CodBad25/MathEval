
\subsection*{1.}

On a :
\begin{align*}
&M(x \,;\, y) \in \mathcal{C} \\
\iff &AM^2 = 5^2 \\
\iff &(x - 2)^2 + (y - 5)^2 = 25 \\
\iff &x^2 - 4x + 4 + y^2 - 10y + 25 = 25 \\
\iff &x^2 + y^2 - 4x - 10y = -4.
\end{align*}

\subsection*{2.}

\begin{align*}
&B(5 \,;\, 9) \in \mathcal{C} \\
\iff &5^2 + 9^2 - 4 \times 5 - 10 \times 9 = -4 \\
\iff &25 + 81 - 20 - 90 = -4 \\
\iff &106 - 110 = -4,
\end{align*}
qui est vrai.

\subsection*{3.}

A est le centre du cercle et B est un point de ce cercle. On sait que la tangente en un point du cercle est perpendiculaire au rayon contenant ce point \(B\).

\subsection*{4.}

Si \(\mathcal{T}_B\) est cette tangente, on a :
\[
M(x \,;\, y) \in \mathcal{T}_B \iff (\overrightarrow{BM}) \perp (\overrightarrow{AB}) \iff \overrightarrow{BM} \cdot \overrightarrow{AB}.
\]
Avec \(\overrightarrow{BM} \begin{pmatrix} x - 5 \\ y - 9 \end{pmatrix}\) et \(\overrightarrow{AB} \begin{pmatrix} 3 \\ 4 \end{pmatrix}\), on a donc :
\begin{align*}
&\overrightarrow{BM} \cdot \overrightarrow{AB} = 0 \\
\iff &3(x - 5) + 4(y - 9) = 0 \\
\iff &3x + 4y - 15 - 36 = 0 \\
\iff &3x + 4y - 51 = 0.
\end{align*}

\subsection*{5.}

Un point de l'axe des ordonnées est caractérisé par son abscisse nulle (\(x = 0\)), donc :
\[
\begin{cases}
x^2 + y^2 - 4x - 10y = -4 \\
x = 0
\end{cases}
\Rightarrow y^2 - 10y = -4
\iff y^2 - 10y + 4 = 0.
\]
Pour cette équation :
\[
\Delta = 100 - 4 \times 4 = 84 = 4 \times 21 = (2\sqrt{21})^2 > 0,
\]
il y a donc deux racines :
\[
y_1 = \frac{10 + 2\sqrt{21}}{2} = 5 + \sqrt{21} \quad \text{et} \quad y_2 = \frac{10 - 2\sqrt{21}}{2} = 5 - \sqrt{21}.
\]
Le cercle \(\mathcal{C}\) a deux points communs avec l'axe des ordonnées de coordonnées :
\[
(5 + \sqrt{21} \,;\, 0) \quad \text{et} \quad (5 - \sqrt{21} \,;\, 0).
\]

