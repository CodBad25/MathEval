
\medskip

Ce QCM comprend 5 questions. Pour chacune des questions, une seule des quatre réponses proposées est correcte. Les cinq questions sont indépendantes.

Pour chaque question, indiquer le numéro de la question et recopier sur la copie la lettre correspondant à la réponse choisie. Aucune justification n'est demandée.

Chaque réponse correcte rapporte $1$ point. Une réponse incorrecte ou une absence de réponse n'apporte ni ne retire de point.

\medskip

\textbf{Question 1}

\medskip

Lors d'une même expérience aléatoire, deux évènements $A$ et $B$ vérifient:

\[P(A) = 0,4\quad ;\quad P(B) = 0,6\quad ;\quad  P\left( A \cap \overline{B}\right )= 0,3\]

Alors:

\begin{center}
\begin{tabularx}{\linewidth}{|*{4}{X|}}\hline
\textbf{a.~~}$P(A \cap B)= 0,1$&\textbf{b.~~}$P(A \cap B)= 0,24$&\textbf{c.~~}$P(A \cup B)=1$&\textbf{d.~~}$P(A \cup B)= 0,7$\\ \hline
\end{tabularx}
\end{center}

\medskip

\textbf{Question 2}

\medskip

On considère la fonction $f$ définie sur $\R$ par $f(x) = x^2 - 3x + 4$. L'abscisse du minimum de $f$ est :

\begin{center}
\begin{tabularx}{\linewidth}{|*{4}{X|}}\hline
\textbf{a.~~}$-\dfrac{3}{2}$&\textbf{b.~~}$\dfrac{2}{3}$&\textbf{c.~~}$\dfrac{3}{2}$&\textbf{d.~~}$1$\rule[-3mm]{0mm}{8mm}\\ \hline
\end{tabularx}
\end{center}

\medskip

\textbf{Question 3}

\medskip

Soit $\left(u_n\right)$ une suite arithmétique telle que $u_5 = 26$ et $u_9 = 8$. La raison de $\left(u_n\right)$ vaut :

\begin{center}
\begin{tabularx}{\linewidth}{|*{4}{X|}}\hline
\textbf{a.~~}$- 18$&\textbf{b.~~}$\dfrac{8}{26}$&\textbf{c.~~}$4,5$&\textbf{d.~~}$- 4,5$\rule[-3mm]{0mm}{8mm}\\ \hline
\end{tabularx}
\end{center}

\medskip

\textbf{Question 4}

\medskip

On considère l'algorithme suivant, écrit en langage usuel:

\begin{center}
\begin{tabularx}{0.45\linewidth}{X}
\texttt{Suite(N)} \\
\quad \texttt{A $\gets$ 10}\\
\quad \texttt{Pour k de 1 à N}\\
\qquad \texttt{A $\gets$ 2*A-4}\\
\quad \texttt{Fin Pour}\\
\quad \texttt{Renvoyer A}\\
\end{tabularx}
\end{center}

Pour la valeur $N = 4$ le résultat affiché sera:

\begin{center}
\begin{tabularx}{\linewidth}{|*{4}{X|}}\hline
\textbf{a.~~}$4$&\textbf{b.~~}$100$&\textbf{c.~~}$52$&\textbf{d.~~}$196$\\ \hline
\end{tabularx}
\end{center}

\medskip

\textbf{Question 5}

\medskip

On considère un rectangle ABCD tel que AB $= 3$ et AD $= 2$.

\medskip

\begin{center}
\psset{unit=1cm}
\begin{pspicture}(3,2)
\psframe(3,2)
\uput[ul](0,2){A}\uput[ur](3,2){B}
\uput[dl](0,0){D}\uput[dr](3,0){C}
\psline(0,2)(3,0) \psline(0,0)(3,2) 
\end{pspicture}
\end{center}


Alors le produit scalaire $\vect{\text{AC}} \cdot \vect{\text{DB}}$ vaut:

\begin{center}
\begin{tabularx}{\linewidth}{|*{4}{X|}}\hline
\textbf{a.~~}$0$&\textbf{b.~~}$5$&\textbf{c.~~}$6$&\textbf{d.~~}$- 6$\\ \hline
\end{tabularx}
\end{center}

\bigskip

