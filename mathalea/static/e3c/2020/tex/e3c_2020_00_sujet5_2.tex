
\medskip

\emph{Dans tout l'exercice, les résultats seront arrondis, si nécessaire, au dix millième.}

\medskip

On étudie un test de dépistage pour une certaine maladie dans une population donnée. On sait que 1\,\% de la population est atteint de la maladie. Des études ont montré que si une personne est malade, alors le test se révèle positif dans 97\,\% des cas et si une personne n'est pas malade, le test est négatif dans 98\,\% des cas.

Pour une personne à qui on fait passer le test de dépistage on associe les évènements :

\setlength\parindent{9mm}
\begin{itemize}
\item[$\bullet~~$] $M$ : la personne est malade,
\item[$\bullet~~$] $T$ : le test est positif.
\end{itemize}
\setlength\parindent{0mm}

\medskip

\begin{enumerate}
\item Recopier et compléter sur la copie l'arbre de probabilité suivant en utilisant les données de l'exercice.

\begin{center}
\pstree[treemode=R,nodesepA=0pt,nodesepB=3pt]{\TR{}}
{\pstree{\TR{$M~~$}}
	{\TR{$T~~$}
	\TR{$\overline{T}~~$}
	}
\pstree{\TR{$\overline{M}~~$}}
	{\TR{$T~~$}
	\TR{$\overline{T}~~$}
	}
}
\end{center}


\item Justifier que $P\left (\overline{M} \cap T\right ) = \np{0,0198}$.
\item Montrer que  $P(T) = \np{0,0295}$.
\item Calculer $P_T(M)$.
\item Une personne dont le test se révèle positif est-elle nécessairement atteinte par cette maladie?
\end{enumerate}

\bigskip

