
\subsection*{1.}

\begin{center}
\psset{xunit=0.04cm,yunit=0.04cm,labelFontSize=\scriptstyle}
\begin{pspicture}(-130,-130)(130,130)
\psaxes[linewidth=1.25pt,Dx=10,Dy=10,labels=none]{->}(0,0)(-120,-120)(120,120)
\psset{unit=0.04cm}
\pscircle[linewidth=1.25pt,linecolor=blue](0,0){110}
\uput[dl](0,0){0}\uput[d](10,0){10}
\uput[l](0,10){10}\uput[d](115,0){$x$}
\uput[l](0,115){$y$}\uput[u](-100,-90){\blue \large\boldmath{$\mathcal{C}$}}
\end{pspicture}
\end{center}

\paragraph{a.}
\begin{align*}
&M(x\,;\,y) \in \mathcal{C} \\
\iff &OM^2 = 110 \\
\iff &(x - 0)^2 + (y - 0)^2 = 110^2 \\
\iff &x^2 + y^2 = 12100.
\end{align*}

\paragraph{b.} On a \(\overrightarrow{OA} \begin{pmatrix} -30 \\ 15 \end{pmatrix}\) et \(\overrightarrow{OD} \begin{pmatrix} 80 \\ -40 \end{pmatrix}\).

Comme :
\[
\det(\overrightarrow{OA} \,;\, \overrightarrow{OD}) = (-30)(-40) - 15 \times 80 = 1200 - 1200 = 0,
\]
les vecteurs sont colinéaires donc les points \(O\), \(A\) et \(D\) sont alignés.

\subsection*{2.}

\paragraph{a.} Avec \(\overrightarrow{AG} \begin{pmatrix} 20 \\ -25 \end{pmatrix}\), on a :
\[
\overrightarrow{AG} \cdot \overrightarrow{AO} = 20 \times 30 + (-25) \times (-15) = 600 - 750 = -150.
\]

\paragraph{b.} L'équation réduite de la droite \((AD)\) est \( y = -\dfrac{1}{2}x \).

Soit \(H\) le projeté orthogonal de \(G\) sur la droite \((AD)\). Quel que soit le point \(M(x\,;\,y)\) de la droite \((AD)\), le triangle \(GHM\) est rectangle en \(H\) et on a \(GH < GM\) (l'hypoténuse est le côté le plus grand) : le point \(H\) est donc le point de \((AD)\) le plus proche de \(G\).

Avec \(H(x\,;\,-\dfrac{1}{2}x)\), on a :
\begin{align*}
&\overrightarrow{\text{GH}} \cdot \overrightarrow{\text{OA}} = 0 \\
\iff &(x + 10) \times (-30) + \left( -\frac{1}{2}x + 10 \right) \times 15 = 0 \\
\iff &-30x - 300 - 7{,}5x + 150 = 0 \\
\iff &-150 = 37{,}5x \\
\iff &x = -\frac{150}{37{,}5} = -4.
\end{align*}

Le point de \( (AD) \) le plus proche de \( G \) est donc \( H(-4 \,;\, 2) \) : ce n'est pas \( O \).

