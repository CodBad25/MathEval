
\medskip

Une entreprise fabrique des pièces en acier, toutes identiques, pour l’industrie
aéronautique.

Ces pièces sont coulées dans des moules à la sortie du four. Elles sont stockées dans un
entrepôt dont la température ambiante est maintenue à 25\textcelsius.

Ces pièces peuvent être modelées dès que leur température devient inférieure ou égale à
600\textcelsius{} et on peut les travailler tant que leur température reste supérieure ou égale à 500\textcelsius.

La température de ces pièces varie en fonction du temps.

On admet que la température en degré Celsius de ces pièces peut être modélisée par la
fonction $f$ définie sur l’intervalle $[0 ; +\infty[$ par :
\[f(t) = \np{1 375}\e^{-0,075t} + 25,\]
où $t$ correspond au temps, exprimé en heures, mesuré après la sortie du four.

\begin{enumerate}
\item  Calculer la température des pièces à la sortie du four.
\item Étudier le sens de variation de la fonction $f$ sur l’intervalle $[0 ; +\infty[$.

 Ce résultat était-il prévisible dans le contexte de l’exercice ?
\item Les pièces peuvent-elles être modelées 10 heures après la sortie du four ? Après 14 heures ?
\item On souhaite déterminer le temps minimum d’attente en heures après la sortie du four avant
de pouvoir modeler les pièces.

\begin{enumerate}
\item  Compléter l’algorithme donné ci-dessous pour qu’il renvoie ce temps minimum d’attente en heure (arrondi par excès à 0,1 près).

\begin{tabular}[]{l}
from math import exp\\
def f(t) :\\
\phantom{xxxx}return 1375*exp (-0.075 t)+25\\
def seuil(()\\
\phantom{xxxx}t=\dotfill\\\
\phantom{xxxx}temperature =\dotfill\\
\phantom{xxxx}while temperature >=\dotfill\\
\phantom{xxxxxx}t=t+0.1\\
\phantom{xxxxxx}temperature =\dotfill\\
\phantom{xxxx}return t\\
\end{tabular}

\item Déterminer ce temps minimum d’attente. On arrondira au dixième.
\end{enumerate}
\end{enumerate}

\vspace{0,5cm}

