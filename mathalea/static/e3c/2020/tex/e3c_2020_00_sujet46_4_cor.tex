
\subsection*{Partie A}

\paragraph{1.}
On lit :
\begin{itemize}
    \item à l'instant \( t = 0 \), environ 2{,}1 milliers de pucerons ;
    \item à l'instant \( t = 6 \), environ 5 milliers de pucerons.
\end{itemize}

\paragraph{2.}
La tangente contient les points \( (0\,;\,2{,}1) \) et \( (2\,;\,4{,}3) \) ; sa pente est donc égale à :
\[
\dfrac{4{,}3 - 2{,}1}{2 - 0} = \dfrac{2{,}2}{2} = 1{,}1,
\]
donc une vitesse de prolifération de 1100 pucerons par jour.

\subsection*{Partie B}

\(f(t) = 0{,}003t^3 - 0{,}12t^2 + 1{,}1t + 2{,}1\).

\paragraph{1.}
On dérive la fonction polynôme \( f \) :
\[
f'(t) = 0{,}009t^2 - 0{,}24t + 1{,}1.
\]

\paragraph{2.}
Pour le trinôme \( 0{,}009t^2 - 0{,}24t + 1{,}1 \) :
\begin{align*}
\Delta &= 0{,}24^2 - 4 \times 0{,}009 \times 1{,}1 \\
&= 0{,}0576 - 0{,}0396 \\
&= 0{,}018 > 0.
\end{align*}

Le trinôme a donc deux racines :

\[
t_1 = \dfrac{0{,}24 + \sqrt{0{,}018}}{2 \times 0{,}009} \approx 20{,}8 \quad \text{et} \quad t_2 = \dfrac{0{,}24 - \sqrt{0{,}018}}{2 \times 0{,}009} \approx 5{,}9.
\]

On sait que \( f'(t) \) a le signe de \( a = 0{,}009 \) donc est positive sauf entre les racines.

