  
\medskip

Un globe-trotter a comme objectif de parcourir \np[km]{2000}  à pied. Il peut parcourir \np[km]{50}  en
une journée, mais, la fatigue s'accumulant, la distance qu'il parcourt diminue de 2\,\% chaque
nouvelle journée.

On note la distance $d_n$ la distance parcourue durant le $n$-ième jour.

Le premier jour de son périple, il parcourt donc $d_1 = \np[km]{50}$.

\medskip

\begin{enumerate}
\item Calculer la distance parcourue le deuxième jour.
\item Quelle est la nature de la suite $\left(d_n\right)$ ? Donnez ses éléments caractéristiques.
\item Pour tout entier naturel $n\geqslant 1$, déterminer l'expression de $d_n$ en fonction de $n$.
\item Pour calculer le nombre de jours qu'il faudra au globe-trotter pour atteindre son objectif, on
a écrit le programme Python suivant :

\begin{python}
def nb jours :
	j=1
	u=50
	S=50
	While ...... :
		u=0,98*u
		S=S+u
		j=...... 
	return j
\end{python}

Compléter les deux lignes incomplètes de ce programme.

\item \begin{minipage}[t]{4cm}

À l'aide de l'extrait de tableur ci-contre, déterminer
quand le globe-trotter aura atteint son objectif.
\end{minipage}\hspace{2.01cm}
\begin{minipage}[t]{9.215cm}
\begin{tabularx}{0.5\linewidth}{|>{\columncolor{lightgray}}m{0.35cm}|*{3}{>{\centering \arraybackslash}X|}}
\hline
\rowcolor{lightgray}&A&\cellcolor{blue}B&C\\\hline
1&\cellcolor{green}j&\cellcolor{green}u&\cellcolor{green}s\\\hline
2&1&50&50\\\hline
\multicolumn{4}{c}{}\\\hline
76&75&11&\np{1951}\\\hline
77&76&11&\np{1962}\\\hline
78&77&11&\np{1972}\\\hline
79&78&11&\np{1983}\\\hline
80&79&10&\np{1993}\\\hline
81&80&10&\np{2003}\\\hline
82&81&10&\np{2013}\\\hline
\end{tabularx}
\end{minipage}
\end{enumerate}

\vspace{1cm}

