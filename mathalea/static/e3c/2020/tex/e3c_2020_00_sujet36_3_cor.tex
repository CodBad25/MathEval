
\subsection*{1.}

Baisser de 1 \%, c'est multiplier par \(1 - \dfrac{1}{100} = 1 - 0{,}01 = 0{,}99\).

Donc \( u_1 = u_0 \times 0{,}99 = 0{,}848 \times 0{,}99 = 0{,}83952 \), soit environ 83{,}95 \%.

\subsection*{2.}

D'une année sur l'autre pendant 10 ans, on passe donc du taux de l'année à celui de l'année suivante par produit par \(0{,}99\). On a donc pour tout naturel \( 0 \leqslant n \leqslant 9 \), \( u_{n+1} = 0{,}99 u_n \), égalité qui montre que la suite \( (u_n) \) est une suite géométrique de raison \(q = 0{,}99\) et de premier terme \( u_0 = 0{,}848 \).

\subsection*{3.}

À la fin de l'exécution, on obtient le nombre d'années au bout duquel le taux de scolarisation passera sous 80 \% (on aura \( n = 6 \)).

\subsection*{4.}

On sait que pour \( 0 \leqslant n \leqslant 9 \), \( u_n = 0{,}848 \times 0{,}99^n \).

\subsection*{5.}

2005 correspond à \( n = 10 \), d'où \( u_{10} = 0{,}848 \times 0{,}99^{10} \approx 0{,}76692 \), soit environ 76{,}69 \%.

