
\medskip

Un parent d'élèves propose un jeu pour la fête de l'école.

Une urne opaque contient $100$ billes indiscernables au toucher : 10 billes rouges, 30 billes blanches et 60 billes vertes.

\medskip

Pour une partie, chaque joueur doit miser 2 jetons. Ensuite, le joueur prélève une bille au hasard dans l'urne.

\medskip

\setlength\parindent{1cm}
\begin{itemize}[label=\textbullet]
\item Si la bille prélevée est rouge, le joueur récupère $8$ jetons.
\item Si la bille est blanche, le joueur récupère $4$ jetons.
\item Si la bille est verte, le joueur ne gagne rien.
\end{itemize}
\setlength\parindent{0cm}

On note $X$ la variable aléatoire égale au gain algébrique du joueur en nombre de jetons, c'est-à-dire, le nombre de jetons gagnés diminué de la mise.

\medskip

\begin{enumerate}
\item 
	\begin{enumerate}
		\item Établir que la loi de probabilité de $X$ est donnée par :
		
\begin{center}
\begin{tabularx}{\linewidth}{|m{4cm}|*{3}{>{\centering \arraybackslash}X|}}\hline
Valeurs $a$ prises par $X$&$-2$ &2 &6\\ \hline
$p(X = a)$& 0,6 &0,3 &0,1\\ \hline
\end{tabularx}
\end{center}
		\item Démontrer que le jeu est équitable, c'est-à-dire que l'espérance de $X$ est nulle.
		\item Calculer la variance puis l'écart-type de $X$. On arrondira au centième.
	\end{enumerate}
\item Pour financer les différentes actions de l'école, les organisateurs de la fête veulent modifier le jeu pour qu'il leur devienne favorable. Ils décident alors d'ajouter des billes vertes dans l'urne.

Combien de billes vertes doit-on ajouter dans l'urne pour que l'espérance du jeu soit égale à $-1$ ?	
\end{enumerate}
