
\medskip

\Oij{} est un repère orthonormé du plan.

On considère les points A, B et C de coordonnées respectives $(-2~;~0)$, (6~;~0) et (0~;~6).

Les points A$'$, B$'$ et C$'$ milieux respectifs des segments [BC], [AC] et [AB].

Le cercle $\Gamma$ passant par les points A$'$, B$'$ et C$'$ a pour centre le point I de coordonnées (1~;~2).


\begin{center}
\psset{unit=1cm,arrowsize=2pt 3}
\begin{pspicture*}(-3.5,-0.75)(7.1,6.5)
\psgrid[gridlabels=0pt,subgriddiv=1,gridwidth=0.1pt]
\psaxes[linewidth=0.6pt,labelFontSize=\scriptstyle](0,0)(-3.5,-0.5)(7,6.5)
\psaxes[linewidth=0.6pt,labelFontSize=\scriptstyle]{->}(0,0)(1,1)
\pspolygon(-2,0)(6,0)(0,6)
\psdots(-2,0)(6,0)(0,6)(1,2)(3,3)(-1,3)
\uput[ul](-2,0){A}\uput[ur](6,0){B}\uput[ur](3,3){A$'$}\uput[ur](0,6){C}\uput[ul](-1,3){B$'$}
\uput[ur](1,2){I}\uput[dl](0,0){O}
\uput[d](0.5,0){$\vect{\imath}$}\uput[l](0,0.5){$\vect{\jmath}$}
\pscircle(1,2){2.23607}\uput[r](3,1){$\Gamma$}
\end{pspicture*}
\end{center}

\smallskip

\begin{enumerate}
\item 
	\begin{enumerate}
		\item Calculer le rayon de ce cercle.
		\item En déduire qu'une équation du cercle $\Gamma$ est $(x - 1)^2 + (y - 2)^2 = 5$. 
	\end{enumerate}
\item  Propriété des hauteurs du triangle ABC
	\begin{enumerate}
		\item On admet que O est le pied de la hauteur issue de C. 
		
Montrer que le point O est sur le cercle $\Gamma$.
		\item Soit H$_{\text{A}}$ le pied de la hauteur issue de A. 
		
Montrer que H$_{\text{A}}$ a pour coordonnées (2; 4). 
		\item Justifier que la point H$_{\text{A}}$ est sur le cercle $\Gamma$.
	\end{enumerate}
\end{enumerate}
