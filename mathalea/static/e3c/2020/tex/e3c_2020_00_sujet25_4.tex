
\medskip

Soit $f$ la fonction définie sur $[0~;~+\infty[$ par 

\[f(x)= -x^2 +2x + 4.\]

Dans le plan muni d’un repère orthonormé, on note $\mathcal{C}$  sa courbe représentative.

\medskip

\begin{enumerate}
\item Déterminer les variations de la fonction $f$ sur $[0~;~+\infty[$ .
\item Déterminer la valeur exacte de l’abscisse du point A, intersection de la courbe $\mathcal{C}$ et de l’axe des abscisses, puis en donner une valeur approchée à $10^{-2}$ près.
\item On note $\mathcal{T}$ la tangente à la courbe $\mathcal{C}$ au point B d’abscisse 2.
Déterminer l’équation réduite de la droite $\mathcal{T}$
\item Tracer la droite $\mathcal{T}$ sur le graphique suivant.

\begin{center}

 \textbf{À rendre avec la copie}
 
 \vspace{1.5cm}
 
\psset{unit=1.5cm,labelFontSize=\scriptstyle,showorigin=false}
\begin{pspicture}(-0.8,-0.9)(5,9)
\multido{\n=0+1}{5}{\psline[linewidth=0.45pt](\n,0.2)(\n,8.4)}
\multido{\n=0+1}{9}{\psline[linewidth=0.45pt](0,\n)(4.4,\n)}
\psaxes[linewidth=0.95pt]{->}(0,0)(-0.44,-0.2)(4.4,8.4)
\def\Func{x x neg 2 add  mul 4 add }
\psplot[plotpoints=2000,linewidth=0.65pt,linecolor=blue,linewidth=1.25pt]{0}{3.3361}{\Func}
\uput[dl](-0.2,-0.2){O}\uput[ur](3.2361,0){A}
\psdot[dotstyle=Mul,dotscale=1.6,linecolor=blue](3.2361,0)
\uput[u](0.75,5){\blue $\mathcal{C}_f$}
\end{pspicture}
\end{center}

\item On admet que la courbe $\mathcal{C}$ est toujours en-dessous de la droite $\mathcal{T}$.

La société Logo reçoit une commande de l’entreprise RapidResto, qui lui demande de confectionner des logos dans des plaques rectangulaires de largeur \np[dm ]{4} et de hauteur \np[dm]{8}  selon le modèle ci-dessous.

Le bord supérieur du logo est modélisé par la courbe $\mathcal{C}$ tracée dans le repère orthonormé figurant sur le graphique précédent dont l’unité graphique est le décimètre (dm).

\emph{ Les figures ci-dessous ne sont pas à l’échelle}.

\begin{center}
\begin{tabular}[]{|l|r|} 
\hline
\psset{xunit=0.7cm,yunit=0.7cm,labelFontSize=\scriptstyle,labelsep=0.1pt}
\begin{pspicture}(-0.4,-0.94)(5.6,8.6)
\multido{\n=-0.4+0.2}{24}{\psline[linewidth=0.35pt,linecolor=lightgray](\n,-0.6)(\n,8.4)}
\multido{\n=-0.6+0.2}{44}{\psline[linewidth=0.35pt,linecolor=lightgray](-0.4,\n)(4.2,\n)}
\multido{\n=0+1}{5}{\psline[linewidth=0.45pt](\n,-0.6)(\n,8.4)}
\multido{\n=0+1}{9}{\psline[linewidth=0.45pt](-0.4,\n)(4.2,\n)}
\psaxes[linewidth=0.95pt]{->}(0,0)(4.2,8.4)
\def\Func{x x neg 2 add  mul 4 add }
\psplot[plotpoints=1000,linewidth=0.85pt,linecolor=red]{0}{3.2361}{\Func}
\psdots[dotstyle=Mul,dotscale=1.6,linecolor=blue](0,8)(4,8)(4,0)(0,0)(3.2361,0)(1,5)(0,4)
\uput[ur](0,8){Q}\uput[ur](4,8){P}\uput[u](4,0){M}\uput[ur](0,0){O}\uput[ur](3.2361,0){A}\uput[u](1,5){C}\uput[l](0,4){D} 
\pscustom[fillstyle=solid,fillcolor=blue,linecolor=blue]
{
\psplot[plotpoints=3000,linewidth=1.25pt,linecolor=blue]{0}{3.2361}{\Func}
\lineto(3.2361,0)\lineto(0,0)
\closepath % indispensable !
}
\uput[r](0.5,2){\textbf{\footnotesize RapidResto}}
\end{pspicture}
&
\psset{xunit=0.7cm,yunit=0.7cm}
\begin{pspicture}(-1,-1)(6,9)
\psframe(0,0)(4,8)
\psline(0,8)(4,0)
\uput[dr](0.5,8){\emph{\scriptsize découpe dans}}
\uput[dr](0.6,7.61){\emph{\scriptsize la diagonale}}
\uput[dr](0.7,7.185){\emph{\scriptsize de la plaque}}
\psline{->}(1.8,6.65)(1.8,4.5)
\psline{<->}(-0.4,8)(-0.4,0)\psline{<->}(0,-0.4)(4,-0.4)
\uput[d](2,-0.5){\scriptsize\np[dm]{4}}\uput[l](-0.35,4){\scriptsize\np[dm]{8}}
\end{pspicture}\\
Logo à réaliser &Découpe selon la diagonale \\\hline
\end{tabular}
\end{center}

Dans un souci d’économie, l’entreprise Logo espère pouvoir réaliser deux logos identiques dans une seule plaque, en la coupant dans sa diagonale.

 Est-ce possible ? Justifier à l’aide des questions précédentes.
 \end{enumerate}
 

