
\medskip

L'évolution d'une population de bactéries dépend de l'environnement dans lequel ces bactéries sont placées. Cette population peut être modélisée par la suite $(P_n)$ définie, pour tout entier naturel $n$, par : $P_{n+1}= (1+\alpha)P_n+\beta$, où $\alpha$ et $\beta$ sont des paramètres liés à l'environnement, notamment à la température et à l'humidité.

$P_n$ modélise alors le nombre de bactéries, en milliers, qui composent cette population $n$ jours après les avoir introduites dans un certain environnement.

\medskip

\begin{enumerate}
\item  Une population, initialement composée de 500 mille bactéries, est étudiée dans un
environnement pour lequel $\alpha=0,2$ et $\beta=70$.
\begin{enumerate}
\item  Combien y a-t-il de bactéries dans cet environnement au bout de deux jours ?
\item Recopier et compléter le programme suivant, écrit en langage Python, pour que la
fonction Nombrebacteries renvoie le nombre de bactéries présentes dans cet environnement au bout de N jours.

\begin{center}
\begin{tabular}[]{|l}
def Nombrebacteries(N):\\
	p=500\\
	for i in range(0,N):\\
	\phantom{xxx}		P = $\dots$\\
	return $\dots$\\
\end{tabular}
\end{center}
\end{enumerate}
\item Une autre population, initialement composée de $500$ mille bactéries, est étudiée dans un
nouvel environnement. On constate que le nombre de bactéries de cette population augmente de 9\,\% par jour.
	\begin{enumerate}
		\item Déterminer les valeurs des paramètres $\alpha$ et $\beta$ pour cet environnement.
		\item Quelle est, dans ce cas, la nature de la suite $\left(P_n\right)$ ?
		\item Justifier qu'après 9 jours dans cet environnement, le nombre de bactéries de cette population
a doublé.
	\end{enumerate}
\end{enumerate}
\vspace{0,5cm}

