
\medskip

Un magasin effectue des promotions avant sa liquidation définitive, chaque semaine les prix des articles sont diminués  de 10\,\% par rapport à la semaine précédente. 

Un manteau coûte $200$~\euro{} avant le début de la liquidation, on pose $u_0=200$ et on note $u_n$ son prix lors de la $n$-ième semaine de liquidation.

\medskip

\begin{enumerate}
\item Calculer les termes $u_1$ et $u_2$ de la suite $(u_n)$.
\item Montrer que la suite $\left(u_n\right)$ est une suite géométrique de premier terme $u_0=200$ dont on précisera la raison et exprimer le terme général de la suite $\left(u_n\right)$ en fonction de $n$.
\item La liquidation dure 12 semaines, déterminer le prix du manteau à la fin de la liquidation s'il est toujours en vente. On donnera le résultat arrondi au centime.
\item On considère la fonction suivante, écrite en langage Python :

\begin{center}
\begin{tabular}[]{m{2.5cm}}
def seuil(x) :\\
\hspace{1.2em}u = 200\\
\hspace{1.2em}n = 0\\
 \hspace{1.2em}while \dotfill:\\
\hspace{2.5em}u =\dotfill\\
 \hspace{2.5em}n =\dotfill\\
\hspace{1.2em}return n\\
\end{tabular}
\end{center}

Recopier et compléter sur la copie la fonction afin qu'elle renvoie le nombre de semaines nécessaires pour que le terme général de la suite $\left(u_n\right)$ soit inférieur au nombre réel $x$.
\item Une personne décide d'acheter le manteau dès que son prix sera inférieur à $100$~\euro. Combien de semaines devra-t-elle attendre ?
\end{enumerate}

\vspace{0,5cm}

