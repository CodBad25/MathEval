	\section*{Exercice 4 (5 points)}
	
	\subsection*{1. Donner une équation cartésienne de la droite $(BD)$ et une équation du cercle de diamètre $[AB]$.}
	\definecolor{uququq}{rgb}{0.25,0.25,0.25}
	\definecolor{zzttqq}{rgb}{0.6,0.2,0}
	\definecolor{xdxdff}{rgb}{0.49,0.49,1}
	\definecolor{qqqqff}{rgb}{0,0,1}
	\definecolor{cqcqcq}{rgb}{0.75,0.75,0.75}
	\begin{tikzpicture}[line cap=round,line join=round,>=triangle 45,x=1.0cm,y=1.0cm]
		\draw [color=cqcqcq,dash pattern=on 1pt off 1pt, xstep=1.0cm,ystep=1.0cm] (-4.3,-4) grid (4,6);
		\draw[->,color=black] (-4.3,0) -- (4,0);
		\foreach \x in {-4,-3,-2,-1,1,2,3}
		\draw[shift={(\x,0)},color=black] (0pt,2pt) -- (0pt,-2pt) node[below] {\footnotesize $\x$};
		\draw[->,color=black] (0,-4) -- (0,6);
		\foreach \y in {-4,-3,-2,-1,1,2,3,4,5}
		\draw[shift={(0,\y)},color=black] (2pt,0pt) -- (-2pt,0pt) node[left] {\footnotesize $\y$};
		\draw[color=black] (0pt,-10pt) node[right] {\footnotesize $0$};
		\clip(-4.3,-4) rectangle (4,6);
		\fill[color=zzttqq,fill=zzttqq,fill opacity=0.1] (-1.96,5.27) -- (3,3) -- (-3,-3) -- cycle;
		\draw(0,3) circle (3cm);
		\draw [color=zzttqq] (-1.96,5.27)-- (3,3);
		\draw [color=zzttqq] (3,3)-- (-3,-3);
		\draw [color=zzttqq] (-3,-3)-- (-1.96,5.27);
		\draw [dash pattern=on 3pt off 3pt,domain=-4.3:4] plot(\x,{(--19.92-6*\x)/6});
		\draw [->] (0,0) -- (1,0);
		\draw [->] (0,0) -- (0,1);
		\begin{scriptsize}
			\fill [color=qqqqff] (-3,3) circle (1.5pt);
			\draw[color=qqqqff] (-2.86,3.28) node {$A$};
			\fill [color=qqqqff] (3,3) circle (1.5pt);
			\draw[color=qqqqff] (3.16,3.28) node {$B$};
			\fill [color=qqqqff] (3,-3) circle (1.5pt);
			\draw[color=qqqqff] (3.16,-2.72) node {$C$};
			\fill [color=qqqqff] (-3,-3) circle (1.5pt);
			\draw[color=qqqqff] (-2.84,-2.72) node {$D$};
			\fill [color=xdxdff] (-1.96,5.27) circle (1.5pt);
			\draw[color=xdxdff] (-1.8,5.56) node {$E$};
			\fill [color=uququq] (1.66,1.66) circle (1.5pt);
			\draw[color=uququq] (2.12,1.82) node {$H$};
			\fill [color=uququq] (0,0) circle (1.5pt);
			\draw[color=uququq] (0.32,-0.54) node {$O$};
		\end{scriptsize}
	\end{tikzpicture}
	\begin{itemize}
		\item Équation de $(BD)$ : $B$ et $D$ sont deux points dont les coordonnées sont égales : une équation de la droite $(BD)$ est donc $y = x$.
		\item Équation du cercle $C$ de diamètre $[AB]$ donc de centre $I(0; 3)$ :
		\[
		M(x; y) \in C \iff IM^2 = 3^2 \iff (x - 0)^2 + (y - 3)^2 = 9 \iff x^2 + y^2 - 6y = 0
		\]
	\end{itemize}
	
	\subsection*{2. Montrer que la hauteur du triangle $BDE$ issue de $E$ admet pour équation cartésienne $x + y - (1 + \sqrt{5}) = 0$.}
	
	\[
	M(x; y) \in (EH) \iff \overrightarrow{EM} \cdot \overrightarrow{DB} = 0
	\]
	
	Avec
	\[
	\overrightarrow{EM} = \begin{pmatrix}
		x - (-2) \\
		y - (3 + \sqrt{5})
	\end{pmatrix}
	\quad \text{et} \quad \overrightarrow{DB} = \begin{pmatrix}
		3 - (-3) \\
		3 - (-3)
	\end{pmatrix}
	\]
	
	On a donc :
	\[
	\overrightarrow{EM} \cdot \overrightarrow{DB} = 6(x + 2) + 6(y - 3 - \sqrt{5}) = 0 \iff x + 2 + y - 3 - \sqrt{5} = 0 \iff x + y - (1 + \sqrt{5}) = 0
	\]
	
	\subsection*{3. Déterminer les coordonnées du projeté orthogonal $H$ du point $E$ sur la droite $(BD)$.}
	
	$H$ est le point commun aux droites perpendiculaires $(EH)$ et $(BD)$ ; ses coordonnées $x$ et $y$ vérifient donc les équations de ces deux droites, donc le système :
	\[
	\begin{cases}
		x + y - (1 + \sqrt{5}) = 0 \\
		y = x
	\end{cases}
	\]
	
	d'où par somme :
	\[
	2y = 1 + \sqrt{5} \iff y = \dfrac{1 + \sqrt{5}}{2}
	\]
	
	et puisque $x = y$, $H\left( \dfrac{1 + \sqrt{5}}{2}; \dfrac{1 + \sqrt{5}}{2} \right)$.
	
	\subsection*{4. Calculer l’aire du triangle $BDE$ (en unités d’aire).}
	
	\begin{itemize}
		\item $BD = 6\sqrt{2}$ (diagonale d’un carré de côté 6)
		\item $\overrightarrow{EH} = \begin{pmatrix}
			\dfrac{1 + \sqrt{5}}{2} - (-2) \\
			\dfrac{1 + \sqrt{5}}{2} - (3 + \sqrt{5})
		\end{pmatrix}$, soit $\overrightarrow{EH} = \begin{pmatrix}
			\dfrac{5 + \sqrt{5}}{2} \\
			\dfrac{-5}{2}
		\end{pmatrix}$
	\end{itemize}
	
	Donc
	\[
	EH^2 = \left( \dfrac{5 + \sqrt{5}}{2} \right)^2 + \left( \dfrac{-5}{2} \right)^2 = \dfrac{25 + 5 + 10\sqrt{5} + 25}{4} = \dfrac{55 + 10\sqrt{5}}{4}
	\]
	
	Donc
	\[
	EH = \dfrac{\sqrt{55 + 10\sqrt{5}}}{2}
	\]
	
	Finalement :
	\[
	A(BDE) = \dfrac{BD \times EH}{2} = \dfrac{6\sqrt{2} \times \dfrac{\sqrt{55 + 10\sqrt{5}}}{2}}{2} = 3\sqrt{2} \times \dfrac{\sqrt{55 + 10\sqrt{5}}}{2} \approx 18,7 \text{ unités d’aire}
	\]
	
	\subsection*{5. Montrer que $\overrightarrow{DB} \cdot \overrightarrow{DE} = 42 + 6\sqrt{5}$.}
	
	On admet que $\|\overrightarrow{DE}\| = \sqrt{42 + 12\sqrt{5}}$ ; en déduire la mesure de l’angle $\widehat{BDE}$ au degré près.
	
	On a
	\[
	\overrightarrow{DB} \cdot \overrightarrow{DE} = \overrightarrow{DB} \cdot \overrightarrow{DH}
	\]
	
	Avec
	\[
	\overrightarrow{DH} = \begin{pmatrix}
		\dfrac{1 + \sqrt{5}}{2} + 3 \\
		\dfrac{1 + \sqrt{5}}{2} + 3
	\end{pmatrix} = \begin{pmatrix}
		\dfrac{7 + \sqrt{5}}{2} \\
		\dfrac{7 + \sqrt{5}}{2}
	\end{pmatrix}
	\]
	
	D'où on calcule
	\[
	DH^2 = \left( \dfrac{7 + \sqrt{5}}{2} \right)^2 + \left( \dfrac{7 + \sqrt{5}}{2} \right)^2 = \dfrac{49 + 5 + 14\sqrt{5} + 49 + 5 + 14\sqrt{5}}{4} = \dfrac{108 + 28\sqrt{5}}{4}
	\]
	
	Donc
	\[
	DH = \left( \dfrac{7 + \sqrt{5}}{2} \right) \times \sqrt{2}
	\]
	
	Finalement :
	\[
	\overrightarrow{DB} \cdot \overrightarrow{DE} = 6\sqrt{2} \times \left( \dfrac{7 + \sqrt{5}}{2} \right) \times \sqrt{2} = 6 \times (7 + \sqrt{5}) = 42 + 6\sqrt{5}
	\]
	
	On sait que le produit scalaire peut aussi s’écrire :
	\[
	\overrightarrow{DB} \cdot \overrightarrow{DE} = \|\overrightarrow{DB}\| \times \|\overrightarrow{DE}\| \times \cos(\widehat{BDE})
	\]
	
	ou
	\[
	42 + 6\sqrt{5} = 6\sqrt{2} \times \sqrt{42 + 12\sqrt{5}} \times \cos(\widehat{BDE}) \iff \cos(\widehat{BDE}) = \dfrac{42 + 6\sqrt{5}}{6\sqrt{2} \times \sqrt{42 + 12\sqrt{5}}} \approx 0,787
	\]
	
	La calculatrice donne
	\[
	\widehat{BDE} \approx 38,1^\circ, \text{ soit 38° au degré près}
	\]
	
