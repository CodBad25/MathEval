	

	
	\subsection*{1.}
	
	Diminuer de 2 \% revient à multiplier par \(1 - 0,02 = 0,98\).
	
	\[
	120 \times 0,98 = 117,6 \, \text{kg} \quad \text{et} \quad 117,6 \times 0,98 = 115,248 \, \text{kg}.
	\]
	
	\subsection*{2.}
	\subsubsection*{a.}
	Diminuer de 2 \% revient à multiplier par \(1 - 0,02 = 0,98\).
	
	On a donc pour tout \(n \in \mathbb{N}\), \(a_{n+1} = 0,98a_n\) : la suite \((a_n)\) est une suite géométrique de premier terme \(a_0 = 120\) et de raison 0,98.
	
	\subsubsection*{b.}
	
	On sait alors que quel que soit le naturel \(n\),
	\[
	a_n = a_0 \times 0,98^n = 120 \times 0,98^n.
	\]
	
	\subsubsection*{c.}
	
	On rappelle que :
	
	Soit \((a_n)_{n \in \mathbb{N}}\) une suite géométrique de raison \(q\), \(q \neq 1\). La somme \(S\) de termes consécutifs est égale à
	\[
	S = u_1 + u_2 + \ldots + u_n = u_1 \times \dfrac{1 - q^n}{1 - q}.
	\]
	
	En 2020, la famille A produira :
	\[
	S_{2020} = a_1 + a_2 + \ldots + a_{12} = 120 \times 0,98 + 120 \times 0,98^2 + \ldots + 120 \times 0,98^{12}.
	\]
	
	Or
	\[
	0,98S_{2020} = 120 \times 0,98^2 + \ldots + 120 \times 0,98^{12} + 120 \times 0,98^{13}.
	\]
	
	Par différence entre les deux lignes précédentes :
	\[
	0,02S_{2020} = 120 \times 0,98 - 120 \times 0,98^{13},
	\]
	
	ou encore :
	\[
	0,02S_{2020} = 120 \times 0,98(1 - 0,98^{12}),
	\]
	
	et
	\[
	S_{2020} = \dfrac{120 \times 0,98(1 - 0,98^{12})}{0,02} = 49 \times 120 (1 - 0,98^{12}) \approx 1265,8.
	\]
	
	Donc \(S_{2020} \approx 1266 \, \text{kg}\) de déchets en 2020.
	
	\subsubsection*{d.}
	

	
	S(6) donne la somme \(a_1 + a_2 + \ldots + a_6\), soit la somme des déchets de la famille du premier semestre 2020.
	
