
\medskip

On appelle pourcentage de compression d'une image, le pourcentage de réduction de sa taille en ko (kilo-octets) après compression.

Une image a une taille initiale de 800 ko. Après une première compression, sa taille est de 664 ko.

\medskip

\begin{enumerate}
\item Calculer le pourcentage de réduction associé à cette première compression.
\end{enumerate}

Dans la suite de l'exercice, on fixe le pourcentage de réduction à 17\,\%. On effectue $n$ compressions successives.

Pour tout entier naturel $n$, on note $t_n$ la taille de l'image en ko après $n$ compressions.

On a donc $t_0 = 800$.
\begin{enumerate}[resume]
\item  Pour tout entier naturel $n$, exprimer $t_{n+1}$ en fonction de $t_n$ et en déduire la nature de la suite $\left(t_n\right)$.
\item  Pour tout entier naturel $n$, exprimer $t_n$  en fonction de $n$.

Afin de déterminer le nombre minimal $n$ de compressions successives à effectuer pour que cette image ait une taille finale inférieure à $50$ ko, on considère la fonction Python suivante:

\begin{center}
\fbox{
\begin{tabularx}{0.45\linewidth}{X}%\hline
\texttt{def nombreCompressions(A):} \\
\qquad \texttt{t = 800}\\
\qquad \texttt{n = 0}\\
\qquad \texttt{While t > A:}\\
\qquad \qquad \texttt{t = t*0,83}\\
\qquad \qquad \texttt{n = n+1} \\
\qquad \texttt{return n}\\% \hline
\end{tabularx}
}
\end{center}

\item Préciser, en justifiant, le nombre A de sorte que l'appel \texttt{nombreCompressions(A)} renvoie le nombre de compressions successives à effectuer que l'on cherche à déterminer.
\item Quel est le nombre minimal de compressions successives à effectuer pour que ce fichier ait une taille finale inférieure à $50$ ko ?
\end{enumerate}

\bigskip

