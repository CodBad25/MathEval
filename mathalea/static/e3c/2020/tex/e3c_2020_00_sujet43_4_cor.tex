
\subsection*{1.}

\paragraph{a.} Avec des notations évidentes :
\[
p(V) = \dfrac{60}{100} = 0{,}60 ; \quad p(B) = \dfrac{30}{100} = 0{,}30 \quad \text{et} \quad p(R) = \dfrac{10}{100} = 0{,}10.
\]
Dans ces trois cas, on « gagne » respectivement : \( -2 + 0 = -2 \), \( -2 + 4 = 2 \) et \( -2 + 8 = 6 \) jetons. D'où le tableau de probabilité de la variable \( X \) :

\[
\begin{array}{|c|c|c|c|}
\hline
\text{Valeurs } a \text{ prises par } X & -2 & 2 & 6 \\
\hline
p(X = a) & 0{,}6 & 0{,}3 & 0{,}1 \\
\hline
\end{array}
\]

\paragraph{b.} On a :
\[
E(X) = -2 \times 0{,}6 + 2 \times 0{,}3 + 6 \times 0{,}1 = -1{,}2 + 0{,}6 + 0{,}6 = -1{,}2 + 1{,}2 = 0,
\]
le jeu est équitable.

\paragraph{c.} On a :
\[
V(X) = 0{,}6 \times (-2)^2 + 0{,}3 \times 2^2 + 0{,}1 \times 6^2 = 2{,}4 + 1{,}2 + 3{,}6 = 7{,}2,
\]
\[
\sigma(X) = \sqrt{7{,}2} \approx 2{,}68.
\]

\subsection*{2.}

Si on ajoute \( n \) billes vertes, il y aura \( 100 + n \) billes.

Les probabilités deviendront :
\[
p(R) = \dfrac{10}{100 + n}, \quad p(B) = \dfrac{30}{100 + n} \quad \text{et} \quad p(V) = \dfrac{60 + n}{100 + n}.
\]
L'espérance devient :
\[
E(X) = -2 \times \dfrac{(60 + n)}{100 + n} + 2 \times \dfrac{30}{100 + n} + 6 \times \dfrac{10}{100 + n} = -1,
\]
d'où :
\[
-120 - 2n + 60 + 60 = -(100 + n).
\]
En résolvant, on trouve \( n = 100 \).

