	
	
	\subsection*{Question 1}
	
	
\begin{align*}
	f(x)& = 0,5(x - 2)^2 - 8\\
	& = 0,5(x - 2)^2 - 0,5 \times 16 = 0,5 \left[ (x - 2)^2 - 16 \right] \\
	&= 0,5(x - 2 + 4)(x - 2 - 4)\\
	& = 0,5(x + 2)(x - 6)
\end{align*}
La réponse correcte est \textbf{c.}
	
	\subsection*{Question 2}
On sait que $u_n = u_0 + nr$, donc en particulier :\\
$u_{10} = u_0 + 10 \times 0,5$, donc $-4 = u_0 + 5 \iff u_0 = -9$.\\
 Alors $u_2 = u_0 + 2r = -9 + 1 = -8$.\\
La réponse correcte est \textbf{d.}
	
	\subsection*{Question 3}
	
Pour $x \neq -2$, $f$ est dérivable et
\begin{align*}
	f'(x) &= \dfrac{2(x + 2) - (2x - 1)}{(x + 2)^2}\\
	& = \dfrac{2x + 4 - 2x + 1}{(x + 2)^2} \\
	&= \dfrac{5}{(x + 2)^2}
\end{align*}
	
	La réponse correcte est \textbf{c.}
	
	\subsection*{Question 4}
Une équation de la droite $\Delta$ est $2x + y + c = 0$ \\
et comme $A(-1 ; 3) \in \Delta \iff -2 + 3 + c = 0 \iff c = -1$, \\
une équation de $\Delta$ est $2x + y - 1 = 0$. (ou $-2x - y + 1 = 0$)\\
La réponse correcte est \textbf{d.}
	
	\subsection*{Question 5}
$M(x ; y) \in C \iff AM^2 = 3^2 \iff (x - 2)^2 + (y - 4)^2 = 9 \iff x^2 + y^2 - 4x - 8y + 11 = 0$.\\La réponse correcte est \textbf{c.}
	
