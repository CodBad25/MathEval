
\medskip

Dans le plan muni d'un repère orthonormé, on considère le point A de coordonnées (3~;~1) ainsi que la droite $(d)$ d'équation cartésienne $x - 3y - 4 = 0$.

\medskip

\begin{enumerate}
\item Déterminer les coordonnées du point B d'abscisse 7 appartenant à la droite $(d)$.
\item Donner un vecteur normal à la droite $(d)$.
\item Déterminer une équation de la droite $(\Delta)$ perpendiculaire à la droite $(d)$ passant par le point A.
\item Calculer les coordonnées du projeté orthogonal H du point A sur la droite $(d)$.
\item Calculer la distance AH et en donner une interprétation.
\end{enumerate}
