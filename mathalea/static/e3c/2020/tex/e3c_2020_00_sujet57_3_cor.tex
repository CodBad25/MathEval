
\subsection*{1.}

Ajouter 5 \% à un nombre, c'est le multiplier par \(1 + \dfrac{5}{100} = 1 + 0{,}05= 1{,}05\).

Donc :
\begin{align*}
&u_1 = 5000 \times 1{,}05 = 5250, \\
&u_2 = u_1 \times 1{,}05 = 5250 \times 1{,}05 = 5512{,}50 \text{ (€)}.
\end{align*}

\subsection*{2.}

On a vu que, quel que soit \(n \in \mathbb{N}\), \(u_{n+1} = 1{,}05u_n\).

\subsection*{3.}

La relation précédente montre que la suite \( (u_n) \) est une suite géométrique de raison \(q = 1{,}05\) et de premier terme \(u_0 = 5000\).

\subsection*{4.}

On sait que, quel que soit \(n \in \mathbb{N}\), \(u_n = 5000 \times 1{,}05^n\).

\subsection*{5.}

Au bout de 15 ans, la somme jusque-là immobilisée sera égale à :
\[
u_{15} = 5000 \times 1{,}05^{15} \approx 10394{,}64 \text{ (€)}.
\]
Le capital initial aura plus que doublé.

