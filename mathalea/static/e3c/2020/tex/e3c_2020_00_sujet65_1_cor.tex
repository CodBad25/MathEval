
\subsection*{1.}

\(\textbf{Affirmation 1 :}\)

On a \( \overrightarrow{AB} \begin{pmatrix} 2 \\ 2 \end{pmatrix} \) et \( \overrightarrow{CD} \begin{pmatrix} -7 \\ 6 \end{pmatrix} \), donc :
\[
\overrightarrow{AB} \cdot \overrightarrow{CD} = 2 \times (-7) + 2 \times 6 = -14 + 12 = -2 \neq 0,
\]
le produit scalaire n'est pas nul. Les vecteurs ne sont pas orthogonaux, les droites \((AB)\) et \((CD)\) ne sont pas perpendiculaires.

\(\textbf{Affirmation 2 :}\)

Le point \(E(3 \,;\, -2)\) appartient à la droite d'équation \(y = x - 5\).

Or \( \overrightarrow{CE} \begin{pmatrix} 3 \\ 3 \end{pmatrix} \) et :
\[
\overrightarrow{CE} \cdot \overrightarrow{AB} = 2 \times 3 + 2 \times 3 = 12 \neq 0.
\]
La droite d'équation \(y = x - 5\) contient le point \(C\) mais n'est pas perpendiculaire à la droite \((AB)\).

\(\textbf{Affirmation 3 :}\)

On a :
\[
AB^2 = (4 - 2)^2 + (0 - (-2))^2 = 4 + 4 = 8.
\]
Une équation du cercle de centre \(A\) passant par \(B\) est donc :
\[
(x - 2)^2 + (y - (-2))^2 = 8 \iff (x - 2)^2 + (y + 2)^2 = 8.
\]
Affirmation vraie.

\subsection*{2.}

\(\textbf{Affirmation 4 :}\)

On dérive \(f\) comme quotient de fonctions dérivables, puisque \(x\) est non nul :
\[
f'(x) = \dfrac{\e^x \times x - 1 \times \e^x}{x^2} = \dfrac{\e^x (x - 1)}{x^2}.
\]
D'où \(f'(1) = \dfrac{\e^1 (1 - 1)}{1^2} = \dfrac{0}{1} = 0\).

\subsection*{3.}

\(\textbf{Affirmation 5 :}\)

\(\dfrac{2\pi}{5}\) radians correspondent à \(\dfrac{2 \times 180}{5} = 72\) en degré.

On est donc dans le premier quadrant : le cosinus et le sinus sont tous les deux positifs : l'affirmation est fausse.

