
\subsection*{1.}

Avec \(n = 0\), on calcule : \(u_1 = 0{,}98 \times 800 = 784\) (habitants).

\subsection*{2.}

La relation \(u_{n+1} = 0{,}98 u_n\), vraie pour tout entier naturel \(n\), montre que la suite \((u_n)\) est géométrique de raison \(0{,}98\) et de premier terme \(u_0 = 800\).

\subsection*{3.}

On sait que pour tout entier naturel \(n\), \(u_n = 800 \times 0{,}98^n\).

\subsection*{4.}

2025 correspond à \(n = 4\), d'où :
\[
u_5 = 800 \times 0{,}98^4 \approx 723{,}1,
\]
soit 723 habitants à l'unité près.

\subsection*{5.}

\begin{python}
def u(n) :	
	u = 800
	for i in range(1,n) :
		u = u * 0.98
	return u
\end{python}

