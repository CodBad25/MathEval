
\medskip
On considère la fonction $f$ définie et dérivable sur l'intervalle $[0~;~ +\infty[$ par $f(x) =x^3 -x^2 -x-1$.

\medskip

\begin{enumerate}
\item On note $f'$ la fonction dérivée de $f$.
	\begin{enumerate}
		\item Montrer que, pour tout réel $x$,\: $f'(x) = 3\left(x + \dfrac{1}{3}\right)(x - 1)$
		\item En déduire le tableau de variation de $f$ sur $[0~;~ +\infty[$.
		\item Déterminer l'abscisse du point de la courbe représentative de $f$ pour lequel le
coefficient directeur de la tangente vaut $7$.
	\end{enumerate}
\item On note $x_0$ l'unique solution de l'équation $f(x) = 0$. On admet que $x_0 \in [1~;~2]$.

 On considère la fonction suivante définie en langage Python.
 
 \begin{center}
 \begin{tabularx}{0.55\linewidth}{|c X|}\hline
1& \texttt{{\blue def} zero\_de\_f(n) :}\\
2& \texttt{a = 1}\\
3& \texttt{b = 2}\\
4&\texttt{{\blue for} k {\blue in range}(n) :}\\
5&\quad\texttt{x = (a + b)/2}\\
6&\quad\texttt{{\blue if} x$**$3 - x$**$2 - x - 1 < 0 :}\\
7&\qquad\texttt{a = x}\\
8&\quad\texttt{{\blue else} :}\\ 
9&\qquad\texttt{b = x}\\
10&\texttt{{\blue return} a, b}\\ \hline
\end{tabularx}
\end{center}

	\begin{enumerate}
		\item On applique cette fonction pour $n = 3$. Reproduire sur la copie et compléter le tableau suivant, jusqu'à l'arrêt de l'algorithme.
		
 \begin{center}
 \begin{tabularx}{\linewidth}{|*{6}{>{\centering \arraybackslash}X|}}\hline
Itération	&$x = \dfrac{a+b}{2}$	&$f(x)< 0$ ?&$a$&$b$&Amplitude de $[a~;~b]$\\ \hline
$k = 0$		&1,5					&OUI		&1,5& 2	&0,5 \\ \hline
$k = 1$		&						&			&	&	&\\ \hline
$k = 2$		&						&			&	&	&\\ \hline
\end{tabularx}
\end{center}
		\item En déduire un encadrement de $x_0$, d'amplitude $0,125$, par deux nombres décimaux.
	\end{enumerate}
\end{enumerate}
