	\section*{Exercice 4 (5 points)}
	
	\begin{enumerate}
\item 


	\definecolor{ffqqqq}{rgb}{1,0,0}
		\definecolor{uququq}{rgb}{0.25,0.25,0.25}
		\definecolor{zzttqq}{rgb}{0.6,0.2,0}
		\definecolor{qqqqff}{rgb}{0,0,1}
		\definecolor{cqcqcq}{rgb}{0.75,0.75,0.75}
	\begin{center}
			\begin{tikzpicture}[line cap=round,line join=round,>=triangle 45,x=1.0cm,y=1.0cm,scale=0.8]
			\draw [color=cqcqcq,dash pattern=on 2pt off 2pt, xstep=1.0cm,ystep=1.0cm] (-4,-5.24) grid (8,6);
			\draw[->,color=black] (-4,0) -- (8,0);
			\foreach \x in {-4,-3,-2,-1,1,2,3,4,5,6,7}
			\draw[shift={(\x,0)},color=black] (0pt,2pt) -- (0pt,-2pt) node[below] {\footnotesize $\x$};
			\draw[->,color=black] (0,-5.24) -- (0,6);
			\foreach \y in {-5,-4,-3,-2,-1,1,2,3,4,5}
			\draw[shift={(0,\y)},color=black] (2pt,0pt) -- (-2pt,0pt) node[left] {\footnotesize $\y$};
			\draw[color=black] (0pt,-10pt) node[right] {\footnotesize $0$};
			\clip(-4,-5.24) rectangle (8,6);
			\fill[color=zzttqq,fill=zzttqq,fill opacity=0.1] (-3,1) -- (3,5) -- (7,1) -- cycle;
			\draw [color=zzttqq] (-3,1)-- (3,5);
			\draw [color=zzttqq] (3,5)-- (7,1);
			\draw [color=zzttqq] (7,1)-- (-3,1);
			\draw [dash pattern=on 2pt off 2pt] (2,-5.24) -- (2,6);
			\draw [dash pattern=on 2pt off 2pt,domain=-4:8] plot(\x,{(-12--6*\x)/-4});
			\draw [dash pattern=on 2pt off 2pt,domain=-4:8] plot(\x,{(-8--4*\x)/4});
			\draw [line width=1.2pt,color=ffqqqq] (2,0) circle (5.1cm);
			\begin{scriptsize}
				\fill [color=qqqqff] (-3,1) circle (1.5pt);
				\draw[color=qqqqff] (-3.3,1.27) node {$A$};
				\fill [color=qqqqff] (3,5) circle (1.5pt);
				\draw[color=qqqqff] (3.1,5.27) node {$B$};
				\fill [color=qqqqff] (7,1) circle (1.5pt);
				\draw[color=qqqqff] (7.3,1.27) node {$C$};
				\fill [color=qqqqff] (2,0) circle (1.5pt);
				\draw[color=qqqqff] (2.12,-0.71) node {$I$};
				\fill [color=qqqqff] (0,3) circle (1.5pt);
				\draw[color=qqqqff] (0.22,3.3) node {$C'$};
					\fill [color=qqqqff] (0,3) circle (1.5pt);
				\draw[color=qqqqff] (4,-2) node {$(\Delta)$};
					\fill [color=qqqqff] (5,3) circle (1.5pt);
				\draw[color=qqqqff] (5.12,3.4) node {$A'$};
				\fill [color=qqqqff] (0,3) circle (1.5pt);
				\draw[color=qqqqff] (6.5,4) node {$(\Delta^\prime)$};
					\fill [color=qqqqff] (2,1) circle (1.5pt);
				\draw[color=qqqqff] (2.32,1.4) node {$B'$};
			\end{scriptsize}
		\end{tikzpicture}
	\end{center}
		
	\item 	 Soit \(C'\) le milieu du segment \([AB]\) : \\
		 Ses coordonnées sont la moyenne de celles des extrémités de celles de $A$ et $B$ donc \(C'(0 ; 3)\).\\
		  Les coordonnées du vecteur \(\overrightarrow{AB}\) sont \(\overrightarrow{AB}(6 ; 4)\).\\
		   Si \((\Delta)\) est la médiatrice de \([AB]\), on a :		
	\begin{align*}
	M(x ; y) \in \Delta\\
	 \iff& \overrightarrow{C'M} \cdot \overrightarrow{AB} = 6(x - 0) + 4(y - 3) = 0 \\
	 \iff& 6x + 4y - 12 = 0\\
	  \iff& 3x + 2y - 6 = 0.
	\end{align*}
	
		
		
	\item  Ses coordonnées sont la moyenne de celles des extrémités de celles de $A$ et $C$  donc \(B'(2 ; 1)\)
		
		\item Soit \(A'\) le milieu de \([BC]\) :  \(A'(5 ; 3)\) et \(\overrightarrow{BC}(4 ; -4)\).\\
		 Si \((\Delta')\) est la médiatrice de \([BC]\), alors
	\begin{align*}
	&	M(x ; y) \in \Delta'\\
	 \iff &\overrightarrow{A'M} \cdot \overrightarrow{BC}\\
	  \iff& 4(x - 5) - 4(y - 3) = 0 \\
	  \iff& 4x - 20 - 4y + 12 = 0\\
	   \iff& 4x - 4y - 8 = 0 \\
	   \iff& x - y - 2 = 0.
\end{align*}
		
		Si \(I\) est le centre du cercle circonscrit au triangle \(ABC\), on sait que c’est le point commun aux trois médiatrices ; donc ses coordonnées vérifient les équations de la médiatrice de \([AB]\) et de la médiatrice de \([BC]\). \\
		Il faut donc résoudre le système :
		
		\[
		\begin{cases}
			3x + 2y - 6 = 0 \\
			x - y - 2 = 0
		\end{cases}
		\iff
		\begin{cases}
			3x + 2y - 6 = 0 \\
			x - 2 = y
		\end{cases}
		\iff
		\begin{cases}
			3x + 2(x - 2) - 6 = 0 \\
			x - 2 = y
		\end{cases}
		\iff
		\begin{cases}
			3x + 2x - 4 - 6 = 0 \\
			x - 2 = y
		\end{cases}
		\iff
		\begin{cases}
			5x = 10 \\
			x - 2 = y
		\end{cases}
		\iff
		\begin{cases}
			x = 2 \\
			y = 0
		\end{cases}.
		\]
		
		Les coordonnées de \(I\) sont \((2 ; 0)\).
		
		\item Le rayon est par exemple \(IA\).
		
		\[
		IA^2 = (-3 - 2)^2 + (1 - 0)^2 = 25 + 1 = 26. \quad Donc \quad IA = \sqrt{26}.
		\]
	\end{enumerate}
	

	
