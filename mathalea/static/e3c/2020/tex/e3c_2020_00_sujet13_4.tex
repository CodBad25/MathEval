
\medskip

\begin{enumerate}
\item Étudier le signe de la fonction $P$ définie sur $\R$ par $P(x) = x^2 + 4x + 3$.


\end{enumerate}

On considère la fonction $f$ définie sur l'intervalle $]- 2~;~ +\infty[$ par

\[f(x) = \dfrac{x^2 +x - 1}{x + 2}\]

et on note $\mathcal{C}_f$ sa courbe représentative dans un repère orthogonal du plan. On admet que la fonction $f$ est dérivable sur l'intervalle $]- 2~;~ +\infty[$. 

\medskip

\begin{enumerate}[resume]
\item Montrer que pour tout réel $x$ de l'intervalle $]- 2~;~ +\infty[$,

\[f'(x) = \dfrac{P(x)}{(x + 2 )^2}\]

où $f'$ est la fonction dérivée de $f$.

\item Étudier le signe de $f'(x)$ sur $]- 2~;~ +\infty[$ et construire le tableau de variations de la fonction $f$ sur $]- 2~;~ +\infty[$.
\item Donner le minimum de la fonction $f$ sur $]- 2~;~ +\infty[$ et la valeur pour laquelle il est atteint (on donnera les valeurs exactes).
\item Déterminer le coefficient directeur de la tangente $T$ à la courbe $\mathcal{C}_f$ au point d'abscisse $2$.
\end{enumerate}
