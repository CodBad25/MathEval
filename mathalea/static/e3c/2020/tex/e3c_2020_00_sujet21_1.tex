
\medskip

Cet exercice est un questionnaire à choix multiples (QCM). 

Les cinq questions sont indépendantes.

Pour chacune des questions, une seule des quatre réponses est exacte.

Le candidat indiquera sur sa copie le numéro de la question et la lettre correspondant à la
réponse exacte.

\textbf{Aucune justification n’est demandée.}

Une réponse exacte rapporte un point, une réponse fausse ou une absence de réponse ne rapporte ni n’enlève aucun point.

\medskip

\textbf{Question 1}

\medskip 


$\left(u_n\right)$ est la suite arithmétique telle que $u_4 = 3$ et $u_{10} = 18$. On peut affirmer que :

\medskip

\begin{tabularx}{\linewidth}{*{4}{X}}
\textbf{a.~~} $ u_0 = 7$ &\textbf{b.~~}$u_7 = 20,5$&\textbf{c.~~}$u_{12} = 23$& \textbf{d.~~}$u_{14} = -28$.
\end{tabularx}

\medskip

\textbf{Question 2}

\medskip 

$2 + 3 + 4 + \dots + 999 + \np{1000}$ est égal à :

\medskip

\begin{tabularx}{\linewidth}{*{4}{X}}
\textbf{a.~~} $\np{500500}$ &\textbf{b.~~} $\np{498999}$&\textbf{c.~~}$\np{499000}$& \textbf{d.~~} $\np{500499}$.
\end{tabularx}

\medskip

\textbf{Question 3}

\medskip


$\left(v_n\right)$ est la suite géométrique de raison $0,3$ telle que $v_0 = -3$.

On conjecture que la suite $\left(v_n\right)$ a pour limite : 

\medskip

\begin{tabularx}{\linewidth}{*{4}{X}}
\textbf{a.~~} $0$ &\textbf{b.~~} $+\infty$&\textbf{c.~~}$-\infty$& \textbf{d.~~} $-3$.
\end{tabularx}

\medskip

\textbf{Question 4}

\medskip 

$f$ est la fonction définie sur $\R$ par $f(x) = -2(x + 2)^2 - 3$. On peut affirmer qu’elle est :

\medskip

\begin{tabularx}{\linewidth}{*{2}{X}}
\textbf{a.~~}décroissante sur $]-\infty~;~+\infty[$&\textbf{c.~~}croissante sur $]-\infty~;~2[$\\
\textbf{b.~~}décroissante sur $]-2~;~+\infty[$&\textbf{d.~~}décroissante sur $]-3~;~+\infty[$ .\\
\end{tabularx}

\medskip

\textbf{Question  5}

\medskip 

L’ensemble des solutions de l’inéquation $x^2 -5x + 6 < 0 $ est

\medskip

\begin{tabularx}{\linewidth}{*{4}{X}}
\textbf{a.~~} $ ]-\infty~;~2[ \cup ]3~;~+\infty[$ &\textbf{b.~~}$]-\infty~;~-1[ \cup ]6~;~+\infty[ $ &\textbf{c.~~}$]2~;~3[$& \textbf{d.~~}$ ]-1~;~6[$ .
\end{tabularx}

\vspace{0,5cm}

