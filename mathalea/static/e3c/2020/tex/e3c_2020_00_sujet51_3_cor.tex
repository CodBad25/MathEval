
\subsection*{1.}

Augmenter de 3 \% c'est multiplier par \(1 + \dfrac{3}{100} = 1 + 0{,}03 = 1{,}03\).

Donc on saisit dans B3 : \texttt{=B2*1,03}.

\subsection*{2.}

\paragraph{a.}
\begin{itemize}
    \item La relation \( u_{n+1} = 1{,}03 u_n \) quel que soit le naturel \(n \geqslant 1\) : ceci montre que la suite \((u_n)\) est une suite géométrique de raison \(1{,}03\) et de premier terme \(u_1 = 1000\).
    \item On a \( v_{n+1} = v_n + 40 \) pour \(n \geqslant 1\) : la suite \((v_n)\) est donc une suite arithmétique de raison \(40\) et de premier terme \(v_1 = 1000\).
\end{itemize}

\paragraph{b.} On a pour \(n \geqslant 1\) :
\[
v_n = 1000 + (n - 1) \times 40 \quad \text{ou} \quad v_n = 1000 + 40n - 40 = 960 + 40n.
\]

\paragraph{c.} On sait que pour \(n \geqslant 1\) :
\[
u_n = 1000 \times 1{,}03^{n-1}.
\]

\subsection*{3.}

On voit qu'à partir de la 21e semaine, \(w_{21} < 0\), soit \(v_{21} - u_{21} < 0\) ou \(v_{21} < u_{21}\).

