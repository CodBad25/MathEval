
\medskip

\emph{Ce QCM comprend 5 questions indépendantes.}

\emph{Pour chacune d'elles, une seule des réponses proposées est exacte.}

\emph{Indiquer pour chaque question sur la copie la lettre correspondant à la réponse
choisie. Aucune justification n'est demandée.}

\emph{Chaque réponse correcte rapporte 1 point. Une réponse incorrecte ou une absence de
réponse n'apporte, ni ne retire de point.}

\medskip

\textbf{Question 1}

\medskip

Dans un repère orthonormé, on considère la parabole $\mathcal{P}$ d'équation $y = 2x^2 + 4x - 11$, de
sommet S et d'axe de symétrie la droite $\mathcal{D}$. Quelle est la bonne proposition ?

\begin{description}
\item[A.~~] S$(-4~;~5)$ et $\mathcal{D}$ a pour équation $y = 5$.
\item [B.~~] $S (-1~;~-17)$ et $\mathcal{D}$ a pour équation $x = -1$.
\item[C.~~] $S (-1~;~-13)$ et $\mathcal{D}$ a pour équation $x = -1$.
\item [D~~.] $S (-1~;~-13)$ et $\mathcal{D}$ a pour équation $y = -1$.
\end{description}

\medskip

\textbf{Question 2}

\medskip

Une expérience aléatoire met en jeu des évènements $A$ et $B$ et leurs évènements contraires
$\overline{A}$ et $\overline{B}$. L'arbre pondéré ci-dessous traduit certaines données de cette expérience aléatoire.

On a alors :
\begin{multicols}{2}
\begin{description}
\item[A.~~]$ p(B) = 0,5$
\item[B.~~]$p(A \cap B) = 0,9$
\item[C.~~]$p_A(B)  = 0,18$
\item[D.~~]$p_B(A) =  \dfrac{9}{13}$
\end{description}
\begin{minipage}[t]{4.5cm}
\psset{nodesepA=0pt,nodesepB=3pt,treesep=0.75,labelsep=0.1pt,levelsep=2.5cm}
\pstree[treemode=R]{\TR{}}
{\pstree{\TR{$A$~}\taput{$\np{0,6}$}}
	{
	\TR{$B$}\taput{$\np{0.3}$}
	\TR{$\overline{B}$}\tbput{$\np{0.7}$}
	}
\pstree{\TR{$\overline{A}$~}\tbput{$\np{0,4}$}}
	{\TR{$B$}\taput{$\np{0,2}$}
	\TR{$\overline{B}$}\tbput{$\np{0,8}$}
	}
}
\end{minipage}
\end{multicols}

\medskip

\textbf{Question 3}

\medskip

On considère le nombre réel $a = \dfrac{18\pi}{5}$.


Un des nombres réels suivants a le même point image que le nombre réel $a$ sur le cercle
trigonométrique. Lequel ?

\medskip

\begin{tabularx}{\linewidth}{*{4}{X}}
\textbf{A.~~} $\dfrac{3\pi}{5}$ &\textbf{B.~~} $\dfrac{63\pi}{5}$&\textbf{C.~~}$\dfrac{-12\pi}{5}$& \textbf{D.~~}$\dfrac{-3\pi}{5}$
\end{tabularx}


\medskip

\textbf{Question 4}

\medskip

On considère la fonction $f$ définie sur $\R$ par $f(x) = x\e^x$.

La fonction dérivée de la fonction $f$ est notée $f'$. On a alors :

\medskip

\begin{tabularx}{\linewidth}{*{4}{X}}
\textbf{A.~~} $f'(x) = \e^x$ &\textbf{B.~~} $f'(x) = (1 + x)\e ^x $&\textbf{C.~~}$f'(x) = x\e^x$& \textbf{D.~~} $f'(x) = 2x\e^x$.
\end{tabularx}

\medskip

\textbf{Question 5}

\medskip

Parmi les relations suivantes, quelle est celle qui permet de définir une suite géométrique de
terme général $u_n$ ?

\medskip

\begin{tabularx}{\linewidth}{*{4}{X}}
\textbf{A.~~} $u_n =\dfrac{u_{n-1}}{2}$ &\textbf{B.~~} $u_n = u_{n-1} + 2$&\textbf{C.~~}$ u_ n = 2u_{ n-1}^2$& \textbf{D.~~} $u_n = 2u_{n-1} + 10$.
\end{tabularx}

\vspace{1cm}

