 
\medskip

La bibliothèque municipale étant devenue trop petite, une commune a décidé d'ouvrir une médiathèque qui pourra contenir \np{100000} ouvrages au total. Pour l'ouverture prévue le 1\up{er} janvier 2020, la médiathèque dispose du stock de \np{35000} ouvrages de l'ancienne bibliothèque, augmenté de \np{7000} ouvrages supplémentaires neufs offerts par la commune.

\medskip

\textbf{Partie A}

\medskip

Chaque année, le bibliothécaire est chargée de supprimer 5\,\% des ouvrages, trop vieux ou abîmés, et d'acheter \np{6 000} ouvrages neufs.

On appelle $u_n$ le nombre, en milliers, d'ouvrages disponibles le 1\up{er} janvier de l'année (2020+$n$).

On donne $u_0 = 42$.

\medskip
 
\begin{enumerate}
\item  Justifier que, pour tout entier naturel n, on a $u_{n+1} = u_n\times  0,95+6$.
\item On propose ci-dessous un programme en langage Python :
\begin{python}
def suite(n) :
u=42
for i in range(n) :
u=0.95*u+6
return u
\end{python}

Expliquer ce que permet de déterminer ce programme.
\end{enumerate}

\medskip

\textbf{Partie B}

\medskip

La commune doit finalement revoir ses dépenses à la baisse, elle ne pourra financer que \np{4000} nouveaux ouvrages par an au lieu des \np{6000} prévus.

On appelle $v_n$ le nombre, en milliers, d'ouvrages disponibles le 1\up{er} janvier de l'année (2020+$n$).

\medskip

\begin{enumerate}
\item  On admet que $v_{n+1} = 0,95\times v_n + 4$ pour tout entier naturel $n\geqslant 0$ avec $v_0 = 42$.

On considère la suite $(w_n)$ définie, pour tout entier naturel $n$, par $w_n= v_n-80$.
	\begin{enumerate}
		\item Montrer que $\left(w_n\right)$ est une suite géométrique de raison $q = 0,95$ et préciser son premier terme $w_0$.
		\item En déduire l'expression de $w_n$ puis de $ v_n$ en fonction de $n$.

	\end{enumerate}
\item On donne ci-dessous un programme en langage Python.

\begin{center}
\begin{python}
def objet(A) :
v=42
n=0
while v<A :
v=0.95*v+4
n=n+1
return n 
\end{python}
\end{center}

L'appel à la fonction objet(70) renvoie 27. Interpréter ce résultat dans le contexte de l'exercice.
\end{enumerate}
