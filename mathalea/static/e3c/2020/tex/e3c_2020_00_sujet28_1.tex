
\medskip

Cet exercice est un QCM comportant 5 questions. 

Pour chacune des questions, une seule des quatre réponses proposées est correcte. 

Les questions sont indépendantes.

Pour chaque question, indiquer le numéro de la question et recopier sur la copie la lettre correspondante à la réponse choisie.

Aucune justification n'est demandée mais il peut être nécessaire d'effectuer des recherches au brouillon pour aider à déterminer votre réponse.

Chaque réponse correcte rapporte 1 point. Une réponse incorrecte ou une question sans réponse n'apporte ni ne retire de point.

\bigskip

Le plan est muni d'un repère orthonormé.

\medskip

\textbf{Question 1}

\medskip 

La droite $\mathcal{D}$ de vecteur directeur $\vv{u}\ \binom{-3}{1}$ passant par $A(-1~;~2)$ a pour équation :

\medskip

\begin{tabularx}{\linewidth}{*{4}{X}}
\textbf{a.~~} $-3x+y-5=0 $ &\textbf{b.~~} $x+3y-5=0 $&\textbf{c.~~} $x-3y-5=0 $& \textbf{d.~~} $3x+y+1=0  $.
\end{tabularx}

\medskip

\textbf{Question 2}

\medskip 

On considère la droite $\mathcal{D}$ d'équation $5x-8y+9=0$ Alors :

\medskip

\begin{tabularx}{\linewidth}{*{2}{X}}
\textbf{a.~~} $ A(6~;~7) $ appartient à $\mathcal{D}$ &\textbf{b.~~} $\vv{n}\ \binom{5}{8})$ est un vecteur normal à $\mathcal{D}$ \\
\textbf{c.~~}$\mathcal{D}$ coupe l'axe des ordonnées au point $B\ (0~;~1) $& \textbf{d.~~} $\mathcal{D}$ est parallèle à la droite $\mathcal{D}'$ d'équation 

$2,5x - 4y + 2 = 0$.
\end{tabularx}

\medskip

\textbf{Question 3}

\medskip 

\begin{minipage}[]{8cm}
On considère la fonction $f$ dont la représentation graphique $\mathcal{C}_f$ est donnée ci-contre. La droite $\mathcal{D}$ est la tangente à $\mathcal{C}_f$ au point A(1~;~1). Le point $B\ (0~;~-1)$ appartient à la droite $\mathcal{D}$. Le nombre dérivé $f'(1)$ est égal à :

\medskip

\begin{tabularx}{\linewidth}{*{4}{X}}
\textbf{a.~~} $1$&\textbf{b.~~} $\frac{1}{2}$&\textbf{c.~~}$2$& \textbf{d.~~} $-2$.
\end{tabularx}
\end{minipage}
\hfill
\begin{minipage}[]{5cm}

\psset{xunit=0.97cm,yunit=0.97cm,labelFontSize=\scriptstyle,showorigin=false}
\begin{pspicture}(-2.34,-1.7)(2.8,3.6)
\multido{\n=-2.2+0.2}{26}{\psline[linewidth=0.35pt,linecolor=lightgray](\n,-1.6)(\n,3.4)}
\multido{\n=-1.6+0.2}{26}{\psline[linewidth=0.35pt,linecolor=lightgray](-2.4,\n)(2.8,\n)}
%\multido{\n=-4+1}{7}{\psline[linewidth=0.45pt](\n,-2)(\n,2.4)}
%\multido{\n=-2+1}{5}{\psline[linewidth=0.45pt](-4.4,\n)(2.2,\n)}
\psaxes[linewidth=0.95pt]{->}(0,0)(-2.2,-1.6)(2.8,3.6)
\def\Func{x x mul }
\psplot[plotpoints=2000,linewidth=1.25pt,linecolor=orange]{-1.85}{1.85}{\Func}
\psplot[plotpoints=2000,linewidth=0.85pt]{-0.35}{2.15}{x 2 mul 1 sub}
\psdots[dotstyle=Mul,dotscale=1.6,linecolor=blue](1,1)(0,-1)
 \uput[r](1,1){A}\uput[r](0,-1){B}\uput[l](-1.6,2){\red $\mathcal{C}_f$}\uput[r](2,3.2){$\mathcal{D}$}
\end{pspicture}
\end{minipage}

\medskip

\textbf{Question 4}

\medskip 

On considère une fonction $f$ polynôme du second degré dont le tableau de signes est donné ci-après :
\begin{center}
\begin{tabular}[]{r|ccccccc}
$x$&$-\infty$&&$-1$&&2&& $+\infty$\\\hline
$f(x)$&&$-$ &0& + & 0 &$-$&\\
\end{tabular}
\end{center}

Une expression de $f(x)$ peut être :

\medskip

\begin{tabularx}{\linewidth}{*{4}{X}}
\textbf{a.~~} $2x^2+5x-2 $ &\textbf{b.~~} $-x^2+1 $&\textbf{c.~~}$-x^2+x+2 $& \textbf{d.~~} $x^2+x-2  $.
\end{tabularx}

\medskip

\textbf{Question 5}

\medskip 

On considère la fonction $f$ définie sur $\R$ par $f(x)= x\e^x$.
Alors la fonction dérivée de $f$, notée $f'$, est définie sur $\R$ par :

\medskip

\begin{tabularx}{\linewidth}{*{4}{X}}
\textbf{a.~~} $ f'(x)= \e^x$ &\textbf{b.~~} $f'(x)=(x+1)\e^x$&\textbf{c.~~}$f'(x)=\e $& \textbf{d.~~} $f'(x)=   x^2\e^x   $.
\end{tabularx}

\vspace{0,5cm}

