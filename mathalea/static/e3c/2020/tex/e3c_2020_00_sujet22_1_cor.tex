	
	\textbf{Question 1}
	
	Avec les points $B(2; 3)$ et $A(4; -1)$, on trouve que le coefficient directeur de la droite $D$, égal au nombre dérivé $f'(4)$, est donné par : 
	\begin{align*}
		f'(4) &= \dfrac{-1-3}{4-2} \\
		&= \dfrac{-4}{2} \\
		&= -2.
	\end{align*}
	
	\textbf{Question 2}
	
	On sait qu'une équation de la tangente à la courbe représentative de la fonction $f$ au point d'abscisse 1 est : 
	\[
	y  = f'(1)(x - 1)+f(1).
	\]
	Avec $f(1) = 1-2+1 = 0$ et $f'(x) = 3x^2 - 4x$, d'où $f'(1) = 3-4 = -1$, l'équation devient :
	\[
	y  = -1(x - 1) \quad \text{donc} \quad y = -x + 1.
	\]
	
	\textbf{Question 3}
	
	Pour tout réel $x$, 
	\begin{align*}
		\dfrac{e^x \times e^{-3x}}{e^{-x}} &= e^x \times e^{-3x} \times e^x \\
		&= e^{x - 3x + x} \\
		&= e^{-x}.
	\end{align*}
	
	\textbf{Question 4}
	
	La fonction est décroissante puis croissante. Le coefficient $a > 0$ et la courbe coupe l'axe des abscisses aux points d'abscisses respectives $-2$ et $1$. De plus, $f(0) = -4$. Donc, la réponse est $c$.
	
	\textbf{Question 5}
	
	Pour l'équation dans $\mathbb{R}$, $-x^2 - 2x + 8 = 0$, on calcule :
	\begin{align*}
		\Delta &= 4 + 4 \times 8 \\
		&= 4 \times (1 + 8) \\
		&= 4 \times 9 \\
		&= 36 > 0.
	\end{align*}
	
	L'équation a donc deux solutions :
	\begin{align*}
		x_1 &= \dfrac{2-6}{-2} = 2, \\
		x_2 &= \dfrac{2 + 6}{-2} = -4.
	\end{align*}
	Le signe du trinôme est celui de $a = -1 < 0$, donc la fonction est négative sauf sur l'intervalle $]-4; 2[$ où le trinôme est positif. La réponse est $b$.
	
