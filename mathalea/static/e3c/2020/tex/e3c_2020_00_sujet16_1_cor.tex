	
	\subsection*{Question 1}
	
	\(EFG\) est un triangle tel que \(EF = 8\), \(FG = 5\) et \(\widehat{EFG} = \dfrac{3\pi}{4}\).
	
	\[
	\overrightarrow{FE} \cdot \overrightarrow{FG} = FE \times FG \times \cos(\widehat{EFG} = 8 \times 5 \times \left( -\dfrac{\sqrt{2}}{2} \right) = -20 \sqrt{2}.
	\]
	
	\subsection*{Question 2}
	
	Le nombre dérivé \(f'(0)\) est le coefficient directeur de la tangente au point d’abscisse 0, soit \(-1\).
	
	\subsection*{Question 3}
	
	On se place dans un repère orthonormé. Une équation du cercle de centre \(B(2 ; 3)\) et de rayon 4 est :
	
	\[
	M(x ; y) \in (C) \iff BM^2 = 4^2 \iff (x - 2)^2 + (y - 3)^2 = 16.
	\]
	
	\subsection*{Question 4}
	
	Il y a deux points d’ordonnée \(-3\), l’un d’abscisse 0, l’autre d’abscisse 1.
	
	\subsection*{Question 5}
	
	Un vecteur directeur de la droite est \(\overrightarrow{u}(2 ; -3)\). Il est orthogonal au vecteur \(\overrightarrow{v}(3 ; 2)\) car
	
	\[
	\overrightarrow{u} \cdot \overrightarrow{v} = 6 - 6 = 0.
	\]
	
