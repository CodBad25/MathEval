
\subsection*{Question 1}

Le produit est nul si l'un des facteurs est nul :
\[
x - 1 = 0 \text{ si } x = 1,
\]
\[
x^2 + x + 1 = 0. \text{ On a } \Delta = 1 - 4 = -3 < 0,
\]
ce trinôme n'a donc pas de racines.

L'équation a donc une seule solution : \(1\).

\subsection*{Question 2}

Comme, quel que soit le réel \(x\), \(\e^x > 0\), on a \(\e^x + 1 > 1 > 0\), \(f(x)\) ne peut s'annuler que si : \[
7x - 23 = 0 \iff x = \dfrac{23}{7}.
\]

\subsection*{Question 3}

le cercle de centre \(A(-4\,;\,2)\) et de rayon \(r = \sqrt{2}\) a pour équation :
\begin{align*}
&M(x\,;\,y) \in \mathcal{C}(A\,;\,R = \sqrt{2}) \\
\iff &AM^2 = (\sqrt{2})^2 \\
\iff &(x + 4)^2 + (y - 2)^2 = 2.
\end{align*}

\subsection*{Question 4}
 
\(\vec{u}\) et \(\vec{v}\) sont orthogonaux si et seulement si leur produit scalaire est nul, soit si :
\begin{align*}
&\vec{u} \cdot \vec{v} = 0 \\
\iff &m(m + 1) - 2 = 0 \\
\iff &m^2 + m - 2 = 0.
\end{align*}
Pour le trinôme, \(\Delta = 1 + 8 = 9 = 3^2 > 0\), il a donc deux racines :
\[
m_1 = \dfrac{-1 + 3}{2} = 1 \quad \text{et} \quad m_2 = \dfrac{-1 - 3}{2} = -2.
\]
Réponse : \(\textbf{b.}\)

\subsection*{Question 5}

Une équation cartésienne de la droite \(\mathcal{D}\) passant par le point \(A(-2\,;\,5)\) et admettant pour vecteur normal \(\vec{n} \begin{pmatrix} -1 \\ 3 \end{pmatrix}\) est :
\begin{align*}
&M(x\,;\,y) \in \mathcal{D} \\
\iff &\overrightarrow{AM} \cdot \vec{n} = 0 \\
\iff &-1(x - (-2)) + 3(y - 5) = 0 \\
\iff &-x - 2 + 3y - 15 = 0 \\
\iff &-x + 3y - 17 = 0 \\
\iff &x - 3y + 17 = 0.
\end{align*}

