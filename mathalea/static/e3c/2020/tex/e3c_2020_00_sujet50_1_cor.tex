
\subsection*{Question 1}

\(\e^x \times \e^{x + 2} = \e^{x + x + 2} = \e^{2x + 2}\).

\subsection*{Question 2}

Une équation de la tangente au point \((1 \,;\, g(1))\) est :
\[
y - g(1) = g'(1)(x - 1) \quad \text{ou} \quad y = g'(1)(x - 1) + g(1).
\]

\subsection*{Question 3}

On sait qu'une équation de \((d)\) est \(7x - 4y + c = 0\).

Or :
\begin{align*}
&A(-2\,;\,3) \in (d) \\
\iff &-2 \times 7 - 4 \times 3 + c = 0 \\
\iff &-26 + c = 0 \\
\iff &c = 26.
\end{align*}

Une équation de \((d)\) est donc \(7x - 4y + 26 = 0\) ou \(-7x + 4y - 26 = 0\).

\subsection*{Question 4}

La fonction cosinus est périodique de période \(2\pi\), donc \(\cos(t + 4\pi) = \cos(t)\).

La fonction cosinus est paire donc \(\cos(-t) = \cos(t)\).

D'où :
\begin{align*}
\cos(t + 4\pi) + \cos(-t) &= \cos(t) + \cos(t) \\
&= 2\cos(t) \\
&= 2 \times \dfrac{2}{3} \\
&= \dfrac{4}{3}.
\end{align*}

\subsection*{Question 5}

\begin{align*}
y 1= -x^2 + 6x - 9 \\
1= -(x^2 - 6x + 9) \\
1= -(x - 3)^2,
\end{align*}
donc \(y = 0\) si et seulement si :
\[
(x - 3)^2 = 0 \quad \iff \quad x - 3 = 0 \quad \iff \quad x = 3.
\]

Le seul point commun à \((P)\) et à l'axe des abscisses est le point de coordonnées \((3 \,;\, 0)\).

