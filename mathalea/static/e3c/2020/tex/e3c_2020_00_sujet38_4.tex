
\medskip

\begin{enumerate}
\item Soit $f$ la fonction définie sur l'intervalle [0~;~2] par $f(x) = 8x - 2x^3$.
\begin{enumerate}
\item Montrer que pour tout réel $x$ de [0; 2], $f'(x)$ a le même signe que $4 - 3x^2$.
\item Étudier les variations de la fonction $f$ sur [0~;~2].
\end{enumerate}
\item Dans un repère orthonormal, on considère la parabole $p$ d'équation $y = x^2$ et la droite $\mathscr{D}$ d'équation $y = 4$.

On considère le rectangle $MSFE$ tel que :
\begin{itemize}
\item $M$ un point de $p$ dont l'abscisse $x$ est un réel de $]0~;~2[$.
\item $S$ est le symétrique de $M$ par rapport à l'axe des ordonnées.
\item $E$ et $F$ sont respectivement les projetés orthogonaux de $M$ et $S$ sur la droite $\mathscr{D}$.
\end{itemize}

\begin{center}



\psset{xunit=1cm,yunit=1cm,labelFontSize=\scriptstyle,showorigin=false}
\begin{pspicture}(-4.4,-1.5)(4.5,5.6)
\psframe[fillstyle=solid, fillcolor= blue](-1,1)(1,4)
\multido{\n=-4+0.2}{40}{\psline[linewidth=0.25pt,linecolor=lightgray](\n,-0.99)(\n,5.2)}
\multido{\n=-0.8+0.2}{30}{\psline[linewidth=0.25pt,linecolor=lightgray](-4,\n)(4,\n)}
\multido{\n=-4+1}{9}{\psline[linewidth=0.45pt](\n,-0.9)(\n,5.2)}
\multido{\n=0+1}{6}{\psline[linewidth=0.45pt](-4,\n)(4,\n)}
\psaxes[linewidth=0.95pt]{-}(0,0)(-4,-0.99)(4,5.2)
\def\Func{x x  mul  }
\psplot[plotpoints=2000,linewidth=0.85pt,linecolor=red]{-2.27}{2.27}{\Func}
\uput[dl](0,0){O}
\psdots[dotstyle=Bullet,dotscale=1.5,](1,1)(-1,1)(-1,4)(1,4)
\uput[dr](1,1){\footnotesize $M$}\uput[ur](1,4){\footnotesize $E$}\uput[ul](-1,4){\footnotesize $F$}\uput[dl](-1,1){\footnotesize $S$}
\psline[linewidth=1.25pt](-4,4)(4,4)\uput[u](3.8,4){$\mathscr{D}$}
\end{pspicture}
\end{center}

	\begin{enumerate}
		\item  Lorsque l'abscisse $x$ du point $M$ varie dans ]0~;~2[, l'aire du rectangle $MSFE$ est-elle constante ?
		\item Montrer que l'aire du rectangle $MSFE$ en fonction de l'abscisse $x$ de $M$ est $8x - 2x^3$.
		\item Montrer que l'aire maximale du rectangle $MSFE$ est $\dfrac{32}{3\sqrt{3}}$.
	\end{enumerate}
\end{enumerate}
