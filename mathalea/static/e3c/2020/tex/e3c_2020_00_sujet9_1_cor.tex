	\section*{Exercice 1 (5 points)}
	
\[
	\begin{array}{l}
		P(E) = P(A \cap E) + P(B \cap E) = 0,6 \times 0,5 + 0,4 \times 0,3 = 0,3 + 0,12 = 0,42
	\end{array}
	\]
La réponse correcte est \textbf{b.}
	
	\subsection*{Question 2}

	\begin{itemize}
		\item $3 \times 6 - 15 - 2 \neq 0.$ Réponse a fausse.
		\item	$3 \times 12 + 1 \times 4 \neq 0$. Réponse b fausse.
			\item $(1; 3)$  n’est pas un vecteur directeur de $(D)$ . Réponse c fausse.
			\item $(3; 1)$ est un vecteur directeur des droites perpendiculaires à  $(D)$ . Réponse d juste.
	\end{itemize}
La réponse correcte est \textbf{d.}
	
	\subsection*{Question 3}
Cette équation admet une infinité de solutions dans l’ensemble des réels : tous les réels de la forme $\dfrac{\pi}{2} + 2k\pi$, avec $k \in \mathbb{Z}$. La réponse correcte est \textbf{b.}
	
	\subsection*{Question 4}
La fonction $f$ est dérivable sur $\mathbb{R}$ puisque $x^2 + 1 > 0$.\\
Sur cet intervalle $f'(x) = \dfrac{2(x^2 + 1) - 2x \times 2x}{(x^2 + 1)^2} = \dfrac{2 - 2x^2}{(x^2 + 1)^2}$.
	
	\begin{itemize}
		\item L’information a. est donc fausse ;
		\item L’équation de la tangente au point d’abscisse 0, avec $f(0) = 0$ et $f'(0) = 2$ est : $y - 0 = 2(x - 0)$, soit $y = 2x$ : vraie.
	\end{itemize}
La réponse correcte est \textbf{b.}
	
	\subsection*{Question 5}
	
Sur $]-2; +\infty[$, on a : $f'(x) = \dfrac{1(x + 2) - 1(x - 3)}{(x + 2)^2} = \dfrac{5}{(x + 2)^2}$.\\
La réponse correcte est \textbf{c.}
	
