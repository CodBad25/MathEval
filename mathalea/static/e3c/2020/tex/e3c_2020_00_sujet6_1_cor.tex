	\section*{Exercice 1 (5 points)}
	
	\subsection*{Question 1}
	
	Un vecteur normal à une équation cartésienne de droite du type $ax+by+c=0$ est $\displaystyle\binom{-b}{a}$.
	$2x - 5y + 3 = 0$ a donc pour vecteur normal $\displaystyle\binom{5}{2}$.
	La réponse correcte est \textbf{c.}
	
	\subsection*{Question 2}
	
	\[
	x^2 + y^2 + 6x - 8y = 0 \iff (x + 3)^2 - 9 + (y - 4)^2 - 16 = 0 \iff (x + 3)^2 + (y - 4)^2 = 25
	\]
	
	\noindent Cette équation signifie que $(x, y)$ appartient au cercle de centre $A(-3, 4)$ et de rayon 5. La réponse correcte est \textbf{b.}
	
	\subsection*{Question 3}
	
	D'après la formule d'Al-Kashi, on a $BC^2 = AB^2 + AC^2 - 2AB \times AC \cos \widehat{BAC}$,
	soit $25 = 9 + 36 - 2 \times 3 \times 5 \times \cos \widehat{BAC}$
	ou encore $30 \cos \widehat{BAC} = 45 - 25 = 20$
	et enfin $\cos \widehat{BAC} = \dfrac{20}{30} = \dfrac{2}{3}$.
	On a donc $\overrightarrow{AB} \cdot \overrightarrow{AC} = AB \times AC \times \cos \widehat{BAC} = 3 \times 6 \times \dfrac{2}{3} = 12$. La réponse correcte est \textbf{b.}
	
	\subsection*{Question 4}
	
	On a $-\dfrac{3\pi}{4} + 4\pi = -\dfrac{3\pi}{4} + \dfrac{16\pi}{4} = \dfrac{13\pi}{4}$. La réponse correcte est \textbf{c.}
	
	\subsection*{Question 5}
	
	$g(x)=u(x)^3$ avec $u(x) = 4x - 7$ et $u'(x) = 4$ ;
	Donc $g'(x) = u'(x) \times 3u^2(x) = 4 \times 3(4x - 7)^2 = 12(4x - 7)^2$. La réponse correcte est \textbf{d.}
	
