
\medskip

Dans un pays, le nombre de créations d'entreprise augmente $1,5\,\%$ par mois.

En janvier 2018 on compte \np{50000} créations d'entreprise.

On modélise le nombre de créations d'entreprise au $n$-ième mois par une suite $\left(u_n\right)$ telle que \[u_0 = 50 \quad \text{et} \quad u_{n+1} = u_n \times 1,015,\]

où $u_n$ est exprimé en milliers d'euros.

\medskip

\begin{enumerate}
\item 
	\begin{enumerate}
		\item Calculer $u_1$.
		\item Interpréter ce résultat dans le contexte de l'exercice.
	\end{enumerate}
\item	
	\begin{enumerate}
		\item Quelle est la nature de la suite $\left(u_n\right)$ ?

		\item Exprimer $u_n$ en fonction de $n$.
		\item Un journaliste annonce qu'au total dans l'année 2018, près de \np{652000} entreprises se sont
créées.

Donner un calcul permettant de justifier les propos du journaliste.
	\end{enumerate}
\end{enumerate}

\medskip

