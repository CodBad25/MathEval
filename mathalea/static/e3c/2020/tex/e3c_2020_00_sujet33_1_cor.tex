
\subsection*{Question 1}

On sait qu'un vecteur directeur de la droite \((\textit{d})\) est \(\vec{u} \begin{pmatrix} -b \\ a \end{pmatrix}\), soit ici \(\vec{u} \begin{pmatrix} 3 \\ 2 \end{pmatrix}\).

\subsection*{Question 2}

Un vecteur directeur de la droite est \(\vec{u} \begin{pmatrix} 3 \\ 2 \end{pmatrix}\), donc un vecteur normal est par exemple \(\vec{v} \begin{pmatrix} -2 \\ 3 \end{pmatrix}\) ou encore \(-2 \vec{v} \begin{pmatrix} 1 \\ -\dfrac{3}{2} \end{pmatrix}\).

\subsection*{Question 3}

\(\overrightarrow{EB} = \overrightarrow{BA}\) entraîne en faisant intervenir \(\ C\) :
\(\overrightarrow{EC} + \overrightarrow{CB} = \overrightarrow{BC} + \overrightarrow{CA}\),

d'où : \(\overrightarrow{AC} = 2 \overrightarrow{BC} + \overrightarrow{CE}\).

De même : \(\overrightarrow{ED} = 2 \overrightarrow{BC}\) entraîne \(\overrightarrow{EC} + \overrightarrow{CD} = 2 \overrightarrow{BC}\),

d'où : \(\overrightarrow{CD} = 2 \overrightarrow{BC} + \overrightarrow{CE}\).

Conclusion : \(\overrightarrow{AC} = \overrightarrow{CD}\),

ce qui démontre que \(\ C \ \text{est le milieu de} \ [AD]\).

\subsection*{Question 4}

\(\vec{u}\) et \(\vec{v}\) sont orthogonaux, donc \(\vec{u} \cdot \vec{v} = 0\), soit :
\begin{align*}
&9(-x + 4) + 7(2x - 5) = 0 \\
\iff &-9x + 36 + 14x - 35 = 0 \\
\iff &5x + 1 = 0 \\
\iff &x = -\dfrac{1}{5}.
\end{align*}

\subsection*{Question 5}

Avec \(\overrightarrow{AC} \begin{pmatrix} 4 \\ 1 \end{pmatrix}\) et \(\overrightarrow{BD} \begin{pmatrix} -5 \\ 4 \end{pmatrix}\), on obtient :
\[
\overrightarrow{AC} \cdot \overrightarrow{BD} = 4 \times (-5) + 1 \times 4 = -20 + 4 = -16.
\]

