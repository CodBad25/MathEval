	\section*{Exercice 4 (5 points)}
	
	\subsection*{1. Déterminer les coordonnées du point $B$ d’abscisse 7 appartenant à la droite $(d)$.}
	
$x - 3y - 4 = 0$ peut s’écrire $3y = x - 4$, donc si $x = 7$, $3y = 7 - 4 = 3$ et $y = 1$. $B(7, 1)$.
	
	\subsection*{2. Donner un vecteur normal à la droite $(d)$.}
	
On sait que le vecteur $\overrightarrow{n} \left( \begin{array}{c} 3 \\ 1 \end{array} \right)$ est un vecteur normal à la droite $(d)$.
	
	\subsection*{3. Déterminer une équation de la droite $(\Delta)$ perpendiculaire à la droite $(d)$ passant par le point $A$.}
	
Le vecteur $\overrightarrow{u} \left( \begin{array}{c} 1 \\ -3 \end{array} \right)$ est un vecteur directeur de la droite $(d)$. On a donc $M(x, y) \in (\Delta) \Leftrightarrow \overrightarrow{AM} \cdot \overrightarrow{u} = 0 \Leftrightarrow 3(x - 3) + 1(y - 1) = 0 \Leftrightarrow 3x - 9 + y - 1 = 0 \Leftrightarrow 3x + y - 10 = 0$.
	
	\subsection*{4. Calculer les coordonnées du projeté orthogonal $H$ du point $A$ sur la droite $(d)$.}
	
Puisque $A$ appartient à la perpendiculaire à la droite $(d)$, son projeté sur $(d)$ est le point d’intersection de $(d)$ et de cette perpendiculaire.\\
Ses coordonnées vérifient donc le système :
	\[
	\begin{cases}
		x - 3y - 4 = 0 \\
		3x + y - 10 = 0
	\end{cases}
	\]
En multipliant la première équation par 3 et en la soustrayant de la deuxième, on obtient :
	\[
	\begin{cases}
		-3x + 9y + 12 = 0 \\
		3x + y - 10 = 0
	\end{cases}
	\]
D'où :
	\[
	10y + 2 = 0 \Leftrightarrow y = -\dfrac{1}{5}, \quad \text{puis} \quad x = 3y + 4 = -\dfrac{3}{5} + 4 = -\dfrac{3}{5} + \dfrac{20}{5} = \dfrac{17}{5}
	\]
	
	$H \left( \dfrac{17}{5}, -\dfrac{1}{5} \right)$.
	
	\subsection*{5. Calculer la distance $AH$ et en donner une interprétation.}
On a :
	\[
	AH^2 = \left( \dfrac{17}{5} - 3 \right)^2 + \left( -\dfrac{1}{5} - 1 \right)^2 = \left( \dfrac{17}{5} - \dfrac{15}{5} \right)^2 + \left( -\dfrac{1}{5} - \dfrac{5}{5} \right)^2 = \left( \dfrac{2}{5} \right)^2 + \left( -\dfrac{6}{5} \right)^2 = \dfrac{4}{25} + \dfrac{36}{25} = \dfrac{40}{25}
	\]
Donc
	\[
	AH = \sqrt{\dfrac{40}{25}} = \dfrac{\sqrt{40}}{5} = \dfrac{2\sqrt{10}}{5} \approx 1,265
	\]
Cette distance est la distance (la plus petite) du point $A$ à la droite $(d)$.
	
