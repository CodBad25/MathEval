
\subsection*{Question 1}

Le coefficient directeur de la tangente en un point d'abscisse \(a\) est le nombre dérivé \(f'(a)\). Donc, \(f'(2) = -1\) est vraie.

\subsection*{Question 2}

On sait que \(\sin^2 x + \cos^2 x = 1\), soit :
\[
\dfrac{1}{4} + \cos^2 x = 1,
\]
d'où :
\[
\cos^2 x = 1 - \dfrac{1}{4} = \dfrac{3}{4}.
 \]
Or, pour \( x \in \left[\dfrac{\pi}{2} \,;\, \dfrac{3\pi}{2}\right] \), on sait que \(\cos x < 0\), donc :
\[
\cos x = -\sqrt{\dfrac{3}{4}} = -\dfrac{\sqrt{3}}{\sqrt{4}} = -\dfrac{\sqrt{3}}{2}.
\]

\subsection*{Question 3}

\(\overrightarrow{OA} \cdot \overrightarrow{OB} = \overrightarrow{OH} \cdot \overrightarrow{OB} = 3 \times 4 = 12\). Proposition \textbf{A} fausse.

\(\sin \widehat{AOB} = \dfrac{AH}{OA} = \dfrac{4}{5}\). Proposition \textbf{B} fausse, proposition \textbf{D} juste.

\(\cos \widehat{AOB} = \dfrac{OH}{OA} = \dfrac{3}{5}\). Proposition \textbf{C} fausse.

\subsection*{Question 4}

Un vecteur directeur de la droite \((d)\) est \(\vec{u} \begin{pmatrix} -2 \\ 3 \end{pmatrix}\).

\begin{align*}
&M(x \,;\, y) \in (d') \\
\iff &\overrightarrow{AM} \cdot \vec{u} = 0 \\
\iff &-2(x - 1) + 3(y - 2) = 0 \\
\iff &-2x + 2 + 3y - 6 = 0 \\
\iff &-2x + 3y - 4 = 0,
\end{align*}

ou encore \( 2x - 3y + 4 = 0 \).

\subsection*{Question 5}

Avec \(C\) centre du cercle on a \( C(3 \,;\, 0) \).

D'autre part \(AB^2 = 4^2 + (-4)^2 = 16 + 16 = 32\),

d'où \(AB = \sqrt{32} = \sqrt{16 \times 2} = 4\sqrt{2}\), donc \(R = 2\sqrt{2}\).

\begin{align*}
&M(x \,;\, y) \in (C) \\
\iff &CM^2 = R^2 = (2\sqrt{2})^2 = 4 \times 2 = 8 \\
\iff &(x - 3)^2 + (y - 0)^2 = 8 \\
\iff &x^2 + 9 - 6x + y^2 = 8 \\
\iff &x^2 + y^2 - 6x + 1 = 0.
\end{align*}

