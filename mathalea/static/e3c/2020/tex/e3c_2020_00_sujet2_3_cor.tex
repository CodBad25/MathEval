	\section*{Exercice 3 (5 points)}
	
	\subsection*{Partie A}
	\subsubsection*{1.}
	On sait que pour une suite géométrique de premier terme $u_0$ et de raison $q$, le terme général de rang $n$, s’écrit $u_n = u_0 \times q^n$.
	\[
	u_8 = 0,2 \times 2^{18} = 0,2 \times 262144 = 52428,8.
	\]
	\[
	u_{50} = 0,2 \times 2^{50} \approx 2,25 \times 10^{14}.
	\]
	
	\subsubsection*{2.}
	$S = u_0 + u_1 + u_2 + u_3 + u_4 + \ldots + u_{18} = 0,2 + 0,4 + 0,8 + 1,6 + \ldots + 52428,8$.
	\[
	2S = 0,4 + 0,8 + 52428,8 + 104857,6.
	\]
	Par différence (deuxième ligne moins première ligne) :
	\[
	S = 104857,6 - 0,2 = 104857,4.
	\]
	
	\subsubsection*{3.}
\begin{center}
	$\begin{array}{|l|}\hline
		U \gets  0,2\\
		S \gets 0,2\\
		N \gets 0\\
		~\\
		\text{Tant que \ldots \ldots}\\
		\quad U ← U * 2\\
		\quad S ← S + U\\
		\quad N \gets N + 1\\
		~\\
		
		\text{Fin tant que}~~~~~~~\\
		\text{Afficher}\: N\\ \hline
	\end{array}$
\end{center}

	
		

	
	\subsection*{Partie B}

	On retrouve dans les sommes versées par Claude exactement les premiers termes de la suite de la partie A et le total des sommes versées d’après le résultat de la question 3 de la partie A est 104857,40 €. Camille pourra donc acheter l’appartement.
	
