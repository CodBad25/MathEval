
\medskip
 
Une entreprise fabrique un engrais biologique. Chaque jour, le volume d'engrais fabriqué est compris entre $5$ m$^3$ et 60 m$^3$.

Le coût moyen quotidien de production de cet engrais, exprimé en \textbf{centaines d'euros}, est modélisé par la fonction $f$ définie sur l'intervalle [5~;~60] par:

\[f(x) = \dfrac{x^2- 15x +400}{x}\]

où $x$ est le volume quotidien d'engrais fabriqué, exprimé en m$^3$.

\medskip

\begin{enumerate}
\item Déterminer le coût moyen quotidien pour la production de $5$~m$^3$ d'engrais.
\item Quels volumes d'engrais faut-il fabriquer pour avoir un coût moyen de
production égal à \np{4300}~\euro{}($43$ centaines d'euros) ?
\item Pour quel volume d'engrais fabriqué le coût moyen de production est-il
minimal? 

Déterminer ce coût moyen minimal.
\end{enumerate}
