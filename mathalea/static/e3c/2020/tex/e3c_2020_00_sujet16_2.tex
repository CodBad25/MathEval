
\medskip

\textbf{Partie A}

\vspace{0.3cm}

On considère la fonction polynôme du second degré $P$ définie sur $ \R $ par :

\[P(x)=x^2-7x+6\]

\begin{enumerate}
\item  Résoudre l’équation $P(x)= 0$.
\item Étudier le signe de $P$ sur $\R$.
\end{enumerate}

\medskip

\textbf{Partie B}

\medskip

On considère la fonction polynôme du troisième degré $f$ définie sur $\R$ par :
\[f(x) = 2x^3 -21x^2 + 36x\]
\begin{enumerate}
\item  Calculer la dérivée $f'$ de $f$ et vérifier que $f'(x) = 6P(x)$
\item Étudier les variations de la fonction $f$.
\item On se place dans un repère du plan. Déterminer une équation de la tangente
T \end{enumerate}à la courbe représentative de $f$ au point B d’abscisse 3.

\vspace{0,5cm}

