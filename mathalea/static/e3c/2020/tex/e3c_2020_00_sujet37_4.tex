
\medskip

On considère la fonction $f$ définie sur l'intervalle $[-4~;~3]$ \footnote{Le texte donnait $[-4~;~4]$, en contradiction avec le graphique} par 

\[f(x) = x^3 + 3x^2 - 9x - 20.\]

On admet que la fonction $f$ est dérivable sur l'intervalle $[-4~;~3]$ et on note $f'$ sa fonction dérivée.

La courbe représentative de la fonction $f$, notée $\mathcal{C}$, est tracée dans le repère ci-dessous.

La droite $\mathcal{T}$ tracée dans le repère est la tangente à la courbe $\mathcal{C}$ au point d'abscisse 0.

\begin{center}
\psset{xunit=1.4cm,yunit=0.28cm,labelFontSize=\scriptstyle,showorigin=false}
\begin{pspicture}(-4.6,-28)(3.5,10)
\multido{\n=-4+1}{8}{\psline[linewidth=0.45pt,linecolor=lightgray](\n,-26)(\n,9)}
\multido{\n=-26+1}{36}{\psline[linewidth=0.45pt,linecolor=lightgray](-4.4,\n)(3.15,\n)}
\psdots[dotstyle=BoldCircle,fillcolor=blue](-4,0)(3,7)
\uput[ul](-4,0){$\mathcal{C}$}\uput[u](-3,8){$\mathcal{T}$}\uput[dl](0,0){O}
\psplot[plotpoints=2000,linewidth=1.25pt,linecolor=orange,linestyle=dashed]{-3.2}{0.6}{x 9 neg  mul 20 sub}
\psaxes[linewidth=1.25pt,Dy=2]{->}(0,0)(-4.4,-26.2)(3.3,9.4)
\def\Func{ x x x 3 add mul 9 sub mul 20 sub }
\psplot[plotpoints=2000,linewidth=1.25pt,linecolor=blue]{-4}{3}{\Func}
\end{pspicture}
\end{center}

\medskip

\begin{enumerate}
\item Déterminer graphiquement les extremums de la fonction $f$.
\item Déterminer l'expression de $f'(x)$ sur $[-4~;~3]$.
\item Étudier le signe de $3x^2 + 6x - 9$ en fonction de $x$ sur $[-4~;~3]$.
\item En déduire le tableau de variations de $f$ sur $[-4~;~3]$ et retrouver les résultats de la question \textbf{1.}
\item Déterminer l'équation réduite de la droite $\mathcal{T}$, tangente à la courbe $\mathcal{C}$ au point d'abscisse 0.
\end{enumerate}
