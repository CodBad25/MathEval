
\medskip

Une culture de pois comporte des pois de couleur \og jaune \fg{} ou \og vert \fg{} et de forme \og lisse \fg{} ou \og ridé \fg.

Le tableau ci-dessous est partiellement renseigné à partir des observations effectuées sur
un grand nom\-bre de pois de cette culture.

\renewcommand\arraystretch{2.1}
\begin{center}
\begin{tabular}{|*{3}{>{\centering \arraybackslash}m{3.9cm}|}c|} 
 \cline{2-4}
\multicolumn{1}{c|}{}&Nombre de pois jaunes & Nombre de pois verts& Total\\
\hline
Nombre de pois ridés	&100	&? 	&600\\\hline
Nombre de pois lisses	& ?		&?	& ?\\\hline
Total 					&300	&? 	&\np{10000}\\\hline
\end{tabular}
\end{center}

\smallskip

\begin{enumerate}
\item  Compléter le tableau suivant.

\begin{center}
\begin{tabularx}{\linewidth}{|*{3}{>{\centering \arraybackslash} X|}m{0.9cm}|} 
 \cline{2-4}
\multicolumn{1}{c|}{}	&Nombre de pois jaunes 	& Nombre de pois verts	& Total\\
\hline
Nombre de pois ridés	&100					& 						&600\\\hline
Nombre de pois lisses	& 						&						& \\\hline
Total 					&300					& 						&\np{10000}\\\hline
\end{tabularx}
\end{center}
\end{enumerate}

On choisit au hasard un pois de la culture et on s'intéresse aux évènements suivants :

\begin{itemize}
\item $J$ : \og le pois est jaune \fg ;
\item $R$ : \og le pois est ridé \fg.
\end{itemize}

L'échantillon étudié est suffisamment important pour être considéré comme représentatif de l'ensemble de la culture de pois.

\begin{enumerate}[resume]
\item Quelle est la probabilité que le pois soit vert et lisse ?
\item Calculer la probabilité que le pois soit vert.
\item Calculer la probabilité qu'un pois soit jaune sachant qu'il est ridé, et en déduire la probabilité qu'un pois soit vert sachant qu'il est ridé.
\item Calculer $P_J(R)$ et en donner une interprétation dans le contexte de l'énoncé.
\end{enumerate}

