	
	\subsection*{Question 1}
	
	On considère la fonction \(f\) définie sur \(\mathbb{R}\) par \(f(x) = 2x^2 + 6x - 8\).
	
	Pour l’équation \(f(x) = 0\), on a \(\Delta = 6^2 - 4 \times 2 \times (-8) = 36 + 64 = 100 = 10^2 > 0\).
	
	L’équation a deux solutions :
	
	\[
	x_1 = \dfrac{-6 + 10}{2 \times 2} = 1 \quad \text{et} \quad x_2 = \dfrac{-6 - 10}{2 \times 2} = -4.
	\]
	
	On sait alors que \(f(x)\) se factorise en : \(f(x) = 2(x - 1)(x + 4)\).
	
	\subsection*{Question 2}
	
	\[
	\left( e^x \right)^2 e^{-x} = e^{2x} \times e^x = e^{2x + x} = e^{3x}.
	\]
	
	\subsection*{Question 3}
	
	On sait qu’une équation de la tangente au point d’abscisse 0 est :\\
$y =  g'(0)(x - 0)+g(0)$,  soit $ y - e^0 = e^0 x$  et enfin  $y = x + 1$.
	
	
	\subsection*{Question 4}
	
	La fonction \(f\) est un produit de fonctions dérivables sur \(\mathbb{R}\). On a donc pour tout réel :
	\[
	f'(x) = -1 e^x + (-x + 1) e^x = e^x (-1 - x + 1) = e^x (-x) = -x e^x.
	\]
	
	\subsection*{Question 5}
	
	Les propositions a., c. et d. sont vraies. Quant à \(f'(3) = -2\), ce nombre dérivé est bien négatif mais bien plus grand que -2.
	
