 
\medskip

Ce QCM comprend $5$ questions.

\emph{Pour chacune des questions, une seule des quatre réponses proposées est correcte.}

\emph{Les questions sont indépendantes.}

\emph{Pour chaque question, indiquer le numéro de la question et recopier sur la copie la lettre correspondante à la réponse choisie.}

\emph{Aucune justification n’est demandée mais il peut être nécessaire d’effectuer des recherches au brouillon pour aider à déterminer votre réponse.}

\emph{Chaque réponse correcte rapporte 1 point. Une réponse incorrecte ou une question sans réponse n’apporte ni ne retire de point.}

\medskip
 \textbf{Question 1 :}
\medskip

 On considère la suite $\left(u_n\right)$ définie par $u_0=  100$ et pour tout entier naturel $n$, $u_{n+1}=u_n-\frac{13}{100}u_n$. 

Quelle est la nature de la suite $\left(u_n\right)$ ?


\begin{tabularx}{\linewidth}{*{1}{X}}
\textbf{A.~~} géométrique de raison 1\\
\textbf{B.~~} arithmétique de raison $-\dfrac{13}{100}$\\
\textbf{C.~~} géométrique de raison 1 et arithmétique de raison $-\dfrac{13}{100}$\\
\textbf{D.~~} géométrique de raison $0,87$\\
\end{tabularx}

\medskip
 \textbf{Question 2 :}
\medskip

On considère la variable aléatoire $X$ qui prend les valeurs $x_i$ pour $i$ entier naturel allant de 1 à 5.

La loi de probabilité incomplète de la variable aléatoire $X$ est donnée ci-dessous :

\begin{center}
\begin{tabular}[]{|*{6}{c|}}
\hline
$X = x_i$				&$- 6$	&$- 3$	&0	&3	&$x_5$\\\hline
$P\left(X = x_i\right)$	&0,2	&0,1	&0,2&0,4&0,1\\\hline
\end{tabular}
\end{center}

L’espérance de la variable aléatoire $X$ est égale à $0,7$.

Quelle est la valeur $x_5$ prise par la variable aléatoire $X$ ?


\begin{tabularx}{\linewidth}{*{4}{X}}
\textbf{A.~~} $6$ &\textbf{B.~~} $1$&\textbf{C.~~}$10$& \textbf{D.~~} $100$.
\end{tabularx}

\medskip
 \textbf{Question 3 :}
\medskip

 Soit $f$ la fonction dérivable définie sur $\left]-\dfrac{7}{3} ; +\infty\right[$ par $f(x)= \dfrac{2x+3}{3x+7}$ et $f'$ sa fonction dérivée.
 
 \begin{center}
 \begin{tabularx}{\linewidth}{*{4}{X}}
\textbf{A.~~} $f'(x)=\dfrac{2}{3} $ &\textbf{B.~~} $f'(x) =\dfrac{23}{(3x+7)^2}$&\textbf{C.~~}$f'(x)=\dfrac{5}{(3x+7)^2} $& \textbf{D.~~} $f'(x)=\dfrac{5}{3x+7}$.
\end{tabularx}
 \end{center}
 
\medskip
 \textbf{Question 4 :} 
\medskip

De 2017 à 2018, le prix d’un article a augmenté de 10\,\%. En 2019, ce même article a retrouvé son prix de 2018. Quelle a été l’évolution du prix entre 2018 et 2019 ?

\begin{tabularx}{\linewidth}{*{1}{X}}
\textbf{A.~~} une baisse de 10\,\%\\
\textbf{B.~~} une baisse de plus de 10\,\%\\
\textbf{C.~~} on ne peut pas savoir\\
\textbf{D.~~} une baisse de moins de 10\,\%.\\
\end{tabularx}

\medskip

\textbf{Question 5 :}

\medskip

Soit $\left(u_n\right)$ la suite définie par $u_0 = 4$ et pour tout entier naturel $n$ par $u_{n+1} = 3u_n - 5$.

On souhaite qu'à la fin de l'exécution de l'algorithme, la valeur contenue dans la variable $u$ soit celle de $u_5$. Quel algorithme doit-on choisir ?

\medskip

\begin{tabularx}{\linewidth}{*{2}{X}}
\textbf{A.~~}
\fbox{
\begin{minipage}[t]{3cm}
$u=4$ \\
$n=0$\\
For $k$ in range (5) :\\
\hspace*{0.3cm}$u=3*n-5$\\
\hspace*{0.3cm}$n=n+1$
\end{minipage}}&
\textbf{B.~~} \fbox{
\begin{minipage}[t]{3cm}
 $u=4$\\
 $n=0$ \\
 For $k$ in range (5) :\\
\hspace*{0.3cm}$u_{n+1}=3*u_n-5$\\
\hspace*{0.3cm}$n=n+1$
 \end{minipage}}\\
 &\\
\textbf{C.~~}\fbox{\begin{minipage}[t]{3cm}
$u=4$ \\
For $k$ in range (5) :\\
\hspace*{0.3cm}$u=3*u-5$
\end{minipage}}&
\textbf{D.~~}\fbox{\begin{minipage}[t]{3cm}
$u=4$\\
$n=0$\\
While $\leqslant$ 5 :\\
\hspace*{0.3cm}$u=3*u-5$\\
\hspace*{0.3cm}$n=n+1$
\end{minipage}}
\end{tabularx}

\vspace{1cm}


