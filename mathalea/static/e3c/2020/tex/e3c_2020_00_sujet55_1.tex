
\medskip

\emph{Cet exercice est un QCM et comprend cinq questions. Pour chacune des questions, une seule
des quatre réponses proposées est correcte. Les questions sont indépendantes.}

\emph{Pour chaque question, indiquer le numéro de la question et recopier sur la copie la lettre
correspondante à la réponse choisie.}

\emph{Aucune justification n'est demandée mais il peut être nécessaire d'effectuer des recherches
au brouillon pour aider à déterminer votre réponse.}

\emph{Chaque réponse correcte rapporte un point. Une réponse incorrecte ou une question sans
réponse n'apporte ni ne retire de point.}

\medskip

\textbf{Question 1}

\medskip

Une équation de la tangente à la courbe représentative de la fonction exponentielle au point
d'abscisse 0 est :

\begin{tabularx}{\linewidth}{*{4}{X}}
\textbf{a.~~}$y=x+1$  &\textbf{b.~~}$y=\e x$ &\textbf{c.~~$y=\e^x $}& \textbf{d.~~}$y= x-1 $ .
\end{tabularx}

\smallskip

\textbf{Question 2}

\medskip

La fonction $f$ définie sur $\R$ par : $f(x) = \e^{-2x+6}$ admet pour dérivée la fonction $f'$ définie
sur $\R$ par :
\begin{tabularx}{\linewidth}{*{2}{X}}
\textbf{a.~~}$f'(x) = \e^{-2x+6}$  &\textbf{b.~~} $f'(x) = -2\e^{-2x+6}$\\\textbf{c.~~}$f'(x) = -2x\e^{-2x+6}$& \textbf{d.~~}$f'(x) = (-2x + 6)\e^{-2x+6}$ .
\end{tabularx}

\medskip

\textbf{Question 3}

\medskip
Dans le repère orthonormé \Oij, le vecteur $\vv{AB}$ représenté ci-dessous est égal à :

\begin{center}
\psset{labelFontSize=\scriptstyle,showorigin=false}
\begin{pspicture}(-0.75,-0.5)(7.5,3.5)
\multido{\n=0+1}{8}{\psline[linewidth=0.75pt,linecolor=lightgray](\n,0)(\n,3.2)}
\multido{\n=0+0.2}{37}{\psline[linewidth=0.5pt,linecolor=lightgray](\n,0)(\n,3.2)}
\multido{\n=0+1}{4}{\psline[linewidth=0.75pt,linecolor=lightgray](0,\n)(7.2,\n)}
\multido{\n=0+0.2}{17}{\psline[linewidth=0.5pt,linecolor=lightgray](0,\n)(7.2,\n)}
\psaxes[linewidth=0.95pt]{->}(0,0)(7.4,3.3)
\psdots[dotstyle=+,dotscale =1.4,dotangle=45](1,3)(7,1)
\uput[l](1,3){A} \uput[dl](7,1){B}\psset{arrowscale=1.8}
\psline[linewidth=0.8pt,linecolor=darkgray]{->}(0,0)(0,1)\psline[linewidth=0.8pt,linecolor=darkgray]{->}(0,0)(1,0)
\psline[linewidth=0.8pt,linecolor=darkgray]{->}(1,3)(7,1)
\uput[dl](0,0){O}\uput[d](0.5,0){$\vv{\imath}$}\uput[l](0,0.5){$\vv{\jmath}$}

\end{pspicture}
\end{center}

\medskip

\begin{tabularx}{\linewidth}{*{4}{X}}
\textbf{a.~~}$-2\vv{\imath} + 6 \vv{\jmath}$  &\textbf{b.~~}$-6\vv{\imath}+ 2\vv{\jmath}$ &\textbf{c.~~}$2\vv{\imath} -6\vv{\jmath}$& \textbf{d.~~}$6\vv{\imath}- 2\vv{\jmath}$ .
\end{tabularx}

\medskip

\textbf{Question 4}

\medskip

On considère la fonction $f$ définie pour tout réel $x$ par $f(x)=\sin x -\cos x$. Parmi les quatre propositions suivantes, une seule est correcte. Laquelle ?

\medskip

\begin{tabularx}{\linewidth}{*{1}{X}}
\textbf{a.~~} $f$ est une fonction paire. \\\textbf{b.~~}$f$ est une fonction impaire. \\\textbf{c.~~}$f$ n'est ni paire ni impaire.\\ \textbf{d.~~}$f (0)=0$ .
\end{tabularx}

\medskip

\textbf{Question 5}

\medskip

Dans le plan muni d'un repère, on considère la droite $(d)$ d'équation : $5x-2y+8=0$.  

La droite $(d)$ a pour coefficient directeur :

\medskip

\begin{tabularx}{\linewidth}{*{4}{X}}
\textbf{a.~~} $\vv{u}(2 ;5)$ &\textbf{b.~~}$\dfrac{5}{2}$ &\textbf{c.~~}$\dfrac{ 2}{5}$& \textbf{d.~~} $ -2$.
\end{tabularx}



\vspace{0.75cm}

