
\medskip
Dans un repère orthonormé, on considère les points A$(-1~;~3)$, B(5~;~0) et C(9~;~3).

\medskip

\begin{enumerate}
\item Déterminer une équation cartésienne de la droite (AB).
\item Déterminer une équation cartésienne de la droite $D$ passant par le point C et de vecteur normal $\vect{n}\begin{pmatrix}-1\\3\end{pmatrix}$.
\item Démontrer que les droites $D$ et (AB) ne sont pas parallèles.

On admet que le point E(3~;~1) est le point d'intersection de ces deux droites.
\item Les droites D et (AB) sont-elles perpendiculaires ?
\item On donne AE$= 2\sqrt{5}$ et EC $= 2\sqrt{10}$.

Calculer la mesure en degrés de l'angle $\widehat{\text{AEC}}$.
\end{enumerate}

\bigskip

