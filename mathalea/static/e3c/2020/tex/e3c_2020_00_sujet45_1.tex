
\medskip

Cet exercice est un questionnaire à choix multiple (QCM) comportant 5 questions.

Pour chacune des questions, une seule des quatre réponses proposées est correcte.

Les questions sont indépendantes.

Pour chaque question, indiquer le numéro de la question et recopier sur la copie la lettre correspondante à la réponse choisie.

Aucune justification n'est demandée mais il peut être nécessaire d'effectuer des recherches au brouillon pour aider à déterminer votre réponse.

Chaque réponse correcte rapporte $1$ point. Une réponse incorrecte ou une question sans réponse n'apporte ni ne retire de point.

\medskip

\textbf{Question 1}

\medskip

Dans le plan muni d'un repère orthonormé, on considère les vecteurs $\vect{u}(-2~;~4)$ et $\vect{v}(3~;~-6)$. Le produit scalaire $\vect{u} \cdot \vect{v}$ est égal à :

\begin{center}
\begin{tabularx}{\linewidth}{|*{4}{X|}}\hline
\textbf{a.~~}18&\textbf{b.~~}$-30$&\textbf{c.~~}$0$&\textbf{d.~~}$24$\\ \hline
\end{tabularx}
\end{center}

\medskip

\textbf{Question 2}

\medskip

On considère le triangle ABC tel que AB = 5, AC = 7 et BAC $= 60\degres$. Quelle est la longueur du côté BC ?

\begin{center}
\begin{tabularx}{\linewidth}{|*{4}{X|}}\hline
\textbf{a.~~}BC$ = \sqrt{109}$ &\textbf{b.~~}BC$ = \sqrt{74}$&\textbf{c.~~}BC$= -35\sqrt{3} + 74$  &\textbf{d.~~}BC$= \sqrt{39}$\\ \hline
\end{tabularx}
\end{center}

\medskip

\textbf{Question 3}

\medskip

Dans le plan muni d'un repère orthonormé, on considère le cercle $C$ de centre A(2~;~3) et de rayon $R = 4$.

Parmi les équations suivantes, laquelle est une équation du cercle $C$ ?

\begin{center}
\begin{tabularx}{\linewidth}{|*{2}{X|}}\hline
\textbf{a.~~}$x^2 + 4x + y^2 + 6y + 9 = 0$&\textbf{b.~~} $x^2 + 4x + y^2 + 6y - 3 = 0$\\ \hline
\textbf{c.~~}$x^2 - 4x + y^2 - 6y - 3 = 0$&\textbf{d.~~}$x^2 - 4x + y^2 - 6y+ 9=0$\\ \hline
\end{tabularx}
\end{center}

\medskip

\textbf{Question 4}

\medskip

Le réel $\dfrac{- 23\pi}{3}$ a le même point image sur le cercle trigonométrique que le réel :

\begin{center}
\begin{tabularx}{\linewidth}{|*{4}{X|}}\hline
\textbf{a.~~}$\dfrac{-\pi}{3}$ &\textbf{b.~~}$\dfrac{\pi}{3}$&\textbf{c.~~}$\dfrac{-2\pi}{3}$ &\textbf{d.~~}$\dfrac{2\pi}{3}$\rule[-3mm]{0mm}{9mm}\\ \hline
\end{tabularx}
\end{center}

\medskip

\textbf{Question 5}

\medskip

On considère l'algorithme suivant écrit en langage Python:

\begin{center}
\begin{tabular}{|c l}
1& def liste(N) :\\
2&U = 1\\
3&L= [U]\\
4&for i in range(1, N) :\\
5&U = 2*U + 3\\
6&L.append(U)\\ 
7&return(L)\\
8&\\
\end{tabular}
\end{center}

Que contient la variable L à la fin de l'exécution dans le cas où on choisit N$ = 4$ ?

\begin{center}
\begin{tabularx}{\linewidth}{|*{4}{X|}}\hline
\textbf{a.~~}[1, 5, 13, 29, 61]&\textbf{b.~~}[1, 5, 13, 29]&\textbf{c.~~}61  &\textbf{d.~~}9\\ \hline
\end{tabularx}
\end{center}

\bigskip

