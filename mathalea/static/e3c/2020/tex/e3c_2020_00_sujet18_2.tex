  
\medskip

Une banque propose un placement. Le compte est rémunéré et rapporte 5\,\% par an. La banque
prend des frais de gestion qui se montent à 12 euros par an.

Ainsi, chaque année la somme sur le compte augmente de 5\,\% puis la banque prélève 12 euros.

Noémie place la somme de \np{1000} euros dans cette banque.

On appelle $u_n$ la somme disponible sur le compte en banque de Noémie après $n$ années, où
$n$ désigne un entier naturel.

On a donc $u_0 = \np{1000}$ et pour tout entier naturel n, $u_{n+1} = 1,05 u_n-12$

\begin{enumerate}
\item  Avec un tableur on a calculé les premiers termes de la suite
$\left(u_n\right)$ :

\medskip

\begin{minipage}[]{0.425\textwidth}
\begin{enumerate}
\item Quelle formule a-t-on entrée dans la cellule B3 avant de
l’étirer pour obtenir ces résultats ?
\item En utilisant les valeurs calculées de la suite, indiquer à
Noémie combien de temps elle doit attendre pour que son
placement lui rapporte 20\,\%.
\end{enumerate}
\end{minipage}
\hspace{2cm}
\begin{minipage}[]{8.97cm}
\begin{tabularx}{0.445\textwidth}{|>{\columncolor{lightgray}}c|*{2}{>{\raggedleft \arraybackslash}X|}}
\hline
\rowcolor{lightgray}&\centering  A& \centering B\tabularnewline\hline
1&\raggedright n&\raggedright u(n)\tabularnewline\hline
2&0&\np{1000}\\\hline
3&1&\np{1038.00}\\\hline
4&2&\np{1077.90}\\\hline
5&3&\np{1119.80}\\\hline
6&4&\np{1163.78}\\\hline
7&5&\np{1209.97}\\\hline
8&6&\np{1258.47}\\\hline
9&7&\np{1309.40}\\\hline
10&8&\np{1362.87}\\\hline
11&9&\np{1419.01}\\\hline
12&10&\np{1477.96}\\\hline
\end{tabularx}
\end{minipage}

\end{enumerate}

\medskip

On pose $\left(v_n\right)$ la suite définie, pour tout entier naturel $n$, par $v_n = u_n- 240$.

\begin{enumerate}[resume]
\item Montrer que la suite $\left(v_n\right)$ est géométrique de raison $1,05$.
\item Exprimer $v_n$ puis $u_n$ en fonction de l’entier $n$.
\item Calculer à partir de cette dernière formule la somme disponible sur le compte en banque de
Noémie après 20 ans de placement.
\end{enumerate}

\vspace{0,5cm}

