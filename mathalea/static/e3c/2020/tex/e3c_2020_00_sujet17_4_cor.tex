	

	\subsection*{1.}
	
	On a
	\[
	\overrightarrow{OM} \left( \begin{array}{c} 3 \\ -2 \end{array} \right) \quad \text{et} \quad \overrightarrow{DC} \left( \begin{array}{c} 2 \\ 3 \end{array} \right), \quad donc \quad \overrightarrow{OM} \cdot \overrightarrow{DC} = 6 - 6 = 0.
	\]
	
	Les vecteurs sont orthogonaux, donc les droites \((OM)\) et \((DC)\) sont perpendiculaires (autrement dit : dans le triangle \(CDM\), \((MH)\) est la hauteur issue de \(M\).
	
	\subsection*{2.}
	
	On a
$\overrightarrow{CD} \left( \begin{array}{c} -2 \\ -3 \end{array} \right)$ et  $\overrightarrow{CM} \left( \begin{array}{c} 3 \\ -5 \end{array} \right)$,  d'où  $\overrightarrow{CD} \cdot \overrightarrow{CM} = -6 + 15 = 9$.
	
	
	\subsection*{3.}
	
	On sait que
	\[
	\overrightarrow{CD} \cdot \overrightarrow{CM} = \overrightarrow{CD} \cdot \overrightarrow{CH} = CD \times CH.
	\]
	
	Or dans le triangle $OCD$ rectangle en \(O\):\\
	 \(CD^2 = OD^2 + OC^2 = 2^2 + 3^2 = 4 + 9 = 13\),\\ d'où \(CD = \sqrt{13}\).
	
	On a donc
	\[
	\overrightarrow{CD} \cdot \overrightarrow{CM} = 9 = CD \times CH = \sqrt{13} \times CH.
	\]
	
	Donc \(CH = \dfrac{9}{\sqrt{13}} \approx 2,496\).
	
