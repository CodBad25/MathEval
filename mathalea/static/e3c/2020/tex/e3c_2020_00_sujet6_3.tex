
\medskip

Soit la suite $\left(u_n\right)$ de premier terme $u_0= 400$ vérifiant la relation, pour tout entier naturel $n$,

\[u_{n+1} = 0,9u_n +60.\]

Soit la suite géométrique $\left(v_n\right)$ de premier terme $v_0= - 200$ et de raison $0,9$.

\medskip

\begin{enumerate}
\item Calculer $u_2$ et $v_2$.
\item Calculer la somme des 20 premiers termes de la suite $\left(v_n\right)$.
\item  La suite $\left(u_n\right)$ est-elle arithmétique ? La suite $\left(u_n\right)$ est-elle géométrique ?
\item  Recopier et compléter la fonction Suite suivante écrite en Python qui permet
de calculer la somme $S$ des 20 premiers termes de la suite $\left(u_n\right)$.

\begin{center}
\begin{tabularx}{0.4\linewidth}{|X|}\hline
\texttt{def Suite ( ) :}\\
\quad  \texttt{U = 400}\\
\quad \texttt{S = 0}\\
\quad \texttt{for i in range (20)}\\
\qquad \texttt{S = \dotfill}\\
\qquad \texttt{U = \dotfill}\\
\texttt{return (...)}\\\hline
\end{tabularx}
\end{center}

\item On admet que $u_n= v_n + 600$. En déduire $u_{20}$.
\end{enumerate}

\bigskip

