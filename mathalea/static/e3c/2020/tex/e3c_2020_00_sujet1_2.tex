
\medskip

Une entreprise pharmaceutique fabrique un soin antipelliculaire. Elle peut produire entre $200$ et \np{2000}~litres de produit par semaine. Le résultat, en dizaines de milliers d'euros, réalisé pour la production et la vente de $x$ centaines de litres est donné par la fonction $R$ définie par :

\[R(x) = (5x - 30)\text{e}^{-0,25x},\:  \text{pour tout réel }\: x \in  [2~;~20].\]

\smallskip

\begin{enumerate}
\item Calculer le résultat réalisé par la fabrication et la vente de $7$ centaines de litres de produit. On l'arrondira à l'euro près.
\item  Vérifier que pour la fabrication et la vente de $400$ litres de produit, l'entreprise réalise un résultat négatif (appelé déficit).
\item  Résoudre l'inéquation $R(x) \geqslant 0$, d'inconnue $x$. Interpréter dans le contexte de l'exercice.
\item  On note $R'$ la dérivée de la fonction $R$.

Un logiciel de calcul formel donne: $R'(x) = (- 1,25x + 12,5)\text{e}^{-0,25x}$.

En déduire la quantité de produit que l'entreprise doit produire et vendre pour réaliser le résultat maximal.
\end{enumerate}

\bigskip

