
\medskip

Désirant participer à une course de 150~km, un cycliste prévoit l'entraînement suivant :
\begin{itemize}
\item parcourir 30~km en première semaine ;
\item chaque semaine qui suit, augmenter la distance parcourue de 9\,\% par rapport à celle
parcourue la semaine précédente.
\end{itemize}
On modélise la distance parcourue chaque semaine à l'entraînement par la suite $\left(d_n\right)$ où $d_n$
représente la distance en km parcourue pendant la $n$-ième semaine d'entraînement.

On a ainsi $d_1 = 30$.

\medskip

\begin{enumerate}
\item Prouver que $d_3 = 35,643$.
\item Quelle est la nature de la suite $\left(d_n\right)$ ? Justifier.
\item En déduire l'expression de $d_n$ en fonction de $n$.
\item On considère la fonction définie de la façon suivante en langage Python.

\begin{python}
1 def distance (k) :
2   d=30
3   n=1
4   while d<=k :
5      d=d*1.09
6 	   n=n+1
7   return n
\end{python}

Quelle information est obtenue par le calcul de distance(150) ?
\item Calculer la distance totale parcourue par le cycliste pendant les 20 premières semaines
d'entraînement.

\end{enumerate}

\vspace{0,5cm}

