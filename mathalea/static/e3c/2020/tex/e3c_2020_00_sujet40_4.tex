
\medskip

\begin{enumerate}
\item Soit $f$ la fonction définie sur $\R$ par $f(x) = 4x^3 - 48x^2 + 144x.$ 
	\begin{enumerate}
		\item Calculer $f'(x)$ et montrer que $f'(x) = 12\left(x^2 - 8x + 12\right)$.
		\item En déduire le tableau de variations de la fonction $f$ sur $\R$.
	\end{enumerate}
\item Dans une plaque de carton carrée de $12$~cm de côté, on découpe, aux quatre coins, des carrés identiques afin de construire une boîte sans couvercle, comme indiqué sur les figures ci-dessous.

On note $x$ la longueur (en cm) du côté de chacun des carrés découpés.

On admet que $x \in ]0~;~6[$.

\begin{center}
\psset{unit=0.3cm}
\begin{pspicture}(30,16)
%\psgrid
\psframe(1,1)(13,13)
\pspolygon[linestyle=dashed,fillstyle=solid,fillcolor=lightgray](3,1)(3,3)(1,3)(1,11)(3,11)(3,13)(11,13)(11,11)(13,11)(13,3)(11,3)(11,1)
\psline{<->}(11,0.5)(13,0.5)\uput[d](12,0.5){$x$}
\psline{<->}(13.4,1)(13.4,3)\uput[r](13.4,2){$x$}
\psline{<->}(0.3,1)(0.3,13)\psline{<->}(1,13.6)(13,13.6)
\uput[l](0.3,7){12}\uput[u](7,13.6){12}
\psframe[fillstyle=solid,fillcolor=lightgray](17,1)(25,3)
\pspolygon[fillstyle=solid,fillcolor=lightgray](25,1)(30,3)(30,5)(25,3)
\psline(17,3)(22,5)(30,5)
%\psline(12,3.8)(12,2.8)(14.85,2.8)
\psline(22,5)(22,3)
\psarc[linewidth=0.15cm]{<-}(15.5,0){5}{70}{110}
\end{pspicture}
\end{center}


L'objectif est de déterminer la longueur $x$ permettant d'obtenir une boîte de volume maximal.
	\begin{enumerate}
		\item Montrer que le volume de la boîte est égal à $100$ cm$^3$ pour $x = 1$. Détailler le calcul.
		\item Montrer que, pour $x \in ]0~;~6[$, le volume de la boîte est égal à $f(x)$,\: $f$ étant la fonction étudiée à la question 1.
		\item Quelle est la valeur de $x$ permettant d'obtenir une boîte de volume maximal ?
	\end{enumerate}
\end{enumerate}
