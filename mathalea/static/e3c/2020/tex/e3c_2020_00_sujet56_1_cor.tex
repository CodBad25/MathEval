
\subsection*{Question 1}

Seule l'affirmation \textbf{c.} est vraie.

\subsection*{Question 2}

On a \( \cos\left(\dfrac{2\pi}{3}\right) = -\dfrac{1}{2} \) et \( \dfrac{2\pi}{3} \in [0 \,;\, \pi] \). L'affirmation \textbf{b.} est vraie.

\subsection*{Question 3}

Sur la figure, on lit que le coefficient directeur de la tangente à la courbe au point d'abscisse 2 est 1. Donc \( f'(2) = 1 \).

\subsection*{Question 4}

\( g \) est dérivable sur \( \mathbb{R} \) et sur cet intervalle :
\begin{align*}
g'(x) &= 3x^2 - 0{,}0012 \\
&= 3\left(x^2 - 0{,}0004\right) \\
&= 3(x + 0{,}02)(x - 0{,}02).
\end{align*}

\( g'(x) \) est un trinôme du second degré de coefficient principal 3 \(> 0\), donc sa courbe représentative est une parabole dont la concavité est tournée vers le haut.

L'équation \( g'(x) = 0 \) a deux solutions \( -0{,}02 \) et \( 0{,}02 \). Ce trinôme est donc positif sauf entre les racines, donc \( g \) est décroissante sur l'intervalle \( ]-0{,}02 \,;\, 0{,}02[ \).

\subsection*{Question 5}

Seule l'affirmation \textbf{d.} est vraie.

