
\medskip


Ce QCM comprend 5 questions. Pour chacune des questions, une seule des quatre réponses proposées est correcte. Les questions sont indépendantes.

Pour chaque question, indiquer le numéro de la question et recopier sur la copie la lettre correspondant à la réponse choisie. Chaque réponse correcte rapporte $1$ point. Une réponse incorrecte ou une question sans réponse n'apporte ni ne retire de point.

\medskip

\begin{enumerate}
\item 
L'inéquation $2x^2 - 9x +4 \geqslant 0$ a pour ensemble de solutions :
	\begin{enumerate}
		\item $S= \left[\dfrac{1}{2}~;~4\right]$
		\item $S = \left ]-\infty~;~\dfrac{1}{2}\right ] \cup  [4~;~+\infty$[
		\item $S = \emptyset$ 
		\item $S = ]- \infty~;~-4] \cup \left [- \dfrac{1}{2}~;~+\infty\right [$
	\end{enumerate}
	
	\medskip
	
\item On considère la fonction $g$ définie sur l'ensemble des réels $\R$ par 

\[g(x) = -x^2 +4x\]

alors
	\begin{enumerate}
		\item le minimum de la fonction $g$ sur $\R$ est 4
		\item  le maximum de la fonction $g$ sur $\R$ est 4
		\item  le maximum de la fonction $g$ sur $\R$ est 2
		\item  $g$ est décroissante sur l'intervalle $[4~;~ +\infty[$
	\end{enumerate}
	
		\medskip

\item Le plan est rapporté à un repère orthonormé. La droite passant par le point A$(0~;~-7)$ et de vecteur normal $\vect{n}\begin{pmatrix}2\\-5\end{pmatrix}$ a pour équation
	\begin{enumerate}
		\item $2x - 5y - 35 = 0$ 
		\item $2x- 5y + 35 = 0$ 
		\item $-5x - 2y + 14 = 0$ 
		\item $5x+ 2y+ 14 = 0$
	\end{enumerate}
	
		\medskip

\item Le plan est rapporté à un repère orthonormé. L'ensemble des points $M$ de coordonnées $(x~;~y)$ telles que $x^2 - 4x +y^2 + 6y = 12$ est
	\begin{enumerate}
		\item le point de coordonnées (5~:~1)
		\item le cercle de centre A$(2~;~-3)$ et de rayon $\sqrt{12}$
		\item le cercle de centre A$(2~;~-3)$ et de rayon $5$
		\item le cercle de centre B$(-2~;~3)$ et de rayon $5$
	\end{enumerate}
	
		\medskip

\item Le plan est muni d'un repère orthonormé.

On considère la droite $d$ d'équation $2x + 3y - 1 = 0$.
	\begin{enumerate}
		\item La droite $d$ est perpendiculaire à la droite (AB), où A$(-2~;~3)$ et B(2~;~9).
		\item  Le vecteur $\vect{u}\begin{pmatrix}-3\\2\end{pmatrix}$ est un vecteur normal à la droite $d$.
		\item  La droite perpendiculaire à $d$ passant par le point $(-1~;~2)$ admet pour équation \\ $3x- 2y+1=0$.
		\item  La droite parallèle à $d$ passant par le point (2~;~3) admet pour équation $2x + 3y + 13 = 0$.
	\end{enumerate}
\end{enumerate}

\bigskip

