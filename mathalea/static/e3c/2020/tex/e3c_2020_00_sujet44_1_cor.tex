
\subsection*{Question 1}

\[
\cos(x) = -\dfrac{\sqrt{3}}{2} \quad \text{pour} \quad x = \pi - \dfrac{\pi}{6} = \dfrac{5\pi}{6}.
\]

\subsection*{Question 2}

On a \(\overrightarrow{AB} \begin{pmatrix} 4 - (-2) \\ -5 - 7 \end{pmatrix} \text{ soit } \overrightarrow{AB} \begin{pmatrix} 6 \\ -12 \end{pmatrix}\).

\subsection*{Question 3}

La droite a pour vecteur directeur \(\overrightarrow{u}\begin{pmatrix} 1 \\ -2 \end{pmatrix}\).

\subsection*{Question 4}

Le minimum du trinôme est obtenu lorsque le carré est nul, soit lorsque \( x = 2 \) et ce minimum est égal à 1.

L'écriture canonique est donc \( y = (x - 2)^2 + 1 \).

\subsection*{Question 5}

\begin{align*}
&M(x\,;\,y) \in \mathcal{C}_1 \\
\iff &(x + 2)^2 + (y - 3)^2 = 9 \\
\iff &(x - (-2))^2 + (y - 3)^2 = 3^2 \\
\iff &CM^2 = 3^2 \\
\iff &CM = 3,
\end{align*}
les points appartiennent au cercle de centre \( C(-2\,;\,3) \) et de rayon 3.

