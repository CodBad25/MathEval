
\medskip


Ce QCM comprend 5 questions.

Pour chacune des questions, une seule des quatre réponses proposées est correcte.
Les questions sont indépendantes.

Pour chaque question, indiquer le numéro de la question et recopier sur la copie la lettre correspondante à la réponse choisie.

Aucune justification n'est demandée mais il peut être nécessaire d'effectuer des recherches au brouillon pour aider à déterminer votre réponse.

Chaque réponse correcte rapporte 1 point. Une réponse incorrecte ou une question sans réponse n'apporte ni ne retire de point.

Dans cet exercice, on se place dans un repère orthonormé.

\medskip


\textbf{Question 1 :} Un vecteur normal à la droite d'équation cartésienne $2x - 5y + 3 = 0$ a pour coordonnées:

\begin{center}
\begin{tabularx}{\linewidth}{|*{4}{X|}}\hline
\textbf{a.~~}$\begin{pmatrix}-5\\2 \end{pmatrix}$&\textbf{b.~~}$\begin{pmatrix}2\\5 \end{pmatrix}$&\textbf{c.~~}$\begin{pmatrix}5\\2 \end{pmatrix}$&\textbf{d.~~}$\begin{pmatrix}-2\\5 \end{pmatrix}$\\ \hline
\end{tabularx}
\end{center}

\textbf{Question 2 :} Le centre A du cercle d'équation $x^2+y^2 + 6x - 8y = 0$ est:

\begin{center}
\begin{tabularx}{\linewidth}{|*{4}{X|}}\hline
\textbf{a.~~}A(3~;~4)&\textbf{b.~~}A$(-3~;~4)$&\textbf{c.~~}A$(-4~;~3)$&\textbf{d.~~}A$(4~;~-3)$\\ \hline
\end{tabularx}
\end{center}

\textbf{Question 3 :} On considère un triangle ABC tel que AB $= 3$, BC $= 5$ et AC $= 6$, on a alors $\vect{\text{AB}} \cdot \vect{\text{AC}}$ égal à :

\begin{center}
\begin{tabularx}{\linewidth}{|*{4}{X|}}\hline
\textbf{a.~~}$-18$&\textbf{b.~~}$10$&\textbf{c.~~}$26$&\textbf{d.~~}$0$\\ \hline
\end{tabularx}
\end{center}

\textbf{Question 4 :} Le nombre réel $\dfrac{- 3\pi}{4} $ est associé au même point du cercle trigonométrique que le réel:

\begin{center}
\begin{tabularx}{\linewidth}{|*{4}{X|}}\hline
\textbf{a.~~}$\dfrac{- 14\pi\rule{0pt}{12pt}}{4\rule[-5pt]{0pt}{0pt}}$&\textbf{b.~~}$\dfrac{7\pi}{4}$&\textbf{c.~~}$\dfrac{13\pi}{4}$&\textbf{d.~~}$\dfrac{19\pi}{4}$\\ \hline
\end{tabularx}
\end{center}

\textbf{Question 5 :} La fonction $g$ définie sur $\R$ par $g(x) = (4x - 7)^3$ a pour fonction dérivée :

\begin{center}
\begin{tabularx}{\linewidth}{|*{2}{X|}}\hline
\textbf{a.~~}$g'(x) = 3(4x - 7)^2$&\textbf{b.~~}$g'(x) = 12(4x - 7)$\\ \hline
\textbf{c.~~}$g'(x) = 12x - 21$&\textbf{d.~~}$g'(x) =  12(4x - 7)^2$\\ \hline
\end{tabularx}
\end{center}

\bigskip

