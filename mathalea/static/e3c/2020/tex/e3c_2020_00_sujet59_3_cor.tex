
\( f(x) = (x^2 - 2{,}5x + 1) \e^x \).

\subsection*{1.}

\paragraph{a.} \(f\) est dérivable sur \( \mathbb{R} \) comme produit de fonctions dérivables sur cet intervalle :
\begin{align*}
f'(x) &= (2x - 2{,}5) \e^x + (x^2 - 2{,}5x + 1) \e^x \\
&= \e^x (2x - 2{,}5 + x^2 - 2{,}5x + 1) \\
&= \e^x (x^2 - 0{,}5x - 1{,}5).
\end{align*}

\paragraph{b.} On sait que \( \e^x > 0 \) quel que soit le réel \( x \), donc \( f'(x) \) a le signe du trinôme \( x^2 - 0{,}5x - 1{,}5 \).

On a :
\[
\Delta = 0{,}5^2 - 4 \times (-1{,}5) = 0{,}25 + 6 = 6{,}25 = (2{,}5)^2.
\]
L'équation \( x^2 - 0{,}5x - 1{,}5 = 0 \) a donc deux racines :
\[
x_1 = \dfrac{0{,}5 + 2{,}5}{2} = 1{,}5 \quad \text{et} \quad x_2 = \dfrac{0{,}5 - 2{,}5}{2} = -1.
\]

On sait de plus que ce trinôme est positif sauf sur l'intervalle \( \,] -1 \,;\, 1{,}5[ \).

La fonction \(f\) est donc croissante sauf sur l'intervalle \( \,] -1 \,;\, 1{,}5[ \).

\subsection*{2.}

\paragraph{a.} On a :
\[
M(x \,;\, y) \in \mathcal{T} \iff y - f(0) = f'(0)(x - 0).
\]

Avec \(f(0) = 1 \times \e^0 = 1\) et \(f'(0) = -1{,}5 \times \e^0 = -1{,}5\), on a donc :
\[
M(x \,;\, y) \in \mathcal{T} \iff y - 1 = -1{,}5(x - 0) \text{ ou encore } y = -1{,}5x + 1.
\]

\paragraph{b.} Le point d'abscisse \(a\) appartenant à la fois à la courbe \(\mathcal{C}_f\) et à la tangente \(\mathcal{T}\), cette abscisse vérifie à la fois l'équation de \(f\) et celle de \(\mathcal{T}\), soit :
\[
(a^2 - 2{,}5a + 1) \e^a = -1{,}5a + 1 \text{ ou } (a^2 - 2{,}5a + 1) \e^a + 1{,}5a - 1 = 0.
\]

On peut entrer sur la calculatrice la fonction \( x \longmapsto g(x) = (x^2 - 2{,}5x + 1) \e^x + 1{,}5x - 1 \) et essayer de chercher pour quelle valeur cette fonction s'annule (en dehors de 0). On a \( g(1{,}7) \approx -0{,}42 \) et \( g(1{,}8) \approx 0{,}13 \). Donc \( 1{,}7 < a < 1{,}8 \).

\( g(1{,}77) \approx -0{,}0599 \) et \( g(1{,}78) \approx 0{,}0002 \), donc \( 1{,}77 < a < 1{,}78 \).

Donc \( a \approx 1{,}8 \) au dixième près.

