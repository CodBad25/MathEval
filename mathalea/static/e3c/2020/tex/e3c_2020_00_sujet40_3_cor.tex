
\subsection*{1.}

Le prix en 2021 devrait être : \(u_2 = 4200 \times 1{,}03^2 = 4455{,}78\) (€).

\subsection*{2.}

\paragraph{a.} Puisque \( u_{n + 1} = 1{,}03u_n \), la suite \((u_n)\) est une suite géométrique de raison \(q = 1{,}03\).

\paragraph{b.} On sais que quel que soit \(n \in \mathbb{N}\), \(u_n = u_0 \times 1{,}03^n = 4200 \times 1{,}03^n\).

\paragraph{c.} En 2024, le prix au mètre carré (€/m²) sera :
\[
u_5 = 4200 \times 1{,}03^5 \approx 4868{,}95.
\]
Pour 40 m², cela donne :
\[
4868{,}95 \times 40 = 194758 \, \text{€},
\]
donc moins de 200000 €, on pourra acheter l'appartement.

\subsection*{3.}

\begin{center}
\begin{python}
def seuil() :
    u = 4200
    n = 0
    while u <= 8000 :
        u = u * 1.03
        n = n + 1
    return n
\end{python}
\end{center}

