	\section*{Exercice 2 (5 points)}


\subsection*{1. Calculer le nombre d’abonnés en 2020 et 2021.}
	
\[	u_1 = 6000 + 750 = 6750\]
\[	u_2 = 6750 + 750 = 7500\]
	
	\subsection*{2. Exprimer $u_{n+1}$ en fonction de $u_n$.}
$u_{n+1} = u_n + 750$.
\subsection*{3. Quelle est la nature de la suite $(u_n)$ ?}
$(u_n)$ est une suite arithmétique de raison 750 de premier terme 6 000.
\subsection*{4. En déduire une expression de $u_n$ en fonction de $n$.}
On a $u_n = 6000 + 750n$.
\subsection*{5. En quelle année le nombre d’abonnés aura triplé par rapport à l’année 2019 ?}
Il faut trouver $n$ tel que $u_n = 3 \times 6000$,\\ soit $6000 + 750n = 18000$, donc si $750n = 12000$ ou
	\[
	n = \dfrac{12000}{750} = 16
	\]
Le nombre d’abonnés aura triplé en 2035.
	
