
\begin{center}
\textbf{Chute d'une bille}
\end{center}

\begin{minipage}{0.7\linewidth}

On filme la chute d'une bille d'acier dans l'huile d'olive contenue dans une éprouvette graduée. La bille est lâchée sans vitesse initiale par un électroaimant dans le référentiel terrestre supposé galiléen.

\medskip

La vidéo est ensuite analysée image par image à l'aide d'un logiciel approprié qui permet de repérer la position instantanée du centre G de la bille suivant un axe (O$y$) vertical orienté vers le bas.

\medskip

L'évolution, au cours du temps, de la valeur expérimentale de la vitesse $v$ de la bille est représentée ci-dessous :

\end{minipage}\hfill 
\begin{minipage}{0.27\linewidth}
\scalebox{0.7}{
\psset{unit=1cm}
\def\xmin {0}   \def\xmax {5}
\def\ymin {0}   \def\ymax {10}
\begin{pspicture}(\xmin,\ymin)(\xmax,\ymax)
\psframe[fillstyle=solid,fillcolor=lightgray](2,1)(3,7.5)                               
\psline[linewidth=2pt](1.75,8.25)(2,8)(2,1)(1.5,0.75)(3.5,0.75)(3,1)(3,8)(3.25,8.25)
\psdots[dotstyle=o,dotscale=2.5,linewidth=1.5pt](2.5,7.5)(2.5,6.9)(2.5,6.2)(2.5,5.5)(2.5,4.7)(2.5,3.8)(2.5,2.8)(2.5,1.7)
\psline[linewidth=2pt]{->}(4,8)(4,0.5)
\psline[linestyle = dashed](3,7.5)(4,7.5) 
\uput[r](4,7.5){\bf O} \uput[r](4,0.5){$\boldsymbol{y}$}
\uput[u](2.5,8.9){\bf Chronophotographie} \uput[d](2.5,9.1){\bf de la chute de la bille} 
\end{pspicture}
}%%% fin du scalebox
%\end{center}
\end{minipage}

\begin{center}
\psset{xunit=10cm,yunit=5cm,comma=true}
\begin{pspicture}(-0.11,-0.15)(0.9,1.3)
\multido{\n=0.00+0.02}{46}{\psline[linewidth=0.25pt,linecolor=orange](\n,0)(\n,1.2)}
\multido{\n=0.00+0.04}{31}{\psline[linewidth=0.25pt,linecolor=orange](0,\n)(0.9,\n)}
\psaxes[linewidth=1.25pt,Dx=0.1,Dy=0.2]{->}(0,0)(0,0)(0.9,1.3)
\psdots[dotstyle=+](0.04,0.29)(0.08,0.51)(0.12,0.63)(0.16,0.75)(0.2,0.82)
(0.24,0.87)(0.28,0.89)(0.32,0.9)(0.36,0.91)(0.4,0.93)
(0.44,0.93)(0.48,0.95)(0.52,0.95)(0.56,0.95)(0.60,0.96)
(0.64,0.96)(0.68,0.95)(0.72,0.97)(0.76,0.96)(0.8,0.96)
\rput(0.45,1.3){\textbf{Évolution de la vitesse de la bille au cours du temps}}
\rput(0.45,-0.2){\textbf{temps (s)}}
\rput{90}(-0.15,0.6){\textbf{vitesse (m.s$^{-1}$)}}
\end{pspicture}
\end{center}

\medskip

\begin{enumerate}
\item Déterminer graphiquement la valeur de la vitesse limite $v_{\text{lim}}$  atteinte par la bille.
\end{enumerate}

Pour établir l'expression de la vitesse de la bille, les données physiques de l'expérience conduisent à résoudre l'équation différentielle $(E) : y' = - 9y + 8,6$.

\begin{enumerate}[start=5]
\item Déterminer la fonction solution de l'équation différentielle $(E)$ s'annulant en $0$.

\item Montrer que la limite de $0,96 \left(1 - \e^{-9t}\right)$ lorsque $t$ tend vers $+\infty$ est égale à $0,96$.
\end{enumerate}

\bigskip


