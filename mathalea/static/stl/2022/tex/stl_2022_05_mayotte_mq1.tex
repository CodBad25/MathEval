
\medskip

On donne, ci-dessous $\mathcal{C}_h$, la courbe représentative d'une fonction $h$, définie et dérivable sur l'intervalle $[-5~;~1]$. On a tracé une partie de la droite, notée $T$, tangente à la courbe $\mathcal{C}_h$ au point d'abscisse $-2$.

\begin{center}
\psset{unit=1cm,arrowsize=2pt 3}
\begin{pspicture}(-5,-6)(1,1)
\psgrid[gridlabels=0pt,subgriddiv=1,gridwidth=1pt,griddots=10,gridcolor=gray]
\psaxes[linewidth=1.25pt]{->}(0,0)(-5,-6)(1,1)
\uput[u](0.8,0){$x$}
\uput[r](0,0.75){$y$}
\psecurve[linewidth=1.25pt,linecolor=blue](-6,-2.1)(-5,-2)(-4,-2.1)(-3,-2.5)(-2,-4)(-1.5,-4.5)(-1,-4.3)(0,-2.65)(1,-1)(2,0)
\uput[d](-4.5,-2){\blue $\mathcal{C}_h$}\uput[ur](-3,-2){$T$}
\psplot[linewidth=1.25pt]{-3.5}{-1}{2 x mul 8 add neg}
\end{pspicture}
\end{center}

Les points A$(-3~;~-2)$ et B$(-2~;~-4)$ appartiennent à la droite $T$.

\begin{enumerate}
\item Déterminer l'équation réduite de la droite $T$.
\item En déduire la valeur exacte de $h'(-2)$.
\item Déterminer les coordonnées des points d'intersection de la droite $T$ avec chacun des axes du repère.
\end{enumerate}

\bigskip

