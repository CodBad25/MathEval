
\medskip

\textbf{Question 1}

\medskip

Dérivée d'un produit: 
\[f'(x)= 8\times \e^{-x} + (8x-2) \times (-1)\e^{-x}
= (8 -8x+2)\e^{-x}
= (10 -8x)\e^{-x}.\]

\bigskip

\textbf{Question 2}

\medskip

Pour tout réel $x$, on a $\e^{-x}>0$ donc :
\[f(x) = 0 \iff 8x-2 = 0 \iff x = 0,25.\]

\bigskip

\textbf{Question 3}

\medskip

Sur $[0\;;\;4[$, $g'(x) > 0$ donc la fonction $g$ est croissante ; Julien a tort.

\bigskip

\textbf{Question 4}

\medskip

Pour $a>0$, $\ln\left (\sqrt{a}\right )=\dfrac{1}{2} \ln(a)$ donc
$\dfrac{\ln \left(\sqrt{8}\right)}{\ln \left(\sqrt{2}\right)}
= \dfrac{\frac{1}{2} \ln(8)}{\frac{1}{2}\ln(2)} 
= \dfrac{\ln(8)}{\ln(2)}.$

Pour $a>0$, $\ln\left (a^3\right ) = 3\ln(a)$ donc
$\ln(8) = \ln\left (2^3\right ) = 3\ln(2)$ et donc
$\dfrac{\ln(8)}{\ln(2)} = \dfrac{3\ln(2)}{\ln(2)}=3.$

\bigskip

\textbf{Question 5}

\medskip

$\ds\lim_{x\to -\infty} 6x = -\infty$ et
$\ds\lim_{x\to -\infty}\e^{X}=0$ donc
$\ds\lim_{x\to -\infty} \e^{6x}=0$.

On déduit que $\ds\lim_{x\to -\infty} \e^{6x}-1=-1$ et donc que
$\ds\lim_{x\to -\infty} f(x) = -1$.
\bigskip

\textbf{Question 6}

\medskip

Le vecteur $\vectt{BD}$ se projette en $\vectt{BC}$ sur (BF); donc
$\vect{\text{BF}}  \cdot \vect{\text{BD}} = \vectt{BF} \cdot \vectt{BC}$.

Les vecteurs $\vectt{BF}$ et $\vectt{BC}$ sont colinéaires et de sens contraires donc
$\vectt{BF} \cdot \vectt{BC} = - \text{BF} \times \text{BC}$.

D'après la figure; $\text{BF}=3$ et $\text{BC}=4$;
on en conclut que $\vect{\text{BF}}  \cdot \vect{\text{BD}} =-12$.

\bigskip


