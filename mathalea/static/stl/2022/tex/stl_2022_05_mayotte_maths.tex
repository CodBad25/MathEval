
\medskip

\textbf{Vous traiterez 4 questions au choix parmi les 6 questions proposées.}

\bigskip

\textbf{Pour les questions 1 et 2 uniquement :}

\medskip

On donne, ci-dessous $\mathcal{C}_h$, la courbe représentative d'une fonction $h$, définie et dérivable sur l'intervalle $[-5~;~5]$. On a tracé une partie de la droite, notée $T$, tangente à la courbe $\mathcal{C}_h$ au point d'abscisse $-2$.

\begin{center}
\psset{unit=1cm,arrowsize=2pt 3}
\begin{pspicture}(-5,-6)(6,4)
\psgrid[gridlabels=0pt,subgriddiv=1,gridwidth=1pt,griddots=10,gridcolor=gray]
\psaxes[linewidth=1.25pt]{->}(0,0)(-5,-6)(6,4)
\uput[u](5.8,0){$x$}
\uput[r](0,3.75){$y$}
\psecurve[linewidth=1.25pt,linecolor=blue](-6,-2.1)(-5,-2)(-4,-2.1)(-3,-2.5)(-2,-4)(-1.5,-4.5)(-1,-4.3)(0,-2.65)(1,-1)(2,0)(3,1)(4,1.8)(5,2)(6,1.9)
\uput[d](-4.5,-2){\blue $\mathcal{C}_h$}\uput[ur](-3,-2){$T$}
\psplot[linewidth=1.25pt]{-3.5}{-1}{2 x mul 8 add neg}
\end{pspicture}
\end{center}

\bigskip

\textbf{Question 1}

\medskip

Les points A$(-3~;~-2)$ et B$(-2~;~-4)$ appartiennent à la droite $T$.

\begin{enumerate}
\item Déterminer l'équation réduite de la droite $T$.
\item En déduire la valeur exacte de $h'(-2)$.
\item Déterminer les coordonnées des points d'intersection de la droite $T$ avec chacun des axes du repère.
\end{enumerate}

\bigskip

\textbf{Question 2}

\medskip

Soit $H$ une primitive de $h$ sur l'intervalle $[-5~;~5]$.

À l'aide du graphique, donner le sens de variation de la fonction $H$ sur l'intervalle $[-5~;~5]$.

\bigskip

\textbf{Question 3}

\medskip

On considère l'équation différentielle $(E)$ suivante:

\[y' = -0,04y + 0,8 \qquad (E)\]

Déterminer $f$ la solution de l'équation différentielle $(E)$ sur l'intervalle $[0~;+\infty[$, qui vérifie la condition initiale $f(0) = 100$.

\bigskip

\textbf{Question 4}

\medskip

Soit $f$ la fonction définie et dérivable sur $\R$ par :
\[f(x)= (x + 1)\e^{-x}.\]

\begin{enumerate}
\item Montrer que, pour tout $x$ réel, $f'(x) = -x\e^{-x}$.
\item En déduire les variations de $f$ sur $\R$.
\end{enumerate}

\bigskip

\textbf{Pour les questions 5 et 6 uniquement :}

\medskip

On note $L$ le niveau sonore en dB et $I$ l'intensité sonore en W$\cdot \text{m}^{-2}$ d'un son. On désigne par Log la fonction logarithme décimal. On a la relation suivante :
\[L = 10 \text{\,Log\,}\left(\dfrac{I}{I_0}\right), \quad\text{où}  \:I_0 =  10^{-12} \text{ W} \cdot \text{m}^{-2}.\]

\bigskip

\textbf{Question 5}

\medskip

\begin{enumerate}
\item Quel est le niveau sonore $L$ d'un son d'intensité sonore $I = 10^{-5} \text{ W} \cdot \text{m}^{-2}$ ?
\item Une sirène d'alarme a un niveau sonore de $130$ dB.

Quelle est son intensité sonore $I$ ?
\end{enumerate}

\bigskip

\textbf{Question 6}

\medskip

On souhaite faire baisser le niveau sonore de $10$ dB.

On note $L' = L - 10$ et on note $I'$ l'intensité sonore correspondant à $L'$.

C'est-à-dire :
\[L'= 10\text{\,Log\,}\left(\dfrac{I'}{I_0}\right).\]

Exprimer $I'$ en fonction de $I$.

\bigskip


