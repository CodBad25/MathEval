
\medskip

Les solutions de  l'équation différentielle $y'+ay=b$ sur  $\R$ sont les fonctions $y$ définies par :
\[y(x)=C\e^{-ax}+\dfrac{b}{a}, \text{ où } C \text{ est une constante quelconque.}\]

Ici $a=\np{0.04}\ \text{ et } b=\np{0.8}$. Par conséquent, sur $[0~;~+\infty[$ :
\[f(x) = C\e^{-\np{0.04}x} + \dfrac{\np{0.8}}{\np{0.04}} = C \e^{-\np{0.04}x}+20.\]

Déterminons $C$ :
\[f(0) = C\e^{-\np{0.04} \times 0} + 20 = C + 20 = 100 \quad \text{d'où}\quad C = 80.\]

Au final : $f(x) = 80\e^{-0,04 x} + 20.$

\bigskip

