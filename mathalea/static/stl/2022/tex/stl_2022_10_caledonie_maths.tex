
\medskip

\textbf{Vous traiterez 4 questions au choix parmi les 6 questions proposées.}

\medskip

\textbf{Question 1}

\medskip

Écrire sur la copie le numéro de la question ainsi que la lettre correspondant à la bonne réponse. \textbf{Aucune justification n'est attendue.}

\medskip

On donne ci-dessous un tracé de la courbe représentative $\mathcal{C}_f$ d'une fonction $f$ définie sur $\R$ :

\medskip

\begin{minipage}{0.4\linewidth}

$f'(-2) =$

\medskip

\textbf{a.~~} 0

\textbf{b.~~} 2

\textbf{c.~~} $-1$

\textbf{d.~~}  $-2,25$

\end{minipage}\hfill
\begin{minipage}{0.56\linewidth}
\psset{unit=1cm,arrowsize=2pt 3}
\begin{pspicture*}(-3,-3)(2,4)
\psgrid[gridlabels=0pt,subgriddiv=1,gridwidth=0.1pt,gridcolor=orange]
\psaxes[linewidth=1.25pt,labelFontSize=\scriptstyle]{->}(0,0)(-3,-3)(2,4)
\uput[u](1.8,0){$x$} \uput[r](0,3.8){$y$}
\psplot[plotpoints=2000,linewidth=1.25pt]{-3}{0.5}{x 2 mul  3  add }
\psplot[plotpoints=2000,linewidth=1.25pt]{-3}{2}{x 2  add neg}
\psbezier[linewidth=1.25pt,linecolor=blue](-2.5,-3)(-2.05,0)(-1.5,-0.5)(-1,-1)
\psbezier[linewidth=1.25pt,linecolor=blue](-1,-1)(-0.43,-1.45)(-0.56,-1.75)(0,-1)
\psbezier[linewidth=1.25pt,linecolor=blue](0,-1)(1,1)(0.83,5)(0.89,4)
\uput[r](0.75,2.75){\blue $\mathcal{C}_f$}
\psdots(-2,-1)(-1,-1)(0,-1)
%\psplot[plotpoints=2000,linewidth=1.25pt,linestyle=dashed]{-3}{1}{x 3 exp 1.533 mul x dup mul 4.1 mul add x 1.567 mul add 1 sub}
\end{pspicture*}
\end{minipage}

\bigskip

\textbf{Question 2}

\medskip

Écrire sur la copie le numéro de la question ainsi que la lettre correspondant à la bonne réponse. \textbf{Aucune justification n'est attendue.}

\medskip

On considère l'équation $\ln (x) = 7$. Cette équation admet pour solution :

\medskip

\textbf{a.~~} $\ln (7)$

\medskip

\textbf{b.~~} $\ln \left(\text{e}^7\right)$

\medskip

\textbf{c.~~} $\text{e}^7$

\medskip

\textbf{d.~~} $\dfrac17$

\bigskip

\textbf{Question 3}

\medskip

On considère la fonction $f$ définie sur $\R$ par $f(x) = x\text{e}^{2x}$.

Déterminer $f'(x)$, où $f'$ est la fonction dérivée de la fonction $f$.

\textbf{Justifier la réponse}.

\bigskip

\textbf{Question 4}

\medskip

Soit ABCD un carré de côté $4$~cm. Calculer le produit scalaire $\vect{\text{AB}}\cdot \vect{\text{AC}}$.

\textbf{Justifier la réponse}.

\bigskip

\textbf{Question 5}

\medskip

On considère l'équation différentielle suivante : 
\[v' = -4,5v + 6,3 \qquad(E)\]

Déterminer la fonction $v$ solution de l'équation (E) et vérifiant la condition initiale $v(0) = 0$. 

\textbf{Justifier la réponse}.

\bigskip

\textbf{Question 6}

\medskip

Afin d'étudier l'évolution d'une population de bactéries à l'intérieur d'une boîte fermée, on considère la fonction $f$ définie pour tout $t \geqslant 0$ par :
\[f(t) = \dfrac{100}{1 + \text{e}^{- 1,3t}}\]

où $f(t)$ désigne le nombre de bactéries (exprimé en millier) à l'instant $t$ (exprimé en heure).

\medskip

\begin{minipage}{0.58\linewidth}
Le programme en Python ci-contre affiche la valeur de $t$ (arrondie à
l'unité) à partir de laquelle le nombre de bactéries à l'intérieur
de l'enceinte dépasse \np{99000}.

Quelle est la valeur affichée lorsqu'on exécute ce programme ?

\end{minipage}\hfill
\begin{minipage}{0.38\linewidth}
\begin{tabular}{|l|}\hline
from math import exp\\
T=0\\
while 100/(1+exp(-1,3*T)) <= 99 :\\
\qquad T = T+1\\
print (T)\\ \hline
\end{tabular}
\end{minipage}

\bigskip


