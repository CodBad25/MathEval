
\medskip

On s'intéresse à l'énergie stockée dans la batterie d'un téléphone portable. Cette grandeur s'exprime en kW\cdot h. Lorsque la batterie est totalement chargée, l'énergie stockée vaut 0,715 kW\cdot h.

Lors du branchement de la batterie vide sur une borne de recharge, l'énergie stockée dans la batterie (en kW\cdot h) en fonction du temps $t$ (en heure) est modélisée par une fonction $f$ telle que, pour tout nombre réel $t \geqslant 0$ :
\[f(t) = -0,715 \e^{-t} + 0,715.\]

La durée de demi-charge est le temps nécessaire pour charger à 50\,\% une batterie qui était vide au départ. 

\smallskip

Déterminer la durée de demi-charge de la batterie de ce téléphone en minute et seconde, arrondie à la seconde.

\bigskip

