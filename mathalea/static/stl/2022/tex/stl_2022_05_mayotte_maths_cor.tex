
\medskip

\textbf{Question 1}

\medskip

\begin{enumerate}
\item L'équation réduite d'une droite est de la forme $y=mx + p$ où :
\[m = \dfrac{y_{\text{B}}-y_{\text{A}}}{x_{\text{B}}-x_{\text{A}}} = \dfrac{-4-(-2)}{-2-(-3)} = -2\]
 
Écrivons qu'elle passe par A$(-3~;~-2)$ : $-2 = -2 \times (-3) + p$ d'où $p = -8$. 

L'équation réduite de (AB) est : $y = - 2x - 8$.
\item Il en résulte que la valeur exacte de $h'(-2)$ est $-2$, puisque le nombre dérivé de la fonction en $a$ est le coefficient directeur de la tangente à la courbe en ce point.
\item Déterminons les coordonnées des points d'intersection de la droite $T$ avec chacun des axes du repère.
\begin{itemize}
\item Avec l'axe des abscisses, $y = 0$ soit $-2x-8 = 0$ d'où $x = - 4$. Le point a pour coordonnées $(-4~;~0)$.
\item Avec l'axe des ordonnées, $x = 0$ soit $y = -8$. Le point a pour coordonnées $(0~;~-8)$.
\end{itemize}
\end{enumerate}

\bigskip

\textbf{Question 2}

\medskip

Si pour tout $x\in I,\:f'(x) < 0$ alors $f$ est  strictement décroissante sur $I$. 

Sur $]-5~;~2[,\:h(x) < 0$ par conséquent $H$ est strictement décroissante sur cet intervalle.

Si pour tout $x\in I, \:f'(x) > 0$ alors la fonction $f$ est strictement croissante sur $I$.

Sur $]2~;~5],\:h(x)>0$ par conséquent $H$ est strictement croissante sur  cet intervalle.

\bigskip

\textbf{Question 3}

\medskip

Les solutions de l'équation différentielle $y'+ay=b$ sur $\R$ sont les fonctions $y$ définies par :
\[y(x)=C\e^{-ax}+\dfrac{b}{a}, \text{ où } C \text{ est une constante quelconque.}\]

Ici $a=\np{0.04}\ \text{ et } b=\np{0.8}$. Par conséquent, sur $[0~;~+\infty[$ :
\[f(x) = C\e^{-\np{0.04}x} + \dfrac{\np{0.8}}{\np{0.04}} = C \e^{-\np{0.04}x}+20.\]

Déterminons $C$ :
\[f(0) = C\e^{-\np{0.04} \times 0} + 20 = C + 20 = 100 \quad \text{d'où}\quad C = 80.\]

Au final : $f(x) = 80\e^{-0,04 x} + 20.$

\bigskip

\textbf{Question 4}

\medskip

\begin{enumerate}
\item Si $f = uv$ alors $f' = u'v + uv'$, d'où :
\[f'(x) = 1 \times \e^{-x} + (x + 1) \times (-1 \times \e^{-x}) = (1 - x -1)\e^{-x} = -x\e^{-x}.\]
\item Comme $\e^{-x} > 0$, le signe de $f'(x)$ est celui de $-x$.

Pour tout $x\in \R_-^*,\; f'(x)>0$. Par conséquent $f$ est strictement croissante sur $\R_-^*$.

Pour tout $x\in \R_+^*,\; f'(x)<0$. Par conséquent $f$ est strictement décroissante sur $\R_+^*$.
\end{enumerate}

\bigskip

\textbf{Question 5}

\medskip

\begin{enumerate}
\item Remplaçons $I$ par sa valeur :
\[L = 10 \text{\,Log\,}\left(\dfrac{10^{-5}}{10^{-12}}\right) = 10 \text{\,Log\,} 10^7 = 10\times 7 = 70.\]
Le niveau sonore  est de \np[dB]{70}.
\item Résolvons l'équation en $I$ :
\begin{align*}
10 \text{\,Log\,} \dfrac{I}{10^{-12}}&=130\\
\text{\,Log\,} \dfrac{I}{10^{-12}}&=13\\
\dfrac{I}{10^{-12}}&=10^{13}\\
I&=10^{13}\times 10^{-12}\\
I&=10
\end{align*}

Son intensité sonore $I$ est $\np[W\cdot m^{-2}]{10}$.
\end{enumerate}

\bigskip

\textbf{Question 6}

\medskip

Exprimons $I'$ en fonction de $I$ :
\begin{align*}
L' &= L - 10\\
10\text{\,Log\,}\left(\frac{I'}{I_0}\right)&=10\text{\,Log\,}\left(\frac{I}{I_0}\right)-10\\
\text{\,Log\,}\left(\frac{I'}{I_0}\right)&=\text{\,Log\,}\left(\frac{I}{I_0}\right)-1\\
\text{\,Log\,}\left(\frac{I'}{I_0}\right)-\text{\,Log\,}\left(\frac{I}{I_0}\right)&=-1\\
\text{\,Log\,}\dfrac{\left(\frac{I'}{I_0}\right)}{\left(\frac{I}{I_0}\right)}&=-1\\
\dfrac{I'}{I}&=10^{-1}\\
I'&=I\times 10^{-1} = \dfrac{I}{10}.
\end{align*}
Donc baisser le niveau sonore de $10$ dB reviens à diviser l'intensité sonore par $10$.

\bigskip


