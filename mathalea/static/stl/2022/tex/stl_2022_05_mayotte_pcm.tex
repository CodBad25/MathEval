
\begin{center}
\textbf{Le carbure de silicium SIC}
\end{center}

Le carbure de silicium, de formule SIC, a été découvert par Jöns Jacob Berzelius en 1824 lors d'une expérience pour synthétiser du diamant. Il est devenu un matériau
incontournable pour la fabrication d'instruments d'optique. Par exemple, il a été utilisé pour garantir la stabilité thermomécanique du télescope spatial infrarouge Hershel, développé par l'agence spatiale européenne et lancé en 2009. En particulier la face optique des miroirs peut être revêtue de carbure de silicium par dépôt chimique en phase vapeur (ou CVD pour l'anglais \og chemical vapor deposition \fg) afin de masquer toute porosité résiduelle et obtenir une surface polissable parfaite.

Dans ce procédé, un solide inerte servant de support est exposé à une ou plusieurs espèces chimiques en phase gazeuse qui se décomposent à sa surface pour former le matériau désiré. Parmi celles-ci, le méthyltrichlorosilane de formule CH$_3$SiCl$_3$ est très souvent choisi. Par la suite, pour des raisons de simplification, il sera noté MTS.

\medskip

On considère une enceinte vide, de volume constant, thermostatée à la température
$T_2 = \np{1200}$ K, dans laquelle, au temps $t = 0$ min, on introduit une certaine quantité de MTS. 

À cette température, la transformation permettant la formation de carbure de silicium peut être considérée comme totale.

L'équation de la réaction modélisant la transformation chimique au cours de laquelle le MTS se décompose est la suivante :

\[\text{CH}_3\text{SiCl}_3  \text{(g)} \to  \text{SiC}(s) + 3 \text{HCl (g)}\]

On modélise la concentration en MTS exprimée en mol\cdot L$^{-1}$ en fonction du temps $t$ exprimé en minute, par la fonction $C$, définie sur l'intervalle [0~;~50] par :
\[C(t) = 0,30 \cdot \e^{-0,035t}\]

\begin{enumerate}[start=5]
\item On note $C'$ la fonction dérivée de la fonction $C$ sur l'intervalle [0~;~50].

Déterminer l'expression de $C'(t)$ pour $t$ appartenant à [0~;~50].
\item On rappelle que la vitesse de disparition de MTS est égale à l'opposé de la fonction dérivée $C'$. On note $C''$ la fonction dérivée de $C'$.

On admet que $C''(t) = 3,675 \cdot 10^{-4}\e^{-0,035t}$ pour $t$ appartenant à [0~;~50].

Etudier le sens de variation de la vitesse de réaction au cours du temps.
\item On considère que la transformation chimique de décomposition de MTS peut être stoppée lorsqu'il ne reste que 10\,\% de la concentration initiale de MTS.

Déterminer l'instant $t$ à partir duquel la transformation chimique peut être stoppée. On donnera la valeur exacte, puis la valeur arrondie à la minute près.
\end{enumerate}

\bigskip


