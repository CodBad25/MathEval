
\medskip

On s'intéresse à l'énergie stockée dans la batterie d'un téléphone portable. Cette grandeur s'exprime en kW\cdot h. Lorsque la batterie est totalement chargée, l'énergie stockée vaut 0,715 kW\cdot h.

Lors du branchement de la batterie vide sur une borne de recharge, l'énergie stockée dans la batterie (en kW\cdot h) en fonction du temps $t$ (en heure) est modélisée par une fonction $f$ telle que, pour tout nombre réel $t \geqslant 0$ :
\[f(t) = -0,715 \e^{-t} + 0,715.\]

On considère la fonction en langage Python suivante :

\begin{center}
\fbox{\begin{tabular}{l}
from math import exp\\
def temps(pourcentage) :\\
\quad t = 0\\
\quad y = 0\\
while y < pourcentage$*0.715$:\\
\qquad t = t+1/60\\
\qquad y = - 0.715*exp(-t)+0.715\\
return(t)
\end{tabular}
}
\end{center}

\smallskip

Que renvoie l'exécution de l'instruction {\ttfamily temps(0.15)} ? 

Interpréter ce résultat dans le contexte de l'exercice.

\bigskip

