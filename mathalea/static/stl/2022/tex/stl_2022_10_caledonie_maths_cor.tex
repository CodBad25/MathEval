
\medskip

\textbf{Question 1}

\medskip

\begin{minipage}{0.4\linewidth}

$f'(-2)$ est, par définition, égal au coefficient directeur $m$ de la droite (AB).

\medskip

\[m = \dfrac{y_B - y_A}{x_B - x_A} = \dfrac{3-(-1)}{0-(-2)} = 2\]

\medskip

D'où : $f'(-2) = 2$

La bonne réponse est \textbf{b.}

\end{minipage}\hfill
\begin{minipage}{0.56\linewidth}
\psset{unit=1cm,arrowsize=2pt 3}
\begin{pspicture*}(-3,-3)(2,4)
\psgrid[gridlabels=0pt,subgriddiv=1,gridwidth=0.1pt,gridcolor=orange]
\psaxes[linewidth=1.25pt,labelFontSize=\scriptstyle]{->}(0,0)(-3,-3)(2,4)
\uput[u](1.8,0){$x$} \uput[r](0,3.8){$y$}
\psplot[plotpoints=2000,linewidth=1.25pt,linecolor=red]{-3}{0.5}{x 2 mul  3  add }
\psbezier[linewidth=1.25pt,linecolor=blue](-2.5,-3)(-2.05,0)(-1.5,-0.5)(-1,-1)
\psbezier[linewidth=1.25pt,linecolor=blue](-1,-1)(-0.43,-1.45)(-0.56,-1.75)(0,-1)
\psbezier[linewidth=1.25pt,linecolor=blue](0,-1)(1,1)(0.83,5)(0.89,4)
\uput[r](0.75,2.75){\blue $\mathcal{C}_f$}
\psdots[linecolor=red](-2,-1)(0,3)
\uput[ul](-2,-1){\red A} \uput[dr](0,3){\red B}
\end{pspicture*}
\end{minipage}

\bigskip

\textbf{Question 2}

\medskip

L'équation $\ln (x) = 7$ admet pour solution $\e^7$.

La bonne réponse est \textbf{c.}

\bigskip

\textbf{Question 3}

\medskip

Si $f = uv$ alors $f' = u'v + uv'$, d'où :
\[f'(x) = 1 \times \e^{2x} + x \times (2\e^{2x}) = (2x+1)\e^{2x}.\]

\bigskip

\textbf{Question 4}

\medskip

B est le projeté orthogonal de C sur (AB).

Il en résulte :
\[\vv{\text{AB}}\cdot\vv{\text{AC}}=\text{AB}\times \text{AB}= 4^2 = 16.\]

\bigskip

\textbf{Question 5}

\medskip

Les solutions de l'équation différentielle $y'+ay=b$ sur $\R$ sont les fonctions $y$ définies par :
\[y(x) = C\e^{-ax} + \dfrac{b}{a}, \text{ où } C \text{ est une constante quelconque.}\]

Ici $a =\np{4.5}\ \text{ et } b = \np{6.3}$. Par conséquent, sur $\R$ :
\[v(x) = C \e^{-\np{4.5}x}+\dfrac{\np{6.3}}{\np{4.5}} = C \e^{-\np{4.5}x}+1,4.\]

Déterminons $C$ :
\[v(0) = C\e^{-0,45 \times 0} + 1,4 = 0 \quad \text{d'où}\quad C = -1,4.\]

Au final : $v(x) = -1,4\e^{-4,5x} + 1,4.$

\bigskip

\textbf{Question 6}

\medskip

La valeur affichée lorsqu'on exécute le programme est $4$.

\bigskip


