
\medskip

\textbf{Vous traiterez 4 questions au choix parmi les 6 questions proposées.}

\medskip

\textbf{Pour les questions 1 à 3, on considère la fonction suivante :}

\medskip

Soit la fonction $f$ définie sur $[-1~;~+\infty[$ par :
\[f(x) = (4x - 1)\e^x.\]

\textbf{Question 1}

\medskip

Calculer $f(-1)$.

\bigskip

\textbf{Question 2}

\medskip

Justifier que la limite de la fonction $f$ en $+\infty$ est $+\infty$.

\bigskip

\textbf{Question 3}

\medskip

On admet que la fonction $f$ est dérivable sur l'intervalle $[-1~;~+\infty[$ et on note $f'$ sa fonction dérivée.

\begin{enumerate}
\item Montrer que pour tout $x$ appartenant à $[-1~;~+\infty[$,\, $f'(x) = \e^x(4x + 3)$.
\item Établir le tableau de variations de la fonction $f$ sur $[-1~;~+\infty[$.
\end{enumerate}

\bigskip

\textbf{Question 4}

\medskip

On considère l'intégrale $I$ suivante : $I = \displaystyle\int_{-1}^2 (4x - 1)\:\text{d}x$.

Montrer que $I = 3$.

\bigskip

\textbf{Question 5}

\medskip

Montrer en détaillant vos calculs que $\ln(576) = 6\ln(2) + 2\ln(3)$.

\bigskip

\textbf{Question 6}

\medskip

ABC est un triangle tel que : AB = 10, BC = 4, $\widehat{\text{ABC}} = 60^\circ$. Déterminer la longueur AC.

\begin{center}
\psset{unit=0.5cm}
\begin{pspicture}(-0.5,-0.5)(10.5,4)
\psdot[dotstyle=x,dotsize=8pt,linecolor=red](0,0)
\psdot[dotstyle=x,dotsize=8pt,linecolor=red](10,0)
% (10+4*cos(120)=8,4*sin(120)=3.464)
\psdot[dotstyle=x,dotsize=8pt,linecolor=red](8,3.464)
\psline(0,0)(10,0)(8,3.464)(0,0)
\psline(0,0)(10,0)
\uput[-135](0,0){A}
\uput[-45](10,0){B}
\uput[45](8,3.464){C}
\psarc[linecolor=blue](10,0){0.90}{120}{180}
\rput(8.5,0.5){\textcolor{blue}{$60\degres$}}
\end{pspicture}
\end{center}

\bigskip


