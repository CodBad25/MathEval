
\begin{center}
\textit{\textbf{Corrigé officiel (éduscol)}}
\end{center}

\begin{enumerate}[start=4]
\item
	\begin{enumerate}
	\item $(I)$ est une équation différentielle de la forme $y' = ay + b$ avec $a = -17 \times 10^9$ et $b = 55 \times 10^6$.

La solution générale est donc une fonction définie sur $\mathbb{R}$ et définie par :
\[t \mapsto k. \e^{at} - \dfrac{b}{a},\]
où $k$ est un réel.

En remplaçant $a$ et $b$ par les valeurs numériques du problème, la solution générale de $(I)$ est de la forme :
\[t \mapsto k. \e^{-17 \times 10^9t} + 3,2 \times 10^{-3},\]
où $k$ est un réel.

	\item $v$ est solution de $(I)$ donc il existe un réel $k$ tel que, pour tout réel $t$,
\[v(t) = k. \e^{-17 \times 10^9t} - 3,2 \times 10^{-3}.\]

La condition $v(0) = 0$ se traduit par l'équation d'inconnue $k: 0 = k + 3,2 \times 10^{-3}$ de solution $k = -3,2 \times 10^{-3}$ ce qui, après factorisation par $3,2 \times 10^{-3}$, confirme l'expression proposée.

	\item $\ds\lim_{t \to \infty} \e^{-17 \times 10^9 t} = 0$ car $-17 \times 10^9 < 0$.
	
	Par opérations sur les limites, on obtient :
\[\lim_{t \to \infty} v(t) = 3,2 \times 10^{-3}.\]

	\item Le problème revient à résoudre l'équation :
\[v(t_0) = 0,63 \times 3,2 \times 10^{-3}\]

qui se ramène à :
\[1 - \e^{-17 \times 10^9 t_0} = 0,63\]

On trouve la valeur :
\[t_0 = -\dfrac{\ln 0,37}{17} \times 10^{-9} \approx 5,8 \times 10^{-11} \text{ s}.\]

Cette valeur est négligeable devant la durée de l'expérience, on peut donc considérer que la valeur limite est immédiatement atteinte.
	\end{enumerate}
\end{enumerate}

\bigskip


