
\begin{center}
\textbf{Modèle de la vitesse de chute d'une hématie dans un plasma sanguin}
\end{center}

La détermination de la vitesse de sédimentation d'une hématie (globule rouge) est une analyse médicale mise en oeuvre pour détecter un état inflammatoire chez un patient. Initialement en suspension dans le plasma sanguin, les hématies d'un échantillon de sang anti-coagulé chutent verticalement dans le plasma et se déposent, c'est la sédimentation.

\medskip

L'objectif de cet exercice est d'étudier un modèle de l'évolution de la vitesse $v(t)$ de chute d'une hématie dans un plasma. $v(t)$ suit l'équation différentielle :
\[\dfrac{dv}{dt} + 1,125 \times 10^6 \times v = 1,811\]

\begin{enumerate}[start=4]
\item
	\begin{enumerate}[start=3]
	\item Donner l'ensemble des solutions de cette équation différentielle.
	\item Justifier que parmi l'ensemble des solutions de cette équation, la fonction $v$ est la solution qui vérifie la condition $v(0) = 0$. En déduire que, pour tout réel $t$ de $[0~;~+\infty[$,
\[v(t) = 1,610 \times 10^{-6} \times \left(1 - e^{-1,125 \times 10^6t}\right).\]
	\end{enumerate}
\item Déterminer la valeur de la limite de $v(t)$ lorsque $t$ tend vers $+\infty$. Préciser la signification physique de cette valeur dans le cadre de ce modèle.
\end{enumerate}

\bigskip


