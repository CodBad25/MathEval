
\begin{center}
\textit{\textbf{Corrigé officiel (éduscol)}}
\end{center}

\begin{enumerate}
\item Lorsqu'on donne l'instruction \texttt{meth\_rect(2)}, le pas choisi est de 2 -- contrairement aux figures 3 et 5 -- et le premier rectangle est construit avec $x = 1$ et est donc aplati, contrairement à la figure 2. C'est donc la figure 4 qui représente la situation calculée par \texttt{meth\_rect(2)}.

\item D'après l'analyse faite à la question \textbf{1.}, la méthode des rectangles implémentée en Python donne une valeur inférieure à $A_2$ donc $A_2 > \texttt{9.307920700315046}$.
\end{enumerate}

\bigskip

