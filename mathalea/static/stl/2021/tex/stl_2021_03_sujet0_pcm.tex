
\begin{center}
\textbf{Séparation d'un mélange d'acides aminés}
\end{center}

Les acides aminés sont des espèces chimiques organiques essentielles au bon fonctionnement de notre organisme. En laboratoire, l'électrophorèse des mélanges d'acides aminés permet de les séparer, de les identifier et d'estimer leur quantité.

\medskip

Dans le référentiel supposé galiléen d'un laboratoire, on étudie le mouvement d'une molécule d'acide aminé (la glycine NH$_3^+$--CH$_2$--CO$_2$H) dans un capilaire.

\medskip

Dans les conditions de l'expérience, la vitesse de déplacement de la glycine obéit à l'équation différentielle :
\[(I) : \dfrac{dv}{dt} = 55 \times 10^6 - 17 \times 10^9 v,\]
où la vitesse $v$ est exprimée en m\cdot s$^{-1}$ et le temps $t$ en s.

\begin{enumerate}[start=4]
\item
	\begin{enumerate}
	\item Déterminer la solution générale de l'équation différentielle $(I)$ de fonction inconnue $v$.
	\item Sachant que $v(0) = 0$, montrer que la vitesse $v$ de l'espèce chimique est, avec des coefficients exprimés avec deux chiffres significatifs :
\[v(t) = 3,2 \times 10^{-3} \left(1 - \e^{-17 \times 10^9 t}\right).\]
	\item Calculer $\lim\limits_{t \to \infty} v(t)$.
	\item Déterminer l'instant $t_0$ où la vitesse atteint 63\% de la vitesse limite. Commenter le résultat en le comparant à la durée de l'expérience.
	\end{enumerate}
\end{enumerate}

\bigskip


