
\medskip

\textbf{Vous traiterez 4 questions au choix parmi les 6 questions proposées.}

\medskip

\textbf{Pour les questions 1 et 2, on considère la fonction suivante :}

\medskip

Soit $g$ la fonction définie sur l'intervalle $[0~;~+\infty[$ par :
\[g(x) = (2x - 1)\e^{-x}.\]

\textbf{Question 1}

\medskip

Calculer $g(0)$.

\bigskip

\textbf{Question 2}

\medskip

On admet que la fonction $g$ est dérivable sur l'intervalle $[0~;~+\infty[$ et on note $g'$ sa fonction dérivée.

\begin{enumerate}
\item Montrer que, pour tout réel $x$ appartenant à $[0~;~+\infty[$,\, $g'(x) = (- 2x + 3)\e^{-x}$.
\item Justifier que $g(x) < 2\e^{-\frac{3}{2}}$ pour $x > \dfrac{3}{2}$.
\end{enumerate}

\bigskip

\textbf{Question 3}

\medskip

Sachant que $\cos \left(\dfrac{9\pi}{5}\right) = \dfrac{\sqrt{5} + 1}{4}$, exprimer $\cos \left(\dfrac{\pi}{5}\right)$ en fonction de $\sqrt{5}$.

\bigskip

\textbf{Question 4}

\medskip

On considère l'intégrale $I$ suivante : $I = \displaystyle\int_0^2 (2x - 1)\:\text{d}x$.

Montrer que $I = 2$.

\bigskip

\textbf{Question 5}

\medskip

Simplifier le nombre suivant en détaillant les calculs :
\[A = 5\ln \left(\e^3\right) - 4 \ln \left(\dfrac{1}{\text{e}^2}\right).\]

\bigskip

\textbf{Question 6}

\medskip

ABCD est un carré de côté 3 cm et DCE est un triangle rectangle et isocèle en C.

\begin{center}
\psset{unit=1cm}
\begin{pspicture}(7,3.5)
\pspolygon(0,0)(6.4,0)(3.2,3.2)(0,3.2)%BEDA
\psline(3.2,0)(3.2,3.2)%CD
\uput[ul](0,3.2){A} \uput[dl](0,0){B} \uput[d](3.2,0){C} 
\uput[ur](3.2,3.2){D} \uput[d](6.4,0){E}
\def\barre{\psline(-0.15,-0.15)(0.15,0.15)}
\rput(0,1.6){\barre}\rput(3.2,1.6){\barre} \rput(1.6,0){\barre}\rput(1.6,3.2){\barre}
\psframe(0.2,0.2)\psframe(0,3)(0.2,3.2)\psframe(3.2,3.2)(3,3)\psframe(3.2,0)(3.4,0.2)
\psarc(6.4,0){0.45}{135}{180}\rput(5.75,0.25){$45\degres$}
\rput(4.8,0){\barre}
\end{pspicture}
\end{center}

\medskip

Donner la valeur du produit scalaire $\vect{\text{EB}} \cdot \vect{\text{ED}}$.

\bigskip


