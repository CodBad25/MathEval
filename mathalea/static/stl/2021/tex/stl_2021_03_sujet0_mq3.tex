
\medskip

\begin{center}
\psset{xunit=1cm,yunit=1cm,labelFontSize=\scriptstyle,comma=true}
\begin{pspicture}(-1,-0.75)(11,3.75)
\multido{\n=0+1}{11}{\psline[linewidth=0.75pt,linecolor=lightgray](\n,0)(\n,3.5)}
\multido{\n=0+1}{4}{\psline[linewidth=0.75pt,linecolor=lightgray](0,\n)(10.25,\n)}
\psframe[linewidth=0.8pt, linecolor=gray](9,0)(9.3,0.3)
\psaxes[linewidth=0.8pt,Dx=1,Dy=1]{->}(0,0)(10.25,3.5)
\pstVerb{ /xmax 9 ln 1 add def }
\pscustom[fillstyle=hlines, hatchcolor=cyan, hatchwidth=0.8pt, hatchsep=5.5pt, hatchangle=-45]{
	\psplot[plotpoints=1000]{1}{9}{x ln}
	\psline(9,0)
	\psplot[plotpoints=1000]{9}{1}{0}
	\closepath
}
\psplot[plotpoints=1000,linewidth=1.25pt,linecolor=blue]{1}{9}{x ln}
\psplot[plotpoints=1000,linewidth=1.25pt,linecolor=blue]{1}{9}{0}
\psline[linecolor=blue,linewidth=1.25pt](9,0)(!9 9 ln)
\uput[ul](1,0){A}\uput[dr](9,0){B}\uput[ur](!9 9 ln){C}\uput[ur](5.4,2.4){$C_f$}
\psline[arrows=->,linewidth=1pt](5.5,2.5)(!5 5 ln)
\psdot[dotsize=4pt,linecolor=red](1,0)
\psdot[dotsize=4pt,linecolor=red](9,0)
\psdot[dotsize=4pt,linecolor=red](!9 9 ln)
\end{pspicture}
\end{center}

\begin{itemize}
\item Sur la figure ci-dessus, l'unité de longueur est le centimètre ;
\item la courbe $C_f$ tracée est celle de la fonction $f$ définie sur $[1~;~9]$ par $f(x) = \ln(x)$ ;
\item le point A a pour coordonnées $(1~;~0)$, le point B a pour coordonnées $(9~;~0)$ ;
\item C est le point de $C_f$ d'abscisse 9.
\end{itemize}

On désigne par $\Delta$ le domaine hachuré sur la figure, délimité par la courbe $C_f$, l'axe des abscisses et le segment [BC]. On note $A_2$ l'aire de $\Delta$, exprimée en cm$^2$.

\smallskip

Calcul de la valeur exacte de $A_2$ :
\begin{enumerate}
\item Démontrer que la fonction $F$ définie sur $[1~;~9]$ par $F(x) = x \ln(x) - x$ est une primitive de la fonction $f$ sur $[1~;~9]$.
\item En déduire la valeur exacte de $A_2$.
\end{enumerate} 

\bigskip

