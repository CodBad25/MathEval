
\medskip

\begin{center}
\psset{xunit=1cm,yunit=1cm,labelFontSize=\scriptstyle,comma=true}
\begin{pspicture}(-1,-0.75)(11,3.75)
\multido{\n=0+1}{11}{\psline[linewidth=0.75pt,linecolor=lightgray](\n,0)(\n,3.5)}
\multido{\n=0+1}{4}{\psline[linewidth=0.75pt,linecolor=lightgray](0,\n)(10.25,\n)}
\psframe[linewidth=0.8pt, linecolor=gray](9,0)(9.3,0.3)
\psaxes[linewidth=0.8pt,Dx=1,Dy=1]{->}(0,0)(10.25,3.5)
\psplot[linewidth=0.75pt, linestyle=dashed]{0}{4.5}{x 1 sub}
\psplot[linewidth=0.75pt, linestyle=dashed]{-0.75}{10.25}{9 ln}
\pstVerb{ /xmax 9 ln 1 add def }
\pscustom[fillstyle=solid,fillcolor=cyan!30,linestyle=none]{
	\psplot[plotpoints=1000]{1}{9}{x ln}
	\psplot[plotpoints=1000]{9}{xmax}{9 ln}
	\psplot[plotpoints=1000]{xmax}{1}{x 1 sub}
	\closepath
}
\pscustom[fillstyle=solid,fillcolor=cyan!30,linestyle=none]{
	\psplot[plotpoints=1000]{1}{9}{x ln}
	\psline(9,0)
	\psplot[plotpoints=1000]{9}{1}{0}
	\closepath
}
\psplot[plotpoints=1000,linewidth=1.25pt,linecolor=blue]{1}{9}{x ln}
\psplot[plotpoints=1000,linewidth=1.25pt,linecolor=blue]{1}{xmax}{x 1 sub}
\psplot[plotpoints=1000,linewidth=1.25pt,linecolor=blue]{xmax}{9}{9 ln}
\psplot[plotpoints=1000,linewidth=1.25pt,linecolor=blue]{1}{9}{0}
\psline[linecolor=blue,linewidth=1.25pt](9,0)(!9 9 ln)
\uput[ul](4,3){$T$}\uput[ul](1,0){A}\uput[dr](9,0){B}\uput[ur](!9 9 ln){C}\uput[ul](!xmax 9 ln){D}\uput[ur](5.4,2.4){$C_f$}
\psline[arrows=->,linewidth=1pt](5.5,2.5)(!5 5 ln)
\psdot[dotsize=4pt,linecolor=red](1,0)
\psdot[dotsize=4pt,linecolor=red](9,0)
\psdot[dotsize=4pt,linecolor=red](!9 9 ln)
\psdot[dotsize=4pt,linecolor=red](!xmax 9 ln)
\end{pspicture}
\end{center}

\begin{itemize}
\item Sur la figure ci-dessus, l'unité de longueur est le centimètre ;
\item la courbe $C_f$ tracée est celle de la fonction $f$ définie sur $[1~;~9]$ par $f(x) = \ln(x)$ ;
\item la droite $T$ est la tangente à la courbe $C_f$ au point A d'abscisse 1 ;
\item le point B a pour coordonnées $(9~;~0)$ ;
\item C est le point de $C_f$ d'abscisse 9 ;
\item la parallèle à l'axe des abscisses passant par C coupe la droite $T$ au point D.
\end{itemize}

\smallskip

Calcul de l'aire $A_1$ du trapèze ABCD :
\begin{enumerate}
\item Justifier que la tangente $T$ a pour équation réduite $y = x - 1$.
\end{enumerate}

\smallskip

On admet que le point D a pour coordonnées : $(2 \ln(3) + 1~;~2 \ln(3))$.

\begin{enumerate}[resume]
\item Démontrer que la valeur de $A_1$, exprimée en cm$^2$, est égale à :
\[16 \ln(3) - 2 (\ln(3))^2.\]
\end{enumerate}

\bigskip

