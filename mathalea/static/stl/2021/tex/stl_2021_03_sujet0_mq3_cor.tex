
\begin{center}
\textit{\textbf{Corrigé officiel (éduscol)}}
\end{center}

\begin{enumerate}
\item On commence par dériver la fonction $x \mapsto x \ln(x)$ comme produit de deux fonctions puis on obtient, pour tout réel $x$ de l'intervalle $[1~;~9]$,
 
$F'(x) = \ln(x) = f(x)$ et $F$ est donc une primitive de $f$ sur $[1~;~9]$.

\item $A_2$ est l'aire du domaine délimité par l'axe des abscisses, la courbe de la fonction $f$ et la droite verticale d'équation $x = 9$.
 
De plus $f$ est positive sur $[1~;~9]$ et $f(1) = 0$.

Comme la fonction $f$ est dérivable sur $[1~;~9]$,

$A_2 = \int_{1}^{9} f(x) dx$.

Enfin, $F$ étant une primitive de $f$ sur l'intervalle $[1~;~9]$,  

$A_2 = F(9) - F(1)$ soit $A_2 = 9 \ln(9) - 10 = 18 \ln(3) - 10$.
\end{enumerate}

\bigskip

