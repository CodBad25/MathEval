
\begin{center}
\textbf{Décomposition de l'eau oxygénée}
\end{center}

L'eau oxygénée, utilisée comme désinfectant, est une solution de peroxyde d'hydrogène HzOz. Son efficacité diminue au cours du temps à cause de la réaction de dismutation de cette espèce. 
L'équation de réaction associée est la suivante:

\[2 \text{H}_2\text{O}_2(\text{aq}) \to  2 \text{H}_2\text{O}(\text{l}) + \text{O}_2(\text{g})\]

Lors d'une activité expérimentale au lycée, les élèves étudient la cinétique de cette réaction catalysée par la présence d'ions fer (III).

On fait l'hypothèse d'une cinétique d'ordre 1 par rapport au peroxyde d'hydrogène pour la réaction de dismutation étudiée. Dans ce cas, en posant $f(t) = \left(\text{H}_2\text{O}_2\right)(t)/\left[\text{H}_2\text{O}_2\right]_0$, on montre que l'équation différentielle vérifiée par la fonction $f$ est:

\[\dfrac{\text{d}f}{\text{d}t} +k \times f = 0\]

\begin{enumerate}[start=8]
\item Vérifier que la fonction $f$ définie par $f(t) = \e^{-kt}$ est solution de l'équation différentielle :
\[y' + ky = 0.\]
\end{enumerate}

On admet que $\ln f(t) = - k \times t$.

\begin{enumerate}[start=9]
\item En utilisant le graphe de la figure suivante, obtenu à partir des résultats expérimentaux, justifier que la pente de la droite est voisine de $-0,08$.

En déduire une valeur approchée de $k$.

\begin{center}
\psset{xunit=0.2cm,yunit=2cm,comma=true}
\begin{pspicture}(-5,-3.7)(45,0.2)
\multido{\n=0+5}{10}{\psline[linewidth=0.2pt](\n,-3.7)(\n,0.2)}
\multido{\n=-3.5+0.5}{8}{\psline[linewidth=0.2pt](0,\n)(45,\n)}
\psaxes[linewidth=1.25pt,Dx=5,Dy=0.5](0,-3.7)(45,0.2)
\psline[linecolor=blue](0,0)(45,-3.6)
\psdots[dotstyle=+,dotscale=1.5,linecolor=red,dotsize=0pt 10](0,0)(5,-0.4)(10,-0.75)(15,-1.3)(20,-1.7)(25,-2)(30,-2.3)(40,-3)
\uput[d](22.5,-4){temps $t$ en min}\rput{90}(-5,-1.75){$\ln f(t)$}
\end{pspicture}

\vspace{1.2cm}

Figure - Évolution de $\ln f(t) = \ln\left(\left[\text{H}_2\text{O}_2\right](t)/\left[\text{H}_2\text{O}_2\right]_0\right)$ en fonction du temps.
\end{center}
\end{enumerate}

L'expression de la concentration en quantité de matière de peroxyde d'hydrogène à un instant $t$ peut s'écrire: 
\[\left[\text{H}_2\text{O}_2\right](t) = \left[\text{H}_2\text{O}_2\right]_0 \times \e^{-kt}.\]

\begin{enumerate}[start=10]
\item Montrer que le temps de demi-réaction peut s'exprimer par la relation : $t_{\frac12} = \dfrac{\ln 2}{k}$.
\item Calculer la valeur du temps de demi-réaction.
\end{enumerate}

\bigskip


