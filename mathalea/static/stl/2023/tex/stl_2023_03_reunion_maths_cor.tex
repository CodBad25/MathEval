
\medskip

\textbf{Question 1}

\medskip

Appliquons aux deux membres de l'inégalité la fonction $\ln$.

Étant une fonction strictement croissante sur $\R_+^*$ elle conserve l'ordre.

Nous obtenons :
\begin{align*}
\ln(\e^{2t}) > \ln (0,12) \\
2t > \ln (0,12) \\
t > \dfrac{\ln (0,12)}{2} \\
\left(\dfrac{\ln (0,12)}{2} \approx - 1,06\right)
\end{align*}

\bigskip

\textbf{Question 2}

\medskip

\begin{enumerate}
\item La dérivée de la fonction $F$ est : $F'(t)=a\times 2 \e^{2t + 6} = f(t) = 6\e^{2t+6}$

Par conséquent $2a = 6$ d'où $a = 3$.
\item Une autre primitive de la fonction $f$, est la fonction $G$ définie par $G(t)= 3\e^{-2t+6}+\lambda$ où $\lambda$ est un nombre réel quelconque.
\end{enumerate}

\bigskip

\textbf{Question 3}

\medskip

On sait que quel que soit $t \in \R$ :
\begin{align*}
\e^{0,6 - 0,2t} &> 0 \\
\iff 1 + \e^{0,6 - 0,2t} &> 1 + 0 \\
\iff 1 + \e^{0,6 - 0,2t} &> 1 \\
\iff \dfrac{1}{1 + \e^{0,6 - 0,2t}} &< 1 \\
\iff \dfrac{94,6}{1 + \e^{0,6 - 0,2t}} &< 94,6
\end{align*}

soit $f(t) < 94,6$.

\medskip

\textit{Correction alternative}

\medskip

Déterminons la limite de $f$ lorsque $t$ tend vers $+ \infty$.

\[\displaystyle \lim_{t\to +\infty} 1+\e^{0,6-0,2t}=1 \text{ puisque }  \lim_{t\to +\infty} \e^{0,6-0,2t}=0.\]

Il en résulte :
\[\displaystyle \lim_{t\to +\infty} \dfrac{94,6}{1+\e^{0,6-0,2t}}=94,6.\]

Le taux d'équipement ne pourra donc jamais dépasser $94,6\,\%$.

\bigskip

\textbf{Question 4}

\medskip

\[f(x)=\dfrac{\e^x}{x^2\left(1+\frac{26}{x}\right)}=\dfrac{\e^x}{x^2}\times \dfrac{1}{1+\frac{26}{x}}.\]

On sait que :
\[\displaystyle  \lim_{x\to +\infty} \dfrac{26}{x} = 0,\]
d'où par somme de limites :
\[\displaystyle  \lim_{x\to +\infty} 1 + \dfrac{26}{x} = 1,\]
et, par quotient de limites :
\[\displaystyle  \lim_{x\to +\infty} \dfrac{1}{1+\frac{26}{x}} = 1.\]

On sait également (croissances comparées) que :
\[\displaystyle  \lim_{x\to +\infty} \dfrac{\e^x}{x^2} = + \infty,\]
donc finalement par produit de limites :
\[\displaystyle  \lim_{x\to +\infty}\dfrac{\e^x}{x^2\left(1+\frac{26}{x}\right)} = + \infty.\]

\bigskip


