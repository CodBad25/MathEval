
\medskip
 
\begin{enumerate}[start=5]
\item Résolvons sur $[0~;~+ \infty[$ cette équation différentielle.

Les solutions de  l'équation différentielle $y'= ay + b$ sur  $\R$ sont les fonctions $y$ définies par :
\[y(x)=C\e^{ax}-\dfrac{b}{a},\]
où $C$ est une constante quelconque.

$a=-7\times 10^{-4},~b=0$, par conséquent sur $[0~;~+\infty[$ $y(t)=C \e^{-7\times 10^{-4}t}$, où $C$ est une constante quelconque.

Les solutions de l'équation (E) sont les fonctions f définies par $f(t)=C\,\e^{-7\times 10^{-4} t }$.

\item Sachant que pour $t = 0$, la concentration initiale du saccharose vaut
0,4 mol$\cdot \text{L}^{-1}$, déterminons la constante C.

$f(0)=C\times \e^{0}$ d'où $C = 0,4$.

Par conséquent,l'unique solution de l'équation $(E)$ est la fonction $c$ définie sur $[0~;~+ \infty[$ par :
\[c(t) = 0,4 \times \e^{-7\times 10^{-4}\times t}.\]

\item Déterminons la limite de $c(t)$ lorsque $t$ tend vers $+ \infty$.

\[\displaystyle \lim_{t\to +\infty}\e^{-7\times 10^{-4}\times t}=0 \text{, il en résulte que } \lim_{t\to +\infty} c(t)=0.\]

\item Ce résultat dans le contexte de la production réalisée en laboratoire montre qu'à très long terme le saccharose aura disparu.
\end{enumerate}

\bigskip


