
\medskip

\begin{enumerate}
\item $\ds \lim_{x\to+\infty}f(x)=\lim_{x\to +\infty}\e^{-x}+\lim_{x\to +\infty} 0,5x-3=0+(+\infty)=+\infty.$

\item Prenons deux points appartenant à cette droite et calculons son coefficient directeur.

Par exemple M$(-2~;~-1)$ et N$(2~;~-3)$, le coefficient directeur de (MN) est  : 
\[\dfrac{-3+1}{2+2}=-\dfrac{1}{2}\ \text{d'où }\ f'(0)=-\dfrac{1}{2}.\]

\item $f'(x)=-\e^{-x}+\np{0.5}\ \text{d'où }\ f'(0)=-\e^0+\np{0.5}=-1+\np{0.5}=-\np{0.5}.$ 

Nous obtenons le même résultat qu'à la question 2.

\item Étudions le signe de $f'(x)$ :

\begin{align*}
-\e^{-x}+\dfrac{1}{2}&>0 \\
-\e^{-x}&>-\dfrac{1}{2} \\
\e^{-x}&<\dfrac{1}{2} \\
-x&<-\ln 2 \\
x&>\ln 2
\end{align*}

Si pour tout $x\in I,\:f'(x)< 0$ alors $f$ est  strictement décroissante sur $I$.

Sur $]-\infty~;~\ln 2[,\:f'(x)<0$ par conséquent $f$ est strictement décroissante sur cet intervalle.

Si pour tout $x\in I, \:f'(x)>0 $ alors la fonction $f$ est strictement croissante sur $I$.

Sur $]\ln 2~;~+\infty[,\:f'(x)>0$ par conséquent $f$ est strictement croissante sur  cet intervalle.

Dressons le tableau de variations :

\begin{center}
\psset{xunit=1cm,yunit=1.3cm}
\begin{pspicture}(12,3)
%\psgrid
\psframe(12,3)
\psline(0,1.5)(12,1.5)\psline(0,2.5)(12,2.5)  \psline(3.5,0)(3.5,3)
\uput[u](1.75,2.5){$x$} \uput[ur](3.35,2.5){$-\infty$}\uput[u](11.6,2.5){$+\infty$}\uput[u](7.8,2.5){$\ln 2$}
\uput[u](1.3,1.75){Signe de $f'(x)$}\uput[u](1.3,0.5){Variations de $f$}
\psline[linestyle=dotted,linewidth=0.5pt](7.8,1.5)(7.8,2.5)
\uput[u](7.8,1.75){$0$} 
\uput[u](5.7,1.75){$-$}
\uput[u](10.2,1.75){$+$}
%\psline{->}(4,0.5)(8.2,1.25)
\psline{->}(4,1.25)(7.1,0.5)\psline{->}(8.5,0.5)(11,1.25)
\uput[dl](4.2,1.5){\small $+\infty$} \uput[u](8,0.15) {\small $\dfrac{\ln 2-5}{2}$}
\uput[d](11.5,1.5){$+\infty$}
\end{pspicture}
\end{center}

$f(\ln 2) =\e^{-\ln 2}+\frac{\ln 2}{2}-3=-\frac{1}2+\frac{\ln 2}{2}-3=\frac{\ln 2-5}{2}\approx -\np{2.15}$

\item Dans le contexte de l'exercice la valeur $-1,29$ à $10^{-2}$ près renvoyée par l'exécution de l'instruction {\ttfamily abscisse()} est l'abscisse du point A.

\item En rouge et gras, les modifications du programme Python :

\begin{center}
\begin{tabular}{|>{\ttfamily}l|}\hline
from math import exp\\
def abscisse():\\
\quad $x=~$\textbf{\textcolor{red}{5.5}}\\
\quad while exp$(-x)+0.5*x-3~\textbf{\textcolor{red}{<}}~0:$\\
\quad \quad $x = x + 0.01$\\
\quad return $x$ \\ \hline
\end{tabular}
\end{center}
\end{enumerate}

\bigskip


