
\medskip

\textbf{Les questions 1, 2, 3 et 4 sont indépendantes les unes des autres.}

\medskip

\textbf{Question 1}

\medskip

Soit la fonction $f$ définie sur $[0~;~+\infty[$ par :
\[f(x)=(3x + 5)\e^x.\]

Vérifier que $f(0)$ est un nombre entier que l'on précisera.

\bigskip

\textbf{Question 2}

\medskip

Soit la fonction $f$ définie et dérivable sur $[0~;~+\infty[$ par :
\[f(x) = (x - 5)\e^{3x}.\]

On note $f'$ sa fonction dérivée.

Démontrer que pour tout $x$ appartenant à l'intervalle $[0~;~+\infty[$, $f'(x)=(3x-14)\e^{3x}$.

\bigskip

\textbf{Question 3}

\medskip

On donne : $\mathcal{A} = \ln \left(\dfrac{25}{8}\right)$.

En détaillant les calculs, écrire $\mathcal{A}$ sous la forme $a\ln(2)+ b\ln(5)$, $a$ et $b$ étant deux nombres entiers relatifs.

\bigskip

\textbf{Question 4}

\medskip

On considère l'équation différentielle (E): $y'= 3y - 12$,
où $y$ est une fonction de variable $x$, définie et dérivable sur $\R$.

Déterminer la fonction $f$ définie et dérivable sur $\R$, solution de (E), qui vérifie $f(0) = 8$.

\bigskip


