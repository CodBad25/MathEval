
\begin{center}
\textbf{Mouvement d'une voiture miniature}
\end{center}

Lors d'une séance expérimentale, un binôme d'élèves réalise la vidéo du mouvement d'une voiture miniature en roue libre. L'étude est menée dans le référentiel du sol supposé galiléen. Le mouvement de la voiture est rectiligne et s'effectue selon un axe horizontal $(\text{O}x)$ fixe.

\medskip

L'analyse de la vidéo obtenue par les élèves a permis de modéliser la position du centre de masse $G$ de la voiture au cours du temps par la fonction polynomiale :
\[x(t) = -0,58 \times t^2 + 0,65 \times t,\]
définie sur l'intervalle de temps $[0~;~0,50]$, la position $x$ étant exprimée en mètres et le temps $t$ en secondes.

\medskip

La fonction $x$ est dérivable sur l'ensemble des réels. On note $x'$ sa dérivée.

\begin{enumerate}[start=3]
\item Déterminer $x'(t)$ pour tout réel $t$.
\item Calculer $x'(0)$.
\item Nommer la grandeur physique à laquelle fait référence $x'(0)$.
\item Déduire de la question 3 la valeur de l'accélération définie sur l'intervalle de temps [0 ; 0,50], le temps étant exprimé en secondes. Interpréter le signe dans la situation étudiée.
\end{enumerate}

\bigskip


