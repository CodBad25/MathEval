
\medskip

\textbf{Les questions 1, 2, 3 et 4 sont indépendantes les unes des autres.}

\medskip

\textbf{Question 1}

\medskip

Soit la fonction $f$ définie sur $[0~;~ +\infty[$ par :
\[f(x) = (4x + 8)\e^x.\]

Vérifier que $f(0)$ est un nombre entier que l'on précisera.

\bigskip

\textbf{Question 2}

\medskip

Soit la fonction $f$ définie sur $[0~;~ +\infty[$ et $\mathcal{C}_f$ sa courbe représentative donnée sur le graphique ci-dessous.

On admet que $f$ est dérivable sur $[0~;~ +\infty[$ et on note $f'$ sa dérivée.

Soit $T$ la tangente à la courbe $\mathcal{C}_f$ au point d'abscisse 2.

Déterminer par lecture graphique $f(2)$ et $f'(2)$.

\begin{center}
\psset{unit=0.75cm,arrowsize=2pt 3}
\begin{pspicture*}(-2.1,-6.2)(10.2,2.2)
\psgrid[gridlabels=0pt,subgriddiv=1,gridwidth=0.2pt]
\psaxes[linewidth=1.25pt,labelFontSize=\scriptstyle]{->}(0,0)(-1.95,-6.2)(10.2,2.2)
\psaxes[linewidth=1.25pt,labelFontSize=\scriptstyle](0,0)(-2,-6.2)(10,2.2)
\psplot[plotpoints=2000,linewidth=1.25pt, linecolor=blue]{0}{10}{15 x mul 2.71828 x neg exp mul neg}
\psplotTangent{2}{7}{15 x mul 2.71828 x neg exp mul neg}
\uput[dr](4.1,-1){\blue $\mathcal{C}_f$}
\end{pspicture*}
\end{center}

\bigskip

\textbf{Question 3}

\medskip

Un triangle ABC est tel que AB $= 5$, BC $= 8$ et AC $= 10$.

Déterminer le cosinus de l'angle $\widehat{\text{BAC}}$ en utilisant une formule d'Al-Kashi.

\bigskip

\textbf{Question 4}

\medskip

On considère la fonction $f$ définie et dérivable sur $\R$ par :
\[f(x) = -3x^2 + 8x.\]

Démontrer que la fonction $F$ définie et dérivable sur $\R$ par :
\[F(x) = -x^3 +4x^2 + \np{1789},\]
est une primitive de $f$ sur $\R$.

\bigskip


