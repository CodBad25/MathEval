
\begin{center}
\textbf{Oxydation des ions iodure}
\end{center}

L'iodure de potassium est utilisé comme complément alimentaire. Il est notamment intégré au sel de table pour prévenir les carences en iode chez les populations ne consommant que peu de fruits de mer et de poissons. L'exposition du sel iodé à l'air libre provoque l'oxydation lente des ions iodure.

\medskip

Au laboratoire, on met en oeuvre l'oxydation des ions iodure $\mathrm{I^{-}(aq)}$ par les ions peroxodisulfate $\mathrm{S_2O_8^{2-}(aq)}$ pour estimer la durée de l'oxydation.

\medskip

On note $C$ la fonction définie sur l'intervalle $[0~;~+\infty[$ modélisant la concentration en ions peroxodisulfate $C(t)$ (exprimée en $\mathrm{mol \cdot L^{-1}}$) du milieu réactionnel en fonction du temps $t$ (exprimé en seconde).

\medskip

La concentration initiale en ions peroxodisulfate vaut $C(0) = 0,0042~\mathrm{mol \cdot L^{-1}}$.

\medskip

Pour une évolution de la concentration donnée par une relation d'ordre 1, les données physiques de l'expérience conduisent à résoudre l'équation différentielle $(E)$ :
\[y' = - \np{0,0085}y.\]

\begin{enumerate}[start=6]
\item Déterminer la fonction $C$, solution de l'équation différentielle $(E)$ vérifiant $C(0) = \np{0,0042}$.
\item Résoudre l'équation :
\[C(t) = \np{0,00021},\]
et donner une valeur approchée à la seconde près de la durée nécessaire pour que la concentration résiduelle en peroxodisulfate, correspondant à une oxydation de $95\,\%$ du réactif limitant, soit égale à \np{0,00021} $\mathrm{mol \cdot L^{-1}}$.
\end{enumerate}

\bigskip


