
\medskip

La fonction $f$ est définie sur $[0~;~ +\infty[$ par : $f(x) = x\e^{\np{0.02}x}- \np{10 000}$.

\begin{enumerate}
\item Déterminer $\displaystyle \lim_{x\to +\infty}f(x)$.
\item On note $f'$ la fonction dérivée de $f$ sur $[0~;~ +\infty[$.

Justifier que pour tout nombre réel $x\geqslant 0,\  f'(x) = (1 + 0,02x )\,\e^{0,02x}$.
\item En déduire le sens de variation de $f$ sur $[0~;~+\infty[$.
\item L'affirmation suivante est-elle vraie ou fausse ? Justifier.

\og Tout nombre réel $x$, compris entre 0 et \np{1000}, a une image négative par $f$. \fg
\item Quatre fonctions A, B, C et D sont écrites dans le même programme Python
ci-dessous.

Laquelle de ces quatre fonctions permet de déterminer la plus petite valeur
entière dont l'image par $f$ est positive ?
\end{enumerate}
\begin{center}
\fbox{
\begin{minipage}{6cm}
from math import exp\\
def A(~):\\
\phantom{de}n = 0\\
\phantom{de}return n * exp(0.02 * n) – 10000\\

def B(~):\\
\phantom{de}n = 0\\
\phantom{de}f = – 10000\\
\phantom{de}while f < 0:\\
\phantom{dem}n = n + 1\\
\phantom{dem}f = n * exp(0.02 * n) – 10000\\
\phantom{de}return n\\

def C(~):\\
\phantom{de}f = – 10000\\
\phantom{de}for n in range(0,1000):\\
\phantom{demn}f = n * exp(0.02 * n) – 10000\\
return f\\
\phantom{xxxx}\\
def D(~):\\
\phantom{de}n = 0\\
\phantom{de}f = – 10000\\
\phantom{de}if f < 0:\\
\phantom{denm}n = n + 1\\
\phantom{denm}f = n * exp(0.02 * n) – 10000\\
return n\\
\end{minipage}
}
\end{center}

\bigskip


