
\medskip

On considère la fonction $f$ définie sur $\R$ par : $f(x) = \e^{-x} + 0,5x - 3$,

dont la courbe représentative $\mathcal{C}_f$ est donnée dans le repère orthonormé du plan ci-dessous.

\begin{center}
\psset{unit=1cm,arrowsize=2pt 3}
\begin{pspicture*}(-3,-3)(8.5,2.5)
\psgrid[gridlabels=0pt,subgriddiv=1,gridcolor=lightgray]
\psaxes[linewidth=1.25pt]{->}(0,0)(-3,-3)(8.5,2.5)
\psplot[plotpoints=2000,linewidth=1.25pt,linecolor=red,labelFontSize=\scripstyle]{-3}{8.5}{0.5 x mul 3 sub 2.71828 x neg exp add}
\psplotTangent{-1.293}{4}{0.5 x mul 3 sub 2.71828 x neg exp add}
\psplotTangent{0}{4}{0.5 x mul 3 sub 2.71828 x neg exp add}
\uput[ur](-1.293,0){\small A}\uput[ur](0,-2){\small D}
\uput[d](8.3,0){$x$} \uput[r](0,2.25){$y$}\uput[r](-1.75,2){\red $\mathcal{C}_f$}
\psdots(-1.293,0)(0,-2)
\end{pspicture*}
\end{center}

Le point d'intersection de $\mathcal{C}_f$ avec l'axe des ordonnées est nommé D.

Les tangentes à la courbe $\mathcal{C}_f$ en A et D sont représentées.

On note $f'$ la fonction dérivée de $f$ sur $\R$.

\begin{enumerate}
\item Déterminer $f'(0)$ par lecture graphique.
\item Calculer $f'(x)$ et vérifier par le calcul le résultat obtenu à la question 1.
\end{enumerate}

\bigskip

