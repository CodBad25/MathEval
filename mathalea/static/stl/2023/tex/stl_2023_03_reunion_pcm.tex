
\begin{center}
\textbf{Production de sucre inverti}
\end{center}

La production de sucre inverti est réalisée en laboratoire lors de la transformation chimique du saccharose en milieu acide, en chauffant.

On définit la vitesse $v$ de disparition du saccharose de concentration $c$ en quantité de matière par:

\[v = - \dfrac{\text{d} c}{\text{d} t}.\]

La cinétique de l'hydrolyse du saccharose peut être modélisée par l'équation différentielle :
\begin{center} $(E) :\quad  \dfrac{\text{d} c}{\text{d} t} = - k \times c$ (soit en mathématiques $y' = - k \times y$),\end{center}
où $k = 7 \times 10^{-4}$.

\begin{enumerate}[start=5]
\item Résoudre sur $[0~;~+ \infty[$ cette équation différentielle.
\item Sachant que pour $t = 0$, la concentration initiale du saccharose vaut
0,4 mol$\cdot \text{L}^{-1}$, montrer que l'unique solution de l'équation $(E)$ est la fonction $c$ définie sur $[0~;~+ \infty[$ par 
\[c(t) = 0,4 \times \e^{-7\times 10^{-4}\times t}.\]

\item Déterminer la limite de $c(t)$ lorsque $t$ tend vers $+ \infty$.
\item Interpréter ce résultat dans le contexte de la production réalisée en laboratoire.
\end{enumerate}

\bigskip


