
\medskip

\textbf{Les questions 1, 2, 3 et 4 sont indépendantes les unes des autres.}

\medskip

\textbf{Question 1}

\medskip

Résoudre dans $\R$ l'inéquation :
\[\e^{2t} > 0,12.\]

\bigskip

\textbf{Question 2}

\medskip

On considère la fonction $F$ définie sur $\R$ par :
\[F(t)= a\e^{2t+6}.\]

\begin{enumerate}
\item $F$ est une primitive de la fonction $f$ définie sur $\R$ par $f(t) = 6 \e^{2t+6}$.

Déterminer la valeur de $a$.
\item Donner une autre primitive de la fonction $f$.
\end{enumerate}

\bigskip

\textbf{Question 3}

\medskip

On s'intéresse à l'équipement des habitants d'une grande ville en ordinateurs depuis 2000.

La part (exprimée en \,\%) des habitants de cette ville ayant au moins un ordinateur est modélisée par la fonction $f$ définie sur $[0~;~+ \infty]$ par :
\[f(t) = \dfrac{94,6}{1 + \e^{0,6 - 0,2t}}\]

où $t$ est la durée écoulée (en année) depuis l'année 2000.

Montrer que le taux d'équipement ne peut jamais être supérieur à 94,6\,\%.

\bigskip

\textbf{Question 4}

\medskip

Soit $f$ la fonction définie sur $]0~;~+ \infty]$ par:

\[f(x) = \dfrac{\e^x}{x^2 + 26x}.\]

Déterminer la limite de la fonction $f$ lorsque $x$ tend vers $+ \infty$.

\bigskip


