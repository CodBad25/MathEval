
\medskip

\begin{enumerate}[start=6]
\item Les solutions de  l'équation différentielle $y'+ay=0$ sur  $\R$ sont les fonctions $y$ définies par  

$y(x)=\lambda\e^{-ax}$ où $\lambda$ est une constante quelconque.

$a=\np{0.0085}$  par conséquent sur $[0~;~+\infty[$ $C(t)=\lambda\e^{-\np{0.0085}t}$ 
 où $\lambda$ est une constante quelconque.

Déterminons $\lambda$ sachant que :
\begin{align*}
C(0)&=\np{0.0042} \\
\lambda\e^{-\np{0.0085}\times 0}&=\np{0.0042} \\
\lambda\e^0&=\np{0.0042} \\
\lambda&=\np{0.0042}
\end{align*}

La fonction $C$ est la fonction définie sur $[0~;~+\infty[$ par $C(t)=\np{0.0042}\e^{-\np{0.0085}t}$.

\item Résolvons l'équation :
\begin{align*}
C(t) &= \np{0,00021} \\
\np{0.0042}\e^{-\np{0.0085}t}&=\np{0.00021} \\
\e^{-\np{0.0085}t}&=\dfrac{\np{0.00021}}{\np{0.0042}} \\
\e^{-\np{0.0085}t}&=\dfrac{1}{20} \\
-\np{0.0085}t&=-\ln(20) \\
t&=\dfrac{\ln(20)}{\np{0.0085}} \\
t&=\dfrac{\np{10000}\times\ln(20)}{\np{85}} \\
t&\approx \np{352.439}
\end{align*}

Une valeur approchée à la seconde près de la durée nécessaire pour que la concentration résiduelle en peroxodisulfate, correspondant à une oxydation de $95\,\%$ du réactif limitant, soit égale à \np{0,00021} mol.L$^{-1}$ est $\np[s]{352}$ .
\end{enumerate}

\bigskip


