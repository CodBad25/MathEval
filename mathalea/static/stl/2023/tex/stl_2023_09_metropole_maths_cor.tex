
\medskip

\textbf{Question 1}

\medskip

$f(0) = (4 \times 0 + 8) \times e^0 = 8 \times 1 = 8.$

$f(0)$ est bien un nombre entier, égal à 8.

\bigskip

\textbf{Question 2}

\medskip

On lit graphiquement sur la courbe $\mathcal{C}_f$ que :
$f(2) \approx -4.$

Soit A$(2~;~-4)$ et B$(4~;~0)$ deux points de $T$ (lecture graphique). Le coefficient directeur de $T$ est égale à :
\[a = \dfrac{y_{\text B}-y_{\text A}}{x_{\text B}-x_{\text A}}
=\dfrac{0 - (-4)}{4 - 2} = \dfrac{4}{2} = 2.\]

Comme le nombre dérivé $f'(x_\text{A})$ d'une fonction $f$ en un point A d'abscisse $x_\text{A}$ est le coefficient directeur de la tangente à la courbe représentative de $f$ au point d'abscisse $x_\text{A}$, il vient :
\[f'(2) \approx 2.\]

\bigskip

\textbf{Question 3}

\medskip

On applique la formule d'Al-Kashi :
\begin{align*}
\cos(\widehat{BAC}) &= \dfrac{AB^2 + AC^2 - BC^2}{2 \times AB \times AC} \\
&= \dfrac{5^2 + 10^2 - 8^2}{2 \times 5 \times 10} \\
&= \dfrac{25 + 100 - 64}{100} \\
&= \dfrac{61}{100} \\
&= 0,61.
\end{align*}

\bigskip

\textbf{Question 4}

\medskip

Calcul de la dérivée de $F$ :
\begin{align*}
F'(x) &= -1 \times 3 \times x^{3 - 1} + 4 \times 2 \times x^{2 - 1} \\
&= -3x^2 + 8x \\
&= f(x).
\end{align*}

$F(x)$ est bien une primitive de $f(x)$ sur $\R$.

\bigskip


