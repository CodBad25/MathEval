
\medskip

\begin{enumerate}[start=8]
\item Calculons la dérivée de $f(t)$ :
\[f'(t) = -k \times e^{-kt} = -k f(t).\]

On a donc :
\[f'(t) + k f(t) = -k f(t) + k f(t) = 0.\]

Ainsi $f(t)$ est bien solution de l'équation différentielle.

\item Soit les points A$(0~;~3,7)$ et B$(25~;~1,7)$ du graphique (semblant appartenir à la droite, éloignés, lecture facile des coordonnées). 

\[a = \dfrac{y_{\text B}-y_{\text A}}{x_{\text B}-x_{\text A}}
=\dfrac{1,7 - 3,7}{25 - 0} = - \dfrac{2}{25} = -\np{0,08}\]

La pente $a$ de la droite étant d'environ $-0,08$, on a :
$a = -k \Rightarrow k \approx 0,08.$

\item On sait que le temps de demi-réaction \( t_{\frac12} \) vérifie :
\[[H_2O_2](t_{\frac12}) = \dfrac{1}{2}[H_2O_2]_0.\]

On part de l'expression :
\[[H_2O_2](t) = [H_2O_2]_0 \cdot e^{-kt}.\]

En remplaçant :
\[\dfrac{1}{2}[H_2O_2]_0 = [H_2O_2]_0 \cdot e^{-k t_{\frac12}} \Rightarrow \dfrac{1}{2} = e^{-k t_{\frac12}}.\]

On prend le logarithme népérien :
\[\ln \left( \dfrac{1}{2} \right) = -k t_{\frac12} \Rightarrow -\ln 2 = -k t_{\frac12} \Rightarrow t_{\frac12} = \dfrac{\ln 2}{k}.\]

\item $t_{\frac12} = \dfrac{\ln 2}{k} = \dfrac{\ln 2}{0,08} \approx 8,66\ \text{min}.$
\end{enumerate}

\bigskip


