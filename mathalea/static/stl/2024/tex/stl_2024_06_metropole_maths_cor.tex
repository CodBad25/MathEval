
\medskip

\textbf{Question 1}

\medskip

D'après le graphique: $f(0)=-2$.

\begin{center}
\psset{xunit=1.5cm,yunit=1.5cm,labelFontSize=\scriptstyle,showorigin=false}
\begin{pspicture*}(-2.5,-3.5)(2.30,1.7)
\psgrid[subgriddiv=5,,gridlabels=0,gridcolor=gray,subgridcolor=lightgray] (-3,-4)(3,2)
\psaxes[linewidth=1.25pt]{->}(0,0)(-2.5,-3.5)(2.30,1.7)
\def\Func{x 2 exp x 2 mul  add 2 sub}
\psplot[plotpoints=1000,linewidth=1.25pt,linecolor=blue]{-2}{1}{\Func}
\uput[dl](0,0){O}
\psdots[linecolor=red](0,-2)
\end{pspicture*}
\end{center}

\bigskip

\textbf{Question 2}

\medskip

D'après le cours: $\ds\lim_{x\to - \infty} \e^{x}=0$ donc $\ds\lim_{x\to - \infty} 2\e^{x}=0$

Or $\ds\lim_{x\to - \infty} 3x=-\infty$

Donc par somme: $\ds\lim_{x\to - \infty} 2\e^{x} + 3x - 2=-\infty$ et donc: $\ds\lim_{x\to - \infty} f(x)=-\infty$

\bigskip

\textbf{Question 3}

\medskip

$f(x) = (3x + 2)\e^{x-1}$ donc $f(1)=\left (3\times 1 +2\right ) \e^{1-1}=5\e^{0}$

Or $\e^{0}=1$ donc $f(1)=5$ donc $f(1)$ est un entier.

\bigskip

\textbf{Question 4}

\medskip

La fonction $x \longmapsto 2x+1$ a pour primitive la fonction $x\longmapsto x^2 + x$.

La fonction $x\longmapsto \dfrac{1}{x}$ a pour primitive la fonction $x\longmapsto \ln (x)$ sur $]0~;~+\infty[$.

Donc la fonction $F$ définie par $F(x)=x^2+x-\ln(x)$ est une primitive de la fonction $f$ sur $]0~;~+\infty[$.

\bigskip


