
\begin{center}
\textbf{Stabilité d'un antibiotique}
\end{center}

L'amoxicilline (noté ici AMOX) est un antibiotique qui possède un large spectre d'action sur certaines infections bactériennes, mais son action peut être altérée par des enzymes produites par certaines bactéries résistantes.

L'amoxicilline peut être utilisée sous forme de poudre. Après ajout d'eau et agitation, on obtient une solution facilement assimilable. Cependant l'amoxicilline est peu stable en milieu aqueux : elle subit une réaction de dégradation avec l'eau (hydrolyse).

\bigskip

\textbf{Dégradation de l'amoxicilline en solution aqueuse}

\medskip

La dégradation de l'amoxicilline est étudiée au laboratoire, à 30\textcelsius{} et à un pH valant 3,5.

La valeur de la concentration initiale en amoxicilline vaut $C_0 = \np{1600}$ \textmu g $\cdot$ mL$^{-1}$.

La concentration de l'amoxicilline à l'instant $t$, notée $C_{\text{Amox}}(t)$, est évaluée toutes les vingt-quatre heures.

\medskip

On fait l'hypothèse que la dégradation de l'amoxicilline suit une loi cinétique d'ordre 1.

\begin{enumerate}[start=2]
\item Établir l'équation différentielle du premier ordre vérifiée par la fonction $C_{\text{Amox}}(t)$. On notera $k_{\text{Amox}}$ la constante de vitesse.
\end{enumerate}

Pour une loi cinétique d'ordre 1, les solutions générales $C(t)$ de l'équation différentielle vérifient l'égalité $\ln \left(\frac{C(t)}{C(0)}\right) = - kt$ pour une certaine valeur de $k$.

\medskip

L'ajustement linéaire de points relevés expérimentalement dans les conditions opératoires données permet d'obtenir une droite passant par les points 0(0~;~0) et A$(96~;~-0,70)$ dont l'abscisse est $t$ en h et l'ordonnée est $\ln\left(\frac{C_{\text{Amox}}(t)}{C_0}\right)$.

\begin{enumerate}[start=4]
\item Déterminer une valeur arrondie à $10^{-4}$ du coefficient directeur de la droite (OA).
En utilisant cette valeur arrondie, en déduire que la droite (OA) a pour équation :
\[y = -0,0073t\]

\item L'ajustement précédent nous permet d'écrire  \mbox{$\ln \left(\frac{C_{\text{Amox}}(t)}{C_0}\right)  = -0,0073t$}, pour tout $t$ appartenant à $ [0~;~+\infty[$.
	\begin{enumerate}
		\item  En déduire que $C_{\text{\text{Amox}}}(t)= \np{1 600}\times \e^{-\np{0,0073}t}$ pour tout $t$ appartenant à $[0~;~+ \infty[$.
		\item Déterminer la limite de la fonction $C_{\text{Amox}}$ en +$\infty$.
		\item Dresser le tableau des variations de la fonction $C_{\text{Amox}}$ sur $[0~;~+\infty[$.
	\end{enumerate}
\end{enumerate}

\bigskip


