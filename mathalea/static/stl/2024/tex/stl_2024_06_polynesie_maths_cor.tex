
\medskip

\begin{enumerate}

\item La bonne réponse est \textbf{a.}

\begin{center}
\fbox{\parbox{5cm}{\ttfamily
from math import exp\\ 
for k in range(10) :\\
\hspace*{1cm}x=k/10       \\
\hspace*{1cm}y=5*exp(2*x+1)  \\        
\hspace*{1cm}print(y) }}
\end{center}

\item On résout dans $\R $ l'équation $f(x) = 5$ :

\[f(x)=5
\iff 5\e^{2x+1}=5
\iff \e^{2x+1}=1
\iff 2x+1=0
\iff x=-\dfrac{1}{2}.\]

\item $f(0)=5\e^{2\times 0+1} = 5\e \approx 13,6>5$, donc l'affirmation est \textbf{fausse}.

\item 
\begin{enumerate}
	\item $F'(x) = \dfrac{5}{2}\times 2\e^{2x+1} = 5 \e^{2x+1}  = f(x)$, \\
donc la fonction $F$ est une primitive sur $\R$ de la fonction $f$.
		
	\item 
$\ds\int_0^1 f(x) \d x
= F(1)-F(0)
= \left ( \dfrac{5}{2}\e^{2\times 1+1} \right ) - \left ( \dfrac{5}{2}\e^{2\times 0+1} \right )
= \dfrac{5}{2} \left ( \e^{3} - \e\right )\approx 43$.		
\end{enumerate}
\end{enumerate}

\bigskip


