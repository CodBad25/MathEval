
\medskip

\textbf{Les questions 1, 2, 3 et 4 sont indépendantes les unes des autres.}

\medskip

\textbf{Question 1}

\medskip

On considère ci-dessous la courbe représentative d'une fonction $f$ définie sur [—2 ; 1].

Par lecture graphique, déterminer $f(0$).

\begin{center}
\psset{xunit=1.5cm,yunit=1.5cm,labelFontSize=\scriptstyle,showorigin=false}
\begin{pspicture}(-2.5,-3.5)(2.30,1.7)
\multido{\n=-2.4+0.2}{24}{\psline[linewidth=0.75pt,linecolor=lightgray](\n,-3.4)(\n,1.6)}
\multido{\n=-3.4+0.2}{25}{\psline[linewidth=0.75pt,linecolor=lightgray](-2.5,\n)(2.3,\n)}
\multido{\n=-2+1}{5}{\psline[linewidth=0.75pt](\n,-3.4)(\n,1.6)}
\multido{\n=-3+1}{5}{\psline[linewidth=0.75pt](-2.5,\n)(2.3,\n)}
\psaxes[linewidth=1.25pt]{->}(0,0)(-2.5,-3.5)(2.3,1.6)
\def\Func{x 2 exp x 2 mul  add 2 sub}
\psplot[plotpoints=1000,linewidth=1.25pt,linecolor=blue]{-2}{1}{\Func}
\uput[dl](0,0){O}
\end{pspicture}
\end{center}

\bigskip

\textbf{Question 2}

\medskip

Soit $f$ la fonction définie sur $\R$ par $f(x) = 2\e^{x}+3x-2$.

Déterminer, en la justifiant, la limite de la fonction $f$ lorsque $x$ tend vers $-\infty$.
 
\bigskip

\textbf{Question 3}

\medskip

Soit $f$ la fonction définie sur $\R$ par $f(x) = (3x + 2)\e^{x-1}$.

En détaillant les calculs, justifier que $f(1)$ est un entier.

\bigskip

\textbf{Question 4}

\medskip

Soit $f$ la fonction définie sur $]0~;~+\infty[$ par $f(x) = 2x +1-\dfrac{1}{x}$.

Déterminer une primitive $F$ de la fonction $f$ sur $]0~;~ +\infty[$.

\bigskip


