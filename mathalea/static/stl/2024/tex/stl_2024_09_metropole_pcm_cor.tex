
\medskip

\begin{enumerate}[start=3]
\item Calculons la dérivée de $f(t)$ :
\[f'(t) = 41 \times (-k) \times e^{-kt} = -41k e^{-kt}.\]

On a donc :
\[f'(t) + k \times f(t) = -41k e^{-kt} + k \times 41 e^{-kt} = 0.\]

Ainsi, $f(t)$ est bien solution de l'équation différentielle.

De plus :
\[f(0) = 41 e^{-k \times 0} = 41 \times 1 = 41.\]
La condition initiale $f(0) = 41$ est vérifiée.

\medskip

\textit{Correction alternative :}

La solution générale de différentielle linéaire du premier ordre :
\[y' + k \times y = 0,\]

est de la forme :
\[y(t) = C e^{-k t}.\]

On impose ici la condition initiale : $f(0) = 41$, soit :
\[f(0) = C e^{-k \times 0} = C \times 1 = C \Rightarrow C = 41\]

On en déduit l'unique solution vérifiant la condition initiale :
\[f(t) = 41 e^{-k t}.\]

Ainsi, la fonction $f(t) = 41 e^{-k t}$ est bien la solution de l'équation différentielle $y' + k y = 0$ qui vérifie la condition initiale $f(0) = 41$.

\item On a :
\[f(t) = 41 e^{-kt} \Rightarrow \ln(f(t)) = \ln(41 \times e^{-kt}) = \ln(41) + \ln(e^{-kt}) = \ln(41) - kt.\]

Donc :
\[\ln(f(t)) = -kt + \ln(41).\]
\end{enumerate}

\begin{enumerate}[start=6]
\item Soit les points A$(4~;~3,5)$ et B$(44~;~1)$ du graphique (semblant appartenir à la droite, éloignés, lecture facile des coordonnées). 

\[a = \dfrac{y_{\text B}-y_{\text A}}{x_{\text B}-x_{\text A}}
=\dfrac{1 - 3,5}{44 - 4} = - \dfrac{2,5}{40} = -\np{0,0625}\]

Donc $-\np{0,063}$ est une valeur arrondie à $10^{-3}$ du coefficient directeur $a$ de la droite (AB).

\item On déduit de la question précédente :
\[k = - a = -(-0,063) = 0,063 \, \text{min}^{-1}.\]

\end{enumerate}

\bigskip


