
\vspace{0.25cm}

\emph{Les calculs seront détaillés sur la copie.}

\smallskip

Pour limiter les bouteilles en plastique, une association souhaite offrir une gourde en métal à chaque élève de sixième.\\
Voici les prix que propose un commerçant.

\begin{center}
Facture

%\smallskip

\renewcommand{\arraystretch}{1.4}
\begin{tabular}{|>{\cellcolor{lightgray}} c |c|c|>{\raggedleft\arraybackslash} m{3.5cm}|>{\raggedleft\arraybackslash} m{3cm}|}
\hline
\rowcolor{lightgray} & A & B & \centering C & \centering\arraybackslash  D\\
\hline
1 & Articles & Quantités & \centering Prix unitaire en F (prix d'une gourde) & \centering\arraybackslash Montant total \newline{} (en F)\\
\hline
2 & Gourdes vertes & 45 & \np{1500,00} & \np{67500,00}\\
\hline
3 & Gourdes bleues & 29 & \np{1200,00} & \np{34800,00}\\
\hline
4 & Gourdes rouges & 36 & \np{1300,00} & \\
\hline
5 & Gourdes grises & & \np{1125,00} & \\
\hline
6 & \multicolumn{2}{c|}{~} & \centering Total HT & \np{227850,00}\\
\cline{1-1} \cline{4-5} 
7 & \multicolumn{2}{|c|}{~} & \centering TVA 13\,\% & \\
\cline{1-1} \cline{4-5} 
8 & \multicolumn{2}{|c|}{~} & \centering Total TTC & \\
\cline{1-1} \cline{4-5} 
\end{tabular}
\end{center}

\begin{flushleft}
\textbf{Partie A}
\end{flushleft}

\begin{enumerate}
\item  \textbf{Justifier} par un calcul le montant total pour les gourdes vertes.
\item \textbf{Compléter} le tableau précdent. \textbf{Détailler} les calculs sur la copie.

\item \textbf{Recopier} la formule que l'on doit choisir dans la cellule D8 parmi les 3 propositions suivantes :

\hfill\hfill \fbox{\texttt{= SOMME(D2:D5)}} \hfill \fbox{\texttt{= D6 + D7}} \hfill \fbox{\texttt{= D6 + 13}}\hfill~

\end{enumerate}

\begin{flushleft}
\textbf{Partie B}
\end{flushleft}

Les gourdes sont toutes distribuées aux élèves. Parmi les 180 élèves de sixième, on choisit un élève au hasard.

\begin{enumerate}[resume]
\item  \textbf{Calculer} la probabilité qu'il ait une gourde rouge.
\item \textbf{Déduire} la probabilité qu'il ait une gourde d'une autre couleur.
\end{enumerate}


