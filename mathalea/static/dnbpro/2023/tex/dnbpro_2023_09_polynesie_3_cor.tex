
\vspace{0.25cm}

De 2000 à 2015, on a noté l'évolution des températures moyennes à Tahiti au mois de janvier.

%\begin{center}
%\psset{xunit=0.8cm,yunit=1.2cm}
%\begin{pspicture}(0,26)(17,30)
%\psgrid[subgriddiv=5,  gridlabels=0, subgridcolor=lightgray,gridcolor=gray] 
%\psaxes[labelFontSize=\scriptstyle,arrowsize=3pt 3, ticksize=-2pt 2pt, labels=y,Oy=26]{->}(0,26)(0,26)(17,30)
%\def\data{1 28.6 2 27.6 3 28 4 28 5 28.1 6 28.3 7 27.8 8 27.6 9 26.9 10 27.9 11 28.7 12 27.2 13 26.9 14 26.8 15 27.4 16 28.1}
%\psset{dotstyle=x,dotscale=2,linecolor=blue}
%\listplot[plotstyle=line,showpoints]{\data}
%\multido{\i=1+1,\I=2000+1}{16}{\uput[d](\i,26){\scriptsize \I}}
%\uput*[r](0.2,29.6){\small Températures moyennes (en \degres{}C)}
%\uput*[u](16,26){\small Années} 
%\end{pspicture}
%\end{center}

\medskip

\begin{enumerate}
\item À  l'aide du graphique, on peut dire que:
\begin{enumerate}
\item la température moyenne en janvier 2002 est de 28\,\degres{}C;
\item la température moyenne en janvier 2006 est de $27,8$\,\degres{}C;
\item l'année où la température moyenne en janvier est la plus basse est 2013 avec une température de  $26,8$\,\degres{}C.
\end{enumerate}
\item% \textbf{Décrire}, à l'aide du graphique, l'évolution des températures moyennes en janvier de 2013 à 2015.
De 2013 à 2015 l'évolution des températures moyennes en janvier est croissante.

\item On place les points dans le repère ci-après correspondant aux températures moyennes en janvier pour 2016, 2017, 2018 et 2019.

\begin{center}
\begin{tabular}{|c|c|}
\hline
\hspace*{0.5cm}\textbf{Date}\hspace*{0.5cm} & \textbf{Températures moyennes en janvier (en \degres{}C)}\\
 \hline
 2016 & $28$\\
 \hline
 2017 & $27,8$\\
 \hline
 2018 & $28$\\
 \hline
 2019 & $28,2$\\
 \hline
\end{tabular}
\end{center}
\end{enumerate}

{\psset{xunit=0.7cm,yunit=1.2cm}
\begin{pspicture}(0,26)(21,30)
\psgrid[subgriddiv=5,  gridlabels=0, subgridcolor=lightgray,gridcolor=gray] 
\psaxes[labelFontSize=\scriptstyle,arrowsize=3pt 3, ticksize=-2pt 2pt, labels=y,Oy=26]{->}(0,26)(0,26)(21,30)
\def\data{1 28.6 2 27.6 3 28 4 28 5 28.1 6 28.3 7 27.8 8 27.6 9 26.9 10 27.9 11 28.7 12 27.2 13 26.9 14 26.8 15 27.4 16 28.1}
\listplot[plotstyle=line,showpoints,dotstyle=x,dotscale=2,linecolor=blue]{\data}
\def\data{17 28 18 27.8 19 28 20 28.2}
\listplot[plotstyle=line,linecolor=red]{16 28.1 17 28}
\listplot[plotstyle=line,showpoints,dotstyle=x,dotscale=2,linecolor=red]{\data}
\multido{\i=1+1,\I=2000+1}{16}{\uput[d](\i,26){\scriptsize \I}}
\multido{\i=17+1,\I=2016+1}{4}{\uput[d](\i,26){\scriptsize\red \I}}
\uput*[r](0.2,29.6){\small Températures moyennes (en \degres{}C)}
\uput*[u](20,26){\small Années} 
\end{pspicture}}

\vspace{0.5cm}

