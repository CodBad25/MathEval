
\vspace{0.25cm}

\begin{minipage}{0.6\linewidth}
L'association souhaite installer un poulailler identique au modèle ci-contre.
\end{minipage}
\hfill
\begin{minipage}{0.35\linewidth}
\includegraphics[width=0.9\linewidth]{MetroPro23_2}
\end{minipage}

\medskip

\includegraphics[width=\linewidth]{MetroPro23_3}

%\medskip

\begin{enumerate}
\item  Nommer les figures planes qui composent la vue éclatée du poulailler de la figure 2.
\item La figure 3 ci-dessus représente le cadre latéral du poulailler.
\begin{enumerate}
\item  En utilisant la relation de Pythagore dans le triangle PRL, montrer que la longueur RL arrondie au centième vaut $0,71$ m.
\item En déduire la largeur TL du cadre du poulailler posé au sol.
\item Calculer l'aire de la surface du sol délimitée par le cadre du poulailler.
\end{enumerate}
\item L'association achète un modèle dont les dimensions au sol sont :

~\hfill Longueur = $2,50$ m\qquad\qquad Largeur = $1,42$ m\hfill~

Un membre de l'association affirme qu'il est possible de placer six poulaillers sur le terrain. Justifier qu'il a raison en faisant un schéma sur la copie.

\smallskip

\textbf{Indication}: on pourra utiliser la figure d'aide à la résolution ci-dessous sachant que chaque poulailler peut être disposé dans le sens de la longueur ou de la largeur.

\parbox{4.4cm}{
Exemple d'un premier poulailler placé dans le sens de la longueur.\\
\\
Échelle: 1cm pour 1 m\\
}\hfill
\parbox{9cm}{
\begin{flushright}
\psset{unit=1cm,arrowsize=3pt 2}
\def\xmin{-1}   \def\xmax{8} \def\ymin{-1}   \def\ymax{5}
\begin{pspicture}(\xmin,\ymin)(\xmax,\ymax)
\psframe[fillstyle=solid,fillcolor=lightgray,linecolor=gray!30](0,0)(7,4)
\psline{<->}(0,-0.3)(7,-0.3) \uput[d](3.5,-0.3){7 m}
\psline{<->}(7.3,0)(7.3,4) \uput[r]{-90}(7.4,2){4 m}
\psframe(0,4)(2.5,2.58)
\psline{<->}(0,4.3)(2.5,4.3) \uput[u](1.25,4.3){$2,50$ m}
\psline{<->}(-0.3,4)(-0.3,2.58) \uput[l]{90}(-0.3,3.29){$1,42$ m}
{\footnotesize\uput[u](1.25,3.2){Cadre posé} \uput[d](1.25,3.35){au sol} }
\end{pspicture}
\end{flushright}}
\end{enumerate}


