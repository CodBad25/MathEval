
\vspace{0.25cm}

\begin{minipage}{0.65\linewidth}
\emph{Les calculs seront détaillés sur la copie.}

\medskip

Tehani souhaite une étagère murale composée de 2 planches en manguier et d'une corde.\\
Les deux planches représentées par les segments [MN] et [BC] mesurent respectivement 20 cm et 40 cm.\\
Pour des raisons d'esthétique, elle décide d'espacer les 2 planches et le point de fixation A de 30 cm à chaque fois.\\
On donne : AP = PH = 30 cm, (MN) // (BC) et (AH) $\perp$ (BC).
\end{minipage}
\hfill
\begin{minipage}{0.3\linewidth}
\begin{flushright}
\psset{unit=1cm,radius=0pt}
\scalebox{0.6}{
\begin{pspicture}(-1,-0.5)(7,5.5)
%\psgrid[subgriddiv=1,  gridlabels=0, gridcolor=lightgray] 
\Cnode*(0,0){B} \Cnode*(6,0){C} \Cnode*(3,5){A} \Cnode*(3,0){H}
\uput[u](A){A} \uput[190](B){B} \uput[-10](C){C} \uput[d](H){H} 
\psframe(H)(2.7,0.3) \psline(B)(A)(C) \psline[linewidth=2pt](B)(C)
\Cnode*(1.5,2.5){M} \Cnode*(4.5,2.5){N} \Cnode*(3,2.5){P}
\uput[l](M){M} \uput[r](N){N} \uput[ur](P){P} 
\psline[linewidth=2pt](M)(N) \psline(A)(H)
\rput(1.5,0){\pmb /} \rput(4.5,0){\pmb /} 
\uput[d](3,-0.5){Le schéma n'est pas à l'échelle.}
\end{pspicture}}
\end{flushright}
\end{minipage}

\begin{enumerate}
\item  %\textbf{Donner} les mesures des longueurs MN et BC, exprimées en cm.
D'après le texte: MN = 20 cm et BC = 40 cm.
\item %\textbf{Calculer} la longueur BH. \textbf{Exprimer} le résultat en cm.
D'après la figure, BH = HC donc $\text{BH}=\dfrac{\text{BC}}{2}$ donc BH = 20 cm.
\item %\textbf{Calculer} la longueur AH. \textbf{Exprimer} le résultat en cm.
AH = AP + PH; or AP = PH = 30 cm donc AH = 60 cm.
 \end{enumerate}
 
 Pour terminer son étagère, elle doit rajouter une corde (représentée par les segments [AB] et [AC]).

\begin{enumerate}[resume]
 \item  %\textbf{Calculer} la longueur AB. \textbf{Exprimer} le résultat en cm
Le triangle ABH est rectangle en H donc, d'après le théorème de Pythagore, on a:\\
$\text{AB}^2=\text{AH}^2+\text{BH}^2$.

Or AH = 60 et BH = 20, donc $\text{AB}^2=60^2+20^2= \np{3600} + 400=\np{4000}$.\\ 
On en déduit que $\text{AB}=\ds\sqrt{\np{4000}}$ soit environ $63,2$ cm.

\item %\textbf{Justifier} si une corde de 100 cm est assez longue.
Le triangle ABC est isocèle donc AB = AC.

La corde doit donc avoir une longueur au moins égale à 2 fois AB, soit $2\times 63,2$ c'est-à-dire $126,4$ cm.

Une corde de 1 m n'est donc pas assez longue.

\end{enumerate}

\vspace{0.5cm}

