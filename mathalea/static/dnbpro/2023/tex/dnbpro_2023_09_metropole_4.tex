
\vspace{0.25cm}

Dans un jeu vidéo réalisé avec le logiciel Scratch, l'avatar d'un joueur au maillot foncé prend le ballon à la sortie d'une mêlée puis se déplace vers la zone grisée. Une capture d'écran de la situation de jeu est donnée ci-dessous.

\medskip

Les croix représentent la position des joueurs de chaque équipe. Ces joueurs ne participent pas à la mêlée.

\medskip

\includegraphics[width=\linewidth]{MetroProSept23_4}


La partie est gagnée lorsque l'avatar entre dans la zone grisée en évitant les autres joueurs.

Les programmes 1, 2 et 3 ci-dessous simulent différents chemins du joueur qui se déplace.

\begin{enumerate}
\item Tracer à main levée, sur la capture d'écran précédente, le chemin parcouru par l'avatar du joueur lorsqu'on utilise le programme 3 ci-dessous.

La ressource d'aide ci-dessous précise les commandes  \og s'orienter \fg{}  et \og tourner \fg{}.

\smallskip

\begin{tabularx}{\linewidth}{ |*3{>{\centering\arraybackslash} X|}}
\hline
\textbf{Programme 1} & \textbf{Programme 2} & \textbf{Programme 3}\\
\hline
&&\\
\begin{scratch}
\blockinit{quand \greenflag est cliqué}
\blockmove{aller à x: \ovalnum{20} y: \ovalnum{20}}
\blockmove{s'orienter à \ovalnum{90}}
\blockmove{avancer de \ovalnum{120} pas}
\blockmove{tourner \turnleft{} de \ovalnum{90} degrés}
\blockmove{avancer de \ovalnum{40} pas}
\blockmove{tourner \turnright{} de \ovalnum{90} degrés}
\blockmove{avancer de \ovalnum{20} pas}
\blockmove{tourner \turnleft{} de \ovalnum{90} degrés}
\blockmove{avancer de \ovalnum{20} pas}
\end{scratch}
&
\begin{scratch}
\blockinit{quand \greenflag est cliqué}
\blockmove{aller à x: \ovalnum{20} y: \ovalnum{20}}
\blockmove{s'orienter à \ovalnum{90}}
\blockmove{avancer de \ovalnum{40} pas}
\blockmove{tourner \turnleft{} de \ovalnum{90} degrés}
\blockmove{avancer de \ovalnum{80} pas}
\blockmove{tourner \turnright{} de \ovalnum{90} degrés}
\blockmove{avancer de \ovalnum{40} pas}
\blockmove{tourner \turnleft{} de \ovalnum{90} degrés}
\blockmove{avancer de \ovalnum{90} pas}
\end{scratch}
&
\begin{scratch}
\blockinit{quand \greenflag est cliqué}
\blockmove{aller à x: \ovalnum{20} y: \ovalnum{20}}
\blockmove{s'orienter à \ovalnum{90}}
\blockmove{avancer de \ovalnum{80} pas}
\blockmove{tourner \turnleft{} de \ovalnum{90} degrés}
\blockmove{avancer de \ovalnum{80} pas}
\blockmove{tourner \turnright{} de \ovalnum{90} degrés}
\blockmove{avancer de \ovalnum{100} pas}
\blockmove{tourner \turnleft{} de \ovalnum{90} degrés}
\blockmove{avancer de \ovalnum{20} pas}
\end{scratch}\\
&&\\
\hline
\end{tabularx}

\smallskip

\begin{center}
\textbf{Ressources}
\end{center}

\setlength{\columnseprule}{1pt}
\begin{multicols}{2}
\begin{center}
\begin{scratch}\blockmove{s'orienter à \ovalnum{90}}\end{scratch}
\parbox{3cm}{\psset{unit=0.7cm,arrowsize=2pt 1}
\begin{pspicture}(-2,-2)(2,2)
%\psgrid
\psframe[fillstyle=solid,fillcolor=gray!70,linecolor=gray!70](-1.7,-1.7)(1.7,1.7)
\pspolygon[fillstyle=solid,fillcolor=gray!70,linecolor=gray!70](-0.3,1.7)(0,2)(0.3,1.7)
\pscircle[fillstyle=solid,fillcolor=gray,linecolor=gray](0,0){1.2}
\pswedge[fillstyle=solid,fillcolor=gray!50,linecolor=gray!50](0,0){1.2}{0}{90}
\multido{\i=0+15}{24}{\psline[linecolor=white](0.9;\i)(1.1;\i)}
\psline[linecolor=white,linewidth=1.5pt](0,1.2)(0,0)(1.2,0)
\psdots[dotscale=1.5,linecolor=white](0,0)
\pscircle[fillstyle=solid,fillcolor=white,linecolor=white](1.2,0){0.3}
\psline[linewidth=1.5pt]{->}(1,0)(1.4,0)
\end{pspicture}}

\smallskip

\emph{Le joueur s'oriente pour courir\\ dans le sens de la flèche}
\end{center}

\columnbreak

\begin{center}

\vspace*{0.48cm}

\begin{scratch}\blockmove{tourner \turnleft{} de \ovalnum{90} degrés}\end{scratch}

\smallskip

\emph{Le joueur tourne de 90\degres{} \\
dans le sens de la flèche}
\end{center}
\end{multicols}

\item Choisir parmi les trois programmes proposés celui qui permet à l'avatar de gagner.

Indiquer ce choix sur la copie. Justifier en traçant le chemin correspondant sur la capture d'écran précédente.

\textbf{Remarque} : Les chemins des 3 programmes se superposent en début de parcours.

\end{enumerate}


