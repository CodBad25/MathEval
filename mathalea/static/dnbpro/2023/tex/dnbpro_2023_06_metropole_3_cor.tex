
\vspace{0.25cm}

Les jardins partagés d'une commune sont gérés par une association. Celle-ci compte 60 membres qui adhèrent pour des objectifs différents. Le document ci-dessous regroupe ces objectifs et les effectifs correspondants.


\begin{figure}[t!]
\begin{multicols}{2}

\psset{xunit=0.7cm,yunit=0.2cm,arrowsize=3pt 2}
\def\xmin{-1}   \def\xmax{10} \def\ymin{-2}   \def\ymax{36}
\begin{pspicture}(\xmin,\ymin)(\xmax,\ymax)
%\psset{yMaxValue=\ymax,yMinValue=\ymin}
%\psgrid[xunit=10cm,yunit=0.4cm,subgriddiv=1, gridlabels=0, subgridcolor=lightgray, gridcolor=gray](0,0)(0.7,16)
\psaxes[labelFontSize=\scriptstyle, ticksize=-2pt 2pt,ticks=y,labels=none](0,0)(0,0)(\xmax,32)
\multido{\i=0+2}{17}{\psline[linecolor=gray](0,\i)(10,\i) \uput[l](0,\i){\scriptsize \i}}
\multido{\i=1+2,\I=1+1}{5}{\uput[d](\i,0){\scriptsize Objectif \I}}
\psaxes[labelFontSize=\scriptstyle, ticksize=-2pt 2pt,ticks=y,labels=none](0,0)(0,0)(\xmax,32)
\listplot[plotstyle=bar,barwidth=0.6,fillstyle=solid,fillcolor=lightgray]{1 3 3 9 5 12 7 6 9 30}
\uput[r](-1,34){Effectifs}
\end{pspicture}
\mbox{\textbf{Document 1 : Répartition des objectifs d'adhésion des membres du jardin partagé}}

\columnbreak

\begin{flushright}
\vspace*{0.6cm}

\fbox{
\parbox{6cm}{
\medskip

~\hfill\textbf{Légende}\hfill~\\

\small

Objectif 1: Être autosuffisant\\

Objectif 2: Profiter d'un loisir\\
 
Objectif 3: Agir pour l'environnement\\

Objectif 4: Partager avec les autres\\

Objectif 5: Être en contact avec la nature\\
}}
\end{flushright}
\end{multicols}
\end{figure}


\begin{enumerate}
\item  Il y a 12 membres qui ont adhéré pour l'objectif 3.

\item Il y a 30 membres qui ont adhéré pour l'objectif 5.
Or $\dfrac{30}{60}\times 100=50$ 
donc le pourcentage de membres ayant adhéré pour l'objectif 5 est de $50\,\%$.

\item On s'intéresse à la répartition des âges des adhérents de l'association.
\begin{enumerate}
\item On complète la valeur manquante en cellule \texttt{B4} du tableur.

\begin{center}
\textbf{Tableau de répartition par classe d'âge}

\smallskip

{\renewcommand{\arraystretch}{1.1}
\begin{tabular}{>{\cellcolor{lightgray!50}} c| >{\centering\arraybackslash}m{5cm} | >{\centering\arraybackslash}m{2cm}|}
\hline
\rowcolor{lightgray!50} & A & B\\
\hline
1 & Classe d'âge des membres & Effectifs\\
\hline
2 & Moins de 20 ans & 12\\
\hline
3 & De 20 à 60 ans inclus & 29\\
\hline
4 & Plus de 60 ans & $\blue 19$\\
\hline
5 & Total & 60\\
\hline
\end{tabular}}
\end{center}

\item Parmi les formules proposées, on coche  celle à saisir dans la cellule \texttt{B4} pour obtenir la valeur manquante.

\begin{tabular}[t]{*3{m{0.3\linewidth}}}
$\square$ \texttt{= B2 + B3 - B5} & $\blue\boxtimes\texttt{ = B5 - (B2 + B3)}$ & $\square$ \texttt{= B5 - B3 + B2}
\end{tabular}

\item  On complète le diagramme circulaire

%\begin{center}
\hfill{\footnotesize
\psset{unit=0.65cm}
\begin{pspicture}(-4,-4.1)(4,4)
%\psgrid
\pscircle(0,0){4}
\pswedge[fillstyle=solid,fillcolor=lightgray!50](0,0){4}{90}{204}
\pswedge[fillstyle=solid,fillcolor=lightgray](0,0){4}{204}{378}
\rput(1.8,1.5){Moins de 20 ans}
\rput*(-2,1){$\blue\text{Plus de 60 ans}$}
\rput*(0,-2){$\blue\text{De 20 à 60 ans inclus}$}
\end{pspicture}}
%\end{center}

\item Un adhérent affirme : \og   Plus d'un quart des membres a moins de 20 ans.  \fg{}

%Cette affirmation est-elle exacte ? Justifier la réponse.
Un quart de 60 correspond à 15 et il n'y a que 12 membres de moins de 20 ans.

L'affirmation est donc fausse.
\end{enumerate}
\end{enumerate}

%\vspace{0.5cm}

