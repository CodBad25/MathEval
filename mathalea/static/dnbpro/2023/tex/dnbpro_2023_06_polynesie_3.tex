
\medskip
 
\textbf{PARTIE A}

\medskip

Terii vend les produits de sa ferme au marché de Papeete sur Tahiti. Il a relevé et classé, par ordre croissant, les masses de gingembre (en kg) vendues au mois de mai.

Voici les relevés statistiques de 19 ventes réalisées au mois de mai :

\smallskip

\begin{tabularx}{\linewidth}{|*{19}{>{\centering\arraybackslash}X|}}
\hline
3 & 3 & 4 & 4 & 4 & 5 & 5 & 5 & 6 & 6 & 7 & 7 & 7 & 8 & 8 & 9 & 10 & 11 & 12\\
\hline
\end{tabularx}


\begin{enumerate}
\item \textbf{Calculer} l'étendue de cette série statistique.
\item \textbf{Déterminer} la médiane de cette série statique.
\item \textbf{Calculer} la masse moyenne de ces ventes. \textbf{Arrondir} le résultat au dixième.
\item Terii estime que la vente sur un mois est rentable lorsque les masses médiane et moyenne des ventes sont supérieures ou égales à 6 kg. Est-ce le cas pour le mois de mai ?\\
 \textbf{Justifier} la réponse.
\end{enumerate}

\medskip

\textbf{PARTIE B}

\medskip

Terii vend 500 g de gingembre pour \np{1270} F.

Sachant que le prix est proportionnel à la masse de gingembre :

\begin{enumerate}
\item  \textbf{Calculer} le prix pour \np{1000} g de gingembre.
\item \textbf{Compléter} le tableau des prix ci-dessous.

\begin{center}
\renewcommand{\arraystretch}{2}
\begin{tabularx}{0.8\linewidth}{| >{\centering\arraybackslash}m{4cm} | *5{>{\centering\arraybackslash} X |}}
\hline
Masse de gingembre\newline (en grammes) & 100 & 500 & 900 & \np{1000} & $\ldots$\\
\hline
Prix (en F) & $\ldots$ & \np{1270} & $\ldots$ & $\ldots$ & \np{9906}\\
\hline
\end{tabularx}
\end{center}

\item \textbf{Calculer} la masse de gingembre qu'un client peut acheter pour \np{15500} F.\\
 \textbf{Arrondir} le résultat au gramme.
\end{enumerate}

\vspace{0.5cm} 
 
