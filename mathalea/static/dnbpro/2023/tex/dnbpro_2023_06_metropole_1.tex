
\vspace{0.25cm}

Cet exercice est un questionnaire à choix multiple (QCM). \\

Pour chaque question, quatre réponses sont proposées mais \textbf{une seule est exacte}.

Cocher la bonne réponse \textbf{sans la justifier}.

Une réponse juste rapporte 4 points, une réponse fausse ou absente rapporte 0 point.

\begin{enumerate}
\item Sur la figure ci-dessous, la part de la partie grisée par rapport à la surface totale est :

\medskip

\begin{tabular}[t]{*4{m{1.5cm}} c m{0.45\linewidth}}
$\square~\dfrac{1}{8}$ & $\square~\dfrac{8}{22}$ & $\square~\dfrac{8}{30}$ & $\square~\dfrac{22}{30}$&
&
\psset{unit=0.4cm}
\def\xmin{0}   \def\xmax{10} \def\ymin{0}   \def\ymax{3}
\begin{pspicture}(\xmin,\ymin)(\xmax,\ymax)
\psframe[fillstyle=solid,fillcolor=gray](2,1)(10,2)
\psgrid[subgriddiv=1, gridlabels=0, gridcolor=black]
%\psaxes[labelFontSize=\scriptstyle, ticksize=-0pt 0pt](0,0)(\xmin,\ymin)(\xmax,\ymax)
\end{pspicture}
\end{tabular}

\medskip

\item La valeur manquante dans l'égalité incomplète $\dfrac{7}{28}=\dfrac{\ldots}{100}$ s'obtient en effectuant le calcul :

\begin{center}
\begin{tabular}[t]{*4{m{2.5cm}}}
$\square~100\times 28 \div 7$ & $\square~7\times 100 \div 28$ & $\square~100\times 28 \div 7$ & $\square~7\div 100\times 28$
\end{tabular}
\end{center}

\item Le volume de cette boite de forme cubique est égal à :

%\begin{center}
\begin{tabular}[t]{*4{m{1.5cm}} m{4cm}}
$\square~1~\text{cm}^3$ & $\square~2~\text{cm}^3$ & $\square~3~\text{cm}^3$ & $\square~6~\text{cm}^3$
&
\psset{xunit=2cm,yunit=2.4cm,arrowsize=2pt 3}
\begin{pspicture}(0.8,0.3)(3,1.6)
%\psgrid
\pspolygon[fillstyle=solid,fillcolor=lightgray!30](1.6,0.6)(2.4,0.5)(2.9,0.7)(2.9,1.4)(2.1,1.5)(1.6,1.3)
\psline(2.4,0.5)(2.4,1.2)(1.6,1.3)
\psline(2.9,1.4)(2.4,1.2)
\psline[linestyle=dashed](1.6,0.6)(2.1,0.8)(2.9,0.7)
\psline[linestyle=dashed](2.1,0.8)(2.1,1.5)
%\psline{<->}(1.5,0.6)(1.5,1.3)
\uput[d]{-9}(1.95,0.55){\footnotesize 1~cm}
\end{pspicture}
\end{tabular}
%\end{center}


\item À l'issue de 10 lancers d'un dé à 12 faces, on obtient la série de résultats suivants :


\begin{tabular}{|*{10}{>{\centering\arraybackslash} m{0.5cm}|}}
\hline
4 & 8 & 10 & 5 & 3 & 8 & 1 & 8 & 7 & 6\\
\hline
\end{tabular}
\hfill 
\parbox{3cm}{\includegraphics[width=2cm]{MetroPro23_4}}

\medskip

La fréquence d'obtention de la face 8 est :

\begin{center}
\begin{tabular}[t]{*4{m{2.5cm}}}
$\square~0,12$ & $\square~0,30$ & $\square~3$ & $\square~8$
\end{tabular}
\end{center}


\item Dans le triangle rectangle ABC ci-dessous, le cosinus de l'angle $\widehat{\text{ACB}}$ est égal à

\begin{tabular}[t]{*4{m{1.5cm}} c m{4cm}}
$\square~\dfrac{\text{AB}}{\text{AC}}$ & $\square~\dfrac{\text{BC}}{\text{AC}}$ & $\square~\dfrac{\text{AC}}{\text{BC}}$ & $\square~\dfrac{\text{AC}}{\text{AB}}$
&&
\psset{unit=1cm}
\begin{pspicture}(-5,-1)(-1,3)
%\psgrid
\pspolygon[fillstyle=solid,fillcolor=lightgray!30](0,0)(-4,0)(-4,2)
\psframe[fillstyle=solid,fillcolor=lightgray](-4,0)(-3.7,0.3)
\uput[ul](-4,2){A} \uput[dl](-4,0){B} \uput[r](0,0){C}
\pswedge[fillstyle=solid,fillcolor=lightgray](0,0){0.6}{155.9}{180} 
\end{pspicture}
\end{tabular}
%\end{center}

\end{enumerate}

\vspace{0.5cm}

