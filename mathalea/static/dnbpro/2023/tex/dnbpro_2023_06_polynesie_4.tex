
\medskip

Un agriculteur souhaiterait louer un camion pour transporter tous ses matériaux agricoles. Il hésite entre deux entreprises.

\begin{list}{\textbullet}{}
\item Tarif de l'entreprise A : 450 F par kilomètre.
\item Tarif de l'entreprise B : un forfait de \np{8400} F et 250 F par kilomètre.
\end{list}

\begin{enumerate}
\item Pour l'entreprise A et l'entreprise B, \textbf{calculer} le prix qu'il devra payer pour une distance de 100 km.
\item On appelle $x$ la distance exprimée en kilomètre.

\begin{list}{\textbullet}{}
\item La fonction $f$ représente le tarif de l'entreprise A
\item La fonction $g$ représente le tarif de l'entreprise B
\item Les fonctions $f$ et $g$ sont définies, pour toute valeur de $x$ sur l'intervalle $[0\,;\,150]$, par: 

{}\hfill$f(x) = 450 x \text{ et } g(x) = 250 x + \np{8400}$\hfill~
\end{list}

Les fonctions $f$ et $g$ sont représentées graphiquement ci-dessous.

\begin{center}
\psset{xunit=0.1cm,yunit=0.6cm,arrowsize=3pt 2}
\def\xmin{0}   \def\xmax{110} \def\ymin{0}   \def\ymax{16}
\begin{pspicture}(\xmin,\ymin)(\xmax,\ymax)
\psset{yMaxValue=\ymax,yMinValue=\ymin}
\psgrid[xunit=1cm,yunit=0.6cm,subgriddiv=5, gridlabels=0, subgridcolor=lightgray, gridcolor=gray](0,0)(11,16)
\multido{\i=0+1,\I=0+2000}{16}{\uput[l](0,\i){\footnotesize \np{\I}}}
\psaxes[labelFontSize=\scriptstyle, ticksize=-0pt 0pt,Dx=10,labels=x]{->}(0,0)(\xmin,\ymin)(\xmax,\ymax)
\psplot[linewidth=1.2pt,linecolor=red]{\xmin}{\xmax}{450 x mul 2000 div}
\psplot[linewidth=1.2pt,linecolor=blue]{\xmin}{\xmax}{8400 250 x mul add 2000 div}
\rput*(12,15){Prix (en F)} \rput*(90,1){Distance (en km)}  
\rput*(45,14){\parbox{1cm}{\rule[-0.4cm]{0cm}{1cm}\centering $\ldots$}} 
\rput*(82,12){\parbox{1cm}{\rule[-0.4cm]{0cm}{1cm}\centering $\ldots$}} 
\end{pspicture}
\end{center}


\textbf{Indiquer} le nom de chaque fonction représentée dans le graphique précédent.

\item À l'aide du graphique:
\begin{enumerate}
\item \textbf{Déterminer} le tarif pour lequel le prix payé est proportionnel à la distance exprimée en kilomètre. \textbf{Justifier} la réponse.
\item \textbf{Déterminer} l'entreprise qui a le tarif le moins cher si la distance à parcourir est de 30 km. \textbf{Tracer} les traits de lecture sur le graphique.
\item \textbf{Déterminer} la distance exprimée en kilomètre pour laquelle les deux tarifs sont égaux.
\item \textbf{Indiquer} le tarif le moins cher en fonction de la distance parcourue, exprimée en kilomètre.
\end{enumerate}
\end{enumerate}

\vspace{0.5cm} 
 
