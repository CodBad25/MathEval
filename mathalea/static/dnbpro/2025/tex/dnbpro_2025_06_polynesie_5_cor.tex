
\smallskip

Un sondage a été fait sur le port de l'uniforme dans un collège.

Sur 800 élèves, 540 élèves sont d'accord pour le port de l'uniforme, 180 élèves ne sont pas d'accord et le reste des élèves n'a donné aucune réponse.

On interroge un élève du collège au hasard.

\medskip

\begin{enumerate}
\item %Calculer la probabilité que l'élève interrogé soit d'accord pour porter l'uniforme.

Sur 800 élèves 540 sont d'accord pour le port de l'uniforme, la probabilité est donc égale à $\dfrac{540}{…800} = \dfrac{54}{80} = \dfrac{27}{40} = 0,675$ (ou 67,5\,\%).
\item %La probabilité que l'élève donne un avis positif ou négatif sur le port de l'uniforme est de $0,9$. Justifier cette réponse par le calcul.
Ont répondu par oui ou non $540 + 180 = 720$ élèves sur 800.

La probabilité que l'élève donne un avis positif ou négatif sur le port de l'uniforme est donc égale à $\dfrac{720}{800} = \dfrac{72}{80} = \dfrac{8 \times 9}{8 \times 10} = \dfrac{9}{10} = 0,9$.

\item %Calculer la probabilité que l'élève ne donne pas d'avis sur le port de l'uniforme.
La probabilité que l'élève ne donne pas d'avis sur le port de l'uniforme est égale à $1 - 0,9 = 0,1$.
\end{enumerate}

