
\smallskip

Pour récolter de l'argent, une association veut acheter à un commerçant des gâteaux et les revendre par la suite pour faire des bénéfices.

Pour cela, le commerçant lui propose la facture suivante dans laquelle certaines données manquent:

\begin{tabularx}{\linewidth}{|c|c|*{3}{>{\centering \arraybackslash}X|}}\hline
	&A				&B						&C			&D\\ \hline
1	&Gâteau			&Prix à l'unité (en F)	&Quantité	&Montant total (en F)\\ \hline
2	&Au beurre		&						&35			&\\ \hline
3	&À la banane	&900 					&22			&\np{19800}\\ \hline
4	&À la vanille	& \np{1100}				&15			&\np{16500}\\ \hline
5	&Au chocolat	&\np{1200}				& 28		&\\ \hline
6	&				&						&			&\\ \hline
7	&				&Montant total HT (hors taxe)		&\np{104900}&\\ \hline
8	&				&Montant de la TVA (5\,\%)&			&\\ \hline 
9	&				&Montant total TTC		&			&\\ \hline
\end{tabularx}

\begin{enumerate}
\item Justifier par un calcul le montant total pour les gâteaux à la banane.
\item Recopier sur la copie la formule à insérer dans la cellule D5, parmi les trois
propositions suivantes :

\begin{center}$\fbox{=B5*C5}$\qquad  $\fbox{=B5+C5}\qquad $\fbox{=C5+1 000}\end{center}
\item Finalisation de la facture correspondant à la commande 
	\begin{enumerate}
		\item \textbf{Compléter} le tableau ci-dessous.
		
		\begin{tabularx}{\linewidth}{|c|c|*{3}{>{\centering \arraybackslash}X|}}\hline
	&A				&B						&C			&D\\ \hline
1	&Gâteau			&Prix à l'unité (en F)	&Quantité	&Montant total (en F)\\ \hline
2	&Au beurre		&						&35			&\\ \hline
3	&À la banane	&900 					&22			&\np{19800}\\ \hline
4	&À la vanille	& \np{1100}				&15			&\np{16500}\\ \hline
5	&Au chocolat	&\np{1200}				& 28		&\\ \hline
6	&				&						&			&\\ \hline
7	&				&\multicolumn{2}{|c|}{Montant total HT (hors taxe)}&\np{104900}\\ \hline
8	&				&\multicolumn{2}{|c|}{Montant de la TVA (5\,\%)}&\\ \hline 
9	&				&\multicolumn{2}{|c|}{Montant total TTC}&\\ \hline
\end{tabularx}
		\item \textbf{Détailler} le calcul du montant de la TVA sur la copie.
	\end{enumerate}
\item Calculer la quantité totale de gâteaux achetés au commerçant.
\end{enumerate}

Pour la revente des gâteaux, l'association fixe le prix à \np{1 400} F l'unité quel que soit le 
gâteau.

\begin{enumerate}[resume]
\item En supposant que tous les gâteaux seront vendus, calculer le montant total de la revente. \textbf{Exprimer} le résultat en F.
\item \textbf{Calculer} le bénéfice réalisé par l'association. Exprimer le résultat en F.

\textbf{Donnée} : bénéfice = montant total de la revente - montant total TTC de la facture du commerçant.
\item L'association souhaite faire un bénéfice de \np{30000} F{}.

\textbf{Indiquer} si l'objectif est atteint. \textbf{Justifier} la réponse.
\end{enumerate}

\medskip

