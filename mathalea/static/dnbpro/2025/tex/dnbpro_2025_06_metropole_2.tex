
\medskip

Dans le cadre de son travail, Elsa doit se déplacer régulièrement à Paris. Elle voyage en train.

Le billet coûte $80$~\euro.

Elsa peut acheter une carte à l'année qui coûte $49$~\euro{} et qui lui permet d'obtenir toute l'année
une réduction de 30\,\% sur les billets.

\medskip

\begin{enumerate}
\item Étude des tarifs :
	\begin{enumerate}
		\item Calculer le prix qu'Elsa paiera pour 3 billets sans carte de réduction.
		\item Justifier que le prix d'un billet de train après une remise de 30\,\% est $56$~\euro.
		\item Calculer le prix total payé par Elsa pour trois billets avec la carte, achat de la carte compris.
	\end{enumerate}
\item Comparaison des tarifs :

Elsa achète $x$ billets.

On nomme :

\begin{itemize}[label=$\bullet~$]
\item $f$ la fonction qui associe à $x$ le montant total que paie Elsa dans le cas où elle n'achète pas la carte de réduction.
\item $g$ la fonction qui associe à $x$ le montant total que paie Elsa dans le cas où elle achète la carte de réduction et en tenant compte de l'achat de la carte.
\end{itemize}

Dans le repère ci-dessous sont représentées les fonctions $f$ et $g$.


\begin{center}
\psset{xunit=1cm,yunit=0.01cm,arrowsize=2pt 3}
\begin{pspicture}(-1,-50)(13,1000)
\psaxes[linewidth=1.25pt,Dy=100,labelFontSize=\scriptstyle]{->}(0,0)(0,0)(13,1000)
\multido{\n=0+1}{14}{\psline[linewidth=0.25pt](\n,0)(\n,1000)}
\multido{\n=0+100}{11}{\psline[linewidth=0.25pt](0,\n)(13,\n)}
\uput[u](10.8,0){Nombre de billets achetés}\uput[r](0,950){Prix à payer (en \euro)}
\uput[r](13,0){$x$}\uput[u](0,1000){$y$}
\psplot[plotpoints=600,linewidth=1.25pt,linecolor=red]{0}{12}{80 x mul}\rput{38}(10,833){\red Fonction : \ldots}
\psplot[plotpoints=600,linewidth=1.25pt,linecolor=blue]{0}{12}{56 x mul 49 add}\rput{30}(10,635){\blue Fonction : \ldots}
\end{pspicture}
\end{center}
		

	\begin{enumerate}
		\item Noter pour chacune des deux droites le nom de la fonction représentée par cette droite sur le graphique ci-dessus.
		
		
		\item Choisir et recopier sur la copie l'expression algébrique de la fonction $g$ :
		
Choix 1 : $g(x) =56x+49$ 

Choix 2: $g(x) =56x$ 

Choix 3: $g(x) = 80x$ 

Choix 4 : $g(x) = 49x +56$
		\item Calculer $g(8)$.
		\item Indiquer le prix à payer pour $8$ billets avec la carte de réduction.
		\item Sachant qu'Elsa achètera plus de $8$ billets dans l'année, déterminer le tarif le plus
avantageux pour elle.

Justifier la réponse en expliquant la méthode utilisée.
	\end{enumerate}
\end{enumerate}


