
\medskip

Cet exercice est un questionnaire à choix multiples (QCM).

Pour chaque question, quatre réponses sont proposées mais \textbf{une seule est exacte}. Cocher la bonne réponse \textbf{sans justification}.

Une réponse correcte apporte 4 points, une réponse fausse ou l'absence de réponse ne rapporte aucun point.

\medskip

\begin{enumerate}
\item $25\,\%$ de $340$ s'obtient en effectuant le calcul suivant :

\begin{tabularx}{\linewidth}{*{4}{>{{$\square$~~}}X}}
$340 \times \dfrac{25}{100}$&$340 + \dfrac{25}{100}$&$340 \times \dfrac{100}{25}$&$340 + \dfrac{100}{25}$\\
\end{tabularx}
\item $3^5$ est égal à :

\begin{tabularx}{\linewidth}{*{4}{>{{$\square$}~~}X}}
$3+3+3+3+3$	&$3\times 5$ & $3 - 5$ & $3 \times 3\times 3\times 3\times 3$\\
\end{tabularx}

\item Le tableau suivant correspond à une situation de proportionnalité : \begin{tabular}{c |c}
19&2\\ \hline
$N$&6\\
\end{tabular}

\begin{tabularx}{\linewidth}{*{4}{>{{$\square$~~}}X}}
$N = 12,5$& $N = 23$& $N = 3,5$&$N = 57$
\end{tabularx}

\item Le volume du cube de côté 3 cm est égal à :

\hspace{8cm} \psset{unit=1cm}
\begin{pspicture}(2.3,2.5)
\psframe(0,0.4)(1.6,2)
\psline(1.6,0.4)(2.1,0.9)(2.1,2.5)(1.6,2)
\psline(2.1,2.5)(0.5,2.5)(0,2)
\psline[linestyle=dashed](0,0.4)(0.5,0.9)(2.1,0.9)
\psline[linestyle=dashed](0.5,0.9)(0.5,2.5)
\uput[d](0.8,0.4){3 cm}
\end{pspicture}

\begin{tabularx}{\linewidth}{*{4}{>{{$\square$~~}}X}}
9~cm$^3$& 18~cm$^3$ & 27~cm$^3$&81~cm$^3$
\end{tabularx}

\item La solution de l'équation $25x + 4 = 108 - x$ est :

\begin{tabularx}{\linewidth}{*{4}{>{{$\square$~~}}X}}
$x = 3$ &$x = 4$&$x = 5$&$x = 6$
\end{tabularx}

\end{enumerate}

\bigskip

