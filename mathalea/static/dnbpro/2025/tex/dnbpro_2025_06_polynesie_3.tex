
\smallskip

Heimana a conçu un programme constitué d'un script principal et d'un bloc motif présenté ci-dessous.

%\begin{tabular}[t]{|m{7cm}|m{7cm}|}\hline
\begin{tblr}{vlines,hlines,
  colspec={Q[7cm,c] Q[7cm,c]},
  rowspec={Q Q}}
\textbf{Bloc Motif}&\textbf{Script principal}\\
\begin{scratch}[scale=0.8]
\initmoreblocks{définir \namemoreblocks{Motif}}
\blockmove{s'orienter à ~\ovalnum{0}}
\blockmove{avancer de \ovalnum{120} pas}
\blockmove{s'orienter à ~\ovalnum{90}}
\blockmove{avancer de \ovalnum{20} pas}
\blockmove{s'orienter à ~\ovalnum{180}}
\blockmove{avancer de \ovalnum{40} pas}
\blockmove{s'orienter à ~\ovalnum{90}}
\blockrepeat{répéter \ovalnum{2} fois}
{\blockmove{avancer de \ovalnum{30} pas}
\blockmove{s'orienter à ~\ovalnum{180}}
\blockmove{avancer de \ovalnum{20} pas}
\blockmove{s'orienter à ~\ovalnum{-90}}
\blockmove{avancer de \ovalnum{30} pas}
\blockmove{s'orienter à ~\ovalnum{180}}
\blockmove{avancer de \ovalnum{20} pas}
\blockmove{s'orienter à ~\ovalnum{90}}
}
\end{scratch}&
\begin{scratch}[scale=0.8]
\blockinit{quand \greenflag est cliqué}
\blockmove{aller à x: \ovalnum{0} y: \ovalnum{0}}
\blockpen{effacer tout}
\blockpen{stylo en position d'écriture}
\blockmoreblocks{Motif}
\end{scratch}

\vspace{2cm}
Petits rappels

\begin{scratch}[scale=0.8]\blockmove{s'orienter à ~\ovalnum{90}}\end{scratch} s'orienter vers la droite 

\begin{scratch}[scale=0.8]\blockmove{s'orienter à ~\ovalnum{0}}\end{scratch} s'orienter vers le haut 

\begin{scratch}[scale=0.8]\blockmove{s'orienter à ~\ovalnum{-90}}\end{scratch} s'orienter vers la gauche 

\begin{scratch}[scale=0.8]\blockmove{s'orienter à ~\ovalnum{-180}}\end{scratch} s'orienter vers le bas\\ 
%\end{tabular}
\end{tblr}

\begin{enumerate}
\item \textbf{Tracer} sur le quadrillage ci-dessous la figure 1 correspondant au programme de Heimana.

\psset{unit=0.6cm,arrowsize=1.5pt 2}
\begin{pspicture}(10,15)
\psgrid[gridlabels=0pt,subgriddiv=1,gridwidth=0.4pt]
\psdot[dotstyle=+,dotangle=45,dotscale=1.8](2,2)\uput[d](2,2){Départ}
\psline[linewidth=1.5pt]{<->}(8,1)(9,1)\uput[d](8.5,1){\small 10 pas}
\end{pspicture}

\item Le camarade de Heimana a la figure 2 ci-dessous en tête mais n'arrive pas à faire le programme.

\begin{center}
\psset{unit=1.8cm}
\begin{pspicture}(6,1)
\def\Motif{\psline(0,0)(0,1)(1,1)(1,0)(2,0)}
\multido{\n=0+2}{3}{\rput(\n,0){\Motif}}
\end{pspicture}
\end{center}

\textbf{Compléter} le programme ci-dessous pour obtenir la figure 2 ci-dessus sachant que chaque segment fait 40 pas.

\begin{scratch}[scale=0.9]
\blockinit{quand \greenflag est cliqué}
\blockmove{aller à x: \ovalnum{0} y: \ovalnum{0}}
\blockpen{stylo en position d'écriture}
\blockrepeat{répéter \ovalnum{} fois}{
\blockmove{s'orienter à ~\ovalnum{0}}
\blockmove{avancer de \ovalnum{40} pas}
\blockmove{s'orienter à ~\ovalnum{90}}
\blockmove{avancer de \ovalnum{} pas}
\blockmove{s'orienter à ~\ovalnum{180}}
\blockmove{avancer de \ovalnum{40} pas}
\blockmove{s'orienter à ~\ovalnum{}}
\blockmove{avancer de \ovalnum{40} pas}
}
\end{scratch}

\item \textbf{Calculer} la longueur totale de la figure 2 sachant que 1 cm vaut $10$ pas.

\textbf{Exprimer} le résultat en cm.
\end{enumerate}

\medskip

