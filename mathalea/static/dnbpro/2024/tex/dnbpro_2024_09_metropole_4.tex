
\medskip

Gabin installe une cuve de récupération d'eau pour arroser son potager. Cette cuve est
représentée sur la figure 1 ci-dessous par le pavé droit ABCDIJGH. \\
La figure 2 représente une vue de côté de l'installation.\\
La cuve est protégée par le toit rectangulaire incliné FKLE.

%fig
%\begin{center}
\psset{unit=1cm,arrowsize=2pt 3}
\begin{pspicture}(0,0)(15,9.8)
%\psgrid[subgriddiv=5,  gridlabels=0, gridcolor=gray, subgridcolor=lightgray] 
\pspolygon(0.3,1.4)(4.3,1.4)(0.3,7.7)%DEF
\psline(4.3,1.4)(8,2.9)(4,9.2)(0.3,7.7)%ELKF
\psline(1.9,1.4)(1.9,5.2)(0.3,5.2)%CBA
\psline{<->}(4.7,1.2)(8.1,2.6)\rput{20}(7.,1.7){4,0 m}
\psline[linestyle=dashed](0.3,1.4)(4,2.9)(8,2.9)%DIL
\psline[linestyle=dashed](1.9,1.4)(5.6,2.9)(5.6,6.7)(1.9,5.2)%CHGB
\psline[linestyle=dashed](0.3,5.2)(4,6.7)(5.6,6.7)%AJG
\psline[linestyle=dashed](4,2.9)(4,9.2)%IK
\psframe(0.3,5.2)(0.55,5.45)\psframe(0.3,1.4)(0.55,1.65)\psframe(1.9,1.4)(2.15,1.65)
\rput(2.3,0.8){Figure 1}
\uput[l](0.3,5.2){A} \uput[dl](1.9,5.2){B} \uput[ul](1.9,1.4){C} \uput[dl](0.3,1.4){D}
\uput[dr](4.3,1.4){E} \uput[ul](0.5,7.7){F} \uput[ur](5.6,6.7){G} \uput[ur](5.9,2.9){H}
\uput[ul](4,2.9){I} \uput[ul](4,6.7){J} \uput[u](4,9.2){K} \uput[ur](8,2.9){L}
%% gouttière
{\psset{linewidth=0.7pt}
\psplot{0.3}{4.4}{ 15 x mul 37 div 1402 185 div add }
\psarc(0.5,7.7){0.2}{-180}{0} \psarc(4.6,9.36){0.2}{-180}{0}
\psline(0.7,7.7)(4.8,9.36) \psline(0.5,7.5)(4.6,9.16)
\psline(0.3,7.7)(0.7,7.7) \psline(4.4,9.36)(4.8,9.36)
\psline(4.5,9.2)(4.5,6.4)\psline(4.7,9.2)(4.7,6.4)
\psarc(4.6,6.45){0.1}{-145}{-35}}
\psline{->}(6,8.8)(5,9.2) \uput[r](6,8.8){Gouttière}
%%%%%%%%%%%
\pspolygon(10,2.1)(14,2.1)(10,8.3)%DEF
\psline(11.6,2.1)(11.6,5.8)(10,5.8)%CBA
\psline{<->}(10,1.9)(14,1.9)\uput[d](12,1.9){2,5 m}
\psline{<->}(14.3,2.3)(10.4,8.5)\rput(12.7,5.7){4,6 m}
\psline{<->}(9.5,5.8)(9.5,8.4)\rput{90}(9.3,7.1){1,6 m}
\psframe(10,5.8)(10.25,6.05)\psframe(11.6,2.1)(11.85,2.35)\psframe(10,2.1)(10.25,2.35)
\uput[l](10,5.8){A} \uput[dl](11.6,5.8){B} \uput[ul](11.6,2.1){C} \uput[dl](10,2.1){D}
\uput[d](14,2.1){E} \uput[u](10,8.3){F}
\uput[u](7.4,0){Les figures 1 et 2 ne sont pas à l'échelle}
\rput(11.8,0.8){Figure 2}
\uput{2pt}[u](13,9){Les points D, A et F} \uput{2pt}[d](13,9){sont alignés}
\uput{2pt}[u](13,8){Les points E, B et F} \uput{2pt}[d](13,8){sont alignés}
\end{pspicture}
%\end{center}

\begin{enumerate}
\item Indiquer sur la copie la nature géométrique du solide EDFKLI en choisissant parmi les
noms suivants :

$\bullet~$ cube\hfill $\bullet~$ triangle\hfill $\bullet~$ prisme droit \hfill $\bullet~$ cylindre\hfill~

\item On considère le triangle EDF rectangle en D représenté sur la figure 2.

En utilisant le théorème de Pythagore, vérifier que la longueur DF arrondie au dixième est $3,9$~m.
\item Calculer, en mètre, la longueur AD.
\item Les droites (AB) et (DC) sont parallèles.

Montrer, en utilisant le théorème de Thalès, que la longueur AB arrondie à l'unité est égale à 1~m.
\end{enumerate}
\begin{minipage}{0.65\linewidth}
\begin{enumerate}[resume]
\item Calculer, en mètre cube $\left(\text{m}^3\right)$, le volume du solide ABCDIJGH.

\textbf{Indication :} Volume d'un parallélépipède rectangle : $V = L \times l \times h$.
\item En déduire le volume, en litre, du récupérateur d'eau.

\textbf{Indication :} 1 m$^3 = \np{1000}$~L
\end{enumerate}
\end{minipage}\hfill
\begin{minipage}{0.25\linewidth}
\begin{center}
\psset{unit=0.85cm,arrowsize=2pt 3}
\begin{pspicture}(4.5,4.5)
%\psgrid
\psframe(0.4,0)(1.9,2.7)
\psline(1.9,0)(4.2,1.3)(4.2,4)(1.9,2.7)
\psline(4.2,4)(2.5,4)(0.4,2.7)
\psline[linestyle=dashed](0.4,0)( 2.5,1.3)(2.5,4)
\psline[linestyle=dashed]( 2.5,1.3)( 4.2,1.3)
\psline{<->}(0.25,0)(0.25,2.7)\uput[l](0.25,1.35){$h$}
\psline{<->}(0.25,2.75)(2.45,4.1)\uput[ul](1.35,3.55){$L$}
\psline{<->}(2.5,4.15)(4.2,4.15)\uput[u](3.35,4.15){$l$}
\end{pspicture}
\end{center}
\end{minipage}

\vspace{0.5cm}

