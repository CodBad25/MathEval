
\medskip

Comme chaque dimanche, Maui se rend au marché de Papeete pour faire quelques achats.

\begin{tabular}{@{} l !{\textbullet} l}
Il achète& une pièce de « Pua'a roti » à \np{1760} F le morceau,\\
& deux paquets de « Firi-firi » à 500 F le paquet,\\
& deux poissons perroquet à \np{1200} F l'unité,\\
& un paquet de « Taro » à 800 F le Paquet,\\
& un tas de « Fe'i » à 400 F le tas,\\
& une bouteille de « Miti haari » à 500 F la bouteille.
\end{tabular}

La facture incomplète des achats de Maui au marché de Papeete est réalisée sur un tableur.

\begin{enumerate}
\item  On complète la facture.

\begin{center}
\renewcommand{\arraystretch}{1.3}
\begin{tabular}{| >{\cellcolor{lightgray}} c | *4{c|}}
\hline
\rowcolor{lightgray} & A & B & C & D\\
\hline
1 & \textbf{Aliment} & \textbf{Quantité} & \textbf{Prix unitaire en F} & \textbf{Prix en F}\\
\hline
2 & Pièce de Pua'a roti & 1 & $\blue \np{1760}$ & $\blue \np{1760}$ \\
\hline
3 & Paquet de Fri fri & $\blue 2$ & 500 & $\blue \np{1000}$ \\
\hline
4 & Poisson perroquet &  $\blue 2$ & \np{1200} & $\blue \np{2400}$ \\
\hline
5 & Paquet de Taro &  $\blue 1$ & \np{800} & $\blue \np{800}$ \\
\hline
6 & Tas de Fe'i & 1 & $\blue \np{400}$ & $\blue \np{400}$ \\
\hline
7 & Bouteille de \og Miti haari \fg{} & 1 & $\blue \np{500}$ & $\blue \np{500}$  \\
\hline
8 & \multicolumn{2}{c|}{~~} &  \textbf{PRIX TOTAL en F} & $\blue \np{6860}$ \\
\hline
\end{tabular}
\end{center}

\item La formule à insérer dans la cellule D3, parmi celles proposées ci-dessous, est celle du milieu.

\begin{center}
{\ttfamily\Large
\begin{tabular}{| p{3cm} | c | p{3cm} | c | p{3cm} |}
\cline{1-1} \cline{3-3} \cline{5-5} 
= 1 * 500 & & \textcolor{blue}{= B3 * C3} & & = B3 + C3\\
\cline{1-1} \cline{3-3} \cline{5-5}
\end{tabular} }
\end{center}
\end{enumerate}

On admet que le montant total de la facture s'élève à \np{6860} F{}.\\
Une remise de 15\,\% est accordée à Maui.

\begin{enumerate}[resume]
\item  %On calcule le montant de la remise. Exprimer le résultat en F{}.
$\np{6860}\times \dfrac{15}{100}= \np{1029}$

La remise est de \np[F]{1029}.

\item% Calculer le prix payé par Maui. Exprimer le résultat en F{}.
$\np{6860} - \np{1029} = \np{5831}$

Le prix payé par Maui est de \np[F]{5831}.

\end{enumerate}

\bigskip

