
\smallskip

Lors d'une compétition de surf, quand une compétitrice surfe une vague, cinq juges attribuent une note entre 0 et 10.


\begin{center}
\fbox{\parbox{0.8\linewidth}{
\begin{list}{$\bullet$}{Détermination du score pour chaque vague:}
\item La plus grande note et la plus petite note sont éliminées.
\item Le score de la vague surfée est la moyenne des trois notes restantes arrondie au
dixième.
\end{list}}}
\end{center}

Lors de la compétition Tahiti Pro à Teahupo'o, une surfeuse a obtenu les scores suivants en finale pour la 4\up{e} vague surfée :

\begin{center}
\begin{tabular}{|c|c|}
\hline
Numéro du juge&Notes  sur 10\\ \hline
1 &6,7\\ \hline
2 &5,4\\ \hline
3 &7,5\\ \hline
4 &8,2\\ \hline
5 &7,7\\ \hline
\end{tabular}
\end{center}

\begin{enumerate}
\item %Expliquer, à l'aide d'un calcul, pourquoi le score obtenu par cette surfeuse est 7,3 pour la 4\up{e} vague.
En retirant la plus grande note $8,3$ et la plus petite $5,4$, il reste $6,7$, $7,5$ et $7,7$ dont la moyenne est:
$\dfrac{6,7+7,5+7,7}{3} = \dfrac{21,9}{3}=7,3$.

Donc  le score obtenu par cette surfeuse pour la 4\up{e} vague est $7,3$.

\item Pour la suite de la compétition, les juges calculent les scores des compétitrices pour toutes les vagues surfées.

Deux surfeuses ont obtenu les scores suivants en finale à la compétition de Teahupo'o :

\begin{center}
\begin{tabularx}{0.8\linewidth}{|m{2.5cm}|*{5}{>{\centering \arraybackslash}X|}}\hline
Épreuve					&Vague \no 1& Vague \no 2	&Vague \no 3&Vague \no 4&Vague \no 5\\ \hline
Score de la\newline surfeuse 1	&6,8		& 8,5			&8,8		& 6,7		& 7,4\\ \hline
Score de la\newline surfeuse 2	& 1,9		& 4,8			& 0,2		& 7,3		& 7,3\\ \hline
\end{tabularx}
\end{center}

\begin{list}{$\bullet$}{\textbf{Détermination du résultat de fin de session par surfeuse:}}
\item Le résultat d'une surfeuse est la somme des deux meilleurs scores. 
\item Le plus grand résultat désigne la gagnante.
\end{list}

	\begin{enumerate}
		\item% \textbf{Calculer} le résultat de la surfeuse 1. \textbf{Écrire} le calcul sur la copie.
Les deux meilleures scores de  la surfeuse 1 sont $8,5$ et $8,8$. 

Or $8,5+8,8=17,3$; donc le résultat de la surfeuse 1 est $17,3$.
		
		\item La surfeuse 2 a obtenu le résultat de $14,6$.
		
$17,3 > 14,6$ donc c'est  la surfeuse 1 qui a remporté la finale. 

%\textbf{Justifier} la réponse.
	\end{enumerate}
\end{enumerate}


