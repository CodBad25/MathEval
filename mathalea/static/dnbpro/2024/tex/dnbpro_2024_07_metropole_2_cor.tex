
\medskip

\begin{enumerate}
\item Les épreuves de natation des Jeux Olympiques ont lieu dans des piscines olympiques.


 \begin{minipage}[t]{0.5\linewidth}
\vspace{0pt} 
\begin{list}{\textbullet}{ La plupart des piscines olympiques sont des pavés droits avec les caractéristiques suivantes:}
\item longueur $L$ : 50 m;
\item largeur $\ell$: 25 m;
\item hauteur d'eau $h$: 3 m.
\end{list}
\end{minipage}
\hfill
 \begin{minipage}[t]{8cm}
\vspace{0pt}
\scalebox{0.6}{
\psset{xunit=1cm,yunit=0.4cm,radius=0pt,arrowsize=3pt 2}
\def\xmin {-1}   \def\xmax {9} \def\ymin {-2}   \def\ymax {11}
\begin{pspicture}(\xmin,\ymin)(\xmax,\ymax)
%\psgrid[subgriddiv=1, gridlabels=0, gridcolor=lightgray] 
\Cnode*(0,0){A}  \Cnode*(2,2){D}\Cnode*(6,0){B} \Cnode*(8,2){C} 
\Cnode*(6,6){F}    \Cnode*(8,8){G} \Cnode*(0,6){E} \Cnode*(2,8){H} 
\pspolygon[linecolor=white,fillstyle=solid,fillcolor=lightgray!30](A)(B)(C)(G)(H)(E)
\psline(G)(C)(B)(A)(E)(F)(G)(H)(E) \psline(B)(F)
{\psset{linestyle=dashed}
\psline(A)(D)(C) \psline(D)(H)}
\psline{<->}(0,-1)(6,-1)\uput[d](3,-1){$L$} 
\psline{<->}(-0.3,0)(-0.3,6) \uput[l](-0.3,3){$h$} 
\psline{<->}(7,0)(9,2) \uput[dr](8,1){$\ell$}
\end{pspicture}
}% fin du scalebox
\end{minipage}

\begin{enumerate}
\item Le volume d'eau contenu dans une piscine olympique est, en m$^3$, de:\\
$L \times \ell \times h = 50\times 25 \times 3 = \np{3750}$
soit $\np{3750000}$ litres.

%Donner la réponse en mètre cube (m$^3$), puis en litres. (\textbf{Rappel} : 1 m$^3$ = \np{1000} L) 

\item 
\begin{list}{\textbullet}{Les piscines municipales les plus courantes ont les dimensions suivantes:}
\item longueur $L$ : 25 m ; 
\item largeur $\ell$ : 12,5 m ;
\item hauteur d'eau $h$ : 3 m.
\end{list}

%Lucas affirme que ce type de piscine contient 4 fois moins d'eau qu'une piscine olympique. 
%Indiquer si cette affirmation est vraie. Justifier la réponse.

Le volume de la piscine municipale est, en m$^3$, de: $25\times 12,5 \times 3 = 937,5 = \dfrac{\np{3750}}{4}$.

L'affirmation de Lucas est vraie.

\end{enumerate}
\end{enumerate}

La nage papillon est la plus spectaculaire.
C'est aussi la deuxième plus rapide après le crawl.

Aux JO de Tokyo en 2021, la canadienne Margaret MacNeil a remporté l’épreuve du 100 m papillon en 56 secondes.

\begin{enumerate}[start=2]
\item La vitesse moyenne de Margaret MacNeil sur cette épreuve est, en m/s, de;
$\dfrac{100}{56}$ soit environ $1,79$.
%Arrondir le résultat au centième.

%\textbf{Rappel} : formule de la vitesse $v$ (en m/s) en fonction de la distance parcourue $d$ en mètre (m) et du temps de parcours $t$ en seconde (s) : $v=\frac{d}{t}$.

\item On a : 1 m/s = $3,6$ km/h. \\
La vitesse de Margaret MacNeil en km/h est donc de:
$1,79\times 3,6 \approx 6,44$.

%Arrondir au centième.

\item L'australienne Emma MacKeon, médaille d'or en nage libre (crawl) a parcouru
100 m à la vitesse de $1,92$ m/s.

\begin{enumerate}
\item  Le temps mis par Emma MacKeon sur cette épreuve est, en seconde, le nombre $t$ tel que: $1,96=\dfrac{100}{t}$.
Donc: $t=\dfrac{100}{1,96}\approx 51,02$.

%Arrondir au centième.

\item On dit qu'une personne qui marche vite, à 7 km/h, est plus rapide sur 100 m qu'une personne nageant le crawl.

La vitesse d'Emma MacKeon est de $1,92$ m/s soit en km/h:
$1,92\times 3,6 = 6,912<7$.

L'affirmation est donc vraie.

%Justifier la réponse.
\end{enumerate}
\end{enumerate}
 
\bigskip

