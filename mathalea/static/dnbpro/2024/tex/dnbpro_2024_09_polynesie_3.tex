
\smallskip

Lors d'une compétition de surf, quand une compétitrice surfe une vague, cinq juges attribuent une note entre 0 et 10.


\begin{center}
\fbox{\parbox{0.8\linewidth}{
\begin{list}{$\bullet$}{Détermination du score pour chaque vague:}
\item La plus grande note et la plus petite note sont éliminées.
\item Le score de la vague surfée est la moyenne des trois notes restantes arrondie au
dixième.
\end{list}}}
\end{center}

Lors de la compétition Tahiti Pro à Teahupo'o, une surfeuse a obtenu les scores suivants en finale pour la 4\up{e} vague surfée :

\begin{center}
\begin{tabular}{|c|c|}
\hline
Numéro du juge&Notes  sur 10\\ \hline
1 &6,7\\ \hline
2 &5,4\\ \hline
3 &7,5\\ \hline
4 &8,2\\ \hline
5 &7,7\\ \hline
\end{tabular}
\end{center}

\begin{enumerate}
\item Expliquer, à l'aide d'un calcul, pourquoi le score obtenu par cette surfeuse est 7,3 pour la 4\up{e} vague.
\item Pour la suite de la compétition, les juges calculent les scores des compétitrices pour toutes les vagues surfées.

Deux surfeuses ont obtenu les scores suivants en finale à la compétition de Teahupo'o :

\begin{center}
\begin{tabularx}{\linewidth}{|m{3cm}|*{5}{>{\centering \arraybackslash}X|}}\hline
Épreuve					&Vague \no 1& Vague \no 2	&Vague \no 3&Vague \no 4&Vague \no 5\\ \hline
Score de la\newline surfeuse 1	&6,8		& 8,5			&8,8		& 6,7		& 7,4\\ \hline
Score de la\newline surfeuse 2	& 1,9		& 4,8			& 0,2		& 7,3		& 7,3\\ \hline
\end{tabularx}
\end{center}

\textbf{Détermination du résultat de fin de session par surfeuse :}

\begin{itemize}[label=$\bullet~$]
\item Le résultat d'une surfeuse est la somme des deux meilleurs scores. 
\item Le plus grand résultat désigne la gagnante.
\end{itemize}

	\begin{enumerate}
		\item \textbf{Calculer} le résultat de la surfeuse 1. \textbf{Écrire} le calcul sur la copie.
		\item La surfeuse 2 a obtenu le résultat de $14,6$.
		
\textbf{Indiquer} qui de la surfeuse 1 ou de la surfeuse 2 a remporté la finale. \textbf{Justifier} la réponse.
	\end{enumerate}
\end{enumerate}

\bigskip

