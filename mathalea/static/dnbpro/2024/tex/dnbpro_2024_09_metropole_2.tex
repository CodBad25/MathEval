
\medskip

Gabin souhaite réduire son impact sur l'environnement. Il a réalisé auprès d'un organisme spécialisé une estimation de la quantité de dioxyde de carbone (CO$_2$) qu'il émet en une année.

Les résultats sont donnés dans le tableau ci-dessous.

\begin{center}
\begin{tabularx}{0.8\linewidth}{|X|c|} \hline
\textbf{Domaine}&\multicolumn{1}{|m{5cm}|}{\textbf{Masse de dioxyde de carbone}\newline\hspace*{2cm} \textbf{(en tonne)}}\\ \hline
Transport &3,6\\ \hline
Logement& 2,2 \\ \hline
Vie quotidienne &1,4\\ \hline
Alimentation&\ldots\\ \hline
Émissions indirectes &2,9\\ \hline
\textbf{Total}&12,1\\ \hline
\end{tabularx}
\end{center}

\medskip

\begin{enumerate}
\item Indiquer à l'aide d'une phrase, la masse totale, en tonne, de dioxyde de carbone (CO$_2$) émis par Gabin en une année.
\item Calculer la masse de CO$_2$ du domaine Alimentation.
\item Calculer le pourcentage de CO$_2$ correspondant à l'alimentation par rapport au total des émissions. Arrondir le résultat au dixième.
\end{enumerate}

L'objectif de Gabin est d'émettre moins de 10 tonnes de CO$_2$ par an. Pour atteindre cet objectif, il effectue des travaux d'isolation et change son mode de chauffage. Ses émissions de CO$_2$ dues au logement diminuent de 50\,\%.

\begin{enumerate}[resume]
\item Calculer la masse de CO$_2$ émis par an pour le domaine Logement après les travaux réalisés et le changement de mode de chauffage.
\item Calculer alors la nouvelle masse totale de CO$_2$ émis par Gabin en une année. 
\item Indiquer si l'objectif de Gabin est atteint. Justifier la réponse.
\end{enumerate}

\vspace{0.5cm}

