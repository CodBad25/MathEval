
\medskip

Dans la commune de Gabin, le tarif de ramassage des bacs à ordures ménagères est composé:\\
\hspace*{2cm}\textbullet~~d'une partie fixe de 115~\euro{} par an,\\
\hspace*{2cm}\textbullet~~d'une partie variable de 5~\euro{} par ramassage.

Le tableau ci-dessous, donne des coûts à l'année en fonction du nombre de ramassages.


{\renewcommand{\arraystretch}{1.5}
\begin{tabularx}{\linewidth}{|m{5cm}|*{4}{>{\centering \arraybackslash}X|}}\hline
Nombre de ramassages (x)&1&10&14&20\\ \hline
Coût à l'année en euros (\euro)&135 &165&\ldots&215\\ \hline
Point de coordonnées $(x~;~y)$&A(4~;~135)&B(10~;~165)&C(14~;~\ldots)&D(20~;~215)\\ \hline
\end{tabularx}}

\medskip

\begin{enumerate}
\item Vérifier par un calcul que Gabin paie 185~\euro{} pour $14$~ramassages dans l'année.

Compléter les éléments vides du tableau précédent.
\item Compléter le graphique suivant en plaçant les points C et D et tracer la droite passant par les points A, B, C et D.

\begin{center}
\psset{xunit=0.5cm,yunit=0.1cm,arrowsize=2pt 3}
\begin{pspicture}(-1,-5)(25,140)
\uput[u](22,0){\footnotesize Nombre de ramassages ($x$)}
\uput[r](0,137){\footnotesize Coût payé (\euro) ($y$)}
\psaxes[linewidth=1.25pt,labelFontSize=\scriptstyle,Oy=100,Dy=5]{->}(0,0)(0,0)(25,140)
\multido{\n=1+1}{25}{\psline[linewidth=0.15pt](\n,0)(\n,140)}
\multido{\n=0+5}{29}{\psline[linewidth=0.15pt](0,\n)(25,\n)}
\psdots[dotstyle=+,dotscale=1.5](4,35)(10,65)
\uput[ur](4,35){A}\uput[ur](10,65){B}
\end{pspicture}
\end{center}
\item Le coût à payer en euros en fonction du nombre $x$ de ramassages dans l'année peut être modélisé par la fonction $f$ d'expression $f(x) = 5x + 115$.

Indiquer si cette fonction est une fonction linéaire. Justifier la réponse.
\item Gabin ne souhaite pas dépasser 195~\euro{} cette année.\\
Déterminer le nombre maximum de ramassages correspondant à cet objectif. \\
Justifier la réponse.

\smallskip

\emph{Indication} : la résolution peut se faire par le calcul ou à l'aide de la représentation graphique que vous avez réalisée, en laissant apparents les traits de lecture.
\end{enumerate}

\vspace{0.5cm}

