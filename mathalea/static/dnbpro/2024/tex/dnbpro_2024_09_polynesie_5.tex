
\smallskip

\begin{minipage}{.65\linewidth}
Mathis veut personnaliser son surfboard avec un motif géométrique.

\medskip

\begin{enumerate}
\item \textbf{Indiquer}, sur la copie, la valeur qu'il faut mettre sur les pointillés du programme ci-contre pour que la figure soit un triangle équilatéral. \textbf{Justifier} la réponse.

Pour former son motif, Mathis hésite entre les trois motifs de triangles suivants :
\end{enumerate}
\end{minipage}\hfill
\begin{minipage}{.3\linewidth}
\begin{scratch}[scale=0.8]
\blockinit{quand \greenflag est cliqué}
\blockpen{stylo en position d'écriture}
\blockrepeat{répéter \ovalnum{\ldots} fois}
{
\blockmove{avancer de \ovalnum{100} pas}
\blockmove{tourner \turnleft{} de \ovalnum{120} degré}
}
\end{scratch}
\end{minipage}

\medskip

\begin{center}
\begin{tabularx}{\linewidth}{|*{3}{>{\centering \arraybackslash}X|}}\hline
motif 1 &motif 2& motif 3\\ \hline
\psset{unit=1cm}
\begin{pspicture}(-2.2,0)(2.2,1.9)
\def\tri{\pspolygon(1;-30)(1;90)(1;210)}
\rput(-1.2,0.8){\tri}\rput(1.2,0.8){\tri}
\end{pspicture}&
\psset{unit=1cm}
\begin{pspicture}(-2.2,0)(2.2,1.9)
\def\tri{\pspolygon(1;-30)(1;90)(1;210)}
\rput(-0.86,0.8){\tri}\rput(0.86,0.8){\tri}
\end{pspicture}&
\psset{unit=1cm}
\begin{pspicture}(-2.2,0)(2.2,1.9)
\def\tri{\pspolygon(1;-30)(1;90)(1;210)}
\rput(-0.3,0.8){\tri}\rput(0.3,0.8){\tri}
\end{pspicture}\\ \hline
\end{tabularx}
\end{center}

\medskip
 
Il décide de réaliser un programme Scratch pour chaque motif.

\begin{center}
\begin{tabularx}{\linewidth}{|*{3}{>{\centering \arraybackslash}X|}}\hline
Programme 1 : & Programme 2 : & Programme 3 :\\ 
\begin{scratch}[scale=0.7]
\blockinit{quand \greenflag est cliqué}
\blockpen{effacer tout}
\blockpen{stylo en position d'écriture}
\blockrepeat{répéter \ovalnum{3} fois}
{
\blockmove{avancer de \ovalnum{100} pas}
\blockmove{tourner \turnleft{} de \ovalnum{120} degrés}
}
\blockmove{avancer de \ovalnum{100} pas}
\blockrepeat{répéter \ovalnum{3} fois}
{
\blockmove{avancer de \ovalnum{100} pas}
\blockmove{tourner \turnleft{} de \ovalnum{120} degrés}
}
\end{scratch}
&
\begin{scratch}[scale=0.7]
\blockinit{quand \greenflag est cliqué}
\blockpen{effacer tout}
\blockpen{stylo en position d'écriture}
\blockrepeat{répéter \ovalnum{3} fois}
{
\blockmove{avancer de \ovalnum{100} pas}
\blockmove{tourner \turnleft{} de \ovalnum{120} degrés}
}
\blockmove{avancer de \ovalnum{40} pas}
\blockrepeat{répéter \ovalnum{3} fois}
{
\blockmove{avancer de \ovalnum{100} pas}
\blockmove{tourner \turnleft{} de \ovalnum{120} degrés}
}
\end{scratch}
&
\begin{scratch}[scale=0.7]
\blockinit{quand \greenflag est cliqué}
\blockpen{effacer tout}
\blockpen{stylo en position d'écriture}
\blockrepeat{répéter \ovalnum{3} fois}
{
\blockmove{avancer de \ovalnum{100} pas}
\blockmove{tourner \turnleft{} de \ovalnum{120} degrés}
}
\blockpen{relever le stylo}
\blockmove{avancer de \ovalnum{140} pas}
\blockpen{stylo en position d'écriture}
\blockrepeat{répéter \ovalnum{3} fois}
{
\blockmove{avancer de \ovalnum{100} pas}
\blockmove{tourner \turnleft{} de \ovalnum{120} degrés}
}
\end{scratch}
\\ \hline
\end{tabularx}
\end{center}

\begin{enumerate}
\setcounter{enumi}{1}
\item \textbf{Associer} le programme Scratch correspondant à chaque motif. \textbf{Écrire} les réponses sur la copie.
\end{enumerate}

Mathis a choisi le motif 2, mais il s'aperçoit que le motif n'est pas assez grand pour recouvrir son surfboard. Il décide donc de faire un motif avec cinq triangles au lieu de deux, comme schématisé ci-dessous.

\begin{center}
\psset{unit=1cm}
\begin{pspicture}(9,1.6)
\def\tri{\pspolygon[linecolor=blue](1;-30)(1;90)(1;210)}
\multido{\n=1+1.732}{5}{\rput(\n,0.5){\tri}}
\end{pspicture}
\end{center}

\begin{enumerate}[resume]
\item Parmi les trois programmes Scratch suivants, \textbf{indiquer} celui qui correspond au motif de Mathis. \textbf{Écrire} la réponse sur la copie.
\end{enumerate}

\begin{center}
\begin{tabularx}{\linewidth}{|*{3}{>{\centering \arraybackslash}X|}}\hline
Programme 1 : &Programme 2 :& Programme 3 :\\
\begin{scratch}[scale=0.7]
\blockinit{quand \greenflag est cliqué}
\blockpen{effacer tout}
\blockpen{stylo en position d'écriture}
\blockrepeat{répéter \ovalnum{5} fois}
{\blockrepeat{répéter \ovalnum{3} fois}
	{\blockmove{avancer de \ovalnum{100} pas}
\blockmove{tourner \turnleft{} de \ovalnum{120} degré}
\blockmove{avancer de \ovalnum{100} pas}
	}
}
\end{scratch}&
\begin{scratch}[scale=0.7]
\blockinit{quand \greenflag est cliqué}
\blockpen{effacer tout}
\blockpen{stylo en position d'écriture}
\blockrepeat{répéter \ovalnum{5} fois}
{\blockrepeat{répéter \ovalnum{3} fois}
	{\blockmove{avancer de \ovalnum{100} pas}
\blockmove{tourner \turnleft{} de \ovalnum{120} degré}
	}
\blockmove{avancer de \ovalnum{100} pas}
}
\end{scratch}&
\begin{scratch}[scale=0.7]
\blockinit{quand \greenflag est cliqué}
\blockpen{effacer tout}
\blockpen{stylo en position d'écriture}
\blockrepeat{répéter \ovalnum{3} fois}
{\blockmove{avancer de \ovalnum{100} pas}
\blockmove{tourner \turnleft{} de \ovalnum{120} degré}
}
\blockrepeat{répéter \ovalnum{5} fois}
{\blockmove{avancer de \ovalnum{100} pas}
}
\end{scratch}\\ \hline
\end{tabularx}
\end{center}

