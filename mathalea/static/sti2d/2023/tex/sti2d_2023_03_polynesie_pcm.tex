
\begin{center}
\textbf{Sécurité d'un four à pyrolyse}
\end{center}

La fonction $\theta$, représentée ci-dessous, modélise l'évolution de la température du four (exprimée en degré Celsius) en fonction du temps t (exprimé en minute) écoulé depuis la fin de la pyrolyse. L'instant initial $t = 0$ correspond au début de la phase de refroidissement.

\begin{center}
\psset{unit=0.015cm,arrowsize=2pt 4}
\begin{pspicture}(-80,-20)(660,600)
\psaxes[linewidth=1.25pt,Dx=60,Dy=100]{->}(0,0)(0,0)(660,600)
\multido{\n=0+12}{56}{\psline[linewidth=0.2pt](\n,0)(\n,600)}
\multido{\n=0+20}{31}{\psline[linewidth=0.2pt](0,\n)(660,\n)}
\psplot[plotpoints=2000,linewidth=1.25pt,linecolor=blue]{0}{660}{480 2.71828 x 95 div neg exp mul 20 add}
\uput[d](560,-30){Temps (en min)}
\rput{90}(-70,400){Température (en $\degres C$}
\end{pspicture}

\bigskip

\emph{Evolution de la température en fonction du temps lors de la phase de refroidissement}
\end{center}

\begin{enumerate}[start=4]
\item Déterminer graphiquement $\displaystyle\lim_{t \to + \infty} \theta(t)$.
\item Interpréter cette limite dans le contexte de l'exercice.
\end{enumerate}
La fonction $\theta$ utilisée pour cette modélisation est définie sur $[~0~; +\infty[$ par : $\theta(t) = 480\e^{-\frac{1}{95}t} + 20.$
\begin{enumerate}[start=6]
\item Calculer la valeur exacte de la solution de l'équation $\theta(t) = 280$.
\end{enumerate}
Pour des raisons de sécurité, le fabricant impose que la porte du four reste verrouillée tant que la température du four est supérieure à $280\degres C$.
\begin{enumerate}[start=7]
\item Au bout de combien de temps la porte se déverrouille-t-elle ?
\end{enumerate}

\bigskip


