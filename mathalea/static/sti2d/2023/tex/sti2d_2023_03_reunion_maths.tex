
\medskip

\textbf{Les questions $1$ et $2$ sont indépendantes.}

\bigskip

\textbf{Question 1}

\medskip

Simplifier l'écriture de l'expression suivante : 
\[A(x) = - \ln (9) + 2 \ln(3x).\]

\bigskip

\textbf{Question 2}

\medskip

Le plan complexe est rapporté à un repère orthonormé \Ouv.

\medskip

Le point M d'affixe $z_{\text M}$ vérifie les conditions suivantes :
\begin{itemize}
\item[$\bullet~~$] M appartient au cercle de centre O et de rayon 6 ;
\item[$\bullet~~$] la partie réelle de $z_{\text M}$ est négative;
\item[$\bullet~~$] la partie imaginaire de $z_{\text M}$ est égale à 3.
\end{itemize}

\medskip

\begin{enumerate}
\item Soit $\theta$ la mesure dans $[0~;~ 2\pi[$ de l'argument du nombre complexe $z_{\text M}$.

Déterminer $\sin (\theta)$.

\item À l'aide du demi-cercle trigonométrique ci-dessous, donner la valeur exacte de $\theta$.

Justifier.

\item En déduire l'écriture exponentielle de $z_{\text M}$.
\end{enumerate}

\begin{center}
\psset{unit=0.9cm}
\begin{pspicture}(-7.5,-1)(7.5,7.5)
\psaxes[linewidth=1.25pt,Dx=10,Dy=10](0,0)(-7,0)(7.5,7)
\psarc(0,0){7}{0}{180}
\uput[ur](7;30){$\frac{\pi}{6}$}\uput[ur](7;45){$\frac{\pi}{4}$}\uput[ur](7;60){$\frac{\pi}{3}$}
\uput[u](7;90){$\frac{\pi}{2}$}\uput[ul](7;120){$\frac{2\pi}{3}$}\uput[ul](7;135){$\frac{3\pi}{4}$}
\uput[ul](7;150){$\frac{5\pi}{6}$}\uput[l](7;180){$\pi$}
\uput[d](6.062,0){$\frac{\sqrt{3}}{2}$}\uput[d](4.95,0){$\frac{\sqrt{2}}{2}$}\uput[d](3.5,0){$\frac12$}
\uput[d](-6.062,0){$\frac{-\sqrt{3}}{2}$}\uput[d](-4.95,0){$-\frac{\sqrt{2}}{2}$}\uput[d](-3.5,0){$-\frac12$}
\uput[l](0,6.062){$\frac{\sqrt{3}}{2}$}\uput[l](0,4.95){$\frac{\sqrt{2}}{2}$}\uput[l](0,3.5){$\frac12$}
\psline[linestyle=dashed](-6.062,0)(-6.062,3.5)(-0.5,3.5)
\psline[linestyle=dashed](6.062,0)(6.062,3.5)(0,3.5)
\psline[linestyle=dashed](-4.95,0)(-4.95,4.95)(-1,4.95)
\psline[linestyle=dashed](4.95,0)(4.95,4.95)(0,4.95)
\psline[linestyle=dashed](-3.5,0)(-3.5,6.062)(-1,6.062)
\psline[linestyle=dashed](3.5,0)(3.5,6.062)(0,6.062)
\end{pspicture}
\end{center}

\bigskip


