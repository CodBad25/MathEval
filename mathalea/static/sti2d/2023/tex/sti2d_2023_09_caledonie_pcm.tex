
\begin{center}
\textbf{Changement de carburant d’un moteur thermique}
\end{center}

La température du moteur (exprimée en $\degres$C) est modélisée par une fonction $\theta$ dépendant du temps (exprimé en secondes) écoulé depuis le démarrage du moteur. \\
On admet que la fonction $\theta$, définie et dérivable sur $[0~;~+\infty[$, est une solution sur cet intervalle de l'équation différentielle suivante :

\[y' = - \dfrac{1}{180}y + \dfrac49.\]

\begin{enumerate}[start=7]
\item Déterminer les solutions sur $[0~;~+\infty[$ de cette équation différentielle.

À $t = 0$, la température du moteur est de $20\degres$ C.

\item Montrer alors que la fonction $\theta$ est définie sur $[0~;~+\infty[$ par: 

\[\theta (t) = 80 - 60\e^{-\frac{1}{180}t}.\]

\item Résoudre sur $[0~;~+\infty[$ l'équation $\theta(t) = 79$.
\end{enumerate}

\medskip

Le changement de carburant ne doit pas modifier la montée en température du moteur. La température optimale de fonctionnement du moteur est de $79\degres$ C.\\
Cette température doit être atteinte en moins de vingt minutes.

\begin{enumerate}[start=10]
\item Indiquer si cette condition est respectée.
\end{enumerate}

\bigskip


