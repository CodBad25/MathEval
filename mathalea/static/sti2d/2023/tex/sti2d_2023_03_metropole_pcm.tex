
\begin{center}
\textbf{Le viscosimètre à chute de bille}
\end{center}

La viscosité d'une huile, notée $\nu$, est un paramètre exprimé en kg.m$^{-1}$.s$^{-1}$, dont la connaissance est essentielle pour toute utilisation de cette huile.

Cet exercice propose un exemple de méthode de mesure de la valeur de la viscosité d'une huile de moteur Diesel du commerce.

Pour réaliser cette mesure, on utilise un \og viscosimètre à chute de bille \fg, constitué d'une éprouvette remplie d'huile de moteur dans laquelle est lâchée une bille métallique sphérique.

On se place dans le référentiel terrestre supposé galiléen et la bille est lâchée sans vitesse initiale depuis la position $z = 0$.

\begin{center}
\psset{unit=1cm,arrowsize=2pt 3}
\begin{pspicture}(6,6)
\psline(0.8,5.8)(0.8,0)(1.8,0)(1.8,5.8)
\psline(0.5,0)(2.1,0)
\psframe[fillstyle=solid,fillcolor=lightgray](0.8,0)(1.8,5.8)
\multido{\n=0.0+0.9}{6}{\psline(0.8,\n)(1.2,\n)}
\pscircle*(1.3,4){0.2}
\psline[linewidth=0.6pt]{->}(0.1,5.8)(0.1,0.6)
\uput[u](0.1,5.8){$z = 0$}\uput[d](0.1,0.6){$z$}
\rput(4.4,4){Bille en acier}
\rput(4.4,2.15){Éprouvette}
\rput(4.4,1.8){graduée remplie}
\rput(4.4,1.45){d'huile moteur}
\psline{->}(3.3,4)(1.6,4)
\psline{->}(3.,1.8)(1.8,1.8)
\end{pspicture}
\end{center}

\textbf{Données :}

\begin{itemize}
\item Rayon de la bille utilisée : $R = 1,1$ cm.
\item Volume de la bille : $V = 5,6$~cm$^3 = 5,6 \times 10^{-6}$~m$^3$.
\item Masse de la bille métallique : $m = 20,1$~g.
\item Masse volumique de l'huile étudiée : $\rho_{\text{huile}} = 8,40 \times 10^2$~kg.m$^{-3}$.
\item Intensité de la gravitation : $g = 9,8$~m.s$^{-2}$.
\end{itemize}

\medskip

On note $v$ la fonction définie sur $[0~;~+\infty[$ comme la projection du vecteur vitesse $\vect{v}$ sur l'axe (O$z$). 

On admet que $v$ est solution de l'équation différentielle $(E)$ suivante où $v(t)$ est exprimée en m.s$^{-1}$ et $t$ en s :
\[(E) : \qquad \dfrac{\text{d}v}{\text{d}t} = - 6,8v + 7,5.\]

\begin{enumerate}[start=4]
\item Au début de l'expérience, la bille est introduite dans l'éprouvette avec une vitesse nulle. 

Démontrer que la solution $v$ de cette équation sur $[0~;~+\infty[$ vérifiant cette condition initiale est définie par :

\[v(t) = - \dfrac{75}{68}\text{e}^{-6,8t} + \dfrac{75}{68}.\]

\item Déterminer la valeur exacte de $\displaystyle\lim_{t \to + \infty} v (t)$ notée $v_{\text{lim}}$ exprimée en m.s$^{-1}$.

\item  On mesure expérimentalement une vitesse limite $v_{\text{lim}} = 1,1~$m.s$^{-1}$.

On peut en déduire la valeur de la viscosité $\eta$ par la relation suivante :

\[\eta = \dfrac{\left(m - \rho_{\text{huile}} V\right)g}{6\pi R v_{\text{lim}}}.\]

Calculer cette valeur et comparer le résultat à la valeur $\eta =  0,66$~kg.m$^{-1}$.\,s$^{-1}$ fournie par le fabricant.
\end{enumerate}

\bigskip


