
\medskip

\begin{enumerate}[start=4]
\item Il semble que $\displaystyle\lim_{t \to + \infty} \theta(t) = 20$.
\item La température après 10 heures de refroidissement va se rapprocher de $20\degres$ C.
\item $\theta(t) = 280 \iff 480\e^{-\frac{1}{95}t} + 20 = 280 \iff 480\e^{-\frac{1}{95}t} = 260 \iff \e^{-\frac{1}{95}t} = \dfrac{260}{480} = \dfrac{26}{48} = \dfrac{13}{24}$.

En utilisant la croissance de la fonction logarithme népérien, on a donc :

$-\frac{1}{95}t = \ln \dfrac{13}{24} \iff t = \dfrac{\ln \dfrac{13}{24}}{-\frac{1}{95}} = - 95 \ln \dfrac{13}{24}$ (environ 58,24 min).

\item On a $8,24 = 58 + 0,24 \times 60 \approx 58$ min 14 s.
\end{enumerate}

\bigskip


