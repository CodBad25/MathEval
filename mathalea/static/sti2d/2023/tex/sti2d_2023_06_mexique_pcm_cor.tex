
\medskip
 
\begin{enumerate}[start=5]
\item On détermine une équation de la tangente à la courbe représentative de la fonction $f$ au point d'abscisse 0.% est :  $y = -\np{0,02576}x + 2,3.$

$f'(x)=2,3\times (-\np{0,0112}) \e^{-\np{0,0112}x} = -\np{0,02576}\e^{-\np{0,0112}x}$
donc $f'(0)=-\np{0,02576}$.

$f(x) = 2,3\e^{-\np{0,0112}x}$ donc $f(0)=2,3$.

La tangente a pour équation: $y=f'(0)(x-0)+f(0)$ soit $y=-\np{0,02576} x + 2,3$.
 
%On rappelle qu'une équation de la tangente à la courbe représentative d'une fonction $f$ au point d'abscisse $a$ est
%\begin{center} $y = f'(a)(x - a) + f(a)$ où $f'$ est la fonction dérivée de $f$.\end{center}
\item L'abscisse $\tau$ du point d'intersection de cette tangente avec l'axe des abscisses est la solution de l'équation: $0 =-\np{0,02576} x + 2,3$; donc $\tau=\dfrac{2,3}{\np{0,02576}}\approx 89,3$.

%On donnera une valeur approchée à $10^{-1}$ près.
\item On sait que: $\tau = RC$ et $R = 0,235~\Omega$.

Donc $C=\dfrac{\tau}{R}=\dfrac{89,3}{0,235} \approx 380$ F.

La valeur obtenue à partir de ce modèle est un peu supérieure à celle donnée par le  constructeur (372~F).
\end{enumerate}

\bigskip


