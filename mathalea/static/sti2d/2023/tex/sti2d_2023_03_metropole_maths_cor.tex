
\bigskip

\textbf{Question 1}

\medskip

$\dfrac{\left(\text{e}^{-3x}\right)^2 \times \left(\text{e}^{2x}\right)^{-3}}{\text{e}^{5x} \times \text{e}^{6x}} = \dfrac{\text{e}^{-6x}\times \text{e}^{-6x}}{\text{e}^{11x}} = \dfrac{\text{e}^{-12x}}{\text{e}^{11x}} = \text{e}^{-12x - 11x} = \text{e}^{- 23x} \rightarrow$ réponse \textbf{D}.

\bigskip

\textbf{Question 2}

\medskip

En posant $u(x) = \text{e}^{2x}$, d'où $u'(x) = 2\text{e}^{2x}$ et $v(x) = - 3x + 1$, d'où $v'(x) = - 3$, on a :
\begin{align*}
f'(x) &= (uv)'(x) = u'(x) \times v(x) + u(x) v'(x) \\
&= 2\text{e}^{2x}(-3x + 1) - 3\text{e}^{2x} \\
&= \text{e}^{2x}[2(- 3x + 1)- 3] \\
&= \text{e}^{2x}(-6x + 2 - 3) \\
&= \text{e}^{2x}(- 6x - 1).
\end{align*}

\bigskip

\textbf{Question 3}

\medskip

Avec $z = \sqrt 3 + \text{i}$, on a :
\[|z| = \sqrt{\left(\sqrt 3  \right)^2 + 1^2} = \sqrt{3 + 1} = 2.\]

On peut alors en factorisant 2, écrire :
\[z = 2\left( \dfrac{\sqrt 3}{2} + \dfrac{1}{2}\text{i} \right).\]

Or on sait que $\cos \dfrac{\pi}{6} = \dfrac{\sqrt 3}{2}$ et $\sin \dfrac{\pi}{6} = \dfrac{1}{2}$, donc :
\[z = 2\left(\cos \dfrac{\pi}{6} + \text{i}\sin \dfrac{\pi}{6}  \right) = 2 \text{e}^{\frac{\pi}{6}}.\]

\bigskip

\textbf{Question 4}

\begin{align*}
&\dfrac{2}{3\ln (10)} \ln (x) - 2,88 = 4 \\
\iff &\dfrac{2}{3\ln (10)} \ln (x) = 6,88 \\
\iff &\dfrac{1}{3\ln (10)}\ln (x) = 3,44 \\
\iff &\ln (x) = 3 \times \dfrac{3,44 \ln (10)}{2} \\
\iff &\ln (x) = 10,32 \ln (10) \\
\iff &\ln (x) = \ln \left(10^{10,32} \right) \\
\iff &x = 10^{10,32}.
\end{align*}

\bigskip


