
\medskip

\begin{enumerate}
\item Le 1\ier{} janvier 2020 correspond à $t=0$; $f(0)=10\e^{0}=10$.

Donc le nombre d'habitants donné par ce modèle au 1\up{er} janvier 2020 est de 10 millions.

Le 1\ier{} janvier 2021 correspond à $t=1$; $f(1)=10\e^{0,02} \approx 10,2$.

Donc le nombre d'habitants donné par ce modèle au 1\up{er} janvier 2021 est d'environ $10,2$ millions.

\item Pour déterminer l'année durant laquelle l'effectif de la population dépassera 20 millions d'habitants, on cherche la plus petite valeur de $t$ pour laquelle $f(t)>20$.

$f(t)>20
\iff 10\e^{0,02t}>20
\iff \e^{0,02t}>2
\iff 0,02t>\ln(2)
\iff t>\dfrac{\ln(2)}{0,02}$

$\dfrac{\ln(2)}{0,02} \approx 34,66$ donc c'est au cours de la 34\ieme{} année, c'est-à-dire en $2054$ que l'effectif dépassera 20 millions.
\end{enumerate}

