
\medskip

\textbf{Les questions $1$ et $2$ sont indépendantes.}

\bigskip

\textbf{Question 1}

\medskip

Soit la fonction $f$ définie et dérivable sur $[0~;~ +\infty[$ par 
\[f(x) = x\e^{-x}.\]

\begin{enumerate}
\item Donner la limite de $f$ en $+ \infty$.
\item Montrer que pour tout réel $x$ appartenant à $[0~;~ +\infty[$, \: $f'(x) = \e^{-x}(1 -
x)$, où $f'$ désigne la fonction dérivée de $f$.
\item En déduire le tableau complet des variations de la fonction $f$ sur $[0~;~ +\infty[$.
\end{enumerate}

\bigskip

\textbf{Question 2}

\medskip

On considère les nombres complexes $z_1 = 6\e^{\text{i}\frac{\pi}{4}}$ et $z_2 = - \sqrt 3 + \text{i}$, où i désigne le nombre complexe de module 1 et d'argument $\dfrac{\pi}{2}$.

\medskip

\begin{enumerate}
\item Écrire $z_2$ sous forme exponentielle. Détailler les calculs.
\item En déduire une écriture du nombre complexe $Z = \dfrac{z_1}{z_2^3}$ sous forme exponentielle.
\end{enumerate}

\bigskip


