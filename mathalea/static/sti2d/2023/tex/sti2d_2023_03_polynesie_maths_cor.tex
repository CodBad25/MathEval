
\bigskip

\textbf{Question 1}

\medskip

\begin{enumerate}
\item On a $f(x) = \dfrac{x}{\e^x}$; or on sait que $\displaystyle\lim_{x \to + \infty}  \dfrac{\e^x}{x} = + \infty$, donc $\displaystyle\lim_{x \to + \infty}  \dfrac{x}{\e^x} = 0$.
\item La fonction $f$ produit de deux fonctions dérivables sur $\R$ est dérivable en particulier sur $[0~;~ +\infty[$ et sur cet intervalle, en dérivant comme un produit :
$f'(x) = 1\e^{-x} - 1\times x\e^{-x} = \e^{-x} (1 - x)$.
\item On sait que quel que soit $x \in [0~;~ +\infty[$, \: $\e^{- x} > 0$ : le signe de $f'(x)$ est donc celui de $1 - x$ :

$\bullet~~1 - x > 0 \iff 1 > x$ : on a donc $f'(x) > 0$ sur $[0~;~1[$ ; la fonction est croissante sur $[0~;~1[$ ;

$\bullet~~1 - x < 0 \iff 1 < x$ : on a donc $f'(x) < 0$ sur $[1~;~+\infty]$ ; la fonction est décroissante sur $[1~;~+\infty]$ ;

$\bullet~~1 - x = 0 \iff 1 = x$ : on a donc $f'(1) = 0$ ; la fonction $f$ a un maximum en 1 égal à $f(1) = 1\e^{-1} = \dfrac{1}{\text{e}} \approx 0,368$.

$\bullet~~$D'autre part $f(0) = 0 \times \e^{0} = 0 \times 1 = 0$. D'où le tableau de variations :

\begin{center}
\psset{unit=1cm,arrowsize=2pt 3}
\begin{pspicture}(7,2.5)
\psframe(7,2.5)\psline(0,1.5)(7,1.5)\psline(0,2)(7,2)\psline(1,0)(1,2.5)
\uput[u](0.5,1.9){$x$} \uput[u](1.15,1.9){$0$} \uput[u](4,1.9){$1$} \uput[u](6.5,1.9){$+\infty$} 
\rput(0.5,1.75){$f'(x)$}\rput(2.5,1.75){$+$}\rput(4,1.75){0}\rput(5.5,1.75){$-$}
\uput[u](1.15,0){0}\uput[d](4,1.5){$\e^{-1}$}\uput[u](6.5,0){0}
\psline{->}(1.5,0.5)(3.5,1)\psline{->}(4.5,1.)(6.5,0.5)
\rput(0.5,0.75){$f(x)$}
\end{pspicture}
\end{center}
\end{enumerate}

\bigskip

\textbf{Question 2}

\medskip

\begin{enumerate}
\item $z_2 = - \sqrt 3 + \text{i}$.

Donc $\left|z_2\right|^2 = \left(- \sqrt 3\right)^2 + 1^2 = 3 + 1 = 4 = 2^2$, donc $\left|z_2\right| = 2$.

On peut en factorisant ce module 2 dans l'écriture de $z_2$, écrire :

$z_2 = 2\left(- \dfrac{\sqrt{3}}{2} + \text{i}\dfrac12\right)$.

Or $- \dfrac{\sqrt{3}}{2} = \cos \dfrac{5\pi}{6}$ et $\dfrac12 = \sin \dfrac{5\pi}{6}$, donc :

$z_2 = 2\left(\cos \dfrac{5\pi}{6} + \text{i}\sin \dfrac{5\pi}{6}\right) = 2\e^{\frac{5\pi}{6}}$.

\item En se servant du résultat précédent : 

$Z = \dfrac{z_1}{z_2^3} = \dfrac{6\e^{\text{i}\frac{\pi}{4}}}{\left(2\e^{\text{i}\frac{5\pi}{6}}\right)^3} =  \dfrac68 \dfrac{\e^{\text{i}\frac{\pi}{4}}}{\e^{\text{i}\frac{5\pi}{2}}}$.

Or $\e^{\text{i}\frac{5\pi}{2}} = \e^{\text{i}\frac{\pi}{2}}$, donc 

$Z = \dfrac34\e^{\text{i}\left(\frac{\pi}{4} - \frac{\pi}{2} \right)} = \dfrac34\e^{-\text{i}\frac{\pi}{4}}$.
\end{enumerate}

\bigskip


