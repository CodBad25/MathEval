
\medskip

\begin{enumerate}
\item Écrivons $z_1$ sous forme exponentielle, en détaillant les calculs.

Une forme exponentielle d'un nombre complexe z non nul est $z=\vert z\vert \e^{\text{i}\theta}$.

$\vert z\vert = \sqrt{\sqrt{2}^2+\sqrt{2}^2}=\sqrt{2+2}=2$

$\cos \theta =\dfrac{\sqrt{2}}{2}$ et $\sin \theta =\dfrac{\sqrt{2}}{2}$. Il en résulte $\theta= \dfrac{\pi}{4}$.

Une écriture sous forme exponentielle de $z$ est , par conséquent $z=2\e^{\text{i}\frac{\pi}{4}}$

\item Montrons que $2 z_2^3 = z_1$.

En utilisant la formule de Moivre $\left(\e^{\text{i}\theta}\right)^n=\left(\e^{n\text{i}\theta}\right)$, nous avons :

\[2 z_2^3=2\left(\e^{\text{i}\frac{\pi}{12}}\right)^3 =2\left(\e^{3\text{i}\frac{\pi}{12}}\right)=2\left(\e^{\text{i}\frac{\pi}{4}}\right)=z_1.\]                                                                                                                                                                                                                                                                                             \end{enumerate}

\bigskip

