
\begin{enumerate}[start=4]
\item Avec la précision donnée par le graphique, la courbe coupe la droite d'équation $y= \np{10000}$ au point d'abscisse $T = 25$ (en \textcelsius).

\item 

\begin{align*}
\np{28785} \e^{-0,042\times T}&=\np{10000}\\
 \e^{-0,042\times T}&=\dfrac{\np{10000}}{\np{28785}}\\
  \e^{-0,042\times T}&=\frac{\np{2000}}{\np{5757}}\\
  -0,042\times T&=\ln \frac{\np{2000}}{\np{5757}}\\
T&=-\dfrac{\ln \frac{\np{2000}}{\np{5757}}}{\np{0.042}}\\
T& \approx \np{25.173}.
\end{align*}

Cette valeur est approximativement la même que celle lue sur le graphique en réponse à la question \textbf{4}.

\item 

\[R'(T)=-\np{0.042} \times \np{28785} \times  \e^{-0,042\times T}= -\np{1208.97}\e^{-0,042\times T}\]

\item Comparons $S(30)$ et $S(90)$.

\[\dfrac{S(30)}{S(90)}=\dfrac{-\left(-\np{1208.97}\e^{-0,042\times 30}\right)}{-\left(-\np{1208.97}\e^{-0,042\times 90}\right)}=\dfrac{\e^{-0,042\times 30}}{\e^{-0,042\times 90}} \approx 12,43.\]

Nous avons donc : $S(30) \approx 12 \times S(90)$.
\end{enumerate}

\bigskip


