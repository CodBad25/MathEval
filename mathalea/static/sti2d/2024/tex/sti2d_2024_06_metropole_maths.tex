
\medskip

\textbf{Dans cet exercice, les questions 1, 2, 3 et 4 peuvent être traitées de façon indépendante les unes des autres.}

Un parachutiste est en chute libre dans l'air jusqu'à l'instant $t = 0$ où il ouvre son parachute. Sa vitesse est alors de 50~m.s$^{-1}$. On admet par la suite que sa vitesse $v$, en m.s$^{-1}$, en fonction du temps $t$, en $s$, est solution de l'équation différentielle sur l'intervalle $[0\;;\;+\infty[$:

\[(E)\;:\quad y'=-5y+10.\]


\begin{flushleft}
\textbf{Question 1}
\end{flushleft}

La fonction constante $g$ définie sur l'intervalle $[0\;;\;+\infty[$ par $g(t)=2$ est-elle une solution de l'équation différentielle $(E)$ ? 
Justifier la réponse.

\begin{flushleft}
\textbf{Question 2}
\end{flushleft}

Montrer que les solutions de l'équation différentielle $(E)$ sur l'intervalle $[0\;;\;+\infty[$ sont les fonctions $f$ définies sur cet intervalle par $f(t)=k\e^{-5t}+2$, où $k$ est un nombre réel donné.

\begin{flushleft}
\textbf{Question 3}
\end{flushleft}

En admettant le résultat de la question précédente, montrer que la fonction $v$ est donnée sur $[0\;;\;+\infty[$ par $v(t) = 48 \e^{-5t} + 2$.

\begin{flushleft}
\textbf{Question 4}
\end{flushleft}

La distance parcourue, en mètre, par le parachutiste pendant les 10 premières secondes après ouverture du parachute est donnée  par l'intégrale :

\[\ds\int_{0}^{10} \left ( 48 \e^{-5t} +2 \right ) \d t\]

Calculer cette intégrale (arrondir à $10^{-1}$).




