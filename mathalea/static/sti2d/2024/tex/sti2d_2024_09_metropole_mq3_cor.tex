
\medskip

Forme algébrique de $z_A$ :
\begin{align*}
z_A &= 2\e^{\frac{\pi}{3}\text{i}} \\
&= 2 \left( \cos\left(\dfrac{\pi}{3}\right) + \text{i}\sin\left(\dfrac{\pi}{3}\right) \right) \\
&= 2 \left( \dfrac{1}{2} + \text{i}\dfrac{\sqrt{3}}{2} \right) \\
&= 1 + \text{i}\sqrt{3}
\end{align*}

Forme exponentielle de $z_B = -\sqrt{3} + \text{i}$ :
\[\left|z_B\right| = \sqrt{\left(-\sqrt{3}\right)^2 + 1^2} = \sqrt{3 + 1} = \sqrt{4} = 2.\]

L'argument $\theta$ est l’angle tel que :
\[\cos(\theta) = \frac{-\sqrt{3}}{2} \quad \text{et} \quad \sin(\theta) = \frac{1}{2} \quad \text{soit} \quad \theta = \dfrac{5\pi}{6}\]

Ce qui donne : $z_B = 2e^{\text{i}\frac{2\pi}{3}}$.

\bigskip

