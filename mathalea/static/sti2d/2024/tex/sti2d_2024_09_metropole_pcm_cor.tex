
\medskip

\begin{enumerate}
\item Remplaçons les valeurs connues dans la formule puis résolvons l'équation :
\begin{align*}
&85 = 10 \log \left(\dfrac{I}{10^{-12}}\right) \\
\iff &\log \left(\dfrac{I}{10^{-12}}\right) = 8,5 \\
\iff &\dfrac{I}{10^{-12}} = 10^{8,5} \\
\iff &I = 10^{-12} \cdot 10^{8,5} \\
\iff &I = 10^{-3,5}.
\end{align*}
Soit $I_1 \approx 3,16 \times 10^{-4}$ W.m$^2$.

\item Cette équation est une équation différentielle linéaire du premier ordre, de la forme :

$y' + ay = 0$ où $a = 0,262$.

La solution générale de cette équation est donnée par :

$y(x) = C \e^{-ax}$ où $C$ est une constante d'intégration.

En remplaçant $a = 0,262$, la solution devient : $y(x) = C \e^{-0,262x}$.

\item La fonction proposée est :
$y = f(x) = 3,2 \cdot 10^{-4} \e^{-0,262x}$.

Sa dérivée est :
$y' = f'(x) = 3,2 \cdot 10^{-4} \cdot (-0,262) \e^{-0,262x}$.

Soit $y' = -0,262y$, l'équation différentielle $(E)$ est donc satisfaite.

Calculons $f(0)$ :
$f(0) = 3,2 \cdot 10^{-4} \e^{-0,262 \cdot 0} = 3,2 \cdot 10^{-4} \cdot 1 = 3,2 \cdot 10^{-4}$.

La condition initiale $f(0) = 3,2 \cdot 10^{-4}$ est donc satisfaite.

La fonction $f(x) = 3,2 \cdot 10^{-4} \e^{-0,262x}$ est une solution particulière de $(E)$ vérifiant la condition initiale.

\item
\begin{align*}
&\e^{-0,262x} = 0,5 \\
\iff &-0,262x = \ln(0,5) \\
\iff &x = \dfrac{\ln(0,5)}{-0,262} \\
\iff &x \approx 2,64.
\end{align*}
La distance de propagation $d$ au bout de laquelle l'intensité acoustique est divisée par 2 est :
$d \approx 2,6 \, \text{cm}.$
\end{enumerate}

\bigskip


