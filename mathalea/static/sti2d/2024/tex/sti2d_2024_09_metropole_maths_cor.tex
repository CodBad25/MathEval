
\bigskip

\textbf{Question 1}
\begin{align*}
&3\ln(2x) - \ln(8) \\
&= 3(\ln(2) + \ln(x)) - \ln(2^3) \\
&= 3\ln(2) + 3\ln(x) - 3\ln(2) \\
&= 3\ln(x)
\end{align*}
La bonne réponse est \textbf{B}.

\bigskip

\textbf{Question 2}

\medskip

$g(x)$ est de la forme $u(x) \times v(x)$ avec :
\begin{list}{\textbullet}{}
\item $u(x) = x^2 \quad \text{et} \quad u'(x) = 2x$,
\item $v(x) = \e^{-2x} \quad \text{et} \quad v'(x) = -2e^{-2x}$.
\end{list}
\begin{align*}
g'(x) &= u'(x) \times v(x) + u(x) \times v'(x) \\
&= 2x \times \e^{-2x} + x^2 \times \left(-2\e^{-2x}\right) \\
&= 2x \e^{-2x} - 2x^2 \e^{-2x} \\
&= 2x \e^{-2x} (1 - x).
\end{align*}
La bonne réponse est \textbf{A}.

\bigskip

\textbf{Question 3}

\medskip

Forme algébrique de $z_A$ :
\begin{align*}
z_A &= 2\e^{\frac{\pi}{3}\text{i}} \\
&= 2 \left( \cos\left(\dfrac{\pi}{3}\right) + \text{i}\sin\left(\dfrac{\pi}{3}\right) \right) \\
&= 2 \left( \dfrac{1}{2} + \text{i}\dfrac{\sqrt{3}}{2} \right) \\
&= 1 + \text{i}\sqrt{3}
\end{align*}

Forme exponentielle de $z_B = -\sqrt{3} + \text{i}$ :
\[\left|z_B\right| = \sqrt{\left(-\sqrt{3}\right)^2 + 1^2} = \sqrt{3 + 1} = \sqrt{4} = 2.\]

L'argument $\theta$ est l’angle tel que :
\[\cos(\theta) = \frac{-\sqrt{3}}{2} \quad \text{et} \quad \sin(\theta) = \frac{1}{2} \quad \text{soit} \quad \theta = \dfrac{5\pi}{6}\]

Ce qui donne : $z_B = 2e^{\text{i}\frac{2\pi}{3}}$.

\bigskip

\textbf{Question 4}
\begin{align*}
\displaystyle\int_0^{\frac{\pi}{2}} \cos (2x) \:\text{d}x
&= \left[ \dfrac{1}{2} \sin(2x) \right]_{0}^{\frac{\pi}{2}} \\
&= \dfrac{1}{2} \sin\left(2 \cdot \dfrac{\pi}{2}\right) - \dfrac{1}{2} \sin(2 \cdot 0) \\
&= \dfrac{1}{2} \sin(\pi) - \dfrac{1}{2} \sin(0) \\
&= \dfrac{1}{2} \cdot 0 - \dfrac{1}{2} \cdot 0 \\
&= 0
\end{align*}

\bigskip


