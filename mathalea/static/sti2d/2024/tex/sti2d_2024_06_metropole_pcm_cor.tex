
\begin{enumerate}[start=5]
\item Sur l'intervalle $[1\;;\;+\infty[$, $f'(x)=0-10\times \dfrac{1}{x}=-\dfrac{10}{x}$.
\item On résout l'équation $f_m(x) = g_m(x)$.

$\begin{aligned}
f_m(x) = g_m(x)&
\iff 148 - 10 \ln(x) = 136 - 7,5 \ln(x)
\iff 148-136 = 10\ln(x) - 7,5\ln(x)\\
&
\iff 12 = 2,5 \ln(x)
\iff \dfrac{12}{2,5} = \ln(x)
\iff 4,8 = \ln(x)
\iff \e^{4,8} = x\\
&
\text{donc }x\approx 121,5.
\end{aligned}$

La distance $d_m$ des enceintes à laquelle doit se trouver le public pour que les deux notes aient le même niveau sonore est donc d'environ 121,5~m.

\item Pour les réglages modifiés, le niveau sonore du son reçu par les spectateurs à la distance $d_m$ des enceintes pour chacune des deux notes est, en dB:

\begin{center}
$f_m\left (\e^{4,8}\right ) = 148 - 10\ln\left ( \e^{4,8} \right ) = 148-10\times 4,8 = 100$.
\end{center}
\end{enumerate}


