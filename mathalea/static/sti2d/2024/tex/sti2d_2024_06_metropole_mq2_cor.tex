
D'après le cours, les solutions de l'équation différentielle $y'=ay$ sur l'intervalle $[0\;;\;+\infty[$ sont les fonctions $f$ définies sur cet intervalle par $f(t)=k\e^{at}$,  où $k$ est un nombre réel quelconque, donc les solutions de l'équation différentielle $y'=-5y$ sur l'intervalle $[0\;;\;+\infty[$ sont les fonctions $f$ définies sur cet intervalle par $f(t)=k\e^{-5t}$,  où $k$ est un nombre réel quelconque.

Une solution de l'équation différentielle $y'=-5y+10$ est la somme d'une solution de l'équation différentielle $y'=-5y$ et d'une solution constante de l'équation différentielle $y'=-5y+10$, donc les solutions de l'équation différentielle $(E)$ sur l'intervalle $[0\;;\;+\infty[$ sont les fonctions $f$ définies sur cet intervalle par $f(t)=k\e^{-5t}+2$,  où $k$ est un nombre réel quelconque.

\medskip

