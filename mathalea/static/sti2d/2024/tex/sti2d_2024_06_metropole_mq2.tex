
Un parachutiste est en chute libre dans l'air jusqu'à l'instant $t = 0$ où il ouvre son parachute. Sa vitesse est alors de 50~m.s$^{-1}$. On admet par la suite que sa vitesse $v$, en m.s$^{-1}$, en fonction du temps $t$, en $s$, est solution de l'équation différentielle sur l'intervalle $[0\;;\;+\infty[$:

\[(E)\;:\quad y'=-5y+10.\]

Montrer que les solutions de l'équation différentielle $(E)$ sur l'intervalle $[0\;;\;+\infty[$ sont les fonctions $f$ définies sur cet intervalle par $f(t)=k\e^{-5t}+2$, où $k$ est un nombre réel donné.


