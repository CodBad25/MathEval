
\medskip

La tangente $T$ à $\mathcal{C}_f$ en A a pour équation $y= f'(x_{\text A})\left (x-x_{\text A}\right ) + f(x_{\text A})$ soit $y=f'(\e)\left (x-\e\right ) + f(\e)$.

$f(x)=\ln(x) $ donc $f(\e)=\ln(\e)=1$

$f(x)=\ln(x) $ donc $f'(x)=\dfrac{1}{x}$ et donc $f'(\e)=\dfrac{1}{\e}$

$T$ a donc pour équation $y=\dfrac{1}{\e} \left (x-\e\right ) + 1$ soit $y=\dfrac{1}{\e}x -1+1$ ou encore $y=\dfrac{1}{\e}x$.

Et si $x=0$, alors $y=\dfrac{1}{\e}x = 0$ donc la droite $T$ passe par l'origine du repère.

\bigskip

