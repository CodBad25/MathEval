
\begin{center}
\textbf{Onde sonore et intensité}
\end{center}

Le son est produit par la vibration d'objets et il arrive jusqu'à nos oreilles sous forme d'ondes se propageant dans l'air. Les sons sont perçus de manière plus ou moins intense.

L'intensité sonore, ou intensité acoustique notée $I$ et exprimée en W.m$^{-2}$, caractérise l'intensité du signal perçue par l'oreille.

On calcule le niveau d'intensité sonore noté $L$ en décibels (dB) à partir de l'intensité sonore notée $I$ (W.m$^{-2}$) par la relation: $L = 10 \log \left(\dfrac{I}{I_0} \right)$.

On rappelle que $I_0 = 10^{-12}$ W.m$^{-2}$ (intensité sonore minimale de référence).

\begin{enumerate}
\item Montrer que $I = I_0 \times  10^{\frac{L}{10}}.$

\item Calculer l'intensité sonore pour $L = 50$~dB.

\item L'intensité sonore $I$ double-t-elle lorsque l'on double le niveau d'intensité sonore $L$ ?

\item Pour une distance à la source $d_1$ (resp. $d_2$), on note $L_1$ (resp. $L_2$) le niveau d'intensité sonore à la distance $d_1$ (resp. $d_2$) de la source et $I_1$ (resp. $I_2$) l'intensité sonore à la distance $d_1$ (resp. $d_2$) de la source.

Le niveau d'intensité sonore diminue de $20$~dB lorsque la distance par rapport à la source est multipliée par $10$. Ainsi si $d_2 =10 d_1$, on a : $L_2 = L_1 - 20$~(dB).

Montrer que l'intensité sonore est divisée par $100$ lorsque la distance par rapport à la source est multipliée par $10$.
\end{enumerate}
\
\bigskip


