
\medskip

\begin{enumerate}
\item D'après le cours, l'équation différentielle $a y' + b y = 0$ pour $a\neq 0$ a pour solutions les fonctions $f$ définies sur $\R$ par $f(x)=k\e^{-\frac{b}{a}x}$ avec $k\in\R$.

Donc l'équation différentielle $(E)$ a pour solution dans $[0~;+ \infty[$ les fonctions $P$ définies par $P(x)=k \e^{-0,0434 x}$ avec $k\in\R$. 

$P(0) = 6,75 \iff k\e^{0}= 6,75 \iff k=6,75$.

Sur $[0~;+ \infty[$ la solution $P$ de cette équation différentielle qui vérifie la condition initiale $P(0) = 6,75$ est définie par $P(x)=6,75\e^{-0,0434x}$.

\item La puissance du signal au bout de 1~km est $P(1)\approx \np{6,4633}$, donc la perte de puissance une fois que le signal a parcouru 1~km depuis l'entrée est, en mW, $P(0)-P(1) \approx \np{0,2867}$ soit environ $287~ \mu$W.
\end{enumerate}

\bigskip

