
\medskip

\textbf{Dans cet exercice, seulement 4 questions au choix parmi les 6 questions proposées sont à traiter. Toutes ces questions sont indépendantes les unes des autres.}

On note $\mathbb{C}$ l'ensemble des nombres complexes, et i le nombre complexe de module 1 et d'argument $\dfrac{\pi}{2}$.

\bigskip

\textbf{Question 1}

\medskip

On considère le nombre complexe $z_1 = \dfrac{2 - 6\text{i}}{2 - \text{i}}$.

Déterminer la forme algébrique de $z_1$.

\bigskip

\textbf{Question 2}

\medskip

Soit $z_2$ le nombre complexe défini par: $z_2 = - 2 - 2\text{i}$.

\begin{enumerate}
\item Déterminer la forme exponentielle de $z_2$.
\item Montrer que $z_2^4$ est un nombre réel que l'on déterminera.
\end{enumerate}

\bigskip

\textbf{Question 3}

\medskip

On considère A, B et C les points du plan d'affixes respectives : 

\[z_{\text{A}} = 2 - 2\text{i}, \qquad z_{\text{B}} = - 2 - 2\text{i} \quad \text{et}\quad z_{\text{C}}=- 4\text{i}.\]

\begin{enumerate}
\item Placer les points A, B et C dans le plan complexe rapporté au repère orthonormal direct \Ouv{} d'unité 1~cm.
\item Montrer que le triangle ABC est rectangle et isocèle.
\end{enumerate}

\bigskip

\textbf{Question 4}

\medskip

On considère l'équation différentielle :
\[y' + 5y = 7 \]
où $y$ est une fonction de la variable $t$, définie et dérivable sur $\mathbb{R}$.

\begin{enumerate}
\item Résoudre cette équation différentielle.
\item Préciser l'expression de la solution $f$ vérifiant $f(0) = 4$.
\end{enumerate}

\bigskip

\textbf{Question 5}

\medskip

Soit $g$ la fonction définie sur l'intervalle $]0~;~+\infty[$ par :
\[g(x) = x \ln (x) - x + 4.\]

\begin{enumerate}
\item On admet que $g$ est dérivable sur l'intervalle $]0~;~+\infty[$, et on note $g'$ sa fonction dérivée. 

Montrer que pour tout réel $x$ de l'intervalle $]0~;~+\infty[$,\, $g'(x) = \ln (x)$. 
\item En déduire le sens de variation de $g$ sur l'intervalle $]0~;~+\infty[$.
\end{enumerate}

\bigskip

\textbf{Question 6}

\medskip

On considère la fonction $h$ définie sur $\R$ par :
\[h(x) = x^2\text{e}^{-x}.\]

\begin{enumerate}
\item Calculer la limite de $h$ en $-\infty$.
\item Justifier que $\displaystyle\lim_{x \to + \infty}  h(x) = 0$.
\end{enumerate}

\smallskip

On admet que $h$ est strictement décroissante sur l'intervalle $[2~;~+ \infty[$ et que l'équation 

$h(x) = 0,5$ admet une unique solution dans l'intervalle $[2~;~+ \infty[$ que l'on note $\alpha$.

\begin{enumerate}[resume]
\item Recopier le programme ci-dessous et compléter les pointillés pour que la fonction
\texttt{sol\_bal} détermine une valeur approchée à $10^{-n}$ près de $\alpha$ par balayage.

\begin{center}
\begin{tabularx}{0.4\linewidth}{|X|}\hline
from math import exp\\
~\\
def sol\_bal(n) \\
\quad x = 2\\
\quad while \ldots > 0,5\\
\qquad x = \ldots\\
\quad return x\\ \hline
\end{tabularx}
\end{center}
\end{enumerate}

\bigskip


