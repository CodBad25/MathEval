
\medskip

\begin{enumerate}
\item On admet que $g$ est dérivable sur l'intervalle $]0~;~+\infty[$, et on note $g'$ sa fonction dérivée. 

Pour tout $x>0$, $g'(x)=1\times \ln(x) + x\times \dfrac{1}{x}-1 - 0 = \ln(x) +1-1=\ln(x)$.

\item 
\begin{list}{\textbullet}{On étudie les variations de $g$ :}
\item Sur $]0~;~1[$, $\ln(x)<0$ donc $g'(x)<0$: la fonction $g$ est strictement décroissante.
\item Sur $]1~;~+\infty[$, $\ln(x)>0$ donc $g'(x)>0$: la fonction $g$ est strictement croissante.
\item Pour $x=1$, $\ln(x)=0$ donc $g'$ s'annule et passe de négative à positive: la fonction $g$ admet un minimum égal à $g(1)= 1\times\ln(1) - 1 + 4 = 3$.
\end{list}
\end{enumerate}

\bigskip

