
\begin{center}
\textbf{Four de recuit de détente}
\end{center}

Lors de leur fabrication, certaines pièces métalliques peuvent présenter des faiblesses dues à un refroidissement inégal après la coulée ou lors du fraisage, du tournage ou du rabotage. Pour réduire les contraintes dans la pièce, on procède à un traitement thermique appelé recuit de détente.

Pour réaliser un recuit de détente, on dispose d'un four thermique électrique qui permet d'obtenir progressivement la température souhaitée à l'aide d'une résistance chauffante.

La température au sein du four contenant les pièces en acier, dépendant du temps, est modélisée par une fonction $\theta$. La température est exprimée en degré Celsius et le temps est exprimé en seconde.

On admet que la fonction $\theta$, définie et dérivable sur l'intervalle $[0~;~ +\infty[$, est une solution, sur cet intervalle, de l'équation différentielle suivante :
\[(E) : \quad  800y' + y = 600.\]

À l'instant $t = 0$, on met le four sous tension. La température est alors de $25~\degres$C.

\medskip

\begin{enumerate}
\item À partir de l'équation différentielle ci-dessus, déterminer une durée caractéristique de l'évolution de la température dans le four et la valeur limite atteinte par la température du four.
\item 
	\begin{enumerate}
		\item Montrer que la fonction $\theta$ est définie sur $[0~;~ +\infty[$ par :
		\[\theta(t) = 600 - 575\e^{\np{-0,00125}t}.\]
		
		\item Quelle sera la température du four au bout de $10$~minutes ?
	\end{enumerate}
\item Selon la norme NF EN ISO 4885, le recuit de détente doit se faire lorsque la température du four est comprise entre $550~\degres$C et $650~\degres$C.
	\begin{enumerate}
		\item Selon ce modèle, déterminer le temps d'attente nécessaire pour que le four atteigne la température de $550~\degres$C. On arrondira le résultat à la minute.
		\item Selon ce modèle, la température du four peut-elle dépasser $600~\degres$C ?
	\end{enumerate}
\end{enumerate}

\bigskip


