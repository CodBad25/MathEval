
\medskip

\begin{enumerate}
\item On place les points A, B et C dans un repère :

\begin{center}
\psset{unit=1cm}
\def\xmin {-3}   \def\xmax {3}
\def\ymin {-5}   \def\ymax {2}
\begin{pspicture*}(\xmin,\ymin)(\xmax,\ymax)
\psgrid[subgriddiv=1,  gridlabels=0, gridcolor=lightgray] 
\psaxes[arrowsize=3pt 3, ticksize=-2pt 2pt, labels=none](0,0)(\xmin,\ymin)(\xmax,\ymax) 
\psaxes[linewidth=1.5pt]{->}(0,0)(1,1)[$\vect u$,-90][$\vect v$,180]
\uput[dl](0,0){O}
\pspolygon[showpoints,linecolor=blue](2,-2)(-2,-2)(0,-4)
{\blue
\uput[ur](2,-2){A} \uput[ul](-2,-2){B} \uput[dr](0,-4){C} 
}
\end{pspicture*}
\end{center}

\item On calcule les longueurs AB, BC et AC.

\begin{list}{\textbullet}{}
\item $\text{AB} = \left | z_{\text B} - z_{\text A} \right | = \left | -2-2\text{i} -2+2\text{i} \right | = \left | -4 \right | = 4$
\item  $\text{BC} = \left | z_{\text C} - z_{\text B} \right | = \left | -4\text{i} +2+2\text{i} \right | = \left | 2-2\text{i} \right | = \ds\sqrt{2^2 + (-2)^2} = \ds\sqrt{8} = 2\sqrt{2}$ 
\item $\text{AC} = \left | z_{\text C} - z_{\text A} \right | = \left | -4\text{i} -2+2\text{i} \right | = \left | -2-2\text{i} \right | = \ds\sqrt{(-2)^2 + (-2)^2} = \ds\sqrt{8} = 2\sqrt{2}$ 
\end{list}

$\text{BC} = \text{AC}$ donc le triangle ABC est isocèle.

$\text{BC}^2 + \text{AC}^2 = 8 + 8 = 16 = 4^2 =  \text{AB}^2$ donc d'après la réciproque du théorème de Pythagore, le triangle ABC est rectangle en C.

Donc  le triangle ABC est isocèle rectangle en C.
\end{enumerate}

\bigskip

