
\bigskip

\textbf{Question 1}

\medskip

Forme algébrique de $z_1$ :
\[z_1 = \dfrac{2 - 6\text{i}}{2 - \text{i}}
= \dfrac{\left (2-6\text{i}\right ) \left (2+\text{i}\right )}{\left (2-\text{i}\right ) \left (2+\text{i}\right )}
=\dfrac{4+2\text{i} -12\text{i} -6 \text{i}^2}{4 - \text{i}^2}
= \dfrac{10 -10\text{i}}{5} = \dfrac{5(2 - 2\text{i})}{5}
= 2 -2\text{i}\]

\bigskip

\textbf{Question 2}

\medskip

\begin{enumerate}
\item Forme exponentielle de $z_2$ :

\begin{list}{\textbullet}{}
\item $\left | z_2 \right | = \left | -2-2\text{i} \right | = \ds\sqrt{(-2)^2+(-2)^2} = \ds\sqrt{8} = 2\sqrt{2}$

\item $z_2 = - 2 - 2\text{i} = 2\sqrt{2}\left ( \dfrac{-2}{2\sqrt{2}} - \dfrac{2}{2\sqrt{2}}\text{i}\right )
= 2\sqrt{2} \left ( -\dfrac{\sqrt{2}}{2} - \dfrac{\sqrt{2}}{2}\text{i} \right )$

On cherche $\theta$ tel que $\cos \theta = -\dfrac{\sqrt{2}}{2}$ et $\sin \theta = - \dfrac{\sqrt{2}}{2}$.

Une valeur de $\theta$ est $-\dfrac{3\pi}{4}$.
\end{list}

$z_2$ a pour forme exponentielle $2\sqrt{2} \e^{-\frac{3\pi}{4}\text{i}}$.

\item 
$z_2^4 = \left (2\sqrt{2} \e^{-\frac{3\pi}{4}\text{i}}\right )^4
= \left (2\sqrt{2}\right )^4 \e^{4\times\left (-\frac{3\pi}{4}\text{i}\right )}
= 64 \e^{-3\pi\text{i}}$

$\e^{-3\pi\text{i}} = \e^{\pi\text{i}}=-1$ donc $z_2^4=-64$ et donc $z_2^4$ est un nombre réel.
\end{enumerate}

\bigskip

\textbf{Question 3}

\medskip

\begin{enumerate}
\item On place les points A, B et C dans un repère :

\begin{center}
\psset{unit=1cm}
\def\xmin {-3}   \def\xmax {3}
\def\ymin {-5}   \def\ymax {2}
\begin{pspicture*}(\xmin,\ymin)(\xmax,\ymax)
\psgrid[subgriddiv=1,  gridlabels=0, gridcolor=lightgray] 
\psaxes[arrowsize=3pt 3, ticksize=-2pt 2pt, labels=none](0,0)(\xmin,\ymin)(\xmax,\ymax) 
\psaxes[linewidth=1.5pt]{->}(0,0)(1,1)[$\vect u$,-90][$\vect v$,180]
\uput[dl](0,0){O}
\pspolygon[showpoints,linecolor=blue](2,-2)(-2,-2)(0,-4)
{\blue
\uput[ur](2,-2){A} \uput[ul](-2,-2){B} \uput[dr](0,-4){C} 
}
\end{pspicture*}
\end{center}

\item On calcule les longueurs AB, BC et AC.

\begin{list}{\textbullet}{}
\item $\text{AB} = \left | z_{\text B} - z_{\text A} \right | = \left | -2-2\text{i} -2+2\text{i} \right | = \left | -4 \right | = 4$
\item  $\text{BC} = \left | z_{\text C} - z_{\text B} \right | = \left | -4\text{i} +2+2\text{i} \right | = \left | 2-2\text{i} \right | = \ds\sqrt{2^2 + (-2)^2} = \ds\sqrt{8} = 2\sqrt{2}$ 
\item $\text{AC} = \left | z_{\text C} - z_{\text A} \right | = \left | -4\text{i} -2+2\text{i} \right | = \left | -2-2\text{i} \right | = \ds\sqrt{(-2)^2 + (-2)^2} = \ds\sqrt{8} = 2\sqrt{2}$ 
\end{list}

$\text{BC} = \text{AC}$ donc le triangle ABC est isocèle.

$\text{BC}^2 + \text{AC}^2 = 8 + 8 = 16 = 4^2 =  \text{AB}^2$ donc d'après la réciproque du théorème de Pythagore, le triangle ABC est rectangle en C.

Donc  le triangle ABC est isocèle rectangle en C.
\end{enumerate}

\bigskip

\textbf{Question 4}

\medskip

\begin{enumerate}
\item D'après le cours on sait que pour $a\neq 0$, les solutions de l'équation différentielle :

$y'+ay=b$ sont les fonctions $f$ définies sur $\R$ par: $f(t)=k \e^{-at} + \dfrac{b}{a}$ où $k\in\R$.

Les solutions de l'équation différentielle $y'+5y=7$ sont donc les fonctions $f$ définies sur $\R$ par: $f(t)=k \e^{-5t} + \dfrac{7}{5}$ où $k\in\R$.

\item $f(0) = 4 \iff k\e^{0}+\dfrac{7}{5}=4 \iff k= 4-\dfrac{7}{5} \iff k =\dfrac{13}{5}$

La solution cherchée est la fonction $f$ telle que: $f(t)=\dfrac{13}{5} \e^{-5t} + \dfrac{7}{5}$
\end{enumerate}

\bigskip

\textbf{Question 5}

\medskip

\begin{enumerate}
\item On admet que $g$ est dérivable sur l'intervalle $]0~;~+\infty[$, et on note $g'$ sa fonction dérivée. 

Pour tout $x>0$, $g'(x)=1\times \ln(x) + x\times \dfrac{1}{x}-1 - 0 = \ln(x) +1-1=\ln(x)$.

\item 
\begin{list}{\textbullet}{On étudie les variations de $g$ :}
\item Sur $]0~;~1[$, $\ln(x)<0$ donc $g'(x)<0$: la fonction $g$ est strictement décroissante.
\item Sur $]1~;~+\infty[$, $\ln(x)>0$ donc $g'(x)>0$: la fonction $g$ est strictement croissante.
\item Pour $x=1$, $\ln(x)=0$ donc $g'$ s'annule et passe de négative à positive: la fonction $g$ admet un minimum égal à $g(1)= 1\times\ln(1) - 1 + 4 = 3$.
\end{list}
\end{enumerate}

\bigskip

\textbf{Question 6}

\medskip

\begin{enumerate}
\item Limite de $h$ en $-\infty$.

$\left.
\begin{array}{l}
\ds\lim_{x\to -\infty} x^2 = +\infty\\
\ds\lim_{x\to -\infty} \e^{-x}=+\infty
\end{array}
\right \}
\text{ donc par produit, } \ds\lim_{x\to -\infty} h(x) = +\infty$

\item $\displaystyle\lim_{x \to + \infty}  h(x) = \ds\lim_{x \to + \infty} x^2\e^{-x} = \ds\lim_{x \to + \infty} \dfrac{x^2}{\e^{x}}$

D'après le cours, on sait que pour $n\geqslant 1$, on a $\ds\lim_{x \to + \infty} \dfrac{\e^{x}}{x^n}=+\infty$ donc $\ds\lim_{x \to + \infty} \dfrac{\e^{x}}{x^2}=+\infty$.

On en déduit que $\ds\lim_{x \to + \infty} \dfrac{x^2}{\e^{x}}=0$ et donc que $\ds\lim_{x \to + \infty} h(x)=0$.

\item Programme python complété :
\begin{center}
\fbox{
\begin{tabularx}{0.4\linewidth}{X}%\hline
from math import exp\\
~\\
def sol\_bal(n) \\
\qquad x = 2\\
\qquad while {\blue x**2*exp(-x)} > 0,5\\
\qquad\qquad x = {\blue x+10**(-n)}\\
\qquad return x\\ %\hline
\end{tabularx}}
\end{center}
\end{enumerate}

\bigskip


