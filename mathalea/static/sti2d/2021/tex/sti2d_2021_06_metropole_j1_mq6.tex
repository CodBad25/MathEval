
\medskip

\begin{center}
\fbox{
\begin{minipage}{0.9\textwidth}
\textbf{Rappel :} pour $a$ et $b$ deux réels, on a les formules suivantes :
\[\bullet~~\cos(a + b) = \cos (a) \cos(b) - \sin(a) \sin(b)\]
\[\bullet~~\cos(a - b) = \cos (a) \cos(b) + \sin(a) \sin(b)\]
\[\bullet~~\sin(a + b) = \sin(a) \cos(b) + \cos (a) \sin(b)\]
\[\bullet~~\sin(a - b) = \sin(a) \cos(b) - \cos (a) \sin(b)\]
\end{minipage}}
\end{center}

La tension $u$ aux bornes d'un générateur dépendant du temps $t$ est donnée par:
\[u(t) = 240 \cos (50t) - 240 \sin (50t).\]

La tension $u$ est exprimée en volt et le temps $t$ est exprimé en seconde.

\begin{enumerate}
\item Montrer que pour tout $t$ appartenant à $[0~;~+ \infty[$,
$u(t) = 240 \sqrt{2} \cos \left(50t + \dfrac{\pi}{4}\right)$

\item En déduire la fréquence $f= \dfrac{\omega}{2\pi}$, exprimée en Hz, délivrée par le générateur, où $\omega$ désigne la pulsation. 
On arrondira le résultat à l'unité.
\end{enumerate}

