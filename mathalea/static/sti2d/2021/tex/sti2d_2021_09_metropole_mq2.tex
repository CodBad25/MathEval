
\medskip

On considère la fonction $h$ définie sur $]0~;~+\infty[$ par $h(x) = \ln (2x + 1)$.

On désigne par $\mathcal{C}_h$ sa courbe représentative dans un repère orthonormé d'origine O et d'unité graphique $1$~cm.

On note $M(x~;~y)$ un point de la courbe $\mathcal{C}_h$. On suppose que l'ordonnée $y$ du point $M$
est supérieure à $15$~cm.

\smallskip

\begin{tabularx}{\linewidth}{|X|}\hline
Affirmation 2 :\\
\og L'abscisse $x$ du point $M$ se situe à plus de $16$~km du point O. \fg\\ \hline
\end{tabularx}

\bigskip

