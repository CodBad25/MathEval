
\medskip

On considère la fonction $h$ définie sur $\R$ par :
\[h(x) = x^2\text{e}^{-x}.\]

\begin{enumerate}
\item Calculer la limite de $h$ en $-\infty$.
\item Justifier que $\displaystyle\lim_{x \to + \infty}  h(x) = 0$.
\end{enumerate}

\smallskip

On admet que $h$ est strictement décroissante sur l'intervalle $[2~;~+ \infty[$ et que l'équation 

$h(x) = 0,5$ admet une unique solution dans l'intervalle $[2~;~+ \infty[$ que l'on note $\alpha$.

\begin{enumerate}[resume]
\item Recopier le programme ci-dessous et compléter les pointillés pour que la fonction
\texttt{sol\_bal} détermine une valeur approchée à $10^{-n}$ près de $\alpha$ par balayage.

\begin{center}
\begin{tabularx}{0.4\linewidth}{|X|}\hline
from math import exp\\
~\\
def sol\_bal(n) \\
\quad x = 2\\
\quad while \ldots > 0,5\\
\qquad x = \ldots\\
\quad return x\\ \hline
\end{tabularx}
\end{center}
\end{enumerate}

