
\medskip

\textbf{Dans cet exercice, les questions 1, 2, 3 et 4 sont indépendantes les unes des autres.}

\medskip

\textbf{Question 1}

\medskip

\textit{Pour cette question, indiquer la lettre de la réponse exacte, en expliquant votre choix.}

\begin{center}
\begin{pspicture}(-6,-6)(6,6)
\psgrid[subgriddiv=5,gridcolor=gray,subgridcolor=lightgray,gridlabels=0,gridwidth=1pt,subgridwidth=0.3pt,Dx=1,Dy=1](-6,-6)(6,6)
\psaxes[arrows=->,arrowsize=2pt 4,linewidth=1pt,Dx=1,Dy=1,ticksize=-3pt 3pt,subticks=0,labels=none](0,0)(-6,-6)(6,6)
\uput[dr](1,0){1}
\psline[linewidth=0.8pt](1,-0.1)(1,0.1)
\uput[ul](0,1){1}
\psline[linewidth=0.8pt](-0.1,1)(0.1,1)
\pscircle[linecolor=blue,linewidth=1pt](0,0){1}
\pscircle[linecolor=blue,linewidth=1pt](0,0){2}
\pscircle[linecolor=blue,linewidth=1pt](0,0){3}
\pscircle[linecolor=blue,linewidth=1pt](0,0){4}
\pscircle[linecolor=blue,linewidth=1pt](0,0){5}
\psdot[dotstyle=+,dotscale=2.5,linecolor=red](3.525,-3.525)
\uput[dr](3.535,-3.535){M}
\psline[linecolor=red, linewidth=1pt](0,0)(3.525,-3.525)
\end{pspicture}
\end{center}

On considère le point M représenté dans le plan complexe ci-dessus.

L'affixe du point M est :

\begin{center}
\begin{tabular}{|c|c|c|c|}
\hline
A & B & C & D \\
\hline
$4\e^{-\i\frac{\pi}{4}}$ & $5\e^{\i\frac{\pi}{4}}$ & $5\e^{-\i\frac{\pi}{4}}$ & $-5\e^{-\i\frac{\pi}{4}}$ \\
\hline
\end{tabular}
\end{center}

\bigskip

\textbf{Question 2}

\medskip

Soit l'équation différentielle $y' = 2y - 0,5$.

\begin{enumerate}
    \item Déterminer l'ensemble des fonctions définies sur $\mathbb{R}$ qui sont solutions de cette équation.
    \item Déterminer la fonction $f$, solution de cette équation, avec pour nombre dérivé \mbox{$f'(0) = -3$}.
\end{enumerate}

\bigskip

\textbf{Question 3}

\medskip

On considère la fonction $f$ définie sur $\mathbb{R}$ par $f(x) = \e^{-0,016x} - 2$.

Résoudre dans $\mathbb{R}$ l'équation $f(x) = 0$.

Donner la valeur exacte de la solution puis une valeur approchée à $10^{-2}$ près.

\bigskip

\textbf{Question 4}

\medskip

Montrer que, pour tout $x > 0$, l'égalité suivante est vraie :
\[\ln\left(\dfrac{x^4}{9}\right) - 3 \ln(x) + \ln\left(\dfrac{9}{x}\right) = 0.\]

\bigskip


