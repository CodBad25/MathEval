
\medskip

\begin{enumerate}
\item Par application de la première formule, on peut écrire :

\[u(t)= \ds\sqrt{3}\;\cos (t) - \sin (t) = 2\left(\dfrac{\sqrt{3}}{2}\;\cos (t) - \dfrac12\sin (t)\right),\]

soit puisque $\dfrac{\sqrt{3}}{2} = \cos \dfrac{\pi}{6}$ et $\dfrac{1}{2} = \sin \dfrac{\pi}{6}$,

\[u(t) = 2\left(\cos t \cos \dfrac{\pi}{6} - \sin t \sin \dfrac{\pi}{6} \right),\]

d'où par application de la troisième formule, on peut écrire :

\[u(t) = 2\cos \left(t + \dfrac{\pi}{6} \right).\]

\item $u(t)=1 \iff 2\cos \left(t + \dfrac{\pi}{6} \right) = 1 \iff \cos \left(t + \dfrac{\pi}{6} \right) = \dfrac 12 \iff \cos \left(t + \dfrac{\pi}{6} \right) = \cos \dfrac{\pi}{3},$

d'où deux possibilités :

$\left\{\begin{array}{l}
t + \dfrac{\pi}{6} = \dfrac{\pi}{3} + 2k \pi \\
\\
t + \dfrac{\pi}{6} = -\dfrac{\pi}{3} + 2k \pi
\end{array}\right. \iff \left\{\begin{array}{l}
t = \dfrac{\pi}{3} - \dfrac{\pi}{6} + 2k\pi \\
\\
t = \dfrac{\pi}{3} - \dfrac{\pi}{6} + 2k\pi \\
\end{array}\right. \iff
\left\{\begin{array}{l}
t = \dfrac{\pi}{6} + 2k\pi\\
\\
t = -\dfrac{\pi}{2} + 2k\pi
\end{array}\right.$

$S = \left\{\dfrac{\pi}{6}~;~-\dfrac{\pi}{2} \right\}$ à $2k\pi$ près.
\end{enumerate}

