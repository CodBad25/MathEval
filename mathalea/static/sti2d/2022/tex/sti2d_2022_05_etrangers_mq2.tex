
\medskip

Le plan est muni d'un repère orthonormé \Ouv. 

Soit A et B les points d'affixes respectives:

\[z_{\text{A}} = \e^{\text{i} \frac{5\pi}{6}} \text{ et } z_{\text{B}} = \e^{-\text{i} \frac{2\pi}{3}}.\]

\begin{enumerate}
\item Les points A et B sont correctement représentés sur l'une des figures ci-dessous.

Laquelle ? Aucune justification n'est attendue.

\begin{tabularx}{\linewidth}{X|X|X|X}
\psset{unit=1cm,arrowsize=2pt 2}
\def\xmin {-1.5}   \def\xmax {1.5}
\def\ymin {-2}   \def\ymax {1.5}
\begin{pspicture*}(\xmin,\ymin)(\xmax,\ymax)
%\psgrid[subgriddiv=1,  gridlabels=0, gridcolor=lightgray] 
%\psaxes[arrowsize=3pt 3, ticksize=-2pt 2pt, labels=none]{->}(0,0)(\xmin,\ymin)(\xmax,\ymax) 
\uput[dl](0,0){O} 
\uput[d](0.5,0){$\vec{u}$}  \uput[l](0,0.5){$\vec{v}$} 
\pscircle(0,0){1}
\psline{->}(-1,0)(1,0) \psline{->}(0,-1)(0,1) 
\psdots[dotstyle=x,dotscale=1.4,linecolor=blue](1;-30)(1;-120)
\uput[-30](1;-30){\blue A} \uput[-120](1;-120){\blue B} 
\uput[d](0,-1.5){\bf Figure 1}
\end{pspicture*}
&
\psset{unit=1cm,arrowsize=2pt 2}
\def\xmin {-1.5}   \def\xmax {1.5}
\def\ymin {-2}   \def\ymax {1.5}
\begin{pspicture*}(\xmin,\ymin)(\xmax,\ymax)
%\psgrid[subgriddiv=1,  gridlabels=0, gridcolor=lightgray] 
%\psaxes[arrowsize=3pt 3, ticksize=-2pt 2pt, labels=none]{->}(0,0)(\xmin,\ymin)(\xmax,\ymax) 
\uput[dl](0,0){O} 
\uput[d](0.5,0){$\vec{u}$}  \uput[l](0,0.5){$\vec{v}$} 
\pscircle(0,0){1}
\psline{->}(-1,0)(1,0) \psline{->}(0,-1)(0,1) 
\psdots[dotstyle=x,dotscale=1.4,linecolor=blue](1;150)(1;-120)
\uput[150](1;150){\blue A} \uput[-120](1;-120){\blue B} 
\uput[d](0,-1.5){\bf Figure 2}
\end{pspicture*}
&
\psset{unit=1cm,arrowsize=2pt 2}
\def\xmin {-1.5}   \def\xmax {1.5}
\def\ymin {-2}   \def\ymax {1.5}
\begin{pspicture*}(\xmin,\ymin)(\xmax,\ymax)
%\psgrid[subgriddiv=1,  gridlabels=0, gridcolor=lightgray] 
%\psaxes[arrowsize=3pt 3, ticksize=-2pt 2pt, labels=none]{->}(0,0)(\xmin,\ymin)(\xmax,\ymax) 
\uput[dl](0,0){O} 
\uput[d](0.5,0){$\vec{u}$}  \uput[l](0,0.5){$\vec{v}$} 
\pscircle(0,0){1}
\psline{->}(-1,0)(1,0) \psline{->}(0,-1)(0,1) 
\psdots[dotstyle=x,dotscale=1.4,linecolor=blue](1;-30)(1;120)
\uput[-30](1;-30){\blue A} \uput[120](1;120){\blue B} 
\uput[d](0,-1.5){\bf Figure 3}
\end{pspicture*}
&
\psset{unit=1cm,arrowsize=2pt 2}
\def\xmin {-1.5}   \def\xmax {1.5}
\def\ymin {-2}   \def\ymax {1.5}
\begin{pspicture*}(\xmin,\ymin)(\xmax,\ymax)
%\psgrid[subgriddiv=1,  gridlabels=0, gridcolor=lightgray] 
%\psaxes[arrowsize=3pt 3, ticksize=-2pt 2pt, labels=none]{->}(0,0)(\xmin,\ymin)(\xmax,\ymax) 
\uput[dl](0,0){O} 
\uput[d](0.5,0){$\vec{u}$}  \uput[l](0,0.5){$\vec{v}$} 
\pscircle(0,0){1}
\psline{->}(-1,0)(1,0) \psline{->}(0,-1)(0,1) 
\psdots[dotstyle=x,dotscale=1.4,linecolor=blue](1;150)(1;120)
\uput[150](1;150){\blue A} \uput[120](1;120){\blue B} 
\uput[d](0,-1.5){\bf Figure 4}
\end{pspicture*}
\end{tabularx}

\item Montrer qu'un argument de $\dfrac{z_{\text A}}{z_{\text B}}$ est $\dfrac{-\pi}{2}$.
\end{enumerate}

\bigskip

