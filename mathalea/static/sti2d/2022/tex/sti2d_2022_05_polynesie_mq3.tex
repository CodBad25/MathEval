
\medskip

Une voiture électrique, dont l'accumulateur est totalement déchargé, est branchée à une borne de rechargement. L'énergie emmagasinée par l'accumulateur (en kilowattheure), notée $E$, peut être modélisée en fonction du temps $t$ écoulé (en heure) par la fonction $E$ définie pour $t \in [0~;~ +\infty[$ par :

\[E(t) = 18\left(1 - \text{e}^{-0,45t}\right).\]

On admet que cette voiture a une énergie de stockage limitée à 18 kWh.

Déterminer l'instant $t_0$, arrondi à la minute, à partir duquel la moitié de cette énergie de stockage limite a été emmagasinée.

\bigskip

