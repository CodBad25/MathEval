
\medskip

\begin{enumerate}[start=4]
\item On a :
\begin{align*}
T(0) &= 37\times\mathrm e^{\frac{-20 \times 0}{459}} +26,4\\
&= 37\times\mathrm e^{0} +26,4\\
&= 37\times 1 +26,4\\
&= 63,4
\end{align*}
Lorsque l'on commence la prise de mesures, le lait a une température de $63,4 \degres C$ (ce qui est conforme aux données de l'énoncé).

\item Une propriété du cours dit que $\displaystyle \lim_{t \to +\infty} \mathrm e^{kt} = 0$ lorsque $k < 0$. Ici, $k = \dfrac {20} {459}$ ; ainsi, par produit et somme :
\[\lim_{t\to + \infty} T(t) = 37 \times 0 + 26,4 = 26,4.\]

On peut affirmer qu'alors, la température de l'air pièce est de $26,4 \degres C$ puisque, après un long temps passé, la température du lait se rapproche de celle du milieu ambiant.

\item On a :
\begin{align*}
&&T(t) &= 40\\
\iff&& 37\times\mathrm e^{\frac{-20 t}{459}} +26,4 &= 40\\
\iff&& 37\times\mathrm e^{\frac{-20 t}{459}} &= 40-26,4\\
\iff&& \mathrm e^{\frac{-20 t}{459}} &= \frac{13,6}{37}\\
\iff&& \frac{-20 t}{459} &= \ln\left(\frac{13,6}{37}\right)\\
\iff&& t &= \frac{\ln\left(\frac{13,6}{37}\right)}{\frac{-20}{459}}\\
&&&= - \frac {459}{20} \ln\left(\frac{13,6}{37}\right)\\
&&&\simeq 22,969.
\end{align*}
En convertissant $0,969$ minutes en seconde, on trouve que la température du lait atteint $40 \degres C$ au bout de 22 minutes et 58 secondes.
\end{enumerate}

\bigskip


