
\medskip

On considère la fonction $f$ définie sur $\R$ par $f(x) = a + b\text{e}^{x}$, où $a$ et $b$ sont deux nombres réels.

On considère la fonction $g$ définie sur $\R$ par$g(x) = x^2 - 4x - 1$.

On note $\mathcal{C}_f$ et $\mathcal{C}_g$ les courbes représentatives des fonctions $f$ et $g$, tracées dans le repère orthogonal ci-dessous.

\begin{center}
\psset{xunit=1cm,yunit=0.2cm}
\begin{pspicture*}(-4,-12)(3.2,16)
\psaxes[linewidth=1.25pt,labelFontSize=\scriptstyle,Dy=5]{->}(0,0)(-4,-12)(3.2,16)
\psplot[plotpoints=2000,linewidth=1.25pt,linecolor=red,linestyle=dashed]{-4}{3}{x dup mul x 4 mul sub 1 sub}
\psplot[plotpoints=2000,linewidth=1.25pt,linecolor=blue]{-4}{3}{3 2.71828 x exp 4 mul sub}
\uput[dl](0,-1){\small A}\uput[r](-2.25,15){\small \red $\mathcal{C}_g$}
\uput[u](-3.8,3){\small \blue $\mathcal{C}_f$}
\psplot[plotpoints=2000,linewidth=1pt]{-4}{3}{4 x mul 1 add neg}
\uput[d](-3.8,15){\small $T$}
\end{pspicture*}
\end{center}

\begin{enumerate}
\item On admet que les deux courbes $\mathcal{C}_f$ et $\mathcal{C}_g$ ont un unique point en commun, noté A d'abscisse 0.

Calculer $g(0)$, puis en déduire que $a + b= - 1$.

\item On admet que les deux courbes $\mathcal{C}_f$ et $\mathcal{C}_g$ ont la même tangente $T$ au point A.
	\begin{enumerate}
		\item Donner, pour tout réel $x$, une expression de $g'(x)$ puis calculer $g'(0)$.
		
		\item En déduire la valeur de $b$, puis celle de $a$.
	\end{enumerate}
\end{enumerate}

\bigskip

