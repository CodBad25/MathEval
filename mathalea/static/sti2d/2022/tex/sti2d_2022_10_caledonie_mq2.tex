
\medskip

\emph{Indiquer la lettre de la réponse exacte. Aucune justification n'est demandée.}

\medskip

Soit la fonction $\theta$ définie sur l'intervalle $[0~;~ +\infty[$ par :
$\theta(t) = 580 \text{e}^{-0,1t} + 20.$

\medskip

La pente de la tangente à la courbe représentative de la fonction $\theta$ au point d'abscisse 10 vaut :

\begin{center}
\renewcommand\arraystretch{2.1}
\begin{tabularx}{\linewidth}{*{2}{X}}
\textbf{a.~~}$-\dfrac{58}{\text{e}}$&\textbf{b.~~}$580\text{e}^{-1} + 20$\\
\textbf{c.~~}$- \dfrac{58}{\text{e}} + 20$&\textbf{d.~~}$\dfrac{580}{\text{e}}$
\end{tabularx}
\renewcommand\arraystretch{1}
\end{center}

\bigskip

