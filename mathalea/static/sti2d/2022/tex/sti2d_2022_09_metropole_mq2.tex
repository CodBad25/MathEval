
\medskip

Lors d'une course, on a mesuré la fréquence cardiaque d'un coureur de \np[m]{100}.

Cette fréquence cardiaque, en battements par minute, est modélisée par la fonction
$f$ définie sur [0; 100] par $f(x) = 28\,\ln (x + 1) + 70$ où $x$ est la distance parcourue, en mètre, depuis le départ de la course.
\begin{enumerate}
\item Selon ce modèle, quelle est la fréquence cardiaque de ce coureur au départ de la course ?
\item Selon ce modèle, au bout de combien de mètres la fréquence cardiaque de ce sportif est-elle égale à $185$ battements par minute ? Arrondir à l'unité.
\end{enumerate}

\bigskip

