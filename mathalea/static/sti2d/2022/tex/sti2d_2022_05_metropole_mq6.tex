
\medskip

\begin{center}
\fbox{
\begin{minipage}{0.9\textwidth}
\textbf{Rappel :} pour $a$ et $b$ deux réels, on a les formules suivantes :
\[\bullet~~\cos(a + b) = \cos (a) \cos(b) - \sin(a) \sin(b)\]
\[\bullet~~\cos(a - b) = \cos (a) \cos(b) + \sin(a) \sin(b)\]
\[\bullet~~\sin(a + b) = \sin(a) \cos(b) + \cos (a) \sin(b)\]
\[\bullet~~\sin(a - b) = \sin(a) \cos(b) - \cos (a) \sin(b)\]
\end{minipage}}
\end{center}

La tension $u$, exprimée en volt, aux bornes d'un dipôle en fonction du temps $t$, exprimé en seconde, est donnée par :
\[u(t) = \cos (50t) + \sqrt{3}\sin (50t).\]

\begin{enumerate}
\item Pour tout nombre réel $t$, écrire $u(t)$ sous la forme $u(t) = U_{\text{max}} \cos (\omega t + \varphi)$ où :

\setlength\parindent{1cm}
\begin{itemize}
\item[$\bullet~~$]$U_{\text{max}}$ représente la tension maximale (exprimée en volt) ;
\item[$\bullet~~$]$\omega$ représente la pulsation (exprimée en rad.s$^{-1}$) ;
\item[$\bullet~~$]$\varphi$ représente le déphasage (exprimé en rad).
\end{itemize}
\setlength\parindent{0cm}
\item En déduire la fréquence correspondante $f = \dfrac{\omega}{2\pi}$, exprimée en Hz. Arrondir le résultat à l'unité.
\end{enumerate}

