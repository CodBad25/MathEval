
\medskip

\begin{enumerate}
\item Sur $\R$, on a :

$f(x) = \sin(x) + \cos(x)$ ;

$f'(x) = \cos(x) - \sin(x)$ ;

$f''(x) = - \sin (x) - \cos (x)$.

Donc $f(x) + f''(x) = 0 \iff f$ est solution d' équation différentielle $y'' + y = 0$ sur $\R$.

\item Soit la fonction $g$ définie par : $g(x) = \sqrt{2}\cos\left(x-\frac{\pi}{4}\right)$.

D'après le formulaire :
\begin{align*}
g(x) &= \sqrt{2}\left(\cos (x) \cos \dfrac{\pi}{4} + \sin (x) \sin \dfrac{\pi}{4} \right) \\
&= \sqrt{2}\left( \cos (x) \times \dfrac{\sqrt{2}}{2} + \sin (x) \times \dfrac{\sqrt{2}}{2} \right) \\
&= \cos (x) \times \dfrac{2}{2} + \sin (x) \times \dfrac{2}{2} \\
&= \cos (x) + \sin (x) \\
&= f(x).
\end{align*}
On a donc pour tout $x \in \R, \: f(x) = \sqrt{2}\cos\left(x-\dfrac{\pi}{4}\right)$.
\end{enumerate}

