
\medskip

\begin{enumerate}
\item Soit $z$ le nombre complexe défini par $z = 1 + \text{i} \sqrt 3$.

Ce nombre complexe a pour module $|z| = \sqrt{1^2 + \bigl(\sqrt 3\bigr)^2} = \sqrt {1 + 3} = \sqrt 4 = 2$.

Par ailleurs, le nombre complexe :
\[ \dfrac z {|z|} = \frac{1 + \text{i} \sqrt 3} 2 = \dfrac 1 2 + \text{i} \dfrac{\sqrt 3} 2, \]
est de module $1$ et admet pour argument un nombre $\psi$ tel que $\cos \psi = \dfrac 1 2$ et $\sin \psi = \dfrac{\sqrt 3} 2$, soit $\psi = \dfrac \pi 3$ comme argument principal. Ce nombre $\psi$ est aussi un argument de $z$. 

Alors, il suffit de prendre $U_{\text{max}} = |z| = 2$, $\omega = 50$ et $\varphi = - \psi = - \dfrac \pi 3$ pour obtenir la bonne expression. En effet :
\begin{align*}
2\cos\left(50 t - \dfrac \pi 3\right) &= 2 \left(\cos(50t) \cos\left(\dfrac \pi 3\right) + \sin(50t) \sin\left(\dfrac \pi 3\right)\right)\\
&= 2 \left(\cos(50t) \times \dfrac 1 2 + \sin(50t) \times \dfrac {\sqrt 3} 2 \right)\\
&=\cos(50t) \times 1 + \sin(50t) \times \sqrt 3\\
&= U(t).
\end{align*}

\item Il suffit de calculer $f = \dfrac {50}{2\pi} = \dfrac {25} {\pi} \simeq 8$ Hz, une fois arrondi à l'unité.
\end{enumerate}

