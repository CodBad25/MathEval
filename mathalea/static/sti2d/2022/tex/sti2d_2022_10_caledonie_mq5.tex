
\medskip

\emph{Indiquer la lettre de la réponse exacte. Aucune justification n'est demandée.}

\medskip

Une pièce en acier, initialement à la température de $600\degres$ C, est mise à refroidir à l'air libre dans une pièce à $20\degres$ C.
La température $\theta$ (en degré Celsius) de l'acier en fonction du temps $t$ (en minute) peut être modélisée par la fonction définie sur l'intervalle $[0~;~ +\infty[$ par :
\[\theta(t) = 580 \text{e}^{-0,1t} + 20.\]

La pièce peut être manipulée lorsque sa température devient inférieure à $40~\degres$ C.

Pour déterminer la durée minimale d'attente, à compter de l'instant où elle est mise à refroidir ($t = 0$), on veut mettre en place un algorithme de balayage, écrit en langage Python.

\begin{center}
\begin{tabular}{l l}
1 &from math import exp \\
2&\\
3&def duree\_d\_attente ( ) :\\
4&\qquad t = 0\\
5&\qquad Temperature = 600\\
6&\qquad \dots\dots\dots\\
7&\qquad \qquad t = t + 1\\
8&\qquad \qquad Temperature = 580 * exp(- 0,1*t) + 20\\
9&\qquad return t
\end{tabular}
\end{center}

\medskip

Pour que la valeur renvoyée par la fonction \textbf{duree\_d\_attente} soit la valeur entière minimale de la durée d'attente, la ligne 6 contient :

\begin{center}
\begin{tabularx}{\linewidth}{*{2}{X}}
\textbf{a.~~} while $t > 40$ : &\textbf{b.~~} while Temperature $> 40$ :\\
\textbf{c.~~} while Temperature $< 40$ : &\textbf{d.~~} for i in range(Temperature) :
\end{tabularx}
\end{center}

\bigskip

