
\medskip

\begin{enumerate}
\item On écrit, en multipliant numérateur et dénominateur par $- \text{i}$ :
\[ z = \dfrac{(-1+ \text{i}) (-\text{i})}{(3\text{i})(-\text{i})}    = \dfrac{-(-\text{i}) - \text{i}^2}{-3 \text{i}^2}
= \dfrac {\text{i} + 1}{3} = \dfrac 1 3 + \dfrac 1 3 \text{i}.\]

\item Le module de $z$ est :
\[ |z| = \sqrt{\left(\dfrac 1 3\right)^2 + \left(\dfrac 1 3\right)^2}
= \sqrt{\dfrac 1 9 + \frac 1 9}
= \sqrt{\dfrac 2 9} 
= \dfrac{\sqrt 2} 3.\]

Soit $u = \dfrac z {|z|}$. Ce nombre complexe, de module $1$, a le même argument que $z$.

On a :
\begin{align*}
u &= \dfrac {\dfrac 1 3 + \dfrac 1 3 \text{i}}{\dfrac {\sqrt 2} 3}\\
&= \left(\dfrac 1 3 + \dfrac 1 3  \text{i} \right) \dfrac 3 {\sqrt 2}\\
&= \dfrac 1 3 \times \dfrac 3 {\sqrt 2} + \dfrac 1 3  \text{i} \times \dfrac 3 {\sqrt 2}\\
&= \dfrac 1 {\sqrt 2} + \dfrac 1 {\sqrt 2}  \text{i}\\
&= \dfrac {\sqrt 2} 2 + \dfrac {\sqrt 2} 2  \text{i}.
\end{align*}
Un argument de $u$ est donc $\theta$ tel que $\cos \theta = \dfrac {\sqrt 2} 2$ et $\sin \theta = \dfrac{\sqrt 2} 2$. On prend $\theta = \dfrac \pi 4$.

Ainsi, sous forme exponentielle, on a $z = \dfrac{\sqrt 2} 3 \mathrm e^{\frac \pi 4  \text{i}}$.
\end{enumerate}

\bigskip

