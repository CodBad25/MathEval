
\medskip

\begin{enumerate}
\item Substituons les valeurs de $\alpha = -0,1$ et $A = 20$ dans l'équation différentielle donnée, puis dévelopons :
\[\theta'(t) = -0,1 (\theta(t) - 20) = -0,1 \theta(t) + 2.\]

En comparant cette équation avec les propositions données, la bonne réponse est :  
\[\textbf{c.} \quad y' = -0,1 y + 2.\]

\item La pente est donnée par la dérivée de $\theta(t)$ évaluée en $t=10$.
\[\theta'(t) = -0,1 \times 580 \e^{-0,1t} = -58 \e^{-0,1t}.\]
\[\theta'(10) = -58 \e^{-0,1 \times 10} = -58 \e^{-1} = -\dfrac{58}{\e}.\]

La bonne réponse est \textbf{a.}

\medskip

\item Sur l'intervalle $[0~;~+ \infty[$, la fonction $\theta$ est :

\begin{itemize}
    \item Le terme $\e^{-0,1t}$ est de ma forme $\e^{\text{k}x}$ avec k < 0. Donc, par définition, ce terme décroit.
    \item Le terme \(580 \e^{-0,1t}\) est donc décroissant.
    \item En ajoutant une constante \(20\), cela n'affecte pas la décroissance.
\end{itemize}
La fonction $\theta(t)$ est donc décroissante. La bonne réponse est \textbf{b.}

\item 
\[\displaystyle \lim_{t\to +\infty}\theta(t)=\lim_{t\to +\infty} 580 \text{e}^{-0,1t} +\lim_{t\to +\infty}20=0+20=20.\]
La bonne réponse est \textbf{a.}

\item Pour que la valeur renvoyée par la fonction \textbf{duree\_d\_attente} soit la valeur entière minimale de la durée d'attente, la ligne 6 doit contenir :
\begin{center}
\begin{tabular}{l l}
1 &from math import exp \\
2&\\
3&def duree\_d\_attente ( ) :\\
4&\qquad t = 0\\
5&\qquad Temperature = 600\\
6&\qquad \textbf{while Temperature > 40}\\
7&\qquad \qquad t = t + 1\\
8&\qquad \qquad Temperature = 580 * exp(- 0,1*t) + 20\\
9&\qquad return t
\end{tabular}
\end{center}
La bonne réponse est \textbf{b.}

\item L'inéquation $\theta(t) \leqslant 40$, d'inconnue $t$, admet comme ensemble solution sur $[0~;~+\infty[$ :
\begin{align*}
580 \text{e}^{-0,1t} + 20& \leqslant 40\\
580 \text{e}^{-0,1t} &\leqslant 20\\
 \text{e}^{-0,1t}&\leqslant \dfrac{ 20}{580}\\
 \text{e}^{-0,1t} &\leqslant \dfrac{1}{29}\\
-0,1 t &\leqslant  \ln \left(\dfrac{1}{29}\right)\\
t&\geqslant -10\ln \left(\dfrac{1}{29}\right)\\
\end{align*}
La bonne réponse est \textbf{b.}
\end{enumerate}

\bigskip


