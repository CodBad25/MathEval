Indiquer, en justifiant, si les affirmations suivantes sont vraies ou fausses.

\centering \textit{Les questions 1 et 2 sont indépendantes.}

\begin{enumerate}
\item Afin de lutter contre le dopage dans le sport, un test a été mis en place.

En principe, ce test est POSITIF lorsque le sportif est dopé, et NÉGATIF lorsqu'il n'est pas dopé. \\
Toutefois, ce test peut commettre des erreurs : il peut être positif lorsque le sportif n'est pas dopé, et négatif lorsque le sportif est dopé.

Le tableau ci-dessous donne les résultats recueillis auprès de 200 coureurs ayant participé à un marathon.

\begin{center}
\begin{tabular}{|c|c|c|c|}
\hline
 & Coureur non dopé & Coureur dopé & Total \\
\hline
Test positif & 15 & 5 & 20 \\
\hline
Test négatif & 178 & 2 & 180 \\
\hline
Total & 193 & 7 & 200 \\
\hline
\end{tabular}
\end{center}

\begin{enumerate}
\item On choisit un coureur au hasard parmi les 200 coureurs testés.

\textbf{Affirmation 1 :} La probabilité que le coureur ne soit pas dopé ou soit testé positif est égale à $\dfrac{213}{200}$.

\item On choisit un coureur au hasard parmi ceux ayant eu un test positif.

\textbf{Affirmation 2 :} Il y a 75\% de chances que le coureur ne soit pas dopé.

\item On choisit un coureur au hasard parmi les 200 coureurs testés.

\textbf{Affirmation 3 :} La probabilité que le coureur soit concerné par une erreur de test est égale à 8,5\%.
\end{enumerate}

\item Au tennis, un SERVICE peut être réussi ou manqué. Une joueuse de tennis s'entraîne à faire des services. On admet que :
\begin{itemize}
\item la probabilité que son service soit réussi est égale à 0,9.
\item les services sont indépendants les uns des autres.
\end{itemize}

La joueuse fait deux services.

\textbf{Affirmation 4 :} La probabilité qu'exactement un service soit réussi sur les deux est égale à 0,09.
\end{enumerate}

