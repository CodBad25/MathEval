
\medskip

Victor sort un plat du four. La température du plat est alors égale à $180~\degres$C. II place ce plat dans une pièce dont la température est égale à $25~\degres$C. Le plat refroidit.

Le plat ne pourra être servi que lorsque sa température sera devenue inférieure ou égale à $40~\degres$C.

On étudie le refroidissement du plat selon deux modèles mathématiques.

\bigskip

\textbf{Partie A : Premier modèle}

\medskip

On suppose que la baisse de la température du plat est \textit{proportionnelle} à la durée du refroidissement, c'est-à-dire au nombre de minutes écoulées depuis la sortie du four.

On constate que 3 minutes après la sortie du four, la température du plat est égale à $105~\degres C$.

\begin{enumerate}
\item %De combien de degrés le plat a-t-il baissé en 3 minutes? En 1 minute ?
$180-105=75$ donc en 3 minutes, la température du plat a baissé de 75~$\degres$C.

La baisse de la température du plat est proportionnelle à la durée du refroidissement, donc en 1 minute, la température du plat a baissé de $\dfrac{75}{3}$, soit 25~$\degres$C.

\item %Vérifier que la température du plat, 5 minutes après la sortie du four, est égale à $55~\degres C$.
$5\times 25=125$ donc après 5 minutes, la température du plat a baissé de 125~$\degres$C.

$180-125=55$; après 5 minutes, elle est donc de 55~$\degres$C.

\item %Selon ce modèle, quelle serait la température du plat, 8 minutes après la sortie du four ? Ce premier modèle semble-t-il pertinent ?
$8\times 25=200$ donc après 8 minutes, la température du plat aurait baissé de 200~$\degres$C, ce qui n'est pas possible puisque la température de départ était de 180~$\degres$C.

Ce premier modèle ne semble pas pertinent.

\end{enumerate}


\textbf{Partie B : Second modèle}

\medskip

\begin{list}{\textbullet}{On dispose toujours des données suivantes:}
\item la température de la pièce est égale à $25~\degres$C.
\item la température du plat à la sortie du four est égale à $180~\degres$C.
\item la température du plat, 3 minutes après la sortie du four, est égale à $105~\degres$C.
\end{list}

Pour tout entier naturel $n$ on note $U_{n}$, la différence entre la température du plat et la température de la pièce, $n$ minutes après la sortie du four.

\emph{Exemple} : 3 minutes après la sortie du four, l'écart avec la température de la pièce est égal à $105 - 25 = 80$. On a donc $U_{3} = 80$.

\begin{enumerate}
\item %Justifier que $U_{0} = 155$.
À la sortie du four, pour $n=0$, le plat est à 180~$\degres$C et la pièce est à 25~$\degres$C.

$180-25=155$ donc $U_0=155$.

\item On suppose que chaque minute la différence $U_{n}$ diminue de $20\,\%$.

\begin{enumerate}
\item% Justifier que, pour tout entier naturel $n$, on a $U_{n+1}= 0,8 U_{n}$.
Diminuer de 20\,\%, c'est multiplier par $1-\dfrac{20}{100}$, soit $0,8$.

Donc, pour tout entier naturel $n$, on a $U_{n+1}= 0,8 U_{n}$.

\item %En déduire la nature de la suite $\left(U_{n}\right)$ et donner sa raison.
$U_0=155$ et, pour tout $n$, on a $U_{n+1}=0,8U_n$, donc la suite $\left (U_n\right )$ est géométrique de premier terme $U_0=155$ et de raison $q=0,8$.

\item%  Exprimer $U_{n}$ en fonction de $n$, pour tout entier naturel $n$.
On en déduit que, pour tout $n$, on a:
$U_n=U_0\times q^n=155\times 0,8^n$.

\item On dispose des données suivantes :

\begin{center}
\begin{tabular}{| m{1cm} |*{13}{>{\scriptsize} c|}}\hline
$n$ & 3 & 4 & 5 & 6 & 7 & 8 & 9 & 10 & 11 & 12 & 13 & 14 & 15 \\ \hline
$U_{n}$ \scriptsize arrondi à $10^{-1}$ & 80 & 64 & 51,2 & 41 & 32,8 & 26,2 & 21 & 16,8 & 13,4 & 10,7 & 8,6 & 6,9 & 5,5 \\
\hline
\end{tabular}
\end{center}

La température du plat doit être inférieure à 40~$\degres$C soit 15~$\degres$C au dessus de la température de la pièce. Il faut donc trouver la plus petite valeur de $n$ telle que $U_n<15$.

Victor pourra servir le plat au bout de 11 minutes.

%Au bout de combien de minutes, Victor pourra-t-il servir le plat ?
	\end{enumerate}
\end{enumerate}

\bigskip

