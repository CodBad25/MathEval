
\smallskip

\textbf{Question 1}

Le dimanche le temps passé à faire les devoirs est $\dfrac{25}{100} = \dfrac{25 \times 1}{25 \times 4} = \dfrac 14$ ; sur ce temps le temps consacré a l'exposé est de 80\,\% soir $\dfrac{80}{100} \times \dfrac 14 = \dfrac 14 \times \dfrac{80}{100}$ : réponse B

\textbf{Question 2}

Diminuer un prix de 50\,\% c'est le multiplier par $1 - \dfrac{50}{100} = 1 - 0,5 = 0,5$.

Pour retrouver le prix initial il faut doubler ce pris soit l'augmenter de $1 + 1 = 1 + \dfrac{100}{100}$, soit l'augmenter de 100\,\%.

\textbf{Question 3 }

Pour obtenir 200 à partir de 250, il suffit de multiplier par 200 et de diviser par 250, soit de le multiplier par $200 \times \dfrac{1}{250} = \dfrac{200}{250} = \dfrac{200 \times 4}{250 \times 4} = \dfrac{800}{\np{1000}} = 0,8$

\textbf{Question 4}

On a bien $\dfrac{10^{-5}}{10^{8}} = 10^{-5 - 8} = 10^{- 13}$.

\textbf{Question 5}

L'épaisseur d'une pile de 2 000 feuilles est égale à $\np{2000} \times 70 \times 10^{-3} = 2 \times 7 \times 10^{3 + 1 -3} = 14 \times 10^{1} = 140$~(mm) ou 14~(cm).

\textbf{Question 6}

Terre : $5,973 \times 10^{24}$~(kg) ; Mercure : $3,302 \times 10^{23}$~(kg) ;

Vénus : $\np{4,8685} \times 10^{24}$~(kg) ; Mars : $\np{6,4185} \times 10^{23}$~(kg).

La masse la plus grande est celle de la Terre.

\textbf{Question 7}

On a la somme : $x + 3x + x^2 = 4x + x^2$.

\textbf{Question 8}

La courbe $\mathcal{C}'$ est au dessous de la courbe $\mathcal{C}$ lorsque $x$ appartient à l'intervalle $[-2~;~-1]$ ou lorsque $x$ appartient à l'intervalle $[1~;~2]$.

\textbf{Question 9}

La courbe coupe l'axe des abscisses en deux points d'abscisses négatives.

\textbf{Question 10}

La fonction s'annule en $x = 2$ ce qui élimine B et D.

On a $f(0) > 0$, ce qui élimine C : reste A.

\textbf{Question 11}

On a $C = (1 + t)^2$, donc $C > 0$ et $\sqrt{C}$ existe : on a donc :

$1 + t = \sqrt{C}$ ou $1 + t = - \sqrt{C}$, d'où :

$t = \sqrt{C} - 1$ ou $t = - \sqrt{C} - 1$

\textbf{Question 12}

C'est l'année 2016.

\bigskip

\textbf{DEUXIÈME PARTIE : \hfill (6 pts)}

