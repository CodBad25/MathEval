Un village propose aux participants de la fête du sport deux épreuves : une randonnée et un cross. Il n'est pas possible de s'inscrire aux deux épreuves à la fois.

On dispose des informations suivantes :
\begin{itemize}
\item 90\% des participants ont choisi la randonnée, parmi eux, 5\% sont licenciés dans un club.
\item 10\% des participants ont choisi le cross, parmi eux, 40\% sont licenciés dans un club.
\end{itemize}

Un journaliste interroge un participant au hasard. \\
On considère les événements suivants :
\begin{itemize}
\item[] $E~:~\textit{\og Le participant a choisi la randonnée \fg}~;$
\item[] $L~:~\textit{\og Le participant est licencié dans un club \fg}~.$
\end{itemize}

\begin{enumerate}
\item Par simple lecture de l'énoncé, indiquer :
\begin{enumerate}
\item La probabilité que le participant interrogé soit licencié dans un club sachant qu'il a choisi la randonnée.
\item La probabilité que le participant interrogé soit licencié dans un club sachant qu'il a choisi le cross.
\end{enumerate}
\end{enumerate}

\textit{En prenant connaissance de ces deux probabilités, le journaliste estime que s'il choisit un participant parmi ceux qui sont licenciés dans un club, la probabilité qu'il ait effectué le cross sera largement supérieure à 50\%. L'objectif des questions suivantes est de vérifier si cette intuition est correcte.}

\begin{enumerate}[resume]
\item Représenter la situation par un arbre de probabilité.

\item 
\begin{enumerate}
\item Déterminer la probabilité que le participant interrogé ait choisi le cross et soit licencié dans un club.
\item Vérifier que la probabilité que le participant interrogé soit licencié dans un club est égale à $\dfrac{850}{10\,000}$ soit 8,5\%.
\end{enumerate}

\item Le journaliste interroge un participant licencié dans un club. Déterminer la probabilité que ce participant ait choisi le cross.

L'intuition du journaliste est-elle correcte ?
\end{enumerate}




