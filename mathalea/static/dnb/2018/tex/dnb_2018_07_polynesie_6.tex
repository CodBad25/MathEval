
\medskip

Voici un script saisi par Alice dans un logiciel d'algorithmique.

\medskip

\begin{center}
\begin{tabularx}{0.8\linewidth}{X}
	\begin{scratch}
		\blockinit{quand \greenflag est cliqué}
		\blocksensing{demander \ovalnum{Choisissez un nombre ?} et attendre}
		\blockevent{envoyer à tous \ovalvariable{le nombre a été saisi}}
		\blockvariable{mettre \selectmenu{Nombre} à \ovalmove{réponse}}
		\blockvariable{mettre \selectmenu{Résultat 1} à \ovaloperator{\ovaloperator{2 * \ovalvariable{Nombre}} + \ovalnum{3}}}
		\blockvariable{mettre \selectmenu{Résultat 1} à \ovaloperator{\ovalvariable{Résultat 1} * \ovalvariable{Résultat 1}}}
		\blocklook{dire \ovaloperator{regroupe \ovalnum{le résultat 1 est} et \ovalvariable{Résultat 1}} pendant \ovalnum{2} secondes}
		\end{scratch}\\[5pt]
		\begin{scratch}
		\blockinit{quand je reçois \ovalnum{le nombre a été saisi}}
		\blockvariable{mettre \selectmenu{Résultat 2} à \ovaloperator{\ovalvariable{Nombre} * \ovalvariable{Nombre}}}
		\blockvariable{mettre \selectmenu{Résultat 2} à \ovaloperator{\ovalvariable{Résultat 2} * \ovalnum{4}}}
		\blockvariable{mettre \selectmenu{Résultat 2} à \ovaloperator{\ovalvariable{Résultat 2} + \ovaloperator{\ovalnum{12} * \ovalvariable{Nombre}}}}
		\blockvariable{mettre \selectmenu{Résultat 2} à \ovaloperator{\ovalvariable{Résultat 2} + \ovalnum{9}}}
		\blockcontrol{attendre \ovalnum{3} secondes}
		\blocklook{dire \ovaloperator{regroupe \ovalnum{le résultat 2 est} et \ovalvariable{Résultat 2}}}
		\end{scratch}\\
\end{tabularx}
\end{center}

\begin{enumerate}
\item Alice a choisi 3 comme nombre, calculer les valeurs de Résultat 1 et de Résultat 2.

\emph{Justifier en faisant apparaître les calculs réalisés}.
\item Généralisation
	\begin{enumerate}
		\item En appelant $x$ le nombre choisi dans l'algorithme, donner une expression littérale
traduisant la première partie de l'algorithme correspondant à Résultat 1.
		\item  En appelant $x$ le nombre choisi dans l'algorithme, donner une expression littérale
traduisant la deuxième partie de l'algorithme correspondant à Résultat 2.
\item  Trouver le ou les nombres choisis par Alice qui correspondent au résultat affiché ci-dessous.

\begin{center}
\ovallook{Résultat 2 \ovalnum{9}}
\end{center}
 	\end{enumerate}
\end{enumerate}



