
\medskip

On dispose d'une urne A contenant 6 boules numérotées: 7~;~10 ~;~12~;~15~;~24~;~30 et d'une urne B contenant 9 boules numérotées: 2~;~ 5~;~6~;~8~;~17~;~18~;~21~;~22~;~25. Les boules sont indiscernables au toucher.

\medskip

\begin{enumerate}
\item Il y a 4 nombres pairs sur 6 nombres : la probabilité est donc égale à $\dfrac 46 = \dfrac 23$.
\item Les nombres premiers sont : 2~;~5~;~17 : la probabilité est donc égale à $\dfrac 39 = \dfrac 13$.
\item Dans l'urne A, $12 = 6 \times 2~;~24 = 6 \times 4$ et $30 = 6 \times 5$ sont des multiples de 6.

Dans l'urne B, $6 = 6 \times 1~;~18 = 6 \times 3$ sont des multiples de 6.

C'est donc l'urne A qui contient le plus grand nombre de multiples de 6.
\item Dans l'urne A il y a 2 nombres supérieurs ou égaux à 20  : la probabilité est égale à $\dfrac26 = \dfrac 13$.

Dans l'urne B, il y a 3 nombres supérieurs ou égaux à 20  : la probabilité est égale à $\dfrac39 = \dfrac 13$ :les deux probabilités sont égales.
\item Le tirage dans l'urne A a une probabilité de $\dfrac 37$ celui dans l'urne B aura une probabilité de $\dfrac{4}{10} = 0,4$.

Or $\dfrac 37 \approx 0,428$, les probabilités ne sont plus égales.
\end{enumerate}

\bigskip

