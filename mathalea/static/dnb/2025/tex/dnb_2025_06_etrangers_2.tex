
\medskip

L'entreprise \og Transport Rapide \fg{} doit livrer cinq colis nommés A, B, C, D et E ayant des masses différentes précisées dans le tableau ci-dessous:

\begin{center}
\begin{tabularx}{\linewidth}{|l|*{5}{>{\centering \arraybackslash}X|}}\hline
\textbf{Nom du colis}& A& B& C& D& E\\ \hline
\textbf{Masse en kg}& 4& 9& 2& 7& 11\\ \hline
\end{tabularx}
\end{center}

\smallskip

\begin{enumerate}
\item Calculer la moyenne des masses des colis en kg.
\item Déterminer la médiane des masses des colis en kg. Interpréter ce résultat.
\item Le transporteur choisit au hasard un colis parmi les cinq (A, B, C, D ou E) pour une livraison express.

Calculer la probabilité pour qu'il sélectionne un colis dont la masse est inférieure à $8$ kg.
\end{enumerate}

Les colis ont la forme d'un pavé droit de longueur $L$, de largeur $l$ et de hauteur $h$, représenté ci-dessous.

\begin{center}
\psset{unit=1cm,arrowsize=2pt 3}
\begin{pspicture}(11,4.4)
\psframe(3,0.7)(10.5,1.8)\psline(3,0.7)(0.8,2.8)(0.8,3.9)(3,1.8)%gauche
\psline(0.8,3.9)(8.3,3.9)(10.5,1.8)
\psline[linestyle=dashed](0.8,2.8)(8.3,2.8)(8.3,3.9)\psline[linestyle=dashed](8.3,2.8)(10.5,0.7)
\psline{<->}(3,0.4)(10.5,0.4)\uput[d](6.75,0.4){$L$}
\psline{<->}(0.6,2.8)(0.6,3.9)\uput[l](0.6,3.35){$h$}
\psline{<->}(0.8,2.6)(3,0.5)\uput[dl](1.9,1.55){$l$}
\end{pspicture}
\end{center}

Voici les dimensions des cinq colis.

\begin{center}
\begin{tabularx}{0.87\linewidth}{|c|*{3}{>{\centering \arraybackslash}X|}}\hline
Colis&Longueur L en mètre&Largeur $l$ en mètre&Hauteur h en mètre \\ \hline
A &0,4&0,3&0,5\\ \hline
B &0,5&0,4&0,8\\ \hline
C &0,3&0,1&0,5\\ \hline
D & 0,4 & 0,3 &0,7\\ \hline
E & 0,5&0,4&0,6\\ \hline
\end{tabularx}
\end{center}

\begin{enumerate}[resume]
\item
	\begin{enumerate}
		\item Vérifier que le volume du colis E est de $0,12$~m$^3$.
		\item L'entreprise souhaite calculer la masse volumique d'un colis dont la formule est rappelée
ci-dessous. Montrer que la masse volumique du colis E arrondie au dixième est $91,7$~kg/m$^3$.

On rappelle que la formule qui permet de calculer la masse volumique d'un objet en kg/m$^3$ est:

\begin{center} $\dfrac{\text{masse (en kg)}}{\text{volume (en m}^3)}$\end{center}

		\item  Le transporteur affirme \og Le colis E est plus lourd que le colis C, donc la masse volumique du colis E est plus grande que celle du colis C \fg. A-t-il raison ?
	\end{enumerate}
\end{enumerate}

\bigskip

