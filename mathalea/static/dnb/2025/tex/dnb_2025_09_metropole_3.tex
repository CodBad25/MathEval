
\medskip

\begin{minipage}{0.45\linewidth}
\begin{itemize}[label=$\bullet~$]
\item ABC un triangle rectangle en B ;
\item les points B, E et C sont alignés ainsi que
les points A, D, F et C ;
\item les droites (BD) et (EF) sont parallèles :
\item AB = 10 cm, BC = 7,5 cm, BE = 3 cm,

BD = 6 cm et CF = 2,7 cm.
\end{itemize}
\end{minipage}
\begin{minipage}{0.5\linewidth}
\psset{unit=0.9cm}
\begin{pspicture}(-0.5,-0.5)(8.5,6.5)
%\psgrid
\pspolygon(0,0)(7.8,0)(0,5.8)
\psline(2.78,3.75)
\psline(0,2.4)(1.63,4.6)
\uput[d](7.8,0){A}\uput[dl](0,0){B}\uput[ul](0,5.8){C}\uput[ur](2.78,3.75){D}
\uput[l](0,2.4){E}\uput[ur](1.63,4.6){F}
\psframe(0.3,0.3)
\end{pspicture}
\end{minipage}

\medskip

\begin{enumerate}
\item
	\begin{enumerate}
		\item Montrer que CE $= 4 ,5$ cm.
		\item Démontrer que la longueur EF est égale à $3,6$ cm.
	\end{enumerate}
\item Démontrer que le triangle CEF est rectangle en F.
\item
	\begin{enumerate}
		\item Calculer la mesure de l'angle $\widehat{\text{BCA}}$. Arrondir au degré.
		\item Les triangles ABC et CEF sont-ils semblables ?
	\end{enumerate}
\end{enumerate}

\bigskip

