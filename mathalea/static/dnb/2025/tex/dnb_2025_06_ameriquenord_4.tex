
À l'approche d'une course organisée par son collège, Malo s'entraîne sur un parcours de $13,5$~km.

La courbe ci-dessous représente la distance parcourue par Malo (en kilomètres) en fonction du temps écoulé (en minutes).

\begin{center}
\psset{xunit=0.125cm,yunit=0.6cm,arrowsize=2pt 3}
\begin{pspicture}(-5,-1)(95,16)
{\psset{linecolor=gray}
\multido{\n=0+5}{20}{\psline[linewidth=0.2pt](\n,0)(\n,16)}
\multido{\n=0+1}{17}{\psline[linewidth=0.2pt](0,\n)(95,\n)}
\multido{\n=0.5+1.0}{16}{\psline[linewidth=0.15pt](0,\n)(95,\n)}}
\psaxes[linewidth=1.25pt,Dx=10,labelFontSize=\scriptstyle]{->}(0,0)(0,0)(95,16)
\psline[linewidth=1.25pt,linecolor=magenta](0,0)(10,2)(20,4.5)(30,6.5)(40,6.5)(50,9)(60,11)(80,13.5)
\uput[u](84,0){Temps (en min)}\uput[r](0,15.75){Distance (en km)}
\end{pspicture}
\end{center}

\begin{enumerate}
\item Le temps et la distance parcourue par Malo sont-ils proportionnels ?
\item Quelle distance Malo a-t-il parcourue au bout de 20 minutes ?

Aucune justification n'est attendue.
\item Combien de temps a-t-il mis pour faire les 9 premiers kilomètres ?

Aucune justification n'est attendue.
\item Quelle est la vitesse moyenne de Malo lors de cette course? Exprimer le résultat au dixième de km/h près.
\item Louise et Hillal ont couru sur le même parcours de $13,5$ km. Louise à une vitesse régulière égale à $12$~km/h et Hillal a une vitesse régulière égale à $10$ km/h
	\begin{enumerate}
		\item Sachant que Louise et Hillal sont partis en même temps, qui a été le premier à
franchir la ligne d'arrivée?
		\item Quelle distance sépare Louise et Hillal, lorsque le premier des deux franchit la ligne d'arrivée ?
	\end{enumerate}
\end{enumerate}


