
À l'approche d'une course organisée par son collège, Malo s'entraîne sur un parcours de $13,5$~km.

%La courbe ci-dessous représente la distance parcourue par Malo (en kilomètres) en fonction du temps écoulé (en minutes).
%
%\begin{center}
%\psset{xunit=0.125cm,yunit=0.6cm,arrowsize=2pt 3}
%\begin{pspicture}(-5,-1)(95,16)
%\multido{\n=0+5}{20}{\psline[linewidth=0.2pt](\n,0)(\n,16)}
%\multido{\n=0+1}{17}{\psline[linewidth=0.2pt](0,\n)(95,\n)}
%\psaxes[linewidth=1.25pt,Dx=10,labelFontSize=\scriptstyle]{->}(0,0)(0,0)(95,16)
%\psline[linewidth=1.25pt,linecolor=magenta](0,0)(30,6.5)(40,6.5)(80,13.5)
%\uput[u](84,0){Temps (en min)}\uput[r](0,15.75){Distance (en km)}
%\end{pspicture}
%\end{center}

\begin{enumerate}
\item %Le temps et la distance parcourue par Malo sont-ils proportionnels ?
La représentation graphique de la distance parcourue en fonction du temps n'est pas un segment contenant l'origine : la distance parcourue par Malo n'est pas  proportionnelle au temps de course.
\item %Quelle distance Malo a-t-il parcourue au bout de 20 minutes ?
On lit sur la courbe qu'au bout de 20 minutes, Malo a parcouru 4,5 km.
%Aucune justification n'est attendue.
\item Combien de temps a-t-il mis pour faire les 9 premiers kilomètres ?
Malo a parcouru le 9 premiers kilomètres en 50 minutes.
%Aucune justification n'est attendue.
\item %Quelle est la vitesse moyenne de Malo lors de cette course ? Exprimer le résultat au dixième de km/h près.
Malo a parcouru les 13,5 km en 80 minutes :

$\bullet~~$Sans compter son arrêt de 10 minutes, sa vitesse moyenne a été de $v_1 = \dfrac{13,5}{\frac{70}{60}} = 13,5 \times \dfrac{60}{70} = \dfrac{81}{7} \approx 11,6$~(km/h) ;

$\bullet~~$Avec son arrêt de 10 minutes, sa vitesse moyenne a été de $v_2 = \dfrac{13,5}{\frac{80}{60}} = 13,5 \times \dfrac{60}{80} = \dfrac{81}{8} \approx 10,1$~(km/h) ;

\item %Louise et Hillal ont couru sur le même parcours de $13,5$ km. Louise à une vitesse régulière égale à $12$~km/h et Hillal a une vitesse régulière égale à $10$ km/h
	\begin{enumerate}
		\item %Sachant que Louise et Hillal sont partis en même temps, qui a été le premier à franchir la ligne d'arrivée?
Louise courant plus vite qu'Hillal est arrivée la première !
		\item %Quelle distance sépare Louise et Hillal, lorsque le premier des deux franchit la ligne d'arrivée ?
		Louise  a parcouru les 13,5 km à la vitesse de 12~km/h en un temps $t$ tel que 

$t = \dfrac{13,5}{12}$.

Au bout de ce temps Hillal a parcouru $10 \times \dfrac{13,5}{12} = \dfrac{135}{12} = 11,25$~(km).

Hillal est donc à ce moment à $13,5 - 11,25 = 2,25$~(km) de l'arrivée donc de Louise.
	\end{enumerate}
\end{enumerate}

