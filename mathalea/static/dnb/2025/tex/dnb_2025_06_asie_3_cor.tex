
\subsection*{Partie A}

\begin{enumerate}
	\item Le périmètre de EFGH vaut : \quad $4 \times 2x=4 \times  2 \times 1,5 = \np[cm]{12}$.

	\item On a :\quad $\mathrm{AB} = 16-2x=16 - 2 \times 1,5 = \np[cm]{13}$.

	\item $x=\np[cm]{1,5}=\mathrm{AD}$ et $\mathrm{AB}=\np[cm]{13}$.

	 On construit le rectangle en utilisant son équerre, les lignes de la copie et sa règle graduée.

	 \begin{tikzpicture}[x=1cm,y=1cm]
	 	\draw (0,0) rectangle (13,1.5);
	 \end{tikzpicture}

	\item \begin{multicols}{2}
		D'une part le périmètre de ABCD est :

		\hfill~$\begin{aligned}
			2 \times (\mathrm{AB} + \mathrm{AD})&= 2 \times (1,5+13)\\
			&= 2 \times 14,5\\
			&=\np[cm]{29}
		\end{aligned}$ \hfill~

		\columnbreak

		D'autre part le périmètre de EFGH est d'après la question \textbf{1.} :

		\hfill~
		$\begin{aligned}
			4 \times \mathrm{EF} & =	4 \times (2 \times 1,5)\\
			&= \np[cm]{12}\\
		\end{aligned}$\hfill~
	\end{multicols}

	Donc les périmètres de ABCD et de EFGH ne sont pas égaux quand $x$ vaut $\np[cm]{1,5}$.
\end{enumerate}

\subsection*{Partie B}
\begin{enumerate}
	\item \begin{enumerate}
		\item Le périmètre d'un carré, c'est quatre fois le côté du carré. Ici, le côté du carré, c'est $2x$, avec $x$ qui est renseigné dans la cellule de la ligne \textsf{1}.

		La formule en \textsf{B2} est donc : \quad\textsf{ = 4*2*A1} \quad ou bien \quad \textsf{ = 8*A1}.

		\item Non il n'y a aucune valeur de $x$ dans ce tableau pour laquelle les deux périmètres sont égaux. On trouve des périmètres proches pour $x = 3$, mais ils ne sont pas égaux.
	\end{enumerate}

	\item \begin{enumerate}
		\item Le périmètre du rectangle est donné par :

		$\begin{aligned}[t]
			2 \times (x + 16 - 2x) &= 2 \times (16-x)
			= 2 \times 16- 2 \times x
			=32-2x\\
			&=-2x+32		\end{aligned}$

		\item On veut donc résoudre l'équation suivante :

		\hfill~ $	\begin{aligned}
		\mathcal{P}_{\mathrm{ABCD}}&=\mathcal{P}_{\mathrm{EFGH}}\\
		-2x+32&=4 \times 2 \times x\\
		-2x+32&=8x\\
		-2x+32+2x&=8x+2x\\
		32&=10x\\
		32\div10&=10x \div10\\
		x&=3,2\\
	\end{aligned}$\hfill~

	La solution du problème est $\np[cm]{3,2}$.
	\end{enumerate}
\end{enumerate}


