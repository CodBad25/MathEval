
\medskip

Le dessus d'une table carrée, de côté 80 cm, est composé de quatre plaques rectangulaires en bois identiques et d'une plaque carrée en verre au centre. Chaque plaque en bois a pour longueur $60$~cm et pour largeur $20$~cm.

Voici la vue du dessus de la table :

\begin{center}
\psset{unit=1cm,arrowsize=2pt 3}
\begin{pspicture}(14,5)
%\psgrid
\psframe[fillstyle=solid,fillcolor=lightgray](4.5,0)(8.1,1.2)
\rput(6.3,0.6){1}
\psframe[fillstyle=solid,fillcolor=lightgray](8,0)(9.2,3.6)
\rput(8.6,1.8){2}
\psframe[fillstyle=solid,fillcolor=lightgray](9.2,3.6)(5.6,4.8)
\psframe[fillstyle=solid,fillcolor=lightgray](5.6,4.8)(4.5,1.2)
\rput(2.6,1.4){Plaque en bois}\rput(11,3.2){Plaque en verre}
\psline{->}(3.7,1.2)(5.5,0.6)
\psline{->}(9.8,3.2)(6.8,2.5)
\end{pspicture}
\end{center}

\begin{enumerate}
\item %Montrer que l'aire du dessus de la table est égale à \np{6400} cm$^2$.
L'aire d'un carré de côté $a$ est $a^2$, donc l'aire de la table est : $(60 + 20)^2 = 80^2 = \np{6400}~\left(\text{m}^2\right)$.
\item %Montrer que l'aire de la plaque en verre représente 25\,\% de l'aire totale du dessus de la table.
La mesure des côtés de la plaque en verre est égale à $60 - 20 = 40$~(cm). Son aire est donc égale à $40^2 = \np{1600}~\left(\text{m}^2\right)$.

Or $\dfrac{\np{1600}}{\np{6400}} = \dfrac{\np{1600}\times 1}{\np{1600}\times 4} = \dfrac14 = \dfrac{1 \times 25}{4 \times 25} = \dfrac{25}{100} = 25\,\%$.
\item %Quel est le nom de la transformation géométrique permettant de passer du rectangle \no 1 au rectangle \no 2 ? Aucune justification n'est demandée.
On passe de la plaque 1 à la plaque 2 par une rotation de 90 degrés dans le sens horaire.
\item %On souhaite réaliser un dessin du dessus de cette table avec le logiciel Scratch.

%Le lutin est orienté vers la droite.

%On a créé le bloc ci-dessous permettant de dessiner le rectangle \no 1 de la figure précédente, dans lequel 1 pas correspond à 1~cm.
	\begin{enumerate}
		\item %Recopier et compléter les lignes 3, 5 et 6 du bloc.
		
\begin{scratch}[num blocks]
\initmoreblocks{définir \namemoreblocks{Rectangle}}
\blockpen{stylo en position d'écriture}
\blockrepeat{répéter \ovalnum{2} fois}
{\blockmove{avancer de \ovalnum{60} pas}
\blockmove{tourner \turnleft{} de \ovalnum{ 90} degrés}
\blockmove{avancer de \ovalnum{ 20} pas}
\blockmove{tourner \turnleft{} de \ovalnum{90} degrés}
}
\blockpen{relever le stylo}
\end{scratch}
		\item %Parmi les trois programmes ci-dessous, lequel permet de tracer la vue du dessus de la table ?

%\begin{center}
%\psset{unit=1cm,arrowsize=2pt 3}
%\begin{pspicture}(14,5)
%\psframe[fillstyle=solid,fillcolor=lightgray](4.5,0)(8.1,1.2)
%\psframe[fillstyle=solid,fillcolor=lightgray](8,0)(9.2,3.6)
%\psframe[fillstyle=solid,fillcolor=lightgray](9.2,3.6)(5.6,4.8)
%\psframe[fillstyle=solid,fillcolor=lightgray](5.6,4.8)(4.5,1.2)
%\end{pspicture}
%\end{center}
%
%\medskip
%
%\begin{center}
%\begin{tabularx}{\linewidth}{|*{3}{>{\centering \arraybackslash}X|}}\hline
%Programme A&Programme B&Programme C\\ \hline
%\begin{scratch}[scale=0.8]
%\blockinit{Quand \greenflag est cliqué}
%\blockpen{effacer tout}
%\blockrepeat{répéter \ovalnum{4} fois}
%{\blockmoreblocks{Rectangle}
%\blockmove{tourner \turnleft{} de \ovalnum{90} degrés}
%}
%\end{scratch}
%&
%\begin{scratch}[scale=0.8]
%\blockinit{Quand \greenflag est cliqué}
%\blockpen{effacer tout}
%\blockrepeat{répéter \ovalnum{4} fois}
%{\blockmoreblocks{Rectangle}
%\blockmove{avancer de \ovalvariable{60} pas}
%\blockmove{tourner \turnleft{} de \ovalnum{90} degrés}
%}
%\end{scratch}
%&
%\begin{scratch}[scale=0.8]
%\blockinit{Quand \greenflag est cliqué}
%\blockpen{effacer tout}
%\blockrepeat{répéter \ovalnum{4} fois}
%{\blockmoreblocks{Rectangle}
%\blockmove{avancer de \ovalvariable{80} pas}
%\blockmove{tourner \turnleft{} de \ovalnum{90} degrés}
%}
%\end{scratch}
%\\ \hline
%\end{tabularx}
%\end{center}
C'est le programme C qui permet de dessiner la table.
	\end{enumerate}
\end{enumerate}
