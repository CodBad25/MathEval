
\medskip

%Cette année, les professeurs d'EPS proposent aux élèves un aquathlon (course à pied et natation).
%
%\medskip

\textbf{Partie A : La Course à pied}

\begin{enumerate}
\item On a AD $ = \text{AE} - \text{DE} = 250 - 50 = 200$~(m).

%Les points A, C, B sont alignés dans cet ordre et les points A, D, E sont alignés dans cet ordre

\item Dans le triangle ADC rectangle en A, le théorème de Pythagore permet d'écrire l'égalité :

DC$^2 = \text{DA}^2 + \text{AC}^2 = 200^2 + 480^2 = \np{40000} + \np{230400} = 520^2$.

Donc DC $ = 520$.
\item 
	\begin{enumerate}
		\item Si les droites (CD) et (BE) sont parallèles, les points A, C, B étant alignés dans cet ordre et les points A, D, E étant alignés dans cet ordre, on a une configuration de Thalès si en particulier on a l'égalité des rapports :

$\dfrac{\text{AC}}{\text{AD}}$ et $\dfrac{\text{AD}}{\text{AE}}$, soit d'une part $\dfrac{480}{480 + 120} = \dfrac{480}{600}$ et d'autre part $\dfrac{200}{250} = \dfrac 45$ ou encore en multipliant chaque terme par 12  $ = \dfrac{48}{60}$. Ces deux quotients sont de façon évidente égaux : les droites (CD) et (BE) sont donc parallèles d'après la réciproque du théorème de Thalès.
		\item On a par exemple dans le triangle ACD rectangle en A, $\tan \widehat{\text{ACD}} = \dfrac{\text{CA}}{\text{DA}} = \dfrac{200}{480} = \dfrac{20}{48} = \dfrac{5}{12}$.

La calculatrice donne $\widehat{\text{ACD}} \approx \approx 22,6~\degres$.

Conclusion : les droites (CD) et (BE) sont parallèles et l'angle $\widehat{\text{ACD}}$ a une mesure supérieure à $20\degres$, donc le parcours sera validé.
	\end{enumerate}
\end{enumerate}

\textbf{Partie B : La natation}

\begin{enumerate}
\item Il y a 9 temps rangés dans l'ordre croissant : comme $\dfrac{9-1}{2} = 4$, le 5\up{e} temps 6~min partage l'effectif des temps en deux séries de quatre temps : 4 inférieurs à 6 min et 4 supérieurs à 6 min : ce temps de 6~min est la médiane de la série.

\item %Un poisson rouge nage à la vitesse de 5 km/h. Nage-t-it plus vite que l'élève le plus rapide 
L'élève le plus rapide parcourt 200~m en 5~min 30 ou $5 \times 60 + 30 = 330~$s.

Sa vitesse est donc égale à $\dfrac{200}{330} = \dfrac{20}{33}$~(m/s) soit $\dfrac{20}{33} \times \np{3600}$~(m/h) soit environ \np{2181,8}~(m/h) et enfin environ 2,2~km/h. Le poisson rouge nage plus de deux fois plus vite que l'élève le plus rapide !
\end{enumerate}

\bigskip

