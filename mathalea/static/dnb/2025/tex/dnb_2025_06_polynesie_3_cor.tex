
\begin{enumerate}
\item \textbf{Bonne réponse :} 9, réponse D.

En effet, comme c'est $(-3)^2$, doit $(- 1) \times (- 1)$, le résultat est bien positif.

\item \textbf{Bonne réponse :} $2^3 \times 3^2 \times 5$, réponse D.

En effet, dans deux propositions, on a 9 et 8 qui ne sont pas des nombres premiers. Dans la troisième proposition fausse, le calcul ne donne pas 360 :

$2^3 \times 3^2 \times 7 = 504 \neq 360$.

Par contre : \quad $2^3 \times 3^2 \times 5 = 360$,\quad et les facteurs représentés sont 2, 3 et 5, qui sont bien premiers.

\item \textbf{Bonne réponse :} 45 cm, réponse B.

En effet, l'aire du rectangle est donnée par :\quad $\mathcal{A} = L  \times \ell$, \quad où $L$ est la longueur du rectangle et $\ell$ sa largeur.

En remplaçant les informations connues, on a :\quad $135 = L \times 3$

Donc :\quad $L = 135 \div 3 = \np[cm]{45}$.

\item {Bonne réponse :} $2x + 3$, réponse D.

En effet, les points D et E sont sur le segment [BG], et les longueurs BD et DE sont codées comme étant égales. On a donc :

$\mathrm{BG} = \mathrm{BD} + \mathrm{DE} + \mathrm{EG} = x + x + 3 = 2x + 3$.

\item \textbf{Bonne réponse :} KBOL, réponse C.

En effet, la translation qui transforme D en M transforme G en K, F en B, H en O et I en L.
\end{enumerate}

\bigskip

