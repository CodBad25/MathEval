
\medskip

\begin{enumerate}
\item La moyenne des masses est égale à : $\overline{m} = \dfrac{4 + 9 + 2 + 7 + 11}{5} = \dfrac{33}{5} = 6,6$~(kg).
\item Dans la liste des masses rangées dans l'ordre croissant 2~;~4~;~7~;~9~;~11, la troisième valeur 7 partage l'ensemble des masses en deux ensembles de même effectif : c'est donc la médiane.
\item Il y a 3 colis sur 5 qui ont une masse inférieure à 8 ; la probabilité est donc égale à $\dfrac
 35 = \dfrac{6}{10} = 0,6$.
\item
	\begin{enumerate}
		\item Volume du colis E : $0,5 \times 0,4 \times 0,6 = 0,2 \times 0,6 = 0,12$~m$^3$.
		\item masse volumique du colis E : $\dfrac{11}{0,12} = \dfrac{1100}{12} \approx 91,67$, soit environ 91,7~kg/m$^3$ au dixième près.
		\item Volume du colis C  : $0,3 \times 0,1 \times 0,5 = 0,03 \times 0,015$~m$^3$.

La masse volumique du colis C est égale à : $\dfrac{2}{0,015} = \dfrac{\np{2000}}{15} \approx 133,3$~kg/m$^3$. Donc le transporteur a tort.
	\end{enumerate}
\end{enumerate}

\bigskip

