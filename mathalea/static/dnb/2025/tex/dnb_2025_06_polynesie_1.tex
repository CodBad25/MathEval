
\medskip

L'association sportive d'un collège propose aux élèves une activité escalade.
La feuille de calcul ci-dessous obtenue à l'aide d'un tableur indique la répartition par âge des élèves inscrits à l'escalade.

\begin{center}
\begin{tabularx}{\linewidth}{|>{\cellcolor{gray!20}}c|c|*{7}{>{\centering \arraybackslash}X|}}\hline
\rowcolor[gray]{0.9}&A&B&C&D&E&F&G&H\\ \hline
1&Âge&10&11&12&13&14&15& Total\\ \hline
2& Effectif& 1& 3& 8& 12& 4& 2& \\ \hline
\end{tabularx}
\end{center}

\medskip

\begin{enumerate}
\item Quel est le nombre d'élèves âgés de 12 ans inscrits à l'escalade?
\item Calculer le nombre total d'élèves inscrits à l'escalade.
\item Quelle formule peut-on saisir dans la cellule H2 pour obtenir le nombre total d'élèves inscrits à l'escalade ?
\item Le professeur affirme : \og $\dfrac 15$ des élèves inscrits à l'escalade ont 14 ans ou plus \fg.

A-t-il raison ?

\item L'année dernière, la moyenne des âges des élèves inscrits à l'escalade était de 13 ans.

La moyenne des âges des élèves inscrits à l'escalade cette année a-t-elle augmenté par rapport à l'année dernière ?
\item L'association prévoit une hausse de 10\,\% des inscriptions à l'escalade l'année prochaine.

Déterminer le nombre d'élèves qui seront inscrits à l'escalade l'année prochaine.
\end{enumerate}

%\bigskip

