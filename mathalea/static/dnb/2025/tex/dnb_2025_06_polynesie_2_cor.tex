
\begin{enumerate}
\item Le point E est sur le segment [BD], donc on en déduit :

$\mathrm{BD} = \mathrm{BE} + \mathrm{ED} = 250 + 750 = \np{1000}$~(m).

\item Dans le triangle ABD, rectangle en A, on applique le théorème de Pythagore :

$\mathrm{AB}^2 + \mathrm{AD}^2 = \mathrm{BD}^2$

En remplaçant les grandeurs connues, on a : \quad $500^2 + \mathrm{AD}^2 = \np{1000}^2$

Soit : \quad $\mathrm{AD}^2 = \np{1000}^2 - 500^2 = \np{1000000} - \np{250000} = \np{750000}$.

Comme AD est une longueur, c'est un nombre positif, donc :

$\mathrm{AD} = \sqrt{\np{750000}} \approx 866,03$

En arrondissant au mètre près, on a donc bien AD environ égale à \np[m]{866}.

\item 
	\begin{enumerate}
		\item Dans le triangle EAB, rectangle en E, le côté [AB] est l'hypoténuse du triangle et le côté [EB] est le côté opposé à l'angle $\widehat{\mathrm{EAB}}$.

On a donc : \quad $\sin\left(\widehat{\mathrm{EAB}}\right) = \dfrac{\mathrm{EB}}{\mathrm{AB}}$.

On connaît les deux longueurs, donc, on a :\quad
$\sin\left(\widehat{\mathrm{EAB}}\right) = \dfrac{250}{500} = \dfrac{1}{2}$.

	\item On a donc :\quad$\widehat{\mathrm{EAB}} = \arcsin\left(\dfrac{1}{2}\right) = 30$\degres.
		\end{enumerate}
\item 
	\begin{enumerate}
		\item D'après le codage de la figure, les droites (AB) et (CD) sont perpendiculaires à la même droite (AD). Par propriété, elles sont donc parallèles.

		\item On sait que : les points B, E et D sont alignés, dans cet ordre, et que les points A, E et C dans le même ordre, car les segments [AC] et [DB] se coupent en E.

On sait également que les droites (AB) et (CD) sont parallèles.

Dans cette configuration, le théorème de Thalès permet de dire que les fractions suivantes sont égales :
\quad $\dfrac{\mathrm{EB}}{\mathrm{ED}}=\dfrac{\mathrm{EA}}{\mathrm{EC}}=\dfrac{\mathrm{AB}}{\mathrm{DC}}$.

Notamment : \quad $\dfrac{\mathrm{EB}}{\mathrm{ED}}=\dfrac{\mathrm{AB}}{\mathrm{DC}}$.

Soit, en remplaçant les longueurs connues : \quad $\dfrac{\mathrm{250}}{\mathrm{750}}=\dfrac{\mathrm{500}}{\mathrm{DC}}$.

D'où, par un produit en croix : \quad $\mathrm{DC} = 500\times \dfrac{750}{250} = 500 \times 3 = \np{1500}$~(m).
	\end{enumerate}
\item Si le piéton fait le tour du jardin botanique, la distance $d$ qu'il va parcourir, c'est le périmètre du jardin, soit :

$d = \mathrm{AB} + \mathrm{BC} + \mathrm{CD} + \mathrm{DA} \approx 500 + \np{1323} + \np{1500} + 866 = \np{4189}$~(m).

Puisque la vitesse moyenne du piéton est de 1,1~(m), cela signifie qu'il lui faudra :

$\dfrac{\np{4189}}{1,1} \approx \np[s]{3808}$.

Or, une heure, c'est 60 minutes, soit $60 \times 60 = \np{3600}$~(secondes).

$\np{3808} > \np{3600}$, donc il faudra plus d'une heure au piéton pour faire le tour du jardin botanique : le temps est supérieur à une heure.

\emph{Remarque :} on peut aussi convertir : \np{3808} secondes, c'est 1 heure, 3 minutes et 28 secondes, donc supérieur à une heure.
\end{enumerate}

\bigskip

