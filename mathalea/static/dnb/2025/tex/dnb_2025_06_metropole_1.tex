
On dispose d'une urne A contenant 6 boules numérotées : \quad 7~;~10~;~12~;~15~;~24~;~30

et d'une urne B contenant 9 boules numérotées :\quad  2~;~5~;~6~;~8~;~17~;~18~;~21~;~22~;~25.

Les boules sont indiscernables au toucher.

\medskip

	\begin{enumerate}
		\item On tire une boule dans l'urne A, quelle est la probabilité d'obtenir un nombre pair ?
		\item On tire une boule dans l'urne B, justifier que la probabilité d'obtenir un nombre premier est de $\dfrac{1}{3}$.
		\item Quelle urne contient le plus grand nombre de boules dont le numéro est un multiple de 6?
		\item On tire une boule au hasard dans l'une des urnes. Démontrer que la probabilité d'obtenir un nombre supérieur ou égal à 20 est la même quelle que soit l'urne choisie ?
		\item En repartant avec la composition initiale des urnes A et B on décide d'ajouter une boule numérotée 50 dans chacune d'entre elles. Dans ces conditions, la probabilité d'obtenir un résultat supérieur ou égal à 20 est-t-elle toujours égale quelle que soit l'urne choisie?
	\end{enumerate}

\bigskip

