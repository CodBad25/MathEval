\begin{minipage}[]{0.52\linewidth}

Dans la figure ci-contre qui n'est pas représentée en vraie grandeur:

\begin{itemize}[label=$\bullet$]
\item Les points G, C et E sont alignés ; 
\item Les points F{}, C et D sont alignés ;
\item Les droites (GF) et (DE) sont parallèles.
\item Le triangle CDE est rectangle en D
\item CD = \np[cm]{21,6}, CE = \np[cm]{29,1}, FC = \np[cm]{17,2}.
\end{itemize}
\end{minipage}
\begin{minipage}[]{0.46\linewidth}
\psset{unit=0.8cm}
\begin{pspicture}(-1,-4.5)(7.4,3.2)
\psline(0,0)(7.4,0)
\psline(0,0)(0,3.2)\psline(7.4,-4)(7.4,0)\psline(0,3.2)(7.4,-4) \psframe(7.4,0)(7.2,-0.2)
\uput[d](0,0){F}\uput[ul](0,3.2){G}\uput[ur](7.4,0){D}\uput[r](7.4,-4){E}\uput[dl](3.4,0){C}
\end{pspicture}
\end{minipage}

\begin{enumerate}
\item Montrer que la longueur DE est égale à \np[cm]{19,5}.
\item Calculer l'aire du triangle CDE.
\item Calculer la longueur GF arrondie au millimètre près.
\item On trace une droite (d) perpendiculaire à (FC) avec un logiciel de géométrie dynamique. La droite (d) coupe 
le segment [GC] en A et le segment [FC] en B. En affichant l'aire du triangle ABC à l'aide du logiciel, on obtient $\np[cm^2]{23,4}$.

\begin{center}
\begin{pspicture}(-1,-4)(7.4,3.4)
%\psgrid
\psline(0,0)(7.4,0)
\psline(0,0)(0,3.2)\psline(7.4,-4)(7.4,0)\psline(0,3.2)(7.4,-4)
\uput[d](0,0){F}\uput[ul](0,3.2){G}\uput[ur](7.4,0){D}\uput[r](7.4,-4){E}\uput[ur](3.4,0){C}
\psframe(2,0)(2.2,0.2)
\psline(2,-4)(2,3)\uput[dr](2,2.8){d}\uput[ur](3,2){$\text{Aire}_{\text{ABC}}=\np[cm^2]{23.4}$} \uput[dl](2,0){B}\uput[ur](2,1.25){A}
\end{pspicture}
\end{center}

	\begin{enumerate}
		\item Montrer que l'aire du triangle ABC est égale à $\dfrac{1}{9}$ de l'aire du triangle CDE.
		\item On admet que les triangles ABC et EDC sont semblables.

Déterminer la longueur AB.
	\end{enumerate}
\end{enumerate}

\vspace{0.5cm}

