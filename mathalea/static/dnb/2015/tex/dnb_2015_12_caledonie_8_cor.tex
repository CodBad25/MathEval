
\medskip

%Un site internet propose de télécharger légalement des clips vidéos. Pour cela, sur la page d'accueil, trois choix s'offrent à nous:
%
%\setlength\parindent{8mm}
%\begin{itemize}
%\item[$\bullet~~$] Premier choix : téléchargement \textbf{direct sans inscription}. Avec ce mode, chaque clip peut être téléchargé pour 4~euros.
%\item[$\bullet~~$] Deuxième choix: téléchargement \textbf{membre}. Ce mode nécessite une inscription à 10~euros.
%valable un mois et permet d'acheter par la suite chaque clip pour 2 euros.
%\item[$\bullet~~$] Troisième choix : téléchargement \textbf{premium}. Une inscription à 50~euros permettant de télécharger tous les clips gratuitement pendant un mois.
%\end{itemize}
%\setlength\parindent{0mm}
%
%\medskip

\begin{enumerate}
\item %Je viens pour la première fois sur ce site et je souhaite télécharger un seul clip.

%Quel est le choix le moins cher ?
Pour télécharger un seul titre le moins cher est le direct sans inscription : 4~\euro.
\item 
	\begin{enumerate}
		\item %Compléter le tableau.

\bigskip

\begin{tabularx}{\linewidth}{|m{3cm}|*{5}{>{\centering \arraybackslash}X|}}\hline
Nombre de clips &1 &2 &5 &10 &15\\ \hline
Prix en euros pour le téléchargement direct&4 &8&20&40&60\\ \hline
Prix en euros pour le téléchargement membre&12 &14&20&30&40\\ \hline
Prix en euros pour le téléchargement premium&50 &50&50&50&50\\ \hline
\end{tabularx}

\bigskip
		\item %À partir de combien de clips devient-il intéressant de s'inscrire en tant que membre ?
		Le tableau montre que pour 5 téléchargements les deux premières possibilités coûtent 20~\euro. Donc à partir de $x = 6$, il devient intéressant de prendre l'abonnement membre
	\end{enumerate}
\item Dans cette question, $x$ désigne le nombre de clips vidéos achetés.
	
%$f,\: g$ et $h$ sont trois fonctions définies par :
	
%\setlength\parindent{8mm}
%\begin{itemize}
%\item[$\bullet~~$]$f(x) = 50$
%\item[$\bullet~~$]$g(x) = 4x$
%\item[$\bullet~~$]$h(x) = 2x + 10$
%\end{itemize}
%\setlength\parindent{0mm}

	\begin{enumerate}
		\item %Associer chacune de ces fonctions au choix qu'elle représente (direct, membre ou premium).
$f$ correspond à l'abonnement premium.

$g$ correspond au téléchargement direct sans abonnement.

$h$ correspond à l'abonnement membre.
		\item %Dans le repère, tracer les droites représentant les fonctions $f,\: g$ et $h$.

\bigskip

\psset{xunit=0.5cm,yunit=0.1cm}
\begin{pspicture}(-1,-10)(24,60)
\multido{\n=0+1}{25}{\psline[linewidth=0.2pt](\n,0)(\n,60)}
\multido{\n=0+5}{13}{\psline[linewidth=0.2pt](0,\n)(24,\n)}
\psaxes[linewidth=1.25pt,Dy=100]{->}(0,0)(24,60)
\multido{\n=0+5}{13}{\uput[l](0,\n){\n}}
\uput[d](20.5,-4){Nombre de clips achetés}
\uput[r](0,57.5){Prix en euros}
\psline[linewidth=1.5pt](0,50)(24,50)\uput[u](3,50){$f$}
\psline[linewidth=1.5pt](15,60)\uput[ul](14,56){$g$}
\psplot[linewidth=1.5pt]{0}{24}{2 x mul 10 add}\uput[ul](23,57){$h$}
\end{pspicture}

\bigskip

		\item %À l'aide du graphique, déterminer le nombre de clips à partir duquel l'offre premium devient la moins chère.
		Pour 20 téléchargement les deux abonnements reviennent au même prix. À partir de 21 téléchargements l'abonnement premium est la solution la moins onéreuse.
	\end{enumerate}
\end{enumerate}

\vspace{0,5cm}

