
\medskip

Pour se promener le long d'un canal, deux sociétés proposent une location de bateaux électriques.

Les bateaux se louent pour un nombre entier d'heures.

\medskip

\begin{enumerate}
\item \textbf{Étude du tarif proposé par la société A}

\medskip

Pour la société A, le prix à payer en fonction de la durée de location en heure est donné par ce graphique.

\begin{center}
\textbf{Prix payé pour la location d'un bateau en fonction de la durée de la location}

\medskip

\psset{xunit=1.4cm,yunit=0.028cm,arrowsize=2pt 3}
\begin{pspicture}(-0.5,-20)(8.25,295)
\multido{\n=0.0+0.5}{17}{\psline[linewidth=0.6pt](\n,0)(\n,280)}
\multido{\n=0+10}{29}{\psline[linewidth=0.6pt](0,\n)(8.25,\n)}
\psaxes[linewidth=1.25pt,Dy=50]{->}(0,0)(0,0)(8.25,280)
\psplot[linewidth=1.25pt,linecolor=blue,plotpoints=500]{0}{8.25}{30 x mul}
\uput[d](6.5,-15){Durée de location (en heures)}
\uput[r](0,290){Prix payé (en \euro)}
\rput{31}(8,250){Société A}
\end{pspicture}
\end{center}

Répondre aux questions ci-dessous à l'aide du graphique.

Aucune justification n'est attendue pour les questions a. et b.
	\begin{enumerate}
		\item Quel prix va-t-on payer en louant un bateau pour 2 heures ?
		\item On dispose d'un budget de $100$~\euro, combien d'heures entières peut-on louer un bateau ?
		\item Expliquer pourquoi le prix est proportionnel à la durée de location.
		\item En déduire à l'aide d'un calcul, le prix à payer pour une durée de location de $10$~heures.
	\end{enumerate}
\item \textbf{Étude du tarif proposé par la société B}

\medskip

La société B propose le tarif suivant : $60$~\euro{} de frais de dossier plus $15$~\euro{} par heure de location.
	\begin{enumerate}
		\item Montrer qu'en louant un bateau pour une durée de 2 heures, le prix à payer sera de $90$~\euro.
		\item On désigne par $x$ le nombre d'heures de location. On appelle $f$ la fonction qui, au nombre d'heures de location, associe le prix, en euro, avec le tarif proposé par la société B.

On admet que $f$ est définie par : $f(x) = 15x + 60$.

Sur le graphique, tracer la courbe représentative de la fonction $f$.
		\item Le prix payé est-il proportionnel à la durée de location ?
	\end{enumerate}
\item \textbf{Comparaison des deux tarifs}
	\begin{enumerate}
		\item On souhaite louer un bateau pour une durée de 3 heures.

Quelle société doit-on choisir pour avoir le tarif le moins cher ?

Quel prix va-t-on payer dans ce cas ?
		\item Pour quelle durée de location le prix payé est-il identique pour les deux sociétés ?
	\end{enumerate}
\end{enumerate}



