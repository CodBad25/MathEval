
\medskip

On dispose d'une roue dont les 4 secteurs ont tous la même aire et sont numérotés : 

1 ; 2 ; 3 ;~4.

On dispose également d'une urne contenant 3 boules numérotées: 2 ; 3 et 4.

Les boules sont indiscernables au toucher.

On considère l'expérience aléatoire suivante : 

\og On fait tourner la roue puis on tire au hasard une boule dans l'urne. On forme alors un nombre entier à deux chiffres tel que :

\begin{itemize}
\item[$\bullet~~$]Le chiffre des dizaines est le numéro indiqué par la flèche sur la roue.
\item[$\bullet~~$]Le chiffre des unités est le numéro de la boule tirée dans l'urne. \fg
\end{itemize}

\psset{unit=1cm,arrowsize=2pt 3}
\begin{center}
\begin{pspicture}(13.5,4)
%\psgrid
\pscircle(4.5,2.2){1.3}
\rput(3.9,3){1} \rput(3.9,3){1} \rput(5.2,3){4} \rput(5.2,1.5){3} \rput(3.9,1.5){2}
\rput(4.5,0.4){La roue: chiffre des dizaines}
\psline[linewidth=2.5pt]{->}(3,4)(3.7,3.22)
\pspolygon(10,0.8)(12.6,0.8)(12.1,3.3)(10.5,3.3)
\rput(11.4,0.4){L'urne: chiffre des unités}
\pscircle(10.8,1.6){0.4} \pscircle(11.8,1.6){0.4}\pscircle(11.4,2.6){0.4}
\rput(10.8,1.6){2} \rput(11.8,1.6){4} \rput(11.4,2.6){3}
\end{pspicture}
\end{center}

\emph{Exemple} : Si la flèche indique le numéro 1 sur la roue et que la boule tirée dans l'urne porte le numéro 3, on forme le nombre 13.

\medskip

\begin{enumerate}
\item Écrire la liste des 12 issues possibles.
\item Déterminer la probabilité de l'évènement: \og Obtenir un nombre impair \fg.
\item On considère l'évènement $A$ : \og Le nombre formé est un nombre premier et inférieur à 30 \fg.
	\begin{enumerate}
		\item Quelle est la probabilité de l'évènement $A$ ?
		\item Quelle est la probabilité de son évènement contraire ?
	\end{enumerate}
\end{enumerate}

À l'aide de cette expérience aléatoire, on crée un jeu de hasard.

Le joueur gagne s'il obtient un multiple de 11.

\begin{enumerate}[resume]
\item Montrer que la probabilité d'obtenir un multiple de 11 est égale à $0,25$.
\item On souhaite simuler ce jeu à l'aide d'un logiciel de programmation.

On a rédigé le script ci-dessous:

\begin{center}
\setscratch{scale=.8}
\begin{scratch}[num blocks]
\blockinit{quand \greenflag est cliqué}
\blockvariable{mettre \ovalcontrol*{Gagné} à \ovalnum{0}}
\blockrepeat{répéter \ovalnum{100} fois}
{%% début repeat
\blockvariable{mettre \ovalcontrol*{Chiffre des dizaines} à \ovaloperator{nombre aléatoire entre \ovalnum{1} et \ovalnum{4}}}
\blockvariable{mettre \ovalcontrol*{Chiffre des unités} à \ovaloperator{nombre aléatoire entre \ovalnum{...} et \ovalnum{...}}}
\blockif{si \ovalnum{....................} = \ovalnum{....................} alors}
{% début if
\blockvariable{ajouter \ovalnum{1} à \ovalnum{Gagné}}
}% fin if
}% fin repeat
\blockstop{dire  regrouper \ovalnum{La fréquence d'apparition d'un multiple de 11 est de :} et \ovalcontrol{Gagné} / \ovalnum{100} pendant \ovalnum{2} secondes}
\end{scratch}
\end{center}

Information:

\ovaloperator{nombre aléatoire entre \ovalnum{1} et \ovalnum{4}} renvoie au hasard un nombre parmi 1, 2, 3, 4.
	\begin{enumerate}
		\item Écrire sur la copie comment compléter les deux cases vides de la ligne 5.

\emph{Ne pas justifier}.
		\item Écrire sur la copie comment compléter les deux cases vides de la ligne 6.

\emph{Ne pas justifier}.
		\item On a cliqué sur le drapeau et voici le résultat du programme :

\og La fréquence d'apparition d'un multiple de $11$ est $0,23$. \fg

Pourquoi le résultat est-il différent de celui obtenu dans la question 4 ?
	\end{enumerate}
\end{enumerate}
