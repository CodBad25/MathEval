
\medskip

\textbf{Pour chacune des affirmations, indiquer si elle est vraie ou fausse en justifiant la réponse.}

\medskip

\begin{enumerate}
\item On considère le tableau ci-dessous :
\begin{center}
\begin{tabularx}{0.8\linewidth}{|l|*{4}{>{\centering \arraybackslash}X|}}\hline
Nombre de baguettes &1 &2 &3 &4 \\ \hline
Prix en \euro{}& 1,10& 2,20& 3,30& 4\\ \hline
\end{tabularx}
\end{center}

\textbf{Affirmation 1 :} \og Le prix est proportionnel au nombre de baguettes. \fg

\item On considère ci-dessous le point A sur une droite graduée:

\begin{center}
\psset{xunit=5cm,arrowsize=2pt 3}
\begin{pspicture}(0.75,-0.25)(2.75,0.25)
\psaxes[linewidth=2pt,ticks=x,subticks=8,xsubticksize=2,Ox=1](1,0)(0.75,0.)(2.5,0.0)
\uput[u](2.25,0){A}
\end{pspicture}
\end{center}

\textbf{Affirmation 2 :} \og L'abscisse du point A est un nombre décimal. \fg


\item \begin{minipage}[t]{9.5cm}
On considère cet engrenage qui est composé d'une roue A à 8 dents et d'une roue B à 12 dents.

\textbf{Affirmation 3 :}

\og Cet engrenage sera dans la même position au bout de 6 tours pour la roue A et de 4 tours pour la roue B. \fg
\end{minipage}
%
%
\hspace{1cm}
%
%
\begin{tabular}[t]{|c|} \hline
\includegraphics[width=6cm]{engrenages}\\ \hline
\end{tabular}

\item \textbf{Affirmation 4 :}

\og Pour tout nombre $x$, l'égalité suivante est vraie:
\begin{center}$(x + 8)(2x - 1) = 2x^2 -(8 - 15x)$.\fg\end{center}
\end{enumerate}

\medskip

