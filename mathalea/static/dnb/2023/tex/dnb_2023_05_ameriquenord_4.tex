
\medskip

À l'aide d'un logiciel de programmation, on veut réaliser le motif \og Fleur\fg suivant.

\begin{center}
\begin{tabular}{|c|}\hline
\textbf{Motif \og Fleur \fg}\\
\psset{unit=1cm}
\begin{pspicture}(-1.6,-1.5)(1.5,1.5)
\def\para{\pspolygon(0,0)(0.9,0)(1.1,0.4)(0.2,0.4)}
\multido{\n=0+72}{5}{\rput{\n}(0,0){\para}}
\rput(-1,-1.35){Un pétale}\psline[linewidth=1.25pt]{->}(-1,-1.1)(-0.4,-0.5)
\end{pspicture}\\ \hline
\end{tabular}
\end{center}

\begin{enumerate}
\item 
	\begin{enumerate}
		\item Le parallélogramme KLMN ci-dessous représente un des pétales du motif \og Fleur \fg.

Construire ce parallélogramme sur la copie en prenant $1$~cm pour 5 pas.

\begin{center}
\psset{unit=0.8cm}
\begin{pspicture}(9.5,4)
\pspolygon(0.2,0.4)(7.2,0.4)(9.2,3.8641)(2.2,3.8641)
\uput[dl](0.2,0.4){K} \uput[dr](7.2,0.4){L} \uput[ur](9.2,3.8641){M} \uput[ul](2.2,3.8641){N} \uput[d](3.7,0.4){35 pas} \uput[ul](7,0.6){$120\degres$} \uput[dl](9,3.6641){$60\degres$} \uput[r](8.2,2.13){20 pas} 
\psarc(7.2,0.4){0.5}{60}{180}\psarc(9.2,3.8641){0.4}{180}{240}
\end{pspicture}
\end{center}
	\end{enumerate}
\end{enumerate}

\medskip

\begin{minipage}{0.56\linewidth}

\textbf{b.~} On définit le bloc \og Pétale \fg{} ci-contre afin de dessiner ce parallélogramme.

On commence la construction du parallélogramme au point K en s'orientant vers la droite.

Par quelles valeurs doit-on compléter les lignes 4, 5, 6, et 7 du bloc \og Pétale \fg{} ci-contre ?

\emph{Aucune justification n'est attendue, écrire sur la copie le numéro de la ligne du bloc \og Pétale\fg{} et la valeur correspondante.}
\end{minipage}\hfill
\begin{minipage}{0.4\linewidth}

\begin{tabular}{|l|}\hline
\multicolumn{1}{|c|}{Bloc \og Pétale \fg}\\
\begin{scratch}[num blocks]
\initmoreblocks{définir \namemoreblocks{Pétale}}
\blockpen{stylo en position d'écriture}
\blockrepeat{répéter \ovalnum{2} fois}
{\blockmove{avancer de \ovalnum{} pas}
\blockmove{tourner \turnleft{} de \ovalnum{} degr\'es}
\blockmove{avancer de \ovalnum{} pas}
\blockmove{tourner \turnleft{} de \ovalnum{} degr\'es}
}
\end{scratch}\\ \hline
\end{tabular}
%\end{enumerate}
\end{minipage}

\begin{enumerate}[resume]
\item Le bloc ci-dessous permet de construire un motif \og Fleur\fg{} en partant de son centre.

\begin{minipage}{0.48\linewidth}
\begin{tabular}{|l|}\hline
\multicolumn{1}{|c|}{Bloc \og Fleur \fg}\\
\begin{scratch}[num blocks]
\initmoreblocks{définir \namemoreblocks{Fleur}}
\blockrepeat{r\'ep\'eter \ovalnum{} fois}
{\blocklook{Pétale}
\blockmove{tourner \turnright{} de \ovalnum{72} degr\'es}
}
\end{scratch}\\ \hline
\end{tabular}
\end{minipage}\hfill
\begin{minipage}{0.48\linewidth}
\begin{tabular}{|c|}\hline
\textbf{Motif \og Fleur \fg}\\
\psset{unit=1cm}
\begin{pspicture}(-1.5,-1.5)(1.5,1.5)
\def\para{\pspolygon(0,0)(0.9,0)(1.1,0.4)(0.2,0.4)}
\multido{\n=0+72}{5}{\rput{\n}(0,0){\para}}
\end{pspicture}\\ \hline
\end{tabular}
\end{minipage}
	\begin{enumerate}
		\item Par quelle valeur doit-on compléter la ligne 2 du bloc \og Fleur \fg{} ci-dessus ? 
\emph{Aucune justification n'est attendue.}
		\item Expliquer le choix de la valeur \og 72 \fg{} dans la ligne 4.
		\item On modifie le bloc \og Fleur \fg{} pour construire le motif suivant:

\begin{center}
\psset{unit=0.2cm}
\begin{pspicture}(-7.75,-7.75)(7.75,7.75)
\def\para{\pspolygon(0,0)(0.9,0)(1.12,0.4)(0.22,0.4)}
%%
\def\para2{\pspolygon(0,0)(7,0)(9,3.4641)(2,3.4641)}
\multido{\n=0+30}{12}{\rput{\n}(0,0){\para2}}
\end{pspicture}
\end{center}

Quelles sont alors les modifications à apporter aux lignes 2 et 4 du bloc \og Fleur\fg{} ? \emph{Aucune justification n'est attendue}.
	\end{enumerate}
\end{enumerate}

\bigskip

