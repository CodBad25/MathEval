
\medskip	

\begin{enumerate}
	\item Les deux demi cercles forment un cercle complet. On calcule le périmètre de ce cercle.

$ P_{\text{cercle}}=2\times\pi\times40\simeq 251$~m.
		
On ajoute les longueurs des deux segments. Le périmètre total vaut donc :

$ 850\times2+251 \approx \np{1951}$~m.
	\item	
		\begin{enumerate}
			\item $ 2~\text{min}~9~$s $= 129  $~s. La vitesse moyenne est donc :
			
			\(\displaystyle v_{\text{moyenne}}=\dfrac{\text{distance}}{\text{temps}}\approx\dfrac{\np{1951}}{129}\approx 15\)~(m/s).
			\item
				$ 15 $m $ =0,015 $~km. Et $ 1 $ h $= \np{3600} $ s donc la vitesse en km/h est :
				
			\(\displaystyle v_{\text{moyenne}}=0,015\times \np{3600} = 54\). Soit environ $ 54 $~(km/h).
		\end{enumerate}
	\item	
		On calcule le nombre de sacs nécessaires, puis le montant à payer pour chaque marque.
		
-- Marque A :

$ 73~027 : 500 \simeq 146,1 $. On a donc besoin de $ 147 $ sacs.

Le coût est donc $ 147 \times 141,95 = \np{20866,65} $ euros.

-- Marque B :

$ \np{73027} : 400 \approx 182,6 $. On a donc besoin de $ 183 $ sacs.

Le coût est donc $ 183 \times 87,9 = \np{16085,70} $ euros.

-- Marque C :

$ 73~027 : 300 \simeq 243,4 $. On a donc besoin de $ 244 $ sacs.

Le coût est donc $ 244 \times 66,5 = \np{16226}$~euros.

Le tarif le moins cher est donc le tarif B.
\end{enumerate}
