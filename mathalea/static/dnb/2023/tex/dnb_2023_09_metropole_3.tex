
\medskip

Une piscine propose deux tarifs d'entrée pour l'année 2023.

\textbf{Tarif A} : $5,90$~\euro{} l'entrée.

\textbf{Tarif B} : $4,40$~\euro{} l'entrée avec une carte d'abonnement de $30$~\euro{} valable toute l'année.

\medskip

\begin{enumerate}
\item 
	\begin{enumerate}
		\item Quel est le prix total pour $10$ entrées avec le tarif A ? 
		\item Quel est le prix total pour $10$ entrées avec le tarif B ?
	\end{enumerate}
\item  On note $f$ et $g$ les fonctions qui modélisent les prix, en euro, respectivement du tarif A et du tarif B en fonction du nombre $x$ d'entrées.

Donner l'expression de $f(x)$, puis celle de $g(x)$.
\item 
	\begin{enumerate}
		\item Résoudre l'équation $5,90x = 4,40x + 30$.
		\item Quel est le nombre d'entrées pour lequel les tarifs A et B donnent le même prix à payer ?
	\end{enumerate}
\item On relève le nombre d'entrées par mois durant une année.

\begin{center}
\begin{tabularx}{\linewidth}{|m{1.5cm}|*{12}{>{\centering \arraybackslash \footnotesize}X|}}\hline
\footnotesize Mois	&Jan.	& Fév.& Mars& Avril& Mai& Juin& Juillet& Août& Sept.& Oct.& Nov.& Déc.\\ \hline
\footnotesize 
Nombre d'entrées	&\np{12500} &\np{13700} &\np{10400} &\np{13600} &\np{12300} &\np{11700} &\np{10400} &\np{11600} &\np{10200} &\np{13800} &\np{12600} &\np{11800}\\ \hline
\end{tabularx}
\end{center}

	\begin{enumerate}
		\item Calculer le nombre moyen d'entrées par mois.
		\item Calculer l'étendue du nombre d'entrées par mois.
	\end{enumerate}
\item La piscine a la forme d'un pavé droit de longueur $50$~m, de largeur $25$~m et de profondeur $3$~m. En admettant qu'elle soit entièrement remplie, déterminer en m$^3$, le volume d'eau qui sera évacué pour réaliser la vidange.
\end{enumerate}

\bigskip

