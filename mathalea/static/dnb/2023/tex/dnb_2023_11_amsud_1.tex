
\bigskip

\begin{minipage}{7.5cm}
On considère la figure ci-contre dans laquelle:

\begin{itemize}
\item[$\bullet$~] Les points F, G et H sont alignés
\item[$\bullet$~] (LH) est perpendiculaire à (FH)
\item[$\bullet$~] EF $= 18$ cm ; FG $= 24$ cm ; EG $= 30$ cm ;

GH = $38,4$ cm
\item[$\bullet$~] $\widehat{\text{EGF}} = \widehat{\text{LGH}}$.
\end{itemize}
\end{minipage}\hfill
\begin{minipage}{7.5cm}
\psset{unit=0.95cm}
\begin{pspicture}(7.8,4.8)
\pspolygon(0.2,3.8)(0.2,1)(7.5,1)(7.5,4.6)(3.3,1)
\pswedge*(3.3,1){0.4}{0}{39}
\pswedge*(3.3,1){0.4}{141}{180}
\psframe(7.5,1)(7.2,1.3)
\rput(3.9,0.2){\emph{La figure n'est pas en vraie grandeur.}}
\uput[u](0.2,3.8){E} \uput[dl](0.2,1){F} \uput[d](3.3,1){G} \uput[dr](7.5,1){H}\uput[ur](7.5,4.6){L}
\end{pspicture}
\end{minipage}

\begin{enumerate}
\item Montrer que le triangle EFG est rectangle en F{}.
\item Calculer la mesure de l'angle $\widehat{\text{EGF}}$.

Donner l'arrondi au degré près.
\item Montrer que les triangles EGF et LGH sont semblables.
\item Parmi les propositions suivantes, quel est le coefficient d'agrandissement qui permet de
passer du triangle EFG au triangle LHG ?

Expliquer.

\begin{center}
\begin{tabularx}{0.75\linewidth}{|*{4}{>{\centering \arraybackslash}X|}}\hline
0,625 &1,28	& 1,6	&2,6\\ \hline
\end{tabularx}
\end{center}

\item Quel est le périmètre du triangle LGH ?
\end{enumerate}


\medskip

