
\medskip

%Olivia a décidé d'installer sur le sol, plat de son jardin, quatre panneaux photovoltaïque pour produire une partie de l'électricité qu'elle consomme.
%
%\medskip
%
%\begin{tabularx}{\linewidth}{|X|}\hline
%\textbf{Description}\\
%Un panneau photovoltaïque est un dispositif permettant de générer de l'électricité à partir de l'énergie lumineuse.\\
%\textbf{Caractéristiques d'un panneau}\\
%\begin{itemize}
%\item[$\bullet~~$] Longueur 1700 mm
%\item[$\bullet~~$] Largeur 1000 mm
%\item[$\bullet~~$] Épaisseur 40 mm
%\item[$\bullet~~$] Fonctionnement optimal : inclinaison par rapport à l'horizontale comprise entre $30\degres{}$ et $35\degres{}$
%\item[$\bullet~~$] Orientation : Sud
%\end{itemize}\\ \hline
%\end{tabularx}
%
%\smallskip
%
%Pour incliner ses panneaux et obtenir un fonctionnement optimal, Olivia choisit de fabriquer elle-même un support. Pour cela, elle réalise les schémas suivants de support qui sera constitué de trois équerres identiques, reliées entr'elles par trois barres latérales de $4$~m de long.\\
%Chaque support est prévu pour accueillir quatre panneaux.\\
%
%\begin{tabularx}{\linewidth}{p{7cm}|X}
%Plan général du support, un panneau est représenté : &Plan détaillé d'une équerre :\\
%\psset{unit=0.8cm,arrowsize=2pt 3}
%\begin{pspicture}(8.2,4.4)
%%\psgrid
%\pspolygon[linewidth=0.8pt](0,2.8)(0,1.1)(2.2,1.1)(8.2,2.6)(6,4.3)
%\psline[linestyle=dotted,linewidth=1.25pt](0,1.1)(6,2.6)
%\pspolygon[linestyle=dotted,linewidth=1.25pt](3,3.52)(3,1.82)(5.2,1.82)
%\pspolygon[linestyle=dotted,linewidth=1.25pt](6,4.3)(6,2.6)(8.2,2.6)
%\psline[linestyle=dotted,linewidth=1.25pt](0.9,1.1)(0.9,2.1)
%\psline[linestyle=dotted,linewidth=1.25pt](3.8,1.9)(3.8,2.8)
%\psline[linestyle=dotted,linewidth=1.25pt](6.8,2.7)(6.8,3.6)
%\pspolygon[fillstyle=solid,fillcolor=lightgray](0,2.8)(1.5,3.15)(3.7,1.45)(2.2,1.1)
%\rput(6,0.1){Barres latérales}
%\psline{->}(6,0.2)(7,2.3)
%\psline{->}(6,0.2)(4.8,4)
%\psline{->}(6,0.2)(0.6,1.22)
%\psline{<->}(0,2.9)(6,4.4)\uput[u](3,3.65){4~m}
%\end{pspicture}&
%\psset{unit=0.9cm,arrowsize=2pt 3}
%\begin{pspicture}(6.8,4.4)
%\pspolygon[linewidth=1.4pt](1.1,0.8)(6.6,0.8)(1.1,3.9)
%\psline[linewidth=1.4pt](2.2,0.8)(2.2,3.3)
%\psframe(1.1,0.8)(1.3,1)\psframe(2.2,0.8)(2.4,1)
%\uput[ul](1.1,3.9){H} \uput[l](1.1,0.8){P} \uput[ur](2.2,3.3){U} \uput[ul](2.2,0.8){T} \uput[r](6.6,0.8){S}
%\psline[linewidth=0.65pt]{<->}(1.1,0.6)(6.6,0.6)\uput[d](3.85,0.6){140~cm}
%\rput{90}(0.9,2.35){90~cm}
%\end{pspicture}
%\end{tabularx}
%\medskip

\begin{enumerate}
\item 
	\begin{enumerate}
		\item %Vérifier que la distance HS arrondie au millimètre est égale à 166,4 cm.
Le théorème de Pythagore appliqué au triangle HPS rectangle en P donne :
		
HS$^2 = \text{HP}^2 + \text{PS}^2 = 90^2 + 140^2 = \np{8100} + \np{19600} = \np{27700}$.

HS étant positive : HS $= \sqrt{\np{27700}} \approx 166,43$, soit $163,4$~cm au millimètre près.
		\item %Pour que le panneau soit bien tenu, le fabricant conseille que la distance HS du support mesure au moins 95\,\% de la longueur du panneau.
%On rappelle que cette longueur mesure 1700 mm.
%Ce support sera-t-il conforme aux conseils du fabricant ?
\np{1700}~mm = 170~cm (longueur du panneau.

Or $95\,\%$ de $170 = \dfrac{95}{100} \times 170 = 0,95 \times 170 = 161,5$~cm.

Comme $163,4 > 161,5$, le panneau est conforme.
	\end{enumerate}
\item %L'angle d'inclinaison, $\widehat{\text{HSP}}$ permettra-t-il un fonctionnement optimal des panneaux ?
Dans le triangle HPS rectangle en P{}, on a la relation :

$\tan \widehat{\text{HSP}} = \dfrac{\text{HP}}{\text{PS}} = \dfrac{90}{140} = \dfrac{9}{14} \approx 0,643$.

La calculatrice donne $\widehat{\text{HSP}} \approx 32,7\degres{}$.

On a bien $30 < 32,7 < 35$. L'angle d'inclinaison, $\widehat{\text{HSP}}$ permet donc un fonctionnement optimal des panneaux.
\item %Pour consolider l'ensemble, Olivia fixe, à l'intérieur de ses équerres, une barre de renfort de $50$~cm de longueur.

%Sur le plan détaillé d'une équerre, cette barre est représentée par le segment [AT] perpendiculaire au segment [PS].

%Calculer la longueur ST. On arrondira au millimètre.
Les droites (UT) et (HP) sont perpendiculaires à la droite (PS) : elles sont donc parallèles.

S, U, H d'une part S, T et P sont alignés donc le théorème de Thalès permet d'écrire :

$\dfrac{\text{ST}}{\text{SP}} = \dfrac{\text{SU}}{\text{SH}} = \dfrac{\text{UT}}{\text{PT}}$.

En particulier $\dfrac{\text{ST}}{140} = \dfrac{50}{90}$. On en déduit :

ST $ = 140 \times \dfrac{50}{90} = 140 \times \dfrac59 \approx 77,8$~cm au millimètre près.
\item %Olivia, achète des tubes en acier inoxydable de longueur $4,5$~m à $37$~\euro{} l'unité pour fabriquer le support composé de trois équerres et des trois barres latérales.
%Montrer qu'elle doit prévoir un budget minimum de $222$~\euro{} pour l'achat des tubes en acier inoxydable.
Chaque équerre avec sa barre de renfort nécessite une longueur de tube égale à environ :

$140 + 90 + 166,4 + 50 = 446,4$~cm soit environ 4,464~m.

De plus il faut 3 équerres et 3 barres latérales de 4~m, soit $3 \times 4,464 + 3 \times 4 = 25,392$~m.

Un tube mesurant 4,5~m il faut donc $\dfrac{25,392}{4,5} \approx 5,64$ : 6 tubes sont donc nécessaires à 37~\euro{} l'unité ce qui qui représente une dépense de :

$6 \times 37 = 222$~\euro.
\end{enumerate}

\bigskip

