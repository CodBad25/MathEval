
\medskip

\begin{center}
\psset{unit=0.8cm}
\begin{pspicture}(-1,0)(2.1,1.9)
\pspolygon[linecolor=blue](0,0)(2.1;0)(2.1;60)(2.1;120)
\end{pspicture}
\end{center}

\begin{enumerate}
\item On souhaite tracer le losange ci-dessus de côté $50$ pas à l'aide du bloc losange.

On a écrit le script ci-dessous avec le logiciel Scratch.

Les lignes 3 et 6 sont complétées en rouge .

\begin{center}
\begin{scratch}[num blocks]
\initmoreblocks{définir \namemoreblocks{losange}}
\blockpen{stylo en position d'écriture}
\blockrepeat{répéter \ovalnum{\textcolor{red}{2}} fois}
{
\blockmove{avancer de \ovalnum{50} pas}
\blockmove{tourner \turnright{} de \ovalnum{60} degrés}
\blockmove{avancer de \ovalnum{\textcolor{red}{50}} pas}
\blockmove{tourner \turnright{} de \ovalnum{120} degrés}
}
\blockpen{relever le stylo}
\end{scratch}
\end{center}

\item Préciser sur votre copie quelle figure est associée à chaque script 1, 2 ou 3.

Aucune justification n'est demandée.

\begin{center}
\begin{tabularx}{\linewidth}{*{3}{>{\centering \arraybackslash}X}}
Figure A &Figure B& Figure C\\ \hline
\psset{unit=0.3cm}
\begin{pspicture}(-3,0)(6.1,1.9)
\def\los{\pspolygon[linecolor=blue](0,0)(2.1,0)(1.2,1.8)(-0.9,1.8)}
\multido{\n=-3.0+3}{4}{\rput(\n,0){\los}}
\end{pspicture}&
\psset{unit=0.25cm}
\begin{pspicture}(-3,-1)(7.6,5)
\def\los{\pspolygon[linecolor=blue](0,0)(2.1,0)(1.2,1.8)(-0.9,1.8)}
\multido{\n=-3+2.1,\na=-0.5+1.0}{4}{\rput(\n,\na){\los}}
\end{pspicture}&
\psset{unit=0.3cm}
\begin{pspicture}(-3,0)(5,3.9)
\def\los{\pspolygon[linecolor=blue](0,0)(2.1,0)(1.2,1.8)(-0.9,1.8)}
\multido{\n=-2.0+2.1}{4}{\rput(\n,0){\los}}
\end{pspicture}\\
Script 1 &Script 2 &Script 3\\
\begin{scratch}
\blockinit{Quand \ovalnum{1} est pressé}
\blockmove{aller à x: \ovalnum{-220} y: \ovalnum0}
\blockmove{s'orienter à \ovalnum{90} degrés}
\blockpen{effacer tout}
\blockrepeat{répéter \ovalnum{4} fois}
{\blockmove{losange}
\blockmove{avancer de \ovalnum{50} pas}
}
\end{scratch}&
\begin{scratch}
\blockinit{Quand \ovalnum{2} est pressé}
\blockmove{aller à x: \ovalnum{-220} y: \ovalnum0}
\blockmove{s'orienter à \ovalnum{90} degrés}
\blockpen{effacer tout}
\blockrepeat{répéter \ovalnum{4} fois}
{\blockmove{losange}
\blockmove{avancer de \ovalnum{100} pas}
}
\end{scratch}&
\begin{scratch}
\blockinit{Quand \ovalnum{3} est pressé}
\blockmove{aller à x: \ovalnum{-220} y: \ovalnum0}
\blockmove{s'orienter à \ovalnum{90} degrés}
\blockpen{effacer tout}
\blockrepeat{répéter \ovalnum{4} fois}
{\blockmove{losange}
\blockmove{avancer de \ovalnum{50} pas}
\blockmove{ajouter \ovalnum{30} à y}
}
\end{scratch}\\
\end{tabularx}
\end{center}

\medskip

Association des scripts :

Script 1\ figure C \quad Script 2\ figure A\quad Script 3 figure B.

\item  Dans la figure ci-dessous obtenue par le programme associé, la transformation qui permet d'obtenir le losange ABCD à partir du losange EDCF,

est la rotation de centre C et d'angle \ang{60}

%\begin{tabularx}{\linewidth}{*{2}{>{\centering \arraybackslash}X}}
\begin{minipage}{7cm}
\psset{unit=1cm}
\begin{pspicture}(-2.8,-2.8)(2.8,2.8)
\def\los{\pspolygon[linecolor=blue](0,0)(1.4;0)(2.42488;30)(1.4;60)}
\multido{\n=0+60}{6}{\rput{\n}(0,0){\los}}
\uput[u](2.42488;90){A} \uput[ur](1.4;60){B} \uput[d](0,-0.1){C}
\uput[ul](1.4;120){D} \uput[ul](2.42488;150){E} \uput[l](1.4;180){F} 
\end{pspicture}
\end{minipage}\hfill
\begin{minipage}{7cm}
\begin{scratch}
\blockinit{Quand \greenflag est cliqué}
\blockpen{effacer tout}
\blockmove{aller à x: \ovalnum{0} y: \ovalnum0}
\blockrepeat{répéter \ovalnum{6} fois}
{\blockmove{losange}
\blockmove{tourner \turnright{} de \ovalnum{60} degrés}}
\end{scratch}
\end{minipage}
%\end{tabularx}
\end{enumerate}

\bigskip

