
\medskip

\begin{enumerate}
	\item	
		\begin{enumerate}
			\item Programme de construction :

-- avec la règle graduée, tracer le segment $[KL]$, de longueur $ 7 $~cm. ($ 35 : 5 = 7 $).
					
-- avec le rapporteur, tracer l'angle de sommet $ L $.
					
-- avec la règle graduée, placer le point $ M $ à $4$~cm du point $ L $.
					
-- tracer la parallèle à $(KL)$, passant par $M$.

-- reporter la longueur $ KL $ et placer le point $N$ sur cette parallèle.

-- tracer $ [NK] $.
		\item	\textit{ligne 4} : 35
		
\textit{ligne 5} : 60

\textit{ligne 6} : 20

\textit{ligne 7} : 120
	\end{enumerate}
	\item
		\begin{enumerate}
			\item	On complète la \textit{ligne 2} par la valeur $ 5 $.
			\item	Le motif comporte $ 5 $ pétales et on effectue une rotation de $ 360 $ degrés. Or $ 360 : 5 = 72 $. Donc après avoir tracé chaque pétale, on tourne de $ 72 $ degrés.
			\item Il y a maintenant $ 12 $ pétales.

Or, $ 360 : 12 = 30 $. On modifie donc les lignes comme suit :

\textit{ligne 2} : répéter $12$ fois

\textit{ligne 4} : tourner de $ 30 $ degrés

		\end{enumerate}
	\end{enumerate}

\bigskip

