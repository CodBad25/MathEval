
\medskip

{\footnotesize
Une piscine propose deux tarifs d'entrée pour l'année 2023.

\textbf{Tarif A} : $5,90$~\euro{} l'entrée.

\textbf{Tarif B} : $4,40$~\euro{} l'entrée avec une carte d'abonnement de $30$~\euro{} valable toute l'année.
}
\medskip

\begin{enumerate}
\item 
	\begin{enumerate}
		\item Le prix total pour $10$ entrées avec le tarif A est en euros : $\np{5.9}\times 10=59$ 
		\item Le prix total pour $10$ entrées avec le tarif B  est en euros : $\np{4.4}\times 10+30=74$
	\end{enumerate}
\item  On note $f$ et $g$ les fonctions qui modélisent les prix, en euro, respectivement du tarif A et du tarif B en fonction du nombre $x$ d'entrées.

Donnons l'expression de $f(x)$, puis celle de $g(x)$.
 $f(x)=\np{5.9}x\qquad g(x)= \np{4.4}x+30$.
\item 
	\begin{enumerate}
		\item Résolvons l'équation $5,90x = 4,40x + 30$.
		\begin{align*}
			5,90x &=4,40x+30\\
			5,90x-4,40x&=30\\
			1,5x&=30\\
			x&=\dfrac{30}{1,5}=20
		\end{align*}
		L'ensemble des solutions de l'équation est \{20\}.
			\item Le nombre d'entrées pour lequel les tarifs A et B donnent le même prix à payer est 20.
	\end{enumerate}
\item On relève le nombre d'entrées par mois durant une année.

\begin{center}
\begin{tabularx}{\linewidth}{|m{1.5cm}|*{12}{>{\centering \arraybackslash \footnotesize}X|}}\hline
\footnotesize Mois	&Jan.	& Fév.& Mars& Avril& Mai& Juin& Juillet& Août& Sept.& Oct.& Nov.& Déc.\\ \hline
\footnotesize 
Nombre d'entrées	&\np{12500} &\np{13700} &\np{10400} &\np{13600} &\np{12300} &\np{11700} &\np{10400} &\np{11600} &\np{10200} &\np{13800} &\np{12600} &\np{11800}\\ \hline
\end{tabularx}
\end{center}

	\begin{enumerate}
		\item Calculons le nombre moyen d'entrées par mois.

		$\overline{x}=\dfrac{\np{12500} +\np{13700}+\dots  +\np{10200} +\np{13800} +\np{12600} +\np{11800}}{12}=\np{12050}$		
		\item Calculons l'étendue du nombre d'entrées par mois. L'étendue est la différence entre les valeurs extrêmes.

		$\np{13800}-\np{10200}=\np{3600}$.

	\end{enumerate}
\item La piscine a la forme d'un pavé droit de longueur $\np[m]{50}$, de largeur $\np[m]{25}$ et de profondeur $\np[m]{3}$~m. En admettant qu'elle soit entièrement remplie, déterminons en m$^3$, le volume d'eau qui sera évacué pour réaliser la vidange.

Le volume d'un pavé droit est $ L \times \ell\times h$. Nous avons donc $50\times 25\times 3=\np{3750}$.

Le volume d'eau à évacuer est donc de $\np[m^3]{3750}$
\end{enumerate}

\bigskip

