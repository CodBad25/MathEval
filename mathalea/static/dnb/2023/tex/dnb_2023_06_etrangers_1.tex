
\medskip

Cet exercice, en deux parties, est un questionnaire à choix multiples (QCM). Pour chaque question, parmi les réponses proposées, une seule est exacte. Recopier le numéro de la question et indiquer la réponse choisie. 

\textbf{Aucune justification n'est attendue ici}

\medskip

\textbf{Partie A}

\medskip

Dans cette partie, on s'intéresse au programme ci-dessous, composé d'un bloc \og triangle équilatéral\fg{} et d'un script principal:

\begin{center}
\begin{tabular}{m{6.5cm}|m{6.5cm}}
Bloc \og triangle équilatéral\fg 	&Script principal\\
\begin{scratch}
\initmoreblocks{définir \namemoreblocks{triangle équilatéral}}
\blockpen{stylo en position d'écriture}
\blockrepeat{répéter \ovalnum{3} fois}
{\blockmove{avancer de \ovalnum{50} pas}
\blockmove{tourner \turnleft{} de \ovalnum{} degrés}
}
\end{scratch}						&\begin{scratch}
\blockinit{Quand \greenflag est cliqué}
\blockmove{aller à x: \ovalnum0 y: \ovalnum0}
\blockmove{s'orienter à \ovalnum{90} degrés}
\blockpen{effacer tout}
\blockrepeat{répéter \ovalnum{2} fois}
{
\blockpen{stylo en position d'écriture}
\blockmoreblocks{triangle équilatéral}
\blockpen{relever stylo}
\blockmove{tourner \turnleft{} de \ovalnum{60} degrés}
}
\end{scratch}\\
&On rappelle que l'instruction \og s'orienter à 90 \fg{} signifie s'orienter vers la droite.\\
\end{tabular}

\smallskip
\begin{tabularx}{\linewidth}{|m{5cm}|*{3}{>{\centering \arraybackslash}X|}}\hline
Questions&Réponse A&Réponse B&Réponse C\\ \hline
\textbf{1.} On souhaite construire le triangle équilatéral ci-dessous. 

Le stylo est orienté à $90\degres{}$ au départ comme ci-dessous.

Départ\begin{pspicture}(-0.5,-0.1)(2.,1.6)
\psline{->}(-0.5,0)(0,0)
\pspolygon(0;0)(1.7;0)(1.7;60)
\psdots(0.85;0)(1.472;30)(0.85;60)
\end{pspicture}

Compléter le script du bloc \og triangle équilatéral \fg{} avec la valeur qui convient.&$60\degres{}$&$100\degres{}$&$120\degres{}$\\ \hline
\textbf{2.} Parmi les trois figures, laquelle est obtenue avec le script principal ?&\psset{unit=1cm}
\begin{pspicture}(-1,0)(1.3,1.4)
\def\tri{\pspolygon(0;0)(1.2;0)(1.2;60)}
\rput(0,0){\tri}\rput{90}(0,0){\tri}
\end{pspicture} & \psset{unit=1cm}
\begin{pspicture}(-1.3,0)(1.3,1.4)
\def\tri{\pspolygon(0;0)(1.2;0)(1.2;60)}
\rput(0,0){\tri}\rput(-1.2,0){\tri}
\end{pspicture}&\psset{unit=1cm}
\begin{pspicture}(-1,0)(1.3,1.4)
\def\tri{\pspolygon(0;0)(1.2;0)(1.2;60)}
\rput(0,0){\tri}\rput{60}(0,0){\tri}
\end{pspicture}\\ \hline
\textbf{3.} Quel polygone obtient-on si on remplace dans le script principal, la boucle \og répéter 2 fois \fg{} par une boucle \og  répéter 6 fois \fg{} ?&Un parallélogramme
&Un hexagone
&Un losange\\ \hline
\end{tabularx}
\end{center}

\medskip

\textbf{Partie B}

\medskip

\begin{tabularx}{\linewidth}{|m{6cm}|*{3}{>{\centering \arraybackslash}X|}}\hline
Questions&Réponse A&Réponse B&Réponse C\\ \hline
\textbf{1.} $\left(\dfrac23 - \dfrac13 \times\dfrac75\right) \div \dfrac 43 = $&$\dfrac{3}{15} \times \dfrac43$&$\left(\dfrac13 \times \dfrac75 \right) \div \dfrac43$&$\dfrac{3}{15} \times \dfrac34$\\ \hline
\textbf{2.} L'écriture scientifique de $302,4 \times 10^{18}$ est:&$3,024 \times 10^{16}$&$3,024 \times 10^{20}$&$\np{0,3024} \times 10^{21}$\\ \hline
\textbf{3.} On donne ci-dessous la masse de 8 biscuits différents:

12 g ; 10 g ; 18 g ; 8 g ; 12 g ; 15 g ; 11 g ; 13 g

Suite à une erreur de mesure, le biscuit pesant $18$ g pèse en fait $16$ g.

Une fois cette erreur corrigée, la valeur de la médiane sera :&Plus petite.&La même.&Plus grande.\\ \hline
\end{tabularx}

\bigskip

