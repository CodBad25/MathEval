
\medskip

%\textbf{Pour chacune des affirmations, indiquer si elle est vraie ou fausse en justifiant la réponse.}
%
%\medskip
%
\begin{enumerate}
\item %On considère le tableau ci-dessous :
%\begin{center}
%\begin{tabularx}{0.8\linewidth}{|l|*{4}{>{\centering \arraybackslash}X|}}\hline
%Nombre de baguettes &1 &2 &3 &4 \\ \hline
%Prix en \euro{}& 1,10& 2,20& 3,30& 4\\ \hline
%\end{tabularx}
%\end{center}
\textbf{Affirmation 1 :} \og Le prix est proportionnel au nombre de baguettes. \fg

On a bien $2,20 = 2 \times 1,10,\quad 3,30 = 3 \times 1,0$, mais $4 \ne 4 \times 1,10$. 

L'affirmation 1 est fausse.
\item \textbf{Affirmation 2 :} \og L'abscisse du point A est un nombre décimal. \fg
%On considère ci-dessous le point A sur une droite graduée:

%\begin{center}
%\psset{xunit=5cm,arrowsize=2pt 3}
%\begin{pspicture}(0.75,-0.25)(2.75,0.25)
%\psaxes[linewidth=2pt,ticks=x,subticks=8,xsubticksize=2,Ox=1](1,0)(0.75,0.)(2.5,0.0)
%\uput[u](2.25,0){A}
%\end{pspicture}
%\end{center}
L'unité est partagée en 8, donc  $1 = 8 \times 0,125$.

Le point A a onc pour abscisse : $2 + 2 \times 0,125 = 2 + 0,25 = 2,25$ : cette abscisse est bien décimale.



L'affirmation 2 est vraie.
\item \textbf{Affirmation 3 :}

\og Cet engrenage sera dans la même position au bout de 6 tours pour la roue A et de 4 tours pour la roue B. \fg
%\begin{minipage}{7.5cm}
%On considère cet engrenage qui est composé d'une roue A à 8 dents et d'une roue B à 12 dents.

On a bien $6 \times 8 = 4 \times 12 = 48$.

L'affirmation 3 est vraie.
%\end{minipage}
%\begin{minipage}{7.5cm}
%\includegraphics[width=6cm]{engrenages}
%\end{minipage}

\item \textbf{Affirmation 4 :}

\og Pour tout nombre $x$, l'égalité suivante est vraie:
\begin{center}$(x + 8)(2x - 1) = 2x^2 -(8 - 15x)$.\fg\end{center}

On a d'une part :

$(x + 8)(2x - 1) = 2x^2 - x + 16x - 8 = 2x^2 + 15x - 8$ et d'autre part :

$2x^2 -(8 - 15x) = 2x^2 - 8 + 15x = 2x^2 + 15x - 8$.

L'affirmation 4 est vraie.
\end{enumerate}

\medskip

