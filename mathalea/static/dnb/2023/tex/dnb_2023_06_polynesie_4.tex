
\medskip

On considère le programme de calcul suivant :

\medskip

\begin{itemize}
\item[$\bullet~~$]Choisir un nombre
\item[$\bullet~~$]Prendre le carré de ce nombre
\item[$\bullet~~$]Multiplier le résultat par 2
\item[$\bullet~~$]Ajouter le nombre de départ
\item[$\bullet~~$]Soustraire 66
\end{itemize}

\medskip

\begin{enumerate}
\item 
	\begin{enumerate}
		\item Montrer que si le nombre choisi au départ est 4, le résultat obtenu est $-30$.
		\item  Quel résultat obtient-on si le nombre choisi au départ est $-3$ ?
	\end{enumerate}
\item 
	\begin{enumerate}
		\item On s'intéresse au bloc d'instruction ci-dessous intitulé \og Programme de calcul \fg.
On souhaite le compléter pour calculer le résultat obtenu avec le programme de calcul en fonction du nombre choisi au départ.

On précise que deux variables ont été créées: \og nombre choisi\fg{} qui correspond au nombre choisi au départ, et \og Résultat \fg.

\begin{center}
\begin{scratch}
\initmoreblocks{d\'efinir \namemoreblocks{Programme de calcul}}
\blockvariable{mettre \selectmenu{Résultat} à \ovaloperator{\ovalnum{A} * \ovalvariable{Nombre choisi}}}
\blockvariable{mettre \selectmenu{Résultat} à \ovaloperator{\ovalnum{B} * \ovalvariable{Résultat}}}
\blockvariable{mettre \selectmenu{Résultat} à \ovaloperator{\ovalvariable{Résultat} + \ovalvariable{Nombre choisi}}}
\blockvariable{mettre \selectmenu{Résultat} à \ovaloperator{\ovalvariable{Résultat} - \ovalnum{66}}}
\end{scratch}
\end{center}

Écrire sur votre copie le contenu qui doit être inséré dans les emplacements A et B. 
\textbf{Aucune justification n'est attendue pour cette question.}

		\item  Lucie insère le bloc précédent dans le script ci-dessous et observe la réponse donnée par le lutin:

\medskip

\begin{minipage}{0.6\linewidth}
\begin{center}
Script
\end{center}
\begin{scratch}
\blockinit{Quand \greenflag est cliqué}
\blockvariable{mettre \selectmenu{nombre choisi} à \ovalnum{0}}
\blockrepeat{répéter \ovalnum{20} fois}
{
\blocklist{Programme de calcul}
\blockif{si \booloperator{\ovalmove{Résultat} = \ovalnum{0}} alors}
{
\blocksound{dire \ovaloperator{regrouper \ovalnum{On peut choisir comme nombre de départ} et \ovalvariable{nombre choisi}} }
}
\blockvariable{mettre \selectmenu{nombre choisi} à \ovaloperator{\ovalvariable{nombre choisi} + \ovalnum{0,5}}}
}
\end{scratch}
\end{minipage}
\begin{minipage}{0.37\linewidth}
\begin{center}
Réponse du lutin\\
\end{center}
On peut choisir comme nombre de départ $5,5$.\\
\\
\qquad\includegraphics[scale=0.7]{Lutin}\\
\\
\\
\\
\\
\\
\\
\\
\\
\\
\end{minipage}

\medskip

À quoi correspond la valeur 5,5 donnée comme réponse par le lutin avec le programme de Lucie?
	\end{enumerate}
\item On nomme $x$ le nombre choisi au départ.
	\begin{enumerate}
		\item Déterminer l'expression obtenue par ce programme de calcul en fonction de $x$.
		\item On admet que $(2x - 11) (x + 6)$ est la forme factorisée de l'expression trouvée à la question précédente.
		
Pour quelle(s) valeur(s) de $x$, le résultat obtenu avec le programme est-il égal à $0$ ?
	\end{enumerate}
\end{enumerate}

\bigskip

