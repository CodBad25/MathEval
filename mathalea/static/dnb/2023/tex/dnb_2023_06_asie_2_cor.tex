
\medskip

%Cet exercice est un questionnaire à choix multiples (QCM). Aucune justification n'est demandée.
%
%Pour chaque question, trois réponses (A, B et C) sont proposées. \textbf{Une seule réponse est exacte.} Recopier le numéro de la question et la réponse sur la copie.


%{\renewcommand{\arraystretch}{1.3}
%\begin{tabularx}{\linewidth}{|m{7.4cm} |*{3}{>{\centering \arraybackslash}X|}}
%\hline
%\hfill{~}\bf{}Questions\hfill{~} & \bf{}Réponse A & \bf{}Réponse B & \bf{}Réponse C \\
%\hline
%\textbf{1)~~}Un sac de billes opaque  contient deux billes rouges, trois billes vertes et trois billes bleues. On tire au hasard une bille dans ce sac.\newline Quelle est la probabilité d'obtenir une bille rouge? & $\dfrac{2}{5}$ & $ \dfrac{1}{4}$ & $\dfrac{3}{8}$\\
%\hline 
%\textbf{2)~~}Si je souhaite augmenter un prix de 25\,\%, par quel coefficient dois-je multiplier ce prix? & $1,25$ & $0,25$ & $0,75$\\
%\hline 
%\textbf{3)~~}Sur la figure suivante, le triangle \pscirclebox{2}{} est l'image du triangle \pscirclebox{1}{} par une transformation.\newline Quelle est cette transformation?\newline
%
%\psset{unit=0.5cm}
%\def\xmin {-1}   \def\xmax {13}
%\def\ymin {-1}   \def\ymax {7}
%\begin{pspicture}(\xmin,\ymin)(\xmax,\ymax)
%%\psaxes[arrowsize=3pt 3, ticksize=-2pt 2pt, labels=none](0,0)(\xmin,\ymin)(\xmax,\ymax)
%%\uput[dl](0,0){$O$}
%\pspolygon[fillstyle=solid,fillcolor=lightgray,linewidth=0.7pt](6,0)(12,0)(12,6)
%\pspolygon[fillstyle=solid,fillcolor=lightgray,linewidth=0.7pt](2,4)(4,4)(4,6)
%\psgrid[gridlabels=0pt,subgriddiv=1, gridcolor=gray] 
%\psline(-0.5,6)(12.5,6) \psline(-0.5,6.25)(12.5,-0.25) \psline(-0.5,6.5)(6.5,-0.5)
%\uput[ur](0,6){D} \rput(10.5,2.5){\pscirclebox{2}} \rput(3.5,4.75){\pscirclebox{1}}
%\end{pspicture}
%& Une\newline  translation & Une\newline  homothétie de centre D et de rapport $-3$ & Une\newline  homothétie de centre D et de rapport $3$ \\
%\hline 
%\textbf{4)~~}On considère une fonction $f$ définie par:\newline
%\hspace*{1cm} $f(x)=-9-7x$\newline
%Quelle est l'affirmation correcte? & $f$ est une\newline fonction affine & $f$ est une\newline fonction linéaire & $f$ n'est ni une fonction affine ni une fonction linéaire\\
%\hline 
%\textbf{5)~~}Une année-lumière est une unité de longueur égale à environ \np{9461} milliards de kilomètres.\newline
%À quelle distance en mètre cela correspond-il? & $9,461\times 10^{15}$~m & $9,461\times 10^{12}$~m  & $9,461\times 10^{9}$~m \\
%\hline
%\textbf{6)~~}\newline
%\psset{unit=0.6cm}
%\def\xmin {-2}   \def\xmax {5} \def\ymin {-1}   \def\ymax {3}
%\begin{pspicture}(\xmin,\ymin)(\xmax,\ymax)
%\psframe[fillstyle=solid,fillcolor=lightgray,linewidth=0.7pt](0,0)(0.3,0.3)
%\pspolygon(0,0)(4.33,0)(0,2.5)
%%\psgrid[gridlabels=0pt,subgriddiv=5, gridcolor=gray] 
%{\small
%\uput[dl](0,0){A} \uput[dr](4.33,0){B} \uput[ul](0,2.5){C}
%\pswedge[fillstyle=solid,fillcolor=lightgray](4.33,0){0.3cm}{150}{180}
%\uput[ur](2,1.5){5 cm} \rput(2.5,0.5){30\degres}
%} 
%\end{pspicture}
%
%Quelle expression donne la longueur AB en centimètre?
%& $5\times \sin\;30\degres{}$  & $5\times \cos\;30\degres{}$ & $\dfrac{5}{\cos\;30\degres{}}$\\
%\hline
%\end{tabularx}}
\begin{enumerate}
\item Il y a 2 billes rouges pour un total de $2 + 3 + 3 = 8$ billes ; la probabilité d'obtenir une bille rouge est donc égale à $\dfrac{2}{8} = \dfrac14$ ou 0,25. Réponse B.
\item Ajouter 25\,\% c'est multiplier par $1 + \frac{25}{100} = 1 + 0,25 = 1,25$. Réponse A.
\item C'est la réponse C.
\item $f(x)$ est de la forme $ax + b$ : c'est une fonction affine. Réponse A.
\item $\np{9461} \times 10^9$~(km) $ = 9,461 \times 10^3 \times 10^9$~(km) $= 9,461 \times 10^3 \times 10^{12}$~(m) $= 9,461 \times 10^{15}$~(m). Réponse A.
\item Par définition du cosinus :

$\cos \widehat{\text{ABC}} = \dfrac{\text{AB}}{\text{BC}}$, d'où AB $ = \text{BC} \times \cos 30\degres{}$. Réponse B.
\end{enumerate}

\bigskip

