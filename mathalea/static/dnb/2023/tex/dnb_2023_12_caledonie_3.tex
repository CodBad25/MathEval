
\medskip

Matthieu souhaite isoler la toiture de sa maison.

Il compte utiliser de la laine de roche pour le toit de sa terrasse et de la ouate de cellulose pour le toit de la partie habitable.

Pour savoir quelles quantités de matériaux acheter, il doit effectuer des calculs.

Il a noté sur un plan de sa maison ci-dessous (vue de profil), toutes les mesures qu'il connait :

\begin{center}
\psset{unit=1cm,arrowsize=2pt 3}
\begin{pspicture}(-0.2,-0.3)(13,6)
%\psgrid
\pspolygon(0,0)(12.2,0)(12.2,2.8)(8.8,4.8)(0,2.7)%AGFDC
\psline(8.8,4.8)(8.8,0)%DB
\psline[linestyle=dashed](8.8,2.8)(12.2,2.8)%EF
\psframe(8.8,2.8)(9,3)
\psline[linewidth=0.5pt]{<->}(8.6,0)(8.6,4.8) \uput[l](8.6,2.4){4,3 m}
\psline[linewidth=0.5pt]{<->}(0,-0.2)(8.8,-0.2) \uput[d](4.4,-0.2){8 m}
\uput[r](0,1.35){2,5~m}
\psarc(12.2,2.8){0.6}{150}{180}\rput(11.3,3){$30\degres$}
\rput(4,5){Toit de la partie habitable}\psline{->}(4,4.8)(4.5,3.8)
\rput(11,5){Toit de la terrasse}\psline{->}(11,4.8)(10.18,4)
\rput(6.6,5.8){\textbf{Le plan n'est pas à l'échelle}}
\psline(-0.1,1.325)(0.1,1.325)\psline(-0.1,1.375)(0.1,1.375)
\psline(8.7,1.325)(8.9,1.325)\psline(8.7,1.375)(8.9,1.375)
\psline(12.1,1.325)(12.3,1.325)\psline(12.1,1.375)(12.3,1.375)
\uput[dl](0,0){A} \uput[dr](8.8,0){B} \uput[ul](0,2.7){C} 
\uput[u](8.8,4.8){D} \uput[dr](8.8,2.8){E}\uput[ur](12.2,2.8){F} \uput[dr](12.2,0){G} 
\end{pspicture}
\end{center}

\textbf{On donne :}

AC = 2,5 m\quad AB = 8 m \quad BD = 4,3 m \quad $\widehat{\text{EFD}} = 30\degres{}$

Les points D, E, B ainsi que les points A, B, G sont alignés.

\medskip

\begin{enumerate}
\item Justifier que DE $= 1,8$~m.
\item Montrer que la longueur DF du toit de la terrasse est égale à 3,6~m.

\textbf{Rédiger la réponse en faisant apparaître les différentes étapes.}
\end{enumerate}

On considère que :

\begin{itemize}
\item le toit de la terrasse est un rectangle de longueur $12$~m et de largeur $3,6$~m ; 
\item un rouleau de laine de roche couvre 6 m$^2$.
\end{itemize}

\begin{enumerate}[resume]
\item  Déterminer le nombre de rouleaux de laine de roche qu'il doit acheter pour le toit de sa terrasse.
\item Montrer que la longueur CD du toit de la partie habitable est égale à $8,2$~m.

\textbf{Rédiger la réponse en faisant apparaître les différentes étapes.}
\end{enumerate}

On considère que :

\begin{itemize}
\item le toit de la partie habitable est un rectangle de longueur $12$~m et de largeur $8,2$~m ;
\item Matthieu souhaite installer de la ouate de cellulose sur une épaisseur de $10$~cm ; 
\item la densité de la ouate de cellulose est de $40$ kg/m$^3$.
\end{itemize}
\begin{enumerate}[resume]
\item Déterminer la masse, en kg, de ouate de cellulose qu'il doit acheter pour le toit de la partie habitable.

\textbf{Toute trace de recherche, même non aboutie, sera prise en compte.}
\end{enumerate}

\bigskip

