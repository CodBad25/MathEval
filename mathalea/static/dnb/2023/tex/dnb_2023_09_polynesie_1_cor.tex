
\medskip

Cet exercice est un questionnaire à choix multiples (QCM).

Pour chaque question, parmi les réponses proposées, une seule est exacte.

%Recopier le numéro de la question et indiquer la réponse choisie avec la justification.

\medskip

\begin{center}
\begin{tabularx}{\linewidth}{|m{7.5cm}|*{3}{>{\centering \arraybackslash}X|}}\hline
\multicolumn{1}{|c|}{Questions}&Réponse A&Réponse B&Réponse C\\ \hline
\textbf{1.~} Une augmentation de 9\,\% correspond à une multiplication par \ldots&1,9&$\dfrac{9}{100}$&$\blue 1,09$\\ 
\hline
\textbf{2.~} On considère la figure ci-dessous:

\psset{unit=1cm}
\begin{pspicture}(-1.8,-0.4)(5,2.5)
\pspolygon(0,0)(3.7;0)(4.7;25)%ABC
\psline(1.1;0)(1.3973;25)%DE
\uput[dl](0,0){A} \uput[dr](3.7;0){B} \uput[ur](4.7;25){C} \uput[d](1.1;0){D} \uput[u](1.3973;25){E} 
\end{pspicture}

On précise que :

\begin{itemize}
\item[$\bullet~$] (DE) et (BC) sont parallèles;
\item[$\bullet~$] E est un point de [AC];
\item[$\bullet~$] D est un point de [AB];
\item[$\bullet~$] AE = 2~cm, EC = 5~cm, ED = 3~cm.
\end{itemize}

Quelle est la longueur BC ?&7,5~cm&6~cm&$\blue 10,5~\text{cm}$\\ \hline
\textbf{3.~} Le tableau ci-dessous donne la répartition des élèves de 5\up{e} d'un collège en fonction du sexe et de la langue vivante 2 choisie :

\begin{tabular}{|l|*{3}{c|}}\hline
\multicolumn{1}{|l|}{~}	&Allemand 	& Espagnol 	& Italien\\ \hline
Filles 					&10 		&43 		&26 \\ \hline
Garçons					&7			&42			&32\\ \hline
\end{tabular}

On interroge au hasard un élève de 5\up{e} parmi tous les élèves de 5\up{e} de ce collège.

Quelle est la probabilité que l'élève interrogé ait choisi l'italien en deuxième langue vivante ?
&$\dfrac{1}{3}$	&$\blue \dfrac{58}{160}$	&$\dfrac{58}{102}$\\ \hline
\textbf{4.~} On reprend la situation de la question \textbf{3.~}  et on interroge au hasard un élève de 5\up{e} parmi tous les élèves de 5\up{e} de ce collège.

Quelle est la probabilité que l'élève interrogé soit une fille qui ne fait pas d'allemand ?
&$\dfrac{69}{79}$	&$\dfrac{69}{143}$	&$\blue\dfrac{69}{160}$\\ \hline
\end{tabularx}
\end{center}

\begin{enumerate}
\item Augmenter de $t\,\%$, c'est multiplier par $1+\dfrac{t}{100}$, donc augmenter de $9\,\%$, c'est multiplier par $1+\dfrac{9}{100}$, soit $1,09$.
\hfill\textbf{Réponse C}

\item $\text{AC} = \text{AE} + \text{EC}$ donc $\text{AC}=2+5=7$. 

D'après les hypothèses, on peut appliquer le théorème de Thalès aux triangles ABC et ADE; on a donc $\dfrac{\text{BC}}{\text{DE}} = \dfrac{\text{AC}}{\text{AE}}$, c'est-à-dire $\dfrac{\text{BC}}{3}=\dfrac{7}{2}$, et donc $\text{BC}=\dfrac{21}{2} =10,5$.
\hfill\textbf{Réponse C}

\item $10+7+43+42+26+32=160$ donc il y a 160 élèves de 5\ieme{} dans ce collège.

$26+32=58$ donc il y a 58 élèves qui ont choisi l'italien en 2\ieme{} langue vivante.

On interroge au hasard un élève de 5\up{e} parmi tous les élèves de 5\up{e} de ce collège donc il y a équiprobabilité. 
La probabilité que l'élève interrogé ait choisi l'italien en deuxième langue vivante est donc $\dfrac{58}{160}$.
\hfill\textbf{Réponse B}

\item $43+26=69$ donc il y a 69 filles qui ne font pas d'allemand.
La probabilité que l'élève interrogé soit une fille qui ne fait pas d'allemand est donc $\dfrac{69}{160}$.
\hfill\textbf{Réponse C}
\end{enumerate}

\bigskip

