
\medskip

Un centre de loisirs dispose d'un bâtiment et d'un espace extérieur pour accueillir des enfants.

\begin{minipage}{0.5\linewidth}
L'espace extérieur, modélisé par un triangle, est partagé en deux parties: un potager (quadrilatère DEFG hachuré) et une zone de jeux (triangle EFC), comme représenté par la figure ci-contre.\\\\
\begin{list}{\textbullet}{Données:}
\item Les points C, E et D  sont alignés.
\item Les points C, F et G  sont alignés.
\item Les droites (EF) et (DG) sont parallèles.
\item Les droites (DG) et (CD) sont perpendiculaires.
\item $\text{CE}=30$~m; $\text{ED}=10$~m et $\text{DG}=24$~m.
\end{list}
\end{minipage}
%\hfill
\begin{minipage}{0.4\linewidth}
\rotatebox{-20}{
\psset{unit=1cm,radius=1pt}
\def\xmin {-1}   \def\xmax {5}
\def\ymin {-1}   \def\ymax {5}
\begin{pspicture}(\xmin,\ymin)(\xmax,\ymax)
%\psgrid[subgriddiv=10, gridcolor=gray] 
%\psaxes[arrowsize=3pt 3, ticksize=-2pt 2pt, labels=none](0,0)(\xmin,\ymin)(\xmax,\ymax)
%\uput[dl](0,0){$O$}
\Cnode*(0,0){A} \Cnode*(2,0){D} \Cnode*(2,4){C} \Cnode*(0,4){B}
\Cnode*(4.4,0){G} \Cnode*(2,1){E} \Cnode*(3.8,1){F}
\psframe[fillstyle=solid,fillcolor=lightgray](A)(C)
\psline(C)(G)
\pspolygon[fillstyle=hlines,hatchwidth=0.5pt](D)(E)(F)(G)
\psframe(A)(0.2,0.2) \psframe(D)(2.2,0.2) \psframe(B)(0.2,3.8) \psframe(C)(1.8,3.8) 
\uput[l]{20}(A){A} \uput[d]{20}(D){D} \uput[r]{20}(G){G} \uput[l]{20}(E){E} 
\uput[ur]{20}(F){F} \uput[ul]{20}(B){B} \uput[ur]{20}(C){C} 
\uput[d]{20}(1.1,2.5){bâtiment} \rput{20}(2.5,2){zone} \uput[d]{20}(2.75,2){de jeux}
 \rput{20}(3,-0.6){potager} \psline{->}(3,-0.4)(3.2,0.55)
\end{pspicture}
}% fin du rotate
\end{minipage}

\begin{enumerate}
\item %Déterminer la longueur CD.
On a CD = CE + ED $ = 30 + 10 = 40$~(m).
\item %Calculer la longueur CG. Arrondir au dixième de mètre près.
Le théorème de Pythagore appliqué au triangle CDG rectangle en D s'écrit :

CG$^2 = \text{CD}^2 + \text{DG}^2 = 40 ^2+ 24^2 = \np{1600} + 576 = \np{2176}$.

Donc CG $ = \sqrt{\np{2176}} \approx 46,64$, soit 46,4~(m) au décimètre premier.
\item %L'équipe veut séparer la zone de jeux et le potager par une clôture représentée par le segment [EF].\\
%Montrer que la clôture doit mesurer 18~m.
Les droites (DE) et (GF) sont sécantes en C et les droites (EF) et (DG) sont parallèles. le théorème de Thalès permet d'écrire :

$\dfrac{\text{CE}}{\text{CD}} = \dfrac{\text{EF}}{\text{DG}}$ soit $\dfrac{30}{40} = \dfrac{\text{EF}}{24}$. On en déduit EF $= 24 \times \dfrac{30}{40} = 24 \times \dfrac34 = 6 \times 3 = 18$~(m).
\item %Pour semer du gazon sur la zone de jeux, l'équipe décide d'acheter des sacs de 5~kg de graines à $22,90$~€ l'unité. Chaque sac permet de couvrir une surface d'environ 140~m$^2$.\\
%Quel budget doit-on prévoir pour pouvoir semer du gazon sur la totalité de la zone de jeux?
L'aire de la zone  de jeux est égale à :

$\mathcal{A}(\text{CEF}) = \dfrac{\text{CE} \times \text{EF}}{2} = \dfrac{30 \times 18}{2} = 30 \times 9 = 270$~(m$^2$).

Avec deux sacs on peut donc ensemencer l'aire de jeux ; il faut donc prévoir un budget de $2 \times 22,90 = 45,80$~\euro.
\item %La direction du centre affirme que la surface du potager est plus grande que celle de la zone de jeux. A-t-elle raison?
On a $\mathcal{A}(\text{CDG}) = \dfrac{\text{CD} \times \text{DG}}{2} = \dfrac{40 \times 24}{2} = 40 \times 12 = 480$~(m$^2$).

Par différence on a : $\mathcal{A}(\text{DEFG}) = \mathcal{A}(\text{CDG}) - \mathcal{A}(\text{CEF}) = 480 - 270 = 210$~(m$^2$).

On a $210 < 280$, donc la direction du centre a tort.
\end{enumerate}

\bigskip

