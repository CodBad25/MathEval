
\medskip

Dans cette exercice, on étudie la probabilité de gain des deux jeux ci-dessous.

\bigskip

\textbf{Partie A }

\medskip

\begin{tabularx}{\linewidth}{X|X}
Jeu 1&Jeu 2\\
Un sac contient cinq boules indiscernables au toucher, dont une portant la lettre N, deux, portant la lettre G et deux portant la lettre P.& Une roue à six  secteurs angulaires identiques numérotées de un à six.\\
\qquad\includegraphics[scale=0.5]{Jeu1}&\includegraphics[scale=0.5]{Jeu2}
\end{tabularx}

\medskip

\begin{enumerate}
\item On considère le jeu 1.

On pioche une boule au hasard dans ce sac et on note la lettre inscrite sur la boule choisie.

On considère qu'on a gagné si on pioche la lettre G.

Montrer que la probabilité de gagner avec ce jeu est de $\dfrac25$.
\item On considère le jeu 2.

On fait tourner la roue et on note le nombre d'inscrits sur le secteur pointé par la flèche. 

On considère qu'on a gagné si on s'arrête sur un nombre premier.

Quelle est la probabilité de gagner à ce jeu ?
\item 
	\begin{enumerate}
		\item Quel est le jeu qui présente la plus faible probabilité de gagner ?
		\item Proposer une liste de boules à rajouter pour que la probabilité de gagner avec le jeu 1 soit de $\dfrac14$.
	\end{enumerate}
\end{enumerate}

\bigskip

\textbf{Partie B}

\medskip

\textbf{Dans cette partie, toute trace de recherche sera valorisée.}

\medskip

On choisit finalement de combiner ces deux jeux.

Dans un premier temps, le joueur doit tirer une boule dans le sac du jeu 1.

On doit ensuite faire tourner la roue du jeu 2.

Le joueur gagne un lot s'il a tiré une boule portant la lettre G et si la roue s'arrête sur un secteur angulaire dont le numéro est un nombre premier.

Quelle est la probabilité de gagner à cette combinaison des deux jeux ?

\bigskip

