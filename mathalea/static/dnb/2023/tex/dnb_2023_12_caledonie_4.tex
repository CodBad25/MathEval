
\medskip

À quelques kilomètres au nord du village de Hienghène, se trouve une des plus belles
randonnées de Nouvelle-Calédonie appelée \og les roches de la Ouaïème \fg.

Le départ se situe au niveau de la mer près d'une plage de sable blanc.

Le sentier grimpe le long d'un versant de montagne et atteint un point de vue imprenable sur le Mont Panié et le lagon.

Voici quelques informations pratiques sur cette randonnée:

\begin{center}
\begin{tabularx}{0.8\linewidth}{|m{5cm}|X|}\hline
Durée estimée (Aller simple)&2 h 30 min\\
Distance (Aller simple)&3,8 km\\ \hline
Altitude&minimale : 0 m / maximale : 741 m\\ \hline
\end{tabularx}
\end{center}

On considère que la pente de la montagne est rectiligne.

On a schématisé le parcours [DV] de la randonnée par la figure ci-dessous:

\begin{center}
\psset{unit=1cm,arrowsize=2pt 3}
\begin{pspicture}(-1,-0.4)(15,6.)
%\psgrid
\rput(7.5,6){Les points D, N et M sont alignés}
\psline(0,0)(12,4.2)
\psline[linestyle=dashed](0,0)(12,0)(12,4.2)%DMV
\psline[linestyle=dashed](8,0)(8,2.8)%NP
\psframe(8,0)(7.75,0.25)\psframe(12,0)(11.75,0.25)
\uput[d](0,0){D} \uput[d](8,0){N} \uput[dr](12,0){M} \uput[ur](12,4.2){V} \uput[ul](8,2.8){P}
\uput[r](12,2.1){0,741 km}\rput{20}(4,1.65){3 km}\rput{20}(5.7,3.2){3,8 km}
\psdots(0,0)(12,0)(12,4.2)(8,0)(8,2.8)
\psline[linewidth=0.5pt]{<->}(-0.3,0.8)(11.7,5)
\rput(0.1,0.4){Le départ}\rput(12.4,4.8){L'arrivée}\rput(8.8,2.6){Panneau}
\end{pspicture}
\end{center}

Fabienne s'est engagée sur ce parcours en partant du point D.

Au bout de 2 heures, elle arrive au panneau P indiquant qu'elle a déjà parcouru $3$~km.

\begin{enumerate}
\item Justifier que les droites (PN) et (VM) sont parallèles.
\item Déterminer à quelle altitude PN se trouve Fabienne lorsqu'elle se situe au panneau P.

\textbf{Rédiger la réponse en faisant apparaître les différentes étapes.}
\item À quelle vitesse moyenne, en km/h, a-t-elle parcouru le trajet [DP] 
?

Sur la fin du parcours [PV], Fabienne marche à une vitesse moyenne de 1,2 km/h.

On rappelle que la durée de l'aller simple est estimée à 2~h 30~min.
\item A-t-elle dépassé cette durée ?

\textbf{Justifier en faisant apparaître les différentes étapes.}
\end{enumerate}

\bigskip

