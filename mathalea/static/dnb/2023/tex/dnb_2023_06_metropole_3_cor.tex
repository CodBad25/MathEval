
\medskip

Cet exercice est un questionnaire à choix multiple (QCM).

%Pour chaque question, trois réponses (A, B ou C) sont proposées.
%
%\textbf{Une seule réponse est exacte.}
%
%\textbf{Recopier sur la copie} le numéro de la question et la réponse choisie. Aucune justification n'est demandée.

\medskip

%\begin{tabularx}{\linewidth}{|m{6.5cm}|*{3}{>{\centering \arraybackslash}X|}} \hline
%	\centering\textbf{Question}& \textbf{Réponse A}& \textbf{Réponse B}& \textbf{Réponse C}\\ \hline
%	\textbf{1.} Dans une classe de 25 élèves, $60 \%$ des élèves sont des filles.
%
%	Combien y a-t-il de filles dans cette classe ?
%	& 10 & 15 & 20 \\ \hline
%
%	\textbf{2.} Quelle est la décomposition en produit de facteurs premiers de 126 ?
%	& $2 \times 9 \times 7$ & $2^2 \times 5^2 + 2 \times 13$ &$2 \times 3^2 \times 7$ \\ \hline
%
%	\textbf{3.} Dans un sac, il y a 17 jetons rouges, 23 jetons jaunes et 20 jetons bleus, tous indiscernables au toucher. On tire au hasard un jeton du sac.
%
%	Quelle est la probabilité d'obtenir un jeton rouge ou un jeton jaune?
%	&$\dfrac{2}{3}$ & 0,6 & $\dfrac{17}{23}$  \\ \hline
%
%	\textbf{4.} Sur l'octogone régulier ci-dessous, quelle est l'image du segment [DC] par la rotation de centre O qui transforme A en D?
%
%	\hfill~	\begin{tikzpicture}[]
%		\foreach \a/\n in {1/A, 2/B, 3/C, 4/D, 5/E, 6/F, 7/G, 8/H}
%			\draw (0,0)--(-100+45*\a:1.5) node[shift={(-100+45*\a:0.3)}]{\n}--(-55+45*\a:1.5);
%		\node at (-0.1,-0.2){O};
%	\end{tikzpicture}\hfill~
%	& [GE] & [GF] & [AH]\\ \hline
%
%	\textbf{5.} Quel est le volume d'un pavé droit de hauteur \np[m]{1,5} et de base rectangulaire de \np[m]{2} de longueur et \np[m]{1,3} de largeur ?
%
%	{\small \emph{On rappelle que $\np[m^3]{1}=\np[L]{1000}$.}}
%	&\np[m^3]{2,6} & \np[L]{3900}& \np[L]{3000} \\ \hline
%\end{tabularx}
\begin{enumerate}
\item On a $25 \times \dfrac{60}{100} = 25 \times 0,6 = 15$. Réponse B.
\item $126 = 2 \times 63 = 2 \times 9 \times 7 = 2 \times 3^2 \times 7$. Réponse C.
\item Il y a $17 + 23 = 40$ jetons rouges ou jaunes. la probabilité est donc égale à $\dfrac{40}{17 + 23 + 20} = \dfrac{40}{60} = \dfrac46 = \dfrac23$. Réponse A.
\item Chacun des angles au centre de l'octogone a une mesure égale à $\frac{360}{8} = 45$\degres{}.
La rotation transformant A en D est donc une rotation de $3 \times 45 = 135$\degres{} dans le sens anti-horaire.

D a pour image G et C a pour image F, donc [DC] a pour image [GF]. Réponse B.
\item Le volume est égal à $2 \times 1,5 \times 1,3 = 3 \times 1,3 = 3,9$~(m$^3$), soit $3,9 \times \np{1000} = \np{3900}~$L. Réponse B.
\end{enumerate}

\vspace{5mm}

