
\medskip

%Cet exercice est un questionnaire à choix multiples (QCM).
%
%Pour chaque question, une seule des trois réponses proposées est exacte.
%
%\textbf{Sur la copie, indiquer le numéro de la question et la réponse A, B ou C choisie.\\
%Aucune justification n'est demandée.}
%
%Aucun point ne sera enlevé en cas de mauvaise réponse.
%
%\begin{center}
%\begin{tabularx}{\linewidth}{|c|m{8cm}|*{3}{>{\centering \arraybackslash}X|}}\hline
%&\textbf{Questions}&\textbf{Réponse A}&\textbf{Réponse B}&\textbf{Réponse C}\\ \hline
%1 &D'après des chercheurs, la probabilité qu'une personne subisse une attaque mortelle par un requin au cours de sa vie, est de ...&$2,7 \times 10^{-7}$&$2,7\times 10^0$&$2,7 \times 10^7$\\ \hline
%2&$\dfrac35 - \dfrac25 \times \dfrac74$\rule[-4mm]{0mm}{10mm}&$- \dfrac{1}{10}$&$\dfrac{2}{10}$&$\dfrac{7}{20}$\rule[-4mm]{0mm}{10mm}\\ \hline
%3&Sur un site, un pantalon est vendu 60~\euro{} au lieu de 80~\euro.
%
%Le pourcentage de réduction est ...&20\,\%&25\,\%&75\,\%\\ \hline
%4&ABCD est un parallélogramme de centre E.
%
%\begin{center}
%\psset{unit=1cm}
%\begin{pspicture}(7.4,2.3)
%\pspolygon(0.4,0.3)(5.5,0.3)(7.1,2.1)(2,2.1)%ADCB
%\psline(0.4,0.3)(7.1,2.1)\psline(5.5,0.3)(2,2.1)
%\psdots(1.2,1.2)(2.95,0.3)
%\uput[dl](0.4,0.3){A} \uput[ul](2,2.1){B} \uput[ur](7.1,2.1){C} \uput[dr](5.5,0.3){D} \uput[u](3.75,1.2){E} \uput[d](1.2,1.2){F}\uput[d](2.95,0.3){G}
%\psline(1.65,0.2)(1.65,0.4)\psline(1.7,0.2)(1.7,0.4)\psline(1.75,0.2)(1.75,0.4)
%\psline(4.2,0.2)(4.2,0.4)\psline(4.25,0.2)(4.25,0.4)\psline(4.3,0.2)(4.3,0.4)
%\rput(0.8,0.75){$\backslash$}\rput(1.7,1.73){$\backslash$}
%\end{pspicture}
%\end{center}
%
%L'homothétie de centre A qui transforme B en F ...&a pour rapport 2.&trans\-forme G en D.&trans\-forme C en E.\\ \hline
%5&La médiane de la série ci-dessous est ...
%\[11-17-8-14-3-20-5-10-12\]&3&5&11\\ \hline
%\end{tabularx}
%\end{center}

\begin{enumerate}
\item Réponse A : c'est la seule inférieure à 1.
\item $\dfrac35 - \dfrac25 \times \dfrac74 = \dfrac35 - \dfrac{7}{10} = \dfrac{6}{10} - \dfrac{7}{10} = - \dfrac{1}{10}.$ Réponse A.
\item La réduction est de 20 \euro{} sur un prix initial de 80~\euro{}, soit $\dfrac{20}{80} = \dfrac{1 \times 20}{4 \times 20} = \dfrac14 = \dfrac{1 \times 25}{4 \times 25} = \dfrac{25}{100}$, c'est-à-dire 25\,\%.  Réponse B.
\item On sait que ABCD est un parallélogramme donc E est le milieu de la diagonale [AC] : AE $= \dfrac12$ AC, donc l'homothétie de centre A qui transforme B en F transforme C en E puisque F est le milieu de [AB]. Réponse C.
\item Il y a 9 valeurs donc la 5\up{e} (de la suite ordonnée 3 - 5 -  8 - 10 - {\red 11}, etc.) partage la série de notes en deux groupes de même importance. Réponse C.
\end{enumerate}

\bigskip

