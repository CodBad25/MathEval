
\medskip

%La construction du Centre Aquatique Olympique de Saint-Denis a débuté en 2021 pour accueillir les épreuves de natation artistique des Jeux olympiques de Paris 2024.
%
%Alyssa et Jules visitent le Centre Aquatique Olympique et s'installent dans les gradins.
%
%On a schématisé leurs positions par rapport à la piscine olympique sur la figure ci-dessous, qui modélise la situation : Alyssa est installée dans les gradins Nord au point A et Jules est assis dans les gradins Sud au point J.
%
%\emph{La figure n'est pas à l'échelle.}
\begin{center}

\psset{unit=1cm}

\begin{pspicture}(0,-0.8)(10.8,2.6)

%\psgrid

\pspolygon(0.2,0)(10.5,0)(10.5,2.3)(6.8,0)(3.6,0)(0.2,1.9)%CEFDBAC

\psframe[fillstyle=vlines](6.8,0)(3.8,-0.4)

\psline(10.5,1.35)(8.93,1.35)

\psframe(0.2,0)(0.4,0.2)\psframe(10.5,0)(10.3,.20)\psframe(10.5,1.35)(10.3,1.55)

\uput[d](1.7,0){Gradins Nord}\uput[d](8.65,0){Gradins Sud}\uput[d](5.3,-0.4){Piscine olympique}

\uput[ul](0.2,1.9){A} \uput[ur](3.6,0){B} \uput[dl](0.2,0){C} \uput[ul](6.8,0){D}

\uput[dr](10.5,0){E} \uput[ur](10.5,2.3){F} \uput[r](10.5,1.35){H} \uput[ul](8.93,1.35){J}

\end{pspicture}

\end{center}



%On donne : AC = FJ $= 15$ m ; BC $= 27$ m ; FH $= 7$ m ; EF $= 18$ m.
%
%Les points F{}, J et D sont alignés.
%
%Les points F{}, H, et E sont alignés.
%
%Les points C, B, D, E sont alignés.
%
%\medskip

\begin{enumerate}
\item %Jules et Alyssa discutent entre eux pour savoir qui est le mieux placé pour assister à l'événement.
	\begin{enumerate}
		\item %Calculer la distance entre Alyssa et le bord de la piscine, c'est-à-dire calculer la longueur AB.
		Dans le triangle ABC rectangle en C, le théorème de Pythagore permet d'obtenir :
		
		AB$^2 = $ AC$^2 + $ CB$^2 = 15^2 + 27^2 = 225 + 729 = 954$, d'où QB $= \sqrt{954} = \sqrt{9 \times 106} = \sqrt{6} \times \sqrt{106} = 3\sqrt{6} \approx 30,9$~(m) soit 31~(m) au mètre près.
		
%Arrondir le résultat au mètre près.
		\item %Vérifier que la distance entre Jules et le bord de la piscine, c'est-à-dire la longueur JD, est de $24$~m, arrondie au mètre près.
		D'après la figure les droites (JH) et (DE) toutes deux perpendiculaires à la droite (EF) sont parallèles.
		
Avec l'alignement respectif des points F{}, J D d'une part et F{}, H et E de l'autre nous avons don une configuration de Thalès qui permet d'écrire en particulier :

$\dfrac{\text{FJ}}{\text{FD}} = \dfrac{\text{FH}}{\text{FE}}$ ou encore $\dfrac{15}{\text{FD}} = \dfrac{7}{18}$ d'où $15 \times 18 = 7 \text{FD} \iff \text{FD} : \dfrac{15 \times 18}{7}$.

On en déduit que JD $ = \text{FD} - \text{FJ}$, soit JD $= \dfrac{15 \times 18}{7} - 15 = \dfrac{15 \times 18 - 15 \times 7}{7} = \dfrac{15 \times, 11}{7} \approx 23,6$, soit environ 24~(m).
		\item %En déduire lequel des deux amis est le plus proche d'un bord de la piscine.
Jules est donc le plus proche de la piscine.
	\end{enumerate}
\item %Pour respecter les normes de sécurité, l'angle d'inclinaison $\widehat{\text{ABC}}$ des gradins Nord ne doit pas dépasser $35\degres{}$. Les gradins Nord respectent-ils cette norme ?
Dans le triangle ABC rectangle en C, on a : $\tan \widehat{\text{ABC}} = \dfrac{\text{AC}}{\text{BC}} = \dfrac{15}{27} = \dfrac59$.

La calculatrice donne $\widehat{\text{ABC}} \approx 29,1 < 35$ : la norme est respectée
\item %Le toit du Centre Aquatique Olympique a une surface de \np{5000}~m$^2$.

%On estime que \np{4678,4}~m$^2$ de ce toit sont recouverts de panneaux photovoltaïques.

%Voici les caractéristiques d'un panneau photovoltaïque standard fournies par le constructeur:

%\begin{center}
%\begin{tabularx}{\linewidth}{|m{5cm} X|}\hline
%\psset{unit=1cm,arrowsize=2pt 3}
%\begin{pspicture}(4.8,2.4)
%\psframe*[linecolor=gray](0.7,0)(4.4,1.5)
%\multido{\na=0.20+0.35}{4}
%{\multido{\n=0.80+0.37}{10}{\pscircle[fillstyle=solid,fillcolor=white](\n,\na){0.05}}}
%\psline[linewidth=0.5pt]{<->}(0.7,1.6)(4.4,1.6)\uput[u](2.55,1.6){1,7~m}
%\psline[linewidth=0.5pt]{<->}(0.6,0)(0.6,1.5)\rput{90}(0.4,0.75){1~m}
%\end{pspicture}&\textbf{Dimensions :} 1 m de large et 1,7 m de long 
%
%\textbf{Énergie produite:} environ $350$~kWh par an\\ \hline
%\end{tabularx}
%\end{center}

%Montrer que la quantité annuelle d'énergie produite par l'ensemble des panneaux photovoltaïques du toit du Centre Aquatique Olympique est de \np{963200} kilowattheures (kWh).
Un panneau a une aire de 1,7~m$^2$, donc  $\dfrac{\np{4678,4}}{1,7}  = \np{2752}$ est le nombre de panneaux.

Ces \np{2752} panneaux produiront $\np{2752} \times 350 = \np{963200}$~(kWh) par an.
\item %La température règlementaire de l'eau contenue dans la piscine lors des jeux Olympiques doit être comprise entre 25\degres{} et 28\degres{}. Pour respecter cette réglementation, on souhaite que l'eau contenue dans la piscine olympique de Saint-Denis soit à une température de 26\degres{}. On admet que l'eau contenue dans cette piscine occupe un pavé droit dont les dimensions sont:

%\begin{itemize}
%\item Longueur : 50 m
%\item Largeur: 25 m
%\item Profondeur: 3 m
%\end{itemize}
%
%On suppose qu'avant la première mise en chauffe de la piscine olympique, l'eau est à 18\degres{}.
%
%On estime qu'il faut environ $9,3$ kWh pour chauffer 1 m$^3$ d'eau de 18\degres{} jusqu'à 26\degres{}.
%
%Quelle quantité d'énergie, en kWh, sera nécessaire pour chauffer toute l'eau de la piscine olympique jusqu'à 26\degres{} ?
Le volume d'eau dans  la piscine est : $50 \times 25 \times 3 = \np{1250} \times 3 = \np{3750}~\left(\text{m}^3\right)$.

Chaque m$^3$ d'eau nécessitant 9,3~kWh, il faudra pour chauffer la piscine :

\[\np{3750} \times 9,3 = \np{34875}~(\text{kWh}).\]

\end{enumerate}

\bigskip

