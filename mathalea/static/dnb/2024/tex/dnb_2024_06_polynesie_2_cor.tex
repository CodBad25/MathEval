
\medskip

%Les Jeux olympiques (JO) d'été ont généralement lieu tous les 4 ans.
%
%Dans cet exercice, on s'intéresse aux coûts d'organisation des dernières éditions des JO d'été.
%
%On rappelle que le coût est l'ensemble des dépenses entraînées par l'organisation des JO.
%
%On précise que :
%
%\begin{itemize}
%\item le \textbf{coût prévisionnel} désigne les dépenses prévues par les organisateurs avant l'édition des JO ;
%\item le \textbf{coût réel} désigne les dépenses réelles qui ont été nécessaires pour l'organisation des JO.
%\end{itemize}
%
%Le graphique ci-dessous compare ces deux coûts pour les dernières éditions des JO d'été.
%
%\medskip
%
%\begin{tabularx}{\linewidth}{|>{\centering \arraybackslash}X|}\hline
%\textbf{Comparaison entre le coût prévisionnel et le coût réel de chaque édition des JO depuis 1992, en milliard d'euros}\\
%
%\psset{xunit=0.9cm,hatchsep=2pt,yunit=1.25cm}
%\begin{pspicture}(0,-1)(15,4.5)
%%\psgrid
%\psline(0,1.4)(15,1.4)
%\psframe*[linecolor=gray](0.5,1.4)(0.7,1.575)\uput[u](0.6,1.575){\footnotesize 3,5}
%\psframe*[linecolor=gray](2.3,1.4)(2.5,1.49)\uput[u](2.4,1.49){\footnotesize 1,8}
%\psframe*[linecolor=gray](4.1,1.4)(4.3,1.55)\uput[u](4.2,1.55){\footnotesize 3}
%\psframe*[linecolor=gray](5.9,1.4)(6.1,1.65)\uput[u](6.,1.65){\footnotesize 5,3}
%\psframe*[linecolor=gray](7.7,1.4)(7.9,1.53)\uput[u](7.8,1.53){\footnotesize 2,6}
%\psframe*[linecolor=gray](9.5,1.4)(9.7,1.64)\uput[u](9.6,1.64){\footnotesize 4,8}
%\psframe*[linecolor=gray](11.3,1.4)(11.5,1.85)\uput[u](11.4,1.85){\footnotesize 9}
%\psframe*[linecolor=gray](13.1,1.4)(13.3,2.05)\uput[u](13.2,2.05){\footnotesize 13}
%\psframe[fillstyle=hlines](1,1.4)(1.2,1.865)\uput[u](1.1,1.865){\footnotesize 9,3}
%\psframe[fillstyle=hlines](2.8,1.4)(3,1.55)\uput[u](2.9,1.55){\footnotesize 2,3}
%\psframe[fillstyle=hlines](4.6,1.4)(4.8,1.65)\uput[u](4.7,1.65){\footnotesize 5,5}
%\psframe[fillstyle=hlines](6.4,1.4)(6.6,1.9)\uput[u](6.5,1.9){\footnotesize 10}
%\psframe[fillstyle=hlines](8.2,1.4)(8.4,3.3)\uput[u](8.3,3.3){\footnotesize 31}
%\psframe[fillstyle=hlines](10,1.4)(10.2,1.9)\uput[u](10.1,1.9){\footnotesize 11}
%\psframe[fillstyle=hlines](11.8,1.4)(12,2.2)\uput[u](11.9,2.2){\footnotesize 16,5}
%\psframe[fillstyle=hlines](13.6,1.4)(13.8,2)\uput[u](13.7,2){\footnotesize 12,1}
%\psframe*[linecolor=gray](5,4.1)(5.3,4.4)\psframe[fillstyle=hlines](10,4.1)(10.3,4.4)
%\rput(7.2,4.2){Coût prévisionnel}\rput(11.8,4.2){Coût réel}
%\rput(0.8,1){Barcelone}\rput(0.8,0.5){1992}
%\rput(2.6,1){Atlanta}\rput(2.6,0.5){1996}
%\rput(4.4,1){Sydney}\rput(4.4,0.5){2000}
%\rput(6.2,1){Athènes}\rput(6.2,0.5){2004}
%\rput(8,1){Pékin}\rput(8,0.5){2008}
%\rput(9.8,1){Londres}\rput(9.8,0.5){2012}
%\rput(11.6,1){Rio de}\rput(11.6,0.5){Janeiro}\rput(11.6,0){2016}
%\rput(13.4,1){Tokyo}\rput(13.4,0.5){2021}
%\rput(7.5,-0.5){\emph{La crise sanitaire de la Covid-$19$ a décalé à $2021$ les Jeux Olympiques de Tokyo prévus en $2020$.}}
%\end{pspicture}\\ \hline
%\end{tabularx}
%
%\medskip

\begin{enumerate}
\item %Entre 1992 et 2021, combien d'éditions ont eu un coût réel supérieur ou égal à $10$ milliards d'euros ?
4 éditions (5 en comptant Tokyo)  ont eu un coût réel supérieur ou égal à $10$ milliards d'euros
\item %Calculer le pourcentage d'augmentation entre le coût prévisionnel et le coût réel lors de l'édition des JO de Rio de Janeiro 2016, arrondi à l'unité.
Le pourcentage d'augmentation entre le coût prévisionnel et le coût réel lors de l'édition des JO de Rio de Janeiro 2016 est égal à : $\dfrac{16,5 - 9}{9} \times 100 \approx 83,3\,\%$ soit environ 83\,\% à l'unité près.
\item %Montrer que le coût réel moyen entre 1992 et 2021 est $12,2$~milliards d'euros, arrondi au dixième de milliard.
Moyenne du coût réel de 1992 à 2021 :

$\dfrac{9,3 + 2,3 + 5,5 + 10 + 31 + 11 + 16,5 + 12,1}{8} = \dfrac{97,7}{8} = \np{12,2125}$, soit 12,2 au dixième de milliard près.
\item %\textbf{Questions de journalistes}
	\begin{enumerate}
		\item %Un journaliste mentionne que le coût réel moyen des JO sur la période 1992 à 2021 est de $12,2$~milliards d'euros. Il poursuit en affirmant: \og Cela signifie que la moitié des éditions entre 1992 et 2021 ont un coût réel supérieur à 12,2 milliards d'euros. \fg
		
%Que penser de cette affirmation ?
Le journaliste confond moyenne et médiane.
		\item %Le coût prévisionnel moyen entre 1992 et 2024 est de l'ordre de $5,5$ milliards d'euros.
		
%Une journaliste cherche à connaître le coût prévisionnel des JO de Paris 2024 pour préparer son intervention télévisée.
		
%Calculer le coût prévisionnel des JO de Paris 2024 qu'elle devrait annoncer.
En prenant en compte les budgets prévisionnels depuis (et non entre) 1992 jusqu'à 2024 et en nommant $p$ le coût prévisionnel des JOP(aris), on a donc pour calcul de la moyenne :

$\dfrac{3,5 + 1,8 + 3 + 5,3 + 2,6 + 4,8 + 9 + 13 + p}{9} = 5,5$, d'où :

$\dfrac{43,0 + p}{9} = 5,5$ puis $43 + p = 9 \times 5,5$ et $p = 9 \times 5,5 - 43 = 6,5$~(milliards d'euros).
	\end{enumerate}
\end{enumerate}

\bigskip

