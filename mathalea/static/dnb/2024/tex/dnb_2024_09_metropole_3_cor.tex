
\medskip

Une entreprise décide de faire poser sur le toit de son hangar des panneaux solaires.

Pendant une semaine d'utilisation, les productions d'électricité journalières en kilowattheures (kWh) de ces panneaux ont été relevées dans le tableau ci-dessous :

\begin{center}
\begin{tabular}{|m{2cm}|*{7}{c|}}\hline
Jour de la\newline semaine&Lundi&Mardi& Mercredi& Jeudi&Vendredi& Samedi&\small Dimanche\\ \hline
Production d'électricité en kWh&381&363 &322& 329&393& 405& 376\\ \hline
\end{tabular}
\end{center}

%\smallskip

\begin{enumerate}
\item
	\begin{enumerate}
		\item La production d'électricité a été la plus grande le samedi avec 405 kWh.
		\item La production d'électricité a été la plus petite le mercredi avec 322 kWh.
		
$405-322= 83$ donc l'étendue de ces productions d'électricité est 83 kWh.
		\item $\dfrac{381 + 363 + 322 + 329 + 393 + 405 + 376}{7} = \dfrac{\np{2569}}{7} = 367$ donc
		
la production moyenne d'électricité par jour sur cette période est de $367$ kWh.
	\end{enumerate}

\item L'entreprise revend 15\,\% de sa production d'électricité au tarif de 8 centimes le kWh.

La production sur la semaine a été de $\np{2569}$ kWh.

Les 15\,\% de cette production correspondent à $\np{2569}\times \dfrac{15}{100}= 385,35$.

$ 385,35\times 8 = \np{3082,8}$
donc elle a gagné $30,828$\,\euro{} pendant ces 7 jours.

\item Afin que les panneaux solaires aient une production maximale, le toit doit avoir une pente avec l'horizontale comprise entre $30\degres$ et $35\degres$.


%\begin{minipage}{0.42\linewidth}
%Schéma en coupe du hangar.
\begin{multicols}{2}
La pente du toit avec l'horizontale correspond à l'angle $\widehat{\text{OLV}}$.

Dans le triangle OLV rectangle en V, on a: $\sin \left (\widehat{\text{OLV}}\right ) = \dfrac{\text{OV}}{\text{OL}} = \dfrac{7}{13,5}$.

On en déduit que $\widehat{\text{OLV}} \approx 31,2\degres$.

$31,2$ est compris entre 30 et 35 donc, sur ce toit, les panneaux solaires ont une production maximale.
%\end{minipage}\hfill
%\begin{minipage}{0.55\linewidth}

\columnbreak

\scalebox{0.7}{
\psset{unit=1cm,arrowsize=2pt 3}
\begin{pspicture}(8.5,6.6)
%\psgrid
\pspolygon(2,0)(8.2,0)(8.2,3.9)(5.1,6.3)(2,3.9)
\psline(5.1,6.3)(5.1,3.9)\psframe(5.1,3.9)(5.3,4.1)
\psline(8.2,3.9)(2,3.9)
\psline[linewidth=2.5pt](4.7,6)(2.4,4.2)
\uput[d](5.1,3.9){V} \uput[r](8.2,3.9){K} \uput[u](5.1,6.3){O} \uput[l](2,3.9){L} 
\uput[r](5.1,5.1){7 m}\rput{37}(3.56,5.4){13,5 m}
\rput(1.8,5.8){panneaux solaires}\psline[linewidth=1.2pt]{->}(1.8,5.5)(3.2,4.9)
\end{pspicture}
}
%\end{minipage}
\end{multicols}
\end{enumerate}

\vspace{0.5cm}

