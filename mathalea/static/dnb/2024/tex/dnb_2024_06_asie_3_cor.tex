
\medskip

\begin{enumerate}
\item Comme les droites (AB) et (CG) sont perpendiculaires, le triangle BCG est un triangle rectangle en C.

Dans ce triangle rectangle en C, le théorème de Pythagore donne l'égalité suivante :

$\mathrm{BG}^2 = \mathrm{BC}^2 + \mathrm{CG}^2$.

En remplaçant par les valeurs connues, on a : \quad $20^2 = \mathrm{BC}^2 + 10^2$

Soit : \quad $\mathrm{BC}^2 = 20^2-10^2 = 400 - 100 = 300$.

Comme BC est une distance, c'est une valeur positive, donc :\quad $\mathrm{BC} = \sqrt{300} \approx 17,32$.

En arrondissant au millimètre, on a bien BC mesurant environ 17,3 cm.

\item Dans le triangle BAG, on va considérer [AB] comme la base, et [CG] est donc la hauteur associée à cette base.

La longueur AB est le double de la longueur BC, puisque C est le milieu de [AB].

L'aire du triangle est donc : \quad $\dfrac{\mathrm{AB}\times\mathrm{CG}}{2} = \dfrac{2\sqrt{300}\times 10}{2} = 10\sqrt{300} = 100\sqrt{3}\approx 173,2$.

À l'unité près, l'aire du triangle BAG est de \np[cm^2]{173}.

\item \begin{enumerate}
		\item Dans le triangle CGB, rectangle en C, on sait que :

		$\cos\left(\widehat{\mathrm{CGB}}\right)  = \dfrac{\mathrm{CG}}{\mathrm{GB}} = \dfrac{10}{20}=\dfrac{1}{2}=0,5$.

		Soit on sait que l'angle dont le cosinus vaut 0,5 est un angle de 60\degre, soit on utilise sa calculatrice en effectuant $\arccos(0,5)$ (ou $\cos^{-1}(0,5)$, selon le modèle de calculatrice), qui renvoie 60\degre.

		\item Le triangle AGB est un triangle isocèle en G (puisque l'arc de cercle passant par A et B a pour centre G, cela signifie que GA = GB, c'est le rayon de l'arc de cercle).

		La droite (CG) est donc la hauteur issue du sommet principal G, dans ce triangle ABG, isocèle en G.

		Cela signifie que la hauteur est aussi la bissectrice de l'angle $\widehat{\mathrm{AGB}}$.

		Ainsi, l'angle $\widehat{\mathrm{CGB}}$ est donc la moitié de l'angle $\widehat{\mathrm{AGB}}$.

		On a donc, d'après la question précédente : \quad $\widehat{\mathrm{AGB}}=2\times 60 = 120$\degre.
	\end{enumerate}
\item Les trois élèves peuvent effectivement former un disque complet avec leurs trois pièces, puisque un disque complet correspond à un arc de 360\degre, comme ils ont trois portions de disque avec des angles au centre de 120\degre, en mettant leur trois pièces ensemble ils vont reconstituer un angle au centre de $3\times 120 = 360$\degre, donc un disque complet.

\item En mettant les trois pièces, on a donc un disque, de centre G et de rayon $R=\np[cm]{20}$.

La surface de ce disque est donnée par :\quad $\pi \times R^2 = 400\pi$.

Comme ce disque est l'assemblage de trois pièces idendiques (et donc de même surface), chacune des trois pièce à une surface de :\quad $\dfrac{400\pi}{3}\approx 418,9$.

À l'unité près, l'aire de chaque pièce est de \np[cm^2]{419}
\end{enumerate}

