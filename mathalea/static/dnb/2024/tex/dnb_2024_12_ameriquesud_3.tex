
\medskip

\begin{enumerate}
\item 

Le tableau ci-dessous présente, pour quatre félins étudiés, les probabilités d'attraper leur proie quand ils la poursuivent.

\begin{center}
\begin{tabularx}{0.75\linewidth}{|*{2}{>{\centering \arraybackslash}X|}}\hline
Félin étudié &
Probabilité d'attraper la proie qu'il poursuit\\ \hline
Le lion&25\,\%\\ \hline
Le guépard&$\dfrac12$\rule[-10pt]{0pt}{28pt}\\ \hline
Le tigre&$0,1$\\ \hline
Le chat à pieds noirs&$\dfrac{6}{10}$\rule[-10pt]{0pt}{28pt}\\ \hline
\end{tabularx}
\end{center}

Vérifier que, parmi les quatre félins étudiés, le chat à pieds noirs a la probabilité la plus élevée d'attraper sa proie quand il la poursuit.
\item  Le plus souvent, le guépard est le félin le plus rapide avec une vitesse pouvant atteindre 115 km/h.
À cette vitesse, en combien de secondes le guépard parcourt-il 100 mètres?
On donnera une valeur approchée au centième de seconde près.

 Dans un pays d'Afrique, on estimait à :

\begin{itemize}[label=$\bullet~$]
\item \np{1200} guépards en 1999.
\item 170 guépards en 2016.
\end{itemize}

Dans ce pays, est-il vrai que le nombre de guépards a baissé d'environ 86\,\% entre 1999 et 2016 ?

\item Dans le parc national d'Etosha en Namibie, on peut observer des lions et des guépards.
À l'aide de la carte ci-dessous, donner approximativement la latitude et la longitude du parc national d'Etosha.

\includegraphics[width=14cm]{mappemonde.eps}

\end{enumerate}

\bigskip

