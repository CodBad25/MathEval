
\medskip

Une entreprise décide de faire poser sur le toit de son hangar des panneaux solaires.

Pendant une semaine d'utilisation, les productions d'électricité journalières en kilowattheures (kWh) de ces panneaux ont été relevées dans le tableau ci-dessous :

\begin{center}
\begin{tabularx}{\linewidth}{|m{2cm}|*{7}{>{\centering \arraybackslash}X|}}\hline
Jour de la semaine&Lundi&Mardi& Mercredi& Jeudi&Vendredi& Samedi&\small Dimanche\\ \hline
Production d'électricité en kWh&381&363 &322& 329&393& 405& 376\\ \hline
\end{tabularx}
\end{center}

\smallskip

\begin{enumerate}
\item
	\begin{enumerate}
		\item Quel jour la production d'électricité a-t-elle été la plus grande ?
		\item Calculer l'étendue de ces productions d'électricité. 
		\item Quelle est la production moyenne d'électricité par jour sur cette période ?
	\end{enumerate}
\item L'entreprise revend 15\,\% de sa production d'électricité au tarif de 8 centimes le kWh.

Combien a-t-elle gagné en euros pendant ces 7 jours ?
\item Afin que les panneaux solaires aient une production maximale, le toit doit avoir une pente avec l'horizontale comprise entre $30\degres$ et $35\degres$.

\begin{minipage}{0.35\linewidth}
Schéma en coupe du hangar.

La pente du toit avec l'horizontale correspond à l'angle $\widehat{\text{OLV}}$.

\end{minipage}\hfill
\begin{minipage}{0.62\linewidth}
\psset{unit=1cm,arrowsize=2pt 3}
\begin{pspicture}(8.5,6.6)
%\psgrid
\pspolygon(2,0)(8.2,0)(8.2,3.9)(5.1,6.3)(2,3.9)
\psline(5.1,6.3)(5.1,3.9)\psframe(5.1,3.9)(5.3,4.1)
\psline(8.2,3.9)(2,3.9)
\psline[linewidth=2.5pt](4.7,6)(2.4,4.2)
\uput[d](5.1,3.9){V} \uput[r](8.2,3.9){K} \uput[u](5.1,6.3){O} \uput[l](2,3.9){L} 
\uput[r](5.1,5.1){7 m}\rput{37}(3.56,5.4){13,5 m}
\rput(1.8,5.8){panneaux solaires}\psline[linewidth=1.2pt]{->}(1.8,5.5)(3.2,4.9)
\end{pspicture}
\end{minipage}

\medskip

Sur ce toit, les panneaux solaires ont-ils une production maximale?

\bigskip

\end{enumerate}
