
\medskip

\begin{enumerate}
\item 

%Le tableau ci-dessous présente, pour quatre félins étudiés, les probabilités d'attraper leur proie quand ils la poursuivent.
%
%\begin{center}
%\begin{tabularx}{0.75\linewidth}{|*{2}{>{\centering \arraybackslash}X|}}\hline
%Félin étudié &
%Probabilité d'attraper la proie qu'il poursuit\\ \hline
%Le lion&25\,\%\\ \hline
%Le guépard&$\dfrac12$\\ \hline
%Le tigre&$0,1$\\ \hline
%Le chat à pieds noirs&$\dfrac{6}{10}$\\ \hline
%\end{tabularx}
%\end{center}
%
%Vérifier que, parmi les quatre félins étudiés, le chat à pieds noirs a la probabilité la plus élevée d'attraper sa proie quand il la poursuit.
Les probabilités en notation décimale sont respectivement:

\[\dfrac{25}{100} = 0,25\;;\; \quad \dfrac12 = 0,5 \;;\;  \quad 0,1 \;;\; \quad \dfrac{6}{10} = 0,6.\]

La probabilité la plus grande est celle du chat à pieds noirs.
\item  %Le plus souvent, le guépard est le félin le plus rapide avec une vitesse pouvant atteindre 115 km/h.
%À cette vitesse, en combien de secondes le guépard parcourt-il 100 mètres?
%On donnera une valeur approchée au centième de seconde près.
115 km en 60 min ou \np{3600} s soit \np{115000} m en \np{3600} s, soit $\dfrac{\np{115000}}{\np{115000}} = \dfrac{\np{1150}}{36} \approx 31,944$~m.

$v$ étant la vitesse, $d$ la distance et $t$ le temps, on sait que $v = \dfrac{d}{t}$, d'où $t = \dfrac{d}{v}$.

Donc avec $d = 100$ et $v = \dfrac{\np{1150}}{36}$, on obtient $t = \dfrac{100}{\frac{\np{1150}}{36}} = \dfrac{100 \times 36}{\np{1150}}\approx 3,130$~(s).

Le guépard parcourt 100 m en à peu près 3,13 secondes (au centième près).

%Dans un pays d'Afrique, on estimait à :
%
%\begin{itemize}[label=$\bullet~$]
%\item \np{1200} guépards en 1999.
%\item 170 guépards en 2016.
%\end{itemize}
%
%Dans ce pays, est-il vrai que le nombre de guépards a baissé d'environ 86\,\% entre 1999 et 2016 ?
Par rapport à 1999, il y avait $\dfrac{170}{\np{1200}} \approx 0,142$, soit 14,2\,\% : à l'unité près la baisse est bien de $100 - 14 = 86$ pour cent.
\item Dans le parc national d'Etosha en Namibie, on peut observer des lions et des guépards.
%À l'aide de la carte ci-dessous, donner approximativement la latitude et la longitude du parc national d'Etosha.
%
%\includegraphics[width=14cm]{mappemonde}
Longitude du parc : environ $15\degres$ Est et latitude $20\degres$ Sud.
\end{enumerate}

\bigskip

