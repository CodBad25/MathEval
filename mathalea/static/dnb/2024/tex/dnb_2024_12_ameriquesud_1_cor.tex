
\bigskip

%Cet exercice est un Q.C.M. (questionnaire à choix multiple).
%
%Pour chacune des cinq questions, trois réponses sont proposées et une seule convient.
%
%Pour chacune des cinq questions, écrire sur la copie le numéro de la question et la lettre correspondant à la réponse choisie.
%
%\textbf{Aucune justification n'est attendue.}
%
%Une réponse fausse ou l'absence de réponse ne retire pas de point.
%
%\begin{center}
%\begin{tabularx}{\linewidth}{|c|m{4.5cm}|*{3}{>{\centering \arraybackslash}X|}}\hline
%	&&A&B&C\\ \hline
%1	&Une urne contient trois jetons verts et deux jetons blancs. On tire un jeton au hasard.
%
%Quelle est la probabilité d'obtenir un jeton blanc?&
%$\dfrac23$&$\dfrac35$&$\dfrac25$\\ \hline
%2	&\psset{unit=0.6cm,arrowsize=2pt 3}
%\begin{pspicture}(0,-0.6)(5.5,3.5)
%
%\psframe(3,1)\psline(1,1)(1,0)\psline(2,1)(2,0)
%\psline(3,0)(3.4,0.35)(3.4,1.35)(3,1)
%\psline(3.4,1.35)(3.8,1.7)
%\psline{->}(5.5,1)(4,1)
%\psline(1,1)(1.4,1.35)(3.4,1.35)
%\psline(2,1)(2.8,1.7)(3.8,1.7)(3.8,0.7)(3.4,0.35)
%\psline(3.8,0.7)(3.8,-0.3)(3.4,-0.65)(3.4,0.35)
%\psline(3.4,-0.65)(2.4,-0.65)(2.4,0)
%\psline(1.4,1.35)(1.4,3.35)(0.4,3.35)(0,3)(1,3)(1,1)
%\psline(1,3)(1.4,3.35)\psline(0,2)(1,2)(1.4,2.35)
%\psline(0,3)(0,1)
%\end{pspicture}
%
%Quelle est la vue de droite de ce solide ? &\psset{unit=0.6cm}
%\begin{pspicture}(3,5.5)
%\psframe(0,1)(1,5)
%\psline(0,2)(1,2)\psline(0,3)(1,3)
%\psline(1,1)(3,1)(3,2)(1,2)\psline(2,1)(2,0)(3,0)(3,1)\psline(0,4)(1,4)\psline(2,1)(2,2)\end{pspicture}&
%\psset{unit=0.6cm}
%\begin{pspicture}(3,5)\psframe(0,1)(1,4) \psframe(0,1)(2,2)\psline(0,3)(1,3)\psline(1,1)(1,0)(2,0)(2,1) \end{pspicture}&
%\psset{unit=0.6cm}
%\begin{pspicture}(3,5)\psframe(0,1)(2,2)\psframe(1,0)(2,4)\psline(1,3)(2,3) \end{pspicture}\\ \hline
%3&
%\psset{unit=0.9cm}
%\begin{pspicture}(3.5,2.6)
%\pspolygon(0.2,2.3)(0.5,0.2)(3.6,0.2)%ABC
%\uput[r](0.2,2.3){\footnotesize A} \uput[dl](0.5,0.2){\footnotesize B} \uput[dr](3.6,0.2){\footnotesize C} \uput[ur](1,0.2){\footnotesize D} \uput[l](0.44,0.53){\footnotesize H}
%\psline(1,0.2)(0.44,0.53)
%\end{pspicture}
%
% B, H et A sont alignés.
%
%B, D et C sont alignés.
%
%BD = 2 cm ; BC = 10 cm  ; 
%
%AC = 16 cm; (DH) // (AC).
%
%Quelle est la longueur du segment [DH] ?
%&3,2 cm&4 cm&4,8 cm\\ \hline
%
%\end{tabularx}
%\end{center}
%
%\begin{center}
%\begin{tabularx}{\linewidth}{|c|m{4.5cm}|*{3}{>{\centering \arraybackslash}X|}}\hline
%&&A&B&C\\ \hline
%4&Voici un engrenage: 12 dents 9 dents
%\psset{unit=0.3cm}
%\begin{pspicture}(-3.5,-3)(10,4)
%%\psgrid
%\def\denta{\psline(2;78)(2;82)(2.3;82)(3;88)(3;92)(2.3;98)(2;98)(2;102)}
%\def\dentb{\psline(1.5;-12)(1.5;-8)(1.725;-8)(2.25;-2)(2.25;2)(1.725;8)(1.5;8)(1.5;12)}
%\def\roued{\multido{\n=-5+40,\na=0+40}{9}{\rput{\na}(0.2;\n){\dentb}}}%
%\pscircle(0,0){2.57}
%\multido{\n=90+30,\na=0+30}{12}{\rput{\na}(0.7;\n){\denta}}
%\pscircle(5.75,0){1.6}
%\rput(5.75,0){\roued}
%\end{pspicture}
%
% Si la petite roue effectue exactement 4 tours complets, combien de tours complets effectue la grande roue ?&3 tours complets&4 tours complets&6 tours complets\\ \hline
%5&\psset{unit=0.7cm}
%\begin{center}
%\begin{pspicture}(3.3,3.3)
%\psframe(2,2)\psframe(2,2)(3,3)
%\psline(0,2)(2,0)\psline(2,3)(3,2)
%\uput[dl](0,0){\footnotesize F} \uput[d](2,0){\footnotesize E} \uput[l](0,2){\footnotesize G} \uput[ul](2,2){\footnotesize A}
%\uput[r](3,2){\footnotesize D} \uput[ur](3,3){\footnotesize C} \uput[ul](2,3){\footnotesize B}
%\multido{\n=0+1}{4}{\psline[linewidth=0.2pt](\n,0)(\n,3)}
%\multido{\n=0+1}{4}{\psline[linewidth=0.2pt](0,\n)(3.2,\n)}
%\end{pspicture}\end{center}
%\medskip
%
%Le carré AGFE est l'image du carré ADCB par une homothétie de centre A.
%
%Le triangle EGF est l'image d'un triangle par cette même homothétie.
%
%Quel est ce triangle ?
%&GEA&ABD&BDC\\ \hline
%\end{tabularx}
%\end{center}
\begin{enumerate}
\item Il y a 2 jetons blancs pour un total de $2 + 3 = 5$ jetons ; la probabilité est donc égale à $\dfrac25 = \dfrac{4}{10} = 0,4$ : réponse C.
\item La vue de droite est la B.
\item On est dans la situation du théorème de Thalès et d'après celui-ci  :

$\dfrac{\text{DH}}{\text{AC}} = \dfrac{\text{BD}}{\text{BC}}$, soit $\dfrac{\text{DH}}{16} = \dfrac{2}{10}$, d'où DH $ = 16 \times \dfrac{2}{10} = \dfrac{32}{10} = 3,2$~(cm). Réponse A.
\item Si la petite roue fait un tour elle fait tourner la grande de 9 crans, donc en faisant 4 tours elle fait tourner la grande de $4 \times 9$ crans  ; or $4 \times 9 = 4 \times 3 \times 3 = 12 \times 3 = 3 \times 12$ : donc la grande tournera de 3 tours : réponse A.
\item Le carré AGFE est l'image du carré ADCB dans l'homothétie de centre A et de rapport $- 2$ : le triangle EGF est donc l'image du triangle BDC dans cette homothétie. Réponse~C.
\end{enumerate}

\bigskip

