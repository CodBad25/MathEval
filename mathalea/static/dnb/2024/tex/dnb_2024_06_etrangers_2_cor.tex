
\medskip

\begin{enumerate}
\item Le circuit 1, c'est quand on enchaîne cinq fois de suite 40 secondes d'exercice et 16 secondes de repos, soit 5 fois  $40 + 16 = 56$ secondes.

On a donc bien besoin de  :\quad $5 \times 56 = 280$ secondes pour effectuer le circuit 1.

Pour le circuit 2 : même principe, on enchaîne dix fois 30 secondes d'exercice et 5 secondes de repos :

$10 \times (30 + 5 ) = 10\times 35 = 350$

Il faut bien 350 secondes pour effectuer le circuit 2.

\item Donnons la décomposition en produit de facteurs premiers de $280$ et de $350$.

\begin{multicols}{2}
$\aligned[t] 280 &= 4 \times 7 \times 10\\
&= 2 \times 2 \times 7 \times 2 \times 5\\
&= 2^3 \times 5 \times 7 \\\endaligned$

$\aligned[t] 350 &= 5 \times 7 \times 10\\
&= 5 \times 7 \times 2 \times 5\\
&= 2 \times 5^2 \times 7 \\\endaligned$
\end{multicols}

La décomposition de 280 en produit de facteurs premiers est : \quad $280= 2^3 \times 5 \times 7$.

Celle de 350 est :\phantom{sition de 280 en produit de facteurs premiers} \quad $350= 2 \times 5^2 \times 7$
\item \begin{enumerate}
		\item Lorsque \np{2800} secondes se sont écoulées à partir du coup de sifflet, Camille se trouve de nouveau au départ du circuit 1 car $\np{2800} = 10 \times 280$, donc comme 10 est un nombre entier, cela signifie que Camille a effectué 10 fois le circuit 1 complètement, et n'a pas encore commencé la 11\up{e} répétition : Camille est donc à nouveau au départ du circuit 1.

On a :\quad $\dfrac{\np{2800}}{350 } = 8$.

\emph{Rem. } Ou encore $\np{2800} = 7 \times 4 \times 100 = 7 \times 4 \times 10 \times 10 =  7 \times 2^2 \times 2^2 \times 5^2 = 2^4 \times 5^2 \times 7 = 2^3 \times {\red(2 \times 5^2 \times 7)} = 8 \times 350$.

Au bout de ces \np{2800}, Dominique a donc parcouru exactement 8 parcours 2  : elle est donc au départ.
	\item Après le coup de sifflet, la première fois où Camille et Dominique se retrouvent en même temps au départ de leur circuit est pour un nombre de secondes qui est le multiple commun à 280 et à 350 le plus petit possible.

Les facteurs premiers de 280 et de 350 sont les mêmes : 2, 5 et 7.

Pour qu'un nombre soit divisible par 280, il faut au moins trois facteurs 2, au moins un facteur 5 et au moins une fois le facteur 7 au moins une fois.

Pour qu'un nombre soit divisible par 350, il faut au moins un facteur 2, au moins deux facteurs 5 et au moins une fois le facteur 7.

En réunissant ces critères, il faut donc $2^3 \times 5^2 \times 7 = \np{1400}$ secondes pour que Camille et Dominique se retrouvent pour la première en même temps au départ de leur circuit.

Comme $\np{1400} = \np{1200} + 200 = 20 \times 60 + 180 = 20 \times 60 + 3 \times 60 + 20 = 23 \times ,60 + 20$, on a \np{1400}(s) = 23 (min 20~(s).

(C'est logique : après \np[s]{2800} les deux avaient fait un nombre pair de tours complets, donc en divisant le temps par 2, ils ont encore fait un nombre entier de tours complets chacun, et donc se retrouvent au début du circuit).
	\end{enumerate}
\end{enumerate}


