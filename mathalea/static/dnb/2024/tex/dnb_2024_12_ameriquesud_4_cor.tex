
\medskip

%On dispose d'un terrain en pente sur lequel on souhaite construire une maison. Il faut pour cela enlever de la terre afin d'obtenir un terrain horizontal.
%On dispose des informations suivantes :
%
%\medskip
%
%\begin{minipage}{0.3\linewidth}
%La maison sera construite sur le terrain horizontal représenté par le segment [BC].
%Le triangle ABC est rectangle en C et : 
%
%AC $= 2,6$ m
%
%AB $=17$ m
%\end{minipage}\hfill
%\begin{minipage}{0.68\linewidth}
%\psset{unit=0.8cm}
%\begin{pspicture}(12,5.5)
%\pspolygon[fillstyle=crosshatch](0,0)(12,0)(12,1)(3.2,1)(3.2,2.4)(0,2.4)
%\pspolygon[fillstyle=vlines](8.3,1)(3.2,2.4)(3.2,1)%BAC
%\psframe(3.2,1)(3.4,1.2)
%\rput(6,2.5){Terre à enlever}\psline{->}(6,2.4)(5,1.6)
%\pspolygon[fillstyle=solid,fillcolor=lightgray](0.5,2.4)(0.5,3.2)(1.4,4.6)(2.3,3.2)(2.3,2.4)
%\pspolygon[fillstyle=solid,fillcolor=lightgray](9.7,1)(9.7,1.8)(10.6,3.2)(11.7,1.8)(11.7,1)
%\uput[d](8.3,1){B} \uput[u](3.2,2.4){A} \uput[dl](3.2,1){C} 
%\rput(6,5.3){\textbf{Vue en coupe du terrain}}
%\end{pspicture}
%\end{minipage}

\medskip

\begin{enumerate}
\item %Justifier que la longueur CB est égale à 16,8 m.
Le théorème de Pythagore appliqué au triangle ABC rectangle en C s'écrit \\
$\text{AB}^2 = \text{AC}^2 + \text{CB}^2$, d'où 

$\text{CB}^2 = \text{AB}^2 - \text{AC}^2 = 17^2 - 2,6^2 = (17 - 2,6) \times (17 + 2,6) = 14,4 \times 19,6 = 282,24$.

Il en résulte que CB $ = \sqrt{282,24} = 16,8$~(m).
\item %Le coût des travaux pour enlever la terre dépend de la mesure de l'angle $\widehat{\text{ABC}}$.
%Si la mesure de l'angle $\widehat{\text{ABC}}$ est supérieure à $8,5\degres$, cela entraînera un surcoût des travaux (c'est-à-dire que les travaux pour enlever la terre coûteront plus cher).

%Est-ce le cas pour ce terrain?
En utilisant par exemple la tangente, on a :
$\tan\, \widehat{\text{ABC}} = \dfrac{\text{AC}}{\text{BC}} = \dfrac{2,6}{16,8}\approx \np{0,1548}$.

La calculatrice donne $\widehat{\text{ABC}} \approx 8,797~(\degres)$ donc une mesure supérieure à $8,5\degres$ : il y aura surcoût.
\item %On admet que le volume de terre enlevée correspond au volume du prisme droit CBAFED de hauteur [CF] et de bases triangulaires ACB et DFE, comme représenté ci-dessous. On rappelle que les longueurs CF et AD sont égales.

% figure prisme de terre
%\begin{center}
%\psset{unit=1cm}
%\begin{pspicture}(5.4,3.2)
%\pspolygon(0.2,0.5)(2.5,0.2)(0.2,1.7)%CBA
%\psline(2.5,0.2)(4.7,1.2)(2.4,2.7)(0.2,1.7)%BEDA
%\psline[linestyle=dashed](0.2,0.5)(2.4,1.5)(4.7,1.2)%CFE
%\psline[linestyle=dashed](2.4,1.5)(2.4,2.7)%FD
%\uput[ul](0.2,1.7){A} \uput[dr](2.5,0.2){B} \uput[dl](0.2,0.5){C}
%\uput[u](2.4,2.7){D} \uput[r](4.7,1.2){E} \uput[l](2.4,1.7){F}
%\psframe(0.2,0.5)(0.4,0.7)\psframe(2.4,1.5)(2.6,1.7)
%\rput{90}(0,1.1){2,6 m}\rput{-30}(3.55,2.2){17 m}\rput{24}(1.3,2.4){30 m}
%\end{pspicture}
%\end{center}
%
%Déterminer le volume de terre à enlever en m$^3$.
%
%On rappelle la formule:
%
%Volume d'un prisme droit = aire d'une base du prisme $\times$ hauteur du prisme.
Le volume de terre à enlever est donc égal à :

$V = \mathcal{A}(\text{ABC}) \times \text{AD} = \dfrac{\text{AC} \times \text{CB}}{2} 
\times \text{AD} = \dfrac{2,6 \times 16,8}{2} 
\times 30 = 2,6 \times 16,8 \times 15 = 655,2$~m$^3$
\end{enumerate}


\bigskip
