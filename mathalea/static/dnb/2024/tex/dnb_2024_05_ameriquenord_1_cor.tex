
%Voici cinq affirmations. Pour chacune d'entre elles, dire si elle est vraie ou fausse. On rappelle que chaque réponse doit être justifiée.

\medskip

\begin{enumerate}
\item %Voici les prix en euros d'un vêtement relevés dans différents magasins.

Voici les prix en euros rangés dans l'ordre croissant
\[7~;~10~;~12~;~13~;~15\]

\textbf{Affirmation A }: La moyenne des prix est $11,40$~\euro. Vrai :

$\dfrac{7 + 10 + 12 + 13 + 15}{5} = \dfrac{57}{5} = \dfrac{114}{10} = 11,4$.

\textbf{Affirmation B} : La médiane des prix est $10$~\euro. Faux : la médiane est 12.

\item Lors d'un entraînement, une élève court 20~m en 6 secondes.

\textbf{Affirmation C} : %Lors de cet entraînement, sa vitesse moyenne était de $14$~km/h.

20~m en 6 secondes soit 200~m en 60~s ou 1 min donc $60 \times 200 = \np{12000}$~m en 60 min ou 1h donc finalement 12~km/h. Faux.
\item Une urne contient $15$ boules indiscernables numérotées de 1 à 15.


\textbf{Affirmation D} : La probabilité de tirer au hasard une boule sur laquelle apparaît un nombre premier est $\dfrac{7}{15}$.

Les nombres premiers sont : 2~;~3~;~5~;~7~;~11~;~13 : il y en a donc 6 parmi les 15, soit une probabilité de $\dfrac{6}{15} = \dfrac25$ : Faux.

\item Le triangle A$'$B$'$C$'$ est l'image du triangle ABC par l'homothétie de centre O et de rapport $(- 3)$.


%\begin{center}
%\psset{unit=0.8cm}
%\begin{pspicture}(-3.5,-2.6)(9.2,4.5)
%%\psgrid
%\uput[dl](-1.4,-0.4){A} \uput[d](-0.6,-1.4){B} \uput[u](-0.4,0.7){C} 
%\uput[u](0,0){O}
%\psdots[dotstyle=+,dotscale=2,dotangle=45](0,0)(-1.4,-0.4)(-0.6,-1.4)(-0.4,0.7)(3.6,1.2)(1.8,4.2)(1.2,-2.1)
%\pspolygon(-1.4,-0.4)(-0.6,-1.4)(-0.4,0.7)%ABC
%\uput[dl](-1.,-0.9){5 cm} \uput[ul](-0.9,0.15){6 cm} \uput[r](-0.5,-0.35){9 cm} 
%\uput[dr](3.6,1.2){A$'$} \uput[u](1.8,4.2){B$'$} \uput[d](1.2,-2.1){C$'$}
%\pspolygon(3.6,1.2)(1.8,4.2)(1.2,-2.1)%A'B'C'
%\uput[ur](2.7,2.7){15 cm} \uput[dr](2.4,-0.45){18 cm} \uput[ul](1.5,1.05){27 cm} 
%\end{pspicture}
%%\includegraphics[max width=\textwidth]{2024_05_30_e3637aa9ef1bd32cf3c8g-2}
%\end{center}
%
%\emph{Le dessin n'est pas à l'échelle}

\textbf{Affirmation E} : L'aire du triangle A$'$ B$'$ C$'$ est égale à 3 fois l'aire du triangle $A B C$.

L'aire fait intervenir le produit de deux longueurs : chacune d'elles étant 3 fois plus grande, l'aire est $3 \times 3 = 9$ fois plus grande : Faux.
\end{enumerate}

