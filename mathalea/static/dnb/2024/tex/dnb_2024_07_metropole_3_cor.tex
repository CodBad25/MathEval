
\medskip

Sur la figure ci-dessous, on a :

\medskip

\begin{itemize}
\item $\mathcal{C}$ est un cercle de centre O et de rayon $4,5$~cm ;
\item $[\text{AB}]$ est un diamètre de ce cercle et D est un point du cercle ;
\item les points B, E, A sont alignés, ainsi que les points D, F{}, A ;
\item les droites (BD) et (EF) sont parallèles ;
\item BD $= 5,4$ cm ~;~ DA $= 7,2$ cm \:et\: AE $= 2,7$ cm.
\end{itemize}

\begin{center}
\psset{unit=0.75cm}
\begin{pspicture}(-4,-4)(4,4.5)
%\psgrid
\pscircle(0,0){4}\uput[ur](4;50){$\mathcal{C}$}
\pspolygon(4;0)(4;100)(4;180)%ADB
\uput[r](4;0){A} \uput[l](4;180){B} \uput[ul](4;100){D}
\uput[d](1.3,0){E} \uput[ur](2.4,1.34){F} \uput[d](0,0){O}
\psline(1.3,0)(2.4,1.34)\psline(0,.1)(0,-.1)
\end{pspicture}
\end{center}

\begin{enumerate}
\item %Justifier que le diamètre [AB] mesure 9 cm.

%On sait que le triangle ABD inscrit dans un cercle admettant l'un de ces côtés comme diamètre d'un cercle est rectangle ; ce diamètre [AB] est l'hypoténuse et il est donc rectangle en D.
%Le théorème de Pythagore s'écrit alors :
%
%AB$^2 = \text{BD}^2 + \text{DA}^2 = 5,4^2 + 7,2^2 = (0,9 \times 6)^2 + (0,9 \times 8)^2 = 0,9^2 \times 6^2 + 0,9^2 \times 8^2 = 0,9^2\left(6^2 + 8^2 \right) = 0,9^2 \times (36 + 64) = 0,9^2 \times 100 = 0,9^2 \times 10^2 = (0,9 \times 10)^2 =  9^2$
%
%Conclusion AB $ = 9$~(cm).
On a AB $ = 2R = 2 \times 4,5 = 9$~(cm).
\item %Démontrer que le triangle ABD est rectangle en D.
On a AD$^2 = 7,2^2$ et DB$^2 = 5,4^2$, d'où AD$^2 + \text{DB}^2 = 7,2^2 + 5,4^2 = 51,84 + 29,16 = 81 = 9^2 = \text{AB}^2$.

Donc AD$^2 + \text{DB}^2  = \text{AB}^2$ : d'après la réciproque du théorème de Pythagore le triangle ABD est rectangle en D ; [AB] est l'hypoténuse.
\item %Calculer AF{}.
Comme les points B, E, A sont alignés, ainsi que les points D, F{}, A et
que les droites (BD) et (EF) sont parallèles on est dans une situation de Thalès et on a donc les égalités :

$\dfrac{\text{AE}}{\text{AB}} = \dfrac{\text{AF}}{\text{AD}} = \dfrac{\text{EF}}{\text{BD}}$.

En particulier $\dfrac{\text{AE}}{\text{AB}} = \dfrac{\text{AF}}{\text{AD}}$ ou $\dfrac{2,7}{9} = \dfrac{\text{AF}}{7,2}$ soit $0,3 = \dfrac{\text{AF}}{7,2}$, d'où AF $ = 0,3 \times 7,2 = 2,16$~(cm).
\item 
	\begin{enumerate}
		\item %Justifier que l'aire du triangle ABD est égale à $19,44~\text{cm}^2$.
		Si $\mathcal{A}$ est l'aire du triangle ABD, on sait que $\mathcal{A} = \dfrac{\text{BD} \times \text{AD}}{2} = \dfrac{5,4 \times 7,2}{2} = 5,4 \times 3,6 = 19,44~\text{cm}^2$.
		\item %Calculer l'aire du disque, arrondie au centième.
		
L'aire du disque est égale à $\pi \times R^2 = \pi \times 4,5^2 = \left(\dfrac92 \right)^2 \times \pi  = \dfrac{81}{4}\pi \approx 63,617$, soit 63,62 au centième de cm$^2$.
	\end{enumerate}
	
%\emph{Rappel} : l'aire du disque est égale à $\pi \times R^2$, où $R$ est le rayon du disque. 
\item Quel pourcentage de l'aire du disque représente l'aire du triangle ABD ?
L'aire du triangle ABD représente pour l'aire du disque un pourcentage égal à environ :

$\dfrac{19,44}{63,62} \times 100$ soit environ 30,6\,\%.
\end{enumerate}

\bigskip

