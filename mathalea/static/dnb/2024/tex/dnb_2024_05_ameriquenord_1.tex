
Voici cinq affirmations. Pour chacune d'entre elles, dire si elle est vraie ou fausse. On rappelle que chaque réponse doit être justifiée.

Aucun point ne sera enlevé en cas de mauvaise réponse.

\medskip

\begin{enumerate}
\item Voici les prix en euros d'un vêtement relevés dans différents magasins.

\[12~;~15~;~10~;~7~;~13\]

\textbf{Affirmation A }: La moyenne des prix est $11,40$~\euro.

\textbf{Affirmation B} : La médiane des prix est $10$~\euro.

\item Lors d'un entraînement, une élève court 20~m en 6 secondes.

\textbf{Affirmation C} : Lors de cet entraînement, sa vitesse moyenne était de $14$~km/h.

\item Une urne contient 15 boules indiscernables numérotées de 1 à 15.

\textbf{Affirmation D} : La probabilité de tirer au hasard une boule sur laquelle apparaît un nombre premier est $\dfrac{7}{15}$.


\item Le triangle A$'$B$'$C$'$ est l'image du triangle ABC par l'homothétie de centre O et de rapport $(- 3)$.


\begin{center}
\psset{unit=0.8cm}
\begin{pspicture}(-3.5,-2.6)(9.2,4.5)
%%\psgrid
\uput[dl](-1.4,-0.4){A} \uput[d](-0.6,-1.4){B} \uput[u](-0.4,0.7){C} 
\uput[u](0,0){O}
\psdots[dotstyle=+,dotscale=2,dotangle=45](0,0)(-1.4,-0.4)(-0.6,-1.4)(-0.4,0.7)(3.6,1.2)(1.8,4.2)(1.2,-2.1)
\pspolygon(-1.4,-0.4)(-0.6,-1.4)(-0.4,0.7)%ABC
\uput[dr](3.6,1.2){A$'$} \uput[u](1.8,4.2){B$'$} \uput[d](1.2,-2.1){C$'$}
\pspolygon(3.6,1.2)(1.8,4.2)(1.2,-2.1)%A'B'C'
\end{pspicture}
\end{center}

\emph{Le dessin n'est pas à l'échelle}

\textbf{Affirmation E} : L'aire du triangle A$'$ B$'$ C$'$ est égale à 3 fois l'aire du triangle ABC.

\end{enumerate}

