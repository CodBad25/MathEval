
On donne le programme suivant.

\textbf{Rappel}

\begin{scratch}
\blockmove{s'orienter à \ovalnum{90} }
\end{scratch} : On s'oriente vers
la droite.\psset{unit=1cm,arrowsize=2pt 3}
\begin{pspicture}(-1.6,-1.6)(1.6,1.6)
\pscircle(0,0){1.2}
\multido{\n=0+15}{25}{\psline(0.8;\n)(01;\n)}
\psline(1.6;90)(0;0)(1.6;0)
\pscircle[fillstyle=solid,fillcolor=white](1.2,0){0.3}
\psline{->}(1.,0)(1.4,0)
\end{pspicture}

\begin{center}
\begin{tabularx}{\linewidth}{|*{2}{>{\centering \arraybackslash}X|}}\hline
Script principal& Motif\\
\begin{scratch}
\blockinit{Quand \greenflag est cliqué}
\blockmove{aller à x: \ovalnum{- 100} y: \ovalnum0}
\blockmove{s'orienter à \ovalnum{90} }
\blockpen{effacer tout}
\blockvariable {mettre \selectmenu{côté } à \ovalnum{80}}
\blockmoreblocks{Motif}
\end{scratch}
&
\begin{scratch}
\initmoreblocks{définir \namemoreblocks{Motif}}
\blockpen{stylo en position d'écriture}
\blockrepeat{répéter \ovalnum{3} fois}
{\blockmove{avancer de \ovalnum{côté} pas}
\blockmove{tourner \turnleft{} de \ovalnum{120} degrés}
}
\blockrepeat{répéter \ovalnum{3} fois}
{\blockmove{avancer de \ovalnum{côté} pas}
\blockmove{tourner \turnright{} de \ovalnum{120} degrés}
}
\blockpen{relever le stylo}
\end{scratch}
\\ \hline
\end{tabularx}
\end{center}

\textbf{Dans cet exercice, aucune justification n'est attendue.}

\medskip

\begin{enumerate}
\item À quelles coordonnées le lutin se positionne-t-il juste après avoir cliqué sur le drapeau vert ? 
\item En prenant 1 cm pour 20 pas, dessiner en vraie grandeur la figure obtenue en exécutant le script principal.
\item On modifie le script principal de trois façons différentes. Associer chaque script à la figure qui lui correspond.

\begin{center}
\begin{tabularx}{\linewidth}{|c|>{\centering \arraybackslash}X|c|}\hline
\begin{scratch}[scale=0.9]
\blockinit{Quand \greenflag est cliqué}
\blockmove{aller à x: \ovalnum{- 100} y: \ovalnum0}
\blockmove{s'orienter à \ovalnum{90} }
\blockpen{effacer tout}
\blockvariable{mettre \ovalvariable{côté} à \ovaloperator{80}}
\blockrepeat{répéter \ovalnum{3} fois}
{\blockmoreblocks{Motif}
\blockmove{avancer de \ovalnum{100} pas}
}
\end{scratch}&
\begin{scratch}[scale=0.9]
\blockinit{Quand \greenflag est cliqué}
\blockmove{aller à x: \ovalnum{- 100} y: \ovalnum{0}}
\blockmove{s'orienter à \ovalnum{90}}
\blockpen{effacer tout}
\blockvariable{mettre \ovalvariable{côté} à \ovalnum{80}}
\blockrepeat{répéter \ovalnum{3} fois}
{
\blockmoreblocks{Motif}
\blockvariable{mettre \ovalvariable{côté} à \ovaloperator{\ovalvariable{côté} *\ovalnum{1.2}}}
}
\end{scratch}&
\begin{scratch}[scale=0.9]
\blockinit{Quand \greenflag est cliqué}
\blockmove{aller à x: \ovalnum{- 100} y: \ovalnum{0}}
\blockmove{s'orienter à \ovalnum{90}}
\blockpen{effacer tout}
\blockvariable{mettre \ovalvariable{côté} à \ovalnum{80}}
\blockrepeat{répéter \ovalnum{3} fois}
{
\blockmoreblocks{Motif}
\blockmove{tourner \turnleft{} de \ovalnum{120} degrés}
}
\end{scratch}\\ \hline
Figure A&Figure B &Figure C\\ 
\begin{pspicture}(0,-1.6)(1.8,1.6)
\pspolygon(0,0)(1.2,0)(0.6,1.039)\pspolygon(0,0)(1.5,0)(0.75,1.2999)\pspolygon(0,0)(1.8,0)(0.9,1.559)
\pspolygon(0,0)(1.2,0)(0.6,-1.039)\pspolygon(0,0)(1.5,0)(0.75,-1.2999)\pspolygon(0,0)(1.8,0)(0.9,-1.559)
\end{pspicture}&
\psset{unit=0.8cm}
\begin{pspicture}(0,-1.6)(4.7,1.6)
\pspolygon(0,0)(1.2,0)(0.6,1.039)\pspolygon(0,0)(1.2,0)(0.6,-1.039)
\pspolygon(1.7,0)(2.9,0)(2.3,1.039)\pspolygon(1.7,0)(2.9,0)(2.3,-1.039)
\pspolygon(3.4,0)(4.6,0)(4,1.039)\pspolygon(3.4,0)(4.6,0)(4,-1.039)
\end{pspicture}&
\begin{pspicture}(-1.3,-1.2)(1.3,1.2)
\pspolygon(1.2;0)(1.2;60)(1.2;120)(1.2;180)(1.2;2400)(1.2;300)
\multido{\n=0+60}{6}{\psline(1.2;\n)}
\end{pspicture}\\ \hline
\end{tabularx}
\end{center}
\item Dans cette question on s'intéresse au script \no 2.
	\begin{enumerate}
		\item Combien de fois le bloc \og motif \fg{} est-il exécuté ?
		\item Quelle est la valeur de la variable \og côté\fg{} à la fin de ce script ?
	\end{enumerate}
\end{enumerate}
