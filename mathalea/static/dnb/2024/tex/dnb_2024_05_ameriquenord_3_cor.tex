
%Un cinéma propose trois tarifs :
%
%Tarif \og Classique\fg{} : La personne paye chaque entrée $11$~\euro.
%
%Tarif \og Essentiel \fg{} : La personne paye un abonnement annuel de $50$~\euro{} puis chaque entrée coûte $5$~\euro.
%
%Tarif \og Liberté \fg : La personne paye un abonnement annuel de $240$~\euro{} avec un nombre d'entrées illimité.

\begin{enumerate}
\item %Avec le tarif \og Classique \fg, une personne souhaite acheter trois entrées au cinéma.

%Combien va-t-elle payer ?

On a $3 \times 11 = 33$~\euro.

\item %Avec le tarif \og Essentiel \fg, une personne souhaite aller huit fois au cinéma.

%Montrer qu'elle va payer $90$~\euro.
On a $50 + 8 \times 5 = 50 + 40 = 90$~\euro.
\item %Dans la suite, $x$ désigne le nombre d'entrées au cinéma.

%On considère les trois fonctions $f, g$ et $h$ suivantes :


\[f: x \longmapsto 50 + 5x \quad g: x \longmapsto 240 \quad h: x \longmapsto 11 x
\]

%Associer, sans justifier, chacune de ces fonctions au tarif correspondant.
$f$ correspond au tarif \og Essentiel \fg ;

$g$ correspond au tarif \og Liberté \fg ;

$h$ correspond au tarif \og Classique \fg.

%Le graphique ci-dessous représente le prix à payer en fonction du nombre d'entrées pour chacun de ces trois tarifs.

\begin{center}
%\includegraphics[max width=\textwidth]{2024_05_30_e3637aa9ef1bd32cf3c8g-4}
\psset{xunit=0.25cm,yunit=0.03cm,arrowsize=2pt 3}
\begin{pspicture*}(-5,-30)(50,290)
\psaxes[linewidth=1.25pt,Dx=5,Dy=50]{->}(0,0)(0,0)(50,290)
\psline(0,240)(45,240)\uput[ul](23,260){$d_1$}
\psline(0,50)(50,300)\uput[ul](45,270){$d_2$}
\psline(25,275)\uput[r](45,240){$d_3$}
\uput[u](40,0){Nombre d'entrées}
\uput[r](0,285){Prix à payer en \euro}
\psline[linecolor=red,ArrowInside=->](0,150)(20,150)(20,0)
\psline[linecolor=blue,ArrowInside=->](0,240)(38,240)(38,0)
\psline[linecolor=cyan,ArrowInside=->](0,200)(30,200)(30,0)
\end{pspicture*}
\end{center}

La droite ($d_{1}$) représente la fonction correspondant au tarif \og Classique\fg.

La droite ($d_{2}$) représente la fonction correspondant au tarif \og Essentiel\fg.

La droite $(d_{3})$ représente la fonction correspondant au tarif \og Liberté \fg.

\item %Quel tarif propose un prix proportionnel au nombre d'entrées ?
C'est le tarif \og Classique \fg{} qui propose un prix proportionnel au nombre d'entrées (la fonction $h$ est linéaire).
\item Pour les questions suivantes, aucune justification n'est attendue.
	\begin{enumerate}
		\item %Avec $150$~\euro, combien peut-on acheter d'entrées au maximum avec le tarif \og Essentiel \fg ?
La droite horizontale d'équation $y = 150$ coupe la droite $(d_2)$ au point d'abscisse 20. 
On peut acheter 20 places au maximum au tarif \og Essentiel \fg.
		\item %À partir de combien d'entrées, le tarif \og Liberté \fg{} devient-il le tarif le plus intéressant?
La droite horizontale d'équation $y = 240$ coupe la droite $(d_2)$ au point d'abscisse 38.  .À partir de 39 places le tarif \og Liberté\fg est le moins onéreux.
		\item %Si on décide de ne pas dépasser un budget de $200$~\euro, quel est le tarif qui permet d'acheter le plus grand nombre d'entrées ?
		La dernière droite coupée par la droite d'équation $y = 200$ est la droite $(d_2)$.
		
		Pour 200~\euro{} c'est le tarif \og Essentiel \fg{} qui donne le plus grand nombre de places.
	\end{enumerate}
\end{enumerate}
	
