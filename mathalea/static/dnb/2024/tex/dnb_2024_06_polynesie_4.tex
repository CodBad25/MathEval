
\medskip

On dispose de deux boîtes contenant des boules numérotées, indiscernables au toucher.

\begin{minipage}{0.55\linewidth}
La première boîte contient trois boules numérotées 2, 3 et 5.

La deuxième boîte contient deux boules numérotées 3 et 5.
\end{minipage}\hfill
\begin{minipage}{0.42\linewidth}
\psset{unit=1cm}
\begin{pspicture}(6.2,1.3)
\psline[linewidth=1.5pt](0,1.3)(0,0)(2.7,0)(2.7,1.3)
\rput(0.6,0.4){\Large \ding{206}}\rput(1.5,0.4){\Large \ding{203}}\rput(2.3,0.4){\Large \ding{204}}
\psline[linewidth=1.5pt](3.4,1.3)(3.4,0)(6.1,0)(6.1,1.3)
\rput(4.2,0.4){\Large \ding{204}}\rput(5.2,0.4){\Large \ding{206}}
\end{pspicture}
\end{minipage}

\medskip

On tire au hasard une boule dans la première boîte puis une boule dans la deuxième boîte.

On s'intéresse au produit des nombres inscrits sur ces deux boules.

Par exemple, si on tire la boule numérotée 2 dans la première boîte puis la boule numérotée 5 dans la deuxième boîte, on obtient comme résultat: $2 \times 5 = 10$.

\medskip

\begin{enumerate}
\item Compléter le tableau à double entrée afin de faire apparaître tous les résultats possibles de cette expérience.

\medskip

\begin{minipage}{0.6\linewidth}
\begin{tabularx}{\linewidth}{|l|*{2}{>{\centering \arraybackslash}X|}}\hline
\diagbox{1\up{er} tirage}{2\up{e} tirage}&3&5\\ \hline
5	& 	& \\ \hline
2	&	&10\\ \hline
3	&	& \\ \hline
\end{tabularx}
\end{minipage}\hfill
\begin{minipage}{0.37\linewidth}
\begin{pspicture}(-1.5,-0.7)(1.5,0.7)
\rput(0,0){\psellipticarc(1.2,0.8){190}{175}}
\psline(1.2;175)(3;190)(1.2;190)\rput(0,0){$2 \times 5 = 10$}
\end{pspicture}
\end{minipage}

\medskip

\item Quelle est la probabilité d'obtenir $15$ comme résultat ?
\item L'affirmation suivante est-elle vraie ?

\textbf{Affirmation :} Il y a 2 chances sur 3 d'obtenir un multiple de 3.
\item On ajoute une troisième boîte contenant deux boules numérotées avec des nombres entiers.

On tire au hasard une boule dans la première boîte, puis une boule dans la deuxième boîte, puis une boule dans la troisième boîte.

On multiplie les nombres inscrits sur ces boules et on s'intéresse au produit de ces trois nombres. Anissa a obtenu comme résultat $165$ et Bilel a obtenu $78$.

Quels sont les nombres inscrits sur les boules de la troisième boîte?
\end{enumerate}

\bigskip

