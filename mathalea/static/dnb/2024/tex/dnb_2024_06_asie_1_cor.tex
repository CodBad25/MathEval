
\emph{Bien qu'on ne demande pas de justification sur votre copie, on en donne dans ce corrigé.}

\medskip

\textbf{Question 1}\quad
Le nombre premier est 37 : \textbf{réponse C}.

En effet :
\begin{itemize}[label=\textbullet]
	\item 1 n'est pas premier, il n'a qu'un seul diviseur entier naturel;
	\item 21 est divisible par 1, 3, 7 et 21 : il a plus de deux diviseurs entiers naturels;
	\item 54 est divisible, entre autres, par 1, 2 et 54 (il y a 8 diviseurs entiers naturels, en tout) : il a plus de deux diviseurs entiers naturels.
\end{itemize}

Contrairement à tous les autres nombres, 37 a exactement deux diviseurs naturels : 1 et 37.

\medskip

\textbf{Question 2}\quad L'aire totale du patron est de \np[cm^2]{150} : \textbf{réponse B}.

En effet un cube a six faces, chaque face étant un carré dont le côté est l'arête du cube, donc ici \np[cm]{5}.

Chaque face a une aire de :\quad $5^2 = 25 \:(\text{cm}^2)$.

L'ensemble des six faces a donc une aire totale de :\quad $6\times 25 = 150 \:(\text{cm}^2)$.

\medskip

\textbf{Question 3}\quad Une forme factorisée de l'expression est $(2x-3)(2x+3)$ : \textbf{réponse B}.

Dans l'expression littérale $4x^2 - 9$ on reconnaît une identité remarquable :

$\aligned[t] 4x^2 - 9&= (2x)^2 - 3^2\\
&= \big((2x) - 3\big)\big((2x) + 3\big)\\
&=(2x - 3)(2x + 3)\endaligned$

\medskip

\textbf{Question 4}\quad La largeur de l'écran est d'environ \np[cm]{62} : \textbf{réponse A}.

Si la longueur et la largeur $\ell$ de l'écran sont dans le ratio $16 : 9$, alors on peut compléter le tableau de proportionnalité :

\begin{center}
	\begin{tabularx}{5cm}{|l|*{2}{>{\centering\arraybackslash}X|}} \hline
		Longueur&16&110\\ \hline
		Largeur&9&$\ell$ \\	\hline
	\end{tabularx}
\end{center}

Avec un produit en croix, on a :\quad $\ell = \dfrac{110 \times 9}{16}=61,875$

Au centimètre près, cela donne donc \np[cm]{62}.

\medskip

\textbf{Question 5}\quad La médiane est 4,1 \textbf{réponse B}

On range les valeurs dans l'ordre croissant :  \quad  3,4\quad $\leqslant$\quad 3,67\quad $\leqslant$\quad 4,1\quad $\leqslant$\quad 4,23\quad $\leqslant$\quad 4,5

Comme il y a 5 valeurs et que 5 est un nombre impair, la médiane est la valeur en position $\dfrac{5 + 1}{2}=3$, dans la série ordonnée, ici, c'est donc 4,1.

\bigskip

