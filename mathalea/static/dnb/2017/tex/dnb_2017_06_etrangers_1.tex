
\medskip

 Pour chacune des affirmations suivantes, dire si elle est vraie ou fausse. 

 Chaque réponse doit être justifiée. 

\medskip

\parbox{0.55\linewidth}{\textbf{Affirmation 1 :} 

Un menuisier prend les mesures suivantes dans le coin d'un mur à 1 mètre au-dessus du sol pour construire une étagère $ABC$ : 

$AB = 65$ cm ; $AC = 72$ cm et $BC = 97$ cm 

Il réfléchit quelques minutes et assure que l'étagère a un angle droit.}
\hfill
\parbox{0.45\linewidth}{
\begin{tikzpicture}[scale=0.9]

  % Points initiaux
  \coordinate (b) at (0,0);
  \coordinate (a) at (0,5);
  \coordinate (c) at (3.7,1.5);
  \coordinate (d) at (6.5,0.5);

  % Translations
  \coordinate (f) at ($(c)+(a)-(b)$);
  \coordinate (e) at ($(d)+(a)-(b)$);

  % Segments
  \draw (b) -- (c);
  \draw (d) -- (c);
  \draw (b) -- (a);
  \draw (d) -- (e);
  \draw (a) -- (f);
  \draw (f) -- (c);
  \draw (f) -- (e);

  % Homothétie A
  \coordinate (A) at ($(c)!.34!(f)$);
  \draw[black, thick] (A)++(-2pt,-2pt) -- ++(4pt,4pt); % Croix
  \draw[black, thick] (A)++(-2pt,2pt) -- ++(4pt,-4pt); % Croix
  \node[above left] at (A) {\tiny A};

  % h = translation de A selon fe
  \coordinate (h) at ($(A)+(e)-(f)$);

  % C = hom_{A,h}^{.32}
  \coordinate (C) at ($(A)!.32!(h)$);
  \draw[black, thick] (C)++(-2pt,-2pt) -- ++(4pt,4pt); % Croix
  \draw[black, thick] (C)++(-2pt,2pt) -- ++(4pt,-4pt); % Croix
  \node[right] at (C) {\tiny C};

  % g = translation de A selon fa
  \coordinate (g) at ($(A)+(a)-(f)$);

  % B = hom_{A,g}^{.36}
  \coordinate (B) at ($(A)!.36!(g)$);
  \draw[black, thick] (B)++(-2pt,-2pt) -- ++(4pt,4pt); % Croix
  \draw[black, thick] (B)++(-2pt,2pt) -- ++(4pt,-4pt); % Croix
  \node[above right] at (B) {\tiny B};

  % Triangle ABC
  \draw (A) -- (B);
  \draw (C) -- (B);
  \draw (A) -- (C);

\end{tikzpicture}
}

\medskip

 \textbf{Affirmation 2 :} 

 Les normes de construction imposent que la pente d'un toit représentée ici par l'angle $\widehat{CAH}$ doit avoir une mesure comprise entre 30$^\circ$ et 35$^\circ$. \\

\parbox{0.43\linewidth}{Une coupe du toit est représentée ci-contre : 

$AC = 6$ m et $AH = 5$ m. 

 

$H$ est le milieu de $[AB]$. 

 }
\hfill
\parbox{0.52\linewidth}{
\begin{tikzpicture}[scale=1.3]

  % Points principaux
  \coordinate (H) at (0,0);
  \coordinate (A) at (-2.6,0);
  \coordinate (B) at (2.6,0);
  \coordinate (C) at (0,1.3);

  % Points d et e
  \coordinate (d) at (-0.6,0);
  \coordinate (e) at (0.6,0);

  % Projections orthogonales
  \path[name path=AC] (A) -- (C);
  \path[name path=perp_d] (d) -- ($(d)!2!(C)$);
  \path[name intersections={of=AC and perp_d, by=g}];

  \path[name path=BC] (B) -- (C);
  \path[name path=perp_e] (e) -- ($(e)!2!(C)$);
  \path[name intersections={of=BC and perp_e, by=f}];

  % Triangle et hauteurs
  \draw (A) -- (B);
  \draw (A) -- (C);
  \draw (B) -- (C);
  \draw[dashed] (C) -- (H);
  \draw (C) -- (d) -- (g);
  \draw (C) -- (e) -- (f);

  % Marquage angle droit à H
  \draw ($(H)+(0.2,0)$) -- ++(0,0.2) -- ++(-0.2,0);

  % Points avec symbole "x"
  \node at (A) {\large $\times$};
  \node at (B) {\large $\times$};
  \node at (C) {\large $\times$};
  \node at (H) {\large $\times$};

  % Étiquettes des points
  \node at ($(A)+(-0.2,0.2)$) {\scriptsize A};
  \node at ($(B)+(0.2,0.2)$) {\scriptsize B};
  \node at ($(C)+(0,0.2)$) {\scriptsize C};
  \node at ($(H)+(0,-0.25)$) {\scriptsize H};

\end{tikzpicture}

}

Le charpentier affirme que sa construction respecte la norme.\\


\medskip

\textbf{Affirmation 3 : }

 Un peintre souhaite repeindre les volets d'une maison. Il constate qu'il utilise $\dfrac{1}{6}$ du pot pour mettre une couche de peinture sur l'intérieur et l'extérieur d'un volet. Il doit peindre ses 4 paires de volets et mettre sur chaque volet 3 couches de peinture. 

 Il affirme qu'il lui faut 2 pots de peinture.

\bigskip

