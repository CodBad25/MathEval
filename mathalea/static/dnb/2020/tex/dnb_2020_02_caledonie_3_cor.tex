
\medskip

%Un aquaculteur étudie l'évolution de la masse moyenne des crevettes dans un bassin.
%
%Il dispose de valeurs théoriques.
%
%On donne la représentation graphique de la masse moyenne théorique des
%crevettes (en grammes) en fonction du temps passé dans le bassin (en jours).
%
%\medskip

\begin{enumerate}
\item %Répondre aux questions suivantes en utilisant le graphique.
	\begin{enumerate}
		\item %La masse moyenne théorique des crevettes est-elle proportionnelle au nombre de jours passés dans le bassin? Justifier la réponse.
La proportionnalité se traduirait par une droite représentation graphique : ce n'est pas le cas. Il n'y a pas proportionnalité.
		\item %Au bout de $80$ jours, quelle est la masse moyenne théorique des crevettes ?
On lit pour 80 jours une masse approximative de 11~g.
		\item %La pêche dans un bassin peut être effectuée lorsque la masse moyenne des crevettes atteint $20$ grammes.
		
%Au bout de combien de jours peut-on envisager la pêche dans ce bassin ?
La masse de 20~g est obtenue au bout de 125~jours.
	\end{enumerate}
\item  %L'aquaculteur effectue régulièrement des relevés dans son bassin pour suivre son évolution.

%Voici les résultats de ses derniers relevés:
%
%\begin{center}
%\begin{tabularx}{0.75\linewidth}{|p{3cm}|*{3}{>{\centering \arraybackslash}X|}}\hline
%Nombre de jours						& 120	& 145	& 175\\ \hline
%Masse moyenne relevée(en grammes)	& 23	& 31	& 38\\ \hline
%\end{tabularx}
%\end{center}
	\begin{enumerate}
		\item Les points A(120~;~23), B(145~;~31) et C(175~;~38) sont sur le graphique ci-dessous.

\psset{xunit=0.06cm,yunit=0.24cm}
\begin{pspicture}(-20,-5)(200,40)
\multido{\n=0+5}{41}{\psline[linewidth=0.2pt](\n,0)(\n,40)}
\multido{\n=0+20}{11}{\psline[linewidth=0.4pt](\n,0)(\n,40)}
\multido{\n=0+1}{41}{\psline[linewidth=0.2pt](0,\n)(200,\n)}
\multido{\n=0+5}{9}{\psline[linewidth=0.4pt](0,\n)(200,\n)}
\psaxes[linewidth=1.25pt,Dx=20,Dy=5]{->}(0,0)(0,0)(200,40)
\psaxes[linewidth=1.25pt,Dx=20,Dy=5](0,0)(0,0)(200,40)
\pscurve[linecolor=blue,linewidth=1.25pt](0,0)(10,0.2)(20,0.5)(40,3.5)(60,7)(80,11)(100,14.9)(120,19)(140,23.1)(160,27.1)(180,31)(192.5,32.8)
\uput[d](100,-4){Nombre de jours}
\psdots[linecolor=red](120,23)(145,31)(175,38)
\rput{90}(-15,20){Masse (en grammes)}
\end{pspicture}	
	
		\item %Comparer les masses moyennes relevées par rapport aux masses moyennes théoriques.
Les masses moyennes relevées sont toutes supérieures aux masses moyennes théoriques espérées, ce qui est une bonne nouvelle pour l'éleveur.
	\end{enumerate}
\end{enumerate}

\vspace{0,5cm}

